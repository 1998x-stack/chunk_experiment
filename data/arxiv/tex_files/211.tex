\documentclass[reqno,12pt]{amsart}
%\documentclass{article}

\usepackage{psfig}

\setlength{\oddsidemargin}{0.0in}
\setlength{\evensidemargin}{0.0in}
\setlength{\textwidth}{6.5in}

\usepackage{graphicx}
\usepackage{amssymb,amsmath,amstext}
\usepackage{latexsym}

\usepackage[norelsize]{algorithm2e}
%\usepackage[ruled]{algorithm2e}
\usepackage{color}
\usepackage{algorithmic}
\renewcommand{\algorithmcfname}{ALGORITHM}

\setlength{\topmargin}{-0.5in}
\setlength{\textheight}{9.5in}
\setlength{\textwidth}{5.85in}
\setlength{\oddsidemargin}{0.325in}
\setlength{\evensidemargin}{0.325in}
\setlength{\marginparwidth}{1.0in}
\newtheorem{theorem}{Theorem}
\newtheorem{lemma}{Lemma}
\newtheorem{corollary}{Corollary}
\begin{document}


\title[A Dense Initialization for Limited-Memory Quasi-Newton Methods]
{A Dense Initialization for Limited-Memory Quasi-Newton Methods}

%\thanks{This research is support in part by National Science Foundation grants
%CMMI-1333326 and CMMI-1334042.}


\author[J. Brust]{Johannes Brust}
\email{jbrust@ucmerced.edu}
\address{Applied Mathematics, University of California, Merced, Merced, CA 95343}

\author[O. Burdakov]{Oleg Burdakov}
\email{oleg.burdakov@liu.se}
\address{Department of Mathematics, Link\"{o}ping University, SE-581 83 Link\"{o}ping, Sweden}

\author[J. Erway]{Jennifer B. Erway}
\email{erwayjb@wfu.edu}
\address{Department of Mathematics, Wake Forest University, Winston-Salem, NC 27109}

\author[R. Marcia]{Roummel F. Marcia}
\email{rmarcia@ucmerced.edu}
\address{Applied Mathematics, University of California, Merced, Merced, CA 95343}

\thanks{J.~B. Erway is supported in part by National Science Foundation grants
CMMI-1334042 and IIS-1741264}
\thanks{R.~F. Marcia is supported in part by National Science Foundation grants
CMMI-1333326 and IIS-1741490}

\date{\today}

\keywords{Large-scale nonlinear optimization, limited-memory quasi-Newton methods, trust-region methods, quasi-Newton matrices, shape-changing norm}


\begin{abstract}
  We consider a family of dense initializations for
  limited-memory quasi-Newton methods.  
  %The proposed initialization
  %uses two parameters to approximate the curvature{\color{blue}s} of the 
  %{\color{blue}objective function} %Hessian 
  %in two complementary subspaces.  
  The proposed initialization %separates the full
%space 
exploits an eigendecompo\-sition-based separation of the full space
into two complementary subspaces, 
assigning a different initialization parameter to
each subspace. This family of dense
  initializations is proposed in the context of 
  a limited-memory Broyden-Fletcher-Goldfarb-Shanno ({L-BFGS}) 
  trust-region method that makes use of a shape-changing norm to
  define each subproblem. 
As with {L-BFGS} methods that traditionally
  use diagonal initialization, the dense initialization and the
  sequence of generated quasi-Newton matrices are never explicitly
  formed.  Numerical experiments on the CUTEst test set suggest that
  this initialization together with the shape-changing trust-region
  method outperforms other L-BFGS
  methods for solving general nonconvex unconstrained optimization
  problems.  While this dense initialization is proposed in the
  context of a special trust-region method, it has broad applications
  for more general quasi-Newton trust-region and line search methods.
  In fact, this initialization is suitable for use with any
  quasi-Newton update that admits a compact representation and, in
  particular, any member of the Broyden class of updates.
\end{abstract}

\maketitle


\newcommand{\mgap}{\;\;}
\newcommand{\bgap}{\;\;\;}
\newcommand{\qDef}{{\mathcal Q}}
\newcommand{\defined}{\mathop{\,{\scriptstyle\stackrel{\triangle}{=}}}\,}
\newcommand{\diag}{\text{diag}}
\renewcommand{\algorithmcfname}{ALGORITHM}

\makeatletter
\newcommand{\minimize}[1]{{\displaystyle\minim_{#1}}}
\newcommand{\minim}{\mathop{\operator@font{minimize}}}
\newcommand{\subject}{\mathop{\operator@font{subject\ to}}}  
\newcommand{\words}[1]{\mgap\text{#1}\mgap}
\def\BFGS{{\small BFGS}}
\def\LBFGS{{\small L-BFGS}}
\def\LSR{{\small L-SR1}}
\def\SR{{\small SR1}}
\def\CG{{\small CG}}
\def\DFP{{\small DFP}}
\def\OBS{{\small OBS}}
\def\OBSSC{{\small OBS-SC}}
\def\SCSR1{{\small SC-SR1}}
\def\PSB{{\small PSB}}
\def\QR{{\small QR}}
\def\MATLAB{{\small MATLAB}}
\renewcommand{\algorithmcfname}{ALGORITHM}
\renewcommand{\vec}[1]{#1}
%\newcommand{\bp}{\mathbf{p}}
%\newcommand{\bv}{\mathbf{v}}

\makeatother

\pagestyle{myheadings}
\thispagestyle{plain}
\markboth{J. BRUST, O. BURDAKOV, J. B. ERWAY AND R. F. MARCIA}{Dense initializations for limited-memory quasi-Newton methods}

\section{Introduction}
\label{sec:intro}
\input{1-Introduction}

\section{Background}
\label{sec:background}
\input{2-Background}

\section{The Proposed Method}
\label{sec:method}
\input{3-Method}

\section{Numerical Experiments}
\label{sec:numexp}
\input{4-NumericalExperiments}

\section{Conclusion}
\input{5-Conclusion}


%%%%%%%%%%%%%%%%%%%%%%%%%%%%%%%%%%%%%%%%%%%%%%
%\section{Acknowledgments}
%%%%%%%%%%%%%%%%%%%%%%%%%%%%%%%%%%%%%%%%%%%%%
%This research is support in part by National Science Foundation grants
%CMMI-1333326 and CMMI-1334042.

\bibliographystyle{abbrv}
\bibliography{myrefs}


\end{document}
% end of file template.tex

