% !TEX root = bottom.tex

%%%%%%%%%%%%%%%%%%%%%%%%%%%%%%%%%%%%%%%%%%%%%%%%%%%%%%
\section{Introduction}
\label{sec:intro}
%%%%%%%%%%%%%%%%%%%%%%%%%%%%%%%%%%%%%%%%%%%%%%%%%%%%%%

Ultrarelativistic heavy-ion collision (URHIC) experiments being performed at the Large Hadron Collider (LHC) at CERN and the Relativistic Heavy Ion Collider (RHIC) at Brookhaven National Laboratory aim to recreate a primordial state of nuclear matter known as the quark-gluon plasma (QGP). Comparisons between hydrodynamic simulations that describe the evolution of the bulk matter and experimental data suggest that LHC URHICs produce a QGP with an initial temperature on the order of $T_{0}=600{-}700$ MeV at $\tau_0 = 0.25$ fm/c \cite{Romatschke:2009im,Heinz:2013AnnRevNuc,Schenke:2011tv,*Gale:2013IntJModPhys,*Ryu:2015vwa,Niemi:2015qia,*Niemi:2015voa,Alqahtani:2017jwl,Alqahtani:2017tnq}, for beam energies in the range of 2.76 - 5.02 TeV/nucleon.  In addition, analysis of the collective flow of the matter produced in URHICs indicates that the QGP behaves like a nearly inviscid relativistic fluid \cite{Romatschke:2009im,Schenke:2011tv,Heinz:2013AnnRevNuc,Gale:2013IntJModPhys,Ryu:2015vwa,Niemi:2015qia,Niemi:2015voa,Alqahtani:2017jwl,*Alqahtani:2017tnq}.  Although the light hadronic states, such as pions, are disassociated at temperatures around the pseudocritical temperature $T_c \simeq 155$ MeV, it was predicted that, due to their large binding energies, bound states of heavy quarks could survive up to temperatures on the order of 600 MeV.  Due to their short formation times, such heavy-quark bound states probe the history of the QGP and their suppression relative to production in pp collisions was proposed to be a clear signal of the creation of the QGP stemming from temperature-dependent screening of the color force \cite{Matsui:1986dk,Karsch:1987pv}.  In practice, experimentalists do see a reduced yield of heavy quarkonium compared to elementary collisions both at RHIC and LHC \cite{Andronic:2015wma,Mocsy:2013syh}. 

As we have learned over the course of the last years, however, the suppression of heavy quarkonium may be accompanied by the regeneration of the observed quarkonium states (a) at the phase boundary between the QGP and the hadronic phase \cite{BraunMunzinger:2000ep,Thews:2000rj,Grandchamp:2002wp} or (b) dynamically due to in-medium recombination of $Q\bar{Q}$ pairs \cite{Young:2008he,*Young:2009tj,Emerick:2011xu}. If a sizable number of $Q\bar{Q}$ pairs is created in the initial stages of a collision, the probability for a free quark antiquark pair to coalesce into a bound state at some point during the QGP lifetime can become significant. Measurements of charmonium yields e.g. have shown that for this lighter flavor indeed regeneration becomes quite important at LHC energies \cite{Rapp:2017chc}. For bottomonium, only recently have there been studies carried out that include a regeneration component, see e.g. \cite{Du:2017qkv}.  In the case of bottomonium, one expects regeneration to be less important due to its much heavier rest mass and large vacuum binding energy ($\lesssim$ 1 GeV) and this is borne out by detailed calculations.  That said, it is desirable to have a unified framework that includes the effects of both in-medium suppression and regeneration.

Recent measurements at LHC have also shown an unambiguous signal for the elliptic flow of the $J/\Psi$ particle, which implies that charm quarks are at least partially kinetically equilibrated with the surrounding medium \cite{ALICE:2013xna,Liu:2009gx,Zhao:2012gc}. A similar observation for bottomonium has not been made and it is expected that bottomonium does not yet participate in the collective motion of the bulk at 5.02 TeV. Since equilibration is intimately related to a loss of memory, charmonium at LHC appears to provide us with a window into the late stages of the collision, while bottomonium is considered to act as probe of the full evolution of the QGP.  As a result, a detailed understanding of heavy quarkonium in-medium will open the possibility to infer the time-dependent properties of the nuclear matter produced in a heavy-ion collision. It is an open question which properties the different species are most sensitive to, as candidates of course temperature or the shear viscosity to entropy ratio come to mind. 

In this paper, we focus solely on bottomonium and investigate its suppression at both RHIC and LHC energies. To this end we combine a non-relativistic description of the quark-antiquark bound state in terms of an in-medium potential with a realistic dynamical model of the bulk matter created in the collision.   The potential model used is vetted by comparing it to lattice QCD calculations of the real and imaginary parts of the heavy-quark potential, providing the first model to attempt to constrain the full complex potential using lattice input.  In addition, since bottomonium states are believed to be formed early on in a collision ($\tau < 1$ fm/c), they can be sensitive to the early-time non-equilibrium dynamics of the QGP.  Of particular importance is the large pressure anisotropy of the QGP in the local rest frame, ${\cal P}_L \ll {\cal P}_T$, which is induced by the rapid longitudinal expansion of the QGP created in URHICs \cite{Ryblewski:2013jsa,Strickland:2014pga}.  This pressure anisotropy leads to potentially important non-equilibrium corrections to the widths of the various bottomonia states \cite{Dumitru:2007hy,Burnier:2009yu,Dumitru:2009fy,Strickland:2011mw,Strickland:2011aa,Krouppa:2015yoa,Du:2016wdx,Krouppa:2016jcl,Du:2016wdx,Biondini:2017qjh,Krouppa:2017lsw,Nopoush:2017zbu} which we will take into account using information provided by perturbative calculations of the screening masses in an anisotropic QGP \cite{Romatschke:2003ms,Dumitru:2007hy,Dumitru:2009fy,Strickland:2011aa,Nopoush:2017zbu} coupled with a realistic 3+1d dissipative anisotropic hydrodynamic evolution for the QGP background~\cite{Martinez:2010sc,Florkowski:2010cf,Nopoush:2014pfa,Alqahtani:2015qja,Alqahtani:2016rth,Alqahtani:2017jwl,Alqahtani:2017tnq}.  

The structure of our paper is as follows.  In section~\ref{sec:pot} we review how the complex-valued heavy-quark potential used for the description of bottomonium is defined and extracted using lattice QCD and appropriately parametrized for use in phenomenological applictions. Section~\ref{sec:dynamics} collects the relevant details on the dynamical evolution model of anisotropic hydrodynamics and how it is connected to the lattice-vetted potential via a spacetime-dependent anisotropic Debye mass. We present the results of our computation in section~\ref{sec:results} and conclude with a discussion and outlook in section~\ref{sec:con}.