
\chapter{Proofs for Chapter~\ref{chap:interpretation}}
\label{chap:appendixC5}

\section{Proof of Proposition~\ref{prop:semweakening2}}
\label{sect:proofofprop:semweakening2}

{
\renewcommand{\thetheorem}{\ref{prop:semweakening2}}
\begin{proposition}[Semantic weakening]
Given value contexts $\Gamma_1$ and $\Gamma_2$, a value type $A$, and a value variable $x$ such that $\sem{\Gamma_1,\Gamma_2} \in \mathcal{B}$ and $\sem{\Gamma_1, x \!:\! A, \Gamma_2} \in \mathcal{B}$, then we have: 
\begin{enumerate}[(a)]
\item Given a value type $B$ such that $\sem{\Gamma_1,\Gamma_2;B} \in \mathcal{V}_{\sem{\Gamma_1,\Gamma_2}}$, then 
\[
\sem{\Gamma_1, x \!:\! A,\Gamma_2;B} = \sproj {\Gamma_1} x A {\Gamma_2}^*(\sem{\Gamma_1,\Gamma_2;B}) \in \mathcal{V}_{\sem{\Gamma_1, x : A,\Gamma_2}}
\]
\item Given a computation type $\ul{C}$ such that $\sem{\Gamma_1,\Gamma_2;\ul{C}} \in \mathcal{C}_{\sem{\Gamma_1,\Gamma_2}}$, then
\[
\sem{\Gamma_1, x \!:\! A,\Gamma_2;\ul{C}} = \sproj {\Gamma_1} x A {\Gamma_2}^*(\sem{\Gamma_1,\Gamma_2;\ul{C}}) \in \mathcal{C}_{\sem{\Gamma_1, x : A,\Gamma_2}}
\]
\item Given a value term $V$ such that $\sem{\Gamma_1,\Gamma_2;V} : 1_{\sem{\Gamma_1,\Gamma_2}} \longrightarrow B$, then
\[
\sem{\Gamma_1, x \!:\! A,\Gamma_2;V} = \sproj {\Gamma_1} x A {\Gamma_2}^*(\sem{\Gamma_1,\Gamma_2;V}) : 1_{\sem{\Gamma_1, x : A,\Gamma_2}} \longrightarrow \sproj {\Gamma_1} x A {\Gamma_2}^*(B)
\]
\item Given a computation term $M$ such that $\sem{\Gamma_1,\Gamma_2;M} : 1_{\sem{\Gamma_1,\Gamma_2}} \longrightarrow U(\ul{C})$, then 
\[
\sem{\Gamma_1, x \!:\! A,\Gamma_2;M} = \sproj {\Gamma_1} x A {\Gamma_2}^*(\sem{\Gamma_1,\Gamma_2;M}) : 1_{\sem{\Gamma_1, x : A,\Gamma_2}} \longrightarrow U(\sproj {\Gamma_1} x A {\Gamma_2}^*(\ul{C}))
\]
\item Given a computation variable $z$, a computation type $\ul{C}$, and a homomorphism term $K$ such that $\sem{\Gamma_1, \Gamma_2; z \!:\! \ul{C}; K} : \sem{\Gamma_1,\Gamma_2;\ul{C}} \longrightarrow \ul{D}$ in $\mathcal{C}_{\sem{\Gamma_1,\Gamma_2}}$, then

\[
\begin{array}{c}
\hspace{-4.5cm}
\sem{\Gamma_1, x \!:\! A, \Gamma_2; z \!:\! \ul{C}; K} = \sproj {\Gamma_1} x A {\Gamma_2}^*(\sem{\Gamma_1, \Gamma_2; z \!:\! \ul{C}; K}) 
\\
\hspace{5.5cm}
: \sproj {\Gamma_1} x A {\Gamma_2}^*(\sem{\Gamma_1, \Gamma_2; \ul{C}}) \longrightarrow \sproj {\Gamma_1} x A {\Gamma_2}^*(\ul{D})
\end{array}
\]
\end{enumerate}
where we use the notation
\[
\sem{\Gamma_1, x \!:\! A,\Gamma_2;B} = \sproj {\Gamma_1} x A {\Gamma_2}^*(\sem{\Gamma_1,\Gamma_2;B}) \in \mathcal{V}_{\sem{\Gamma_1, x : A,\Gamma_2}}
\]
to mean that $\sem{\Gamma_1, x \!:\! A,\Gamma_2;B}$ is defined and that it is equal to $\sproj {\Gamma_1} x A {\Gamma_2}^*(\sem{\Gamma_1,\Gamma_2;B})$ as an object of $\mathcal{V}_{\sem{\Gamma_1, x : A,\Gamma_2}}$. We also use analogous notation for terms and morphisms.
\end{proposition}
\addtocounter{theorem}{-1}
}

\begin{proof}
We prove $(a)$--$(e)$ simultaneously, by induction on the sum of the sizes of the arguments to $\sem{-}$. We omit the cases involving the MLTT fragment of eMLTT because in the setting of contextual categories these proofs can be found in~\cite[Chapter~III]{Streicher:Semantics}.

The proofs of all the cases (both those covered in op.~cit. and the ones discussed below) follow the same general pattern: they rely on the semantic structures we use in the definitions being split, i.e.,  preserved on-the-nose by reindexing functors.

\vspace{0.2cm}
\noindent
\textbf{Type of thunked computations:}
In this case, we assume that 
\[
\sem{\Gamma_1,\Gamma_2;U\ul{C}} \in \mathcal{V}_{\sem{\Gamma_1,\Gamma_2}}
\]
and we need to show that 
\[
\sem{\Gamma_1, x \!:\! A,\Gamma_2;U\ul{C}} = \sproj {\Gamma_1} x A {\Gamma_2}^*(\sem{\Gamma_1,\Gamma_2;U\ul{C}}) \in \mathcal{V}_{\sem{\Gamma_1, x : A,\Gamma_2}}
\] 

First, by inspecting the definition of $\sem{-}$ for $U\ul{C}$, the assumption gives us that 
\[
\sem{\Gamma_1,\Gamma_2;\ul{C}} \in \mathcal{C}_{\sem{\Gamma_1,\Gamma_2}}
\]
on which we can use $(b)$ to get that 
\[
\sem{\Gamma_1, x \!:\! A,\Gamma_2;\ul{C}} = \sproj {\Gamma_1} x A {\Gamma_2}^*(\sem{\Gamma_1,\Gamma_2;\ul{C}}) \in \mathcal{C}_{\sem{\Gamma_1, x : A,\Gamma_2}}
\]

Now, by applying the functor $U$ to both sides of this last equation, we get
\[
U(\sem{\Gamma_1, x \!:\! A,\Gamma_2;\ul{C}}) = U(\sproj {\Gamma_1} x A {\Gamma_2}^*(\sem{\Gamma_1,\Gamma_2;\ul{C}})) \in \mathcal{V}_{\sem{\Gamma_1, x : A,\Gamma_2}}
\]

Next, as $U$ is a split fibred functor, we also have that
\[
U(\sem{\Gamma_1, x \!:\! A,\Gamma_2;\ul{C}}) = \sproj {\Gamma_1} x A {\Gamma_2}^*(U(\sem{\Gamma_1,\Gamma_2;\ul{C}})) \in \mathcal{V}_{\sem{\Gamma_1, x : A,\Gamma_2}}
\]

Finally, by using the definition of $\sem{-}$ for $U\ul{C}$, we get that
\[
\sem{\Gamma_1, x \!:\! A,\Gamma_2;U\ul{C}} = \sproj {\Gamma_1} x A {\Gamma_2}^*(\sem{\Gamma_1,\Gamma_2;U\ul{C}}) \in \mathcal{V}_{\sem{\Gamma_1, x : A,\Gamma_2}}
\]

\vspace{0.2cm}
\noindent
\textbf{Homomorphic function space:}
In this case, we assume that
\[
\sem{\Gamma_1,\Gamma_2;\ul{C} \multimap \ul{D}} \in \mathcal{V}_{\sem{\Gamma_1,\Gamma_2}}
\]
and we need to show that
\[
\sem{\Gamma_1, x \!:\! A, \Gamma_2; \ul{C} \multimap \ul{D}} = \sproj {\Gamma_1} x A {\Gamma_2}^*(\sem{\Gamma_1,\Gamma_2;\ul{C} \multimap \ul{D}}) \in \mathcal{V}_{\sem{\Gamma_1, x : A,\Gamma_2}}
\]

First, by inspecting the definition of $\sem{-}$ for $\ul{C} \multimap \ul{D}$, the assumption gives us that
\[
\sem{\Gamma_1,\Gamma_2;\ul{C}} \in \mathcal{C}_{\sem{\Gamma_1,\Gamma_2}}
\qquad
\sem{\Gamma_1,\Gamma_2;\ul{D}} \in \mathcal{C}_{\sem{\Gamma_1,\Gamma_2}}
\]
on which we can use $(b)$ to get that
\[
\begin{array}{c}
\sem{\Gamma_1, x \!:\! A, \Gamma_2; \ul{C}} = \sproj {\Gamma_1} x A {\Gamma_2}^*(\sem{\Gamma_1,\Gamma_2;\ul{C}}) \in \mathcal{C}_{\sem{\Gamma_1, x : A,\Gamma_2}}
\\[3mm]
\sem{\Gamma_1, x \!:\! A, \Gamma_2; \ul{D}} = \sproj {\Gamma_1} x A {\Gamma_2}^*(\sem{\Gamma_1,\Gamma_2;\ul{D}}) \in \mathcal{C}_{\sem{\Gamma_1, x : A,\Gamma_2}}
\end{array}
\]

Next, by applying the functor $\multimap$ to both sides of these equations, we get 
\[
\begin{array}{c}
\hspace{-6.5cm}
\sem{\Gamma_1, x \!:\! A, \Gamma_2; \ul{C}} \multimap \sem{\Gamma_1, x \!:\! A, \Gamma_2; \ul{D}} = 
\\
\hspace{3cm}
\sproj {\Gamma_1} x A {\Gamma_2}^*(\sem{\Gamma_1,\Gamma_2;\ul{C}}) \multimap \sproj {\Gamma_1} x A {\Gamma_2}^*(\sem{\Gamma_1,\Gamma_2;\ul{D}}) \in \mathcal{V}_{\sem{\Gamma_1, x : A,\Gamma_2}}
\end{array}
\]

Next, as $\multimap$ is a split fibred functor, we also have that
\[
\begin{array}{c}
\hspace{-6.5cm}
\sem{\Gamma_1, x \!:\! A, \Gamma_2; \ul{C}} \multimap \sem{\Gamma_1, x \!:\! A, \Gamma_2; \ul{D}} = 
\\[-1mm]
\hspace{5.25cm}
\sproj {\Gamma_1} x A {\Gamma_2}^*(\sem{\Gamma_1,\Gamma_2;\ul{C}} \multimap \sem{\Gamma_1,\Gamma_2;\ul{D}}) \in \mathcal{V}_{\sem{\Gamma_1, x : A,\Gamma_2}}
\end{array}
\]

Finally, by using the definition of $\sem{-}$ for $\ul{C} \multimap \ul{D}$, we get that
\[
\sem{\Gamma_1, x \!:\! A, \Gamma_2; \ul{C} \multimap \ul{D}} = \sproj {\Gamma_1} x A {\Gamma_2}^*(\sem{\Gamma_1,\Gamma_2;\ul{C} \multimap \ul{D}}) \in \mathcal{V}_{\sem{\Gamma_1, x : A,\Gamma_2}}
\]

\vspace{0.2cm}
\noindent
\textbf{Type of free computations over a value type:}
We omit the proof for this case because it is analogous to the case for the type of thunked computations. 

\vspace{0.2cm}
\noindent
\textbf{Computational $\Sigma$-type:}
In this case, we assume that 
\[
\sem{\Gamma_1,\Gamma_2;\Sigma\, y \!:\! B .\, \ul{C}} \in \mathcal{C}_{\sem{\Gamma_1,\Gamma_2}}
\]
and we need to show that
\[
\sem{\Gamma_1, x \!:\! A,\Gamma_2;\Sigma\, y \!:\! B .\, \ul{C}} = \sproj {\Gamma_1} x A {\Gamma_2}^*(\sem{\Gamma_1,\Gamma_2;\Sigma\, y \!:\! B .\, \ul{C}}) \in \mathcal{C}_{\sem{\Gamma_1, x : A,\Gamma_2}}
\]

First, by inspecting the definition of $\sem{-}$ for $\Sigma\, y \!:\! B .\, \ul{C}$, the assumption gives us that 
\[
\sem{\Gamma_1,\Gamma_2;B} \in \mathcal{V}_{\sem{\Gamma_1,\Gamma_2}} 
\qquad
\sem{\Gamma_1,\Gamma_2, y \!:\! B;\ul{C}} \in \mathcal{C}_{\ia {\sem{\Gamma_1,\Gamma_2;B}}}
\]
on which we can use $(a)$ and the induction hypothesis, respectively, to get that
\[
\begin{array}{c}
\sem{\Gamma_1, x \!:\! A,\Gamma_2;B} = \sproj {\Gamma_1} x A {\Gamma_2}^*(\sem{\Gamma_1,\Gamma_2;B}) \in \mathcal{V}_{\sem{\Gamma_1, x : A,\Gamma_2}}
\\[3mm]
\sem{\Gamma_1, x \!:\! A,\Gamma_2, y \!:\! B;\ul{C}} = \sproj {\Gamma_1} x A {\Gamma_2, y : B}^*(\sem{\Gamma_1,\Gamma_2, y \!:\! B;\ul{C}}) \in \mathcal{C}_{\ia {\sem{\Gamma_1, x : A,\Gamma_2; B}}}
\end{array}
\]

Now, by using the definition of $\sproj {\Gamma_1} x A {\Gamma_2, y : B}$ in the second equation, we get that
\[
\begin{array}{c}
\hspace{-0.25cm}
\sem{\Gamma_1, x \!:\! A,\Gamma_2, y \!:\! B;\ul{C}} = (\ia {\overline{\sproj {\Gamma_1} x A {\Gamma_2}}(\sem{\Gamma_1, \Gamma_2;B})})^*(\sem{\Gamma_1,\Gamma_2, y \!:\! B;\ul{C}}) \in \mathcal{C}_{\ia {\sem{\Gamma_1, x : A,\Gamma_2; B}}}
\end{array}
\]

Next, by applying the functor $\Sigma_{\sem{\Gamma_1, x \!:\! A,\Gamma_2;B}}$ to both sides of this equation, we get that
\[
\begin{array}{c}
\hspace{-8cm}
\Sigma_{\sem{\Gamma_1, x : A,\Gamma_2;B}} (\sem{\Gamma_1, x \!:\! A,\Gamma_2, y \!:\! B;\ul{C}}) = 
\\
\hspace{1.5cm}
\Sigma_{\sem{\Gamma_1, x : A,\Gamma_2;B}} ((\ia {\overline{\sproj {\Gamma_1} x A {\Gamma_2}}(\sem{\Gamma_1, \Gamma_2;B})})^*(\sem{\Gamma_1,\Gamma_2, y \!:\! B;\ul{C}})) \in \mathcal{C}_{\sem{\Gamma_1, x : A,\Gamma_2}}
\end{array}
\]


Next, by using the equation we got from using $(a)$ above, we get that
\[
\begin{array}{c}
\hspace{-8cm}
\Sigma_{\sem{\Gamma_1, x : A,\Gamma_2;B}} (\sem{\Gamma_1, x \!:\! A,\Gamma_2, y \!:\! B;\ul{C}}) = 
\\
\Sigma_{\sproj {\Gamma_1} x A {\Gamma_2}^*(\sem{\Gamma_1,\Gamma_2;B})} ((\ia {\overline{\sproj {\Gamma_1} x A {\Gamma_2}}(\sem{\Gamma_1, \Gamma_2;B})})^*(\sem{\Gamma_1,\Gamma_2, y \!:\! B;\ul{C}})) \in \mathcal{C}_{\sem{\Gamma_1, x : A,\Gamma_2}}
\end{array}
\]

Now, by using the split Beck-Chevalley condition for $\Sigma_{\sem{\Gamma_1, x : A,\Gamma_2;B}}$, we get that
\[
\begin{array}{c}
\hspace{-6.5cm}
\Sigma_{\sem{\Gamma_1, x : A,\Gamma_2;B}} (\sem{\Gamma_1, x \!:\! A,\Gamma_2, y \!:\! B;\ul{C}}) = 
\\
\hspace{4.75cm}
\sproj {\Gamma_1} x A {\Gamma_2}^*(\Sigma_{\sem{\Gamma_1,\Gamma_2;B}} (\sem{\Gamma_1,\Gamma_2, y \!:\! B;\ul{C}})) \in \mathcal{C}_{\sem{\Gamma_1, x : A,\Gamma_2}}
\end{array}
\]

Finally, by using the definition of $\sem{-}$ for $\Sigma\, y \!:\! B .\, \ul{C}$, we get that
\[
\sem{\Gamma_1, x \!:\! A,\Gamma_2;\Sigma\, y \!:\! B .\, \ul{C}} = \sproj {\Gamma_1} x A {\Gamma_2}^*(\sem{\Gamma_1,\Gamma_2;\Sigma\, y \!:\! B .\, \ul{C}}) \in \mathcal{C}_{\sem{\Gamma_1, x : A,\Gamma_2}}
\]

\vspace{0.2cm}
\noindent
\textbf{Computational $\Pi$-type:}
We omit the proof for this case because it is analogous to the case for the computational $\Sigma$-type.

\vspace{0.2cm}
\noindent
\textbf{Thunking a computation:}
In this case, we assume that 
\[
\sem{\Gamma_1,\Gamma_2;\thunk M} : 1_{\sem{\Gamma_1,\Gamma_2}} \longrightarrow U(\ul{C})
\]
and we need to show that 
\[
\begin{array}{c}
\hspace{-3.25cm}
\sem{\Gamma_1, x \!:\! A,\Gamma_2;\thunk M} = \sproj {\Gamma_1} x A {\Gamma_2}^*(\sem{\Gamma_1,\Gamma_2;\thunk M}) 
\\
\hspace{7.5cm}
: 1_{\sem{\Gamma_1, x : A,\Gamma_2}} \longrightarrow \sproj {\Gamma_1} x A {\Gamma_2}^*(U(\ul{C}))
\end{array}
\]

First, by inspecting the definition of $\sem{-}$ for $\thunk M$, the assumption gives us
\[
\sem{\Gamma_1,\Gamma_2;M} : 1_{\sem{\Gamma_1,\Gamma_2}} \longrightarrow \ul{C}
\]

Now, by using $(d)$ on this morphism, we get that
\[
\sem{\Gamma_1, x \!:\! A,\Gamma_2;M} = \sproj {\Gamma_1} x A {\Gamma_2}^*(\sem{\Gamma_1,\Gamma_2;M}) : 1_{\sem{\Gamma_1, x : A,\Gamma_2}} \longrightarrow \sproj {\Gamma_1} x A {\Gamma_2}^*(\ul{C})
\]

Next, by using the definition of $\sem{-}$ for $\thunk M$, we get that
\[
\begin{array}{c}
\hspace{-2.75cm}
\sem{\Gamma_1, x \!:\! A,\Gamma_2;\thunk M} = \sproj {\Gamma_1} x A {\Gamma_2}^*(\sem{\Gamma_1,\Gamma_2;\thunk M}) 
\\
\hspace{7.7cm}
: 1_{\sem{\Gamma_1, x : A,\Gamma_2}} \longrightarrow U(\sproj {\Gamma_1} x A {\Gamma_2}^*(\ul{C}))
\end{array}
\]

Finally, as $U$ is a split fibred functor, we get that
\[
\begin{array}{c}
\hspace{-2.75cm}
\sem{\Gamma_1, x \!:\! A,\Gamma_2;\thunk M} = \sproj {\Gamma_1} x A {\Gamma_2}^*(\sem{\Gamma_1,\Gamma_2;\thunk M}) 
\\
\hspace{7.7cm}
: 1_{\sem{\Gamma_1, x : A,\Gamma_2}} \longrightarrow \sproj {\Gamma_1} x A {\Gamma_2}^*(U(\ul{C}))
\end{array}
\]

\vspace{0.2cm}
\noindent
\textbf{Homomorphic lambda abstraction:}
In this case, we assume that
\[
\sem{\Gamma_1,\Gamma_2;\lambda\, z \!:\! \ul{C} .\, K} : 1_{\sem{\Gamma_1,\Gamma_2}} \longrightarrow \sem{\Gamma_1,\Gamma_2;\ul{C}} \multimap \ul{D}
\]
and we have to show that
\[
\begin{array}{c}
\hspace{-3cm}
\sem{\Gamma_1, x \!:\! A,\Gamma_2;\lambda\, z \!:\! \ul{C} .\, K} = \sproj {\Gamma_1} x A {\Gamma_2}^*(\sem{\Gamma_1,\Gamma_2;\lambda\, z \!:\! \ul{C} .\, K}) 
\\
\hspace{5.5cm}
: 1_{\sem{\Gamma_1, x : A,\Gamma_2}} \longrightarrow \sproj {\Gamma_1} x A {\Gamma_2}^*(\sem{\Gamma_1,\Gamma_2;\ul{C}} \multimap \ul{D})
\end{array}
\]

First, by inspecting the definition of $\sem{-}$ for $\lambda\, z \!:\! \ul{C} .\, K$, the assumption gives us that
\[
\sem{\Gamma_1,\Gamma_2;z \!:\! \ul{C}; K} : \sem{\Gamma_1,\Gamma_2;\ul{C}} \longrightarrow \ul{D}
\]

Now, by using $(e)$ on this morphism, we get that
\[
\begin{array}{c}
\hspace{-3.5cm}
\sem{\Gamma_1, x \!:\! A, \Gamma_2; z \!:\! \ul{C}; K} = \sproj {\Gamma_1} x A {\Gamma_2}^*(\sem{\Gamma_1, \Gamma_2; z \!:\! \ul{C}; K}) 
\\
\hspace{5.5cm}
: \sproj {\Gamma_1} x A {\Gamma_2}^*(\sem{\Gamma_1, \Gamma_2; \ul{C}}) \longrightarrow \sproj {\Gamma_1} x A {\Gamma_2}^*(\ul{D})
\end{array}
\]

Finally, we show that
\[
\begin{array}{c}
\hspace{-3cm}
\sem{\Gamma_1, x \!:\! A,\Gamma_2;\lambda\, z \!:\! \ul{C} .\, K} = \sproj {\Gamma_1} x A {\Gamma_2}^*(\sem{\Gamma_1,\Gamma_2;\lambda\, z \!:\! \ul{C} .\, K}) 
\\
\hspace{5.5cm}
: 1_{\sem{\Gamma_1, x : A,\Gamma_2}} \longrightarrow \sproj {\Gamma_1} x A {\Gamma_2}^*(\sem{\Gamma_1,\Gamma_2;\ul{C}} \multimap \ul{D})
\end{array}
\]
by proving that the next diagram commutes, in which we write $\mathsf{pr}$ for $\sproj {\Gamma_1} x A {\Gamma_2}$. To improve the readability of this diagram, we aggregate some small proof steps.
\[
\vspace{0.3cm}
\xymatrix@C=11.5em@R=6em@M=0.5em{
1_{\sem{\Gamma_1, x : A, \Gamma_2}} \ar[r]^-{\sem{\Gamma_1, x : A,\Gamma_2;\lambda\, z : \ul{C} .\, K}} \ar[dr]_>>>>>>>>>>>>>>>>{\xi^{-1}_{\sem{\Gamma_1, x : A, \Gamma_2},\mathsf{pr}^*(\sem{\Gamma_1, \Gamma_2; \ul{C}}),\mathsf{pr}^*(\ul{D})}(\mathsf{pr}^*(\sem{\Gamma_1, \Gamma_2; z : \ul{C}; K}))\qquad\qquad\qquad} \ar@/_3pc/[dd]_-{=} & \mathsf{pr}^*(\sem{\Gamma_1,\Gamma_2;\ul{C}} \multimap \ul{D})
\\
& \mathsf{pr}^*(\sem{\Gamma_1,\Gamma_2;\ul{C}}) \multimap \mathsf{pr}^*(\ul{D}) \ar[u]_-{=}^>>>>>{\dcomment{\text{def. of } \sem{\Gamma_1, x \!:\! A,\Gamma_2;\lambda\, z \!:\! \ul{C} .\, K}} \quad}^>>>>>>>>>>>>>{\dcomment{(e)}\qquad\quad} \ar@{}[d]_<<<<<<<{\dcomment{\xi^{-1} \text{ is preserved on-the-nose by reindexing}}\qquad} \ar@{}[dd]_>>>>>>>>>>>>>>>>>>>>>>>>>>>>>>{\dcomment{\text{def. of } \sem{\Gamma_1,\Gamma_2;\lambda\, z \!:\! \ul{C} .\, K}}\qquad\qquad\qquad}
\\
\mathsf{pr}^*(1_{\sem{\Gamma_1,\Gamma_2}}) \ar@/^2pc/[r]^-{\mathsf{pr}^*(\xi^{-1}_{\sem{\Gamma_1,\Gamma_2}, \sem{\Gamma_1,\Gamma_2;\ul{C}},\ul{D}}(\sem{\Gamma_1,\Gamma_2; z : \ul{C} ; K}))} \ar@/_2pc/[r]_{\mathsf{pr}^*(\sem{\Gamma_1,\Gamma_2;\lambda\, z : \ul{C} .\, K})} & \mathsf{pr}^*(\sem{\Gamma_1,\Gamma_2;\ul{C}} \multimap \ul{D}) \ar@/_6pc/[uu]_>>>>>>>>>{\!\!\!\!\id_{\mathsf{pr}^*(\sem{\Gamma_1,\Gamma_2;\ul{C}} \multimap \ul{D})}}
\\
&
}
\vspace{-1.5cm}
\]

\vspace{0.1cm}
\noindent
\textbf{Returning a value:}
In this case, we assume that 
\[
\sem{\Gamma_1,\Gamma_2;\return V} : 1_{\sem{\Gamma_1,\Gamma_2}} \longrightarrow U(F(B))
\]
and we need to show that
\[
\begin{array}{c}
\hspace{-3.5cm}
\sem{\Gamma_1, x \!:\! A,\Gamma_2;\return V} = \sproj {\Gamma_1} x A {\Gamma_2}^*(\sem{\Gamma_1,\Gamma_2;\return V}) 
\\
\hspace{7cm}
: 1_{\sem{\Gamma_1, x : A,\Gamma_2}} \longrightarrow U(\sproj {\Gamma_1} x A {\Gamma_2}^*(F(B)))
\end{array}
\]

First, by inspecting the definition of $\sem{-}$ for $\return V$, the assumption gives us
\[
\sem{\Gamma_1,\Gamma_2;V} : 1_{\sem{\Gamma_1,\Gamma_2}} \longrightarrow B
\]

Next, by using $(c)$ on this morphism, we get that
\[
\sem{\Gamma_1, x \!:\! A,\Gamma_2;V} = \sproj {\Gamma_1} x A {\Gamma_2}^*(\sem{\Gamma_1,\Gamma_2;V}) : 1_{\sem{\Gamma_1, x : A,\Gamma_2}} \longrightarrow \sproj {\Gamma_1} x A {\Gamma_2}^*(B)
\]

Finally, we show that
\[
\begin{array}{c}
\hspace{-2.5cm}
\sem{\Gamma_1, x \!:\! A,\Gamma_2;\return V} = \sproj {\Gamma_1} x A {\Gamma_2}^*(\sem{\Gamma_1,\Gamma_2;\return V}) 
\\
\hspace{7cm}
: 1_{\sem{\Gamma_1, x : A,\Gamma_2}} \longrightarrow U(\sproj {\Gamma_1} x A {\Gamma_2}^*(F(B)))
\end{array}
\]
by showing that the next diagram commutes.
\[
\xymatrix@C=11.5em@R=6em@M=0.5em{
1_{\sem{\Gamma_1, x : A,\Gamma_2}} \ar[r]^-{\sem{\Gamma_1, x : A,\Gamma_2;\return V}} \ar[d]_>>>>>>>>>{\sem{\Gamma_1, x : A,\Gamma_2;V}}^-{\quad\qquad\qquad\dcomment{\text{def. of } \sem{\Gamma_1, x \!:\! A,\Gamma_2;\return V}}} \ar@/_7pc/[dd]_-{=} & U(\sproj {\Gamma_1} x A {\Gamma_2}^*(F(B)))
\\
\sproj {\Gamma_1} x A {\Gamma_2}^*(B) \ar[r]^-{\eta^{F \,\dashv\, U}_{\sproj {\Gamma_1} x A {\Gamma_2}^*(B)}} \ar[dr]^>>>>>>>>>>>>>>>>{\quad\sproj {\Gamma_1} x A {\Gamma_2}^*(\eta^{F \,\dashv\, U}_{B})}_>>>>>>>>>>>>>>>>>>>>>>{\dcomment{\text{functoriality of } \sproj {\Gamma_1} x A {\Gamma_2}^*}\qquad\qquad\qquad} & U(F(\sproj {\Gamma_1} x A {\Gamma_2}^*(B))) \ar[u]_-{=}
\\
\sproj {\Gamma_1} x A {\Gamma_2}^*(1_{\sem{\Gamma_1, \Gamma_2}}) \ar[u]_>>>>>>>>>>>>>>{\sproj {\Gamma_1} x A {\Gamma_2}^*(\sem{\Gamma_1,\Gamma_2;V})}^>>>>>>>{\dcomment{(c)}\qquad} \ar[r]_-{\sproj {\Gamma_1} x A {\Gamma_2}^*(\eta^{F \,\dashv\, U} \,\comp\, \sem{\Gamma_1,\Gamma_2;V})} \ar@/_5pc/[r]_-{\sproj {\Gamma_1} x A {\Gamma_2}^*(\sem{\Gamma_1,\Gamma_2;\return V})} & \sproj {\Gamma_1} x A {\Gamma_2}^*(U(F(B))) \ar[u]_-{=}^>>>>>{\dcomment{\text{Proposition~\ref{prop:fibrednaturaltransformationspreserved}}}\qquad\qquad}
\\
& \ar@{}[u]^>>>>>>>{\dcomment{\text{def. of } \sem{\Gamma_1,\Gamma_2;\return V}}\qquad\qquad\qquad\,\,}
}
\]


\noindent
\textbf{Sequential composition for computation terms:}
In this case, we assume that 
\[
\sem{\Gamma_1,\Gamma_2; \doto M {y \!:\! B} {\ul{C}} N} : 1_{\sem{\Gamma_1,\Gamma_2}} \longrightarrow U(\sem{\Gamma_1,\Gamma_2;\ul{C}})
\]
and we need to show that
\[
\begin{array}{c}
\hspace{-1.5cm}
\sem{\Gamma_1, x \!:\! A,\Gamma_2;\doto M {y \!:\! B} {\ul{C}} N} = \sproj {\Gamma_1} x A {\Gamma_2}^*(\sem{\Gamma_1,\Gamma_2;\doto M {y \!:\! B} {\ul{C}} N}) 
\\
\hspace{6cm}
: 1_{\sem{\Gamma_1, x : A,\Gamma_2}} \longrightarrow U(\sproj {\Gamma_1} x A {\Gamma_2}^*(\sem{\Gamma_1,\Gamma_2;\ul{C}}))
\end{array}
\]

First, by inspecting the definition of $\sem{-}$ for $\doto M {y \!:\! B} {\ul{C}} N$, the assumption gives us that
\[
\begin{array}{c}
\sem{\Gamma_1,\Gamma_2; M} : 1_{\sem{\Gamma_1,\Gamma_2}} \longrightarrow U(F(\sem{\Gamma_1,\Gamma_2;B}))
\\[3mm]
\sem{\Gamma_1,\Gamma_2, y \!:\! B; N} : 1_{\sem{\Gamma_1,\Gamma_2, y : B}} \longrightarrow U(\pi^*_{\sem{\Gamma_1,\Gamma_2;B}}(\sem{\Gamma_1,\Gamma_2;\ul{C}}))
\end{array}
\]

Next, by using the induction hypothesis on these morphisms, we get that
\[
\begin{array}{c}
\hspace{-5.4cm}
\sem{\Gamma_1, x \!:\! A,\Gamma_2;M} = \sproj {\Gamma_1} x A {\Gamma_2}^*(\sem{\Gamma_1,\Gamma_2;M}) 
\\
\hspace{5.5cm}
: 1_{\sem{\Gamma_1, x : A,\Gamma_2}} \longrightarrow U(\sproj {\Gamma_1} x A {\Gamma_2}^*(F(\sem{\Gamma_1,\Gamma_2;B})))
\\[3mm]
\hspace{-3.4cm}
\sem{\Gamma_1, x \!:\! A,\Gamma_2, y \!:\! B;N} = \sproj {\Gamma_1} x A {\Gamma_2, y : B}^*(\sem{\Gamma_1,\Gamma_2, y \!:\! B;N}) 
\\
\hspace{3cm}
: 1_{\sem{\Gamma_1, x : A,\Gamma_2, y : B}} \longrightarrow U(\sproj {\Gamma_1} x A {\Gamma_2, y : B}^*(\pi^*_{\sem{\Gamma_1,\Gamma_2;B}}(\sem{\Gamma_1,\Gamma_2;\ul{C}})))
\end{array}
\]

Finally, we show that 
\[
\begin{array}{c}
\hspace{-0.5cm}
\sem{\Gamma_1, x \!:\! A,\Gamma_2;\doto M {y \!:\! B} {\ul{C}} N} = \sproj {\Gamma_1} x A {\Gamma_2}^*(\sem{\Gamma_1,\Gamma_2;\doto M {y \!:\! B} {\ul{C}} N}) 
\\
\hspace{6cm}
: 1_{\sem{\Gamma_1, x : A,\Gamma_2}} \longrightarrow U(\sproj {\Gamma_1} x A {\Gamma_2}^*(\sem{\Gamma_1,\Gamma_2;\ul{C}}))
\end{array}
\]
by proving that the next diagram commutes, in which we write $\mathsf{pr}_{\Gamma_2}$ for $\sproj {\Gamma_1} x A {\Gamma_2}$ and $\mathsf{pr}_{\Gamma_2, y : B}$ for $\sproj {\Gamma_1} x A {\Gamma_2, y : B}$. For better readability, we aggregate small proof steps.

\[
\hspace{-0.3cm}
\scriptsize
\xymatrix@C=2em@R=4.5em@M=0.5em{
1_{\sem{\Gamma_1, x : A,\Gamma_2}} 
\ar[d]_-{\sem{\Gamma_1, x : A, \Gamma_2;M}}^<<<<{\!\qquad\qquad\qquad\dscomment{\text{use of the induction hypothesis on } \sem{\Gamma_1,\Gamma_2;M}}}^>>>>{\qquad\qquad\qquad\qquad\qquad\dscomment{U \text{ is split fibred}}} \ar[r]^-{=} & \mathsf{pr}_{\Gamma_2}^*(1_{\sem{\Gamma_1,\Gamma_2}}) \ar[d]^-{\mathsf{pr}^*_{\Gamma_2}(\sem{\Gamma_1,\Gamma_2;M})}
\\
U(\mathsf{pr}_{\Gamma_2}^*(F(\sem{\Gamma_1,\Gamma_2;B}))) \ar[d]_-{=} \ar[r]_-{=} & \mathsf{pr}^*_{\Gamma_2}(U(F(\sem{\Gamma_1,\Gamma_2;B}))) \ar[dd]^-{\mathsf{pr}^*_{\Gamma_2}(U(F(\langle \id_{\sem{\Gamma_1,\Gamma_2;B}} , ! \rangle)))}_<<<<<<<{\dscomment{F \text{ and } U \text{ are split fibred}} \qquad\qquad\qquad\qquad\quad}_<<<<<<<<<<<<<<<<<<<<<<<<<{\dscomment{\text{the fibred Cartesian products in } p \text{ are split}} \qquad\qquad\qquad\,\,\,\,}
\\
U(F(\mathsf{pr}_{\Gamma_2}^*(\sem{\Gamma_1,\Gamma_2;B}))) \ar[d]_-{U(F(\langle \id_{\mathsf{pr}_{\Gamma_2}^*(\sem{\Gamma_1,\Gamma_2;B})} , ! \rangle))} & 
\\
U(F(\Sigma_{\mathsf{pr}_{\Gamma_2}^*(\sem{\Gamma_1,\Gamma_2;B})}(\pi^*_{\mathsf{pr}_{\Gamma_2}^*(\sem{\Gamma_1,\Gamma_2;B})}(1_{\sem{\Gamma_1, x : A, \Gamma_2}})))) \ar[d]_-{=} \ar[r]_-{=} & \mathsf{pr}^*_{\Gamma_2}(U(F(\Sigma_{\sem{\Gamma_1,\Gamma_2;B}}(\pi^*_{\sem{\Gamma_1,\Gamma_2;B}}(1_{\sem{\Gamma_1,\Gamma_2}}))))) \ar[d]^-{=}_<<<<{\dscomment{\text{split Beck-Chevalley}}\qquad\qquad}_<<<<{\dscomment{\text{def. of } \sproj {\Gamma_1} x A {\Gamma_2, y : B}}\qquad\qquad\qquad\qquad\qquad\qquad\qquad}_<<<<<<<<<<{\dscomment{F \text{ and } U \text{ are split fibred}} \qquad\qquad\qquad\qquad\quad\!\!\!\!}
\\
U(F(\Sigma_{\mathsf{pr}_{\Gamma_2}^*(\sem{\Gamma_1,\Gamma_2;B})}(1_{\sem{\Gamma_1, x : A, \Gamma_2, y : B}}))) \ar[dd]^<<<<<{U(F(\Sigma_{\mathsf{pr}_{\Gamma_2}^*(\sem{\Gamma_1,\Gamma_2;B})}(\sem{\Gamma_1, x : A,\Gamma_2,y : B;N})))} \ar[r]_-{=} & \mathsf{pr}^*_{\Gamma_2}(U(F(\Sigma_{\sem{\Gamma_1,\Gamma_2;B}}(1_{\sem{\Gamma_1,\Gamma_2,y : B}})))) \ar[dd]_>>>>>{\mathsf{pr}^*_{\Gamma_2}(U(F(\Sigma_{\sem{\Gamma_1,\Gamma_2;B}}(\sem{\Gamma_1,\Gamma_2,y:B;N}))))}_>>>>>>>>>>>{\dscomment{\text{split Beck-Chevalley}}\qquad\qquad\qquad\qquad\quad}_>>>>>>>>>>>{\dscomment{\text{def. of } \sproj {\Gamma_1} x A {\Gamma_2, y : B}}\quad\qquad\qquad\qquad\qquad\qquad\qquad\qquad\qquad}_>>>>>>>>>>>{\dscomment{F \text{ and } U \text{ are s. fib.}} \qquad\!\!\!\!}_<<<<<<<<<<<<{\dscomment{\text{use of the induction hypothesis on } \sem{\Gamma_1,\Gamma_2, y \!:\! B;N}}\qquad\qquad\quad}
\\
&
\\
U(F(\Sigma_{\mathsf{pr}_{\Gamma_2}^*(\sem{\Gamma_1,\Gamma_2;B})}(U(\mathsf{pr}^*_{\Gamma_2, y : B}(\pi^*_{\sem{\Gamma_1,\Gamma_2;B}}(\sem{\Gamma_1, \Gamma_2;\ul{C}})))))) \ar[d]_-{=}^-{\qquad\qquad\qquad\qquad\qquad\qquad\qquad\dscomment{\text{split Beck-Chevalley}}}^-{\qquad\qquad\dscomment{\text{def. of } \sproj {\Gamma_1} x A {\Gamma_2, y : B}}} \ar[r]_-{=} & \mathsf{pr}^*_{\Gamma_2}(U(F(\Sigma_{\sem{\Gamma_1,\Gamma_2;B}}(U(\pi^*_{\sem{\Gamma_1,\Gamma_2;B}}(\sem{\Gamma_1,\Gamma_2;\ul{C}})))))) \ar[dd]^-{=}
\\
U(F(\Sigma_{\mathsf{pr}_{\Gamma_2}^*(\sem{\Gamma_1,\Gamma_2;B})}(U(\pi^*_{\mathsf{pr}_{\Gamma_2}^*(\sem{\Gamma_1,\Gamma_2;B})}(\mathsf{pr}^*_{\Gamma_2}(\sem{\Gamma_1,\Gamma_2;\ul{C}})))))) \ar[d]_{=}^-{\qquad\qquad\qquad\qquad\quad\,\,\,\,\dscomment{F \text{ and } U \text{ are split fibred}}}
\\
U(F(\Sigma_{\mathsf{pr}_{\Gamma_2}^*(\sem{\Gamma_1,\Gamma_2;B})}(\pi^*_{\mathsf{pr}_{\Gamma_2}^*(\sem{\Gamma_1,\Gamma_2;B})}(U(\mathsf{pr}^*_{\Gamma_2}(\sem{\Gamma_1,\Gamma_2;\ul{C}})))))) \ar[d]_-{U(F(\varepsilon^{\Sigma_{\mathsf{pr}_{\Gamma_2}^*(\sem{\Gamma_1,\Gamma_2;B})} \,\dashv\, \pi^*_{\mathsf{pr}_{\Gamma_2}^*(\sem{\Gamma_1,\Gamma_2;B})}}_{U(\mathsf{pr}^*_{\Gamma_2}(\sem{\Gamma_1,\Gamma_2;\ul{C}}))}))} \ar[r]_-{=} & \mathsf{pr}^*_{\Gamma_2}(U(F(\Sigma_{\sem{\Gamma_1,\Gamma_2;B}}(\pi^*_{\sem{\Gamma_1,\Gamma_2;B}}(U(\sem{\Gamma_1,\Gamma_2;\ul{C}})))))) \ar[d]^-{\mathsf{pr}^*_{\Gamma_2}(U(F(\varepsilon^{\Sigma_{\sem{\Gamma_1,\Gamma_2;B}} \,\dashv\, \pi^*_{\sem{\Gamma_1,\Gamma_2;B}}}_{U(\sem{\Gamma_1,\Gamma_2;\ul{C}})})))}_>>>>{\dscomment{\text{split Beck-Chevalley}}\qquad\qquad\qquad\qquad\quad}_>>>>{\dscomment{\text{def. of } \sproj {\Gamma_1} x A {\Gamma_2, y : B}}\quad\qquad\qquad\qquad\qquad\qquad\qquad\qquad\qquad}_>>>>{\dscomment{F \text{ and } U \text{ are s. fib.}} \qquad\!\!\!\!}_<<<<{\dscomment{\text{Proposition~\ref{prop:BCfordepsums} for } \varepsilon^{\Sigma_{\mathsf{pr}_{\Gamma_2}^*(\sem{\Gamma_1,\Gamma_2;B})} \,\dashv\, \pi^*_{\mathsf{pr}_{\Gamma_2}^*(\sem{\Gamma_1,\Gamma_2;B})}}}\qquad\qquad\quad}
\\
U(F(U(\mathsf{pr}^*_{\Gamma_2}(\sem{\Gamma_1,\Gamma_2;\ul{C}})))) \ar[d]_-{U(\varepsilon^{F \,\dashv\, U}_{\mathsf{pr}^*_{\Gamma_2}(\sem{\Gamma_1,\Gamma_2;\ul{C}})})} \ar[r]_-{=} & \mathsf{pr}^*_{\Gamma_2}(U(F(U(\sem{\Gamma_1,\Gamma_2;\ul{C}})))) \ar[d]^-{\mathsf{pr}^*_{\Gamma_2}(U(\varepsilon^{F \,\dashv\, U}_{\sem{\Gamma_1,\Gamma_2;\ul{C}}}))}_>>>>{\dscomment{\text{split Beck-Chevalley}}\qquad\qquad\qquad\qquad\quad}_>>>>{\dscomment{\text{def. of } \sproj {\Gamma_1} x A {\Gamma_2, y : B}}\quad\qquad\qquad\qquad\qquad\qquad\qquad\qquad\qquad}_>>>>{\dscomment{F \text{ and } U \text{ are s. fib.}} \qquad\!\!\!\!}_<<<<{\dscomment{\text{Proposition~\ref{prop:fibrednaturaltransformationspreserved} for } \varepsilon^{F \,\dashv\, U}}\qquad\qquad\qquad\qquad\quad}
\\
U(\mathsf{pr}_{\Gamma_2}^*(\sem{\Gamma_1,\Gamma_2;\ul{C}})) \ar[r]_-{=} & \mathsf{pr}^*_{\Gamma_2}(U(\sem{\Gamma_1,\Gamma_2;\ul{C}}))
}
\]

We conclude by observing that the left-hand side top-to-bottom composite morphism is equal to  $\sem{\Gamma_1, x \!:\! A, \Gamma_2;\doto M {y \!:\! B} {\ul{C}} N}$, and that the right-hand side top-to-bottom composite morphism is equal to $\sproj {\Gamma_1} x A {\Gamma_2} ^* (\sem{\Gamma_1,\Gamma_2;\doto M {y \!:\! B} {\ul{C}} N})$. 

\vspace{0.2cm}
\noindent
\textbf{Computational pairing for computation terms:}
In this case, we assume that 
\[
\sem{\Gamma_1,\Gamma_2; \langle V , M \rangle_{(y : B).\, \ul{C}}} : 1_{\sem{\Gamma_1,\Gamma_2}} \longrightarrow U(\Sigma_{\sem{\Gamma_1,\Gamma_2;B}}(\sem{\Gamma_1,\Gamma_2, y \!:\! B ;\ul{C}}))
\]
and we need to show that
\[
\begin{array}{c}
\hspace{-2.5cm}
\sem{\Gamma_1, x \!:\! A,\Gamma_2; \langle V , M \rangle_{(y : B).\, \ul{C}}} = \sproj {\Gamma_1} x A {\Gamma_2}^*(\sem{\Gamma_1,\Gamma_2;\langle V , M \rangle_{(y : B).\, \ul{C}}}) 
\\
\hspace{4cm}
: 1_{\sem{\Gamma_1, x : A,\Gamma_2}} \longrightarrow U(\sproj {\Gamma_1} x A {\Gamma_2}^*(\Sigma_{\sem{\Gamma_1,\Gamma_2}}(\sem{\Gamma_1,\Gamma_2, y \!:\! B ;\ul{C}})))
\end{array}
\]

First, by inspecting the definition of $\sem{-}$ for $\langle V , M \rangle_{(y : B).\, \ul{C}}$, the assumption  gives us
\[
\begin{array}{c}
\sem{\Gamma_1,\Gamma_2;V} : 1_{\sem{\Gamma_1,\Gamma_2}} \longrightarrow \sem{\Gamma_1,\Gamma_2;B}
\\[3mm]
\sem{\Gamma_1,\Gamma_2;M} : 1_{\sem{\Gamma_1,\Gamma_2}} \longrightarrow U((\mathsf{s}(\sem{\Gamma_1,\Gamma_2;V}))^*(\sem{\Gamma_1,\Gamma_2, y \!:\! B ;\ul{C}}))
\end{array}
\]
on which we can use $(c)$ and the induction hypothesis, respectively, to get that
\[
\begin{array}{c}
\sem{\Gamma_1, x \!:\! A,\Gamma_2;V} = \sproj {\Gamma_1} x A {\Gamma_2}^*(\sem{\Gamma_1,\Gamma_2;V}) : 1_{\sem{\Gamma_1, x : A,\Gamma_2}} \longrightarrow \sproj {\Gamma_1} x A {\Gamma_2}^*(B)
\\[3mm]
\hspace{-6cm}
\sem{\Gamma_1, x \!:\! A,\Gamma_2;M} = \sproj {\Gamma_1} x A {\Gamma_2}^*(\sem{\Gamma_1,\Gamma_2;M})
\\
\hspace{2.5cm}
: 1_{\sem{\Gamma_1, x : A, \Gamma_2}} \longrightarrow U(\sproj {\Gamma_1} x A {\Gamma_2}^*((\mathsf{s}(\sem{\Gamma_1,\Gamma_2;V}))^*(\sem{\Gamma_1,\Gamma_2, y \!:\! B ;\ul{C}})))
\end{array}
\]

Finally, we show that 
\[
\begin{array}{c}
\hspace{-1.5cm}
\sem{\Gamma_1, x \!:\! A,\Gamma_2; \langle V , M \rangle_{(y : B).\, \ul{C}}} = \sproj {\Gamma_1} x A {\Gamma_2}^*(\sem{\Gamma_1,\Gamma_2;\langle V , M \rangle_{(y : B).\, \ul{C}}}) 
\\
\hspace{3.5cm}
: 1_{\sem{\Gamma_1, x : A,\Gamma_2}} \longrightarrow U(\sproj {\Gamma_1} x A {\Gamma_2}^*(\Sigma_{\sem{\Gamma_1,\Gamma_2;B}}(\sem{\Gamma_1,\Gamma_2, y \!:\! B ;\ul{C}})))
\end{array}
\]
by proving that the next diagram commutes, in which we write $\mathsf{pr}_{\Gamma_2}$ for $\sproj {\Gamma_1} x A {\Gamma_2}$ and $\mathsf{pr}_{\Gamma_2, y : B}$ for $\sproj {\Gamma_1} x A {\Gamma_2, y : B}$. For better readability, we aggregate small proof steps.
\[
\scriptsize
\xymatrix@C=4em@R=7em@M=0.5em{
1_{\sem{\Gamma_1, x : A,\Gamma_2}} 
\ar[r]^-{=} \ar[d]_-{\sem{\Gamma_1,x:A,\Gamma_2;M}}^<<<<<<{\qquad\qquad\quad\!\!\dscomment{\text{use of the induction hypothesis on } \sem{\Gamma_1,\Gamma_2;M} }}^>>>>>>>{\quad\qquad\qquad\qquad\qquad\dscomment{U \text{ is split fibred}}}
& 
\mathsf{pr}_{\Gamma_2}^*(1_{\sem{\Gamma_1,\Gamma_2}})
\ar[d]^-{\mathsf{pr}_{\Gamma_2}^*(\sem{\Gamma_1,\Gamma_2;M})}
\\
U(\mathsf{pr}_{\Gamma_2}^*((\mathsf{s}(\sem{\Gamma_1,\Gamma_2;V}))^*(\sem{\Gamma_1,\Gamma_2,y \!:\! B;\ul{C}}))) \ar[r]^-{=} \ar[d]_-{=}^{\,\,\,\quad\qquad\qquad\dscomment{\text{Proposition~\ref{prop:BCfordepcompsums} for } \eta^{\Sigma_{\sem{\Gamma_1,\Gamma_2;B}} \,\dashv\, \pi^*_{\sem{\Gamma_1,\Gamma_2;B}}}}} 
&
\mathsf{pr}_{\Gamma_2}^*(U((\mathsf{s}(\sem{\Gamma_1,\Gamma_2;V}))^*(\sem{\Gamma_1,\Gamma_2,y \!:\! B;\ul{C}}))) \ar[ddd]_>>>>>>{\mathsf{pr}_{\Gamma_2}^*(U((\mathsf{s}(\sem{\Gamma_1,\Gamma_2;V}))^*(\eta^{\Sigma_{\sem{\Gamma_1,\Gamma_2;B}} \,\dashv\, \pi^*_{\sem{\Gamma_1,\Gamma_2;B}}}_{\sem{\Gamma_1,\Gamma_2,y \!:\! B;\ul{C}}})))}
\\
U((\mathsf{s}(\mathsf{pr}_{\Gamma_2}^*(\sem{\Gamma_1,\Gamma_2;V})))^*(\mathsf{pr}_{\Gamma_2, y : B}^*(\sem{\Gamma_1,\Gamma_2, y \!:\! B;\ul{C}}))) \ar[dd]^<<<<<{U((\mathsf{s}(\mathsf{pr}_{\Gamma_2}^*(\sem{\Gamma_1,\Gamma_2;V})))^*(\eta^{\Sigma_{\mathsf{pr}_{\Gamma_2}^*(\sem{\Gamma_1,\Gamma_2;B})} \,\dashv\, \pi^*_{\mathsf{pr}_{\Gamma_2}^*(\sem{\Gamma_1,\Gamma_2;B})}}_{\mathsf{pr}_{\Gamma_2, y : B}^*(\sem{\Gamma_1,\Gamma_2, y : B;\ul{C}})}))}^>>>>>>>>>>>>>>>>>>>>>{\qquad\qquad\dscomment{(*)}}^>>>>>>>>>>>>>>>>>>>>>{\quad\qquad\qquad\qquad\dscomment{\text{split Beck-Chevalley}}}^>>>>>>>>>>>>>>>>>>>>>{\quad\qquad\qquad\qquad\qquad\qquad\qquad\qquad\dscomment{U \text{ is split fibred}}}
\\
&
\\
\txt<12pc>{
$U((\mathsf{s}(\mathsf{pr}_{\Gamma_2}^*(\sem{\Gamma_1,\Gamma_2;V})))^*(\pi^*_{\mathsf{pr}_{\Gamma_2}^*(\sem{\Gamma_1,\Gamma_2;B})}($
\\
$\Sigma_{\mathsf{pr}_{\Gamma_2}^*(\sem{\Gamma_1,\Gamma_2;B})}(\mathsf{pr}_{\Gamma_2, y : B}^*(\sem{\Gamma_1,\Gamma_2, y \!:\! B;\ul{C}})))))$
} 
\ar[r]^-{=} \ar[d]_-{=}^<<<<<{\qquad\qquad\dscomment{\mathsf{s}(\mathsf{pr}_{\Gamma_2}^*(\sem{\Gamma_1,\Gamma_2;V})) \text{ is a section of } \pi_{\mathsf{pr}_{\Gamma_2}^*(\sem{\Gamma_1,\Gamma_2;B})}}}^>>>>>>{\,\,\,\,\,\quad\qquad\qquad\dscomment{\mathsf{s}(\sem{\Gamma_1,\Gamma_2;V}) \text{ is a section of } \pi_{\sem{\Gamma_1,\Gamma_2;B}}}}
&
\txt<10.5pc>{
$\mathsf{pr}_{\Gamma_2}^*(U((\mathsf{s}(\sem{\Gamma_1,\Gamma_2;V}))^*(\pi^*_{\sem{\Gamma_1,\Gamma_2;B}}($
\\
$\Sigma_{\Gamma_1,\Gamma_2;B}(\sem{\Gamma_1,\Gamma_2,y \!:\! B;\ul{C}})))))$
}
\ar[dd]^-{=}
\\
U(\Sigma_{\mathsf{pr}_{\Gamma_2}^*(\sem{\Gamma_1,\Gamma_2;B})}(\mathsf{pr}_{\Gamma_2, y : B}^*(\sem{\Gamma_1,\Gamma_2, y \!:\! B;\ul{C}}))) \ar[d]_-{=}^-{\quad\qquad\qquad\dscomment{\text{split Beck-Chevalley}}}^-{\qquad\qquad\qquad\qquad\qquad\qquad\qquad\dscomment{U \text{ is split fibred}}}
\\
U(\mathsf{pr}_{\Gamma_2}^*((\Sigma_{\Gamma_1,\Gamma_2;B}(\sem{\Gamma_1,\Gamma_2,y \!:\! B;\ul{C}}))) \ar[r]_-{=}
&
\mathsf{pr}_{\Gamma_2}^*(U(\Sigma_{\Gamma_1,\Gamma_2;B}(\sem{\Gamma_1,\Gamma_2,y \!:\! B;\ul{C}})))
}
\]

We conclude by observing that the left-hand side top-to-bottom composite morphism is equal to $\sem{\Gamma_1, x \!:\! A,\Gamma_2; \langle V , M \rangle_{(y : B).\, \ul{C}}}$, and that the right-hand side top-to-bottom composite morphism is equal to $\sproj {\Gamma_1} x A {\Gamma_2}^*(\sem{\Gamma_1,\Gamma_2;\langle V , M \rangle_{(y : B).\, \ul{C}}})$. 

In the above diagram, and in other cases of this proof, we use $(*)$ to refer to the following commuting diagram:
\[
\scriptsize
\xymatrix@C=6.5em@R=6em@M=0.5em{
&
\\
\sem{\Gamma_1,\Gamma_2} \ar[r]_-{\eta^{1 \,\dashv\, \ia -}_{\sem{\Gamma_1,\Gamma_2}}} \ar@/^3pc/[rrr]^-{\mathsf{s}(\sem{\Gamma_1,\Gamma_2;V})} \ar@{}[u]_<<<<<<{\!\!\!\qquad\qquad\qquad\qquad\qquad\qquad\qquad\qquad\qquad\dscomment{\text{def. of } \mathsf{s}(\sem{\Gamma_1,\Gamma_2;V})}} & \ia {1_{\sem{\Gamma_1,\Gamma_2}}} \ar[r]^-{\ia {\sem{\Gamma_1,\Gamma_2;V}}} & \ia {\sem{\Gamma_1,\Gamma_2;B}} \ar[r]^-{=} & \sem{\Gamma_1,\Gamma_2,y \!:\! B}
\\
& \ia {\mathsf{pr}_{\Gamma_2}^*(1_{\sem{\Gamma_1,\Gamma_2}})} \ar[u]_<<<<{\ia {\overline{\mathsf{pr}_{\Gamma_2}}(1_{\sem{\Gamma_1,\Gamma_2}})}}^<<<<<<<{\dscomment{1 \text{ is s. fib.}}\,\,\,}_-{\quad\qquad\dscomment{\text{def. of } \mathsf{pr}_{\Gamma_2}^*(\sem{\Gamma_1,\Gamma_2;V})}} \ar[r]_-{\ia {\mathsf{pr}_{\Gamma_2}^*(\sem{\Gamma_1,\Gamma_2;V})}} & \ia {\mathsf{pr}_{\Gamma_2}^*(\sem{\Gamma_1,\Gamma_2;B})} \ar[u]^>>>>{\ia {\overline{\mathsf{pr}_{\Gamma_2}}(\sem{\Gamma_1,\Gamma_2;B})}} &
\\
\sem{\Gamma_1, x \!:\! A, \Gamma_2} \ar[r]^-{\eta^{1 \,\dashv\, \ia -}_{\sem{\Gamma_1, x : A,\Gamma_2}}} \ar[uu]^-{\mathsf{pr}_{\Gamma_2}}_>>>>>>>>>>>{\,\,\,\quad\dscomment{\text{nat. of } \eta^{1 \,\dashv\, \ia -}}} \ar@/_3pc/[rrr]_-{\mathsf{s}(\sem{\Gamma_1, x : A, \Gamma_2;V})} \ar@{}[d]^<<<<<{\quad\qquad\qquad\qquad\qquad\qquad\qquad\qquad\qquad\dscomment{\text{def. of } \mathsf{s}(\sem{\Gamma_1, x \!:\! A,\Gamma_2;V})}} & \ia {1_{\sem{\Gamma_1, x : A, \Gamma_2}}} \ar[r]_-{\ia {\sem{\Gamma_1, x : A,\Gamma_2;V}}} \ar[u]^-{=}_-{\,\,\,\qquad\qquad\qquad\dscomment{(c)}} \ar@/^4pc/[uu]^-{\ia {1(\mathsf{pr}_{\Gamma_2})}} & \ia {\sem{\Gamma_1, x \!:\! A, \Gamma_2;B}} \ar[r]_-{=} \ar[u]_-{=} & \sem{\Gamma_1, x \!:\! A, \Gamma_2, y \!:\! B} \ar[uu]_-{\mathsf{pr}_{\Gamma_2, y : B}}^-{\dscomment{\text{def. of } \mathsf{pr}_{\Gamma_2, y : B}}\qquad\quad}
\\
&
}
\vspace{-0.5cm}
\]

\vspace{0.2cm}
\noindent
\textbf{Computational pattern-matching for computation terms:}
In this case, we assume that
\[
\sem{\Gamma_1,\Gamma_2; \doto M {(y \!:\! B, z \!:\! \ul{C})} {\ul{D}} K} : 1_{\sem{\Gamma_1,\Gamma_2}} \longrightarrow U(\sem{\Gamma_1,\Gamma_2; \ul{D}})
\]
and we need to show that
\[
\begin{array}{c}
\hspace{-6.5cm}
\sem{\Gamma_1, x \!:\! A,\Gamma_2; \doto M {(y \!:\! B, z \!:\! \ul{C})} {\ul{D}} K} = 
\\
\hspace{-3cm}
\sproj {\Gamma_1} x A {\Gamma_2}^*(\sem{\Gamma_1,\Gamma_2; \doto M {(y \!:\! B, z \!:\! \ul{C})} {\ul{D}} K}) 
\\
\hspace{5.5cm}
: 1_{\sem{\Gamma_1, x : A,\Gamma_2}} \longrightarrow U(\sproj {\Gamma_1} x A {\Gamma_2}^*(\sem{\Gamma_1,\Gamma_2; \ul{D}}))
\end{array}
\]

First, by inspecting the definition of $\sem{-}$ for $\doto M {(y \!:\! B, z \!:\! \ul{C})} {\ul{D}} K$, the assumption gives us that
\[
\begin{array}{c}
\sem{\Gamma_1,\Gamma_2; M} : 1_{\sem{\Gamma_1,\Gamma_2}} \longrightarrow U(\Sigma_{\sem{\Gamma_1,\Gamma_2;B}}(\sem{\Gamma_1,\Gamma_2,y \!:\! B;\ul{C}}))
\\[3mm]
\sem{\Gamma_1,\Gamma_2, y \!:\! B; z \!:\! \ul{C}; K} : \sem{\Gamma_1,\Gamma_2,y \!:\! B;\ul{C}} \longrightarrow \pi^*_{\sem{\Gamma_1,\Gamma_2;B}}(\sem{\Gamma_1,\Gamma_2; \ul{D}})
\end{array}
\]
on which we can use the induction hypothesis and $(e)$, respectively, to get that
\[
\begin{array}{c}
\hspace{-6cm}
\sem{\Gamma_1, x \!:\! A,\Gamma_2;M} = \sproj {\Gamma_1} x A {\Gamma_2}^*(\sem{\Gamma_1,\Gamma_2;M}) 
\\
\hspace{3.5cm}
: 1_{\sem{\Gamma_1, x : A,\Gamma_2}} \longrightarrow U(\sproj {\Gamma_1} x A {\Gamma_2}^*(\Sigma_{\sem{\Gamma_1,\Gamma_2;B}}(\sem{\Gamma_1,\Gamma_2,y \!:\! B;\ul{C}})))
\\[4mm]
\hspace{-2.4cm}
\sem{\Gamma_1, x \!:\! A,\Gamma_2, y \!:\! B; z \!:\! \ul{C};K} = \sproj {\Gamma_1} x A {\Gamma_2, y : B}^*(\sem{\Gamma_1,\Gamma_2, y \!:\! B; z \!:\! \ul{C};K})
\\
\hspace{1cm}
: \sproj {\Gamma_1} x A {\Gamma_2, y : B}^*(\sem{\Gamma_1,\Gamma_2,y \!:\! B;\ul{C}}) \longrightarrow \sproj {\Gamma_1} x A {\Gamma_2, y : B}^*(\pi^*_{\sem{\Gamma_1,\Gamma_2;B}}(\sem{\Gamma_1,\Gamma_2; \ul{D}}))
\end{array}
\vspace{0.25cm}
\]

Finally, we show that 
\[
\begin{array}{c}
\hspace{-5.5cm}
\sem{\Gamma_1, x \!:\! A,\Gamma_2; \doto M {(y \!:\! B, z \!:\! \ul{C})} {\ul{D}} K} = 
\\
\hspace{-2cm}
\sproj {\Gamma_1} x A {\Gamma_2}^*(\sem{\Gamma_1,\Gamma_2; \doto M {(y \!:\! B, z \!:\! \ul{C})} {\ul{D}} K}) 
\\
\hspace{5.5cm}
: 1_{\sem{\Gamma_1, x : A,\Gamma_2}} \longrightarrow U(\sproj {\Gamma_1} x A {\Gamma_2}^*(\sem{\Gamma_1,\Gamma_2; \ul{D}}))
\end{array}
\]
by proving that the next diagram commutes, in which we write $\mathsf{pr}_{\Gamma_2}$ for $\sproj {\Gamma_1} x A {\Gamma_2}$ and $\mathsf{pr}_{\Gamma_2, y : B}$ for $\sproj {\Gamma_1} x A {\Gamma_2, y : B}$. For better readability, we aggregate small proof steps.

\vspace{0.3cm}
\[
\scriptsize
\xymatrix@C=2em@R=5em@M=0.5em{
1_{\sem{\Gamma_1, x : A, \Gamma_2}} 
\ar[r]^-{=} \ar[d]_-{\sem{\Gamma_1, x : A, \Gamma_2;M}}^<<<<{\qquad\qquad\dscomment{\text{use of the induction hypothesis on } \sem{\Gamma_1,\Gamma_2;M} }}^>>>>>{\qquad\qquad\qquad\qquad\dscomment{U \text{ is split fibred}}}
& 
\mathsf{pr}_{\Gamma_2}^*(1_{\sem{\Gamma_1,\Gamma_2}})
\ar[d]^-{\mathsf{pr}_{\Gamma_2}^*(\sem{\Gamma_1,\Gamma_2;M})}
\\
U(\mathsf{pr}_{\Gamma_2}^*(\Sigma_{\sem{\Gamma_1,\Gamma_2;B}}(\sem{\Gamma_1,\Gamma_2, y \!:\! B;\ul{C}})))
\ar[r]^-{=} \ar[d]_-{=}^-{\,\,\,\,\,\,\,\qquad\qquad\dscomment{\text{split Beck-Chevalley}}}^-{\,\,\,\,\,\,\,\qquad\qquad\qquad\qquad\qquad\qquad\dscomment{U \text{ is split fibred}}}
&
\mathsf{pr}_{\Gamma_2}^*(U(\Sigma_{\sem{\Gamma_1,\Gamma_2;B}}(\sem{\Gamma_1,\Gamma_2,y \!:\! B;\ul{C}})))
\ar[dd]^-{\mathsf{pr}_{\Gamma_2}^*(U(\Sigma_{\sem{\Gamma_1,\Gamma_2;B}}(\sem{\Gamma_1,\Gamma_2, y : B;z : \ul{C}; K})))}
\\
U(\Sigma_{\mathsf{pr}_{\Gamma_2}^*(\sem{\Gamma_1,\Gamma_2;B})}(\mathsf{pr}^*_{\Gamma_2,y : B}(\sem{\Gamma_1,\Gamma_2,y \!:\! B;\ul{C}})))
\ar[d]_-{U(\Sigma_{\mathsf{pr}_{\Gamma_2}^*(\sem{\Gamma_1,\Gamma_2;B})}(\sem{\Gamma_1, x : A,\Gamma_2, y : B; z : \ul{C}; K}))}^-{\,\,\,\,\,\,\,\qquad\dscomment{\text{use of the induction hypothesis on } \sem{\Gamma_1, x \!:\! A, \Gamma_2; y \!:\! B; K}}}
\\
U(\Sigma_{\mathsf{pr}_{\Gamma_2}^*(\sem{\Gamma_1,\Gamma_2;B})}(\mathsf{pr}^*_{\Gamma_2,y : B}(\pi^*_{\sem{\Gamma_1,\Gamma_2;B}}(\sem{\Gamma_1,\Gamma_2; \ul{D}})))) \ar[r]^-{=} \ar[d]_-{=}^<<<<{\,\,\,\,\,\,\,\qquad\qquad\dscomment{\text{def. of } \mathsf{pr}_{\Gamma_2, y : B}}}^<<<<{\,\,\,\,\,\,\,\qquad\qquad\qquad\qquad\qquad\dscomment{\mathcal{P}(\overline{\mathsf{pr}_{\Gamma_2}}(\sem{\Gamma_1,\Gamma_2;B}))}}^>>>>>{\,\,\,\,\,\,\,\,\,\,\,\quad\qquad\qquad\qquad\dscomment{U \text{ is split fibred}}}
&
\mathsf{pr}_{\Gamma_2}^*(U(\Sigma_{\sem{\Gamma_1,\Gamma_2;B}}(\pi^*_{\sem{\Gamma_1,\Gamma_2;B}}(\sem{\Gamma_1,\Gamma_2; \ul{D}})))) \ar[dd]^-{\mathsf{pr}_{\Gamma_2}^*(U(\varepsilon^{\Sigma_{\sem{\Gamma_1,\Gamma_2;B}} \,\dashv\, \pi^*_{\sem{\Gamma_1,\Gamma_2;B}}}_{\sem{\Gamma_1,\Gamma_2; \ul{D}}}))}
\\
U(\Sigma_{\mathsf{pr}_{\Gamma_2}^*(\sem{\Gamma_1,\Gamma_2;B})}(\pi^*_{\mathsf{pr}_{\Gamma_2}^*(\sem{\Gamma_1,\Gamma_2;B})}(\mathsf{pr}_{\Gamma_2}^*(\sem{\Gamma_1,\Gamma_2; \ul{D}})))) \ar[d]_-{U(\varepsilon^{\Sigma_{\mathsf{pr}_{\Gamma_2}^*(\sem{\Gamma_1,\Gamma_2;B})} \,\dashv\, \pi^*_{\mathsf{pr}_{\Gamma_2}^*(\sem{\Gamma_1,\Gamma_2;B})}}_{\mathsf{pr}^*_{\Gamma_2}(\sem{\Gamma_1,\Gamma_2; \ul{D}})})}^<<<{\,\,\,\,\,\,\,\,\,\,\,\,\,\,\,\qquad\qquad\qquad\dscomment{\text{split Beck-Chevalley}}}^>>>>>{\,\,\,\,\,\,\,\qquad\qquad\dscomment{\text{Proposition~\ref{prop:BCfordepcompsums} for } \varepsilon^{\Sigma_{\sem{\Gamma_1,\Gamma_2;B}} \,\dashv\, \pi^*_{\sem{\Gamma_1,\Gamma_2;B}}}}}
\\
U(\mathsf{pr}_{\Gamma_2}^*(\sem{\Gamma_1,\Gamma_2; \ul{D}}))
\ar[r]_-{=}
&
\mathsf{pr}_{\Gamma_2}^*(U(\sem{\Gamma_1,\Gamma_2; \ul{D}}))
}
\vspace{0.3cm}
\]

To conclude, we observe that the left-hand side top-to-bottom composite morphism is equal to $\sem{\Gamma_1, x \!:\! A,\Gamma_2; \doto M {(y \!:\! B, z \!:\! \ul{C})} {\ul{D}} K}$, and that the right-hand side top-to-bottom composite morphism is equal to $\sproj {\Gamma_1} x A {\Gamma_2}^*(\sem{\Gamma_1,\Gamma_2;\doto M {(y \!:\! B, z \!:\! \ul{C})} {\ul{D}} K})$. 

\vspace{0.2cm}
\noindent
\textbf{Computational lambda abstraction for computation terms:}
In this case, we assume that 
\[
\sem{\Gamma_1,\Gamma_2; \lambda \, y \!:\! B .\, M} : 1_{\sem{\Gamma_1,\Gamma_2}} \longrightarrow U(\Pi_{\sem{\Gamma_1,\Gamma_2;B}}(\ul{C}))
\]

\pagebreak\noindent
and we need to show that
\[
\begin{array}{c}
\hspace{-3.5cm}
\sem{\Gamma_1, x \!:\! A,\Gamma_2; \lambda \, y \!:\! B .\, M} = 
\sproj {\Gamma_1} x A {\Gamma_2}^*(\sem{\Gamma_1,\Gamma_2; \lambda \, y \!:\! B .\, M}) 
\\
\hspace{5.75cm}
: 1_{\sem{\Gamma_1, x : A,\Gamma_2}} \longrightarrow U(\sproj {\Gamma_1} x A {\Gamma_2}^*(\Pi_{\sem{\Gamma_1,\Gamma_2;B}}(\ul{C})))
\end{array}
\]

First, by inspecting the definition of $\sem{-}$ for $\lambda \, y \!:\! B .\, M$, the assumption gives us that
\[
\sem{\Gamma_1,\Gamma_2, y \!:\! B ; M} : 1_{\sem{\Gamma_1,\Gamma_2, y : B}} \longrightarrow U(\ul{C})
\]

Next, by using the induction hypothesis on this morphism, we get that
\[
\begin{array}{c}
\hspace{-3.25cm}
\sem{\Gamma_1, x \!:\! A,\Gamma_2, y \!:\! B ; M} = 
\sproj {\Gamma_1} x A {\Gamma_2, y : B}^*(\sem{\Gamma_1,\Gamma_2, y \!:\! B ; M}) 
\\
\hspace{6.75cm}
: 1_{\sem{\Gamma_1, x : A,\Gamma_2, y : B}} \longrightarrow U(\sproj {\Gamma_1} x A {\Gamma_2, y : B}^*(\ul{C}))
\end{array}
\]

Finally, we show that
\[
\begin{array}{c}
\hspace{-3.15cm}
\sem{\Gamma_1, x \!:\! A,\Gamma_2; \lambda \, y \!:\! B .\, M} = 
\sproj {\Gamma_1} x A {\Gamma_2}^*(\sem{\Gamma_1,\Gamma_2; \lambda \, y \!:\! B .\, M}) 
\\
\hspace{5.75cm}
: 1_{\sem{\Gamma_1, x : A,\Gamma_2}} \longrightarrow U(\sproj {\Gamma_1} x A {\Gamma_2}^*(\Pi_{\sem{\Gamma_1,\Gamma_2;B}}(\ul{C})))
\end{array}
\]
by proving that the next diagram commutes, in which we write $\mathsf{pr}_{\Gamma_2}$ for $\sproj {\Gamma_1} x A {\Gamma_2}$ and $\mathsf{pr}_{\Gamma_2, y : B}$ for $\sproj {\Gamma_1} x A {\Gamma_2, y : B}$. For better readability, we aggregate small proof steps.

\vspace{0.3cm}
\[
\scriptsize
\xymatrix@C=2em@R=7em@M=0.5em{
1_{\sem{\Gamma_1, x : A, \Gamma_2}} 
\ar[r]^{=} \ar[d]_-{\eta^{\pi^*_{\mathsf{pr}_{\Gamma_2}^*(\sem{\Gamma_1,\Gamma_2;B})} \,\dashv\, \Pi_{\mathsf{pr}_{\Gamma_2}^*(\sem{\Gamma_1,\Gamma_2;B})}}_{1_{\sem{\Gamma_1, x : A, \Gamma_2}}}}^-{\!\!\!\!\!\!\!\qquad\qquad\dscomment{\text{Proposition~\ref{prop:BCfordepproducts} for } \eta^{\pi^*_{\sem{\Gamma_1,\Gamma_2;B}} \,\dashv\, \Pi_{\sem{\Gamma_1,\Gamma_2;B}}}}}
& 
\mathsf{pr}_{\Gamma_2}^*(1_{\sem{\Gamma_1,\Gamma_2}})
\ar[d]^-{\mathsf{pr}^*(\eta^{\pi^*_{\sem{\Gamma_1,\Gamma_2;B}} \,\dashv\, \Pi_{\sem{\Gamma_1,\Gamma_2;B}}}_{1_{\sem{\Gamma_1,\Gamma_2}}})}
\\
\Pi_{\mathsf{pr}_{\Gamma_2}^*(\sem{\Gamma_1,\Gamma_2;B})}(\pi^*_{\mathsf{pr}_{\Gamma_2}^*(\sem{\Gamma_1,\Gamma_2;B})}(1_{\sem{\Gamma_1, x : A, \Gamma_2}})) \ar[r]^-{=} \ar[d]_-{=}^-{\quad\qquad\dscomment{\text{split Beck-Chevalley}}}^-{\quad\qquad\qquad\qquad\qquad\qquad\dscomment{1 \text{ is split fibred}}}
&
\mathsf{pr}_{\Gamma_2}^*(\Pi_{\sem{\Gamma_1,\Gamma_2;B}}(\pi^*_{\sem{\Gamma_1,\Gamma_2;B}}(1_{\sem{\Gamma_1,\Gamma_2}}))) \ar[d]^-{=}
\\
\Pi_{\mathsf{pr}_{\Gamma_2}^*(\sem{\Gamma_1,\Gamma_2;B})}(1_{\sem{\Gamma_1, x : A, \Gamma_2, y : B}}) \ar[r]^-{=} \ar[d]_-{\Pi_{\mathsf{pr}_{\Gamma_2}^*(\sem{\Gamma_1,\Gamma_2;B})}(\sem{\Gamma_1, x : A, \Gamma_2, y : B; M})}^<<<<<<{\!\!\!\!\qquad\dscomment{\text{use of the induction hypothesis on } \sem{\Gamma_1, \Gamma_2, y \!:\! B; M}}}^>>>>>>>{\qquad\qquad\qquad\dscomment{\text{split Beck-Chevalley}}}
&
\mathsf{pr}_{\Gamma_2}^*(\Pi_{\sem{\Gamma_1,\Gamma_2;B}}(1_{\sem{\Gamma_1,\Gamma_2,y : B}})) \ar[d]^-{\mathsf{pr}_{\Gamma_2}^*(\Pi_{\sem{\Gamma_1,\Gamma_2;B}}(\sem{\Gamma_1,\Gamma_2, y : B; M}))}
\\
\Pi_{\mathsf{pr}_{\Gamma_2}^*(\sem{\Gamma_1,\Gamma_2;B})}(U(\mathsf{pr}_{\Gamma_2, y : B}^*(\ul{C}))) \ar[r]^-{=} \ar[d]_-{(\zeta^{-1}_{\Pi,\mathsf{pr}^*_{\Gamma_2}(\sem{\Gamma_1,\Gamma_2;B})})_{\mathsf{pr}_{\Gamma_2, y : B}^*(\ul{C})}}^>>>>>>>{\quad\qquad\dscomment{\text{split Beck-Chevalley}}}^>>>>>>>{\quad\qquad\qquad\qquad\qquad\qquad\dscomment{U \text{ is split fibred}}}^<<<<<<{\quad\dscomment{\text{expanding the defs. of } \zeta^{-1}_{\Pi,\mathsf{pr}^*_{\Gamma_2}(\sem{\Gamma_1,\Gamma_2;B})} \text{ and } \zeta^{-1}_{\Pi,\sem{\Gamma_1,\Gamma_2;B}}}}
&
\mathsf{pr}_{\Gamma_2}^*(\Pi_{\sem{\Gamma_1,\Gamma_2;B}}(U(\ul{C}))) \ar[d]^-{\mathsf{pr}_{\Gamma_2}^*((\zeta^{-1}_{\Pi,\sem{\Gamma_1,\Gamma_2;B}})_{\ul{C}})}
\\
U(\Pi_{\mathsf{pr}_{\Gamma_2}^*(\sem{\Gamma_1,\Gamma_2;B})}(\mathsf{pr}_{\Gamma_2, y : B}^*(\ul{C}))) \ar[r]_-{=}
&
\mathsf{pr}_{\Gamma_2}^*(U(\Pi_{\sem{\Gamma_1,\Gamma_2;B}}(\ul{C})))
}
\vspace{0.3cm}
\]

\pagebreak
We conclude by observing that the left-hand side top-to-bottom composite morphism is equal to $\sem{\Gamma_1, x \!:\! A,\Gamma_2; \lambda \, y \!:\! B .\, M}$, and that the right-hand side top-to-bottom composite morphism
is equal to $\sproj {\Gamma_1} x A {\Gamma_2}^*(\sem{\Gamma_1,\Gamma_2;\lambda \, y \!:\! B .\, M})$. 

\vspace{0.2cm}
\noindent
\textbf{Computational function application for computation terms:}
In this case, we assume that 
\[
\sem{\Gamma_1,\Gamma_2; M(V)_{(y : B).\,\ul{C}}} : 1_{\sem{\Gamma_1,\Gamma_2}} \longrightarrow U((\mathsf{s}(\sem{\Gamma_1,\Gamma_2;V}))^*(\sem{\Gamma_1,\Gamma_2,y \!:\! B;\ul{C}}))
\]
and we need to show that
\[
\begin{array}{c}
\hspace{-3.5cm}
\sem{\Gamma_1, x \!:\! A,\Gamma_2; M(V)_{(y : B).\,\ul{C}}} = 
\sproj {\Gamma_1} x A {\Gamma_2}^*(\sem{\Gamma_1,\Gamma_2; M(V)_{(y : B).\,\ul{C}}}) 
\\
\hspace{2.5cm}
: 1_{\sem{\Gamma_1, x : A,\Gamma_2}} \longrightarrow U(\sproj {\Gamma_1} x A {\Gamma_2}^*((\mathsf{s}(\sem{\Gamma_1,\Gamma_2;V}))^*(\sem{\Gamma_1,\Gamma_2,y \!:\! B;\ul{C}})))
\end{array}
\]

First, by inspecting the definition of $\sem{-}$ for $M(V)_{(y : B).\,\ul{C}}$, the assumption gives us that
\[
\begin{array}{c}
\sem{\Gamma_1,\Gamma_2; V} : 1_{\sem{\Gamma_1,\Gamma_2}} \longrightarrow \sem{\Gamma_1,\Gamma_2;B}
\\[3mm]
\sem{\Gamma_1,\Gamma_2; M} : 1_{\sem{\Gamma_1,\Gamma_2}} \longrightarrow U(\Pi_{\sem{\Gamma_1,\Gamma_2;B}}(\sem{\Gamma_1,\Gamma_2,y \!:\! B;\ul{C}}))
\end{array}
\]
on which we can use $(c)$ and the induction hypothesis, respectively, to get that
\[
\begin{array}{c}
\sem{\Gamma_1, x \!:\! A,\Gamma_2; V} = 
\sproj {\Gamma_1} x A {\Gamma_2}^*(\sem{\Gamma_1,\Gamma_2; V}) 
: 1_{\sem{\Gamma_1, x : A,\Gamma_2}} \longrightarrow \sproj {\Gamma_1} x A {\Gamma_2}^*(\sem{\Gamma_1,\Gamma_2;B})
\\[3mm]
\hspace{-6.8cm}
\sem{\Gamma_1, x \!:\! A,\Gamma_2; M} = 
\sproj {\Gamma_1} x A {\Gamma_2}^*(\sem{\Gamma_1,\Gamma_2; M}) 
\\
\hspace{3.8cm}
: 1_{\sem{\Gamma_1, x : A,\Gamma_2}} \longrightarrow U(\sproj {\Gamma_1} x A {\Gamma_2}^*(\Pi_{\sem{\Gamma_1,\Gamma_2;B}}(\sem{\Gamma_1,\Gamma_2,y \!:\! B;\ul{C}})))
\end{array}
\]

Finally, we show that 
\[
\begin{array}{c}
\hspace{-2.5cm}
\sem{\Gamma_1, x \!:\! A,\Gamma_2; M(V)_{(y : B).\,\ul{C}}} = 
\sproj {\Gamma_1} x A {\Gamma_2}^*(\sem{\Gamma_1,\Gamma_2; M(V)_{(y : B).\,\ul{C}}}) 
\\
\hspace{2.5cm}
: 1_{\sem{\Gamma_1, x : A,\Gamma_2}} \longrightarrow U(\sproj {\Gamma_1} x A {\Gamma_2}^*((\mathsf{s}(\sem{\Gamma_1,\Gamma_2;V}))^*(\sem{\Gamma_1,\Gamma_2,y \!:\! B;\ul{C}})))
\end{array}
\]
by proving that the next diagram commutes, in which we write $\mathsf{pr}_{\Gamma_2}$ for $\sproj {\Gamma_1} x A {\Gamma_2}$ and $\mathsf{pr}_{\Gamma_2, y : B}$ for $\sproj {\Gamma_1} x A {\Gamma_2, y : B}$. For better readability, we aggregate small proof steps.
\[
\scriptsize
\xymatrix@C=5em@R=6.5em@M=0.5em{
1_{\sem{\Gamma_1,x:A,\Gamma_2}} 
\ar[r]^-{=} \ar[d]_-{=}^<<<<<<{\!\!\!\!\!\!\!\!\qquad\qquad\qquad\dscomment{\mathsf{s}(\sem{\Gamma_1, x \!:\! A, \Gamma_2;V}) \text{ is a section of } \pi_{\mathsf{pr}_{\Gamma_2}^*(\sem{\Gamma_1,\Gamma_2;B})}}}^>>>>>>>{\!\!\!\!\quad\qquad\qquad\qquad\dscomment{\mathsf{s}(\sem{\Gamma_1, \Gamma_2;V}) \text{ is a section of } \pi_{\sem{\Gamma_1,\Gamma_2;B}}}}
& 
\mathsf{pr}_{\Gamma_2}^*(1_{\sem{\Gamma_1,\Gamma_2}})
\ar[d]^-{=}
\\
(\mathsf{s}(\sem{\Gamma_1, x \!:\! A, \Gamma_2;V}))^*(\pi^*_{\mathsf{pr}_{\Gamma_2}^*(\sem{\Gamma_1,\Gamma_2;B})}(1_{\sem{\Gamma_1,x:A,\Gamma_2}})) 
\ar[r]^-{=} \ar[dd]^<<<<<<{(\mathsf{s}(\sem{\Gamma_1, x : A, \Gamma_2;V}))^*(\pi^*_{\mathsf{pr}_{\Gamma_2}^*(\sem{\Gamma_1,\Gamma_2;B})}(\sem{\Gamma_1,x : A, \Gamma_2;M}))}^<<<<<<<<<<<<<{\!\!\!\!\!\!\!\!\qquad\qquad\qquad\dscomment{\text{use of the induction hypothesis on } \sem{\Gamma_1,\Gamma_2;M}}}^<<<<<<<<<<<<<<<<<<<{\,\,\,\,\,\,\,\qquad\dscomment{(*) \text{ from above}}}^<<<<<<<<<<<<<<<<<<<{\,\,\,\,\,\,\,\,\,\,\,\,\,\qquad\qquad\qquad\qquad\dscomment{\text{def. of } \mathsf{pr}_{\Gamma_2, y : B}}}^<<<<<<<<<<<<<<<<<<<{\,\,\,\,\,\,\,\,\,\,\,\,\,\,\,\qquad\qquad\qquad\qquad\qquad\qquad\qquad\dscomment{\mathcal{P}(\overline{\mathsf{pr}_{\Gamma_2}}(\sem{\Gamma_1,\Gamma_2;B}))}}^<<<<<<<<<<<<<<<<<<<<<<<<<{\,\,\,\,\,\,\,\,\,\,\,\qquad\qquad\qquad\qquad\dscomment{U \text{ is split fibred}}}
&
\mathsf{pr}_{\Gamma_2}^*((\mathsf{s}(\sem{\Gamma_1, \Gamma_2;V}))^*(\pi^*_{\sem{\Gamma_1,\Gamma_2;B}}(1_{\sem{\Gamma_1,\Gamma_2}})))
\ar[dd]_>>>>>>{\mathsf{pr}_{\Gamma_2}^*((\mathsf{s}(\sem{\Gamma_1, \Gamma_2;V}))^*(\pi^*_{\sem{\Gamma_1,\Gamma_2;B}}(\sem{\Gamma_1,\Gamma_2,y : B;M})))}
\\
&
\\
\txt<12pc>{
$(\mathsf{s}(\sem{\Gamma_1, x \!:\! A, \Gamma_2;V}))^*(\pi^*_{\mathsf{pr}_{\Gamma_2}^*(\sem{\Gamma_1,\Gamma_2;B})}($
\\
$U(\mathsf{pr}_{\Gamma_2}^*(\Pi_{\sem{\Gamma_1,\Gamma_2;B}}(\sem{\Gamma_1,\Gamma_2,y \!:\! B;\ul{C}})))))$
}
\ar[r]^-{=} \ar[d]_-{=}^<<<<<<{\,\,\,\,\,\,\,\qquad\dscomment{(*) \text{ from above}}}^<<<<<<{\,\,\,\,\,\,\,\,\,\,\,\,\,\qquad\qquad\qquad\qquad\dscomment{\text{def. of } \mathsf{pr}_{\Gamma_2, y : B}}}^<<<<<<{\,\,\,\,\,\,\,\,\,\,\,\,\,\,\,\qquad\qquad\qquad\qquad\qquad\qquad\qquad\dscomment{\mathcal{P}(\overline{\mathsf{pr}_{\Gamma_2}}(\sem{\Gamma_1,\Gamma_2;B}))}}^>>>>>>>{\,\,\,\,\,\,\,\,\,\,\,\qquad\qquad\qquad\qquad\dscomment{U \text{ is split fibred}}}
&
\txt<12pc>{
$\mathsf{pr}_{\Gamma_2}^*((\mathsf{s}(\sem{\Gamma_1, \Gamma_2;V}))^*(\pi^*_{\sem{\Gamma_1,\Gamma_2;B}}($
\\
$U(\Pi_{\sem{\Gamma_1,\Gamma_2;B}}(\sem{\Gamma_1,\Gamma_2,y \!:\! B;\ul{C}})))))$
}
\ar[d]^-{=}
\\
\txt<12pc>{
$U((\mathsf{s}(\sem{\Gamma_1, x \!:\! A, \Gamma_2;V}))^*(\pi^*_{\mathsf{pr}_{\Gamma_2}^*(\sem{\Gamma_1,\Gamma_2;B})}($
\\
$\mathsf{pr}_{\Gamma_2}^*(\Pi_{\sem{\Gamma_1,\Gamma_2;B}}(\sem{\Gamma_1,\Gamma_2,y \!:\! B;\ul{C}})))))$
}
\ar[r]^-{=} \ar[d]_-{=}^<<<<<<{\,\,\,\,\,\,\,\qquad\dscomment{(*) \text{ from above}}}^<<<<<<{\,\,\,\,\,\,\,\,\,\,\,\,\,\qquad\qquad\qquad\qquad\dscomment{\text{def. of } \mathsf{pr}_{\Gamma_2, y : B}}}^<<<<<<{\,\,\,\,\,\,\,\,\,\,\,\,\,\,\,\qquad\qquad\qquad\qquad\qquad\qquad\qquad\dscomment{\mathcal{P}(\overline{\mathsf{pr}_{\Gamma_2}}(\sem{\Gamma_1,\Gamma_2;B}))}}^>>>>>>>{\,\,\,\,\,\,\,\,\,\,\,\qquad\qquad\qquad\qquad\dscomment{U \text{ is split fibred}}}
&
\txt<12pc>{
$\mathsf{pr}_{\Gamma_2}^*(U((\mathsf{s}(\sem{\Gamma_1, \Gamma_2;V}))^*(\pi^*_{\sem{\Gamma_1,\Gamma_2;B}}($
\\
$\Pi_{\sem{\Gamma_1,\Gamma_2;B}}(\sem{\Gamma_1,\Gamma_2,y \!:\! B;\ul{C}})))))$
}
\ar[ddd]_>>>>>>>{\mathsf{pr}_{\Gamma_2}^*(U((\mathsf{s}(\sem{\Gamma_1, \Gamma_2;V}))^*(\varepsilon^{\pi^*_{\sem{\Gamma_1,\Gamma_2;B}} \,\dashv\, \Pi_{\sem{\Gamma_1,\Gamma_2;B}}}_{\sem{\Gamma_1,\Gamma_2,y \!:\! B;\ul{C}}})))}
\\
\txt<12pc>{
$U((\mathsf{s}(\sem{\Gamma_1, x \!:\! A, \Gamma_2;V}))^*(\mathsf{pr}_{\Gamma_2, y : B}^*($
\\
$\pi^*_{\sem{\Gamma_1,\Gamma_2;B}}(\Pi_{\sem{\Gamma_1,\Gamma_2;B}}(\sem{\Gamma_1,\Gamma_2,y \!:\! B;\ul{C}})))))$
}
\ar[dd]^<<<<<{U((\mathsf{s}(\sem{\Gamma_1, x \!:\! A, \Gamma_2;V}))^*(\mathsf{pr}_{\Gamma_2, y : B}^*(\varepsilon^{\pi^*_{\sem{\Gamma_1,\Gamma_2;B}} \,\dashv\, \Pi_{\sem{\Gamma_1,\Gamma_2;B}}}_{\sem{\Gamma_1,\Gamma_2,y \!:\! B;\ul{C}}})))}^-{\qquad\qquad\qquad\dscomment{\text{Proposition~\ref{prop:BCfordepproducts} for } \varepsilon^{\pi^*_{\sem{\Gamma_1,\Gamma_2;B}} \,\dashv\, \Pi_{\sem{\Gamma_1,\Gamma_2;B}}}}}
\\
&
\\
U((\mathsf{s}(\sem{\Gamma_1, x \!:\! A, \Gamma_2;V}))^*(\mathsf{pr}_{\Gamma_2, y : B}^*(\sem{\Gamma_1,\Gamma_2,y \!:\! B;\ul{C}})))
\ar[r]_-{=}
&
\mathsf{pr}_{\Gamma_2}^*(U((\mathsf{s}(\sem{\Gamma_1, \Gamma_2;V}))^*(\sem{\Gamma_1,\Gamma_2,y \!:\! B;\ul{C}})))
}
\]

We conclude by observing that the left-hand side top-to-bottom composite morphism is equal to $\sem{\Gamma_1, x \!:\! A,\Gamma_2; M(V)_{(y : B).\,\ul{C}}}$, and that the right-hand side top-to-bottom\linebreak

\pagebreak\noindent 
composite morphism is equal to $\sproj {\Gamma_1} x A {\Gamma_2}^*(\sem{\Gamma_1,\Gamma_2;M(V)_{(y : B).\,\ul{C}}})$. 

\vspace{0.2cm}
\noindent
\textbf{Forcing a thunked computation:}
We omit the proof for this case because it is analogous to the case for thunking a computation.

\vspace{0.2cm}
\noindent
\textbf{Homomorphic function application for computation terms:}
In this case, we assume that
\[
\sem{\Gamma_1,\Gamma_2; V(M)_{\ul{C}, \ul{D}}} : 1_{\sem{\Gamma_1,\Gamma_2}} \longrightarrow U(\sem{\Gamma_1,\Gamma_2;\ul{D}})
\]
and we need to show that
\[
\begin{array}{c}
\hspace{-4cm}
\sem{\Gamma_1, x \!:\! A,\Gamma_2; V(M)_{\ul{C}, \ul{D}}} = \sproj {\Gamma_1} x A {\Gamma_2}^*(\sem{\Gamma_1,\Gamma_2; V(M)_{\ul{C}, \ul{D}}}) 
\\
\hspace{6.5cm}
: 1_{\sem{\Gamma_1, x : A,\Gamma_2}} \longrightarrow U(\sproj {\Gamma_1} x A {\Gamma_2}^*(\sem{\Gamma_1,\Gamma_2;\ul{D}}))
\end{array}
\]

First, by inspecting the definition of $\sem{-}$ for $V(M)_{\ul{C}, \ul{D}}$, the assumption gives us
\[
\begin{array}{c}
\sem{\Gamma_1,\Gamma_2; V} : 1_{\sem{\Gamma_1,\Gamma_2}} \longrightarrow \sem{\Gamma_1,\Gamma_2;\ul{C}} \multimap \sem{\Gamma_1,\Gamma_2;\ul{D}}
\\[3mm]
\sem{\Gamma_1,\Gamma_2; M} : 1_{\sem{\Gamma_1,\Gamma_2}} \longrightarrow U(\sem{\Gamma_1,\Gamma_2;\ul{C}})
\end{array}
\]
on which we can use $(c)$ and the induction hypothesis, respectively, to get that
\[
\begin{array}{c}
\sem{\Gamma_1, x : A,\Gamma_2; V} : 1_{\sem{\Gamma_1, x : A,\Gamma_2}} \longrightarrow \sproj {\Gamma_1} x A {\Gamma_2}^*(\sem{\Gamma_1,\Gamma_2;\ul{C}} \multimap \sem{\Gamma_1,\Gamma_2;\ul{D}})
\\[3mm]
\sem{\Gamma_1, x : A,\Gamma_2; M} : 1_{\sem{\Gamma_1, x : A,\Gamma_2}} \longrightarrow U(\sproj {\Gamma_1} x A {\Gamma_2}^*(\sem{\Gamma_1,\Gamma_2;\ul{C}}))
\end{array}
\]

Finally, we show that 
\[
\begin{array}{c}
\hspace{-3cm}
\sem{\Gamma_1, x \!:\! A,\Gamma_2; V(M)_{\ul{C}, \ul{D}}} = \sproj {\Gamma_1} x A {\Gamma_2}^*(\sem{\Gamma_1,\Gamma_2; V(M)_{\ul{C}, \ul{D}}}) 
\\
\hspace{6.5cm}
: 1_{\sem{\Gamma_1, x : A,\Gamma_2}} \longrightarrow U(\sproj {\Gamma_1} x A {\Gamma_2}^*(\sem{\Gamma_1,\Gamma_2;\ul{D}}))
\end{array}
\]

\pagebreak\noindent
by proving that the next diagram commutes, in which we write $\mathsf{pr}$ for $\sproj {\Gamma_1} x A {\Gamma_2}$. To improve the readability of this diagram, we aggregate small proof steps.

\[
\hspace{0.5cm}
\xymatrix@C=11.5em@R=8em@M=0.5em{
1_{\sem{\Gamma_1, x : A, \Gamma_2}} \ar[r]^-{\sem{\Gamma_1, x : A,\Gamma_2; V(M)_{\ul{C}, \ul{D}}}} \ar[d]^-{\sem{\Gamma_1,x : A, \Gamma_2;M}}^<<<{\,\,\,\quad\qquad\dcomment{\text{def. of } \sem{\Gamma_1, x \!:\! A,\Gamma_2; V(M)_{\ul{C}, \ul{D}}}}}^<<<<<<<<<<<<{\qquad\qquad\qquad\dcomment{(c)}} \ar@/_6pc/[ddd]^-{=} & U(\mathsf{pr}^*(\sem{\Gamma_1,\Gamma_2;\ul{D}}))
\\
U(\mathsf{pr}^*(\sem{\Gamma_1,\Gamma_2;\ul{C}})) \ar[ur]_<<<<<<<<<<<<<<<<{\quad\qquad\qquad\qquad U(\xi_{\sem{\Gamma_1,\Gamma_2},\mathsf{pr}^*(\sem{\Gamma_1,\Gamma_2;\ul{C}}),\mathsf{pr}^*(\sem{\Gamma_1,\Gamma_2;\ul{D}})}(\mathsf{pr}^*(\sem{\Gamma_1,\Gamma_2;V})))} \ar[d]_-{=}_-{\dcomment{\text{use of i.h.}}\quad}^-{\qquad\dcomment{U \text{ is split fibred}}}^-{\qquad\qquad\qquad\qquad\qquad\dcomment{\xi \text{ is preserved on-the-nose by reindexing}}}
\\
\mathsf{pr}^*(U(\sem{\Gamma_1,\Gamma_2;\ul{C}})) \ar[dr]^<<<<<<<<<<<<<<<<{\qquad\qquad\qquad\mathsf{pr}^*(U(\xi_{\sem{\Gamma_1,\Gamma_2}, \sem{\Gamma_1,\Gamma_2;\ul{C}}, \sem{\Gamma_1,\Gamma_2;\ul{D}}}(\sem{\Gamma_1,\Gamma_2;V})))}
\\
\mathsf{pr}^*(1_{\sem{\Gamma_1,\Gamma_2}}) \ar[r]_-{\mathsf{pr}^*(\sem{\Gamma_1,\Gamma_2; V(M)_{\ul{C}, \ul{D}}})} \ar[u]_-{\mathsf{pr}^*(\sem{\Gamma_1,\Gamma_2;M})}_<<<<<<{\quad\qquad\dcomment{\text{def. of } \sem{\Gamma_1,\Gamma_2; V(M)_{\ul{C}, \ul{D}}}}} & \mathsf{pr}^*(U(\sem{\Gamma_1,\Gamma_2;\ul{D}})) \ar@/_6pc/[uuu]^-{=}
}
\]

\vspace{0.2cm}
\noindent
\textbf{Computation variables:}
In this case, we assume that
\[
\sem{\Gamma_1,\Gamma_2;z \!:\! \ul{C};z} : \sem{\Gamma_1,\Gamma_2;\ul{C}} \longrightarrow \sem{\Gamma_1,\Gamma_2;\ul{C}}
\]
and we need to show that 
\[
\begin{array}{c}
\hspace{-4.5cm}
\sem{\Gamma_1, x \!:\! A,\Gamma_2;z \!:\! \ul{C};z} = \sproj {\Gamma_1} x A {\Gamma_2}^*(\sem{\Gamma_1,\Gamma_2;z \!:\! \ul{C};z}) 
\\
\hspace{4.5cm}
: \sproj {\Gamma_1} x A {\Gamma_2}^*(\sem{\Gamma_1,\Gamma_2;\ul{C}}) \longrightarrow \sproj {\Gamma_1} x A {\Gamma_2}^*(\sem{\Gamma_1,\Gamma_2;\ul{C}})
\end{array}
\]

\pagebreak 

First, by inspecting the definition of $\sem{-}$ for $z$, the assumption also gives us that 
\[
\sem{\Gamma_1,\Gamma_2;\ul{C}} \in \mathcal{C}_{\sem{\Gamma_1,\Gamma_2}}
\]

Next, by using $(b)$ on this object, we get that
\[
\sem{\Gamma_1, x \!:\! A, \Gamma_2; \ul{C}} = \sproj {\Gamma_1} x A {\Gamma_2}^*(\sem{\Gamma_1,\Gamma_2;\ul{C}}) \in \mathcal{C}_{\sem{\Gamma_1, x : A,\Gamma_2}}
\]

Next, by using the functoriality of $\sproj {\Gamma_1} x A {\Gamma_2}^*$, we get that
\[
\begin{array}{c}
\hspace{-4.5cm}
\id_{\sem{\Gamma_1, x : A, \Gamma_2; \ul{C}}} 
=
\sproj {\Gamma_1} x A {\Gamma_2}^*(\id_{\sem{\Gamma_1,\Gamma_2;\ul{C}}})
\\
\hspace{4cm}
: \sproj {\Gamma_1} x A {\Gamma_2}^*(\sem{\Gamma_1,\Gamma_2;\ul{C}}) \longrightarrow \sproj {\Gamma_1} x A {\Gamma_2}^*(\sem{\Gamma_1,\Gamma_2;\ul{C}})
\end{array}
\]

Finally, by using the definition of $\sem{-}$ for $z$, we get that
\[
\begin{array}{c}
\hspace{-4cm}
\sem{\Gamma_1, x \!:\! A,\Gamma_2;z \!:\! \ul{C};z} = \sproj {\Gamma_1} x A {\Gamma_2}^*(\sem{\Gamma_1,\Gamma_2;z \!:\! \ul{C};z}) 
\\
\hspace{4.5cm}
: \sproj {\Gamma_1} x A {\Gamma_2}^*(\sem{\Gamma_1,\Gamma_2;\ul{C}}) \longrightarrow \sproj {\Gamma_1} x A {\Gamma_2}^*(\sem{\Gamma_1,\Gamma_2;\ul{C}})
\end{array}
\]

\vspace{0.1cm}
\noindent
\textbf{Other cases for homomorphism terms:}
We omit the cases for sequential composition, computational pairing, computational pattern-matching, computational lambda abstraction, and homomorphic function application for homomorphism terms because they are analogous to the corresponding cases for computation terms discussed above. 
\end{proof}

\newpage

\section{Proof of Proposition~\ref{prop:semweakeningandsubstitutioncommuting}}
\label{sect:proofofprop:semweakeningandsubstitutioncommuting}

{
\renewcommand{\thetheorem}{\ref{prop:semweakeningandsubstitutioncommuting}}
\begin{proposition}
Given value contexts $\Gamma_1$ and $\Gamma_2$, value variables $x$ and $y$, value types $A$ and $B$, and a value term $V$ such that $\sem{\Gamma_1,\Gamma_2[V/y]} \in \mathcal{B}$, $\sem{\Gamma_1, y \!:\! B, \Gamma_2} \in \mathcal{B}$, $\sem{\Gamma_1, x \!:\! A,\Gamma_2[V/y]} \in \mathcal{B}$, $\sem{\Gamma_1, x \!:\! A, y \!:\! B,\Gamma_2} \in \mathcal{B}$, and $\sem{\Gamma_1; V} : 1_{\sem{\Gamma_1}} \longrightarrow \sem{\Gamma_1;B}$, then 
\[
\ssubst {\Gamma_1} {y} {B} {\Gamma_2} {V} \comp \sproj {\Gamma_1} {x} {A} {\Gamma_1[V/y]}
=
\sproj {\Gamma_1} {x} {A} {y : B, \Gamma_2} \comp \ssubst {\Gamma_1, x : A} y B {\Gamma_2} V 
\]
\end{proposition}
\addtocounter{theorem}{-1}
}

\begin{proof}
We prove this equation by induction on the length of $\Gamma_2$. Both the base case and the step case of induction are proved similarly, by straightforward diagram chasing.

\noindent
\textit{Base case (with $\Gamma_2 = \diamond$):}
%
\[
\scriptsize
\xymatrix@C=5em@R=6em@M=0.5em{
\sem{\Gamma_1, x \!:\! A} 
\ar[r]^-{\sproj {\Gamma_1} {x} {A} {\diamond}}
\ar@/^0.5pc/[dr]^<<<<<{=}
\ar@/_1.5pc/[ddr]_>>>>>>>>>>>>>>{\eta^{1 \,\dashv\, \ia -}_{\sem{\Gamma_1, x \!:\! A}}\!\!\!\!\!}^-{\,\,\,\,\,\,\dscomment{\text{iso.}}}
\ar@/_4.75pc/[dddd]_<<<<<<<{\ssubst {\Gamma_1, x : A} {y} {B} {\diamond} {V}}
\ar@/^1pc/[ddd]_-{\mathsf{s}(\sem{\Gamma_1, x : A; V})}_>>>>>>>>>>>>>{\dscomment{\text{def. of } \ssubst {\Gamma_1, x : A} {y} {B} {\diamond} {V}}}^>>>>>>>>>>>>{\!\!\!\!\quad\dscomment{\text{def. of } \mathsf{s}(\sem{\Gamma_1, x : A; V})}}
& 
\sem{\Gamma_1} 
\ar[rr]^-{\ssubst {\Gamma_1} {y} {B} {\diamond} {V}}
\ar@/^1pc/[dr]^>>>>>>>{\!\!\!\!\!\!\!\eta^{1 \,\dashv\, \ia -}_{\sem{\Gamma_1}}}_-{\dscomment{\text{iso.}}\,\,\,\,\,\,\,}^>{\dscomment{\text{def. of } \mathsf{s}(\sem{\Gamma_1;V})}}
\ar@/^1pc/[drr]^>>>>>>>>>>>{\,\,\,\,\,\,\mathsf{s}(\sem{\Gamma_1;V})}
&&
\sem{\Gamma_1, y \!:\! B}
\ar[d]^-{=}_<<<{\dscomment{\text{def. of } \ssubst {\Gamma_1} {y} {B} {\diamond} {V}}\qquad}
\\
&
\ia {\sem{\Gamma_1; A}}
\ar[u]^-{\pi_{\sem{\Gamma_1;A}}\!\!\!}^>>>{\dscomment{\text{def. of } {\sproj {\Gamma_1} {x} {A} {\diamond}}}\,\,\,\,\,}
&
\ia {1_{\sem{\Gamma_1}}}
\ar[r]_-{\ia {\sem{\Gamma_1; V}}}
\ar@/^1pc/[ul]^-{\pi_{1_{\sem{\Gamma_1}}}\!\!\!\!}
&
\ia {\sem{\Gamma_1; B}}
\ar@/^0.5pc/[dddd]_-{\id_{\ia {\sem{\Gamma_1; B}}}}
\\
&
\ia {1_{\sem{\Gamma_1, x : A}}}
\ar@/_1pc/[uul]_<<<<<<<<<<<<{\!\!\!\pi_{1_{\sem{\Gamma_1, x : A}}}}
\ar[dl]^-{\,\,\,\,\ia {\sem{\Gamma_1, x : A; V}}}^>>>>>{\qquad\qquad\qquad\dscomment{\text{Proposition~\ref{prop:semweakening2}}}}^>>>>>{\qquad\qquad\qquad\qquad\qquad\qquad\qquad\qquad\dscomment{\text{def. of } \sproj {\Gamma_1} {x} {A} {\diamond}}}
\ar[r]_-{=}
&
\ia {\pi^*_{\sem{\Gamma_1;A}}(1_{\sem{\Gamma_1}})}
\ar[d]_-{\ia {\pi^*_{\sem{\Gamma_1;A}}(\sem{\Gamma_1;V})}}
\ar[u]^-{\ia {\overline{\pi_{\sem{\Gamma_1;A}}}(1_{\sem{\Gamma_1}})}}^>{\dscomment{\mathcal{P}(\overline{\pi_{\sem{\Gamma_1;A}}}(1_{\sem{\Gamma_1}}))}\qquad\quad}_>>>>>>>>>{\!\!\!\!\quad\dscomment{\text{def. of } \pi^*_{\sem{\Gamma_1;A}}(\sem{\Gamma_1;V})}}
\\
\ia {\sem{\Gamma_1, x \!:\! A; B}}
\ar[dr]_-{=}
&
&
\ia {\pi^*_{\sem{\Gamma_1;A}}(\sem{\Gamma_1;B})}
\ar[ddr]_-{\ia {\overline{\pi_{\sem{\Gamma_1;A}}}(\sem{\Gamma_1;B})}}^<<<<<<<<<<{\quad\qquad\dscomment{\text{id. law}}}
\ar@/_1.5pc/[uur]_<<<<<<<<<<{\!\!\!\!\!\ia {\overline{\pi_{\sem{\Gamma_1;A}}}(\sem{\Gamma_1;B})}}
\\
\sem{\Gamma_1, x \!:\! A, y \!:\! B}
\ar[u]^-{=}
\ar[d]_-{\sproj {\Gamma_1} {x} {A} {y : B}}^<<<<{\,\,\,\,\dscomment{\text{def. of } \sproj {\Gamma_1} {x} {A} {y : B}}}
&
\ia {\sproj {\Gamma_1} {x} {A} {\diamond}^*(\sem{\Gamma_1; B})}
\ar[dl]^-{\,\,\,\,\,\,\,\,\,\,\,\ia {\overline{\sproj {\Gamma_1} {x} {A} {\diamond}}(\sem{\Gamma_1;B})}}^<<<{\qquad\qquad\qquad\qquad\qquad\dscomment{\text{def. of } \sproj {\Gamma_1} {x} {A} {\diamond}}}
\ar[ur]_-{=}
\\
\sem{\Gamma_1, y \!:\! B}
\ar[rrr]_-{=}
&
&
&
\ia {\sem{\Gamma_1;B}}
}
\vspace{0.25cm}
\]

\pagebreak

\noindent
\textit{Step case (with $\Gamma_2 = \Gamma'_2, x_n \!:\! A_n$):}
%
\[
\hspace{0.65cm}
\scriptsize
\xymatrix@C=2em@R=6em@M=0.5em{
\sem{\Gamma_1, x \!:\! A, \Gamma'_2[V/y], x_n \!:\! A_n[V/y]}
\ar@/_5.5pc/[dddd]_-{=}
\ar@/_8.5pc/[ddddddd]^>>>>>>>>>>>>>>>>>>>>>>>>>>>>>>>>>>>>>>>>>>>>>>>>>{\ssubst {\Gamma_1, x : A} {y} {B} {\Gamma'_2, x_n : A_n} {V}}
\ar[rr]^-{\sproj {\Gamma_1} {x} {A} {\Gamma'_2, x_n : A_n}}
\ar[d]^-{=}^>>>>>{\quad\qquad\qquad\qquad\qquad\dscomment{\text{def. of } \sproj {\Gamma_1} {x} {A} {\Gamma'_2, x_n : A_n}}}^<<<<<{\!\!\quad\quad\qquad\qquad\qquad\qquad\dscomment{\text{Proposition~\ref{prop:semweakening2}}}}
&
&
\sem{\Gamma_1, \Gamma'_2[V/y], x_n \!:\! A_n[V/y]]}
\ar@/^4.5pc/[ddd]_-{=}
\ar@/^9pc/[ddddddd]_-{\ssubst {\Gamma_1} {y} {B} {\Gamma'_2, x_n : A_n} {V}}
\\
\txt<7pc>{
$\{\sproj {\Gamma_1} {x} {A} {\Gamma'_2[V/y]}^*($\\$\sem{\Gamma_1, \Gamma'_2[V/y]; A_n[V/y]})\}$
}
\ar[d]^-{=}^-{\qquad\qquad\qquad\qquad\qquad\qquad\dscomment{\text{Proposition~\ref{prop:semsubstitution2}}}}_-{\dscomment{p \text{ is a s. fib.}}\quad}
\ar[rr]_-{\ia {\overline{\sproj {\Gamma_1} {x} {A} {\Gamma'_2[V/y]}}(\sem{\Gamma_1, \Gamma'_2[V/y]; A_n[V/y]})}}
&
&
\ia {\sem{\Gamma_1, \Gamma'_2[V/y]; A_n[V/y]}}
\ar[u]_-{=}
\\
\txt<7pc>{
$\{\sproj {\Gamma_1} {x} {A} {\Gamma'_2[V/y]}^*($\\$\ssubst {\Gamma_1} {y} {B} {\Gamma_2} {V}^*($\\$\sem{\Gamma_1, \Gamma'_2[V/y]; A_n[V/y]}))\}$
}
\ar[d]^-{=}_-{\dscomment{\text{i.h.}}\qquad\,\,\,}
\ar[drr]^<<<<<<<<<<<<<{\quad\qquad\qquad\qquad\ia {\overline{\sproj {\Gamma_1} {x} {A} {\Gamma'_2[V/y]}}(\ssubst {\Gamma_1} {y} {B} {\Gamma_2} {V}^*(\sem{\Gamma_1, \Gamma'_2[V/y]; A_n[V/y]}))}}
\\
\txt<7pc>{
$\{\ssubst {\Gamma_1, x : A} {y} {B} {\Gamma'_2} {V}^*($\\$\sproj {\Gamma_1} {x} {A} {y : B, \Gamma'_2}^*($\\$\sem{\Gamma_1, y \!:\! B, \Gamma'_2;A_n}))\}$
}
\ar[d]^-{=}^-{\qquad\qquad\dscomment{p \text{ is a split fibration}}}^-{\qquad\qquad\qquad\qquad\qquad\qquad\qquad\dscomment{\text{induction hypothesis}}}
&
&
\txt<7pc>{
$\{\ssubst {\Gamma_1} {y} {B} {\Gamma'_2} {V}^*($\\$\sem{\Gamma_1, y \!:\! B, \Gamma'_2; A_n})\}$
}
\ar[dd]_-{\ia {\overline{\ssubst {\Gamma_1} {y} {B} {\Gamma'_2} {V}}(\sem{\Gamma_1, y : B, \Gamma'_2; A_n})}}^-{\qquad\dscomment{\text{def. of } \ssubst {\Gamma_1} {y} {B} {\Gamma'_2, x_n : A_n} {V}}}
\\
\txt<7pc>{
$
\{\ssubst {\Gamma_1, x : A} {y} {B} {\Gamma'_2} {V}^*($\\$\sem{\Gamma_1, x \!:\! A, y \!:\! B, \Gamma_2;A_n})\}
$}
\ar[dd]^<<<<<<<<<<<{\ia {\overline{\ssubst {\Gamma_1, x : A} {y} {B} {\Gamma'_2} {V}}(\sem{\Gamma_1, x : A, \Gamma'_2; A_n})}}_<<<<<<{\dscomment{\text{def. of } \ssubst {\Gamma_1, x : A} {y} {B} {\Gamma'_2, x_n : A_n} {V}}\,\,\,\,\,}
\\
&&
\ia {\sem{\Gamma_1, y \!:\! B, \Gamma'_2; A_n}}
\ar[dd]^-{=}
\\
\ia {\sem{\Gamma_1, x \!:\! A, y \!:\! B, \Gamma'_2; A_n}}
\ar[d]^-{=}^>>>>>{\qquad\qquad\qquad\qquad\dscomment{\text{def. of } \sproj {\Gamma_1} {x} {A} {y : B, \Gamma'_2, x_n : A_n}}}^<<<<<<<<{\,\,\,\,\quad\qquad\qquad\qquad\qquad\dscomment{\text{Proposition~\ref{prop:semweakening2}}}}
\ar[r]_-{=}
&
\txt<5.5pc>{
$\{\sproj {\Gamma_1} {x} {A} {y : B, \Gamma'_2}^*($\\$\sem{\Gamma_1, y \!:\! B, \Gamma'_2;A_n})\}$
}
\ar[ur]^-{\ia {\overline{\sproj {\Gamma_1} {x} {A} {y : B, \Gamma'_2}}(\sem{\Gamma_1, y : B, \Gamma'_2; A_n})}\,\,\,\,\,\,\,\,\,}
&
\\
\sem{\Gamma_1, x \!:\! A, y \!:\! B, \Gamma'_2, x_n \!:\! A_n}
\ar[rr]_-{\sproj {\Gamma_1} {x} {A} {y : B, \Gamma'_2, x_n : A_n}}
&
&
\sem{\Gamma_1, y \!:\! B, \Gamma'_2; x_n : A_n}
}
\]
\end{proof}

\newpage 

\section{Proof of Proposition~\ref{prop:reindexingalongkappaandpairing}}
\label{sect:proofofprop:reindexingalongkappaandpairing}

{
\renewcommand{\thetheorem}{\ref{prop:reindexingalongkappaandpairing}}
\begin{proposition}
Given a value context $\Gamma$, value variables $x_1$, $x_2$, and $y$, and \linebreak value types $A_1$, $A_2$, and $B$ such that $x_2 \not\in V\!ars(\Gamma) \cup \{y\}$, $\sem{\Gamma} \in \mathcal{B}$, $\sem{\Gamma;A} \in \mathcal{V}_{\sem{\Gamma}}$, \linebreak $\sem{\Gamma, x_1 \!:\! A_1;A_2} \in \mathcal{V}_{\sem{\Gamma, x_1 \!:\! A_1}}$, and $\sem{\Gamma, y \!:\! (\Sigma\, x_1 \!:\! A_1 .\, A_2) ; B} \in \mathcal{V}_{\sem{\Gamma, y : (\Sigma\, x_1 : A_1 .\, A_2)}}$, then we have
\[
\sem{\Gamma, x_1 \!:\! A_1, x_2 \!:\! A_2, B[\langle x_1, x_2 \rangle/y]} = \kappa_{\sem{\Gamma; A_1},\sem{\Gamma, x_1 : A_1; A_2}}^*(\sem{\Gamma, y \!:\! (\Sigma\, x_1 \!:\! A_1 .\, A_2); B}) 
\]
\end{proposition}
\addtocounter{theorem}{-1}
}

\begin{proof}
We begin by noting that both sides of this equation can be rewritten as follows.

On the one hand, the left-hand side of this equation can be rewritten as
\[
\begin{array}{c}
\hspace{-8cm}
(\mathsf{s}(\sem{\Gamma, x_1 \!:\! A_1, x_2 \!:\! A_2; \langle x_1 , x_2 \rangle}))^*(
\\
\hspace{-4cm}
\ia{\overline{\pi_{\sem{\Gamma, x_1 : A_1 ; A_2}}}(\sem{\Gamma, x_1 \!:\! A_1; \Sigma\, x_1 \!:\! A_1 .\, A_2})}^*(
\\
\hspace{4cm}
\ia{\overline{\pi_{\sem{\Gamma;A_1}}}(\sem{\Gamma; \Sigma\, x_1 \!:\! A_1 .\, A_2})}^*(\sem{\Gamma, y \!:\! (\Sigma\, x_1 \!:\! A_1 .\, A_2); B})))
\end{array}
\]
based on Propositions~\ref{prop:semweakening2} and~\ref{prop:semsubstitution2}, and the definition of morphisms $\sproj {\Gamma_1} {x} {A} {\Gamma_2}$.

On the other hand, the right-hand side of this equation can be rewritten as
\[
\ia {\eta^{\Sigma_{\sem{\Gamma; A_1}} \,\dashv\, \pi^*_{\sem{\Gamma; A_1}}}_{\sem{\Gamma, x_1 : A_1; A_2}}}^*(\ia {\overline{\pi_{\sem{\Gamma;A_1}}}(\sem{\Gamma; \Sigma\, x_1 \!:\! A_1 .\, A_2})}^*(\sem{\Gamma, y \!:\! (\Sigma\, x_1 \!:\! A_1 .\, A_2); B}))
\]
based on the definitions of $\kappa_{\sem{\Gamma; A_1},\sem{\Gamma, x_1 : A_1; A_2}}$ and $\sem{\Gamma;\Sigma\, x_1 \!:\! A_1 .\, A_2}$.

Now, as a result of $p : \mathcal{V} \longrightarrow \mathcal{B}$ being a split fibration, it suffices to show 
\[
\begin{array}{c}
\ia{\overline{\pi_{\sem{\Gamma, x_1 : A_1 ; A_2}}}(\sem{\Gamma, x_1 \!:\! A_1; \Sigma\, x_1 \!:\! A_1 .\, A_2})} 
\comp \mathsf{s}(\sem{\Gamma, x_1 \!:\! A_1, x_2 \!:\! A_2; \langle x_1 , x_2 \rangle})
\\
=
\\
\ia {\eta^{\Sigma_{\sem{\Gamma; A_1}} \,\dashv\, \pi^*_{\sem{\Gamma; A_1}}}_{\sem{\Gamma, x_1 : A_1; A_2}}}
\end{array}
\]
for the required equation to be true, which follows from the commutativity of the following diagram:
\[
\scriptsize
\xymatrix@C=5em@R=5em@M=0.5em{
\sem{\Gamma, x_1, x_2}
\ar[d]^-{\eta^{1 \,\dashv\, \ia -}_{\sem{\Gamma, x_1, x_2}}}^>>>>{\quad\qquad\qquad\qquad\qquad\dscomment{(*)}}
\ar@/_8pc/[dddddd]_<<<<<<<<<<<<{\mathsf{s}(\sem{\Gamma, x_1, x_2; \langle x_1, x_2 \rangle})}
\ar[r]^-{=}
&
\ia {\sem{\Gamma, x_1; A_2}}
\ar[ddddddd]^-{\ia {\eta^{\Sigma_{\sem{\Gamma;A_1}} \,\dashv\, \pi^*_{\sem{\Gamma; A_1}}}_{\sem{\Gamma, x_1 ; A_2}}}}
\\
\ia {1_{\sem{\Gamma, x_1, x_2}}}
\ar[d]^-{\ia {\eta^{\Sigma_{\sem{\Gamma, x_1;A_2}} \,\dashv\, \pi^*_{\sem{\Gamma, x_1; A_2}}}_{1_{\sem{\Gamma, x_1, x_2}}}}}
\\
\ia {\pi^*_{\sem{\Gamma, x_1;A_2}}(\Sigma_{\sem{\Gamma, x_1; A_2}}(1_{\sem{\Gamma, x_1, x_2}}))}
\ar[d]_-{\ia {\pi^*_{\sem{\Gamma, x_1;A_2}}(\mathsf{fst})}}
\\
\ia {\pi^*_{\sem{\Gamma, x_1;A_2}}(\sem{\Gamma, x_1;A_2})}
\ar[d]_-{=}_-{\dscomment{\text{def. of } \mathsf{s}(\sem{\Gamma, x_1, x_2; \langle x_1, x_2 \rangle})}\quad\,\,\,\,}
\ar@/_2pc/[uuur]_<<<<<<<<<<<<<{\,\,\,\,\,\ia {\overline{\pi_{\sem{\Gamma, x_1;A_2}}}(\sem{\Gamma, x_1;A_2})}}
\\
\txt<10pc>{
$\{(\mathsf{s}(\sem{\Gamma, x_1, x_2; x_1}))^*( $\\$ \sem{\Gamma, x_1, x_2, x'_1; A_2[x'_1/x_1]})\}$
}
\ar[d]_-{\dshide{\ia {(\mathsf{s}(\sem{\Gamma, x_1, x_2; x_1}))^*(\eta^{\Sigma_{\sem{\Gamma, x_1, x_2;A_1}} \,\dashv\, \pi^*_{\sem{\Gamma, x_1, x_2;A_1}}}_{\sem{\Gamma, x_1, x_2, x'_1; A_2[x'_1/x_1]}})}}}^-{\qquad\qquad\qquad\qquad\qquad\dscomment{(**)}}
\\
\txt<10pc>{
$\{(\mathsf{s}(\sem{\Gamma, x_1, x_2; x_1}))^*($\\$ \pi^*_{\sem{\Gamma, x_1, x_2;A_1}}( $\\$ \Sigma_{\Gamma, x_1, x_2; A_1}( $\\$ \hspace{0.5cm}\sem{\Gamma, x_1, x_2, x'_1; A_2[x'_1/x_1]})))\}$
}
\ar[d]_-{=}
\\
\ia {\Sigma_{\sem{\Gamma, x_1, x_2; A_1}}(\sem{\Gamma, x_1, x_2, x'_1; A_2[x'_1/x_1]})}
\ar[d]_-{=}
\\
\txt<10pc>{
$\{\pi^*_{\sem{\Gamma, x_1; A_2}}(\Sigma_{\sem{\Gamma, x_1; A_1}}( $\\$ \sem{\Gamma, x_1, x'_1; A_2[x'_1/x_1]}))\}$
}
\ar[d]_-{\ia {\overline{\pi_{\sem{\Gamma, x_1; A_2}}}(\Sigma_{\sem{\Gamma, x_1; A_1}}(\sem{\Gamma, x_1, x'_1; A_2[x'_1/x_1]}))}}
& 
\ia {\pi^*_{\sem{\Gamma;A_1}}(\Sigma_{\sem{\Gamma;A_1}}(\sem{\Gamma, x_1 ; A_2}))}
\\
\ia {\Sigma_{\sem{\Gamma, x_1; A_1}}(\sem{\Gamma, x_1, x'_1; A_2[x'_1/x_1]})}
\ar[r]_-{=}
&
\ia {\pi^*_{\sem{\Gamma;A_1}}(\Sigma_{\sem{\Gamma;A_1}}(\sem{\Gamma, x'_1 ; A_2[x'_1/x_1]}))}
\ar[u]_-{=}
}
\]

\pagebreak

\noindent 
where the subdiagram marked with $(*)$ commutes because we have
\[
\scriptsize
\xymatrix@C=7em@R=3.5em@M=0.5em{
\sem{\Gamma, x_1, x_2}
\ar[d]_-{\eta^{1 \,\dashv\, \ia -}_{\sem{\Gamma, x_1, x_2}}}^-{\qquad\qquad\qquad\dscomment{\eta^{1 \,\dashv\, \ia -}_{\sem{\Gamma, x_1, x_2}} \text{ and } \pi_{1_{\sem{\Gamma, x_1, x_2}}} \text{ form an isomorphism (Proposition~\ref{prop:compcatunitiso})}}}
\ar[rr]^-{=}
&&
\ia {\sem{\Gamma, x_1; A_2}}
\\
\ia {1_{\sem{\Gamma, x_1, x_2}}}
\ar@/^2pc/[ddr]^-{\!\!\!\!\!\!\!\!\id_{\ia {1_{\sem{\Gamma, x_1, x_2}}}}}
\ar[dd]_-{\ia {\eta^{\Sigma_{\sem{\Gamma, x_1;A_2}} \,\dashv\, \pi^*_{\sem{\Gamma, x_1; A_2}}}_{1_{\sem{\Gamma, x_1, x_2}}}}}^>>>>>>>{\qquad\dscomment{\text{def. of } \kappa}}
\ar[dddr]^-{\!\!\kappa_{\sem{\Gamma; A_1},\sem{\Gamma, x_1; A_2}}}^<<<<<<<<<<<<<<<<<<<<{\,\,\,\,\,\,\,\,\,\,\,\dscomment{\kappa \text{ is an iso.}}}
\\
\\
\ia {\pi^*_{\sem{\Gamma, x_1;A_2}}(\Sigma_{\sem{\Gamma, x_1; A_2}}(1_{\sem{\Gamma, x_1, x_2}}))}
\ar[dd]_-{\ia {\pi^*_{\sem{\Gamma, x_1;A_2}}(\mathsf{fst})}}
\ar[dr]_>>>>>>>>>>>>>>{\ia{\overline{\pi_{\sem{\Gamma, x_1;A_2}}}(\Sigma_{\sem{\Gamma, x_1; A_2}}(1_{\sem{\Gamma, x_1, x_2}}))}\qquad\qquad\quad}
&
\ia {1_{\sem{\Gamma, x_1,x_2}}}
\ar[uuur]_-{\!\!\!\!\pi_{1_{\sem{\Gamma, x_1, x_2}}}}_<<<<<{\quad\qquad\qquad\dscomment{\text{def. of } \mathsf{fst}}}
\\
&
\ia {\Sigma_{\sem{\Gamma, x_1; A_2}}(1_{\sem{\Gamma, x_1, x_2}})}
\ar[dr]_-{\ia {\mathsf{fst}}}_<<<<<<{\dscomment{\text{def. of } \pi^*_{\sem{\Gamma, x_1;A_2}}(\mathsf{fst})}\qquad\qquad\qquad\qquad\qquad\qquad\qquad\qquad}
\ar[u]_-{\kappa^{-1}_{\sem{\Gamma; A_1},\sem{\Gamma, x_1; A_2}}}
\\
\ia {\pi^*_{\sem{\Gamma, x_1;A_2}}(\sem{\Gamma, x_1;A_2})}
\ar[rr]_-{\ia {\overline{\pi_{\sem{\Gamma, x_1;A_2}}}(\sem{\Gamma, x_1;A_2})}}
&&
\ia {\sem{\Gamma, x_1;A_2}}
\ar[uuuuu]_-{\id_{\ia {\sem{\Gamma, x_1;A_2}}}}
}
\]
and the subdiagram marked with $(**)$ commutes because we have
\[
\scriptsize
\xymatrix@C=1em@R=4em@M=0.5em{
\ia {\pi^*_{\sem{\Gamma, x_1;A_2}}(\sem{\Gamma, x_1;A_2})}
\ar[d]_-{=}
\ar[rr]^-{\ia {\overline{\pi_{\sem{\Gamma, x_1;A_2}}}(\sem{\Gamma, x_1;A_2})}}
\ar@/^10pc/[dddddr]^<<<<<<<<<<<<<<<<<<<<<<<<<<<<<<<{\!\!\!\!\ia {\pi^*_{\sem{\Gamma, x_1; A_2}}(\eta^{\Sigma_{\sem{\Gamma;A_1}} \,\dashv\, \pi^*_{\sem{\Gamma; A_1}}}_{\sem{\Gamma, x_1 ; A_2}})}}^<<<<<<<<<<<<<<<<<{\qquad\qquad\qquad\dscomment{\text{def. of } \pi^*_{\sem{\Gamma, x_1; A_2}}(\eta^{\Sigma_{\sem{\Gamma;A_1}} \,\dashv\, \pi^*_{\sem{\Gamma; A_1}}}_{\sem{\Gamma, x_1 ; A_2}})}}
&
&
\ia {\sem{\Gamma, x_1; A_2}}
\ar[dddd]^-{\ia {\eta^{\Sigma_{\sem{\Gamma;A_1}} \,\dashv\, \pi^*_{\sem{\Gamma; A_1}}}_{\sem{\Gamma, x_1 ; A_2}}}}
\\
\txt<10pc>{
$\{(\mathsf{s}(\sem{\Gamma, x_1, x_2; x_1}))^*( $\\$ \sem{\Gamma, x_1, x_2, x'_1; A_2[x'_1/x_1]})\}$
}
\ar[d]^-{{\ia {(\mathsf{s}(\sem{\Gamma, x_1, x_2; x_1}))^*(\eta^{\Sigma_{\sem{\Gamma, x_1, x_2;A_1}} \,\dashv\, \pi^*_{\sem{\Gamma, x_1, x_2;A_1}}}_{\sem{\Gamma, x_1, x_2, x'_1; A_2[x'_1/x_1]}})}}}
\\
\txt<10pc>{
$\{(\mathsf{s}(\sem{\Gamma, x_1, x_2; x_1}))^*($\\$ \pi^*_{\sem{\Gamma, x_1, x_2;A_1}}(\Sigma_{\Gamma, x_1, x_2; A_1}( $\\$ \sem{\Gamma, x_1, x_2, x'_1; A_2[x'_1/x_1]})))\}$
}
\ar[d]_-{=}^-{\qquad\qquad\dscomment{\text{Proposition~\ref{prop:semweakening2}}}}^-{\qquad\qquad\qquad\qquad\qquad\qquad\dscomment{\text{Proposition~\ref{prop:semsubstitution2}}}}
\\
\ia {\Sigma_{\sem{\Gamma, x_1, x_2; A_1}}(\sem{\Gamma, x_1, x_2, x'_1; A_2[x'_1/x_1]})}
\ar[d]_-{=}^-{\quad\dscomment{\text{split Beck-Chevalley}}}^{\quad\qquad\qquad\qquad\qquad\dscomment{\text{defs. of } \sproj {\Gamma_1} {x} {A} {\Gamma_2} \text{ and } \ssubst {\Gamma_1} {x} {A} {\Gamma_2} V}}
\\
\txt<10pc>{
$\{\pi^*_{\sem{\Gamma, x_1; A_2}}(\Sigma_{\sem{\Gamma, x_1; A_1}}( $\\$ \sem{\Gamma, x_1, x'_1; A_2[x'_1/x_1]}))\}$
}
\ar[dd]^>>>>>{\ia {\overline{\pi_{\sem{\Gamma, x_1; A_2}}}(\Sigma_{\sem{\Gamma, x_1; A_1}}(\sem{\Gamma, x_1, x'_1; A_2[x'_1/x_1]}))}}^>>>>>>>>>>>{\qquad\qquad\dscomment{\text{Proposition~\ref{prop:semweakening2}}}}^>>>>>>>>>>>{\quad\qquad\qquad\qquad\qquad\qquad\qquad\qquad\dscomment{\text{split Beck-Chevalley}}}^>>>>>>>>>>>{\qquad\qquad\qquad\qquad\qquad\qquad\qquad\qquad\qquad\qquad\qquad\qquad\qquad\dscomment{\text{def. of } \sproj {\Gamma_1} {x} {A} {\Gamma_2}}}
\ar[dr]_-{=}
&
& 
\ia {\pi^*_{\sem{\Gamma;A_1}}(\Sigma_{\sem{\Gamma;A_1}}(\sem{\Gamma, x_1 ; A_2}))}
\\
& \txt<10pc>{
$\{\pi^*_{\sem{\Gamma, x_1; A_2}}(\pi^*_{\sem{\Gamma;A_1}}($\\$\Sigma_{\sem{\Gamma;A_1}}(\sem{\Gamma, x_1 ; A_2})))\}$
}
\ar[ur]_<<<<<<<<<<<{\!\!\!\!\qquad\qquad\ia {\overline{\pi_{\sem{\Gamma, x_1; A_2}}}(\pi^*_{\sem{\Gamma;A_1}}(\Sigma_{\sem{\Gamma;A_1}}(\sem{\Gamma, x_1 ; A_2})))}}
\\
\ia {\Sigma_{\sem{\Gamma, x_1; A_1}}(\sem{\Gamma, x_1, x'_1; A_2[x'_1/x_1]})}
\ar[rr]_-{=}
&&
\txt<5pc>{
$\{\pi^*_{\sem{\Gamma;A_1}}(\Sigma_{\sem{\Gamma;A_1}}($\\$\sem{\Gamma, x'_1 ; A_2[x'_1/x_1]}))\}$
}
\ar@/_3.75pc/[uu]_-{=}
}
\]

To improve the readability of the above proofs, we have omitted the types in value contexts, writing ${\Gamma, x_1, x_2}$ for ${\Gamma, x_1 \!:\! A_1, x_2 \!:\! A_2}$. We have also omitted details of equalities that follow from the use of the split Beck-Chevalley conditions, Propositions~\ref{prop:semweakening2} and~\ref{prop:semsubstitution2}, and the definitions of semantic projection and substitution morphisms, e.g., 
\[
\Sigma_{\sem{\Gamma, x_1,x_2;A_1}}(\sem{\Gamma, x_1, x_2, x'_1; B[x'_1/x_1]}) = \pi^*_{\sem{\Gamma, x_1; B}}(\Sigma_{\sem{\Gamma, x_1;A_1}}(\sem{\Gamma, x_1, x'_1; B[x'_1/x_1]}))
\]
\end{proof}

\renewcommand\thesection{\thechapter.\arabic{section}}

