\section{Introduction}
%
The Roper resonance was discovered in 1963 \cite{Roper:1964zza, BAREYRE1964137, AUVIL196476, PhysRevLett.13.555, PhysRev.138.B190}; and, as we shall relate, its characteristics have been the source of great puzzlement since that time.  It is therefore appropriate here to state the simplest of these characteristics; namely, the Roper is a $J=1/2$ positive-parity resonance with pole mass $\approx 1.37\,$GeV and width $\approx 0.18\,$GeV \cite{Olive:2016xmw}.  In the spectrum of nucleon-like states, \emph{i.e}.\ baryons with isospin\footnote{Isospin is a quantum number associated with strong-interaction bound-states.  Its value indicates the number of electric-charge states that may be considered as (nearly) identical in the absence of electroweak interactions, \emph{e.g}.\ the neutron and proton form an $I=1/2$ multiplet and are collectively described as nucleons.}  %% S= +1  is distinct from S= -1.
%
$I=1/2$, the Roper resonance lies about $0.4\,$GeV above the ground-state nucleon and $0.15\,$GeV below the first $J=1/2$ negative-parity state, which has roughly the same width.  Today, the levels in this spectrum are labelled thus:
%
$N({\rm mass})\,J^P$;
%
and hence the ground-state nucleon is denoted $N(940)\,1/2^+$, the Roper resonance as $N(1440)\,1/2^+$, and the negative-parity state described above is $N(1535)\,1/2^-$.

The search for an understanding of the Roper resonance is the highest profile case in a long-running effort to chart and explain the spectrum and interactions of all strong interaction bound states that are supported by the Standard Model of Particle Physics.  The importance of this effort has long been recognized and cannot be overestimated.  Indeed, baryons and their resonances play a central role in the existence of our universe and ourselves; and therefore \cite{Isgur:2000ad}: ``\emph{\ldots\ they must be at the center of any discussion of why the world we actually experience has the character it does.  I am convinced that completing this chapter in the history of science will be one of the most interesting and fruitful areas of physics for at least the next thirty years.}''

Strong interactions within the Standard Model are described by quantum chromodynamics (QCD), the theory of gluons (gauge fields) and quarks (matter fields).   QCD is conceptually simple and can be expressed compactly in just one line, with two definitions \cite{Wilczek:2000ih}; and yet, nearly four decades after its formulation, we are still seeking answers to such apparently simple questions as what is the proton's wave function and which, if any, of the known baryons is the proton's first radial excitation.  Indeed, numerous problems remain open because QCD is fundamentally different from the Standard Model's other pieces: whilst a perturbation theory exists and is a powerful tool when used in connection with high-energy QCD processes, it is essentially useless when it comes to developing an understanding of strong interaction bound states built from light quarks.\footnote{There are six known quark flavors: $u$ (up) and $d$ (down) quarks are light, with masses far less than the characteristic QCD mass-scale of $\Lambda_{\rm QCD} \approx 0.2\,$GeV; $s$ (strange) quarks lie near the boundary between light and heavy; the $c$ (charm) quarks are relatively heavy, but not heavy enough for non-relativistic approximations to be quantitatively accurate; $b$ (bottom) quarks are practically heavy; and $t$ (top) quark are so heavy that they decay via weak interactions before forming hadron bound-states.}

The study of the properties and interactions of light hadronic systems lies squarely within the purview of strong-QCD [sQCD], \emph{viz}.\ the body of experimental and theoretical methods used to probe and map the infrared domain of Standard Model physics, whereupon emergent phenomena, such as gluon and quark confinement and dynamical chiral symmetry breaking [DCSB], appear to play the dominant role in determining all observable characteristics of the theory.  The nature of sQCD, and its contemporary methods and challenges will become apparent as we recount the history of the Roper resonance and the modern developments that have enabled a coherent picture of this system to emerge and, by analogy, of an array of related resonances.


%%Features:
%%\begin{itemize}
%%\item Big Picture = confinement, DCSB, etc.
%%\item how does the Roper and the spectrum in general give us access to these problems?
%%\end{itemize}

%%Roper resonance \cite{Roper:1964zza, BAREYRE1964137, AUVIL196476, PhysRevLett.13.555, PhysRev.138.B190}

%%Naming convention: historically known as the $P_{11}(1440)$ and now denoted as $N(1440)\,1/2^+$


%%Hey Remark
%%\begin{itemize}
%%\item How wrong can two people be?
%%\item Volker is best placed to summarise the experimental changes.  But they will include filling-in the missing resonances and the discovery of exotics at LHC
%%\item I can succinctly state the evolution of theory \ldots\ replacement of non-R quark models, evolution of lQCD, growth of continuum methods in QCD = running masses, etc.
%%\end{itemize} 