\section{Formal Definitions of the Terms}\label{sec:appendix-prelim}

A \defn{spanning tree} $T$ of an undirected connected graph $G$ is a subgraph that is a tree which includes all of the vertices of $G$.
A \defn{spanning forest} of $G$ contains the union of the spanning trees of all connected components in $G$.
The \defn{lowest-common-ancestor} (LCA) query for two vertices on a rooted spanning tree requires $O(n)$ work and $O(\log n)$ depth on preprocessing, and $O(1)$ query time~\cite{berkman1993recursive,sadakane2002space}.

A \defn{connected component} of $G$ is a subgraph in which any two vertices are connected to each other by paths via edges in the graph.

A \defn{biconnected component} (also known as a block or 2-connected component) of $G$ is a maximal subgraph such that it is still connected after removing any single vertex in the subgraph.
Any connected graph decomposes into a tree of biconnected components called the block-cut tree of the graph.
The blocks are attached to each other at shared vertices called \defn{articulation points}.

A \defn{bridge} of $G$ is an edge whose deletion increases the number of connected components of the graph.
A connected graph is \defn{$k$-edge-connected} if it remains connected whenever fewer than $k$ edges are removed.
An unconnected graph is 0-edge connected; a connected graph with bridges is 1-edge-connected; and a bridge-less graph is at least 2-edge-connected.

