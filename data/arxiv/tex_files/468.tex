%\RequirePackage{lineno}
%\documentclass[pra,aps,twocolumn,preprintnumbers,superscriptaddress,showkeys,longbibliography]{revtex4-1}
\documentclass[journal=aamick,manuscript=article]{achemso}
\usepackage[version=3]{mhchem} % Formula subscripts using \ce{}

\usepackage{graphicx,epstopdf}% Include figure files
\usepackage{amssymb,amsmath}
\usepackage{dcolumn}% Align table columns on decimal point
\usepackage{bm}% bold math
\usepackage{color}
\usepackage{enumerate}
\usepackage{hyperref}
\usepackage{soul}
\usepackage{appendix}
\usepackage{fancyhdr}

\usepackage{color}

\usepackage{comment}

\usepackage[normalem]{ulem} % for strikeout

\newcommand*\mycommand[1]{\texttt{\emph{#1}}}
\renewcommand{\vec}[1]{\mbox{\mathversion{bold}$#1$}}
\renewcommand{\thepage}{S\arabic{page}} 
\renewcommand{\theequation}{S\arabic{equation}}
\renewcommand{\thetable}{S\arabic{table}}  
\renewcommand{\thefigure}{S\arabic{figure}} 



\title
  {Supporting Information\\ for\\ Strong light-matter coupling in carbon nanotubes as a route to exciton brightening}

\author{Vanik A. Shahnazaryan}
\affiliation{ITMO University, St. Petersburg 197101, Russia}

\author{Vasil A. Saroka}
\affiliation{Physics and Astronomy, University of Exeter, Stocker Road, Exeter EX4 4QL, United Kingdom}
\alsoaffiliation{Institute for Nuclear Problems, Belarusian State University, Bobruiskaya 11, 220030 Minsk, Belarus}

\author{Ivan A. Shelykh}
\affiliation{ITMO University, St. Petersburg 197101, Russia}
\alsoaffiliation{Science Institute, University of Iceland IS-107, Reykjavik, Iceland}

\author{William L. Barnes}
\affiliation{Physics and Astronomy, University of Exeter, Stocker Road, Exeter EX4 4QL, United Kingdom}

\author{Mikhail E. Portnoi}
\email{M.E.Portnoi@exeter.ac.uk}
\affiliation{ITMO University, St. Petersburg 197101, Russia}
\alsoaffiliation{Physics and Astronomy, University of Exeter, Stocker Road, Exeter EX4 4QL, United Kingdom}


\keywords{Carbon nanotubes, microcavity, exciton-polariton, ground state brightening}

\begin{document}


\section*{A: Calculation of the dipole matrix element}
\label{appDipole}
In this section we present a derivation of the matrix elements of dipole transitions within the zone-folding tight-binding approximation. For graphene-based nanostructures it is convenient to deal with the matrix elements of the velocity operator, instead of the position matrix element, which allows one to express relevant physical quantities in terms of the graphene Fermi velocity $v_F$.
These two approaches, however, are absolutely equivalent in the dipole approximation~\cite{BookLandauVolIV1997}.
The two matrix elements are related by
\begin{equation}
\label{eq:VMEvsPME}
  \vec{v}_{nm} =  \dfrac{i}{\hbar} \langle n \left| \left[ H, \vec{r} \right] \right| m \rangle = i \omega_{nm} \vec{r}_{nm}\, ,
\end{equation}
where $\omega_{n m} = (E_n - E_m)/\hbar $ is the frequency of the transition.

To calculate velocity operator matrix elements for SWCNTs in the tight-binding model we employ the so-called gradient approximation (the term ``effective mass approximation'' is used in earlier literature)~\cite{Wannier1962,Blount1962a,Dresselhaus1967,LewYanVoon1993,Lin1994,Saroka2017}. The optical selection rules resulting from such a calculation are in agreement with those in Refs.~\cite{Ajiki1994,Milosevic2003,Jiang2004,Malic2006}. Using this approach it is easy to show that the velocity matrix element dependence on the wavevector is not significant in semiconducting SWCNTs in the vicinity of the Dirac point, which corresponds to the conduction and valence band edges. Therefore, the matrix element can be reasonably approximated by a constant value equal to the Fermi velocity of electrons in graphene $v_F$.
We are interested in the transitions from the edges of the closest conduction and valence subbands, i.e., those that occur near the Dirac point, we can estimate the magnitude of the dipole moment operator matrix element as follows:

%
\begin{equation}
\label{eq:DipoleMomentOperatorMatrixElementInTheDiracPoint}
    d_{cv} = e|r_{cv}| = \dfrac{e v_F \hbar}{E_g} = \dfrac{e \dfrac{\sqrt{3} at }{2 \hbar} \hbar}{\dfrac{2 a t}{\sqrt{3} d}} = \dfrac{3 e d}{4} \, ,
\end{equation}
%
where we have used Eq.~\eqref{eq:VMEvsPME} and the fact that the bandgap energy, $E_g$, in the low-energy zone-folding approximation is given by,
\begin{eqnarray}
\label{eq:CNTBandGapsInConicalApproximation}
    E_g = 2 \hbar v_F \Delta = 4 \hbar v_F /(3 d) = 2 a t /\sqrt{3} d,
\\    \nonumber
\end{eqnarray}

\noindent where $v_F = \sqrt{3} a t /(2 \hbar)$ is the graphene Fermi velocity~\cite{Neto2009} in terms of the hopping integral $t=3.033$~eV~\cite{SaitoBook1998} and $\Delta=2/(3d)$ is the shift of the momentum quantization line from the Dirac point for any semiconducting SWCNT~\cite{Saito2000,Samsonidze2003}. %For semiconducting $(n,0)$ zigzag SWCNTs ($n$ is not divisible by 3), $E_g = 2 \pi t /( \sqrt{3} n)$.\\

Although the majority of the results presented in this paper were obtained for a zigzag structure with chirality $(10,0)$, the approach developed here is of a general character and can be applied to semiconducting nanotubes of an arbitrary configuration.  


\section*{B: The dependence of light-matter coupling rate on CNT diameter}
\label{g-d-Dep}
Here we briefly examine the CNT diameter dependence of parameters involved in the light-matter coupling rate, given by Eq.~(10) of the main text. First of all we note, that in near-resonant regime $\Delta\ll E_C(0)$, $E_C(0) \approx E_X =E_g -E_b$. As it was shown in the previous section, the bandgap scales as $E_g \propto 1/d$. From the Eq.~(6) of the main text and the following discussion one founds for the binding energy and Bohr radius: $E_b\propto 1/d$, and $a_0 \propto d$. Thus, we found that $E_C(0)\propto 1/d$. On the other hand, the distance between mirrors $L_C$ is defined through the resonant wavelength: $L_C=2\lambda_C\propto1/E_C(0)\propto d$. Hence, the factor under square root has overall $1/d^2$ dependence, which cancels out with interband transition matrix element, defined by Eq.~(\ref{eq:DipoleMomentOperatorMatrixElementInTheDiracPoint}). Thus, what it remains is to examine the exciton wave function at zero interparticle distance, reading as $\psi_{\alpha}(0) = N_{\alpha} W_{\alpha,1/2}(2\gamma d/\alpha a_0)$, where
%
\begin{equation}
    N_{\alpha}=\left( \alpha a_0 \int\limits_{2\gamma d/\alpha a_0}^\infty |W_{\alpha,1/2}(z)|^2 \mathrm{d} z \right) ^{-1/2}. 
\end{equation}
%
As it follows from the definition of Bohr radius, the factor $2\gamma d/\alpha a_0$ do not depend on tube diameter $d$. Finally, the last parameter depending on $d$ is the normalization constant $N_\alpha \propto d^{-1/2}$ via the Bohr radius $a_0$. 

\section*{C: The brightening dependence on exciton-phonon coupling rate}
\label{appRate}

%
\begin{figure}[h]
	\includegraphics[width=0.9\linewidth]{w_dep.pdf}
    \caption{The luminescence rates $\lambda$, $\lambda^0$ versus exciton-phonon coupling rate $W$. The black line corresponds to the dark regime ($\lambda^0$) and the red line to the bright regime ($\lambda$). Here $W_0 =1.5$ ns $^{-1}$ is the value used in the main text. The other parameters are $P=P_0$ and $\Omega_R=60$ meV.}
\label{fig:W_dep}
\end{figure}
%

In the  following section we discuss the impact of the exciton-phonon effective coupling rate $W$ on the efficiency of the luminescence in both bright and dark regimes. Particularly, in Fig. \ref{fig:W_dep} we plot the luminescence efficiencies $\lambda^0$, $\lambda$ as a function of $W$. We choose a logarithmic scale for the $x$ axis, varying $W$ over several orders of magnitude. 
While the value of luminescence rate in the excitonic regime (i.e. in the absence of a cavity) varies with the coupling rate $W$, it always remains in the low-efficiency regime (black line).
%As expected, the luminescence rate in the excitonic regime (i.e. in the absence of a cavity) is always in the regime of low efficiency (black line). 
On the other hand, over the whole range of $W$ plotted, the impact of the cavity strikingly increases the luminescence rate (red line).


\bibliography{library}

\end{document}

