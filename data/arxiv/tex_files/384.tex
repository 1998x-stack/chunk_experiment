\section{Another lower bound for simulated tempering}
\label{app:other}

\begin{thm}[Comparison theorem using canonical paths, \cite{diaconis1993comparison}]\label{thm:can-path}
Let $(\Om, P)$ be a finite Markov chain with stationary distribution $p$.

Suppose each pair $x,y\in \Om$, $x\ne y$ is associated with a path $\ga_{x,y}$. Define the congestion to be
$$
\rh(\ga) = \max_{z,w\in \Om, z\ne w} \ba{
\fc{\sum_{\ga_{x,y}\ni (z,w) }|\ga_{x,y}|p(x)p(y)}{p(z)P(z,w)}
}.
$$
Then
$$
\Gap(P) \ge \rc{\rh(\ga)}.
$$
\end{thm}

\begin{df}
Say that partition $\cal P$ refines $\cal Q$, written $P\sqsubseteq \cal Q$, if for every $A\in \cal P$ there exists $B\in \cal Q$ such that $A\subeq B$. 

Define a chain of partitions as $\{\cal P_i = \{A_{i,j}\}\}_{i=1}^{L}$, where each $\cal P_i$ is a refinement of $\cal P_{i-1}$:
$$
\cal P_{L} \sqsubseteq \cdots \sqsubseteq \cal P_1.
$$
\end{df}

%\Hnote{Need to redo bound (will be better!) if only transition to $\pm1$ level}

\begin{thm}\label{thm:sim-temp}
Suppose Assumptions~\ref{asm} hold.

Furthermore, suppose that $(\cal P_i)_{i=1}^L$ is a chain of partitions. %each $\cal P_{i+1}$ is a refinement of $\cal P_i$.
Define $\ga$ for the chain of partitions as 
$$
\ga((\cal P_i)_{i=1}^{L}) =\min_{1\le i_1\le i_2\le L}\min_{A\in \cal P_{i_1}}
\fc{p_{{i_1}}(A)}{p_{{i_2}}(A)}.
$$

%Let $M_i=(\Om, P_i), i\in \{0,\ldots, L-1\}$ be a sequence of reversible Markov chains with stationary distributions $p_i$. 
%
%Consider the simulated tempering chain $(\Om\times\{0,\ldots, L-1\}, P_{\st})$ with probabilities $(r_i)_{i=0}^{L-1}$. Let $r = \fc{\min(r_i)}{\max(r_i)}$.  
%
%Let $(\cal P_i)_{i=0}^{L-1}$ be a chain of partitions such that $\cal P_0=\{\Om\}$.
%
%Let $\ga = \ga ((\cal P_i)_{i=0}^{L-1})$ and $\de = \de((\cal P_i)_{i=0}^{L-1})$. 
Then 
\begin{align}
\Gap(M_{\st}) &\ge\fc{r^2\ga \de}{32L^3} \min_{1\le i\le L, A\in \cal P_i} (\Gap(M|_A)).
\end{align}
\end{thm}
\begin{proof}
Let $p_{\st}$ be the stationary distribution of $P_{\st}$.
First note that we can easily switch between $p_i$ and $p_{\st}$ using $p_{\st}(A\times \{i\}) = r_i p_i(A)$. 

Define the partition $\cal P$ on $\Om\times \{0,\ldots, l-1\}$ by 
$$
\cal P = \set{A\times \{i\}}{A\in \cal P_i}.
$$

By Theorem~\ref{thm:gap-product},
\begin{align}\label{eq:st-gap-prod}
\Gap(M_{\st}) & \ge \rc2 \Gap(\ol M_{\st}) 
\min_{B\in \cal P}\Gap(M_{\st}|_B).
\end{align}
We now lower-bound $\Gap(\ol M_{\st}) $. We will abuse notation by considering the sets $B\in \cal P$ as states in $\ol M_{\st}$, and identify a union of sets in $\cal P$ with the corresponding set of states for $\ol{M}_{\st}$. 

Consider a tree with nodes $B\in \cal P$, and edges connecting $A\times\{i\}$, $A'\times\{i-1\}$ if $A\in A'$. Designate $\Om\times \{1\}$ as the root. For $X,Y\in \cal P$, define the canonical path $\ga_{X,Y}$ to be the unique path in this tree.

Note that $|\ga_{X,Y}|\le 2(L-1)$. Given an edge $(A\times\{i\},A'\times\{i-1\})$, consider
\begin{align}\label{eq:st-path}
\fc{\sum_{\ga_{X,Y}\ni(A\times\{i\},A'\times\{i-1\})} |\ga_{X,Y}|p_{\st}(X)p_{\st}(Y)}{p_{\st}(A\times\{i\})P_{\st}(A\times\{i\},A'\times\{i-1\})}
&\le \fc{2(L-1) 2p_{\st}(S)p_{\st}(S^c)}{p_{\st}(A\times\{i\})P_{\st}(A\times\{i\},A'\times\{i-1\})}
\end{align}
where $S= A\times \{i,\ldots,L\}$ is the union of all children of $A\times \{i\}$ (including itself). This follows because the paths which go through $(A\times\{i\},A'\times\{i-1\})$ are exactly those between $X,Y$ where one of $X,Y$ is a subset of $S= A\times \{i,\ldots, L\}$ and the other is not. To upper bound~\eqref{eq:st-path}, we upper-bound $\fc{p(S)}{p(A\times \{i\})}$ and lower-bound $P(A\times \{i\}, A'\times \{i+1\})$. 

We upper-bound by definition of $\ga$,
\begin{align}
\fc{p(S)}{p(A\times \{i\})} &= \fc{\sum_{k=i}^{L} r_ip_i(A)}{r_ip_i(A)}\\
&\le\fc{\max r_i}{\min r_i}
 \fc{\sum_{k=i}^{L} p_i(A)}{p_i(A)}\\
&\le \fc L{r\ga}.
\label{eq:path-bd1}
\end{align}

%We claim that 
%$$
%p(S)\le \fc{L}{\ga}p(A\times \{i\}).
%$$
%We prove the following by induction on $j-i$: for $j>i$, given $A_i\in \cal P_i$, $A_j\in \cal P_j$, %there exists some chain $(A_i,\ldots, A_j)$ such that 
%\begin{align}\label{eq:chain-ratio}
%\min_{\text{chain }(A_i,\ldots, A_j)}
%\prod_{k=i+1}^j \min\bc{\fc{p_{{k-1}}(A_k)}{p_{k}(A_k)},1}
%\le \fc{p_i(A_i)}{p_j(A_i)}.
%\end{align}
%We induct on $j-i$.
%For $j=i$ this is trivial. For $j>i$, by the induction hypothesis on $(i+1,j)$,
%\begin{align}
%\fc{p_i(A_i)}{p_j(A_i)} 
%&\ge
%\min_{A_{i+1}\subeq A_i, A_{i+1}\in \cal P_{i+1}}
%\fc{p_i (A_{i+1})}{p_j(A_{i+1})}\\
%%\fc{p_i(A_i)}{p_{i+1}(A_i)}\fc{p_i(A_i)}{p_j(A_i)}\\
%%& \ge 
%%\min \fc{p_{i 
%&\ge 
%\fc{p_i (A_{i+1})}{p_{i+1}(A_{i+1})}
%\fc{p_{i+1} (A_{i+1})}{p_j(A_{i+1})}\\
%&\ge 
%\fc{p_i (A_{i+1})}{p_{i+1}(A_{i+1})}
%\min_{\text{chain }(A_{i+1},\ldots, A_j)}
%\prod_{k=i+2}^j \min\bc{\fc{p_{{k-1}}(A_k)}{p_{k}(A_k)},1}\\
%&\ge \min_{\text{chain }(A_i,\ldots, A_j)}
%\prod_{k=i+1}^j \min\bc{\fc{p_{{k-1}}(A_k)}{p_{k}(A_k)},1}
%\end{align}
%where in the last step we add $A_i$ to the chain $(A_{i+1},\ldots, A_j)$. 
%
%From~\eqref{eq:chain-ratio} and definition of $\ga$, we get $\fc{p_i(A_i)}{p_j(A_i)}\ge \ga$. Thus,
%\begin{align}
%\fc{p(A\times \{i,\ldots, L-1\})}{p(A\times \{i\})}
%&
%= \fc{\sum_{k=i}^{L-1} r_ip_i(A)}{r_ip_i(A)}\\
%&\le\fc{\max r_i}{\min r_i}
% \fc{\sum_{k=i}^{l-1} p_i(A)}{p_i(A)}\\
%&\le \fc L{r\ga}.
%\label{eq:path-bd1}
%\end{align}

Next we lower bound $P_{\st}(A\times \{i\}, A'\times \{i-1\})$.  There is probability $\rc{2L}$ of proposing a switch to level $i-1$, so
\begin{align}
P_{\st}(A\times \{i\}, A'\times \{i-1\})
&\ge \rc{2L} \int_\Om r_ip_i(x) \min\bc{
\fc{p_{i-1}(x)}{p_i(x)}\fc{r_{i-1}}{r_i}, 1
}\dx/(r_ip_i(A))\\
&=\rc{2L}\int_\Om  \min\bc{p_{i-1}(x)\fc{r_{i-1}}{r_i}, p_i(x)}\dx/p_i(A)\\
&\ge \rc{2L} \fc{\min r_j}{\max r_j} \int_\Om \min\bc{p_{i-1}(x), p_i(x)} \dx /  p_i(A)\\
&\ge \rc{2L} r\de.
\label{eq:path-bd2}
\end{align}
Putting~\eqref{eq:st-path},~\eqref{eq:path-bd1}, and~\eqref{eq:path-bd2} together, 
\begin{align}
\eqref{eq:st-path}&\le 2(L-1)2\pf{L}{r\ga} \pf{2L}{r\de}\\
&\le \fc{8L^3}{r^2\ga \de}.
\end{align}
Using~\eqref{eq:st-gap-prod} and Theorem~\ref{thm:can-path}, 
\begin{align}
\Gap(M_{\st}) &\ge \rc 2 \Gap(\ol M_{\st})\min_{B\in \cal P} \Gap(M_{\st}|_B)\\
&\ge \fc{r^2\ga \de}{16L^3} \min_{B\in \cal P} \Gap(M_{\st}|_B)\\
&\ge \fc{r^2\ga \de}{32L^3} \min_{1\le i\le L, A\in \cal P_i} \Gap(M_{i}|_A)
\end{align}
\end{proof}

By taking all the partitions except the first to be the same, we see that this theorem is an improvement to the bound for simulated tempering in~\cite[Theorem 3.1]{woodard2009conditions}, which gives the bound 
$$
\Gap(P_{\st}) \ge\fc{\ga^{J+3} \de^3}{2^{14}(L+1)^5J^3} \min\bc{
\min_{2\le i\le L, A\in \cal P} (\Gap(M_{i}|_A)), \Gap(M_1)}
$$
when $r=1$, where $J$ is the number of sets in the partition. Most notably, their bound is exponential in $J$, while our bound has no dependence on $J$.
%
%\begin{cor}
%Keep the setting of Theorem~\ref{thm:sim-temp}, except fix the partition $\cal P$. Then 
%\begin{align}
%\Gap(P_{\st}) &\ge\fc{r^2\ga \de}{32L^3} \min\bc{
%\min_{0\le i\le L-1, A\in \cal P} (\Gap(P_{i}|_A)), \Gap(P_0)}.
%\end{align}
%\end{cor}
%\begin{proof}
%Take $(\cal P_i')_{i=0}^{L-1}$ to be the chain with $\cal P_0=\{\Om\}$ and $\cal P_1=\cdots =\cal P_{L-1}=\cal P$ and use Theorem~\ref{thm:sim-temp}.
%\end{proof}
