\vspace{-0.1cm}

\section{Conclusion}

In this work, we presented a grasp planning approach which takes into account global and local object properties to generate stable and robust grasping information for robotic hands. Global information is gathered by analyzing the object's mean curvature skeleton in order to identify suitable regions for applying a grasp. In addition, local information is used to select the grasp type and to align the hand according to local skeleton and surface properties. 
We showed that the approach is capable of generating high quality grasping information for real-world objects as they occur in object modeling databases such as the KIT or the YCB object DB projects.
We evaluated the approach to a wide variety of objects with different robot hand models and showed that the resulting grasps are more robust in the presence of inaccuracies in grasp execution compared to grasp planners which do not consider local object properties.
In addition, we think that if we compare the results of our grasp planner to the results of other approaches, the generated grasping poses look more natural, i.e. more human like, although we did not perform a qualitative evaluation in this sense. In future work, this aspect will be investigated, e.g. with methods provided in \cite{Borst2005} or \cite{Liarokapis2015}.