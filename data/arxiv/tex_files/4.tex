%\listfiles
\documentclass[prb,aps,showpacs,twocolumn,floatfix]{revtex4-1}
\usepackage{amsmath, amsfonts, amssymb, bm, graphicx}
\numberwithin{equation}{section}

% ----------------------------------------------------------------------------- 
\newcommand{\bS}{{\bm S}}
\newcommand{\bk}{{\bm k}}
\newcommand{\bx}{{\bm x}}
\newcommand{\bp}{{\bm p}}
\newcommand{\bq}{{\bm q}}
\newcommand{\bkp}{{{\bm k}^\prime}}
\newcommand{\bsig}{{\bm \sigma}}
\newcommand{\balpha}{{\bm \alpha}}
\newcommand{\e}{{\rm e}}
\newcommand{\ii}{{\rm i}}
\newcommand{\dd}{d^\dag}
\newcommand{\ed}{\hat{e}^\dag}
\newcommand{\cd}{c^\dag}
\newcommand{\fd}{\hat{f}^\dag}
\newcommand{\gd}{\hat{g}^\dag}
\newcommand{\ua}{{\uparrow}}
\newcommand{\da}{{\downarrow}}
\newcommand{\vac}{\vert {\rm vac} \rangle}
\newcommand{\hvac}{\langle {\rm vac} \vert}
\newcommand{\FS}{\vert{\rm FS}\rangle}
\newcommand{\hFS}{\langle{\rm FS}\vert}
\newcommand{\tr}{{\rm Tr}\,}
\newcommand{\ve}{\varepsilon}
\renewcommand{\theequation}{S\arabic{equation}}
\renewcommand{\thefigure}{S\arabic{figure}}
% -----------------------------------------------------------------------------

\begin{document}

\title{Methods and Supplementary material}
\maketitle

\section{Methods}
\renewcommand{\theequation}{M\arabic{equation}}

%
\subsection*{Derivation of the $g$-atom effective Hamiltonian (2)}
%

We focus on the regime $J_e \gg J_g, \, U^-_{e g}$ when the interaction term
$\hat{H}_{\rm int} = U^-_{e g} \sum_{i a} \hat{n}^g_{i a} \hat{n}^e_{i a}$ and
$g$-atom kinetic energy $\hat{H}^g_0$ in Eq. (1) can be treated as a small
correction to the kinetic energy of $e$-atoms $\hat{H}^e_0 = -J_e
\sideset{}{_i} \sum (\ed_{i 1} \hat{e}_{i 2} + {\rm h.c.})$, and use
perturbation theory to compute the effective low-energy model (2).
The zero-order subspace is spanned by the states $\vert \Psi_{e g} \rangle =
\prod_i \vert \lambda \rangle_i \otimes \vert \Psi_g \rangle$.
Because $J_g \ll J_e$, we consider this subspace degenerate for all $\vert
\Psi_g \rangle$.
To determine 1st (2nd) order corrections to the wavefunctions (energy
eigenvalues), we employ the Schrieffer-Wolff transformation (SWT) method
which systematically removes {\it off-diagonal} matrix elements of
$\hat{H}_{\rm int}$ between states $\vert \Psi_{e g} \rangle$ to each order in
$U^-_{e g}$ \cite{bir-1974-1}: $\hat{H} = \hat{H}^e_0 + \hat{H}_{\rm int} \to
\e^{\cal \hat{S}} \hat{H} \e^{-{\cal \hat{S}}}$ with ${\cal \hat{S}} = {\cal
\hat{S}}_1 + {\cal \hat{S}}_2 + \ldots$ being an anti-hermitian generator of
the SWT and ${\cal \hat{S}}_n \sim (U^-_{e g})^n$.
Notice that $\hat{H}^g_0$ acts only on the $g$-atom states $\vert \Psi_g
\rangle$ and contains only {\it diagonal} matrix elements, thus representing a
1st order correction.
On the other hand, $\hat{H}_{\rm int}$ has both off-diagonal and diagonal
elements.
The latter are proportional to $\sum_{i a} \hat{n}^g_{i a}$, i.e. they only
correct the $g$-atom chemical potential and can be omitted.

To the first order in $U^-_{e g}$, ${\cal \hat{S}}$ is given by:
\begin{displaymath}
 {\cal \hat{S}}_1 = U^-_{e g} \sum_{i a, \, E^\prime_e E_e}
 \frac{\hat{P}_{E^\prime_e} \, \widetilde{\hat{n}^e}_{i a} \, \hat{P}_{E_e}}
 {E^\prime_e - E_e} \, \hat{n}^g_{i a}.
\end{displaymath}
where $\hat{P}_{E_e}$ is the projector onto the state $\prod_i \vert \lambda
\rangle_i$ of the $e$-subsystem with an energy $E_e$, and
$\widetilde{\hat{n}^e}_{i a}$ is an off-diagonal part of the matrix ${}_i
\langle \lambda^\prime \vert \hat{n}^e_{i a} \vert \lambda \rangle_i =
\frac{1}{2} \delta_{\lambda^\prime \lambda} + \frac{1}{2}
\sigma^x_{\lambda^\prime \lambda} (\delta_{a 1} - \delta_{a 2})$ ($\delta_{a
b}$ is a Kronecker delta-symbol).

The generator ${\cal \hat{S}}_1$ yields a 2nd order in $U^-_{e g}$ correction
$\hat{H}_2 = \frac{1}{2} \bigl[ {\cal \hat{S}}_1, \hat{H}_{\rm int} \bigr]$
which operates in the degenerate manifold $\lbrace \vert \Psi_{e g} \rangle
\rbrace$:
\begin{align}
 \hat{H}_2 = & \frac{(U^-_{e g})^2}{2} \sum_{\substack{i a b\\ E^\prime_e E_e}}
 {}_i \bigl \langle \lambda \bigl \vert \frac{\hat{P}_{E^\prime_e} \,
 \widetilde{\hat{n}^e}_{i a} \, \hat{P}_{E_e}}{E^\prime_e - E_e}
 \widetilde{\hat{n}^e}_{i b} \bigr \vert \lambda \bigr \rangle_i
 \hat{n}^g_{i a} \hat{n}^g_{i b} + {\rm h.c.} \nonumber \\
 & = \lambda u_{g g} \sideset{}{_i} \sum \hat{n}^g_{i 1} \hat{n}^g_{i 2}
 \nonumber
\end{align}
with $u_{g g} = (U^-_{e g})^2 / 4 J_e$ and $\lambda = \pm 1$.
The $g$-atom interaction can be controlled by preparing the $e$-subsystem in a
particular local state: it is repulsive (attractive) for a (anti-) symmetric
$e$-state $\lambda = +1$ ($\lambda = -1$).

%
\subsection*{Unconstrained Hartree-Fock-Bogoliubov theory in position space}
%

In the present section, we summarize the fully unconstrained
Hartree-Fock-Bogoliubov (HFB) mean-field approach \cite{blaizot-1986-1} which
we used to compute the phase diagram in Fig. 2.
This variational technique treats on an equal footing SF and normal phases,
including various insulating states that break translational symmetry, and
hence provides an additional check of robustness of the TSF phase.
Our method amounts to the following linearization of the four-fermion
interaction:
\begin{align}
 \hat{H}_{\rm int} = & - \! u_{g g} \! \sideset{}{_i} \sum \! \gd_{i 1}
 \gd_{i 2} \hat{g}_{i 2} \hat{g}_{i 1} \! \approx - \! u_{g g} \!
 \sideset{}{_i} \sum \! \bigl[ \hat{h}_H \! + \! \hat{h}_F \! + \! \hat{h}_B
 \bigr]; \nonumber \\
 %
 \hat{h}_{H} = & \langle \hat{n}^g_{i 1} \rangle \gd_{i 2} \hat{g}_{i 2} +
 \langle \hat{n}^g_{i 2} \rangle \gd_{i 1} \hat{g}_{i 1}; \nonumber \\
 %
 \hat{h}_F = & -\langle \gd_{i 1} \hat{g}_{i 2} \rangle \gd_{i 2} \hat{g}_{i 1}
 - {\rm h.c.} = \xi_i \gd_{i 1} \hat{g}_{i 2} + {\rm h.c.}; \nonumber \\
 %
 \hat{h}_B = & \langle \hat{g}_{i 2} \hat{g}_{i 1} \rangle \gd_{i 1} \gd_{i 2}
 + {\rm h.c.} = -\frac{\Delta_i}{u_{g g}} \gd_{i 1} \gd_{i 2} - {\rm h.c.},
 \label{eq:H_int_HFB}
\end{align}
where $\hat{h}_{H, F, B}$ refer to Hartree, Fock and Bogoliubov contributions
with (in general complex) OPs $n^g_{i a} = \langle \gd_{i a} \hat{g}_{i a}
\rangle$, $\xi_i = -\langle \gd_{i 2} \hat{g}_{i 1} \rangle$, and $\Delta_i =
-u_{g g} \langle \hat{g}_{i 2}\hat{g}_{i 1} \rangle$.
The full HFB mean-field Hamiltonian includes $\hat{H}_{\rm int}$ and the
$g$-atom kinetic energy $\hat{H}^g_0$ [see Eq. (2)], and can be written as:
\begin{displaymath}
 \hat{H}_{\rm HFB} = \frac{1}{2} \sum_{\alpha \beta}
 \begin{pmatrix}
  \gd_\alpha & \hat{g}_\alpha
 \end{pmatrix}
 \underbrace{
  \begin{pmatrix}
   X_{\alpha \beta} & D_{\alpha \beta} \\
   -D^*_{\alpha \beta} & -X^*_{\alpha \beta}
  \end{pmatrix}
 }_{{\cal H}_{\alpha \beta}}
 \begin{pmatrix}
  \hat{g}_\beta \\
  \gd_\beta
 \end{pmatrix}
\end{displaymath}
with composite indices $\alpha = \lbrace i a \rbrace$ and $\beta = \lbrace j b
\rbrace$, and $2 N_d \times 2 N_d$ dimensional matrices $D_{\alpha \beta} = \ii
\sigma^y_{\alpha \beta} \Delta_i \delta_{i j}$ and
\begin{align}
 X_{\alpha \beta} \! = \! X_{i a, \, j b} \! = & -J_g \bigl[ \delta_{i j}
 \sigma^x_{a b} + \bigl( \delta_{a 2}^{b 1} \delta_{j, i + 1} +
 \delta_{a 1}^{b 2} \delta_{j, i - 1} \bigr) \bigr] \nonumber \\
 & -\mu \delta_{i j} \delta_{a b} - u_{g g} \delta_{i j}
 \begin{pmatrix}
  n^g_{i 2} & \xi_i \\
  \xi^*_i & n^g_{i 1}
 \end{pmatrix}
 _{a b}. \nonumber
\end{align}
The kernel ${\cal H}_{\alpha \beta}$ has the property $\tau^x {\cal H}^* \tau^x
= -{\cal H}$ with $\tau^x = (\sigma^x \otimes \delta_{\alpha \beta})$, which
guarantees that its spectrum is even w.r.t. zero energy. We shall enumerate its
positive eigenvalues by $\nu = 1, \ldots, n_\nu$ with $n_\nu = 2 N_d$.
$\hat{H}_{\rm HFB}$ can be diagonalized by a Bogoliubov transformation
\begin{displaymath}
 \begin{pmatrix}
  \hat{g}_\alpha \\
  \gd_\alpha
 \end{pmatrix}
 =
 \sum_\nu^{n_\nu} \bigl[ \psi^\nu \hat{\gamma}_\nu + (\tau^x \psi^\nu)^*
 \hat{\gamma}^\dag_\nu \bigr],
\end{displaymath}
where $\hat{\gamma}_\nu$ are new fermionic modes and $\psi^\nu$ is an
eigenstate of ${\cal H}$ corresponding to an energy $E_\nu \geqslant 0$.
These states obey a completeness relation $\sum_\nu^{n_\nu} \bigl[ \psi^{*
\nu}_p \psi^\nu_q + (\tau^x \psi^\nu)_p (\tau^x \psi^\nu)^*_q \bigr] =
\delta_{p q}$ with $p$ and $q = 1, \ldots, 4 N_d$.

Using the above quasiparticle modes, we can write down the self-consistency
relations:
\begin{align}
 \Delta_i = & -\frac{u_{g g}}{2} \sideset{}{^{n_\nu}_\nu} \sum \bigl[
 \psi^{* \nu}_{(i 1) + 2 N_d} \psi^\nu_{(i 2)} - \psi^{* \nu}_{(i 2) + 2 N_d}
 \psi^\nu_{(i 1)} \bigr] \phi_\nu, \nonumber \\
 %
 n^g_{i a} = & \frac{1}{2} - \frac{1}{2} \sideset{}{^{n_\nu}_\nu} \sum \bigl[
 \vert \psi^\nu_{(i a)} \vert^2 - \vert \psi^\nu_{(i a) + 2 N_d} \vert^2 \bigr]
 \phi_\nu, \nonumber \\
 %
 \xi_i = & \frac{1}{2} \sideset{}{^{n_\nu}_\nu} \sum \bigl[
 \psi^{* \nu}_{(i 2)} \psi^\nu_{(i 1)} - \psi^{* \nu}_{(i 1) + 2 N_d}
 \psi^\nu_{(i 2) + 2 N_d} \bigr] \phi_\nu, \nonumber
\end{align}
where $\phi_\nu = 1 - 2 f_\nu$ and $f_\nu = \langle \hat{\gamma}^\dag_\nu
\hat{\gamma}_\nu \rangle = 1 / (\e^{E_\nu / T} + 1)$.

The quantum phase diagram (Fig. 2) was computed by minimizing the grand
potential ${\cal G} = \langle \hat H_{\rm HFB} \rangle$ (notice, $\hat{H}_{\rm
HFB}$ already includes the term $-\mu \sum_{i a} \hat{n}^g_{i a}$):
\begin{align}
 \frac{\cal G}{N_d} = -\mu & - \frac{1}{N_d} \sum_i \biggl[ \frac{\vert
 \Delta_i \vert^2}{u_{g g}} + u_{g g} \bigl( n^g_{i 1} n^g_{i 2} - \vert \xi_i
 \vert^2 \bigr) \biggr] \nonumber \\
 - \frac{1}{2 N_d} & \sum_{\alpha \beta} X_{\alpha \beta}(u_{g g} \! = \! 0) \!
 \sum_\nu^{n_\nu} \! \bigl[ \psi^{* \nu}_\alpha \psi^\nu_\beta -
 \psi^{* \nu}_{\beta + 2 N_d} \psi^\nu_{\alpha + 2 N_d}\bigr]. \nonumber
\end{align}

%
\subsection*{Probing $g$-atom attraction}
%

Here we present a simple derivation of the time dependence of the number of
doubly occupied dimers after the quench protocol described in the main text.
Assume that the $g$-$g$ interactions are Hubbard-like: $\hat{H}_{\rm int} = U
\hat{V} = U \sum_i \hat{n}^g_{i 1} \hat{n}^g_{i 2}$, where $U$ denotes the
interaction strength whose sign is unknown and needs to be determined (in the
main text, $U = -u_{g g} < 0$).
The operator $\hat{V}$ is a projector onto states with only doubly-occupied
dimers.
Our task is to compute $n_2 (t) = \langle \psi (t) \vert \hat{V} \vert \psi (t)
\rangle$, where $\vert \psi (t) \rangle$ is the time-dependent state of the
system.

At time $t = 0$ $g$-atoms fill the Fermi sea $\vert \psi_0 \rangle = \vert \psi
(t = 0) \rangle = \prod _{k < k_F} \fd_{k, -1} \vac$, where $\vac$ is the
vacuum state without particles and $k_F$ is the Fermi momentum.
The wavefunction at $t = t_0$ is $\vert \psi (t_0) \rangle = \e^{-\ii \varphi
\hat{V}} \vert \psi_0 \rangle$ with $\varphi = U t_0$, while for $t > t_0$ it
is given by $\vert \psi (t > t_0) \rangle = \e^{-\ii \hat{H}_0^g (t - t_0)}
\vert \psi (t_0) \rangle$.
Therefore,
\begin{displaymath}
 n_2 (t > t_0) = \langle \psi_0 \vert \e^{\ii \varphi \hat{V}}
 \e^{\ii \hat{H}_0^g (t - t_0)} \hat{V} \e^{-\ii \hat{H}_0^g (t - t_0)}
 \e^{-\ii \varphi \hat{V}} \vert \psi_0 \rangle.
\end{displaymath}
Consider a short time $t_0 \ll 1 / U$ when $\varphi \ll 1$ and $n_2 (t) -
\langle \psi_0 \vert \hat{V} \vert \psi_0 \rangle \approx \ii \varphi \langle
\psi_0 \vert \bigl[ \hat{V}, \e^{\ii \hat{H}_0^g \tau} \hat{V} \e^{-\ii
\hat{H}_0^g \tau} \bigr] \vert \psi_0 \rangle$.
If in addition we limit ourselves to short-time dynamics, $t - t_0 \ll 1 /
J_g$, then $\e^{\ii \hat{H}_0^g (t - t_0)} \hat{V} \e^{-\ii \hat{H}_0^g (t -
t_0)} \approx \hat V + \ii (t - t_0) [\hat{H}_0^g, \hat{V}]$, and the final
expression for $n_2 (t)$ is:
\begin{displaymath}
 n_2 (t) = \langle \hat{V} \rangle + \varphi (t - t_0) \langle [\hat{V},
 [\hat{V}, \hat{H}_0^g]] \rangle = \langle \hat{V} \rangle - \kappa \varphi
 (t - t_0).
\end{displaymath}
Here $\langle \ldots \rangle = \langle \psi_0 \vert \ldots \vert \psi_0
\rangle$ and $\kappa = 2 \bigl[ \langle \hat{V} \hat{H}_0^g \hat{V} \rangle -
E_0 \langle \hat{V}^2 \rangle \bigl]$ ($E_0$ is the GS energy).
It is easy to see that $\kappa > 0$.
Indeed, introducing a normalized wavefunction $\vert \psi^V_0 \rangle = \hat{V}
\vert \psi_0 \rangle / \langle \hat{V}^2 \rangle$, we have $\kappa = 2 \langle
\hat{V}^2 \rangle \bigl[ \langle \psi^V_0 \vert \hat{H}_0^g \vert \psi^V_0
\rangle - E_0 \bigr]$, which is clearly positive.

%
\subsection*{Momentum-resolved spectroscopy signal}
%

Here we derive the expression for ${\cal R}_\tau (\delta, k)$ used in the main
text.
Our approach almost exactly follows Ref. \onlinecite{dao-2009-1}.
We use units such that $\hbar = k_B = 1$.

The transfer of $g$-atoms from the SF phase to $e$-states in an empty tube is
governed by the operator:
\begin{displaymath}
 \hat{V}_L = \Omega \sideset{}{_{i a}} \sum \hat{q}^\dag_{i a} \hat{g}_{i a},
\end{displaymath}
where $\Omega$ is the Rabi frequency and $\hat{q}^\dag_{i a}$ creates an
$e$-atom in well $a$ of the $i$-th dimer in the auxiliary (empty) tube.
The initial state of the system contains $N_g$ $g$-atoms, $\vert \psi_{\rm in}
\rangle = \vert \psi^{i, g}_{N_g} \rangle \otimes \vac$ ($\vac$ is an empty
state of the auxiliary tube).
Its energy is ${\cal E}_{\rm in} = E^g_i + \mu N_g + \omega_0$, where
$\omega_0$ is the laser frequency (note that we work in the grand-canonical
ensemble, so $E^g_i$ already includes the chemical potential offset).
In the final state, there is a single $e$-atom with momentum $k$ and band index
$\tau$ in the auxiliary tube: $\vert \psi_{\rm fi} \rangle = \vert \psi^{f,
g}_{N_g - 1} \rangle \otimes \hat{f}^\dag_{k \tau} \vac$ ($\vert \psi^{f,
g}_{N_g - 1} \rangle$ is an intermediate state of $g$-atoms with $N_g - 1$
particles).
This state has an energy ${\cal E}_{\rm fi} = E^g_f + \mu (N_g - 1) + \nu_{e g}
+ \epsilon^e_{k \tau}$, where $\nu_{e g}$ is the $e$-$g$ transition frequency
($\sim 10^{14}$ Hz for ${}^{87} {\rm Sr}$).
Because $\nu_{e g}$ is a very large energy compared to SF energy scales, we
introduce a laser detuning $\delta = \omega_0 - \nu_{e g}$.
The operators $\hat{f}_{k \tau}$ are similar to the $f$-modes defined in the
text: $\hat{f}_{k \tau} = \frac{1}{\sqrt{2}} \bigl[ \hat{q}_{k 1} -
\frac{\tau}{r_k} (1 + \eta \e^{\ii k}) \hat{q}_{k 2} \bigr]$.
$\eta$ parameterizes the $e$-atom band structure in the empty tube via
$\epsilon^e_{k \tau} = \tau r_k = \tau \sqrt{1 + \eta^2 + 2 \eta \cos k}$.

The transfer rate ${\cal R}$ can be computed with the aid of the Fermi golden
rule \cite{dao-2009-1}:
\begin{displaymath}
 {\cal R}_\tau (\delta, k) = 2 \pi \sideset{}{_{{\rm in}, \, {\rm fi}}} \sum
 \bigl \vert \langle \psi_{\rm fi} \vert \hat{V}_L \vert \psi_{\rm in} \rangle
 \bigr \vert^2 \,\, \frac{\e^{-E_{\rm in} / T}}{\cal Z} \delta
 ({\cal E}_{\rm fi} - {\cal E}_{\rm in}),
\end{displaymath}
where summation is extended over all initial and final states, and ${\cal Z}$
is the grand partition function.
The matrix elements of $\hat{V}$ are given by
\begin{align}
 \langle \psi_{\rm fi} \vert \hat{V}_L \vert \psi_{\rm in} \rangle = & \Omega
 \sideset{}{_{k a}} \sum \langle \psi^{f, g}_{N_g - 1} \vert \hat{g}_{k a}
 \vert \psi^{i, g}_{N_g} \rangle \hvac \hat{f}_{k \tau} \hat{q}^\dag_{k a} \vac
 = \nonumber \\
 = & {\textstyle \frac{\Omega}{\sqrt{2}}} \bigl[ \langle f \vert \hat{g}_{k 1}
 \vert i \rangle - {\textstyle \frac{\tau}{r_k}} (1 + \eta \e^{\ii k}) \langle
 f \vert \hat{g}_{k 2} \vert i \rangle \bigr], \nonumber
\end{align}
where $\langle f \vert \hat{g}_{k a} \vert i \rangle = \langle \psi^{f, g}_{N_g
- 1} \vert \hat{g}_{k a} \vert \psi^{i, g}_{N_g} \rangle$.

Using the single-particle spectral density \cite{zubarev-1960-1},
\begin{align}
 {\cal A}_{a b} (\omega, k) = & 2 \pi \sideset{}{_{n m}} \sum \langle n \vert
 \hat{g}_{k a} \vert m \rangle \frac{\e^{-E_m / T}}{\cal Z} \nonumber \\
 & \times \langle m \vert \gd_{k b} \vert n \rangle \delta (E_m - E_n -
 \omega), \nonumber
\end{align}
we obtain after a straightforward calculation:
\begin{align}
 {\cal R}_\tau (\delta, k) = & {\textstyle \frac{\Omega^2}{2}} \bigl \lbrace
 {\rm Tr} \, {\cal A} (\omega, k) \nonumber \\
 - {\textstyle \frac{\tau}{r_k}} & \bigl[ (1 + \e^{\ii k}) {\cal A}_{1 2}
 (\omega, k) + (1 + \e^{-\ii k}) {\cal A}_{2 1} (\omega, k) \bigr] \bigr
 \rbrace \nonumber
\end{align}
with $\omega = \epsilon^e_{k \tau} - \mu - \delta$.
Noting that ${\cal A}_{21} = {\cal A}^*_{12}$, we arrive at the expression in
the main text.

%
\subsection*{Magneto-electric phenomena}
%

In this section we discuss novel magneto-electric effects that occur as a
result of the {\it odd in momentum} SOC in the effective model (2) when it is
subjected to a weak laser-induced synthetic magnetic field.
We assume this field to be homogeneous: $\delta \hat{H}_{\rm ef} = -{\bm b}
\cdot \sum_k \bsig_{a b} \gd_{k a} \hat{g}_{k b}$.
In particular, we shall derive weak-coupling expressions for the susceptibility
$\chi$ and magneto-electric coefficient $\kappa$ quoted in the main text.

It is well-known that a physical external magnetic field breaks time-reversal
symmetry ${\cal T}$.
However, because ${\bm b}$ is artificial, it does not necessarily break ${\cal
T}$.
Indeed, under time-reversal, the second-quantized operators and $c$-numbers
transform as $\hat{g}_{k a} \to \hat{g}_{-k, a}$ (because $a$ is not a real
spin but the position index inside a dimer) and $c \to c^*$.
Therefore, only the $b_y$-term in $\delta \hat{H}_{\rm ef}$ breaks ${\cal T}$
and is capable of generating a mass current.
Below we focus on the case with $b_x = b_z = 0$ and $b_y \neq 0$ [see Fig.
4(a)].

It is instructive to study the non-interacting system described by the
Hamiltonian $\hat{H}^g_0 + \delta \hat{H}_{\rm ef}$ [$\hat{H}^g_0$ is defined
in Eq. (2)] which is diagonalized by Bogoliubov quasiparticles $\hat{f}_{k
\tau} = \frac{1}{\sqrt{2}} \bigl[ \hat{g}_{k 1} - \frac{\tau}{R_k} (1 +
\e^{-\ii k} - \ii \, b_y) \hat{g}_{k 2} \bigr]$ with energies $\epsilon_{k
\tau} = \tau R_k = \tau \sqrt{2 J_g^2 (1 + \cos k) + b_y^2 + 2 J_g b_y \sin
k}$.
Assuming that $T = 0$ and $f$-particles fill the Fermi sea $\FS$ with a
chemical potential $\mu$, i.e. $\hFS \fd_{k \tau^\prime} \hat{f}_{k \tau} \FS =
\delta_{\tau^\prime \tau} \theta (\mu - \epsilon_{k \tau})$, the average
mass current is (the operator $\hat{K}$ was defined in the main text)
\begin{align}
 \hFS \hat{K} \FS = & {\textstyle \frac{2}{N_d}} \sideset{}{_{k \tau}} \sum
 {\textstyle \frac{\tau}{R_k}} (b_y \cos k - J_g \sin k) \langle \fd_{k \tau}
 \hat{f}_{k \tau} \rangle \nonumber \\
 = {\textstyle \frac{2}{J_g N_d}} & \sideset{}{_{k \tau}} \sum \frac{ \partial
 \epsilon_{k \tau} }{\partial k } \, \theta (\mu \! - \! \epsilon_{k \tau}) =
 {\textstyle \frac{2}{J_g}} (\mu \! - \! \mu) \! = 0. \nonumber
\end{align}
This result is valid even in interacting non-SF systems.
Hence a magnetic field cannot generate a mass current in the absence of SF
correlations \cite{ojanen-2012-1}.

This situation changes dramatically when pairing correlations are taken into
account, technically because the average current can no longer be written as an
integral of the quasiparticle group velocity.
Consider the weak-coupling dilute limit $u_{g g} \ll J_g$ and $n^g \ll 1$.
In this regime, we can project interactions in Eq. (2) onto the lowest band
$\tau = -1$.
At small fields $b_y \ll J_g, \Delta$, the bare fermion operators become
$\hat{g}_{k 1} \approx \frac{1}{\sqrt{2}} \hat{f}_k$ and $\hat{g}_{k 2} \approx
\frac{1}{\sqrt{2}} \e^{\ii k / 2} (1 + \ii b_y / 2 J_g) \hat{f}_k$, where
$\hat{f}_{k, -1} \equiv \hat{f}_k$.
It is easy to check that $b_y$ enters the projected interaction term in Eq. (2)
only via quadratic corrections $\sim b_y^2$.
Therefore, to linear order in $b_y$ the projected Hamiltonian is the same as
Eq. (3) with $\epsilon_k = -2 J_g \cos \frac{k}{2} - b_y \sin \frac{k}{2} =
\epsilon^{(0)}_k - b_y \sin \frac{k}{2}$.
Since $-\epsilon_{-k} = -\epsilon^{(0)}_k - b_y \sin \frac{k}{2}$, the BdG
Hamiltonian is
\begin{displaymath}
 {\cal H}_{\rm BdG} = -b_y \sin {\textstyle \frac{k}{2}} +
 \begin{pmatrix}
  \epsilon^{(0)}_k - \mu & D_k \\
  D_k^* &   \mu-\epsilon^{(0)}_k
 \end{pmatrix}
 ,
\end{displaymath}
see Eq. (S2) in the Supplementary material.
The Bogoliubov transformation diagonalizing this Hamiltonian is given by Eq.
(S3) of the Supplementary material with $\epsilon_k$ and $E_k$ replaced by
$\epsilon^{(0)}_k$ and $E^{(0)}_k = \sqrt{(\epsilon^{(0)}_k - \mu)^2 + \vert
D_k \vert^2}$, respectively.
However, energies of the quasiparticles $\hat{\gamma}_k$ and $\hat{\Gamma}_k$
are now split by a field correction and are given by $E_{k, \pm} = E^{(0)}_k
\mp b_y \sin \frac{k}{2}$, respectively.
If $b_y$ is small, we can assume that ${\rm sign} (E_{k, \pm}) = \pm$, and in
the BCS GS $\langle \hat{\gamma}^\dag_k \hat{\gamma}_k \rangle = \nu (E_{k,
+})$ and $\langle \hat{\Gamma}^\dag_k \hat{\Gamma}_k \rangle = \nu (E_{k, -})$
with the Fermi function $\nu (x) = \bigl( e^{x / T} + 1 \bigr)^{-1}$.

The projected current and $y$-component of the pseudospin $\hat{S}_y$ have the
form: $\hat{K} \approx \frac{1}{N_d} \sum_k \bigl( 2 \sin \frac{k}{2} -
\frac{b_y}{J_g} \cos \frac{k}{2} \bigr) \fd_k \hat{f}_k$ and $\hat{S}_y =
\frac{1}{2 N_d} \sum_k \sigma^y_{a b} \gd_{k a} \hat{g}_{k b} \approx
\frac{1}{4 N_d} \sum_k \bigl( 2 \sin \frac{k}{2} + \frac{b_y}{J_g} \cos
\frac{k}{2} \bigr) \fd_k \hat{f}_k$.
In the BCS GS the first term in both expressions vanishes at $T = 0$ because
$\sideset{}{^\prime} \sum_k \sin \frac{k}{2} \langle \fd_k \hat{f}_k -
\hat{F}^\dag_k \hat{F}_k \rangle = \sideset{}{^\prime} \sum_k \sin \frac{k}{2}
\bigl[ \nu (E_{k, +}) - \nu (E_{k, -}) \bigr]$.
Hence, $\kappa = \langle \hat{K} \rangle / b_y = -4 \langle \hat{S}_y \rangle /
b_y = -4 \chi$.
When $T = 0$, it is easy to show that $\kappa$ is given by the expression in
the main text.

Fig. 4(b) shows $\kappa$ and $\chi$ numerically computed beyond the
weak-coupling dilute limit.
In the simulation we used the same strategy as above: solve the BdG equations
in the field, compute $\langle \hat{K} \rangle$ and $\langle \hat{S}_y \rangle$
as functions of $b_y$, and extract the corresponding linear coefficients.

\section{Supplementary material}
\renewcommand{\theequation}{S\arabic{equation}}

%
\subsection*{BCS theory in a translationally invariant 1D system}
%

The unconstrained mean-field approach presented in Methods can be handled only
numerically.
However, if we restrict the variational space to translationally-invariant
states in a chain with periodic boundary conditions (one can use the momentum
$k$-space representation) and omit the Hartree-Fock terms in $\hat{H}_{\rm
HFB}$, the resulting BCS theory can be treated analytically and a number of
insights can be gained on the structure of the topological SF phase.
In particular, in this section we analyze the singlet-triplet mixing of Cooper
pair states due to the SOC in Eq. (2).

At the BCS level \cite{bardeen-1957-1}, the linearized interaction
$\hat{H}_{\rm int}$ in Eq. (M1) contains only the $\hat{h}_B$ term:
\begin{displaymath}
 \hat{H}_{\rm int} \approx \sideset{}{_k} \sum \bigl[ \Delta \, \gd_{k 1}
 \gd_{-k, 2} + {\rm h.c.} \bigr], \nonumber
\end{displaymath}
where $\Delta = -\frac{u_{g g}}{N_d} \sum_k \langle \hat{g}_{-k, 2} \hat{g}_{k
1} \rangle \equiv -\ii \Delta_0 \e^{\ii \phi_\Delta}$ (real $\Delta_0$) is an
$s$-wave OP.
Because the above interaction couples fermions with opposite momenta, to avoid
overcounting of $k$-states we will restrict momentum summations to a
``positive'' half of the BZ with $k > 0$ (indicated by a prime), and denote
$\hat{g}_{-k, a} \equiv \hat{G}_{k a}$ ($-k$ belongs to the other half of the
BZ): $\sum_k f (\hat{g}_{k a}, \hat{g}_{-k, b}) = \sum_k^\prime \bigl[ f
(\hat{g}_{k a}, \hat{G}_{k b}) + f (\hat{G}_{k a}, \hat{g}_{k b}) \bigr]$.
Then
\begin{align}
 \hat{H}_{\rm int} \approx & \sideset{}{^\prime_k} \sum \bigl[ \Delta \, \bigl(
 \gd_{k 1} \hat{G}^\dag_{k 2} + \hat{G}^\dag_{k 1} \hat{g}^\dag_{k 2} \bigr) +
 {\rm h.c.} \bigr] \nonumber \\
 & = \Delta_0 \, \sigma^y_{a b} \sideset{}{^\prime_k} \sum \bigl[
 \e^{\ii \phi_\Delta} \, \gd_{k a} \hat{G}^\dag_{k b} - {\rm h.c.} \bigr].
 \nonumber
\end{align}
Similarly, the kinetic energy in Eq. (2) can be written as
\begin{displaymath}
 \hat{H}^g_0 = \sideset{}{_k} \sum \epsilon_{a b} (k) \gd_{k a} \hat{g}_{k b}
 = \sideset{}{^\prime_k} \sum \epsilon_{a b} (k) \bigl[ \gd_{k a} \hat{g}_{k b}
 - \hat{G}_{k a} \hat{G}^\dag_{k b} \bigr]
\end{displaymath}
because $\epsilon (k) = -J_g \bigl[ (1 + \cos k) \sigma^x + \sin k \, \sigma^y
\bigr] = \epsilon^T (-k)$.

Up to constant terms, the full BCS Hamiltonian is
\begin{displaymath}
 \hat{H}_{\rm BCS} \! = \!\! \sideset{}{^\prime} \sum_k \!
 \bigl( \gd_{k a} \, \hat{G}_{k a} \bigr) \!
 \underbrace{
  \begin{pmatrix}
   \epsilon_{a b} - \mu \delta_{a b} & \Delta_0 \e^{\ii \phi_\Delta}
   \sigma^y_{a b} \\
   \Delta_0 \e^{-\ii \phi_\Delta} \sigma^y_{a b} & \mu \delta_{a b} -
   \epsilon_{a b}
  \end{pmatrix}
 }_{
  {\cal H}_{\rm BdG}
 }
 \!\!
 \begin{pmatrix}
  \hat{g}_{k, b} \\
  \hat{G}^\dag_{k, b}
 \end{pmatrix}
 .
\end{displaymath}
The BdG Hamiltonian ${\cal H}_{\rm BdG}$ can be easily diagonalized when the
OP $\Delta$ is purely imaginary.
This scenario is still quite general, and below we will focus on the case
$\phi_\Delta = 0$ when ${\cal H}_{\rm BdG} = (\sigma^z \otimes (\epsilon-\mu))
+ \Delta_0 (\sigma^x \otimes \sigma^y)$.
It is diagonalized by performing a unitary transformation ${\cal H}_{\rm BdG}
\to {\cal H}^y_{\rm BdG} = (U^\dag \otimes 1) {\cal H}_{\rm BdG} (U \otimes 1)$
with
$U = \frac{1}{\sqrt{2}}
 \begin{pmatrix}
  1 & 1 \\
  \ii & -\ii
 \end{pmatrix}
$.
The eigenstates $\psi_{\nu \tau}$ are labeled by two indices $\tau$ and $\nu =
\pm 1$, and have the form
\begin{equation}
 \psi_{\nu \tau} = \frac{1}{2 T_{k \tau}}
 \begin{pmatrix}
  \bigl[ \nu T_{k \tau} + h_k \bigr] u_{k \tau} \\
  \ii \bigl[ \nu T_{k \tau} - h_k \bigr] u_{k \tau}
 \end{pmatrix}
 \label{eq:BdG_wf_1d}
\end{equation}
with $h_k = \epsilon (k) - \mu + \ii \Delta_0 \sigma^y$ and the spinor
\begin{displaymath}
 u_{k \tau} = \frac{1}{\sqrt{2 R_k (R_k + \tau D_k)}}
 \begin{pmatrix}
  R_k + \tau D_k \\
  \tau \mu (1 + \e^{\ii k})
 \end{pmatrix}.
\end{displaymath}
The corresponding energies are $E_{k \nu \tau} = \nu T_{k \tau}$ with $T_{k
\tau} = \sqrt{\mu^2 + \Delta_0^2 + 2 J_g^2 (1 + \cos k) + 2 \tau J_g R_k}$,
$R_k = \sqrt{D^2_k + 2 \mu^2 (1 + \cos k)}$ and $D_k = \Delta_0 (1 + \cos k)$.

The parameter $\Delta_0$ is determined from the BCS self-consistency condition:
\begin{align}
 \Delta_0 = & -\frac{u_{g g}}{N_d} \sideset{}{^\prime_k} \sum \sigma^y_{a b}
 \langle \hat{g}_{k, a} \hat{G}_{k, b} \rangle
 \nonumber \\
 & = \frac{\Delta_0 u_{g g}}{4 N_d} \sideset{}{_{k \tau}} \sum \frac{{\rm th}
 \, \frac{\beta T_{k \tau}}{2}}{T_{k \tau}} \biggl[ 1 + \tau \frac{J_g}{R_k} (1
 + \cos k)^2 \biggr] \nonumber
\end{align}
with the inverse temperature $\beta = 1 /T$.
To solve this equation, let us assume that the Fermi level (corresponding to
momentum $k_F$) lies below zero, i.e. $\mu = -J_g \sqrt{2 (1 + \cos k_F)} < 0$.
Near the Fermi level $\delta k = \vert k \vert - k_F$, $T_{k, +1} = 2 \vert
\mu \vert$ and $T_{k, -1} = \sqrt{\Delta_0^2 + v_F^2 \delta k^2}$, with Fermi
velocity $v_F = J_g \sqrt{1 - (\mu / 2 J_g)^2}$.
In the weak coupling and zero-temperature limits, the main contribution to the
sum comes from around the Fermi surface:
\begin{align}
 1 \approx & \frac{u_{g g}}{4 N_d} \sum_k \frac{1}{T_{k, -1}} \biggl[ 1 -
 \frac{J_g}{R_k} (1 + \cos k)^2 \biggr]_{k = k_F} \!\! = \frac{u_{g g}}{4}
 \frac{2}{2 \pi} \nonumber \\
 \times & \int_0^{\lesssim k_F} \frac{d k}{\sqrt{\Delta_0^2 + v_F^2 \delta
 k^2}} \frac{v_F^2}{J_g^2} = \frac{u_{g g}}{2 \pi J_g} \sqrt{1 -
 \frac{\mu^2}{(2 J_g)^2}} \ln \frac{2 \rho}{\Delta_0}. \nonumber
\end{align}
From here, $\Delta_0 = 2 \rho \e^{-2 \pi J_g / u_{g g} \sqrt{1 - (\mu / 2
J_g)^2}}$ with $\rho \sim J_g \gg \Delta_0$ -- a characteristic energy measured
from the Fermi level beyond which the above expansion becomes invalid.

The OP $\Delta_0$ involves $g$-fermions with opposite pseudospin $a$ and
describes an $s$-wave Cooper pairing.
However, due to the SOC inherent in $\hat{H}^g_0$, there is also an admixture
of the $p$-wave pairing states of two $g$-atoms with the same pseudospin which,
because of Fermi statistics, must be antisymmetric in momentum.
To quantify these $p$-wave correlations, we compute a spin-averaged pairing
amplitude ${\cal P} = \frac{1}{2 \Delta_0} \sum_a \langle \hat{g}_{k a}
\hat{G}_{k a} \rangle$ using the BdG wavefunctions \eqref{eq:BdG_wf_1d}:
\begin{displaymath}
 {\cal P} \! = \! \frac{\mu \sin k}{4 R_k} \sideset{}{_\tau} \sum
 \frac{\tau}{T_{k \tau}} {\rm th} \, \frac{\beta T_{k \tau}}{2}.
\end{displaymath}
This result demonstrates that $p$-wave correlations in the system are
enslaved to the $s$-wave OP.
Moreover, the $\tau$-dependence of the quasiparticle energy is essential: if
$T_{k, +1} = T_{k, -1}$, as it would be in the absence of SOC, there is no
$p$-wave pairing and ${\cal P} \equiv 0$.

Finally, we derive the BCS approximation in the weak-coupling limit described
by Eq. (3).
Introducing the pairing OP $\Delta = -\frac{u_{g g}}{4 N_d} \sum_k \e^{-\ii
\frac{k}{2}} \langle \hat{f}_{-k} \hat{f}_k \rangle$, one can linearize the
Hamiltonian (3):
\begin{align}
 \hat{H}_{\rm ef} \approx & \sideset{}{_k} \sum \bigl[ (\epsilon_k - \mu) \fd_k
 \hat{f}_k + \Delta \e^{\ii \frac{k}{2}} \fd_k \fd_{-k} + {\rm h. c.} \bigr]
 \label{eq:H_BCS_f} \\
 = & \sideset{}{^\prime_k} \sum \biggl[ (\epsilon_k - \mu) +
 \begin{pmatrix}
  \fd_k & \!\! \hat{F}_k
 \end{pmatrix}
 \begin{pmatrix}
  \epsilon_k - \mu & D_k \\
  D_k^* & \mu - \epsilon_k
 \end{pmatrix}
 \begin{pmatrix}
  \hat{f}_k \\
  \hat{F}^\dag_k
 \end{pmatrix}
 \biggr],
 \nonumber
\end{align}
where $D_k = 2 \ii \Delta \sin \frac{k}{2}$, $\epsilon_{-k} = \epsilon_k$, as
before $\sum^\prime_k \ldots$ extends over half of the BZ, and $\hat{F}_k =
\hat{f}_{-k}$.
This Hamiltonian is diagonalized by a Bogoliubov transformation from
$f$-fermions to quasiparticles $\hat{\gamma}_k$ and $\hat{\Gamma}_k$
\cite{landau-vol-9}:
\begin{align}
 \begin{pmatrix}
  \hat{f}_k \\
  \hat{F}^\dag_k
 \end{pmatrix}
 = & \frac{1}{\sqrt{2 E_k}} \biggl[ \frac{1}{\sqrt{E_k + (\epsilon_k - \mu)}}
 \begin{pmatrix}
  E_k + (\epsilon_k - \mu) \\
  D^*_k
 \end{pmatrix}
 \hat{\gamma}_k \nonumber \\
 - & \frac{1}{\sqrt{E_k - (\epsilon_k - \mu)}}
 \begin{pmatrix}
  E_k - (\epsilon_k - \mu) \\
  -D^*_k
 \end{pmatrix}
 \hat{\Gamma}^\dag_k
 \biggr].
\label{eq:H_BCS_f_qp}
\end{align}
Here $E_k = \sqrt{(\epsilon_k - \mu)^2 + \vert D_k \vert^2}$ is the
quasiparticle dispersion.
The SF gap has a $p$-wave symmetry: $D_k = -D_{-k}$ and its amplitude $\Delta$
is determined from the self-consistency equation $1 = \frac{u_{g g}}{4 N_d}
\sum_k \sin^2 \frac{k}{2} / E_k$: $\Delta \sim \e^{-2 \pi J_g / u_{g g} \sqrt{1
- (\mu / 2 J_g)^2}}$, which coincides with the result obtained in the
multi-band case.

%
\subsection*{Topological superfluidity in a 2D lattice}
%

The results presented in the main text, can be extended to higher dimensional
systems.
Here we briefly consider a generalization involving a lattice that has the
plaquette structure shown in Fig. \ref{fig:sfig1}(a).
In each plaquette, there is one $e$-atom that can tunnel around the plaquette
(with lattice constant $a_0$).
On the other hand, $g$-fermions move in the simpler square lattice.
As before, all atoms are nuclear-spin-polarized and we use units where $a_0 =
1$.
The Hamiltonian of this system is
\begin{align}
 \hat{H} = & \hat{H}^e_0 + \hat{H}^g_0 + \hat{H}_{\rm int};
 \label{eq:H_2d} \\ 
 \hat{H}^e_0 = & -J_e \sum_{i, \, \Box_{a b}} \bigl( \ed_{i a} \hat{e}_{i b} +
 {\rm h. c.} \bigr); \nonumber \\
 \hat{H}^g_0 = & -J_g \biggl[ \,\, \sum_{i, \, \Box_{a b}} \bigl( \gd_{i a}
 \hat{g}_{i b} + {\rm h.c.} \bigr) \nonumber \\
 + & \!\! \sum_{\langle i j \rangle_x} \! \bigl( \gd_{i 2} \hat{g}_{j 1} +
 \gd_{i 3} \hat{g}_{j 4} \bigr) \! + \!\! \sum_{\langle i j \rangle_y} \!
 \bigl( \gd_{i, 4} \hat{g}_{j, 1} + \gd_{i 3} \hat{g}_{j 2} \bigr) \! +
 \! {\rm h. c.} \biggr]; \nonumber \\
 \hat{H}_{\rm int} = & \,\, U^-_{e g} \sum_{i a} \hat{n}^g_{i a}
 \hat{n}^e_{i a}, \nonumber
\end{align}
where $i = \bx_i = (x_i, y_i)$, $i$, $j = 1, \ldots, N_\Box$ label plaquettes
in the lattice, $a$, $b = 1, 2, 3, 4$ denote wells inside a plaquette,
$\Box_{a b}$ indicates all sides $\langle a b \rangle$ of a square, and
$\langle i j \rangle_{x, y}$ is a link connecting two plaquettes in the $x$ or
$y$ direction [Fig. \ref{fig:sfig1}(a)].
Other notations are the same as in Eq. (1).

In the rest of this section, we shall follow an analysis similar to the one in
the main text:
First, we consider $e$-atom states of an isolated plaquette and identify a
spatial mode that gives rise to the $g$-atom attraction within that plaquette.
Then, we study a weak-coupling dilute limit $u_{g g} \ll J_g$, $n^g \ll 1$ and
demonstrate the stability of a chiral $p_x + \ii p_y$ TSF state.
The study of the phase diagram, magneto-electric effects, etc. is left for a
future investigation.

\paragraph{Pairing of the $g$-atoms.--}
An $e$-atom localized within a plaquette, has four states $\lambda = 0, \ldots,
3$ shown in Fig. \ref{fig:sfig1}(b).
Their wavefunctions are $\vert \lambda \rangle_i = \sum_a X^{(\lambda)}_a
\ed_{i a} \vac$ with $X^{(\lambda)}_a = \frac{1}{2} \e^{\ii \frac{\pi}{2}
\lambda (a - 1)}$ and corresponding energies $E^e_0 = -E^e_2 = -2 J_e$, $E^e_1
= E^e_3 = 0$.
Matrix elements of the density $\hat{n}^e_{i a}$ are ${}_i \langle
\lambda^\prime \vert \hat{n}^e_{i a} \vert \lambda \rangle_i =
X^{(\lambda^\prime) *}_a X^{(\lambda)}_a = \frac{1}{4} \e^{\ii \frac{\pi}{2}
(\lambda - \lambda^\prime) (a - 1)}$.

In the 1D case we saw that an attractive interaction between $g$-atoms occurs
when the $e$-subsystem is prepared in the highest excited kinetic energy state.
An analogous situation happens here: $e$-atoms must fluctuate out of the
$d$-wave $\lambda = 2$ state with energy $2 J_e$.
Using the Schrieffer-Wolff transformation derived in Methods and performing
similar calculations, we obtain the second-order $g$-atom Hamiltonian
\begin{align}
 \hat{H}_{\rm ef} = & \hat{H}^g_0 - u_{g g} \sideset{}{_i} \sum \bigl[
 (\hat{n}^g_{i 1} + \hat{n}^g_{i 3}) (\hat{n}^g_{i 2} + \hat{n}^g_{i 4})
 \nonumber \\
 & + 3 (\hat{n}^g_{i 1} \hat{n}^g_{i 3} + \hat{n}^g_{i 2} \hat{n}^g_{i 4})
 \bigr].
 \label{eq:H_ef-2d}
\end{align}
Here $u_{g g} = (U^-_{e g})^2 / 32 J_e$, and density terms in the 1st (2nd)
line describe attractive interactions along sides (diagonals) of the $i$-th
plaquette [gray ellipses in Fig. \ref{fig:sfig1}(a)].

\begin{figure}[t]
 \begin{center}
  \includegraphics[width = \columnwidth]{./sfig1.pdf}
 \end{center}
 \caption{
  {\bf (a)} Two-dimensional optical superlattice with a plaquette structure.
  The $e$-atoms are confined within red square plaquettes (with one atom per
  cell), $g$-atoms propagate in a simple square lattice (thin blue lines).
  Gray ellipses denote $g$-atom attraction mediated by fluctuations of the
  $e$-subsystem.
  Other notations are as in Fig. 1(a).
  %
  {\bf (b)} States of an $e$-atom within a plaquette.
  Also shown are point symmetries of the wavefunctions $X^{(\lambda)}_a =
  \frac{1}{2} \e^{\ii \frac{\pi}{2} \lambda (a - 1)}$ with colors indicating
  phases $\pm 1$ and $\pm \ii$.
  In particular, the $\lambda = 0$ ($2$) state has $s$- ($d$-) wave symmetry.
  Attractive interactions between $g$-atoms occur when $e$-atoms are prepared
  in the $\lambda = 2$ state.
 }
 \label{fig:sfig1}
\end{figure}

\paragraph{Weak-coupling dilute regime.--}
The $g$-atom kinetic energy can be written as
\begin{displaymath}
 \hat{H}^g_0 = -J_g \sum_\bk \gd_{\bk a}
 \begin{pmatrix}
  0 & \eta_{k_x} & 0 & \eta_{k_y} \\
  \eta^*_{k_x} & 0 & \eta_{k_y} & 0 \\
  0 & \eta^*_{k_y} & 0 & \eta^*_{k_x} \\
  \eta^*_{k_y} & 0 & \eta_{k_x} & 0
 \end{pmatrix}
 _{\!\!\!\! a b} \hat{g}_{\bk b},
\end{displaymath}
where $\hat{g}_{\bk a} = \frac{1}{\sqrt{N_\Box}} \sum_i \e^{-\ii (\bk \cdot
\bx_i)} \hat{g}_{i a}$, momentum $\bk = (k_x, k_y)$ with $k_\alpha \in [-\pi,
\pi]$ ($\alpha = x$, $y$) is defined in a plaquette BZ with $N_\Box$ states and
$\eta_{k_\alpha} = 1 + \e^{-\ii k_\alpha}$.
Eigenvalues of the matrix in $\hat{H}_0^g$ are $\pm 2 J_g \bigl[ \cos
\frac{k_x}{2} \pm \cos \frac{k_y}{2} \bigr]$.
At weak coupling $u_{g g} \ll J_g$ and $n^g \ll 1$, only the lowest band
$\epsilon_\bk = - 2 J_g \bigl[ \cos \frac{k_x}{2} + \cos \frac{k_y}{2} \bigr]
\approx -4 J_g + \frac{k^2}{2 m^*}$ is populated ($m^* = 2 / J_g$ is the
effective mass).
The corresponding eigenvector is $\psi_0 = \frac{1}{2} \bigl( 1, \,
-\e^{-\ii \varphi_x}, \, \e^{-\ii (\varphi_x + \varphi_y)}, \,
-\e^{-\ii \varphi_y} \bigr)^T$ with $\e^{\ii \varphi_\alpha} = \eta_{k_\alpha}
/ \vert \eta_{k_\alpha} \vert$.
At small momenta, $\psi_0 \approx \frac{1}{2} \bigl( 1, \, \e^{\ii k_x / 2}, \,
\e^{\ii (k_x + k_y) / 2}, \, \e^{\ii k_y / 2} \bigr)^T$.

The low-energy effective Hamiltonian can be obtained by replacing $\hat{g}_\bk
\to \psi_0 f_\bk$ ($f_\bk$ is the lowest-band fermion quasiparticle) in Eq.
\eqref{eq:H_ef-2d}:
\begin{displaymath}
 \hat{H}_{\rm ef} \approx \sideset{}{_\bk} \sum \epsilon_\bk \fd_\bk
 \hat{f}_\bk + \frac{u_{g g}}{16 N_\Box} \sideset{}{_{\bk^\prime \bk \bq}} \sum
 \bq^2 \fd_{\bk + \bq} \fd_{\bk^\prime - \bq} \hat{f}_{\bk^\prime} \hat{f}_\bk.
\end{displaymath}
This expression describes a system of spinless fermions interacting via an
attractive $p$-wave coupling.
Within the BCS approximation we have:
\begin{align}
 \hat{H}_{\rm ef} \approx & \sideset{}{_\bk} \sum \bigl[ \xi_\bk \fd_\bk
 \hat{f}_\bk + {\textstyle \frac{1}{2}} (\bk \cdot {\bm \Delta}) \fd_\bk
 \fd_{-\bk} + {\rm h. c.} \bigr] \nonumber \\
 & = \sideset{}{^\prime_\bk} \sum \biggl[ \xi_\bk +
 \begin{pmatrix}
  \fd_\bk & \hat{F}_\bk
 \end{pmatrix}
 \begin{pmatrix}
  \xi_\bk & D_\bk \\
  D^*_\bk & -\xi_\bk
 \end{pmatrix}
 \begin{pmatrix}
  \hat{f}_\bk \\
  \hat{F}^\dag_\bk
 \end{pmatrix}
 \biggr]. \nonumber
\end{align}
Here we switched to a grand-canonical ensemble with a chemical potential $\mu$
and performed the standard replacement $\hat{H}_{\rm ef} \to \hat{H}_{\rm ef} -
\mu \sum_{i a} \hat{n}^g_{i a}$.
$\xi_\bk = \epsilon_\bk - \mu$, $D_\bk = (\bk \cdot {\bm \Delta})$, and other
notations are the same as in the 1D weak-coupling case [see Eq. (1) in the main
text and Methods].
The above Hamiltonian is diagonalized by a Bogoliubov transformation identical
to the one used in the 1D case after a replacement $k \to \bk$.
The pairing OP ${\bm \Delta} = -\frac{u_{g g}}{2 N_\Box} \sum^\prime_\bk \bk
\langle \hat{F}_\bk \hat{f}_\bk \rangle$ obeys the BCS equation
\begin{equation}
 {\bm \Delta} = \frac{u_{g g}}{4 N_\Box} \sideset{}{^\prime_\bk} \sum
 \frac{\bk (\bk \cdot {\bm \Delta})}{E_\bk} = \frac{u_{g g}}{8 N_\Box}
 \sideset{}{_\bk} \sum \frac{\bk (\bk \cdot {\bm \Delta})}{E_\bk},
 \label{eq:bcs-2d}
\end{equation}
where $E_\bk = \sqrt{\xi_\bk^2 + \vert D_\bk \vert^2}$ and in the last sum we
used evenness of the integrand and extended summation over the entire BZ.
Whenever the SF phase develops, the grand potential ${\cal G}_s$ in that phase
is reduced compared to its normal-state value ${\cal G}_n$.
This shift can be computed using the Hellman-Feynman theorem
\cite{landau-vol-9}
\begin{displaymath}
 \frac{\delta {\cal G}}{N_\Box} = \frac{{\cal G}_s - {\cal G}_n}{N_\Box} =
 \int_0^{u_{g g}} \! d \widetilde{u} \, \frac{\langle \hat{H}_{\rm int} \rangle}
 {N_\Box \, \widetilde{u}} = -2 \int_0^{u_{g g}} \! d \widetilde{u} \,
 \frac{\vert \Delta (\widetilde{u}) \vert^2}{\widetilde{u}^2}.
\end{displaymath}
There are two competing SF states characterized by different symmetries
of the OP ${\bm \Delta}$: (i) chiral $p_x \pm \ii p_y$ phase with ${\bm
\Delta}_{p + \ii p} = \Delta_1 ({\bm e}^p_x \mp \ii {\bm e}^p_y)$ ($\bp = p_x
{\bm e}^p_x + p_y {\bm e}^p_y$), and (ii) $p_x$ (or $p_y$) state with ${\bm
\Delta}_{p_x} = \Delta_2 {\bm e}^p_x$ (or $\Delta_2 {\bm e}^p_y$).
Below we assume real amplitudes $\Delta_{1, 2}$ and demonstrate that at weak
coupling, the $p_x + \ii p_y$ state is favored.

\begin{figure}[t]
 \begin{center}
  \includegraphics[width = \columnwidth]{./sfig2.pdf}
 \end{center}
 \caption{
  Inverse compressibility ${\cal X}$ of a chain with periodic and anti-periodic
  boundary conditions (PBC and anti-PBC) computed using exact diagonalization
  for {\bf (a)} $N = 24$ and {\bf (b)} $N = 26$ sites.
  The number of $g$-atoms is even: $N_g = 2, \ldots, 10$.
  Red, blue and green curves correspond to interaction strengths $u_{g g} / J_g
  = 0.03$, $0.17$ and $1.03$, respectively.
  Arrows show direction of increasing $u_{g g}$ in the above interval.
 }
 \label{fig:sfig2}
\end{figure}

\paragraph{$p_x + \ii p_y$ SF.--}
The $u_{g g}$-dependence of $\Delta_1$ can be determined by multiplying Eq.
\eqref{eq:bcs-2d} by ${\bm \Delta}^*$ [notice that $\vert {\bm \Delta} \vert^2
= 2 \Delta_1^2$ and $\vert \bk \cdot {\bm \Delta} \vert^2 = \bk^2 \Delta_1^2$]:
$2 = \frac{u_{g g}}{8 N_\Box} \sum_\bk \frac{\bk^2}{E_\bk} = \frac{u_{g g}}{16
\pi} \int d k \frac{k^3}{E_\bk} \approx \kappa \ln \frac{2 \rho}{k_F
\Delta_1}$, where $\kappa = u_{g g} Q_F$ with $Q_F \equiv \frac{m^* k_F^2}{8
\pi}$, $k_F$ is the Fermi momentum, and $\rho \sim J_g \gg \Delta_1$ is a
characteristic energy cutoff [cf. Methods]. We have:
\begin{displaymath}
 \frac{\delta {\cal G}_1}{N_\Box} = 4 Q_F \int_0^{\Delta_1} d \bigl(
 {\textstyle \frac{1}{\kappa}} \bigr) \Delta^2 = -Q_F \Delta_1^2 =
 -\frac{\rho^2 m^*}{2 \pi} \e^{-4 / \kappa}.
\end{displaymath}

\paragraph{$p_x$ SF.--}
Proceeding in a similar manner as above, we use the BCS equation
\eqref{eq:bcs-2d} to calculate $\Delta_2$
\begin{align}
 1 = & \frac{u_{g g}}{8 N_\Box} \! \sum_\bk \frac{k_x^2}{E_\bk} =
 \frac{u_{g g}}{8} \int \frac{d k \, k}{2 \pi} \oint \frac{d \theta}{2 \pi}
 \frac{k^2 \cos^2 \theta}{\sqrt{\xi_\bk^2 + k^2 \Delta_2^2 \cos^2 \theta}}
 \nonumber \\
 \approx & \kappa \oint \frac{d \theta}{2 \pi} \, \cos^2 \theta \, \ln
 \frac{2 \rho}{k_F \Delta_2 \vert \cos \theta \vert} = \frac{\kappa}{2} \biggl[
 \ln \frac{2 \rho}{k_F \Delta_2} + I \biggr], \nonumber
\end{align}
where $I = -2 \oint \frac{d \theta}{2 \pi} \cos^2 \theta \ln \vert \cos \theta
\vert \approx 0.2$.
$\delta {\cal G}$ can be computed as before
\begin{align}
 \frac{\delta {\cal G}_2}{N_\Box} = & 2 Q_F \int_0^{\Delta_2} d \bigl(
 {\textstyle \frac{1}{\kappa}} \bigr) \Delta^2 = -\frac{Q_F \Delta_2^2}{2} =
 -\frac{\rho^2 m^*}{2 \pi} \frac{\e^{2 I}}{2} \e^{-4 / \kappa}.
 \nonumber
\end{align}
Because $\e^{2 I} / 2 < 1$, $\delta {\cal G}_1 < \delta {\cal G}_2$ and the
$p_x \pm \ii p_y$ SF state is preferred over the striped $p_{x, y}$ phase.

%
\subsection*{Fermion parity switch in the Hamiltonian (2)}
%

In this section, we show that the exact GS of the effective Hamiltonian (2)
realizes a fermionic parity switch \cite{ortiz-2014-1,ortiz-2016-1}, i.e. the
inverse compressibility, ${\cal X} = \frac{1}{2 N} \bigl[ E_0 (N_g + 1) + E_0
(N_g - 1) \bigr] - E_0 (N_g)$, changes sign when the boundary conditions are
switched from periodic to anti-periodic, thus confirming the topological nature
of our system beyond mean-field.
In Fig. \ref{fig:sfig2} we present results of the exact diagonalization (using
the Lanczos technique of Ref. \cite{wu-2000-1}) in systems with $N = 24$ and
$26$ sites (12 and 13 dimers, respectively) and even number of $g$-atoms $N_g =
2, \ldots, 10$.
Clearly, ${\cal X}$ has different sign for the two types of boundary
conditions.

%merlin.mbs apsrev4-1.bst 2010-07-25 4.21a (PWD, AO, DPC) hacked
%Control: key (0)
%Control: author (72) initials jnrlst
%Control: editor formatted (1) identically to author
%Control: production of article title (-1) disabled
%Control: page (0) single
%Control: year (1) truncated
%Control: production of eprint (0) enabled
\begin{thebibliography}{10}%
\makeatletter
\providecommand \@ifxundefined [1]{%
 \@ifx{#1\undefined}
}%
\providecommand \@ifnum [1]{%
 \ifnum #1\expandafter \@firstoftwo
 \else \expandafter \@secondoftwo
 \fi
}%
\providecommand \@ifx [1]{%
 \ifx #1\expandafter \@firstoftwo
 \else \expandafter \@secondoftwo
 \fi
}%
\providecommand \natexlab [1]{#1}%
\providecommand \enquote  [1]{``#1''}%
\providecommand \bibnamefont  [1]{#1}%
\providecommand \bibfnamefont [1]{#1}%
\providecommand \citenamefont [1]{#1}%
\providecommand \href@noop [0]{\@secondoftwo}%
\providecommand \href [0]{\begingroup \@sanitize@url \@href}%
\providecommand \@href[1]{\@@startlink{#1}\@@href}%
\providecommand \@@href[1]{\endgroup#1\@@endlink}%
\providecommand \@sanitize@url [0]{\catcode `\\12\catcode `\$12\catcode
  `\&12\catcode `\#12\catcode `\^12\catcode `\_12\catcode `\%12\relax}%
\providecommand \@@startlink[1]{}%
\providecommand \@@endlink[0]{}%
\providecommand \url  [0]{\begingroup\@sanitize@url \@url }%
\providecommand \@url [1]{\endgroup\@href {#1}{\urlprefix }}%
\providecommand \urlprefix  [0]{URL }%
\providecommand \Eprint [0]{\href }%
\providecommand \doibase [0]{http://dx.doi.org/}%
\providecommand \selectlanguage [0]{\@gobble}%
\providecommand \bibinfo  [0]{\@secondoftwo}%
\providecommand \bibfield  [0]{\@secondoftwo}%
\providecommand \translation [1]{[#1]}%
\providecommand \BibitemOpen [0]{}%
\providecommand \bibitemStop [0]{}%
\providecommand \bibitemNoStop [0]{.\EOS\space}%
\providecommand \EOS [0]{\spacefactor3000\relax}%
\providecommand \BibitemShut  [1]{\csname bibitem#1\endcsname}%
\let\auto@bib@innerbib\@empty
%</preamble>
\bibitem [{\citenamefont {Bir}\ and\ \citenamefont {Pikus}(1974)}]{bir-1974-1}%
  \BibitemOpen
  \bibfield  {author} {\bibinfo {author} {\bibfnamefont {G.}~\bibnamefont
  {Bir}}\ and\ \bibinfo {author} {\bibfnamefont {G.}~\bibnamefont {Pikus}},\
  }\href {http://books.google.com/books?id=38m2QgAACAAJ} {\emph {\bibinfo
  {title} {Symmetry and Strain-Induced Effects in Semiconductors}}}\ (\bibinfo
  {publisher} {John Wiley and Sons},\ \bibinfo {year} {1974})\BibitemShut
  {NoStop}%
\bibitem [{\citenamefont {Blaizot}\ and\ \citenamefont
  {Ripka}(1986)}]{blaizot-1986-1}%
  \BibitemOpen
  \bibfield  {author} {\bibinfo {author} {\bibfnamefont {J.}~\bibnamefont
  {Blaizot}}\ and\ \bibinfo {author} {\bibfnamefont {G.}~\bibnamefont
  {Ripka}},\ }\href@noop {} {\emph {\bibinfo {title} {Quantum Theory of Finite
  Systems}}}\ (\bibinfo  {publisher} {Cambridge, MA},\ \bibinfo {year}
  {1986})\BibitemShut {NoStop}%
\bibitem [{\citenamefont {Dao}\ \emph {et~al.}(2009)\citenamefont {Dao},
  \citenamefont {Carusotto},\ and\ \citenamefont {Georges}}]{dao-2009-1}%
  \BibitemOpen
  \bibfield  {author} {\bibinfo {author} {\bibfnamefont {T.-L.}\ \bibnamefont
  {Dao}}, \bibinfo {author} {\bibfnamefont {I.}~\bibnamefont {Carusotto}}, \
  and\ \bibinfo {author} {\bibfnamefont {A.}~\bibnamefont {Georges}},\ }\href
  {\doibase 10.1103/PhysRevA.80.023627} {\bibfield  {journal} {\bibinfo
  {journal} {Phys. Rev. A}\ }\textbf {\bibinfo {volume} {80}},\ \bibinfo
  {pages} {023627} (\bibinfo {year} {2009})}\BibitemShut {NoStop}%
\bibitem [{\citenamefont {Zubarev}(1960)}]{zubarev-1960-1}%
  \BibitemOpen
  \bibfield  {author} {\bibinfo {author} {\bibfnamefont {D.~N.}\ \bibnamefont
  {Zubarev}},\ }\href {\doibase 10.1070/PU1960v003n03ABEH003275} {\bibfield
  {journal} {\bibinfo  {journal} {Phys. Usp.}\ }\textbf {\bibinfo {volume}
  {3}},\ \bibinfo {pages} {320} (\bibinfo {year} {1960})}\BibitemShut {NoStop}%
\bibitem [{\citenamefont {Ojanen}(2012)}]{ojanen-2012-1}%
  \BibitemOpen
  \bibfield  {author} {\bibinfo {author} {\bibfnamefont {T.}~\bibnamefont
  {Ojanen}},\ }\href {\doibase 10.1103/PhysRevLett.109.226804} {\bibfield
  {journal} {\bibinfo  {journal} {Phys. Rev. Lett.}\ }\textbf {\bibinfo
  {volume} {109}},\ \bibinfo {pages} {226804} (\bibinfo {year}
  {2012})}\BibitemShut {NoStop}%
\bibitem [{\citenamefont {Bardeen}\ \emph {et~al.}(1957)\citenamefont
  {Bardeen}, \citenamefont {Cooper},\ and\ \citenamefont
  {Schrieffer}}]{bardeen-1957-1}%
  \BibitemOpen
  \bibfield  {author} {\bibinfo {author} {\bibfnamefont {J.}~\bibnamefont
  {Bardeen}}, \bibinfo {author} {\bibfnamefont {L.~N.}\ \bibnamefont {Cooper}},
  \ and\ \bibinfo {author} {\bibfnamefont {J.~R.}\ \bibnamefont {Schrieffer}},\
  }\href {\doibase 10.1103/PhysRev.108.1175} {\bibfield  {journal} {\bibinfo
  {journal} {Phys. Rev.}\ }\textbf {\bibinfo {volume} {108}},\ \bibinfo {pages}
  {1175} (\bibinfo {year} {1957})}\BibitemShut {NoStop}%
\bibitem [{\citenamefont {Landau}\ \emph {et~al.}(1980)\citenamefont {Landau},
  \citenamefont {Lifshitz},\ and\ \citenamefont {Pitaevskii}}]{landau-vol-9}%
  \BibitemOpen
  \bibfield  {author} {\bibinfo {author} {\bibfnamefont {L.}~\bibnamefont
  {Landau}}, \bibinfo {author} {\bibfnamefont {E.}~\bibnamefont {Lifshitz}}, \
  and\ \bibinfo {author} {\bibfnamefont {L.}~\bibnamefont {Pitaevskii}},\
  }\href {https://books.google.com/books?id=NaB7oAkon9MC} {\emph {\bibinfo
  {title} {Statistical Physics}}},\ \bibinfo {series} {Course of theoretical
  physics}\ No.\ \bibinfo {number} {pt. 2}\ (\bibinfo  {publisher} {Pergamon
  Press},\ \bibinfo {year} {1980})\BibitemShut {NoStop}%
\bibitem [{\citenamefont {Ortiz}\ \emph {et~al.}(2014)\citenamefont {Ortiz},
  \citenamefont {Dukelsky}, \citenamefont {Cobanera}, \citenamefont {Esebbag},\
  and\ \citenamefont {Beenakker}}]{ortiz-2014-1}%
  \BibitemOpen
  \bibfield  {author} {\bibinfo {author} {\bibfnamefont {G.}~\bibnamefont
  {Ortiz}}, \bibinfo {author} {\bibfnamefont {J.}~\bibnamefont {Dukelsky}},
  \bibinfo {author} {\bibfnamefont {E.}~\bibnamefont {Cobanera}}, \bibinfo
  {author} {\bibfnamefont {C.}~\bibnamefont {Esebbag}}, \ and\ \bibinfo
  {author} {\bibfnamefont {C.}~\bibnamefont {Beenakker}},\ }\href {\doibase
  10.1103/PhysRevLett.113.267002} {\bibfield  {journal} {\bibinfo  {journal}
  {Phys. Rev. Lett.}\ }\textbf {\bibinfo {volume} {113}},\ \bibinfo {pages}
  {267002} (\bibinfo {year} {2014})}\BibitemShut {NoStop}%
\bibitem [{\citenamefont {Ortiz}\ and\ \citenamefont
  {Cobanera}(2016)}]{ortiz-2016-1}%
  \BibitemOpen
  \bibfield  {author} {\bibinfo {author} {\bibfnamefont {G.}~\bibnamefont
  {Ortiz}}\ and\ \bibinfo {author} {\bibfnamefont {E.}~\bibnamefont
  {Cobanera}},\ }\href {\doibase http://dx.doi.org/10.1016/j.aop.2016.05.020}
  {\bibfield  {journal} {\bibinfo  {journal} {Annals of Physics}\ }\textbf
  {\bibinfo {volume} {372}},\ \bibinfo {pages} {357} (\bibinfo {year}
  {2016})}\BibitemShut {NoStop}%
\bibitem [{\citenamefont {Wu}\ and\ \citenamefont {Simon}(2000)}]{wu-2000-1}%
  \BibitemOpen
  \bibfield  {author} {\bibinfo {author} {\bibfnamefont {K.}~\bibnamefont
  {Wu}}\ and\ \bibinfo {author} {\bibfnamefont {H.}~\bibnamefont {Simon}},\
  }\href {\doibase 10.1137/S0895479898334605} {\bibfield  {journal} {\bibinfo
  {journal} {SIAM Journal on Matrix Analysis and Applications}\ }\textbf
  {\bibinfo {volume} {22}},\ \bibinfo {pages} {602} (\bibinfo {year}
  {2000})}\BibitemShut {NoStop}%
\end{thebibliography}%

\end{document}
% vim: set shiftwidth=1 spell spelllang=en_us:
