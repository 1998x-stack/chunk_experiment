%%%%%%%%%%%%%%%%%%%%%%%%%%%%%%%%%%%%%%%%%%%%%%%%%
% conclusions 
%%%%%%%%%%%%%%%%%%%%%%%%%%%%%%%%%%%%%%%%%%%%%%%%%

\section{Conclusions}
\label{sec:conclusions}

%We have characterized when 
%one labelled triangulation of a set of points can be transformed into another labelled triangulation of the same point set by a sequence of flips, and bounded the 
We have characterized when 
two labelled triangulations of a set of $n$ points belong to the same connected component of the labelled flip graph, and proved that the diameter of each connected component is bounded by $O(n^7)$.
We conclude with some open problems:

\begin{enumerate}

\item 
%\medskip
%\noindent{\bf 1.} 
Reduce the gap between the upper bound, $O(n^7)$, and the best known lower bound of $O(n^3)$~\cite{bose2013flipping} on the diameter of a component of the labelled flip graph.
%\note{Reword this if we wish to avoid ``labelled flip graph''.}

\item 
%\comment{
%\noindent{\bf 2.}  
We have studied the case where each edge in a triangulation has a unique label, and given a bound of $O(n^7)$ on the diameter of a component of the labelled flip graph.  The case where edges are unlabelled can be viewed as the case where every edge has the same label---in this case the bound becomes $O(n^2)$.  
A unifying scenario is when the edges have labels and labels may appear on more than one edge.
Is there a bound on the diameter of connected components of the flip graph that depends on the number of labels, or on the maximum number of edges with the same label?
% Note: this is the question that Beppe Liotta asked after Zuzka's talk at SoCG.
%
%Consider partially respected edge labellings: given two labelled triangulations and a real number $r \in [0,1]$, does there exist a flip sequence reconfiguring one triangulation into the other while ignoring at most an $r$-fraction of the labels? In the case of unlabelled triangulations, $r=1$ and the length of such a flip sequence is $\theta(n^2)$. We have investigated the case of $r=0$.
%Investigate the diameter of connected components of a flip graph for triangulations partially respecting the edge labelling   
%}

\item
%\noindent{\bf 3.} 
We did not analyze the run-time of our algorithms in the main text.  A crude bound is $O(n^8)$, with the bottleneck being the explicit construction in the proof of Lemma~\ref{lemma:elem-swap} of the double quadrilateral graph which has $O(n^4)$ vertices and thus $O(n^8)$ edges.  This bound can surely be improved. 

 \item 
 %\noindent{\bf 4.}  
 What is the complexity of the following flip distance problem for labelled triangulations: Given two labelled triangulations and a number $k$, is there a flip sequence of length at most $k$ to transform the first triangulation to the second one?
This problem is NP-complete in the unlabelled setting, but 
knowing the mapping of edges might make the problem easier.
%it is possible that the hardness arises from not knowing the mapping of edges.

\end{enumerate}  

   