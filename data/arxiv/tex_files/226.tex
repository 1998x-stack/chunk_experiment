\section{Spectral gap of simulated tempering}
\label{s:main}

%Recall, the theorem we wish to show is the following.
In this section we prove a lower bound for the spectral gap of simulated tempering given a partition.

\begin{asm}\label{asm}
Let $M_i=(\Om, P_i), i\in [L]$ be a sequence of Markov chains with state space $\Om$ and stationary distributions $p_i$.
Consider the simulated tempering chain $M_{\st}=(\Om\times[L], P_{\st})$ %
 with probabilities $(r_i)_{i=1}^{L}$. Let $r = \fc{\min(r_i)}{\max(r_i)}$.  

Let $\cal P_i$
%$\cal P=\set{A_j}{1\le j\le n}$ 
be a partition\footnote{We allow overlaps on sets of measure 0.} of the ground set $\Om$, for each $i\in [L]$, with $\cal P_1=\{\Om\}$. 

Define the \vocab{overlap parameter} of $(\cal P_i)_{i=1}^L$ to be 
$$
\de((\cal P_i)_{i=1}^L) = \min_{1< i\le L, A\in \cal P} \ba{
\int_{A} \min\{p_{i-1}(x),p_i(x)\} \dx 
}/p_i(A).
$$
\end{asm}

\begin{thm}
Suppose Assumptions~\ref{asm} hold. Define
$$
p_{\min}=\min_{i,A\in \cal P_i} p_i(A). 
$$

Then the spectral gap of the tempering chain satisfies
\begin{align}
\Gap(M_{\st}) &\ge\fc{r^4\de^2p_{\min}^2}{32L^4}
%{512L^2}
\min_{1\le i\le L, A\in \cal P_i} (\Gap(M_{i}|_A)).
\end{align} 
\label{t:temperingnochain}
\end{thm} 


\begin{proof}
Let $p_{\st}$ be the stationary distribution of $P_{\st}$.
First note that we can easily switch between $p_i$ and $p_{\st}$ using $p_{\st}(A\times \{i\}) = w_i p_i(A)$.  Note $w_i\ge \fc rL$. %\Hnote{May be more natural to bound in terms of $\min w_i$ instead of $r$.}

Define the partition $\cal P$ on $\Om\times [L]$ by 
$$
\cal P = \set{A\times \{i\}}{A\in \cal P_i}.
$$

By Theorem~\ref{thm:gap-product},
\begin{align}\label{eq:st-gap-prod0}
\Gap(M_{\st}) & \ge \rc2 \Gap(\ol M_{\st}) 
\min_{B\in \cal P}\Gap(M_{\st}|_B).
\end{align}
The second term $\Gap(M_{\st}|_B)$ is related to $(\Gap(M_{i}|_A))$. We now lower-bound $\Gap(\ol M_{\st}) $. We will abuse notation by considering the sets $B\in \cal P$ as states in $\ol M_{\st}$, and identify a union of sets in $\cal P$ with the corresponding set of states for $\ol{M}_{\st}$. 

We bound $\Gap(\ol M_{\st})$ by bounding the conductance of $\ol M_{\st}$ using Cheeger's inequality (Theorem~\ref{thm:cheeger}). 

Suppose first that $S\subeq \cal P$, $\Om\times \{1\}\nin S$. Intuitively, this means the ``highest temperature'' is not in the set. We will go from top to bottom until we find a partition that is in the set, the interaction of this part and the temperature above it will already provide a large enough cut.

Let $i$ be minimal such that $A\times \{i\}\in S$ for some $A$. There is %$\rc 8$ 
$\rc{2L}$ 
probability of proposing a switch to level $i-1$, so 
%\Hnote{$\rc{2L}$ instead of $\rc 8$ if proposing all $i\in [L]$ equally} 
\begin{align}
%P(S,S^c) \ge 
P_{\st}(A\times \{i\}, \Om\times \{i-1\})
&\ge %\rc{8}
\rc{2L} \frac{1}{r_ip_i(A)}
 \int_A r_ip_i(x) \min\bc{
\fc{p_{i-1}(x)}{p_i(x)}\fc{r_{i-1}}{r_i}, 1
}\dx\\
&=%\rc{8}
\rc{2L}\int_A  \min\bc{p_{i-1}(x)\fc{r_{i-1}}{r_i}, p_i(x)}\dx/p_i(A)\\
&\ge% \rc{8}
\rc{2L} \fc{\min r_j}{\max r_j} \int_A \min\bc{p_{i-1}(x), p_i(x)} \dx /  p_i(A)\\
&\ge %\rc{8}
\rc{2L} r\de.
\label{eq:path-bd2-0}
\end{align}

We have that (defining $Q$ as in Definition~\ref{df:conduct})
\begin{align}
\phi_{\st}(S) & =  \fc{\ol Q_{\st}(S,S^c)}{p(S)}\\
&\ge \fc{p_{\st}(A\times \{i\}) P_{\st}(A\times \{i\}, \Om\times \{i-1\})}{p(S)}\\
&\ge \fc{rp_{\min}}{L}  %\rc{8}
\rc{2L}
 r\de
=\fc{r^2\de p_{\min}}{2L^2}
%{8L}.
\end{align}

Now consider the case that $\Om\times \{1\}\in S$. This case the highest temperature is in the set. We then try to find the part with highest temperature that is not in the set. 

Note that $p_{\st}(\Om\times \{1\}) \ge \fc{r}L$. Define $A\times \{i\}$ as above for $S^c$, then
\begin{align}
\phi_{\st}(S)&
=\fc{\ol Q_{\st}(S,S^c)}{p(S)}\\
&\ge \fc{p_{\st}(\Om\times \{1\}) P_{\st}(A\times \{i\}, \Om\times \{i-1\})}{p(S)}\\
&\ge \fc{r}{L} %\rc 8
\rc{2L} r\de.
\end{align}
Thus by Cheeger's inequality~\ref{thm:cheeger},
\begin{align}
\Gap(\ol M_{\st}) &\ge \fc{\Phi(\ol M_{\st})^2}{2}
=\fc{r^4\de^2p_{\min}^2}{8L^4}
%{128L^2}.
\label{eq:gap-proj}
\end{align}

Therefore we have proved the projected Markov chain (between partitions) has good spectral gap. What's left is to prove that inside each partition the Markov chain 
has good spectral gap, note that
\begin{align}
\Gap(M_{\st}|_{B\times \{i\}}) & \ge
\rc 2\Gap(M_i|_A)
\label{eq:gap-rest}
\end{align}
because the chain $M_{\st}$, on a state in $\Om\times \{i\}$, transitions according to $M_i$ with probability $\rc2$. 
Plugging~\eqref{eq:gap-proj} and~\eqref{eq:gap-rest} into~\eqref{eq:st-gap-prod0} gives the bound.
\end{proof}

\begin{rem}
Suppose in the type 2 transition we instead pick $k'$ as follows: With probability $\rc 2$, let $k'=k-1$, and with probability $\rc2$, let $k'=k+1$. If $k'\nin [L]$, let $k'=k$ instead.

Then the $\rc{2L}$ becomes $\rc4$ in the proof above so we get the improved gap
\begin{align}
\Gap(M_{\st}) &\ge\fc{r^4\de^2p_{\min}^2}{128L^2}
\min_{1\le i\le L, A\in \cal P_i} (\Gap(M_{i}|_A)).
\end{align}
\end{rem}




%\begin{thm}
%Let $M_i, i\in \{0,\ldots, L-1\}$ be a sequence of Markov chains with state space $\Om$ and stationary distributions $\mu_i$.
%Consider the simulated tempering chain $(\Om\times\{0,\ldots, L-1\}, P_{\st})$ with probabilities $(r_i)_{i=0}^{L-1}$. Let $r = \fc{\min(r_i)}{\max(r_i)}$.  
%
%Let $\cal P=\set{A_j}{1\le j\le n}$ be a partition\footnote{We allow overlaps on sets of measure 0.} of the ground set $\Om$. Define a parameter $\ga$ for the tempering chain by
%$$
%\ga(\cal P) =\min_{0\le i_1\le i_2\le L-1}\min_j \fc{\mu_{{i_1}}(A_j)}{\mu_{{i_2}}(A_j)}
%$$
%and the \emph{overlap} parameter $(\cal P_i)_{i=0}^{L-1}$ to be 
%$$
%\de(\cal P) = \min_{1\le i\le L-1, A\in \cal P} \ba{
%\int_{A} \min\{\pi_{i-1}(x),\pi_i(x)\} \dx 
%}/\pi_i(A).
%$$
%Then, the spectral gap of the tempering chain satisfies:
%\begin{align}
%\Gap(P_{\st}) &\ge\fc{r^2\ga \de}{32L^3} \min\bc{
%\min_{0\le i\le L-1, A\in \cal P} (\Gap(P_{i}|_A)), \Gap(P_0)}.
%\end{align} 
%\end{thm} 
%
%In fact, we will prove an even more general theorem \Anote{Do we want to keep it this general? We're not using it anywhere}. 
%Say that partition $\cal P$ refines $\cal Q$, written $P\sqsubseteq \cal Q$, if for every $A\in \cal P$ there exists $B\in \cal Q$ such that $A\subeq B$. 
%Define a chain of partitions as $\{\cal P_i = \{A_{i,j}\}\}_{i=0}^{L-1}$, where each $\cal P_i$ is a refinement of $\cal P_{i-1}$:
%$$
%\cal P_{l-1} \sqsubseteq \cdots \sqsubseteq \cal P_0.
%$$
%
%Define $\ga$ for a chain of partitions as 
%$$
%\ga((\cal P_i)_{i=0}^{L-1}) =\min_{0\le i_1\le i_2\le L-1}\min_{A\in \cal P_{i_1}}
%\fc{\mu_{{i_1}}(A)}{\mu_{{i_2}}(A)}.
%$$
%
%Define the overlap of measures with respect to $(\cal P_i)_{i=0}^{L-1}$ to be 
%$$
%\de((\cal P_i)_{i=0}^{L-1}) = \min_{1\le i\le L-1, A\in \cal P_i} \ba{
%\int_{A} \min\{\pi_{i-1}(x),\pi_i(x)\} \dx 
%}/\pi_i(A).
%$$
%
%\begin{thm}\label{thm:sim-temp}
%Let $M_i=(\Om, P_i), i\in \{0,\ldots, L-1\}$ be a sequence of reversible Markov chains with stationary distributions $\mu_i$. 
%
%Consider the simulated tempering chain $(\Om\times\{0,\ldots, L-1\}, P_{\st})$ with probabilities $(r_i)_{i=0}^{L-1}$. Let $r = \fc{\min(r_i)}{\max(r_i)}$.  
%
%Let $(\cal P_i)_{i=0}^{L-1}$ be a chain of partitions such that $\cal P_0=\{\Om\}$.
%
%Let $\ga = \ga ((\cal P_i)_{i=0}^{L-1})$ and $\de = \de((\cal P_i)_{i=0}^{L-1})$. 
%Then 
%\begin{align}
%\Gap(P_{\st}) &\ge\fc{r^2\ga \de}{32L^3} \min_{0\le i\le L-1, A\in \cal P_i} (\Gap(P_{i}|_A)).
%\end{align}
%\end{thm}
%\begin{proof}
%Let $\mu_{\st}$ be the stationary distribution of $P_{\st}$.
%First note that we can easily switch between $\mu_i$ and $\mu_{\st}$ using $\mu_{\st}(A\times \{i\}) = r_i \mu_i(A)$. 
%
%Define the partition $\cal P$ on $\Om\times \{0,\ldots, l-1\}$ by 
%$$
%\cal P = \set{A\times \{i\}}{A\in \cal P_i}.
%$$
%
%By Theorem~\ref{thm:gap-product},
%\begin{align}\label{eq:st-gap-prod}
%\Gap(P_{\st}) & \ge \rc2 \Gap(\ol P_{\st}) 
%\min_{B\in \cal P}\Gap(P_{\st}|_B).
%\end{align}
%We now lower-bound $\Gap(\ol P_{\st}) $. We will abuse notation by considering the sets $B\in \cal P$ as states in $\ol P_{\st}$, and identify a union of sets in $\cal P$ with the corresponding set of states for $\ol{P}_{\st}$. 
%
%Consider a tree with nodes $B\in \cal P$, and edges connecting $A\times\{i\}$, $A'\times\{i-1\}$ if $A\in A'$. Designate $\Om\times \{0\}$ as the root. For $X,Y\in \cal P$, define the canonical path $\ga_{X,Y}$ to be the unique path in this tree.
%
%Note that $|\ga_{X,Y}|\le 2(L-1)$. Given an edge $(A\times\{i\},A'\times\{i-1\})$, consider
%\begin{align}\label{eq:st-path}
%\fc{\sum_{\ga_{X,Y}\ni(A\times\{i\},A'\times\{i-1\})} |\ga_{X,Y}|\mu(X)\mu(Y)}{\mu(A\times\{i\})P(A\times\{i\},A'\times\{i-1\})}
%&\le \fc{2(l-1) 2\mu(S)\mu(S^c)}{\mu(A\times\{i\})P(A\times\{i\},A'\times\{i-1\})}
%\end{align}
%where $S= A\times \{i,\ldots, l_1\}$ is the union of all children of $A\times \{i\}$ (including itself). This follows because the paths which go through $(A\times\{i\},A'\times\{i-1\})$ are exactly those between $X,Y$ where one of $X,Y$ is a subset of $S= A\times \{i,\ldots, l_1\}$ and the other is not. To upper bound~\eqref{eq:st-path}, we upper-bound $\fc{\mu(S)}{\mu(A\times \{i\})}$ and lower-bound $P(A\times \{i\}, A'\times \{i+1\})$. 
%
%We upper-bound by definition of $\ga$
%\begin{align}
%\fc{\mu(S)}{\mu(A\times \{i\})} &= \fc{\sum_{k=i}^{L-1} r_i\mu_i(A)}{r_i\mu_i(A)}\\
%&\le\fc{\max r_i}{\min r_i}
% \fc{\sum_{k=i}^{L-1} \mu_i(A)}{\mu_i(A)}\\
%&\le \fc L{r\ga}.
%\label{eq:path-bd1}
%\end{align}
%
%%We claim that 
%%$$
%%\mu(S)\le \fc{L}{\ga}\mu(A\times \{i\}).
%%$$
%%We prove the following by induction on $j-i$: for $j>i$, given $A_i\in \cal P_i$, $A_j\in \cal P_j$, %there exists some chain $(A_i,\ldots, A_j)$ such that 
%%\begin{align}\label{eq:chain-ratio}
%%\min_{\text{chain }(A_i,\ldots, A_j)}
%%\prod_{k=i+1}^j \min\bc{\fc{\mu_{{k-1}}(A_k)}{\mu_{k}(A_k)},1}
%%\le \fc{\mu_i(A_i)}{\mu_j(A_i)}.
%%\end{align}
%%We induct on $j-i$.
%%For $j=i$ this is trivial. For $j>i$, by the induction hypothesis on $(i+1,j)$,
%%\begin{align}
%%\fc{\mu_i(A_i)}{\mu_j(A_i)} 
%%&\ge
%%\min_{A_{i+1}\subeq A_i, A_{i+1}\in \cal P_{i+1}}
%%\fc{\mu_i (A_{i+1})}{\mu_j(A_{i+1})}\\
%%%\fc{\mu_i(A_i)}{\mu_{i+1}(A_i)}\fc{\mu_i(A_i)}{\mu_j(A_i)}\\
%%%& \ge 
%%%\min \fc{\mu_{i 
%%&\ge 
%%\fc{\mu_i (A_{i+1})}{\mu_{i+1}(A_{i+1})}
%%\fc{\mu_{i+1} (A_{i+1})}{\mu_j(A_{i+1})}\\
%%&\ge 
%%\fc{\mu_i (A_{i+1})}{\mu_{i+1}(A_{i+1})}
%%\min_{\text{chain }(A_{i+1},\ldots, A_j)}
%%\prod_{k=i+2}^j \min\bc{\fc{\mu_{{k-1}}(A_k)}{\mu_{k}(A_k)},1}\\
%%&\ge \min_{\text{chain }(A_i,\ldots, A_j)}
%%\prod_{k=i+1}^j \min\bc{\fc{\mu_{{k-1}}(A_k)}{\mu_{k}(A_k)},1}
%%\end{align}
%%where in the last step we add $A_i$ to the chain $(A_{i+1},\ldots, A_j)$. 
%%
%%From~\eqref{eq:chain-ratio} and definition of $\ga$, we get $\fc{\mu_i(A_i)}{\mu_j(A_i)}\ge \ga$. Thus,
%%\begin{align}
%%\fc{\mu(A\times \{i,\ldots, L-1\})}{\mu(A\times \{i\})}
%%&
%%= \fc{\sum_{k=i}^{L-1} r_i\mu_i(A)}{r_i\mu_i(A)}\\
%%&\le\fc{\max r_i}{\min r_i}
%% \fc{\sum_{k=i}^{l-1} \mu_i(A)}{\mu_i(A)}\\
%%&\le \fc L{r\ga}.
%%\label{eq:path-bd1}
%%\end{align}
%
%Next we lower bound $P(A\times \{i\}, A'\times \{i-1\})$.  There is probability $\rc{2L}$ of proposing a switch to level $i-1$, so
%\begin{align}
%P(A\times \{i\}, A'\times \{i-1\})
%&\ge \rc{2L} \int_\Om r_i\mu_i(x) \min\bc{
%\fc{\mu_{i-1}(x)}{\mu_i(x)}\fc{r_{i-1}}{r_i}, 1
%}\dx/(r_i\mu_i(A))\\
%&=\rc{2L}\int_\Om  \min\bc{\mu_{i-1}(x)\fc{r_{i-1}}{r_i}, \mu_i(x)}\dx/\mu_i(A)\\
%&\ge \rc{2L} \fc{\min r_j}{\max r_j} \int_\Om \min\bc{\mu_{i-1}(x), \mu_i(x)} \dx /  \mu_i(A)\\
%&\ge \rc{2L} r\de.
%\label{eq:path-bd2}
%\end{align}
%Putting~\eqref{eq:st-path},~\eqref{eq:path-bd1}, and~\eqref{eq:path-bd2} together, 
%\begin{align}
%\eqref{eq:st-path}&\le 2(l-1)2\pf{l}{r\ga} \pf{2L}{r\de}\\
%&\le \fc{8L^3}{r^2\ga \de}.
%\end{align}
%Using~\eqref{eq:st-gap-prod} and Theorem~\ref{thm:can-path}, 
%\begin{align}
%\Gap(P_{\st}) &\ge \rc 2 \Gap(\ol P_{\st})\min_{B\in \cal P} \Gap(P_{\st}|_B)\\
%&\ge \fc{r^2\ga \de}{16L^3} \min_{B\in \cal P} \Gap(P_{\st}|_B)\\
%&\ge \fc{r^2\ga \de}{32L^3} \min_{0\le i\le L-1, A\in \cal P_i} \Gap(P_{i}|_A)
%\end{align}
%\end{proof}
%\begin{cor}
%Keep the setting of Theorem~\ref{thm:sim-temp}, except fix the partition $\cal P$. Then 
%\begin{align}
%\Gap(P_{\st}) &\ge\fc{r^2\ga \de}{32L^3} \min\bc{
%\min_{0\le i\le L-1, A\in \cal P} (\Gap(P_{i}|_A)), \Gap(P_0)}.
%\end{align}
%\end{cor}
%\begin{proof}
%Take $(\cal P_i')_{i=0}^{L-1}$ to be the chain with $\cal P_0=\{\Om\}$ and $\cal P_1=\cdots =\cal P_{L-1}=\cal P$ and use Theorem~\ref{thm:sim-temp}.
%\end{proof}
