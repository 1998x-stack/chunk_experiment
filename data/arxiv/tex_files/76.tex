%\RequirePackage[switch,columnwise]{lineno}
\documentclass[rmp,twocolumn,floatfix,superscriptaddress,showpacs]{revtex4-1}
%\documentclass[aps,rmp,amsmath,amssymb,graphicx,longbibliography]{revtex4-1}
%\documentclass[aps,rmp,preprint,amsmath,amssymb,graphicx,longbibliography]{revtex4-1}
%\documentclass[aps,rmp,reprint,amsmath,amssymb,graphicx,longbibliography]{revtex4-1}
%\documentclass[prl,twocolumn,floatfix,superscriptaddress,showpacs]{revtex4}
%\documentclass[prl,preprint,floatfix,superscriptaddress,showpacs]{revtex4-1}
%\documentclass[aps,prd,twocolumn,showpacs,preprintnumbers,amsmath,amssymb,floatfix]{revtex4}
%\documentclass[showpacs,preprintnumbers,amsmath,amssymb,twocolumn]{revtex4-1}
%\documentclass[prd,onecolumn,floatfix,superscriptaddress,showpacs,preprintnumbers,nofootinbib]{revtex4}

\usepackage{lineno}

\usepackage{fancyhdr}


\usepackage{graphicx,dcolumn,bm}
\usepackage{rotating}
\usepackage{amsmath}
\usepackage{amssymb}
\usepackage{array}
\usepackage{wrapfig}
\usepackage{units}
\usepackage{color}
\usepackage{cancel}
\usepackage{enumerate}
%\usepackage[utf8]{inputenc}
%\usepackage{calc}
%\usepackage{layout}
%\usepackage{epsfig}
%\usepackage{setspace}
%\usepackage{palatino}
%\usepackage{booktabs}
%\usepackage{rotating}
%\usepackage{multirow}
%\usepackage{verbatim}
%%\usepackage{hyperref}
%\usepackage{datetime}



% Some other (several out of many) possibilities
%\documentclass[preprint,aps]{revtex4}
%\documentclass[preprint,aps,draft]{revtex4}
%\documentclass[prb]{revtex4}% Physical Review B

%\usepackage{enumitem} 
%\usepackage{dcolumn}% Align table columns on decimal point
%\usepackage{bm}% bold math
%\usepackage[dvips]{graphicx}
%\usepackage{amsmath}
%\usepackage{epsfig}
%\usepackage{amsfonts}
%\usepackage{amssymb}
%\nofiles


\newcommand{\be}{\begin{equation}}
\newcommand{\ee}{\end{equation}}
\newcommand{\bee}{\begin{equation*}}
\newcommand{\eee}{\end{equation*}}
\newcommand{\bea}{\begin{eqnarray}}
\newcommand{\eea}{\end{eqnarray}}
\newcommand{\bean}{\begin{eqnarray*}}
\newcommand{\eean}{\end{eqnarray*}}

\newcommand{\nn}{\nonumber}
\newcommand{\markx}{$\clubsuit$}


\newcommand{\lp}{\left(}
\newcommand{\rp}{\right)}

\usepackage{color}
\newcommand{\red}[1]{\textcolor{red}{#1}}
\newcommand{\magenta}[1]{\textcolor{magenta}{#1}}
\newcommand{\blue}[1]{\textcolor{blue}{#1}}

\renewcommand{\labelitemi}{}


\newcommand{\ba}{\begin{eqnarray}}
\newcommand{\ea}{\end{eqnarray}}
\newcommand{\s}{\scriptscriptstyle}
%\newcommand{\Tr}{\mathrm{Tr}}
\def\sinhat{\hat{s}^2}
\def\coshat{\hat{c}^2}
\def\sinzero{\sin^2\hat\theta_W (0)}
\def\alphat{\hat\alpha}
\def\qwp{{Q_W(p)}}
\def\qwe{{Q_W(e)}}
\def\qwcs{{Q_W({\rm Cs})}}
\def\qwtl{{Q_W({\rm Tl})}}
\def\sstw{{\sin^2\theta_W}}
\newcommand{\ket}[1]{\left\lvert #1\right\rangle}
\newcommand{\bra}[1]{\left\langle #1\right\rvert}
\newcommand{\vect}[1]{\mathbf{#1}}
\newcommand{\op}[1]{\textsf{#1}}
\newcommand{\abs}[1]{\left\lvert #1\right\rvert}
\newcommand{\boldgamma}{\mbox{\boldmath$\gamma$}}
\newcommand{\diracslash}[1]{\not\!\! #1}
\newcommand{\pd}[2]{\frac{\partial #1}{\partial #2}}
\newcommand{\mcdot}{\!\cdot\!}
\newcommand{\CPV}{CP\!\!\!\!\!\!\!\!\raisebox{0pt}{\small$\diagup$}}

%%%%%%%%%%%%%%%%%%%%%%%%%%%%%%%%%%%%%%%%%%%%%%%%%%%%%%
%  New Commands needed for operator table
%%%%%%%%%%%%%%%%%%%%%%%%%%%%%%%%%%%%%%%%%%%%%%%%%%%%%%
%\def\theequation{\thesection.\arabic{equation}}
%\newcommand{\newsection}[1]{\section{#1}\setcounter{equation}{0}}
%\newcommand{\newappendix}[1]{\section*{#1}\setcounter{equation}{0}}
%
%\def\lb{\linebreak}
%\def\nnb{\nonumber}
%
%\newcommand{\scm}{\scriptscriptstyle\mathrm}
%\newcommand{\f}{\frac}
%
\newcommand{\baa}{\begin{array}}      
\newcommand{\eaa}{\end{array}}
\newcommand{\bit}{\begin{itemize}}    
\newcommand{\eit}{\end{itemize}}
\newcommand{\ben}{\begin{enumerate}}  
\newcommand{\een}{\end{enumerate}}
\newcommand{\bce}{\begin{center}}     
\newcommand{\ece}{\end{center}}
\newcommand{\bfl}{\begin{flushright}} 
\newcommand{\efl}{\end{flushright}}
\newcommand{\btb}{\begin{tabular}}    
\newcommand{\etb}{\end{tabular}}
%
\newcommand{\eps}{\varepsilon}
\newcommand{\vp}{\varphi}
\newcommand{\tvp}{\widetilde{\varphi}}
\newcommand{\D}{\mbox{$\not\!\!D$}}
\newcommand{\Db}{\mbox{${\raisebox{2mm}{\boldmath ${}^\leftarrow$}\hspace{-4mm} D}$}}
\newcommand{\Dfb}{\mbox{$\raisebox{2mm}{\boldmath ${}^\leftrightarrow$}\hspace{-4mm} D$}}
\newcommand{\vpj }{\mbox{${\vp^\dag i\,\raisebox{2mm}{\boldmath ${}^\leftrightarrow$}\hspace{-4mm} D_\mu\,\vp}$}}
\newcommand{\vpjt}{\mbox{${\vp^\dag i\,\raisebox{2mm}{\boldmath ${}^\leftrightarrow$}\hspace{-4mm} D_\mu^{\,I}\,\vp}$}}
%
\def\ocal{{\cal O}}
\def\lcal{{\cal L}}
\def\p{\partial}
\def\wt{\widetilde}
\def\gpbz{{\bar g}_\pi^{(0)}}
\def\gpiz{{\bar g}_\pi^{(0)}}
\def\gpio{{\bar g}_\pi^{(1)}}
\def\gpbo{{\bar g}_\pi^{(1)}}
\def\gpbt{{\bar g}_\pi^{(2)}}
\def\gpbi{{\bar g}_\pi^{(i)}}

\def\gebz{{\bar g}_\eta^{(0)}}
\def\gebo{{\bar g}_\eta^{(1)}}
\def\gebt{{\bar g}_\eta^{(2)}}
\def\gebi{{\bar g}_\eta^{(i)}}

\def\grbz{{\bar g}_\rho^{(0)}}
\def\grbo{{\bar g}_\rho^{(1)}}
\def\grbt{{\bar g}_\rho^{(2)}}
\def\grbi{{\bar g}_\rho^{(i)}}

\def\gobz{{\bar g}_\omega^{(0)}}
\def\gobo{{\bar g}_\omega^{(1)}}
\def\gobt{{\bar g}_\omega^{(2)}}
\def\gobi{{\bar g}_\omega^{(i)}}

%%%%%%%%%%%%%%%%%%%%%%%%%%%%%%%%%%%%%%%%%%%%%%%%%%%%%%

%\DeclareMathOperator{\Real}{Re}
%\DeclareMathOperator{\Imag}{Im}
%\DeclareMathOperator{\Tr}{Tr}

%\newcommand\slurp[1]{#1}
%\newcommand\sslurp[2]{#1{#2}}
%{\catcode`/=\active \expandafter}%
%\slurp{\newcommand/}{/\penalty1000\hskip0pt\relax}
%{\catcode`:=\active \expandafter}%
%\slurp{\newcommand:}{:\penalty1000\hskip0pt\relax}


%\newcommand\addspace{\ifcat\nextchar a\spacefactor999. \else.\fi}
%{\catcode`\.=\active \expandafter}%
%\slurp{\newcommand.}{\futurelet\nextchar\addspace}


%\usepackage[unicode]{hyperref} 
%\newcommand\url[1]{\href{#1}{#1}}
%\ifx\href\undefined\def\href#1{}\fi
%\ifx\texorpdfstring\undefined\def\texorpdfstring#1#2{#1}\fi

%\newcommand\myslash{/} \newcommand\mycolon{:}
%\newcommand\doi{{\catcode`/=\active \catcode`:=\active \expandafter}\sslurp\realdoi}
%{\catcode`/=\active \catcode`:=\active \expandafter}%
%\slurp{\newcommand\realdoi[1]{{\let/=\myslash \let:=\mycolon
 %                              \edef\raw{{http://dx.doi.org/#1}}\expandafter}%
 %                              \expandafter\href\raw{doi:#1}}}
%\newcommand\Lbsm{\Lambda_\mathrm{BSM}}
%\newcommand{\vier}{\\ [4 pt]}
%\newcommand{\acht}{\\ [8 pt]}


\newcommand{\ecm}{\,{\it e}~{\rm cm}}
% \pagestyle{fancy}
%\lhead{ACFI-T17-18}

\begin{document}



\title{Electric Dipole Moments of Atoms, Molecules, Nuclei and Particles}

 \author{T.E. Chupp}
 \affiliation{Department of Physics, University of Michigan, Ann Arbor, Michigan 48109 USA}
 \author{P. Fierlinger}
 \affiliation{Physik Department and Excellence-Cluster ``Universe'', Technische Universit{\" a}t M{\" u}nchen, 85748 Garching bei M{\" u}nchen, Germany}
 \author{M.J. Ramsey-Musolf}
 \affiliation{Amherst Center for Fundamental Interactions and Department of Physics, University of Massachusetts-Amherst, Amherst, Massachusetts 01003 USA}
 \author{J.T. Singh}
 \affiliation{National Superconducting Cyclotron Laboratory and Department of Physics and Astronomy, Michigan State University
East Lansing, Michigan 48824 USA}

\pacs{07.55.Ge, 07.55.Nk,2.10.Dk, 3.15.Kr, 11.30.Er, 12.15.Ji, 21.10.Ky, 29.25.Dz}


\begin{abstract}
A permanent electric dipole moment (EDM) of a particle or system is a separation of charge along its angular-momentum axis and is a direct signal of T-violation and, assuming CPT symmetry, CP violation. For over sixty years EDMs have been studied, first as a signal of a parity-symmetry violation and then as a signal of CP violation that would clarify its role in nature and in theory. Contemporary motivations include the role that CP violation plays in explaining the cosmological matter-antimatter asymmetry and the search for new physics. Experiments on a variety of systems have become ever-more sensitive, but provide only upper limits on EDMs, and theory at several scales is crucial to interpret these limits. Nuclear theory provides connections from Standard-Model and Beyond-Standard-Model physics to the observable EDMs, and atomic and molecular theory reveal how CP-violation is manifest in these systems. EDM results in hadronic systems require that the Standard Model QCD parameter of $\bar\theta$ must be exceptionally small, which could be explained by the existence of axions - also a candidate dark-matter particle. Theoretical results on electroweak baryogenesis show that new physics is needed to explain the dominance of matter in the universe. Experimental and theoretical efforts continue to expand with new ideas and new questions, and this review provides a broad overview of theoretical motivations and interpretations as well as details about experimental techniques, experiments, and prospects. The intent is to provide specifics and context as this exciting field moves forward.
\end{abstract}

\hskip -3.8 truein 
ACFI-T17-18
\hskip 3.3 truein
To be published in Reviews of Modern Physics



%\linenumbers


 \maketitle
  
 \date{\today.  To be submitted to RMP}
 \tableofcontents


\section{Introduction}
\label{sec:Introduction}

The measurement and interpretation of permanent Electric Dipole Moments or EDMs of  particles and quantum systems have been a unique window into the nature of elementary particle interactions from the very first proposal of~\textcite{rf:Purcell1950} to search for a neutron EDM as a signal of parity-symmetry (P) violation.  The neutron EDM was not observed then or since, and we now recognize, as pointed out by~\textcite{LY57a} and \textcite{Lan57c,Lan57b-ru,Lan57b}, that EDMs  also violate time-reversal symmetry (T).  An EDM is  a direct signal of T-violation,
%and not subject, for example, to T-allowed final-state-interaction phase shifts, 
and CPT symmetry (C is charge conjugation), required of any relativistic field theory~\cite{Tureanu:2013psa} therefore implies that observation of a non-zero P-odd/T-odd EDM is also a signal of CP violation. EDMs have become a major focus of contemporary research for several interconnected reasons: 
\begin{enumerate}[i.]
\item EDMs provide a direct experimental probe of CP violation, a feature of the Standard Model (SM) and Beyond-Standard-Model (BSM) physics;
% that introduces new phases into amplitudes for new interactions, 
\item the P-violating and T-violating EDM signal distinguishes the much weaker CP-violating interactions from the dominant strong and electromagnetic interactions; 
\item  CP violation is a required component of Sakharov's recipe~\cite{SakharovBaryogenesis-ru,SakharovBaryogenesis} for the baryon asymmetry, the fact that there is more matter than antimatter in the universe; however SM-CP violation cannot produce the observed asymmetry, and new CP-violating interactions are required.%~\cite{Morrissey:2012db}
%sufficient for baryogenesis at the electroweak phase transition
\end{enumerate}

The EDM of a  system $\vec d$ must be parallel (or antiparallel) to the average angular momentum of the system $\hbar\langle\vec J\rangle$ . Thus, relative to the center of mass ($\vec r=0$):
\begin{equation}
\label{eq:EDMdef}
\vec d =\int\ \vec r\rho_Q d^3r = d \frac{\langle\vec J\rangle}{J}, 
\end{equation}
where $\rho_Q$ is the electric-charge distribution. 
This is analagous to the magnetic dipole moment~\cite{RamseyMolBeamsBookp77}
\begin{equation}
\label{eq:magmomdef}
\vec\mu =\frac{1}{2} \int\ \vec r\times\!\vec J_Q d^3r = \mu  \frac{\langle\vec J\rangle}{J},
\end{equation}
where $\vec J_Q$ is the current density. For a neutral system, {\it e.g.} the neutron, $\vec d$ can be considered a separation of equal charges; however for a  system with charge $Q\ne 0$, $\vec d/Q=\vec r_Q$ is the center of charge with respect to the  center of mass.

The interaction of a fermion with magnetic moment $\mu$ and EDM $d$ with  electric and magnetic fields 
%shown in Figure~\ref{fg:MDMEDM} 
can be written
\begin{equation}
{\cal L_{EM}}=-\frac{\mu}{2}\bar\Psi \sigma^{\mu\nu}F_{\mu\nu}\Psi -i\frac{d}{2}\bar\Psi \sigma^{\mu\nu}\gamma^5 F_{\mu\nu}\Psi,
\label{eq:EMLagrangian}
\end{equation}
%(Alternatively the second term can be written in terms of the dual EM-field tensor $d\F^{\mu\nu}\tilde F_{\mu\nu}}$.)
where $\Psi$ is the fermion field, and  $F_{\mu\nu}=\partial_\mu A_\nu-\partial_\nu A_\mu$ is the electromagnetic field tensor with $A_\mu$ the four-vector electromagnetic potential.
The second term of Eqn.~\ref{eq:EMLagrangian}, first written down in this way by~\textcite{rf:Salpeter1958}\footnote{In his paper ``The Quantum Theory of the Electron'', \textcite{Dirac} revealed the purely imaginary coupling of the electric field to an electric moment. This was considered, at the time, unphysical because it was derived from a real Hamiltonian.} in analogy to the anomalous magnetic moment term, reveals parity and time-reversal violation in the Dirac matrix $\gamma^5$ and the imaginary number $i$, respectively. The corresponding non-relativistic Hamiltonian for a quantum system is
\begin{equation}
H=-(\vec\mu\cdot \vec B+\vec d\cdot\vec E) = -(\mu\vec J\!\cdot\!\vec B+d\vec J\!\cdot\!\vec E)/J.
\label{eq:EDMHamiltonian}
\end{equation}
The magnetic field $\vec B$ and the angular momentum operator $\vec J$ are both even under P but odd under T, while the electric field $\vec E$ is odd under P but even under T. The second term, proportional to $\vec J\cdot\!\vec E$,
%The second term, proportional to  $d\vec J\cdot\vec E$, is the combination of a T-odd axial vector and a T-even polar vector, {\it i.e.} it 
is thus P-odd and T-odd and a direct signal of CP violation assuming CPT invariance.



A common approach to measuring an EDM is to apply a strong electric field and a very well controlled and characterized magnetic field and to measure the shift in the energy, or more  commonly the frequency, of the splitting between magnetic sub-levels when $\vec E$  is changed. For a system with total angular momentum $\hbar J$,  the EDM frequency shift for two adjacent levels ($|\Delta m_J|=1$) is
%\red{MJRM, Peter, JTS: please check}
\begin{equation}
%|\Delta f|=\frac{|dE|}{2\pi\hbar J}|\Delta m_J|.
|\Delta \omega|=\frac{|dE|}{\hbar J}.
\label{eq:EDMFreqEquation1}
\end{equation}
The precision of a single frequency measurement depends on the interrogation time $\tau$ and the signal-to-noise ratio (SNR) for the measurement. The SNR depends on the specifics of the %echnique, and for a counting experiment is ideally proportional to the $\sqrt{N}$ for $N$ particles counted or interrogated, {\it i.e.}  $\sigma_f=\frac{1}{2\pi\tau\sqrt{N}}.$ For a phase-noise limited experiment, for example with SQUID mangetometers the SNR generally depends on the bandwidth of the measurement $B=1/\sqrt{\tau}$, {\it i.e.} for a signal amplitude $V_0$ (volts) and noise density $v_n$ (volts/$\sqrt{\rm Hz}$),   the ideal frequency precision for a constant (non-decaying) signal is given by $\sigma_f\approx\frac{\sqrt{12}v_n}{2\pi V_0}\tau^{-3/2}$ 
technique.
% and is ultimately limited by the particle ``shot noise'' which is given by $\sqrt{N}$ where $N$ is the number of particles being interrogated. 
For a count-rate-limited experiment with $N$ particles measured or interrogated in a single measurement SNR\,$\propto$\, $\sqrt{N}$, and the statistical uncertainty of a single frequency measurement is given by $\sigma_\omega = 1/( \tau \sqrt{N})$. 
In phase-noise-limited experiments, for example those using a SQUID magnetometer, the statistical uncertainty of a single frequency measurement for constant signal and SNR is given 
by $\sigma_\omega = \sqrt{12}/( \tau [{\rm SNR}])$~\cite{rf:ChuppMaser1,clps}, where SNR generally increases as $\sqrt{\tau}$.
The EDM sensitivity for a {\bf pair} of frequency measurements with opposite electric field each lasting a time $\tau$ therefore scales as
\begin{eqnarray}
\sigma_d&\gtrsim&\frac{\hbar J}{E}\frac{1}{\sqrt{2N}}\tau^{-1}\quad\quad\ \ {\rm (counting)},\nonumber\\
\sigma_d&\gtrsim&\frac{\hbar J}{E}\sqrt{\frac{3}{\pi}}\frac{v_n}{V_0}\tau^{-3/2}\quad {\rm (phase\ noise)}.
%\sigma_d\approx {1\over 2 E J}{h\over \tau SNR }
\label{eq:EDMSigmaEquation1}
\end{eqnarray}
Here $V_0$ is the signal size and $v_n \sqrt{B}$ is the noise in a bandwidth $B=1/(2\pi\tau)$, with $B$ in Hz.

The experimental challenges are to have the largest possible electric field magnitude $E$, the longest possible $\tau$ and the highest possible  $N$ or $V_0/v_n$. Additionally an ideally small, stable, and well characterized applied magnetic field is required to suppress frequency fluctuations due to changes in the magnetic moment interaction $\vec\mu\cdot\vec B$. Experimenters also strive to find systems in which the EDM is in some way enhanced, basically due to a large intrinsic (P-even, T-even) electric dipole moment and/or increased electric polarizability of the system, which is the case for a molecule or an atomic nucleus with octupole collectivity.
To date all EDM searches (see Table~\ref{tb:EDMResults}), including the neutron, atoms (Cs, Tl, Xe, Hg, and Ra) and molecules (TlF, YbF, ThO, and HfF$^+$),
%neutron ($d_n$) \cite{Baker:2006ts} atoms (Cs~\cite{Murthy:1989zz}, Xe~\cite{rf:Rosenberry2001}, Tl~\cite{Regan:2002ta}, Hg\cite{Graner:2016ses}) and molecules (TlF~\cite{rf:Cho1991}, YbF~\cite{Hudson:2002az}, ThO~\cite{Baron:2013eja}) 
have results consistent with zero but also consistent with the Standard Model. 


%\begin{figure}[tb]
%\hskip -0.35 truein
%\includegraphics[width= 3.4 truein]{MDMEDM}
%\vskip -0.6 truein
%\caption{\label{fg:MDMEDM} Effective coupling of a neutron with magnetic and electric dipole moments to an external electromagnetic field. The cross represents the CP-violating vertex, which generates an EDM.}
%\end{figure}


Figure~\ref{fg:EDMSubwayMap} 
%illustrates the scope of this review, showing 
shows the connections from fundamental theory, including  Standard Model and Beyond-Standard-Model physics, through a series of theory levels at different energy scales to the experimentally accessible P-odd/T-odd observables in a variety of systems.
SM  CP violation arises from a complex phase in the Cabibbo-Kobayashi-Maskawa (CKM) matrix parameterizing the weak interaction~\cite{Kobayashi:1973fv} and in the gluon
 %CP violation $G{\widetilde G}$ operator in the Standard-Model
$G{\widetilde G}$ contribution to the  strong interaction, which is proportional to the   parameter ${\bar\theta}$~\cite{'tHooft:1976up,Jackiw:1976pf,Callan:1976je}. The CKM contribution to any observable EDM is many orders of magnitude smaller than current upper limits, providing a window of opportunity for discovering EDMs that arise from a non-zero $\bar\theta$ or  BSM physics.
 In contrast to CKM CP-violation, contributions of BSM physics need not be suppressed unless the CP-violating parameters themselves are small, or the mass scales are high. 
New BSM interactions are also required for baryogenesis to account for the cosmic matter-antimatter asymmetry.  EDMs provide a particularly important connection to baryogenesis if the CP-violation energy scale is not too high compared to the scale of electroweak symmetry-breaking, and if the responsible P-odd/T-odd interactions are flavor diagonal~\cite{Morrissey:2012db}.
 
 \begin{figure*}[tb]
\includegraphics[width = 7 truein]{EDMSubwayMap}%{TopDownSubwayMapBW}
\caption{\label{fg:EDMSubwayMap} (Color online) Illustration of the connections from a fundamental theory at a high energy scale to an EDM in a measurable low-energy system. The dashed boxes indicate levels dominated by theory, and the solid boxes identify systems that are the object of current and future experiments. The fundamental CP-violating Lagrangian at the top, a combination of SM and BSM physics, is reduced to the set of effective-field-theory Wilson coefficients that characterize interactions at the electroweak energy scale of $\approx 300$ GeV, the vacuum-expectation value of the Higgs. The  set of low-energy parameters defined in Sec.~\ref{sec:Theory} enter calculations that connect the electroweak-scale Wilson coefficients directly to electrons and nuclei. Finally atomic, molecular and condensed-matter structure calculations connect the low-energy parameters to the observables  in experimentally accessible systems.}
 \end{figure*}


This review is intended as a broad summary of how EDM experiment and theory have reached this point and how it will progress. To do so the motivations and impact of EDM measurements along with a context for interpreting the results in terms of a set of P-odd/T-odd low-energy parameters are presented in Sec.~\ref{sec:Theory}. State-of-the art experimental techniques and improvements that will drive progress are presented in some detail in Sec.~\ref{sec:Techniques}, followed by a review of the current status of all experiments and prospects for new and improved approaches. The interpretation of these experiments and the impact of improvements in the context of the low energy theory parameters are presented in Sec~\ref{sec:Interpretation}. The conclusions emphasize what will be necessary from both theory and experiment for continued progress and, perhaps, the discovery of an EDM.

We also draw attention to a long list of  important and classic reviews  that include or are fully devoted to EDMs and provide a number of different perspectives as well as the history of the motivations and context:~\textcite{Garwin1959},
 \textcite{rf:Shapiro1968,rf:Shapiro1968-ru}, \textcite{Sandars1975,Sandars1993},  \textcite{RamseyEDMReview,doi:10.1146/annurev.ns.0.120190.000245}, \textcite{BM89},  \textcite{Bernreuther:1990jx,Bernreuther:1990jx-erratum}, \textcite{1402-4896-1993-T46-012}, \textcite{rf:GolubLamoreaux},   \textcite{KL1997}, \textcite{COMMINS19991}, \textcite{doi:10.1080/00107510010027781}, \textcite{Ginges:2003qt}, \textcite{Pospelov:2005pr}, \textcite{doi:10.1143/JPSJ.76.111010}, \textcite{LDM-2009},  \textcite{CHUPP2010129}, \textcite{Dubbers:2011ns}, \textcite{Engel:2013lsa},  \textcite{doi:10.1146/annurev-nucl-102014-022331}, and \textcite{Yamanaka:2017mef}.



\subsection{Experimental landscape}
%\subsubsection*{The neutron}
The neutron was the objective of the early direct EDM measurements of Smith, Purcell and Ramsey due to the reasoning that it was a neutral hadronic (weakly interacting) system and would not be accelerated from the measurement region by a large static electric field~\cite{rf:Purcell1950,rf:Smith1957}. The early neutron-EDM measurements~\cite{Miller1967nEDM,rf:Baird1969,rf:Cohen1969} culminating in the measurement reported by~\textcite{rf:Dress1977}, used molecular-beam techniques developed to measure the neutron magnetic moment. This limited the observation time for a neutron transiting a meter-scale apparatus to milliseconds, resulting in linewidths of hundreds of Hz or more. The beam approach also had significant limitations due to a number of systematic effects, including the interaction of the neutron magnetic moment with the motional magnetic field ($\vec E\times\vec v/c^2$) and leakage currents, both of which shifted the energy when the electric field was changed. By 1980, advances in ultra-cold neutron (UCN) production at Leningrad Nuclear Physics Institute (LNPI)~\cite{Altarev1980,Altarev1981b} and at the Institute Laue-Langevin (ILL) in Grenoble, France~\cite{Smith1990}, enabled the first EDM measurements with neutrons stored in a ``cell.'' The much smaller UCN speeds and the nearly zero average velocity mitigated the motional effects, leading to a series of increasingly precise neutron EDM measurements, which are discussed in section~\ref{sec:NeutronEDM}. As the rate of UCN production improved, leading to smaller statistical errors, the control of the magnetic field required advances in magnetic shielding and magnetometry discussed in Sec~\ref{sec:Techniques}. Comagnetometry (the use of a second species  less sensitive to CP-violation but with a similar magnetic moment in the same measurement volume and at the same time) mitigated magnetic-field instability and a number of systematic effects. The neutron-EDM experiments are rate/statistics limited with typically only a few thousand UCN per measurement cycle, and advances require new UCN sources, which are discussed in detail in Sec~\ref{sec:UCNSources}, and corresponding improvements to magnetic shielding, magnetometry and understanding of systematic effects.

%\subsubsection*{Atomic and Molecular EDMs}


 

The earliest limits  on proton and electron EDMs were established by studies of corrections to the Lamb shift in hydrogen respectively by \textcite{Sternheimer1959} and by~\textcite{rf:Salpeter1958} and~\textcite{Feinberg1958}. Limits on the proton EDM were also set by analyzing the out-of-plane component of the proton spin polarization in an scattering asymmetry experiment  from a carbon target.~\cite{Rose:1960cla}, and  electron EDM limits were derived from frequency shifts in electron paramagnetic resonance
\cite{RB63},  anomalous magnetic-moment ($g-2$) measurements~\cite{Nelson1959,WC63}, and scattering measurements with spin-zero targets~\cite{PhysRev.114.1530} of helium~\cite{rf:Goldemberg1963,BURLESON196068,AVAKOV1959685} and carbon~\cite{PhysRev.140.B1605}. An early limit on the EDM of the muon was  derived from from analyzing the vertical component of the muon spin polarization the Nevis Cyclotron and fringe
fields, by the measuring asymmetry of the muon decay electrons.~\cite{BG60muon,PhysRevLett.1.144,PhysRevLett.1.144-err,Charpak1961}.

Starting in the 1960's, experimenters turned their attention to stable atoms and molecules beams in early beams experiments pioneered by~\textcite{rf:Sandars1964}.  It was recognized that these systems provided a rich set of possible contributions to the P-odd/T-odd observables, but the charged constituents, the electron and nucleus, are significantly shielded from the large external field by the polarization of the atom. This is embodied in Schiff's theorem~\cite{rf:Schiff1963}, which % , following Newton's second law, 
states that for a bound system of point-like charged particles the net force and the net electric field at the position of each charged particle are exactly zero. The shielding is not perfect for a nucleus of finite size and in the case of unpaired  electrons (paramagnetic systems) due to relativistic effects. %\cite{rf:Commins2007}. %~\cite{rf:SandarsZ2}.
In fact for paramagnetic atoms there is an effective enhancement of the sensitivity to an electron EDM that is approximately $10Z^3\alpha^2$ as explained by~\textcite{rf:Sandars1965,rf:Sandars1966,rf:Sandars1968I,rf:Sandars1968II},~\textcite{Ignatovich:1969tv,
Ignatovich:1969tv-ru}, and~\textcite{doi:10.1119/1.2710486}. Moreover, an atomic EDM can arise due to T and P violation in the electron-nucleus interaction that may have a scalar or tensor nature,  and these effects also increase with $Z$. 

Paramagnetic systems with one or more unpaired electrons (Cs, Tl, YbF, ThO, HfF$^+$ {\it etc.}) are most sensitive to both the electron EDM $d_e$ and the nuclear spin-independent component of the electron-nucleus coupling ($C_S$), which are likely to be several orders of magnitude stronger than tensor and pseudoscalar contributions, given comparable strength of the intrinsic couplings~\cite{Ginges:2003qt}. 
Diamagnetic systems, including $^{129}$Xe, $^{199}$Hg and $^{225}$Ra atoms, and the molecule TlF are most sensitive to purely hadronic CP-violating sources that couple through the Schiff moment $\vec S$, the $r^2$-weighted electric-dipole charge distribution for a nucleus with $Z$ protons,
\begin{equation}
\vec S =\frac{1}{10}\int\ r^2\vec r\rho_Q d^3r-\frac{1}{6Z} \int r^2d^3r \int\vec r\rho_Q d^3r.
%S=\frac{1}{10}\langle r_p^2\vec r_p\rangle-\frac{1}{6}\vec d_N\langle r_p^2\rangle\langle\vec r_p\rangle.
\label{eq:SchiffDef}
\end{equation}
The EDM of the nucleus $\int\vec r\rho_Q d^3r=\vec d_N$ is unobservable in a neutral atom and the second term is therefore subtracted from $\vec S$. \textcite{flambaum02} have shown that an effective model of the Schiff moment is a constant electric field with the nucleus directed along the nuclear spin, which is probed by the atomic or molecular electrons through the interaction
\begin{equation}
H=-4\pi\vec\nabla\rho_e(0)\cdot\vec S,
\end{equation}
where $\vec\nabla\rho_e(0)$ is the gradient of the electron density at the nucleus.
As the atomic electrons penetrate the nucleus, the Schiff-moment electric force moves the electron cloud with respect to the atom's center of mass and induces an EDM along the spin.\footnote{We note that Schiff's Theorem has been recently reevaluated in work showing that these formulas may be unjustified approximations~\cite{rf:Liu2007}; however, there is disagreement on the validity of this reformulation \cite{rf:Senkov2008}.
}
 In addition, the EDM of a diamagnetic atom or molecule can arise due to the tensor component of the electron-nucleus coupling $C_T$ for atoms and molecules. The electron EDM and $C_S$ contribute to the EDM of diamagnetic atoms in higher order.  The magnetic quadrupole moment, a P-odd and T-odd distribution of currents in the nucleus is not shielded in the same way as electric moments and   induces an atomic EDM by coupling to an unpaired electron~\cite{Flambaum:1984fb,Flambaum:1984fb-ru}. The magnetic quadrupole moment requires a paramagnetic atom with nuclear spin $I>1/2$, and cesium is the only experimental system so far that meets these requirements. \textcite{Murthy:1989zz} have presented an analysis of their experiment on cesium that extracts the magnetic quadrupole moment.


EDM searches are not confined to neutral systems. Charged particles and ions can be contained in storage rings or with time-dependent fields. For example, the paramagnetic molecular ion HfF$^+$ was stored with a rotating electric field~\cite{Cairncross:2017fip}, and the EDM of the muon was measured in conjunction with the $g-2$, magnetic-moment anomaly measurements at Brookhaven~\cite{Bennett:2006fi}. In the muon experiment, spin-precession due to the EDM coupling to the motional electric field ($\vec v\times \vec B$) was measured, and an upper limit on $d_\mu$ was reported~\cite{Bennett:2008dy}. Though not a dedicated EDM measurement, the technique has demonstrated the possibility of a significantly improved measurement, which is motivated by theoretical suggestions that  lepton EDMs may scale with a power of the lepton mass~\cite{Babu:2000cz}. Storage-ring EDM searches have also been proposed for light nuclei, {\it i.e.} the proton, deuteron and helion ($^3$He$^{++}$)~\cite{Farley04,Khriplovich98,Rathmann:2013rqa}. The electron EDM can also be measured in special ferro-electric and paramagnetic solid-state systems with quasi-free electron spins that can be subjected to applied electric and magnetic fields~\cite{Eckel:2012aw}. We also note that the EDM of the $\Lambda$ hyperon was measured in a spin-precession measurement~\cite{rf:Pondrom1981}, and that limits on the $\tau$ lepton EDM~\cite{Inami:2002ah} and on neutrino EDMs have been derived~\cite{COMMINS19991, doi:10.1143/JPSJ.76.111010}.

A compilation of experimental results is presented in Table~\ref{tb:EDMResults}, which separates paramagnetic (electron-spin dependent) systems from diamagnetic (nuclear and nucleon spin-dependent) systems. 
% which are most sensitive to hadronic effects and nuclear-spin-dependent semi-hadronic contributions. 
In order to cast all results consistently, we have expressed the upper limits (u.l.) at 95\% confidence levels.\footnote{The upper limit $l$ is defined by the $\int_{-l}^l P^\prime(x)dx=0.95$ for a normalized Gaussian probability distribution $P^\prime(x)$ with central value and $\sigma$ given by the total error given in Table~\ref{tb:EDMResults}}



\begin{table}
\centering
\begin{tabular}{|c|l|cr|c|}
\hline\hline
  & {Result}&\multicolumn{2}{|c|} {95\% u.l.} & ref. \\
\hline
\multicolumn{5}{|c|}{Paramagnetic systems}\\
\hline
Xe$^m$ & $d_A=(\ \ 0.7\pm 1.4)\times 10^{-22}$ & $3.1\times 10^{-22}$  & \ecm &  $a$\\
\hline
Cs  &  $d_A=(-1.8\pm6.9)\times 10^{-24}$  & $1.4\times 10^{-23}$& \ecm & $b$ \\
& $d_e=(-1.5\pm 5.7)\times 10^{-26}$    & $1.2\times 10^{-25}$ &\ecm &  \\
&$C_S =  {(2.5\pm9.8)}\times 10^{-6}$ & $2\times 10^{-5}$& &\\
&$Q_m = {(3\pm13)}\times 10^{-8} \  $  & $2.6\times 10^{-7}$&$\mu_N R_\mathrm{Cs}$ &\\
\hline
Tl  &$d_A=(-4.0\pm 4.3)\times 10^{-25}$   & $1.1\times 10^{-24}$ &\ecm &$c$ \\
&   $d_e=(\quad 6.9\pm 7.4)\times 10^{-28}$  & $1.9\times 10^{-27}$& \ecm &  \\
 \hline
YbF &  $d_e=(-2.4\pm 5.9)\times 10^{-28}$     & $1.2\times 10^{-27}$ & \ecm& $d$ \\
\hline
%ThO &  $\omega^{\mathcal NE}=2.6\pm 5.8$  mrad/s & & &$e$   \\
 ThO    &  $d_e=(-2.1\pm 4.5)\times 10^{-29}$   &$9.7\times 10^{-29}$ &\ecm & $e$  \\
     &  $C_S=(-1.3\pm 3.0)\times 10^{-9}$ &$6.4\times 10^{-9}$ & &  \\  
\hline
%HfF$^+$ &  $2\pi f^{BD}=0.6\pm 5.6$  mrad/s & & & $f$   \\
HfF$^+$     &  $d_e=(0.9\pm 7.9)\times 10^{-29}$  &$1.6\times 10^{-28}$ &\ecm  & $f$  \\
%     &  $C_S=(3\pm 2.8)\times 10^{-9}$ $^\dagger$ &$7.6\times 10^{-9}$   &&  \\  
\hline
\multicolumn{ 5}{|c|}{Diamagnetic systems}\\
\hline 
 $^{199}$Hg & $d_A=(2.2\pm 3.1)\times 10^{-30}$  &  $7.4\times 10^{-30}$  & \ecm&$g$\\
\hline
$^{129}$Xe &   $d_A=(0.7\pm 3.3)\times 10^{-27}$  & $6.6\times 10^{-27}$ & \ecm&$h$\\
\hline
$^{225}$Ra &   $d_A=(4\pm 6)\times 10^{-24}$ & $1.4\times 10^{-23}$ &\ecm&$i$\\
\hline
TlF &  $d_{\rm\ }=(-1.7\pm 2.9)\times 10^{-23}$   & $6.5\times 10^{-23}$ & \ecm&$j$\\
\hline
n &  $d_n=(-0.21\pm1.82)\times 10^{-26}$     & $3.6\times 10^{-26}$  &\ecm&$k$\\
\hline\multicolumn{ 5}{|c|}{Particle systems}\\
\hline 
$\mu$ & $d_\mu=(0.0\pm 0.9)\times 10^{-19}$  &$1.8\times 10^{-19}$  & \ecm&$l$\\
\hline 
$\tau$ & $Re(d_\tau)=(1.15\pm 1.70)\times 10^{-17}$  &$3.9\times 10^{-17}$  & \ecm&$m$\\
\hline $\Lambda$ & $d_{\Lambda}=(-3.0\pm 7.4)\times 10^{-17}$  & $1.6\times 10^{-16}$& \ecm& $n$\\
\hline
\hline 
\end{tabular}
\caption{Systems with EDM results and the most recent results as presented by the authors. When $d_e$ is presented by the authors, the assumption is $C_S=0$, and for ThO, the $C_S$ result assumes $d_e=0$. $Q_m$ is the magnetic quadrupole moment, which requires a paramagnetic atom with nuclear spin $I>1/2$. ($\mu_N$ and $R_{\rm Cs}$ are the nuclear magneton and the nuclear radius of $^{133}$Cs, respectively.)
We have combined statistical and systematic errors in quadrature for cases where they are separately reported by the experimenters. References;
$a$~\cite{rf:Player1970}; $b$ \cite{Murthy:1989zz}; $c$  \cite{Regan:2002ta}; $d$ \cite{Hudson:2011zz}; $e$  \cite{Baron:2013eja}; $f$  \cite{Cairncross:2017fip}; $g$  \cite{Graner:2016ses-erratum}; $h$ \cite{rf:Rosenberry2001}; $i$ \cite{rf:Parker2015}; $j$ \cite{rf:Cho1991}; $k$ \cite{Afach:2015sja}; $l$ \cite{Bennett:2008dy}; $m$~\cite{Inami:2002ah}; $n$ \cite{rf:Pondrom1981}. %The value of $C_S$ provided for HfF$^+$ makes use of ref.~\cite{Skripnikov2017}.  
 \label{tb:EDMResults}
}
\end{table}
  %L. Pondrom, R. Handler, M. Sheaff, P. T. Cox, J. Dworkin, O. E.Overseth, T. Devlin, L. Schachinger, and K. Heller: Phys. Rev. D 23 (1981) 814.


\subsection{Theoretical interpretation}

\label{sec:TheoreticalInterpretation}

The results on EDMs presented in Table~\ref{tb:EDMResults} have significant theoretical impact in several contexts by constraining explicit parameters of  SM and BSM physics. The Standard Model has two explicit CP-violating parameters: the phase in the CKM matrix, and the coefficient $\bar\theta$ 
%, of the CP-violating $G{\widetilde G}$ 
 in the SM strong interaction Lagrangian.  
EDMs arising from the CKM-matrix vanish up to three loops for the electron~\cite{Bernreuther:1990jx} and up to two loops for quarks~\cite{Shabalin:1978rs,Shabalin:1978rs-ru,Shabalin:1982sg,Shabalin:1982sg-ru}.  The leading SM contributions to the neutron EDM, however, arise from a combination of hadronic one-loop and %\lq\lq pole" 
resonance contributions,  each a combination of two $\Delta S=1$ hadronic interactions (one CP violating and one CP-conserving). The CP-violating $\Delta S=1$ vertex is itself a one-loop effect, arising from the QCD  \lq\lq Penguin" process (See FIG.~\ref{fg:NeutronEDMVertex}). 
The estimate of the corresponding neutron EDM is $(1-6)\times 10^{-32}$ \ecm~\cite{Seng:2014lea}, where the range reflects the present hadronic uncertainties. For both the electron and the neutron, the SM CKM contribution lies several orders of magnitude below the sensitivities of recent and next-generation EDM searches. The Penguin process generated by the exchange of a kaon between two nucleons induces CP-violating effects in nuclei; however \textcite{Donoghue:1987dd} and \textcite{Yamanaka:2015ncb} show that this contribution is also many orders of magnitude below current experimental sensitivity for diamagnetic atom EDMs.
EDMs of the neutron and atoms also uniquely constrain the SM strong-interaction parameter $\bar\theta$ which sets the scale of strong CP violation as discussed in Sec.~\ref{sec:Theory}). 

%The CKM matrix elements including the CP violating complex phase are constrained by a large set of experimental results, and explicit calculations of the CKM contribution to the neutron EDM (at 3 loops) and the electron EDM (at 4 loops) are far below current experimental sensitivity. On the other hand, strong CP-violation is not constrained by other measurements, and therefore the EDM results can be used to show that $\bar\theta<<1$ as described in detail in section~\ref{sec:GlobalAnalysis}. This has led to the strong-CP problem: if $\bar\theta$ is identically zero there must be a symmetry requiring strong-CP conservation;  if it is small but finite suppression of $\bar\theta$ has a dynamical origin, for example in the proposed axion coupling. The axion explanation has been very compelling and has spawned a number of experimental endeavors~\cite{rf:AxionSearchReview}. 

BSM theories generally provide new degrees of freedom and complex CP-violating couplings that often induce EDMs at the one-loop level. The most widely-considered BSM scenarios for which implications have been analyzed include supersymmetry~\cite{Pospelov:2005pr,RamseyMusolf:2006vr}, the two-Higgs model~\cite{Inoue:2014nva}, and left-right symmetric models~\cite{Pati:1974yy,Pati:1974yy-erratum,Mohapatra:1974hk,Senjanovic:1975rk}.% [see~\cite{Engel:2013lsa} for a review of LRSM EDM computations]. Examples are presented in Section~\ref{sec:Theory}
% (see also Refs.~\cite{Pospelov:2005pr,RamseyMusolf:2006vr,Engel:2013lsa}). 

%etc. introduce a number of parameters with possible complex phases that would induce an EDM. In general, there are a very large number of parameters in the model that are not (yet) constrained by experiment, and several authors have provided plausible ranges for the electron and neutron EDM by restricting the model parameters in justifiable ways, however there are no concrete predictions of the scale of an EDM, rather what models can do is establish relations between the effective-field theory parameters that can then be tested against the constraints set by a global analysis of experiments.

A complementary, model-independent framework for EDM interpretation relies on effective field theory (EFT), presented in detail Sec.~\ref{sec:EFTParameters}). % and reviewed in detail in Ref.~\cite{Engel:2013lsa}. 
The EFT approach assumes that the BSM particles are sufficiently heavy that their effects can be compiled into  a  set of residual weak-scale, non-renormalizable operators involving only SM fields. The corresponding operators are dimension six  %- {\it i.e.} scale as the sixth power of energy
 and  effectively depend on $(v/\Lambda)^2$, where %$d$ is the operator's canonical dimension. For $\Lambda$ much greater than the Higgs vacuum expectation value 
$v=246$ GeV is the Higgs vaccum-expectation-value and $\Lambda$ is the energy scale of the new physics. The strength of each operator's contribution is characterized by a corresponding Wilson coefficient.  In addition to $\bar\theta$ there are the following 12 dimension-six BSM Wilson coefficients representing the intrinsic electron EDM, up-quark and down-quark EDMs and CEDMs, one CP-violating three gluon operator, five four-fermion operators, and one quark-Higgs boson interaction operator. Experimental EDM results constrain the Wilson coefficients, while a given BSM theory provides predictions for the Wilson coefficients in terms of the underlying model parameters. %One may, however, endeavor to interpret EDM results entirely in terms of the Wilson coefficients without making any reference to a particular model and subsequently utilize the results within specific model realizations.

Interactions involving light quarks and gluons are, of course, not directly accessible to experiment. Consequently, it is useful to consider their manifestation in a low-energy effective theory (below the hadronic scale $\Lambda_\mathrm{had}\sim 1$ GeV) involving electrons, photons, pions, and nucleons. Hadronic matrix elements of the quark and gluon EFT operators then yield the hadronic operator coefficients. At lowest non-trivial order, one obtains the electron EDM ($d_e$); scalar, pseudoscalar, and tensor electron-nucleon interactions ($C_S$, $C_P$, and $C_T$, respectively\footnote{Each interaction has an isoscalar and isovector component, which we have suppressed here for notational simplicity.}; short-range neutron and proton EDMs ($\bar d_n^{sr}$ and $\bar d_p^{sr}$);  isoscalar, isovector, and isotensor pion-nucleon couplings ($\gpbi$, $i=0,1,2$); and a set of four-nucleon operators. 
%CP-odd pion-nucleon couplings (3), and CP-violating electron-nucleus couplings (4). 
%In general there are multiple sources for the CP-violating EDM or related observable in each system experimentally investigated to date. (The muon is a fundamental fermion and it's intrinsic EDM is the only exception so far.) 
In the context of this hadronic-scale EFT, it is appropriate to express the combination of contributions to a measured atomic  EDM for paramagnetic systems, diamagnetic systems, and nucleons as
%\begin{eqnarray}
%d_{para}\approx \alpha_{d_e} d_e + \alpha_{C_S} C_S  ,
%\\
%d_{dia}\approx \alpha_{d_e} d_e + \alpha_{C_T} C_T + \alpha_{\bar d_n^\mathrm{sr}} \bar d_n^\mathrm{sr} + \alpha_{\bar d_p^\mathrm{sr}} \bar d_p^\mathrm{sr}  + \alpha_{g_\pi^0} \bar g_\pi^0 + \alpha_{g_\pi^1}  \bar g_\pi^1 ,
%\\
%d_{nucleon}=\alpha_{d_e} d_e + \alpha_{C_S} C_S + \alpha_{C_T} C_T + \alpha_{\bar d_n^\mathrm{sr}} \bar d_n^\mathrm{sr} + \alpha_{\bar d_p^\mathrm{sr}} \bar d_p^\mathrm{sr}  + \alpha_{g_\pi^0} \bar g_\pi^0 + \alpha_{g_\pi^1}  \bar g_\pi^1 ,
%\end{eqnarray}
%where $\alpha_{d_e}=\partial d/\partial d_e$, {\it etc.}. This can be compactly written as
\begin{equation}
d_i=\sum_{j} \alpha_{ij} C_j,
\label{eq:d_i}
\end{equation}
where $i$ labels the system, and $j$ labels the specific low-energy parameter ({\em e.g.}, $d_e$, $C_S$, {\em etc.}). 
The $\alpha_{ij}=\partial d_i/\partial C_j$ are provided by theoretical calculations at various scales from atomic to nuclear to short-range and are presented in Sec.~\ref{sec:Theory}. Note that the coefficients $\alpha_{ij}$ have various labels in the literature for notation developed for the different experimental systems. % and Tables~\ref{tb:paramagnetics} and~\ref{tb:diamagnetics} for the nucleons and diamagnetic and paramagnetic atoms and molecules, respectively. %The sensitivity of the EDM for each experimental system to the parameters presented as a best value and a reasonable range as set forth in Ref.~\cite{Engel:2013lsa}. where $k_T^{(0,1)}$ give the sensitivities of the atomic EDM to the isoscalar and isovector electron-quark tensor interactions. 
%A compilation of the $\rho_Z^N$, $\kappa_S$, and $k_T^{(0,1)}$ is given in Table~\ref{tb:rhokappak_ST} and in~\cite{Engel:2013lsa}
%can be found in Ref.~\cite{Engel:2013lsa}\footnote{We note that the values for  the $\kappa_S$ given in that work should be multiplied by an overall factor of $-1$ given the convention used there and in Eqn.~(\ref{eq:diamag}).}.


One approach to interpreting the experimental limits assumes that the EDM in a specific system arises from only one source -- the \lq\lq sole-source" approach. In the sole-source approach,  the constraint on each parameter is derived assuming that all other contributions are negligible and so one experimental result may appear to set limits on a large number of individual CP-violating parameters. An alternative approach  -- the global analysis  presented in~\textcite{Chupp:2014gka} and Sec.~\ref{sec:GlobalAnalysis} -- assumes simultaneous non-zero values of the dominant parameters globally constrained by the experimental results. In the global analysis, paramagnetic systems are used to set limits on the electron EDM $d_e$ and the nuclear spin-independent electron-nucleus coupling $C_S$.  Diamagnetic systems set limits on  four dominant parameters: two pion-nucleon couplings ($\gpbz,\gpbo$), a specific isospin combination of nuclear spin-dependent couplings, and the  \lq\lq short distance" contribution to the neutron EDM, ${\bar d}^{sr}_n$. There is, unfortunately, significant variation and uncertainty in the $\alpha_{ij}$, in particular for the nuclear and hadronic calculations, which soften the constraints on the low-energy parameters.



\begin{figure}[tb]
\vskip -0.75 truein
\centerline{\includegraphics[width=4.25 truein]{TCPenguin}}%{NeutronEDMVertex}
\vskip -0.75 truein
\caption{\label{fg:NeutronEDMVertex}  Penguin diagram  giving the SM-CKM $\Delta S=1$ CP-violating effective interaction. Adapted from \textcite{Pospelov:2005pr}.}
\end{figure}



%Beyond-Standard Model physics generally provides new degrees of freedom and complex amplitudes that naturally introduce CP violation at tree (one-loop) level. While models introduce (generally many) new parameters, effective-field theory provides a context for generalizing the interpretation of experimental results.    A set of dimension-6 effective-field theory (EFT) operators (Sec~\ref{sec:EFTParameters})  gives rise to 12 low-energy parameters representing intrinsic electron and quark EDMs (3), induced quark chromo-EDMs (2) CP-odd pion-nucleon couplings (3), and CP-violating electron-nucleus couplings (4). 
%In general there are multiple sources for the CP-violating EDM or related observable in each system experimentally investigated to date. (The muon is a fundamental fermion and it's intrinsic EDM is the only exception so far.) 
%It is therefore appropriate to express the combination of contributions to a measured atomic  EDM for paramagnetic, diamagnetic and nucleons as
%\begin{eqnarray}
%d_{para}\approx \alpha_{d_e} d_e + \alpha_{C_S} C_S  ,
%\\
%d_{dia}\approx \alpha_{d_e} d_e + \alpha_{C_T} C_T + \alpha_{\bar d_n^\mathrm{sr}} \bar d_n^\mathrm{sr} + \alpha_{\bar d_p^\mathrm{sr}} \bar d_p^\mathrm{sr}  + \alpha_{g_\pi^0} \bar g_\pi^0 + \alpha_{g_\pi^1}  \bar g_\pi^1 ,
%\\
%d_{nucleon}=\alpha_{d_e} d_e + \alpha_{C_S} C_S + \alpha_{C_T} C_T + \alpha_{\bar d_n^\mathrm{sr}} \bar d_n^\mathrm{sr} + \alpha_{\bar d_p^\mathrm{sr}} \bar d_p^\mathrm{sr}  + \alpha_{g_\pi^0} \bar g_\pi^0 + \alpha_{g_\pi^1}  \bar g_\pi^1 ,
%\end{eqnarray}
%where $\alpha_{d_e}=\partial d/\partial d_e$, {\it etc.}. This can be compactly written as
%\begin{equation}
%d_i=\sum_{j} \alpha_{ij} C_j,
%\label{eq:d_i}
%\end{equation}
%where $i$ labels the system, and $j$ labels the physical contribution. 
%The coefficients $\alpha_{ij}=\partial d_i/\partial C_j$ are provided by theoretical calculations at various scales from atomic to nuclear to short-range and are listed in Sec.~\ref{sec:GlobalAnalysis} and Tables~\ref{tb:paramagnetics} and~\ref{tb:diamagnetics} for the nucleons and diamagnetic and paramagnetic atoms and molecules, respectively. %The sensitivity of the EDM for each experimental system to the parameters presented as a best value and a reasonable range as set forth in Ref.~\cite{Engel:2013lsa}. where $k_T^{(0,1)}$ give the sensitivities of the atomic EDM to the isoscalar and isovector electron-quark tensor interactions. 
%A compilation of the $\rho_Z^N$, $\kappa_S$, and $k_T^{(0,1)}$ is given in Table~\ref{tb:rhokappak_ST} and in~\cite{Engel:2013lsa}
%can be found in Ref.~\cite{Engel:2013lsa}\footnote{We note that the values for  the $\kappa_S$ given in that work should be multiplied by an overall factor of $-1$ given the convention used there and in Eqn.~(\ref{eq:diamag}).}.


%One approach to interpreting the experimental limits assumes that the EDM in a specific system arises due to only one source -- the sole-source approach. In the sole-source approach,  the constraint on each parameter is derived assuming that all other contributions are negligible and so one experimental result may appear to set limits on a large number of individual CP-violating parameters. An alternative approach  -- the global analysis  presented in~\cite{Chupp:2014gka} and summarized in Sec.~\ref{sec:GlobalAnalysis}-- allows for simultaneously non-zero values of the dominant parameters to extract values constrained by experiment. In this global analysis, paramagnetic systems are used to set limits on the electron EDM $d_e$ and the nuclear spin-independent electron-nucleus coupling $C_S$  while the diamagnetic systems set limits on the four dominant parameters: two pion-nucleon couplings, a specific isospin combination of nuclear spin-dependent couplings and the intrinsic quark-EDM contribution to the neutron EDM $d_n$. There is, unfortunately, significant variation and uncertainty in the $\alpha_{ij}$, in particular for the nuclear and hadronic calculations, which can be viewed as softening the constraints on the low-energy parameters.


%Models of new physics, e.g. supersymmetry~\cite{rf:SUSY} and and related multi-Higgs models~\cite{rf:multiHiggs}, left-right symmetric models~\cite{rf:LRsym}, etc. introduce a number of parameters with possible complex phases that would induce an EDM. In general, there are a very large number of parameters in the model that are not (yet) constrained by experiment, and several authors have provided plausible ranges for the electron and neutron EDM by restricting the model parameters in justifiable ways, however there are no concrete predictions of the scale of an EDM, rather what models can do is establish relations between the effective-field theory parameters that can then be tested against the constraints set by a global analysis of experiments.

\subsection{Reach and complementarity}
\label{sec:Reach}

%Standard-Model EDMs arising from the CKM-matrix vanish up to three-loops for the electron~\cite{Bernreuther:1990jx} and up to two loops for quarks\cite{Shabalin:1978rs,Shabalin:1982sg}.  The leading contributions to the neutron EDM, however, arise from a combination of hadronic one-loop and \lq\lq pole" (resonance) contributions, where each entails a combination of two $\Delta S=1$ hadronic interactions (one CP violating and one CP-conserving). The CP-violating $\Delta S=1$ vertex is, itself, a one-loop effect, arising from the QCD  \lq\lq Penguin" process (See FIG.~\ref{fg:NeutronEDMVertex}). 
%The estimate of the corresponding neutron EDM is $(1-6)\times 10^{-32}$ \ecm~\cite{Seng:2014lea}, where the range reflects the present hadronic uncertainties. For both the electron and the neutron, the Standard-Model CKM contribution lies several orders of magnitude below the sensitivities of recent and next-generation neutron-EDM searches. The Penguin process generated by the exchange of a kaon between two nucleons also induces an  EDM in diamagnetic atoms; however this contribution is also estimated to fall several orders of magnitude below current and anticipated experimental sensitivity  (See Ref.~\cite{Donoghue:1987dd} and references therein). 

%EDMs of the neutron and atoms provide a unique window on CP-violation in the Standard-Model strong interaction through the QCD \lq\lq $\theta$-term", a renormalizable P-odd/T-odd gluonic operator whose strength is set by the parameter ${\bar\theta}$  (see Section \ref{sec:Theory}). The QCD contribution to theneutron EDM  is given by 
%\begin{equation}
%d_n \approx (10^{-16} {\rm \ecm~})\ {\bar\theta}\ \ \ .
%\end{equation}
%The parameter $\bar\theta$ is na\"ively epected to be of order unity; however  the current upper bound, $d_n\le 3.6\times 10^{-26}$ \ecm~ (95\% C.L.) (Table~\ref{tb:EDMResults}) implies
%$|{\bar\theta}|\lsim 10^{-10}$, assuming this is the only contribution to the neutron EDM. The corresponding bound obtained from the $^{199}$Hg EDM limit is comparable. This severe constraintn on $\bar\theta$ has 
%motivated a variety of  theoretical explanation. The most widely considered is the existence of a spontaneously-broken Peccei-Quinn symmetry and the associated axion, which itself provides a candidate for the observed relic density of cold dark matter. The axion explanation has been very compelling and has spawned a number of experimental endeavors~\cite{rf:AxionSearchReview}. 

EDMs arising from BSM CP violation depend on a combination of factors, including new CP-violating phases, $\phi_{\rm CPV}$, the mass scale $\Lambda$ associated with the new particles, and the underlying dynamics. In general, an elementary fermion EDM can be expressed as
\begin{equation}
d \approx (10^{-16}  \ecm~)  \left(\frac{v}{\Lambda}\right)^2\ (\sin\phi_{\rm CPV}) (y_f\, F) \ \ \ ,
\label{eq:edmbsm}
\end{equation}
where $v=246$ GeV is the Higgs vacuum expectation value, $y_f$ is a Yukawa coupling typically associated with the SM fermions in the system of interest, and $F$ accounts for the dynamics, which may be perturbative or non-perturbative and will differ depending on the system. 

For an electron EDM that arises through perturbative dynamics at the  one-loop level, $F\sim g^2/(16\pi^2)$ where $g$ is the BSM coupling strength. The present electron EDM upper limit of $\approx 1\times 10^{-28}$ \ecm~ 
%$9.7\times 10^{-29}$ \ecm~ from \textcite{Baron:2013eja} 
implies that $\Lambda\gtrapprox 1-2$ TeV for $g$ of order the SM SU(2$)_L$ gauge coupling strength, and $\sin\phi_{\rm CPV}\approx 1$. This energy scale for $\Lambda$, which is comparable to the reach of the neutron and diamagnetic atom EDM limits, rivals the BSM physics reach of the LHC.  % \cite{rf:LHCReachReview}. 
It is important to note that exceptions to these na\"ive estimates of mass scale sensitivity can occur. For example at the level of the underlying elementary particle physics, an EDM may be enhanced by contributions of heavy fermion intermediate states, {\em e.g.}, the top quark, leading to the presence of a larger Yukawa coupling in Eqn.~(\ref{eq:edmbsm}). 
%than one would generically expect. 
In paramagnetic atoms and molecules an EDM
 may also be generated by a  nuclear-spin independent scalar T-odd/P-odd electron-quark interaction at tree-level, which generally scales with the number of nucleons. 
 %may also be enhanced by nuclear coherence \magenta{Michael: what does nuclear coherence mean?} 
In this case, the resulting mass reach for current experimental sensitivies is as high as $\sim 13,000$ TeV, as discussed in Section~\ref{sec:Interpretation}. 
%From Eqn.~(\ref{eq:edmbsm}) and the experimental limits in Table~\ref{tb:EDMResults},   one would generally conclude that the new particles must be heavier than a few TeV in any model that gives rise to one-loop EDMs, assuming $|\sin\phi_{\rm CPV}|\sim \mathcal{O}(1)$. 
Models that generate EDMs at two-loop or higher-loop order allow for lighter BSM particles with CP-violating interactions. 



%***** 
%\magenta{Is this - Schiff theorem - out of place in the "reach and complementarity" section}

%\magenta{On the flip side, the well-known } For diamagnetic atoms with no unpaired electrons, and atomic EDM arises due to T-odd/P-odd nuclear interactions as well as a nuclear-spin-dependent electron-nucleus interaction; however for a neutral, non-relativistic system of point-like constituents that experience only Coulomb interactions, the existence of a non-zero EDM of any constituent will yield no energy (or frequency) shift in the presence of an external electric field as first clarified by Schiff \textcite{Schiff:1963zz}. An observable atomic or molecular EDM can arise only when the system is relativistic, as in the paramagnetic systems; when there exist non-Coulombic interactions ({\em e.g.} a magnetic electron-nucleus interaction); or a deviation from the point-like limit. For the diamagnetic systems of experimental interest, the latter effect arising from finite nuclear size is characterized by the nuclear Schiff moment, $S$. Atomic electrons probe $S$ through the effective interaction
%\begin{equation}
%\label{eq:schiffatom}
%H=-4\pi\vec\nabla\rho(0)\cdot\vec S,
%\end{equation}
%where $\vec\nabla\rho(0)$ is the gradient of the electron density at the nucleus. The Schiff moment itself is effectively an $r^3$-weighted moment of the nuclear charge, in comparison with the $r$-weighted moment that gives the EDM. Under various simplifying assumptions, it can be expressed as {\color{magenta} check \& update}
%\begin{equation}
%S=\frac{1}{10}\langle r^2\vec r_p\rangle-\frac{1}{6}Z\langle r^2\rangle\langle\vec r_p\rangle \ \ ,
%\label{eq:SchiffDef}
%\end{equation}
%where the $\langle\ldots\rangle$ denote corresponding moments with respect the nuclear charge density {\color{magenta} update and make more precise} (Schiff's theorem has been recently reformulated in work showing that these formulas are approximations that may not be justified~\cite{rf:Liu2007,rf:Senkov2008}.). 

For the Schiff moment, the additional power of $r^2$ in Eqn.~(\ref{eq:SchiffDef})
%, as compared to the EDM (\ref{eq:EDMdef})
 implies that for a given underlying source of CP-violation, the contribution to a diamagnetic atomic or molecular EDM is suppressed compared to that of the neutron by $(R_N/R_A)^2$, where $R_N$ and $R_A$ are the nuclear and atomic radii, respectively. As a concrete illustration the bound on $\bar{\theta}$ arising from the $^{199}$Hg EDM limit is comparable to the bound from $d_n$, even though the respective EDM limits differ by nearly four orders of magnitude (see Table~\ref{tb:EDMResults}).

An example of unique constraints set by EDM searches is found in the strong CP contribution to the neutron EDM given by~\textcite{Pospelov:1999ha,Crewther:1979pi,rf:Crewther1979-erratum} and \textcite{Shindler:2015aqa}
\begin{equation}
d_n \approx (10^{-16} e-{\rm cm})\ {\bar\theta}\ \ \ .
\label{eq:dnthetabar}
\end{equation}
The parameter $\bar\theta$ is na\"ively expected to be of order unity. However assuming this is the only contribution to the neutron EDM, the current upper bound, from Table~\ref{tb:EDMResults} implies
$|{\bar\theta}|\lessapprox 10^{-10}$. The corresponding bound obtained from the $^{199}$Hg EDM limit is comparable. This severe constraint on $\bar\theta$ has 
motivated a variety of  theoretical explanations. The most widely considered explanation is the existence of a spontaneously-broken Peccei-Quinn symmetry~\cite{Peccei:1977ur,Peccei:1977hh} and an associated particle - the axion~
\cite{Weinberg:1977ma,Wilczek:1977pj}. The axion is also  a candidate for the observed relic density of cold dark matter. The axion proposal has been very compelling and has spawned a number of experimental endeavors summarized, for example by~\textcite{doi:10.1146/annurev-nucl-102014-022120}. 

 
 
 %; for example, the leading Standard-Model neutron EDM CP-violating vertex  is shown in FIG.~\ref{fg:NeutronEDMVertex}. On the other hand, a one-loop (tree level) contribution would arise from the imaginary component of the coupling to a new heavy particle
%$X$ shown in FIG.~\ref{fg:NeutronEDMVertex}b. In the case of such a particle at mass scale  of mass $\Lambda$. The resulting EDM compared to the anomalous magnetic moment would be given by
%\begin{equation}
%\frac{cd}{\mu}\approx f^{2N}\frac{m_q^2}{\Lambda^2}\sin\phi_{\rm CPV},
%\end{equation}
%or
%\be
%5d \approx (10^{-16} {\rm \ecm~}) (\frac{v}{\Lambda})^2(\sin\phi_{\rm CPV}) (y_fF)
%5\ee
%where $f$ is the coupling strength, $N$ is the number of loops ($N=1$ for this example), and $\phi_{\rm CPV}$ is the CP-violating phase, $v$ the Higgs vacuum-expectation value, and $y_fF$ is the scaling due to dynamics. The current upper limit on the neutron EDM is $d_n\le 3.6\times 10^{-26}$ \ecm~ (95\% C.L.) (Table~\ref{tb:EDMResults}), and ${\mu_n/ c}=2.0\times 10^{-14}$ \ecm, thus. Taking $f\approx \alpha_S$ and $\sin\phi_{\rm CPV}\approx 1$, $\Lambda \ge$ 1-2 TeV, rivaling the LHC in reach to new physics~\cite{rf:LHCReachReview}. A similar conclusion follows for considering the limits on the electron EDM from measurements in paramagnetic systems, and may extend orders of magnitude further when considering the limits on CP-violating electron-nucleus coupling in paramagnetic systems as discussed in section~\ref{sec:Interpretation}.



%\subsection{Techniques and activities}
\begin{figure*}[tb]
%\includegraphics[height=2 truein]{EWBGBubbles}\includegraphics[height=3 truein]{EWBGBubbleWall}
\vskip -1 truein
\includegraphics[width=7 truein]{EWBGBubblesComposite}
\vskip -1.5 truein
\caption{\label{fg:EWBGBubbles}  (Color online) Left side: bubble nucleation during first order electroweak phase transition. Right side: CP- and C-violating interactions in the $\langle\phi_H\rangle = 0$ background near the bubble walls that produce baryons. Figure \copyright~ IOP Publishing Ltd and Deutsche Physikalische Gesellschaft;  reproduced from~\textcite{Morrissey:2012db}  by permission of IOP Publishing. }
\vskip .5 truein
\end{figure*}


\subsection{EDMs and baryogenesis}

Baryogengesis, the generation of a net asymmetry of matter over antimatter in the early universe, requires three components as first explained by~\textcite{SakharovBaryogenesis-ru,SakharovBaryogenesis}: (1) violation of baryon number $B$; (2) departure from thermodynamic equilibrium (assuming CPT invariance); and 3) both C-violating and CP-violating  processes. A number of baryogenesis scenarios that satisfy these requirements have been proposed, each typically focusing on a certain era in cosmic history and corresponding energy scale. Among the most widely considered and experimentally testable scenarios is electroweak baryogenesis. For  recent reviews of electroweak baryogenesis, see \textcite{Morrissey:2012db}; and see~\textcite{Riotto:1999yt} and~\textcite{Dine:2003ax} for more general baryogenesis reviews. 

In the electroweak baryogenesis (EWBG) scenario, the universe proceeds from initial conditions having no net baryon number and vanishing Higgs field expectation value $\langle\phi_H\rangle=0$, implying that the Standard-Model SU(2)$_L\times$U(1)$_Y$ electroweak symmetry has not yet been broken by the Higgs mechanism (giving mass to the $W^\pm$ and $Z$ bosons). As the plasma  cools below the electroweak scale of $\approx$ 100 GeV, $\langle\phi_H\rangle$ becomes non-zero, breaking electroweak symmetry. EWBG requires that the transition be a first order phase transition, proceeding via the nucleation of bubbles of broken symmetry with $\langle\phi_H\rangle \ne 0$ as suggested in the left side of FIG.~\ref{fg:EWBGBubbles}. These bubbles expand and fuse into a single  phase with $\langle\phi_H\rangle \ne 0$. 
%In the SM, however, this transition is a \lq\lq smooth" crossover type for a Higgs boson with mass larger than about 70 GeV. Various well-motivated SM extensions introduce new particles and interactions that can lead to a first order transition even for a Higgs boson with the observed mass of 125 GeV.
CP-violating and C-violating interactions in the $\langle\phi_H\rangle = 0$ background near the bubble walls produce $B\ne0$ in processes that convert baryons to antileptons or antibaryons to leptons illustrated in the  right side of FIG.~\ref{fg:EWBGBubbles}. 
%(conserving charge and the difference of baryon and lepton number $B-L$) 
%at different rates.  
These processes are referred to  as sphaleron transitions that arise from a configuration of the Standard-Model fields in the $\langle\phi_H\rangle=0$ phase that corresponds to a saddle point of the electroweak effective action~\cite{Klinkhamer:1984di}. As the bubbles expand and merge, they sweep up the $B\ne 0$ regions as they eventually coalesce into the universe that persists to the present epoch. However if sphaleron transitions persist inside the bubbles the baryon number would not be preserved.  Thus, the first order transition must be sufficiently \lq\lq strong"  so as to quench the sphaleron transitions inside the broken phase.  

In principle, the SM with CP-violation from the CKM matrix provides all of the ingredients for this scenario; however the phase transition cannot be first order for a Higgs mass greater than $\sim$70 GeV. Given the observed Higgs mass  $m_H=125.09\pm 0.24$ GeV~\cite{rf:HiggsMassdoi10.1103/PhysRevLett.114.191803}, a first order phase transition cannot have occurred in a purely SM universe. Even if the value of $m_H$ were small enough to accommodate a first order electroweak phase transition,  the effects of CKM CP-violation are too feeble to have resulted in the observed matter-antimatter asymmetry. Thus, electroweak baryogenesis requires BSM physics for two reasons: generation of a strong, first order electroweak phase transition and production of sufficiently large CP-violating asymmetries during the transition. New particle searches at colliders may discover new interactions responsible for a first-order phase transition~\cite{Assamagan:2016azc,Contino:2016spe}, but it is EDMs that provide the most powerful probe of the new CP-violating interactions.

Electroweak baryogenesis  provides an additional constraint on the BSM mass scale $\Lambda$ and on CP-violating phase(s) that set the scale of EDMs. Eqn.~\ref{eq:edmbsm} shows that experimental limits on EDMs constrain the ratio $\sin\phi_{\rm CPV}/\Lambda^2$ but do not separately constrain $\Lambda$  and $\sin\phi_{\rm CPV}$. However the requirements for electroweak baryogenesis do provide complementary constraints on the mass scale and CP-violating phases. We illustrate this in FIG.~\ref{fg:MSSMdedn} from~\textcite{Li:2008ez}, which shows constraints on parameters of the minimal supersymmetric Standard Model (MSSM) needed to generate the observed matter-antimatter asymmetry and the corresponding EDMs that would arise. The horizontal and vertical axes give the soft SUSY-breaking {\it bino} mass parameter $M_1$ and the CP-violating \lq\lq  bino phase" $\sin(\Phi_M)=\sin\mathrm{Arg}(\mu M_1b^\ast)$. The green band shows the relationship between these parameters needed to produced the matter-antimatter asymmetry, while the nearly horizontal lines indicate values of the electron (left panel) and neutron (right panel) EDMs. The EDMs have been computed in the limit of heavy {\it sfermions}, which is consistent with LHC results, so that EDMs arise from two-loop graphs containing the electroweak {\it gauginos}. Note that the present $d_e$ limit roughly excludes the region   $d_e>10^{-28}$ \ecm, while the current neutron-EDM bound does not yet constrain the indicated parameter space. The next generation electron and neutron EDM experiments are expected to probe below $d_e<10^{-29}$ \ecm~ and $d_n<10^{-27}$ \ecm.%{\color{magenta} figure from MSSM or other goes here}

\begin{figure*}[htb]
\includegraphics[width=3.25 truein]{M1_de}\quad\includegraphics[width=3.25 truein]{M1_dn}
\caption{\label{fg:MSSMdedn}  (Color online) Sensitivity of the electron EDM (left panel) and neutron EDM (right panel) to the baryon asymmetry in the MSSM.  The horizontal axes give the bino soft mass parameter, $M_1$; the vertical axes give the sine of the relative phase of $M_1$, the supersymmetric $\mu$ parameter, and the soft Higgs mass parameter $b$. The green bands indicate the values of these parameters needed to obtain the observed baryon asymmetry. Nearly horizontal lines give contours of constant EDMs. Figure originally published in \textcite{Li:2008ez}.  }
\end{figure*}


%Baryogengesis, the generation of a net asymmetry of matter over antimatter in the early universe, requires three components as first explained by Sakharov~\cite{rf:SakharovBaryogenesis}: 1) departure from thermodynamic equilibrium, 2) violation of baryon number and 3) both C- and CP-violating  processes. These can be satisfied in the process of electroweak-baryogenesis~\cite{rf:EWBGrefs} from the initial conditions of a hot, radiation-dominated plasma containing zero net baryon number in which SU(2)$\times$U(1) electroweak symmetry has not been broken by the Higgs mechanism~\cite{rf:OtherEWBGrefs}. As the plasma of the early Universe cools below the electroweak scale ($\approx$ 100GeV), the Higgs field takes on a non-zero value  ($\langle\phi\ne 0\rangle$) in a first-order phase transition known as spontaneous  electroweak symmetry breaking. C- and CP-violating interactions before this phase transition can generate a net baryon number that is frozen into the broken-symmetry phase as it propagates in the plasma. In principle, the Standard Model provides all of the ingredients for this recipe; however the phase transition cannot be first order for a Higgs mass greater than 70 GeV. Of course the Higgs mass is  $m_H=125.09\pm 0.24$ GeV~\cite{rf:HiggsMassdoi10.1103/PhysRevLett.114.191803}, and this first order phase transition cannot occur implying that beyond-Standard-Model physics near the electroweak scale is necessary to generate the baryon asymmetry. A number of extensions to the Standard Model have been proposed that introduce either new scalars, which couple to the Higgs field and modify the effective potential or scalar fields with additional coupling to the Higgs affects the effective Higgs potential directly. Both mechanisms introduce new light scalars that may also generate an experimentally observable EDM as well as be detectable as a particle produced at high energy or as a dark-matter candidate (WIMP).





\section{Theoretical background}
\label{sec:Theory}

\subsection{CP/T Violation}

%\noindent{\color{magenta} Is this subsection really needed? It's textbook. And we have already introduced C, P, and T in the introduction. I would do away with it}

Parity (P), Time Reversal (T) and Charge Conjugations (C) are the discrete symmetry transformations of quantum mechanics and quantum field theory. Experiment shows that the strong and electromagnetic interactions are symmetric under C, P and T separately and under CP and T. The weak interaction, which  involves only left-handed neutrinos and right-handed antineutrinos, is maximally antisymmetric under P and C . CP is also violated in weak decays of kaons and b-mesons. Symmetry under the combined transformations of C, P, and T (or CPT) is consistent with experiment and is also required for any Lorentz-invariant quantum field theory as embodied in the CPT theorem~\cite{Tureanu:2013psa,rf:LudersCPTTheorem,Jost:1957zz}.

Parity, a unitary transformation described by $(t,\vec r)\rightarrow (t_P,\vec r_P)=(t,-\vec r)$, reverses the handedness of the coordinate system, {\it i.e.} $\hat x\times\hat y=\hat z$ while $\hat x_P\times \hat y_P=-\hat z_P$. Particles have intrinsic parity, that is the field describing a particle acquires a phase of $\pm1$ under the parity transformation. For fermions, particles and antiparticles have opposite intrinsic parity. 

The time-reversal transformation is described by $(t,\vec r)\rightarrow (t_T,\vec r_T)=(-t,\vec r)$, but when this is applied to wave functions or fields not only is motion reversed (e.g. $\vec p\rightarrow \vec p$ and $\vec J\rightarrow -\vec J$) but the imaginary phase is reversed as well, {\it i.e.} time reversal includes complex conjugation. Moreover for scattering and decay processes, the initial and final state are reversed, which is a complication for interpreting any experiment subject to final sate corrections such as detailed balance or decay correlation measurements~\cite{rf:FSI}.
%, which are subject to final state corrections, for example due to the phase shift of a final state charged particle moving in a coulomb field~\cite{rf:FSI}. 
Thus the {\it anti-unitary} time-reversal transformation involves motion reversal, complex conjugation and reversal of initial and final states. For an EDM, however, the initial and final states are the same, so there is no complication from final-state effects and a definitive observation of an EDM is a direct signature of T-violation and, invoking the CPT theorem, of CP-violation.

Charge conjugation transforms particles into antiparticles, reversing charge without reversing the handedness or spin. It is interesting to note, therefore, that CPT symmetry requires the EDM   of a particle  and the EDM of its antiparticle  be equal in magnitude and opposite in sign:
%($\bar\mu\langle\vec J\rangle=-\mu\langle\vec J\rangle$): 
\begin{equation}
d\langle\vec J\rangle\xrightarrow{C}\bar d\langle\vec J\rangle\xrightarrow{P}+\bar d\langle\vec J\rangle\xrightarrow{T}\bar d\langle-\vec J\rangle=-\bar d\langle\vec J\rangle.
\end{equation}
Similarly, for the magnetic moments:
\begin{equation}
\mu\langle\vec J\rangle\xrightarrow{C}\bar\mu\langle\vec J\rangle\xrightarrow{P}+\bar\mu\langle\vec J\rangle\xrightarrow{T}\bar\mu\langle-\vec J\rangle=-\bar\mu\langle\vec J\rangle.
\end{equation}


\subsection{General framework}

%\noindent{\color{magenta} Everywhere henceforth: change CP-violating to CPV and Standard Model to SM.}

%As indicated above, the two CP-violating Standard-Model contributions to EDMs are from the weak interaction through the imaginary components of the CKM matrix~\cite{Kobayashi:1973fv}  and from the strong interaction through the parameter $\bar\theta$~\cite{'tHooft:1976up,Jackiw:1976pf,Callan:1976je}. 
%The weak CKM-matrix contributes to both the neutron, at the three-loop level, and the electron, at the four-loop level. The parameter $\bar\theta$ contributes to hadronic systems including the neutron, atoms and molecules. 
As indicated by FIG.~\ref{fg:EDMSubwayMap}, EDMs in experimentally accesible systems arise from CP-violation at a fundamental level that is manifest at several energy or length scales. The Lagrangian for a fundamental theory incorporating SM CKM and  $\bar\theta$ and contributions together with BSM physics can be written
\begin{equation}
\label{eq:LCPV1}
\mathcal{L}_\mathrm{\rm CPV} = \mathcal{L}_\mathrm{CKM}+\mathcal{L}_{\bar\theta}
+\mathcal{L}_\mathrm{BSM}. 
\end{equation}
The general framework that connects this to experiment, Effective Field theory (EFT),  absorbs higher-energy processes into a set of operators that contribute at a scale $\Lambda$ resulting in a set of weak scale, non-renormalizable operators involving only SM fields. The corresponding amplitudes scale as $(v/\Lambda)^{d-4}$, where $d$ is the operator's canonical dimension and $v=246$ GeV is the Higgs vacuum expectation value. 
%The CP-violating Lagrangian incorporating Standard Model CKM and  $\bar\theta$ contibutions together with Beyond-Standard-Model physics can be written
%\begin{equation}
%\label{eq:LCPV1}
%\mathcal{L}_\mathrm{\rm CPV} = \mathcal{L}_\mathrm{CKM}+\mathcal{L}_{\bar\theta}
%+\mathcal{L}_\mathrm{BSM}\ . 
%\end{equation}
%The EFT operators are characterized by their dimension $d$, with their contirbution to $\mathcal{L}_\mathrm{\rm CPV}$ on $\Lambda$ given by $ \left(\frac{1}{\Lambda}\right)^{d-4} $.

The $\bar\theta$ term in $\mathcal{L}_\mathrm{\rm CPV}$ enters at EFT dimension four, while  CKM-generated fermion EDMs 
are dimension five, but elecro-weak $SU(2)\times U(1)$ gauge invariance requires coupling through the Higgs field making these effectively dimension six. % along with four-fermion couplings and the Weinberg three-gluon operator. These dimension-six operators 
%contribute to EDMs proportionally to the inverse-square of $\Lambda$.
%  {\it i.e.} the BSM Lagrangian takes the form
BSM physics enters at dimension six and higher
%; however considering only first generation Standard-Model fermions in addition to Standard-Model bosonic fields, it is sufficient to consider only $d=6$,
 {\it i.e},
\begin{equation}
\mathcal{L}_\mathrm{BSM}\rightarrow \mathcal{L}_\mathrm{\rm CPV}^\mathrm{eff} = \sum_{k,d}\ \alpha_k^{(d)} \left(\frac{1}{\Lambda}\right)^{d-4} \mathcal{O}_k^{(d)},
\label{eq:bsmeff}
\end{equation}
where $\alpha_k^{(d)}$ is the Wilson coefficient for each operator $\mathcal{O}_k^{(d)}$,  $k$ denotes all operators for a given $d$ that are invariant under both $SU(2)$ and $U(1)$, and the operators contain only SM fields. However when considering only first generation SM fermions and SM bosons, it is sufficient to consider only $d=6$.
%In this review, we will restrict our attention to dimension six operators. 
At this order, the relevant set of operators, {\it i.e.} the \lq\lq CP-violating sources"  listed in Table~\ref{tb:ThirteenParameters},
 include the fermion SU(2)$_L$ and U(1)$_Y$ electroweak dipole operators and the SU(3$)_C$ chromo-electric-dipole operators; a set of four fermion semi-leptonic and non-leptonic operators; a CP-violating three-gluon operator; and a CP-violating fermion-Higgs operator.  After electroweak symmetry-breaking, the  dipole operators induce the elementary fermion EDMs and Chromo-EDMs (CEDMs) as well as analogous fermion couplings to the massive electroweak gauge bosons that are not directly relevant to the experimental observables discussed in this review. The fermion-Higgs operator induces a four-quark CP-violating operator whose transformation properties are distinct from the other four-quark operators listed in Table~\ref{tb:ThirteenParameters}.

%\subsection{Effective Field Theory}
%\label{sec:EFT}
%Effective field theory (EFT) provides a connection across energy scales through a set of effective operators that contribute at a scale $\Lambda$,  which is considered to lie above the weak scale $v=246$ GeV. The $\bar\theta$ Lagrangian enters at dimension four, and fermion EDMs and quark chromo-EDMs are dimeinsion five, but elecro-weak ($SU(2)\times U(1)$ gauge invariance requires the coupling to through the Higgs field making these effectively dimension six along with four-fermion couplings and the Weinberg three-gluon operator. These dimension-six operators contribute to EDMs proportionally to the inverse-square of $\Lambda$,  {\it i.e.} the BSM Lagrangian takes the form
%\begin{equation}
%\mathcal{L}_\mathrm{BSM}^\mathrm{eff}=\frac{1}{\Lambda^2}\sum_i \alpha_i^n
%\mathcal{O}_i^{(6)}+ \cdots
%\end{equation}
%Table~\ref{tb:ThirteenParameters} lists the dimension-six operators described in the following paragraphs.


\begin{table}[t]
\centering \renewcommand{\arraystretch}{1.5}
\begin{tabular}{||c|c|c||}
\hline\hline
%$\mathcal{O}_{e}$ & $(\bar F \sigma^{\mu\nu} f_R) \Phi\, B_{\mu\nu}$ & electron EDM \\
%\hline
$\mathcal{O}_{fW}$ &$(\bar F \sigma^{\mu\nu} f_R) \tau^I \Phi\, W_{\mu\nu}^I$  & fermion SU(2)$_L$  EDM \\
$\mathcal{O}_{fB}$ & $(\bar F \sigma^{\mu\nu} f_R) \Phi\, B_{\mu\nu}$ & fermion U(1)$_Y$  EDM \\
$\mathcal{O}_{uG}$ & $(\bar Q \sigma^{\mu\nu} T^A u_R) \tvp\, G_{\mu\nu}^A$ & u-quark Chromo EDM\\
$\mathcal{O}_{dG}$ & $(\bar Q \sigma^{\mu\nu} T^A d_R) \vp\, G_{\mu\nu}^A$& d-quark Chromo EDM\\
\hline
$Q_{ledq}$ & $(\bar L^j e_R)(\bar d_R Q^j)$ & CP-violating semi-leptonic \\
$Q_{lequ}^{(1)}$ & $(\bar L^j e_R) \epsilon_{jk} (\bar Q^k u_R)$ & \\
$Q_{lequ}^{(3)}$ & $(\bar L^j \sigma_{\mu\nu} e_R) \epsilon_{jk} (\bar Q^k
\sigma^{\mu\nu} u_R)$ &  \\
\hline
$\mathcal{O}_{\wt G}$ & $f^{ABC} \wt G_\mu^{A\nu} G_\nu^{B\rho} G_\rho^{C\mu} $ & CP-violating 3 gluon \\
$Q_{quqd}^{(1)}$ & $(\bar Q^j u_R) \epsilon_{jk} (\bar Q^k d_R)$ & CP-violating four quark\\
$Q_{quqd}^{(8)}$ & $(\bar Q^j T^A u_R) \epsilon_{jk} (\bar Q^k T^A d_R)$ & \\
\hline
$Q_{\varphi ud}$ & $i\left({\tilde\varphi}^\dag D_\mu \varphi\right) {\bar u}_R\gamma^\mu d_R$ & CP-violating  quark-Higgs\\
\hline\hline
\end{tabular}
\caption{Dimension-six P-odd/T-odd operators that induce atomic, hadronic, and nuclear EDMs. Here $\varphi$ is the SM Higgs doublet, $\tvp=i\tau_2\vp^\ast$, and $\Phi=\vp$ ($\tvp$) for $I_3^f< 0$ ($I_3^f>0$). The notation is adapted from~\textcite{Engel:2013lsa}.
\label{tb:ThirteenParameters}}
\end{table}

%Although our interest will be primarily in the BSM origins of the Wilson coefficients $\alpha_k^{(6)}$, it is important to bear in mind that a subset of them may also be generated by the SM CPV interactions. 
%Before discussion the underlying SM and BSM origins in detail, it is useful to consider the structure of the CPV sources in more detail and to define some notation.  

%To make the connection to the low-energy observables consider, for example, 
The second term of the electromagnetic Lagrangian in Eq~\ref{eq:EMLagrangian} describes the EDM interaction for an elementary fermion $f$, which couples left-handed to right-handed fermions. Letting the Wilson coefficient $\alpha_{fV_k}^{(6)} = g_k C_{fV_k}$, where $k=B,\ W,\ G$ labels the Standard-Model  electroweak ($B$ and $W$) and gluon ($G$) gauge fields 
\bea
{\cal L_\mathrm{EDM}}&=&-i\frac{d_f}{2}\bar\Psi \sigma^{\mu\nu}\gamma^5 F_{\mu\nu}\Psi\nonumber\\
&=&\frac{1}{\Lambda^2}(g_B C_{f B}\mathcal{O}_{fB}+2I_3g_W C_{f W}\mathcal{O}_{fW}),\nonumber\\
\eea
%The operators
%while the latter two 
%$Q_{f W}$ and $Q_{f B}$ generate the elementary fermion EDM interactions of Eqn.~(\ref{eq:edmdef}). 
where $d_f$ is the  fermion EDM, which couples to the EM field $A^\mu=B^\mu\cos\theta_W+W_3^\mu\sin\theta_W$ ($\theta_W$ is the SM Weinberg angle), is
 \bea
\label{eq:prdef}
%d_f & = & \frac{ e d_f^\gamma}{\Lambda} = - \frac{\sqrt{2} e}{v}\ 
%\left(\frac{ v}{\Lambda}\right)^2\ 
%\left[ \mathrm{Im}\ C_{f\wt B}+ (2I_3^f)\mathrm{Im}\ C_{f\wt W}\right]\\
\nonumber
d_f  &=&   - \frac{\sqrt{2} e}{v}\ \left(\frac{ v}{\Lambda}\right)^2\ 
(\mathrm{Im}\ C_{f B}+ 2I_3^f \; \mathrm{Im}\ C_{f W})\\
& = & -(1.13\times 10^{-13}\ e\ \mathrm{fm})\  \left(\frac{ v}{\Lambda}\right)^2\  \mathrm{Im} C_{f\gamma}.
\eea
Here 
\be
\mathrm{Im} C_{f\gamma} =\mathrm{Im}\ C_{f B}+ 2I_3^f \; \mathrm{Im}\ C_{f W}
\ee
%C_{f \gamma} \ ,
%\nonumber
%\eea
%where
%\be
%\label{eq:Cfgammadef}
%\mathrm{Im}\ C_{f \gamma} \equiv 
%\mathrm{Im}\ C_{f B}+ 2I_3^f \; \mathrm{Im}\ C_{f W} \ ,
%\ee
and $I_3^f$ is the third component of weak isospin for fermion $f$.  
%Here, we  have expressed $d_f$ and ${\tilde d}_q$ in terms of the  Fermi scale  $1/v$, a  dimensionless ratio involving the BSM scales $\Lambda$ and $v$, and the dimensionless  Wilson coefficients. Expressing these quantities in units of fm one has
%\bea
%{\tilde d_q} & = &  - (1.13 \times 10^{-3}\ \mathrm{fm}) 
%\left(\frac{ v}{\Lambda}\right)^2\ \mathrm{Im} \ C_{q G} \ ,\\
%d_f & = & - (1.13 \times 10^{-3}\ e \, \mathrm{fm})
%\left(\frac{ v}{\Lambda}\right)^2\ \mathrm{Im}\ C_{f \gamma} \ .
%\eea
%Note that in Eqn. \eqref{eq:edmdef} and those that follow it
%$f$ is not a right-handed field
%as in Table \ref{tab:cpvdim6-1}. 
The CP-violating quark-gluon interaction can be written in terms of an analogous chromo-EDM (CEDM) $\tilde d_q$: %{\color{magenta} move figure to model discussion}
%generated by BSM physics would result in a P-odd/T-odd interaction with the gluon field shown by the example in FIG.~\ref{fg:chromoEDM} with couplings to new SUSY particles. The chromo-EDM-gluon Lagrangian is 
\be
\label{eq:cedmdef}
%\mathcal{L}^\mathrm{CEDM} = -i\sum_q\ \frac{g_3 d_q^G}{2\Lambda}\ 
%{\bar q} \sigma^{\mu\nu} T^A\gamma_5 q\ G_{\mu\nu}^A
\mathcal{L}_\mathrm{CEDM} = -i\sum_q\ \frac{g_3 {\tilde d}_q}{2}\ 
{\bar q} \sigma^{\mu\nu} T^A\gamma_5 q\ G_{\mu\nu}^A \ ,
\ee
where $T^A$ ($A=1, \ldots, 8$) are the generators of the QCD color group. Note that $\tilde d_q$ has dimensions of length, because it couples to gluon fields, not EM fields.

%\begin{figure}[tb]
%\hskip -0.6 truein
%\includegraphics[width= 3.8 truein]{CEDM}
%\vskip -0.5 truein
%\caption{\label{fg:chromoEDM} An example of BSM couplings of the chromo-EDM to the gluon field.}
%\end{figure}


%Before proceeding, 
%As discussed in Ref.~\cite{Engel:2013lsa}, it is useful to observe that the 
Due to electroweak gauge invariance, the coefficients of the operators that generate EDMs and CEDMs   ($Q_{q \wt G}$, $Q_{f \wt W}$, $Q_{f \wt B}$) contain explicit factors of the Higgs 
field with Yukawa couplings $Y_f=\sqrt{2} m_f/v$. We can write
$\mathrm{Im}\ C_{{f\gamma }}  \equiv  Y_f\, {\delta}_f $, {\it etc.} so that
\be
 d_f =  -(1.13 \times 10^{-3}\ e\, \mathrm{fm})  \, 
\left(\frac{v}{\Lambda}\right)^2\, Y_f\,  {\delta}_f \  ,
\label{eq:nda1prime}
\ee
and
\be
{\tilde d}_q = -(1.13 \times 10^{-3}\ \mathrm{fm})\,  
\left(\frac{v}{\Lambda}\right)^2\, Y_q\,  {\tilde\delta}_q.
%\nonumber
\label{eq:nda1}
\ee
%Rewriting $d_f$ and $\tilde d_f$ in terms of  $\delta_f$ and  ${\tilde\delta}_q$ is useful when comparing with the Wilson  coefficients of other dimension-six CPV operators. 
In general, we expect $\delta_q\sim \delta_\ell$, thus the up and down-quark EDMs would be comparable, but
 light quark EDM $d_q$ would be roughly an order of magnitude larger than the  electron EDM. As noted earlier, exceptions to this expectation can arise when the Higgs couples to heavy degrees of freedom in the loop graphs that generate quark EDMs and CEDMs. 
%$
%Y_u,\ Y_d \rightarrow Y_q \equiv \frac{\sqrt{2}{\bar m}}{v}
%$
%with ${\bar m}$ being the average light quark mass.}. 
%In what follows, we will therefore quote constraints on both $d_e$ and $\delta_e$ implied
%by results for paramagnetic systems; for implications of the neutron and diamagnetic results for the quark EDMs ($d_q/\delta_q$) and CEDMs (${\tilde d}_q/{\tilde\delta}_q$)
%we refer the reader to Ref.~\cite{Engel:2013lsa}.

Considering only first generation fermions, there are fifteen independent weak-scale coefficients. Translating the electroweak dipole operators into the elementary fermion EDMs and neglecting couplings to massive gauge bosons leads to a set of twelve $d=6$ CP-violating sources -- in addition to the $\bar{\theta}$ parameter -- that induce atomic, hadronic and nuclear EDMs.

\subsection{Low-energy parameters}% Manifestations}
\label{sec:LowEnergyParameters}
\label{sub:lowE}

As indicated in FIG.~\ref{fg:EDMSubwayMap}, the Wilson coefficients are connected to the experimental observables at the hadronic scale, $\Lambda_\mathrm{had}\sim 1$ GeV, through a set of low-energy parameters involving pions, and nucleons in place of quarks and gluons as well as photons and electrons. %In principle, one may probe the EDM of a free charged lepton directly, though in practice, experimental probes of $d_e$ have relied on bound state paramagnetic systems. The CPV sources involving quarks and gluons will induce T- and P-violating (T-odd/P-oddT-odd/P-odd)  hadronic parameters that enter hadronic, nuclear and atomic systems. 
Considering first purely hadronic interactions, the starting point is a T-odd/P-odd (or TVPV) effective, non-relativistic Lagrangian containing pions and nucleons~\cite{Engel:2013lsa}:
\bea
\label{eq:pinn}
\mathcal{L}_{\pi NN}^\mathrm{TVPV}\!\!\! &=&\!\! -2{\bar N} \left(\bar{d}_0+\bar{d}_1\tau_3\right)S_\mu N v_\nu F^{\mu\nu}\nonumber\\
\nonumber
&+&\!\!\! {\bar N}\left[\gpbz{\vec\tau}\cdot{\vec\pi} +\gpbo \pi^0 + \gpbt\, \left(3\tau_3\pi^0-{\vec\tau}\cdot{\vec\pi}\right)\right] N,\nonumber\\
\eea
where $S_\mu$ is the spin of a nucleon $N$ having velocity $v_\nu$, $\vec\tau$ is the isospin operator, and $\vec\pi={\pi^+,\pi^0,\pi^-}$ represents the pion field. Four-nucleon interactions are currently being studied and are not considered in this discussion.
Combinations $\bar d_0+\bar d_1\tau_3=\bar d_0\mp \bar d_1$ correspond to the short-range contributions to the neutron and proton EDMs.
The quark EDMs contribution to the $\bar{d}_{0,1}$ while the quark CEDMs, the three-gluon operator, and the CP-violating four-quark operators (including the operator induced by $Q_{\varphi ud}$) will contribute to both $\bar{d}_{0,1}$ and $\bar g_\pi^{(0),(1),(2)}$. Generally, the sensitivity of the isotensor coupling $\gpbt$ is significantly suppressed compared to that of $\gpbz$ and $\gpbo$.  %, so we will only include the latter two in the following analysis.
The T-odd/P-odd pion-nucleon interactions parameterized by the couplings $\gpbi$, contribute to  nucleon EDMs as well as to  nucleon-nucleon interactions  that generate the Schiff moment. 




The semi-leptonic operators $\mathcal{O}_{\ell e dq}$ and $\mathcal{O}_{\ell equ}^{(1,3)}$ induce effective nucleon spin-independent (NSID)  and nuclear spin-dependent electron-nucleon interactions, described by the scalar ($S$) and tensor ($T$) interactions:
\bea
\label{eq:NSID}
\mathcal{L}_{S} & = & -\frac{G_F}{\sqrt{2}}
{\bar e}i\gamma_5 e\ {\bar N} \left[ C_S^{(0)} +C_S^{(1)}\tau_3\right] N
%+8\, {\bar e} \sigma_{\mu\nu} e\ v^\nu
%{\bar N} \left[ C_T^{(0)} +C_T^{(1)}\tau_3\right] S^\mu N\Bigr\}
%+\cdots \ ,
\eea
\bea
\label{eq:NSD}
\mathcal{L}_\mathrm{T} & = & \frac{8 G_F}{\sqrt{2}}
%{\bar e}i\gamma_5 e\ {\bar N} \left[ C_S^{(0)} +C_S^{(1)}\tau_3\right] N
{\bar e} \sigma^{\mu\nu} e\ v_\nu
{\bar N} \left[ C_T^{(0)} +C_T^{(1)}\tau_3\right] S_\mu N%\Bigr\}
+\cdots \ ,\nonumber\\
\eea
where the Dirac matrices act on the electron wave function, $G_F$ is the Fermi constant, and $N$ is a nucleon spinor; the sum over all nucleons is implied,  and where the $+\cdots$ indicate sub-leading contributions arising from the electron-scalar--nucleon-pseudoscalar interaction. 
 
The coefficients $C_{S,T}^{(0,1)}$ can be expressed in terms of the underlying semileptonic operator coefficients and the nucleon scalar and tensor form factors:
\begin{eqnarray}
\nonumber
C_S^{(0)}  &=& -g_S^{(0)}\, \left(\frac{v}{\Lambda}\right)^2\,  
\mathrm{Im}\ C_{eq}^{(-)}\\
\nonumber
C_S^{(1)}  &=&  g_S^{(1)}\, \left(\frac{v}{\Lambda}\right)^2\,  
\mathrm{Im}\ C_{eq}^{(+)}  \\
\nonumber
C_T^{(0)} & = & -g_T^{(0)}\, \left(\frac{v}{\Lambda}\right)^2\,  
\mathrm{Im}\ C_{\ell e qu}^{(3)}\\
C_T^{(1)} & = & -g_T^{(1)}\, \left(\frac{v}{\Lambda}\right)^2\,  
\mathrm{Im}\ C_{\ell e qu}^{(3)},
\label{eq:CSi}
\end{eqnarray}
where 
\be
\label{eq:Ceqdef}
C_{eq}^{(\pm)}= C_{\ell e dq} \pm C_{\ell e q u}^{(1)} \ \ \ .
\ee 
The isoscalar and isovector  form factors $g_\Gamma^{(0,1)}$ are given by
\bea
\label{eq:ffdef}
\frac{1}{2} \bra{N} \left[{\bar u} \Gamma u + {\bar d}\Gamma d\right]\ket{N} 
&\equiv& g_\Gamma^{(0)} {\bar \psi_N} \Gamma \psi_N\ ,\\
\frac{1}{2} \bra{N} \left[{\bar u} \Gamma u - {\bar d}\Gamma d\right]\ket{N} 
&\equiv& g_\Gamma^{(1)} {\bar \psi_N} \Gamma \tau_3 \psi_N\ ,
\eea
where $\Gamma = 1$ for $S$  and $\sigma_{\mu\nu}$ for $T$~\cite{Engel:2013lsa}.

%The remaining operators in Table \ref{tb:ThirteenParameters} include $\mathcal{O}_{\wt G}$, the CP-violating Weinberg three-gluon operator (sometimes called the gluon chromo EDM); a set of three semileptonic operators $Q_{ledq}$, $Q_{lequ}^{(1)}$, $Q_{lequ}^{(3)}$ that give rise to P-odd/T-odd electron-nucleus interactions in atoms and molecules; and two four-quark operators $Q_{quqd}^{(1)}$ and $Q_{quqd}^{(8)}$. The quark-Higgs interaction
%combined with exchange of a $W$ boson results in the effective interaction ${\bar u}_R\gamma^\mu d_R {\bar d}_L\gamma_\mu u_R$, which naturally arises in left-right symmetric models  {\em via} mixing of the left- and right-handed $W$ bosons and through the rotations of the left- and right-handed quarks from the weak to mass eigenstate basis. 

%Generally, the Wilson coefficients reflect distinct sources of CP violation, and relationships among them depend on the Lagrangian at high energy scale. We discuss several examples below. Before doing so, however, it is useful to summarize the \lq\lq flow" of the fundamental CPV parameters down to the low-energy systems of experimental interest, illustrated in FIG.~\ref{fg:EDMSubwayMap}. Here, we  show schematically the relationship of physics at various scales to the observable EDMs of atoms, molecules and nucleons. At the top is the intrinsic-short-range physics thtincludes the CKM and $\bar\theta$ terms along with CP-violation in BSM models such as Supersymmetry, Left-Right Symmetry etc. Toward the bottom are the actual experimentally accessible physical systems, {\em i.e.} nucleons, nuclei, atoms and molecules, and solid-state systems. The specific models will induce a subset of the dimension-six weak scale parameters $C_k$. At lower-energy, the latter will first become manifest in the hadronic scale parameters that ultimately generate the EDMs of experimentally accessible systems.

%Several specialties of theory connect the bottom to the top with a connection through effective field theory, which provides a set of fundamental operators describing physics below some energy scale, in our case the electroweak scale. Hadronic theory connects these operators to nucleon EDMs, nuclear physics calculations are required to make the connection to the Schiff moment, etc. and  atomic and molecular theory provide the connection from the electron EDM, the  Schiff moment and P-odd/T-odd electron-nucleus (e-N) forces to the observables in atoms and molecules.  



%\begin{figure*}[tb]
%\includegraphics[width = 6.5 truein]{TopDownSubwayMapColor}%{TopDownSubwayMapBW}
%\caption{\label{fg:EDMSubwayMap} Illustration of the connections from a fundamental theory to an EDM in a measurable   system.}
%\end{figure*}


%From a theoretical perspective, specific models are required to relate the different Wilson coefficients; however there are specific symmetries that can be invoked. For examle a non-zero four-fermion interaction would generate an electron edm in higher order (two loop).
%(Table?).

\subsection{EDMs in the Standard Model}

CP violation in the CKM matrix leads to non-vanishing coefficients of the $d=6$ CP-violating sources at the multi-loop level. The primary theoretical interest has been the elementary fermion EDMs. %Recall that the CKM matrix results from the rotation of the left-handed quark weak (or flavor) eigenstates into the mass eigenstates and characterizes the mismatch between the rotations of the up- and down-type quarks that appears in the charged current weak interaction:
%The CKM matrix represents the mixing of left-handed strong-interaction eigenstates $\begin{bmatrix}u\\d\end{bmatrix}_L$,  
%$\begin{bmatrix}c\\s\end{bmatrix}_L$ and  $\begin{bmatrix}t\\d\end{bmatrix}_L$ by the charged-current weak interaction. This is reflected  in the Lagrangian 
The CKM Lagrangian for mixing of left-handed down-type quarks and up-type quarks is
\begin{equation}
\label{eq:ccint}
\mathcal{L}_\mathrm{CKM} = -\frac{ig_2}{\sqrt{2}}\sum_{p,q} V^{pq} {\bar
U}_L^p \diracslash{W}^+ D_L^q +\mathrm{h.c.}\ .
\end{equation}
Here $g_2$ is the weak coupling constant,  $\diracslash{W}^+=\gamma^\mu W_\mu^{+}$  is the charged $W$-boson coupling,
 $U_L^p=u,c,t$ and $D_L^p=d,s,b$ are a generation-$p$ left-handed up-type
and down-type quark fields, and $V^{pq}$ denotes the element of the CKM matrix.
The constraints from unitarity and quark-field rephasing for the three quark generations allow four free parameters: three magnitudes and a CP-violating phase.  Writing
%It is convenient to use the Wolfenstein parameterization, accurate to $\mathcal{O}(\lambda^4)$:
\begin{eqnarray}
V_\mathrm{CKM}&= &
\begin{bmatrix}
V^{ud} & V^{us} &V^{ub}\\
V^{cd} & V^{cs} & V^{cb}\\
V^{td} & V^{ts} & V^{tb}\\
\end{bmatrix}
%\\
%& &
%\begin{bmatrix}
%1-\lambda^2/2 & \lambda  & A\lambda^3(\rho-i\eta)\\
%-\lambda & 1-\lambda^2/2 & A\lambda^2\\
%A\lambda^3(1-\rho-i\eta) & -A\lambda^2 & 1\\
%\end{bmatrix}\nonumber\\
\end{eqnarray}
%Note that the only complex entries in this approximation enter in the mixing of first and third generations  $V_{td}$,  and  $V_{ub}$ (imaginary components of $V_{cs}$ and $V_{cb}$ enter at $\mathcal{O}(\lambda^4)$). 
%The standard global fits to a large array of flavor-mixing weak decays and CP-violation in the $K$ and $B$ meson systems, assuming unitarity of the CKM matrix, give
%\begin{eqnarray}
%&\lambda&=0.22548^{+0.00068}_{-0.00034},\nonumber\\
%&A&=0.810^{+0.018}_{-0.024},\nonumber\\
%&\bar\rho&=0.145^{+0.013}_{-0.007},\nonumber\\
%&\bar\eta&=0.343^{+0.011}_{-0.012},
%\end{eqnarray}
%where $\bar\rho=\rho(1-\lambda^2/2+...)$, and $\bar\eta=\eta(1-\lambda^2/2+...)$ \cite{rf:PDG2014,CKMFitter}. 
%CKM CP violaiton is proportional to the T-odd 
the CP violating effects are proportional to the Jarlskog invariant 
\be
\bar\delta=\mathrm{Im}(V_{us}V_{cs}^*V_{cb}V_{ub}^*).
\ee
%Note that  $\frac{\eta}{\rho}$,  is of order one indicating a significant imaginary components of $V_{ub}$ and $V_{td}$.
%Note that in the, $\bar\delta$ is sometimes expressed to leading order in the three CP-conserving mixing angles and one CP-violating phase in the CKM matrix as $\bar\delta\approx $.
 A global analysis of experimental determinations of CP-violating observables in the neutral kaon and $B$-meson systems gives $\bar\delta\approx 5\times 10^{-5}$ 
%, and we refer the reader to Refs.
~\cite{rf:PDG2014,CKMFitter}. 
% for detailed reviews. \red{DOESN'T SEEM NECESSARY: The remaining entries in $V_\mathrm{CKM}$ are characterized by three mixing angles $\theta_j$ ($j=1,2,3$).}

%As noted above, elementary fermion EDMs generated by CP violation in the CKM matrix are suppressed because they involve the mixing of first and third generation quarks and require three loops for the neutron (FIG.~\ref{fg:NeutronEDMVertex}) and four loops for the electron. . 

The electron EDM arises at four-loop level and has been estimated  by~\textcite{rf:NgNg1996} to be 
%{\color{magenta} need to see paper to check formula...don't have access}
 \be
 d_e^{\rm CKM}\approx \frac{eG_F}{\pi^2} \left (\frac{\alpha_{EM}}{2\pi}\right )^3 m_e \bar\delta\approx 10^{-38}\ {\rm \ecm}.
 \ee
 For the neutron, the contribution of the valence $u$- and $d$-quarks has been computed by~\textcite{Czarnecki:1997bu} to be 
 %~\cite{rf:SMNeutronEDM1,rf:SMNeutronEDM2}
 \ba
d_d&\approx& \frac{m_dm_c^2\alpha_S G_F^2 \bar\delta}{108\pi^5} \times f\left (\ln\frac {m_b^2}{m_c^2},\ln\frac{m_W^2}{m_b^2}\right )\nonumber\\
%[(\ln(\frac {m_b^2}{m_c^2})^2-2\ln(\frac{m_b^2}{m_c^2})+\frac{\pi^2}{3})\ln\frac{m_W^2}{m_b^2}+\frac{5}{8}(\ln(\frac{m_b^2}{m_c^2})^2\nonumber\\
% &-&(\frac{355}{36}+\frac{2}{3}\pi^2)\ln(\frac{m_b^2}{m_c^2})-\frac{1231}{108}+\frac{7}{8}\pi^2+8\zeta_3]\nonumber\\
&\approx& -0.7\times 10^{-34} \text{\ecm~}
\ea
\ba
d_u&\approx& \frac{m_um_s^2\alpha_S G_F^2 \bar\delta}{216\pi^5}\times f\left (\ln\frac {m_b^2}{m_s^2},\ln\frac{m_c^2}{m_s^2},\ln\frac{m_b^2}{m_c^2},\ln\frac{m_W^2}{m_b^2} \right )\nonumber\\
%&\times&[(-\ln(\frac {m_b^2}{m_s^2})^2+2\ln(\frac{m_b^2}{m_s^2})+2-\frac{2\pi^2}{3})\ln\frac{m_W^2}{m_b^2}\nonumber\\
% &-&\ln(\frac {m_b^2}{m_c^2})\ln(\frac {m_c^2}{m_s^2})^2+2\ln(\frac {m_b^2}{m_c^2})\ln(\frac {m_c^2}{m_s^2})\nonumber\\
 %&-&\frac{5}{8}(\ln(\frac{m_b^2}{m_s^2})^2-(\frac{259}{36}+\frac{1}{3}\pi^2)\ln(\frac{m_b^2}{m_s^2})\nonumber\\
 %&+&(\frac{140}{9}+\pi^2)\ln(\frac{m_c^2}{m_s^2})-\frac{121}{108}+\frac{41}{36}-4\zeta_3]\nonumber\\
 &\approx& -0.15\times 10^{-34} \text{\ecm},
\ea
where the $f's$ are functions of the natural logarithms of the mass ratios $\frac {m_b^2}{m_c^2}$, {\it etc}.
%Here $\alpha_S=0.2$, $m_u$= 5 MeV, $m_d=$ 10 MeV, $m_c=$ 1.5 GeV, $m_s=0.2$ GeV, $m_b=4.5$ GeV, $G_F\approx 10^{-5} m_p^{-2}$ and the Rieman zeta function is $\zeta_3=1.202...$, so that 
The valence-quark contribution to the neutron EDM is
\be
 d_n^{\rm CKM}=\frac{4}{3} d_d-\frac{1}{3}d_u
 \approx -0.9\times 10^{-34}\ \text{\rm \ecm}.
 \ee

A significantly larger contribution to $d_n$ (and $d_p$) arises from \lq\lq long distance" meson-exchange contributions, for example that shown in FIG.~\ref{fg:MDMEDMStrange}, 
%{\color{magenta} need to generate a new figure - chiral loops with strange mesons},
 where the CP-violating $\Delta S=1$ hadronic vertices are generated by the Penguin process of FIG.~\ref{fg:NeutronEDMVertex}, while the CP-conserving $\Delta S=1$ couplings arise from the tree-level strangeness-changing charged current interaction  (Eqn.~\ref{eq:ccint}). A full compilation of diagrams and corresponding results for SM neutron and proton EDMs based on heavy baryon chiral perturbation theory  is provided by~\textcite{Seng:2014lea}:
\be
|d_{n,p}|\approx (1-6)\times 10^{-32}\ \text{\rm \ecm}
%1\times 10^{-32}\  e-\mathrm{cm} \leq \vert d_{p,n}\vert \leq 6\times 10^{-32}\  \mathrm{\ecm}\ \ \ ,
\ee
where the range reflects the present uncertainty in various low energy constants that enter the heavy baryon effective Lagrangian and an estimate of the  higher order terms neglected in the heavy baryon expansion.

\begin{figure}
\vskip -0.5 truein
\centerline{\includegraphics[width = 4.25 truein]{MDMEDMStrange}}
\vskip -0.75 truein
\caption{\label{fg:MDMEDMStrange} Representative chiral loop contribution to the neutron EDM arising from SM CKM CP-violation. The $\otimes$ indicates a CP-violating       $\Delta S=1$ vertex such as that shown in FIG.~\ref{fg:NeutronEDMVertex}, while the \textbullet\ corresponds to a CP-conserving $\Delta S=1$ interaction. Adapted from \textcite{Pospelov:2005pr}. }
\end{figure}


The CKM contribution enters the Schiff moment through the P-odd/T-odd $NN$ interaction mediated by kaon exchange~\cite{Donoghue:1987dd}.  \textcite{flambaum86} present an estimate of $S$ using the one-body effective P-odd/T-odd potential for a valence nucleon 
\be
{\hat W} = \frac{G_F}{\sqrt{2}}\ \frac{\eta_a}{2 m_N}\ {\vec\sigma}_a\cdot{\vec\nabla}\ \rho_A({\vec r})\ \ \ ,
\label{eq:dndp}\ee
where $\rho_A({\vec r})$ is the nuclear density and, for valence nucleon $a=n$ or $p$, the P-odd/T-odd coupling strength is
\bea
\eta_n &=&\left(N/A \right)\, \eta_{nn}+\ \left(Z/A\right)\, \eta_{np},\nonumber\\
\eta_p &=&\left(N/A \right)\, \eta_{pn}+\ \left(Z/A\right)\, \eta_{pp}.%\red{CHECK MINUS SIGN (-N/A)$\eta_{ap}$}\\
\label{eq:etanetap}\eea
In the SM the $\eta_{a}$'s are proportional to 
$G_F {\bar\delta}$. 

For  $^{199}$Hg EDM, which has an unpaired neutron, the resulting SM estimate for the  Schiff moment and atomic EDM  are 
\ba
S(^{199}\mathrm{Hg})&\approx& -1.4\times 10^{-8}\ \eta_{np}\ e\ \mathrm{fm}^3\nonumber\\
d_A(^{199}\mathrm{Hg})&=& 3.9\times 10^{-25} \eta_{np}\ e\ \mathrm{cm},
\ea
where we have used $d_A(^{199}\mathrm{Hg})/S = -2.8\times 10^{-17}$cm/fm$^{3}$, which is given in Table~\ref{tb:SchiffCoef}. 
%(Note that in the literature, $\bar\delta$ is sometimes expressed to leading order in the three CP-conserving mixing angles and CP-violating phase in the CKM matrix.)
\textcite{Donoghue:1987dd} corrected an earlier computation of $\eta_{np}$ by properly taking into account the constraints from chiral symmetry resulting in  $|\eta_{np}| \lesssim 10^{-9}$ and
\be
%|
|d_A(^{199}\mathrm{Hg})^\mathrm{CKM}|
%|
 \lesssim 4\times 10^{-34}\ \ e\ \mathrm{cm}.
\ee


The EDMs of unpaired nucleons also contribute to the Schiff moment and atomic EDM. For $^{199}$Hg the unpaired neutron is dominant~\cite{dmitriev03}, and this contribrution can be estimated using the SM estimate for $d_n$ as
\be
d_A(^{199}\mathrm{Hg})^\mathrm{CKM(\it n)} \approx 4\times 10^{-4} d_n, 
\ee
resulting in
\be
% |
|d_A(^{199}\mathrm{Hg})^\mathrm{CKM(\it n)}|
%|
 \lesssim  2.4\times 10^{-35}\ \ecm.
 \ee
% \red {which is significantly smaller than the pion-nucleon contribution.}



CP violation in the strong-interaction arises from the term in the QCD
Lagrangian formed by gluon field $G_{\mu\nu}$ combined with its dual  ${\tilde G}_{\mu\nu}
=\epsilon_{\mu\nu\alpha\beta}G^{\alpha\beta}/2$:
%\be
%{\tilde G}_{\mu\nu} =
%\frac{1}{2}\epsilon_{\mu\nu\alpha\beta}G^{\alpha\beta} 
%\ee
\begin{equation}
\mathcal{L}_{\bar\theta} =-\frac{\alpha_S}{16\pi^2} {\bar\theta} \,
\mathrm{Tr}\left(G^{\mu\nu}{\tilde G}_{\mu\nu}\right) \ ,
\label{thetaterm}
 \end{equation} 
where $\alpha_S$ is the strong coupling constant.\footnote{Following \cite{Grzadkowski:2010es}, $\epsilon_{0123}=1$ . This sign convention is opposite that used by~\textcite{Pospelov:2005pr} and elsewhere.  Consequently, $\mathcal{L}_{\bar\theta}$ carries an overall $-1$ compared to what frequently appears in the literature.} 
%This interaction will contribute to the neutron and proton EDM directly as well as inducing a nuclear Schiff moment through the T-odd/P-odd (isospin-zero) pion-nucleon coupling~\cite{Pospelov:1999ha,Crewther:1979pi,Shindler:2015aqa}.  %Chiral symmetry considerations imply that the effect on $\gpbo$ and $\gpbt$ is suppressed by an additional power of $m_\pi^2$ while $\gpbt$ is further reduced by the presence of isospin breaking. 
This will contribute to the neutron and proton EDM directly as well as induce a nuclear Schiff moment through the T-odd/P-odd (isospin-zero) pion-nucleon coupling~\cite{Pospelov:1999ha,Crewther:1979pi,rf:Crewther1979-erratum,Shindler:2015aqa}. For the neutron, the results fall in the range
\be
d_n^{\bar\theta}\approx -(0.9-1.2) \times 10^{-16}\bar\theta\ \text{\ecm}.
\ee
Recently~\textcite{Abramczyk:2017oxr} have observed the need to apply a correction to lattice QCD computations of the $d_n^{\bar\theta}$.


Thus experimental constraints on EDMs in hadronic systems can be used to set an upper bound on  ${\bar\theta}$. Assuming this interaction is the sole source of CP-violation, and neglecting uncertainties associated with the hadronic and nuclear physics, limits from  $d_n$ or from $d_A(^{199}\mathrm{Hg})$ imply  ${\bar\theta}\lessapprox 10^{-10}$. As we discuss in Sec.~\ref{sec:GlobalAnalysis}, allowing for multiple sources of CP violation can weaken this upper bound considerably, but the resulting constraint is nonetheless severe:   $\bar\theta\lessapprox 10^{-6}$. Either way, the tiny value allowed for a non-vanishing ${\bar\theta}$ parameter gives rise to the``strong CP problem." This may be addressed by the axion solution, which postulates an axion field $a(x)$  that couples to gluons with the  Lagrangian~\cite{Peccei:1977hh,Peccei:1977ur}
\begin{equation}
\mathcal{L}_{a} =\frac{1}{2}\partial^\mu a\partial_\mu a -V(a)- \frac{a(x)}{f_a}\frac{\alpha_S}{8\pi} G^{\mu\nu}{\tilde G}_{\mu\nu} \ .
\label{axionLagranian}
 \end{equation} 
The first term is the kinetic energy, $V(a)$ is the axion potential, the third term is the axion-gluon coupling, and $f_a$ the axion decay constant, which is analogous to the pion decay constant. The ground state is the minimum of the axion potential, which shifts the value of $\bar\theta\rightarrow \bar\theta+\frac{\langle a \rangle}{f_a}$ and could lead to cancellations that suppress $\bar\theta$.


%\begin{bmatrix}
%0.97428 & 0.2253  & 0.00347\\
%0.2252 & 0.97345 & 0.0410\\
%0.00862 & 0.0403 & 0.99915\\
%\end{bmatrix}

Neutrino masses established by neutrino oscillations give rise to a $3\times3$ neutrino-mixing matrix with a single CP-violating phase analogous to the CKM phase. If neutrinos are Majorana particles,  two-loop contributions to $d_e$ are possible~\cite{rf:NgNg1996}. However, this turns out to make a small contribution unless the neutrino masses are very specifically tuned~\cite{Archambault:2004td}


\subsection{Beyond-Standard-Model Physics}

Observational and theoretical motivations for  Beyond-Standard-Model physics include the need to explain dark matter, non-vanishing neutrino masses, the observed matter-antimatter asymmetry, and considerations of naturalness, which require a mechanism to solve the \lq\lq hierarchy problem" associated with loop corrections to weak-scale physics. 
%(Scenarios introduced to solve the hierarchy problem suggest the existence of new particles with masses $\lessapprox 1$ TeV.) 
In general, BSM scenarios that address these issues
provide new mechanisms of CP-violation that also generate EDMs. 
%\red {The connections (CONNECITONS TO WHAT?) are, perhaps, clearest in the case of baryogenesis (see Section \ref{sec:Introduction}) and the strong CP problem, as the axion also provides a viable dark matter candidate. - SOMETING IS DISCONNECTED HERE} 
% a scale that is currently probed by LHC and EDM searches (assuming $|\sin\phi_\mathrm{\rm CPV}|$ is maximal). 
Here, we discuss the EDM implications of a few representative BSM scenarios of current interest: supersymmetry (SUSY), left-right symmetric models, and extended Higgs sectors.

%On the one hand, suppressing strong CP violation via the axion could also solve the dark matter problem - the axion is the dark-matter; on the other hand extensions to the Standard Model that intoduce new couplings of Standard Model particles and new symmetries  e.g. a heavy right-handed $W$ (Left-Right symmetry) and SUSY are strongly motivated BSM scenarios, which would enter at a mass scale that is currently experimentally bounded by collider experiments  $\Lambda> (1-10)$ TeV. As noted in Sec.~\ref{sec:Reach}, this mass scale is already probed at one loop by EDMs provided the CP-violating phases are of order unity, or conversly, EDMs are beginnning to constrain these phases.

SUSY introduces symmetry between fermions and bosons, postulating an extra Higgs doublet and a set of new particles  -  ``superpartners'' of the SM particles 
%fermions (quarks and leptons) and gauge bosons ($W^\pm,\ Z^0$ and gluons) 
 called squarks, sleptons and gauginos. With this spectrum of new particles come new couplings and, most importantly, new CP-violating phases.
Though there is currently no direct experimental evidence for SUSY or SUSY particles, the theory is well motivated  by providing a mechanism for solving the hierarchy problem, unifying the gauge couplings, and by providing the new particles as potential dark matter candidates. In the MSSM minimal supersymmetric
extension, there exist 40 additional CP-violating phases, %beyond the SM CKM phase, 
a subset of which can induce EDMs at the one-loop level. Representative one-loop contributions to the elementary fermion EDMs and quark CEDMs are shown in Figs.~\ref{fg:GenericSUSYEDM} and \ref{fg:chromoEDM}, respectively. In each case, the external gauge boson can couple to any internal superpartner carrying the appropriate charge (electric charge for the fermion EDM or color for the chromoEDM). 


%To obtain a sense of the magnitudes of these one-loop contributions,
 It is useful to adopt several simplifying assumptions: 
 \begin{enumerate}[i.]
\item there is a single mass scale $M_\mathrm{SUSY}$ common to all superpartners;%\red{carry} a common mass scale, $M_\mathrm{SUSY}$; 
\item there is a common relative phase $\phi_\mu$ between the supersymmetric Higgs/Higgsino mass parameter $\mu$ and the three SUSY-breaking gaugino masses, $M_j$ ($j=1,2,3$); 
\item the SUSY-breaking trilinear interactions involving scalar fermions and the  Higgs have a common phase, $\phi_A$. 
\end{enumerate}
The resulting one-loop EDMs
 and CEDMs 
%are can then be expressed in 
following from Eqns.~(\ref{eq:nda1prime}-\ref{eq:nda1}) with $\Lambda\to M_\mathrm{SUSY}$, $Y_f$  the dimensionless Yukawa coupling for the fermion of interest, %and the $\sin\phi_\mathrm{\rm CPV} F$ functions given by {\color{magenta} need to check the conversion, including sign} \cite{Pospelov:2005pr} {\color{magenta} see also review with Shufang; also note the loop geometric factors need to be put in}
% only six physical CP phases can generate EDMSs through generic couplings of a SUSY fermion $\chi$ (chargino, gaugino or neutralino) and boson ($\bar f^\prime$) (squark or slepton) as  shown   in Fig~\ref{fg:GenericSUSYEDM}, where the external photon can couple to either $\chi$ or $\bar f^\prime$. For example assuming degenerate $\chi$ and $\bar f^\prime$ masses and only two CP phases
%$\theta_\mu$ (the phase of the higgsino mass), and $\theta_A$ (the universal phase of the trilinear
%couplings)  the electron EDM, quark EDM and quark chromo-EDM are (see Sec.~\ref{sec:EFT} and FIG.~\ref{fg:chromoEDM} are~\cite{Pospelov:2005pr} 
%\begin{eqnarray}
%\nonumber
%d_e & \propto &  \frac{1}{\sqrt{2}}\left[\frac{g_1^2}{12} \sin\phi_A + \left(\frac{5g_2^2}{24} + \frac{g_1^2}{24} \right) \sin\phi_\mu \tan\beta\right]\\
%d_q & \propto &   \frac{\sqrt{2}g_3^2}{9} \Big( \sin\phi_\mu R_q - \sin\phi_A \Big) +\cdots \nonumber \\
%\tilde d_q & \propto &  \frac{5g_3^2}{18\sqrt{2}} \Big( \sin\phi_\mu R_q - \sin\phi_A \Big)+\cdots\ \ \ .
%\label{eq:MinimalSUSYEDMs}
%\end{eqnarray}
%\begin{eqnarray}
% &d_e&\approx {e \kappa_e}[\frac{g_1^2}{12} \sin\theta_A + \left(\frac{5g_2^2}{24} + \frac{g_1^2}{24} \right) \sin\theta_\mu \tan\beta] ~, 
% \nonumber\\
% & &\\
% &d_q& \approx {e_q \kappa_q} \frac{2g_3^2}{9} \Big( \sin\theta_\mu R_q - \sin\theta_A \Big)  ~, \\
% & \tilde d_q& {\kappa_q}\frac{5g_3^2}{18} \Big( \sin\theta_\mu R_q - \sin\theta_A \Big)  ~,
%\end{eqnarray}
%
%red{CAN WE PUT PREFACTORS IN - it's nice to see the $1/M_{SUSY}2$}
are \cite{Pospelov:2005pr,RamseyMusolf:2006vr}
\begin{eqnarray}
\nonumber
\delta_e & =&  -\frac{q\kappa_e}{\sqrt{32\pi^2}}\left[\frac{g_1^2}{12} \sin\phi_A + \left(\frac{5g_2^2}{24} + \frac{g_1^2}{24} \right) \sin\phi_\mu \tan\beta\right]\\
\delta_q & =&  -\frac{q_f\kappa_f}{\sqrt{32\pi^2}}\left[ \frac{\sqrt{2}g_3^2}{9} \Big( \sin\phi_\mu R_q - \sin\phi_A \Big) +\cdots\right] \nonumber \\
\tilde \delta_q & =& -\frac{\kappa_f}{\sqrt{32\pi^2}}\left[ \frac{5g_3^2}{18\sqrt{2}} \Big( \sin\phi_\mu R_q - \sin\phi_A \Big)+\cdots\right]\ ,\nonumber \\ 
\label{eq:MinimalSUSYEDMs}
\end{eqnarray} 
where we have followed the opposite sign convention for the trilinear phase $\phi_A$ compared to~\textcite{RamseyMusolf:2006vr}.
In Eqn.~\ref{eq:MinimalSUSYEDMs}, $f$ refers to the fermion (electron, $u$ and $d$ quark), $q_f$ is the fermion charge ($1$, 2/3, and -1/3, respectively for $e$, $u$, and $d$), $m_f$ the fermion mass, and $\kappa_f=\frac{m_f}{16\pi^2\Lambda}$. Also  $g_{1,2,3}$ are the gauge couplings, $\tan\beta=v_u/v_d$ is the ratio of the vacuum expectation values of the two Higgs doublets, the \lq\lq $+\cdots$" indicate  contributions from loops involving electroweak gauginos, 
%
%\begin{equation}
%     \kappa_i \approx \frac{m_i}{M_{\rm SUSY}^2}  \times 1.3 \times 10^{-19}\, \mathrm{TeV-cm}.
%\end{equation}
$R_d = \tan\beta$, and $R_{u} = \cot\beta$ for down quarks and up quarks, respectively. 
%\red{Translating the quark EDM and CEDM results into the neutron EDM and the $^{199}$Hg Schiff moment provides bounds on $(\phi_\mu,\phi_A)$ from the experimental results for $d_e$ as obtained from ThO, $d_n$, and $d_A(^{199}\mathrm{Hg})$.}  
%\red{THE FOLLOWING  INCLUDES A LOT OF INTERPRETATION - REORGANIZE??? - on the other hand it does not fit into the ``General Framework"}
%\red{IS THIS OK HERE? Turning to the 4-fermion and 3-gluon operators: SUSY scenarios with large $\tan\beta$ can generate EDMs from CP- violating 4-fermion operators~\cite{Lebedev:2002ne,Demir:2003js}. The Weinberg three-gluon operator  receives contributions at the two-loop level from squark-gluino loops, and at the three-loop level from diagrams involving the Higgs bosons. For  SUSY with TeV-scale masses, the EDMs are typically small.}

Turning to the 4-fermion and 3-gluon operators: SUSY scenarios with large $\tan\beta$ can generate EDMs from CP- violating 4-fermion operators~\cite{Lebedev:2002ne,Demir:2003js}.
The Weinberg three-gluon operator  receives contributions at the two-loop level from squark-gluino
loops and at the three-loop level from diagrams involving the Higgs bosons. 
Na\"ively  $\sin\phi_{\mu}$ and $\sin\phi_{A}$ are expected to be $\mathcal{O}(1)$, and  electroweak baryogenesis typically requires larger phases and at least a subset  of the superpartner masses to be well below the  TeV scale~\cite{Morrissey:2012db}. 
%\textcite{Pospelov:2005pr} have derived constraints on $\phi_\mu$ and $\phi_A$ based on experimental results from the neutron EDM, $^{199}$Hg and $d_e$ from Tl for $M_\mathrm{SUSY}=NNN$.  
%The resulting constraints in shown in FIG.~{\color{magenta} need figure from Ritz} for $M_\mathrm{SUSY} = NNN$. 
%Note that these results have been obtained using older values for the hadronic and nuclear matrix elements; updated matrix elements are provided by~\textcite{Engel:2013lsa} {\color{magenta} also include new lattice results}. Nonetheless, they indicate the tight constraints on sub-TeV superpartner masses implied by the current experimental EDM results. 
%We note that  $d_n$ and $d_A(^{199}\mathrm{Hg})$ limits do not constrain the individual phases to be small in themselves, but could arise from cancellation contributions, as observed by \textcite{Ibrahim:1998je} {\color{magenta} check ref}; however the relatively strong sensitivity of $d_e$ to $\phi_\mu$ makes this \lq\lq cancellation scenario" somewhat less plausible. \textcite{Li:2010ax} discuss EDM constraints that do not rely on this cancellation scenario or make the phase universality assumption.

%An analysis of EDM results from $d_e$, $d_n$ and $d_A$($^{199}$Hg) by~\textcite{Pospelov:2005pr} \red{assuming $M_{\rm SUSY}\sim NNN$ TeV finds that $\phi_\mu$ and $\phi_A$ are $<<10^{-NNN}|$, DO WE WANT TO SHOW P\&R UPDATED GRAPH???} {\it i.e.} much less than na\"ive expectations, leading to the so-called SUSY-CP problem \cite{Dimopoulos:1995ju}.    \textcite{Ibrahim:1998je}  {\color{magenta} check ref} pointed out that the individual phases need not be small themselves if there are sufficient cancellation contributions; however the dependence of $d_e$ on $\phi_\mu$ (Eqn.~\ref{eq:MinimalSUSYEDMs}) makes this \lq\lq cancellation scenario" somewhat less plausible.

An analysis of EDM results from $d_e$, $d_n$ and $d_A$($^{199}$Hg) by~\textcite{Pospelov:2005pr} found that $|\phi_{\mu,\, A}|\lessapprox 10^{-2}$ for  $M_{\rm SUSY}=500 $ GeV, {\it i.e.} much less than na\"ive expectations, leading to the so-called SUSY-CP problem \cite{Dimopoulos:1995ju}. %The new experimental limits on these EDMs achieved after 2005 only exacerbate the problem. Not surprisingly, 
A more general analysis that does not rely on the assumption of phase universality yields somewhat relaxed constraints but does not eliminate the SUSY CP-problem\cite{Li:2010ax}.
\textcite{Ibrahim:1998je,Ibrahim:1998je-erratum}  %{\color{magenta} check ref} 
have pointed out that the individual phases need not be small themselves if there are sufficient cancellations. However the dependence of $d_e$ on $\phi_\mu$ in Eqn.~\ref{eq:MinimalSUSYEDMs} makes this \lq\lq cancellation scenario" somewhat less plausible. A discussion within the context of $R$-parity violation is presented by~\textcite{Yamanaka:2014nba}.


~\textcite{Giudice:2004tc,Giudice:2004tc-erratum} and \textcite{Kane:2009kv} have considered a scenario in which the squark and slepton masses are considerably heavier than 1 TeV, while the electroweak gauge bosons and Higgsinos remain relatively light, leading to less constrained phases. Present LHC constraints on squark masses are consistent with this possibility, though the LHC slepton mass bounds are much weaker. In this regime of heavy squarks and sleptons, electroweak baryogenesis proceeds via CP-violating bino and/or wino interactions in the early universe, while EDMs of first generation fermions arise at two-loop order through the chargino-neutralino \lq\lq Barr-Zee diagrams" shown in FIG.~\ref{edm:fig:diagram2}~\cite{Barr:1990vd,Barr:1990vd-erratum}. Applying this scenario to supersymmetric baryogenesis  and relaxing the phase universality assumption leads to the results given in FIG.~\ref{fg:MSSMdedn}, showing that improvements in the sensitivities to $d_e$ and $d_n$ by one and two orders of magnitude, respectively, would probe the entire CP-violating parameter space for MSSM baryogenesis~\cite{Morrissey:2012db,Cirigliano:2006dg,Li:2008ez,Cirigliano:2009yd}. Note that supersymmetric electroweak baryogenesis requires not only sufficient CP-violation, but also a strong first order electroweak phase transition. LHC measurements of Higgs boson properties now render this possibility unlikely in the MSSM~\cite{Curtin:2012aa,Katz:2015uja}; however, ~\textcite{Liebler:2015ddv} suggest that there are regions of parameter space that can satisfy both the observed Higgs mass and baryogenesis. On the other hand,  extensions of the MSSM with gauge singlet superfields presently allow for the needed first-order phase transition. In the context of these ``next-to-minimal'' scenarios,  CP-violating sources could give rise to the observed baryon asymmetry as indicated in FIG.~\ref{fg:MSSMdedn}.

%
%the minimal Standard Model extension (Minimal Supersymmetric Standard Model or MSSM)

\begin{figure}[tb]
%\hskip -0.6 truein
\vskip -0.5 truein
\centerline{\includegraphics[width= 4. truein]{GenericOneLoopSUSYEDM}}
\vskip -0.75 truein
\caption{\label{fg:GenericSUSYEDM} One-loop fermion EDM generated by coupling to SUSY particles, $f^\prime$ and $\chi$. Adapted from~\textcite{Ellis:2008zy}).
}
\end{figure}

\begin{figure}[tb]
%\hskip -0.6 truein
\vskip -0.5 truein
\centerline{\includegraphics[width= 4. truein]{ChromoEDM}}
\vskip -0.75 truein
\caption{\label{fg:chromoEDM} An example of BSM couplings of the CEDM to the gluon field $g$. The crossed-circle indicates interactions that mix the left- and right-handed squarks. Figure from~\textcite{TardiffThesis}.
}
\end{figure}

%SUSY contributions to electroweak baryogenesis typically requires  large phases unless cancellations arise which would require near degeneracy of superpartner masses. This suggests that EDM limits are making it difficult for SUSY, which is the so-called SUSY-CP problem~\cite{Dimopoulos:1995ju}. 
%There are, of course, many ways out including fermion partners (squarks and sleptons) with masses greater than 10 TeV, which could lead to suppression of one-loop EDMs while allowing for large phases~\cite{Giudice:2004tc,Kane:2009kv} and relatively light electroweak gauge boson and Higgs boson superpartners for baryogenesis.~\cite{Morrissey:2012db}. In this case, two-loop (Barr-Zee) diagrams~\cite{Barr:1990vd} can generate EDMs  
%Applying this to baryogenesis leads to lower bounds on EDMs that
%are only about 2 orders of magnitude below the current experimental limits
%\cite{Morrissey:2012db,Cirigliano:2006dg,Cirigliano:2009yd}.
%Another possibility is that the mechanism of SUSY-breaking that is responsible for splitting the SM masses from their superpartners may also suppress  CP-violating phases.(iii). Or there may  be cancellations. 
%It was proposed some time ago[24] that contributions to EDMs from different CPV phases or those from different dimension-six CPV operators may cancel leading to a suppression that again allows for O(1) phases and light superpartners.
%An extensive discussion of (i) and (iii) are given in the reviews of Refs. [14, 26] and the more recent analysis of EDMs in SUSY given in Ref. [88]. 

\begin{figure}
%\vskip -1.5 truein
\includegraphics[width=3.4 truein]{BarrZeeDiagram2} 
\vskip -.2 truein
\includegraphics[width=3.4 truein]{BarrZeeDiagram3}
%\vskip -1.2 truein
%\vspace{0.2cm}
\caption{Example two-loop Barr-Zee diagrams that give rise to a fermion EDM (coupling through $\gamma$) or CEDM (coupling through $g$).  Here $\tilde\tau$ is the $\tau$-slepton, $\tilde t$ and $\tilde b$ are squarks and $\chi$ is the chargino.
Adapted from~ \textcite{Barr:1990vd}.
\label{edm:fig:diagram2}}
\end{figure}

%%%%%%%%%%%%%%

%Within the SUSY context, it is worth nothing that  

%Beyond the MSSM, SUSY models with extra Higgs bosons~\cite{Ellwanger:2009dp,Dine:2007xi}, can contain CP-violating phases that generate EDMs at  tree level {\color{magenta} clarify and update the following},  and current limits on EDMs lead to strong constraints on these  phases as well~\cite{Blum:2010by,Altmannshofer:2011rm,Altmannshofer:2011iv}.  Higgs-sector CP models, in which.... \red {CLARIFY}, can produce EDMs that are are consistent with both current experimental limits and with  electroweak baryogenesis~\cite{Huber:2006wf,Blum:2010by,Pietroni:1992in,Davies:1996qn,Huber:2000mg,Kang:2004pp,Menon:2004wv, Kumar:2011np,Huber:2006ma,Profumo:2007wc} An additional scalar also provides a viable low-mass dark-matter candidate~\cite{Carena:2011jy}.

%%%%%%%%%%%%%%

The discovery of the  125 GeV Higgs boson has raised anew the possibility that it might be one of a number of scalars, and  a wide array of possibilities for the ``larger'' Higgs sector have been considered over the years. 
%and it is impossible to do justice to the subject in the context of this review. Instead, we consider 
One scenario that has been studied extensively is the Two-Higgs-Doublet model, %(2HDM). 
%One version of this scenario has already been discussed above in the context of the MSSM, 
wherein the requirements of supersymmetry restrict the form of the scalar potential and the couplings of the two Higgs doublets to the SM fermions. In the more general context, the Two-Higgs-Doublet model allows for a variety of additional CP-violating phases that can give rise to EDMs. The phases may arise in the scalar potential and/or the amplitudes for scalar-fermion interactions. 
The implications of new CP-violating phases in the Two-Higgs-Doublet model have been analyzed by~\textcite{Inoue:2014nva}, who considered a potential that manifests a softly-broken $Z_2$ symmetry in order to avoid constraints from the absence of flavor-changing neutral currents. (A $Z_2$ symmetry is a discrete symmetry under phase reversals of the relevant fields.)
 In the absence of CP-violation the scalar spectrum contains two charged scalars, $H^\pm$, and three neutral scalars: the CP-even $H^0$ and $h^0$ and the CP-odd $A^0$. With the presence of CP-violating phases in the potential, the three neutral scalars mix to form the neutral mass eigenstates, $h_i$, one of which is identified with the 125 GeV  SM Higgs-like scalar. This CP-mixing translates into CP-violating phases in the couplings of the $h_i$ to SM fermions, thereby inducing EDMs. In a variant of the Two-Higgs-Doublet model considered by~\textcite{Inoue:2014nva}, the requirements of electroweak symmetry-breaking imply that there exists only one  CP-violating phase in the scalar sector $\alpha_b$, which is responsible for both CP mixing among the scalars and the generation of EDMs. The latter arise from the Barr-Zee diagrams shown in FIG.~\ref{edm:fig:diagram2}. 

Constraints on $\alpha_b$ as a function of $\tan\beta$ set by present and prospective %and prospective
EDM results are  are shown in FIG.~\ref{fg:combined_S} for the \lq\lq type II" Two-Higgs-Doublet model (for an enumeration of several variants of the Two-Higgs-Doublet model, for exapmle see~\cite{Barger:1989fj}. The type II scenario has the same Yukawa structure as the MSSM.).
 %For purposes of illustration, the \red{ \lq\lq type II"?}-Two-Higgs-Doublet model has been assumed for the scalar-fermion interactions. 
The $d_e$ limit from ThO is generally the most restrictive, except in the vicinity of $\tan\beta\sim 1$ and $\sim 10$. \textcite{Bian:2014zka} have pointed out that the vanishing sensitivity to  $d_e$  near $\tan\beta\sim 1$ arises from a cancellation between the effects of the induced CP-violating couplings of the Higgs-like scalar to the electron and the corresponding couplings to the $h F^{\mu\nu} {\tilde F}_{\mu\nu}$ operator associated with the upper loop of the Barr-Zee diagrams.% In these \lq\lq cancellation regions", the present 
The neutron and $^{199}\mathrm{Hg}$ EDMs are not susceptible to the same cancellation mechanism as the electron and provide additional constraints near $\tan\beta\sim 1$.  The middle and far right panels show the sensitivity of prospective future EDM searches, including anticipated results of from $^{225}$Ra~\cite{PhysRevC.94.025501}. 
%Again, one observes that t
The reach of a ten times more sensitive $d_e$ search would extend somewhat beyond the constraints from  neutron and atomic searches, except in the cancellation regions. More optimistically, any non-zero result could indicate whether or not the observed EDM is consistent with CP violation in the Two-Higgs-Doublet model and help narrow the parameter space.\footnote{As indicated by \textcite{Inoue:2014nva}, there exist considerable hadronic and nuclear theory uncertainties associated with the $d_n$ and $d_A$ sensitivities.}


 \begin{figure*}[tb]
%\hskip -0.6 truein
\vskip -1.25 truein
\includegraphics[width= 7 truein]{combined_S}
\vskip -1.5 truein
\caption{\label{fg:combined_S}  (Color online) EDM results  for the Type II Two-Higgs-Doublet model. Horizontal axes show the ratio of up- and down-type Higgs vacuum expectation values. Vertical axes show the CP-violating phase that mixes of CP-even and CP-odd scalars. Purple regions are excluded by consistency with electroweak symmetry breaking. Left panel: current constraints from $d_e$ (blue), $d_n$ (green), and $d_A(^{199}\mathrm{Hg})$ (pink). Middle panel: impact of improving present constraints by one order of magnitude, where blue dashed line indicates the prospective reach of $d_e$. The yellow region indicates the reach of a future $d_A(^{225}\mathrm{Ra})$ with a sensitivity of $10^{-27}$ \ecm. Right panel gives same future constraints but with $d_n$ having two orders of magnitude better sensitivity than the present limit. 
Figure from~~\textcite{Inoue:2014nva}.
}
\end{figure*}



%%%%%%%%%%%%%%%%%%

%The possibility of addtional gauge-bosons  as well as an extended scalar sector \red{more Higgs?} has long been considered by the theoretical community\red {UGH}. Among the most extensively studied possibilities for the former is t
Left Right Symmetric Models postulate the existence of an SU(2)$_L\times$SU(2)$_R\times$U(1)$_{B-L}$ symmetry, in which parity-violation in the Standard Model arises from spontaneous breakdown of the SU(2)$_R$ symmetry at a scale above the electroweak scale  ($M_{W_R}>>M_{W_L}$)~\cite{Pati:1974yy,Pati:1974yy-erratum,Mohapatra:1974hk,Senjanovic:1975rk}. This gives rise to a  second CKM-like matrix for the right-handed charged-current couplings of $W_L$ with a new CP-violating phase. Spontaneous symmetry breaking induces mixing between left-handed $W_L$ and right-handed $W_R$ gauge bosons, and  the  mass eigenstates become a mixture of $W_L$ and $W_R$:
\begin{eqnarray}
\label{eq:LRSM1}
W_1^+ & = & \cos\xi W_L^+ + \sin\xi e^{-i\alpha} W_R^+\nonumber\\
W_2^+ & = & -\sin\xi e^{i\alpha} W_L^+ +\cos\xi W_R^+
\end{eqnarray}
where $\alpha$ is the CP-violating phase associated with the gauge boson mixing. This phase, along with the left- and right-handed CKM phases, can lead to one-loop quark EDMs arising from $W_{1,2}$ exchange. Retaining only the contribution from $\alpha$ and taking $|\sin\xi|\lessapprox 10^{-3}$ as implied by tests of CKM matrix unitarity, the resulting short-range contribution to the neutron EDM is
\begin{equation}
\label{eq:LRSM2}
| \bar d_n^{sr}|\lessapprox (3\times 10^{-14}\ e\ \mathrm{fm}) \left(1-\frac{M_1^2}{M_2^2}\right)\ \cos\theta_L\cos\theta_R\sin\alpha\ \ \ ,
\end{equation}
where $M_1$ and $M_2$ are the masses of $W_1$ ane $W_2$, respectively, and $\theta_{L,R}$ denote the left-handed and right-handed Cabibbo angles. The upper bound on the contribution to the neutron EDM  is an order of magnitude less than the current limits on $d_n$, though the analysis should be revisited to include  quark CEDM contributions. 

In addition, $W_L$-$W_R$ mixing gives rise to a unique, four-quark CP-violating operator that, in turn, generates the T-odd/P-odd $\pi-NN$ coupling $\gpbo$ discussed in Sec.~\ref{sec:LowEnergyParameters}.
\begin{equation}
\label{eq:LRSM3}
\gpbo\Big\vert^\mathrm{LRSM} \approx 10^{-4}\ \left(1-\frac{M_1^2}{M_2^2}\right) \sin\xi \cos\theta_L\cos\theta_R\sin\alpha\ \ \ .
\end{equation}
The resulting mercury atomic EDM  is
\begin{eqnarray}
\label{eq:LRSM4}
|d_A(^{199}\mathrm{Hg})| & \lessapprox & (1.1 \times 10^{-11}\ e\ \mathrm{fm})\\
\nonumber
&&\times  \left(1-\frac{M_1^2}{M_2^2}\right)\ \cos\theta_L\cos\theta_R\sin\alpha\ \ \ ,
\end{eqnarray}
where we have again used an approximate upper bound $|\sin\xi|\lessapprox 10^{-3}$ and take the $\gpio$ dependence of the Schiff moment as the midpoint of the range given in Table~\ref{tb:SchiffCoef}.
 Given the significantly larger coefficient in Eqn.~(\ref{eq:LRSM4}) compared to that in Eqn.~(\ref{eq:LRSM2}), together with the stronger mercury atom EDM bound, we observe that the atomic EDM results currently place the most severe constraints on the CP violation associated with $W_L$-$W_R$ mixing. There is, however, an important caveat: the contribution of $\gpbo$ to the $^{199}$Hg Schiff moment has significant nuclear theory uncertainties~\cite{rf:deJesus}, and it is possible that the sensitivity is considerably weaker than indicated in  Eqn.~\ref{eq:LRSM4}. On the other hand, the nuclear many-body computations for this contribution to the Schiff moments of other nuclei of experimental interest appear to be more reliable, providing motivation for active pursuit of improved experiments on $^{129}$Xe, $^{221/223}$Rn and $^{225}$Ra~\cite{dobaczewski05,ban10}. 


%\subsection{The EDM -- Flavor Connection}

New phases could in principle also affect CP-violation in flavor-violating process, such as meson mixing  or  rare $B$-meson decays, and give complementary information on the model
parameters~ \cite{Altmannshofer:2008hc} that could push the new
physics scale well beyond 10 TeV.
Even so, due to the generic flavor mixing, the  electron and neutron EDMs  are proportional to heavy-quark and lepton masses, and the experimental limits  probe scales of 1000 TeV in some cases.
An explicit example of such a case is given by the mini-split SUSY framework  for which current EDM bounds already probe masses up to 100~TeV~\cite{Hall:2011jd,Ibe:2011aa,Arvanitaki:2012ps,ArkaniHamed:2012gw,McKeen:2013dma,Altmannshofer:2013lfa}.

Recent work has also considered the constraints that EDMs may place on CP-violating couplings of other SM particles, such as the Higgs boson or top quark. \textcite{McKeen:2012av} computed constraints on the CPV Higgs-diphoton coupling, $h F_{\mu\nu} {\widetilde F}^{\mu\nu}$, and showed that the corresponding relative impact of this operator on the rate for the decay $h\to\gamma\gamma$ is at the $10^{-4}$ level,  well below the expected sensitivity at the LHC or future Higgs factories. This constraint may be weaker in specific models, such as those containing vector-like fermions (see also~\textcite{Chao:2014dpa} for the connection with baryogenesis). \textcite{Chien:2015xha} investigated the constraints on dimension-six operators that couple the Higgs boson to quarks and gluons, and found that the impact of hadronic and nuclear physics uncertainties is pronounced. \textcite{Cirigliano:2016njn,Cirigliano:2016nyn} and~\textcite{Fuyuto:2017xup} considered the constraints on the top quark EDM from $d_e$ and find that the bounds are three orders of magnitude stronger than obtained from other sources. 

\subsection{From theory to experiment}
\label{sec:EFTParameters}
%\subsection{Low-energy parameters}
%\label{sec:LowEnergyParameters}

Experiments probe P-odd/T-odd observables in systems that combine a number of scales as illustrated in FIG.~\ref{fg:EDMSubwayMap}. For the neutron and proton, the fundamental CP-violating interactions discussed above arise from two sources: a short range contribution (denoted by ${\bar d}_{n,p}^\mathrm{sr}$) and a long range contribution arising from the P-odd/T-odd pion-nucleon interactions. 
Storage ring experiments also have the potential to directly probe EDMs of light nuclei, namely, the deuteron ($^2$H$^+$) and helion ($^3$He$^{++}$) discussed in Sec.~\ref{sec:StorageRingEDMs}. The EDMs of these systems arise from the constituent nucleon EDMs as well as P-odd/T-odd nucleon-nucleon interactions arising from  pion-exchange and from four-nucleon contact interactions.  Paramagnetic atoms and molecules are most sensitive to the electron EDM and the nuclear-spin-independent electron-nucleus coupling. In diamagnetic atoms 
% with non-vanishing nuclear spin but no net electron spin, 
the dominant contributions are the nuclear-spin dependent electron-nucleus interaction and the Schiff moment, which also arises from long-range pion exchange and short range four-nucleon interactions. The following summarizes the contributions of these low-energy parameters to the experimentally accessible systems.

\vskip 0.1in

\noindent{\em Nucleons:}

\vskip 0.1in

%The quark CEDMs, the three-gluon operator, and  four-quark operators will induce non-vanishing nucleon EDMs and niuclear Schiff moments through the pion-nucleon couplings $\bar g_\pi^{(0,1,2}$ for isospin couplings 0,1, and 2.
% $\gpbz$ and $\gpbo$.
% A detailed discussion of the relative sensitivities of the low-energy parameters to these CPV sources goes beyond the scope of this work, and we refer the reader to Ref.~\cite{Engel:2013lsa}. 
%Nonetheless, we highlight two features. First, 
Long-range strangeness-conserving pion-nucleon coupling contributions to the nucleon EDMs indicated in FIG.~\ref{fg:MDMEDM} have been computed using chiral perturbation theory.
The  magnitude of $\gpbt$ is expected to be suppressed by a factor of 100 or more relative to $\gpbz$ and $\gpbo$ based on chiral symmetry considerations  
%[see Ref.~\cite{Engel:2013lsa} and references therein],
and is typically neglected in the computation of the nucleon EDMs~
\cite{Chupp:2014gka}. 
% Additionally, the effect of a non-vanishing P-odd/T-odd quark-Higgs coupling $\mathrm{Im} C_{\varphi ud}$ will generate both a neutron and proton EDM and also
 %, to leading order in chiral counting, 
%contribute to $\gpbo$. As indicated by Eqn.~(\ref{eq:dnfull}) the long-range contribution to $d_n$ associated with $\gpbo$ is quadratically suppressed by $m_\pi^2/m_N^2$, whereas the effect on $d_p$ is linearly proportional to $m_\pi/m_N$. %, giving these systems some advantage as probes of $\gpbo$.
%The neutron and proton EDMs arise from two sources. The long-range contributions from the P-odd/T-odd $\pi$-$NN$ interaction have been computed using heavy baryon chiral perturbation theory, with the 
%The remaining short-range contribution is contained in the low-energy coefficients ${\bar d}_n^\mathrm{sr}$ and ${\bar d}_p^\mathrm{sr}$ 
%\cite{Seng:2014pba}:
%\bea
%\label{eq:dnfull}
%d_n  =  {\bar d}_n^\mathrm{sr}-\frac{e g_A}{8\pi^2 F_\pi} &\bigl\{&\gpbz[\ln \frac{m_\pi^2}{m_N^2} -\frac{\pi m_\pi}{2 m_N}] \nonumber\\
%&+& \gpbo\,\frac{(\kappa_1-\kappa_0)}{4}\frac{m_\pi^2}{m_N^2}\ln  \frac{m_\pi^2}{m_N^2} \bigr\} \nonumber\\
%d_n  =  {\bar d}_n^\mathrm{sr}-\frac{e g_A\gpbz}{8\pi^2 F_\pi} &\bigl\{&\ln \frac{m_\pi^2}{m_N^2} -\frac{\pi m_\pi}{2 m_N} \nonumber\\
%&+& \frac{\gpbo}{4\gpbz}\, (\kappa_1-\kappa_0)\frac{m_\pi^2}{m_N^2}\ln  \frac{m_\pi^2}{m_N^2} \bigr\} \nonumber\\
%\label{eq:dpfull}
%d_p  =  {\bar d}_p^\mathrm{sr}+\frac{e g_A}{8\pi^2 F_\pi} &\bigl\{ \gpbz[&\ln \frac{m_\pi^2}{m_N^2} 
%-\frac{2\pi m_\pi}{m_N}]\nonumber\\
%-\frac{\gpbo}{4}\bigl[  \frac{2\pi m_\pi}{m_N}  &+&(\frac{5}{2}+\kappa_0+\kappa_1) \frac{m_\pi^2}{m_N^2}\ln  \frac{m_\pi^2}{m_N^2}\bigr]\bigr\} \ \ \ ,\nonumber\\
%d_p  =  {\bar d}_p^\mathrm{sr}+\frac{e g_A\gpbz}{8\pi^2 F_\pi} &\bigl\{ &\ln \frac{m_\pi^2}{m_N^2} 
%-\frac{2\pi m_\pi}{m_N}\nonumber\\
%-\frac{\gpbo}{4\gpbz}\bigl[  \frac{2\pi m_\pi}{m_N}  &+&(\frac{5}{2}+\kappa_0+\kappa_1) \frac{m_\pi^2}{m_N^2}\ln  \frac{m_\pi^2}{m_N^2}\bigr]\bigr\} \ \ \ ,\nonumber\\
%\eea
%where $g_A\approx 1.27$ is the nucleon isovector axial coupling and  $\kappa_0$ and $\kappa_1$ are the isoscalar and isovector nucleon anomalous magnetic moments, respectively. 
The result to next-to-next-to-leading order is \cite{Seng:2014pba}:
\bea
\label{eq:dnfull}
d_n  =  {\bar d}_n^\mathrm{sr}-\frac{e g_A}{8\pi^2 F_\pi} &\biggl\{&\!\!\!\!\ \gpbz[\ln \frac{m_\pi^2}{m_N^2} -\frac{\pi m_\pi}{2 m_N}] \nonumber\\
&+&\gpbo\,\frac{(\kappa_1-\kappa_0)}{4}\frac{m_\pi^2}{m_N^2}\ln  \frac{m_\pi^2}{m_N^2} \biggr\} \nonumber\\
%d_n  =  {\bar d}_n^\mathrm{sr}-\frac{e g_A\gpbz}{8\pi^2 F_\pi} &\bigl\{&\ln \frac{m_\pi^2}{m_N^2} -\frac{\pi m_\pi}{2 m_N} \nonumber\\
%&+& \frac{\gpbo}{4\gpbz}\, (\kappa_1-\kappa_0)\frac{m_\pi^2}{m_N^2}\ln  \frac{m_\pi^2}{m_N^2} \bigr\} \nonumber\\
\label{eq:dpfull}
d_p  =  {\bar d}_p^\mathrm{sr}+\frac{e g_A}{8\pi^2 F_\pi} &\biggl\{&\!\! \gpbz[\ln \frac{m_\pi^2}{m_N^2}
-\frac{2\pi m_\pi}{m_N}]\nonumber\\
-\frac{\gpbo}{4}\bigl[  \frac{2\pi m_\pi}{m_N} \!\!\!&+& \!\! (\frac{5}{2}+\kappa_0+\kappa_1) \frac{m_\pi^2}{m_N^2}\ln  \frac{m_\pi^2}{m_N^2}\bigr]\biggr\} ,\nonumber\\
%d_p  =  {\bar d}_p^\mathrm{sr}+\frac{e g_A\gpbz}{8\pi^2 F_\pi} &\bigl\{ &\ln \frac{m_\pi^2}{m_N^2}
%-\frac{2\pi m_\pi}{m_N}\nonumber\\
%-\frac{\gpbo}{4\gpbz}\bigl[  \frac{2\pi m_\pi}{m_N}  &+&(\frac{5}{2}+\kappa_0+\kappa_1) \frac{m_\pi^2}{m_N^2}\ln  \frac{m_\pi^2}{m_N^2}\bigr]\bigr\} \ \ \ ,\nonumber\\
\eea
where $g_A$ is the nucleon isovector axial coupling,  $\kappa_0$ and $\kappa_1$ are the isoscalar and isovector nucleon anomalous magnetic moments, respectively, and the low-energy coefficients ${\bar d}_n^\mathrm{sr}$ and ${\bar d}_p^\mathrm{sr}$ account for remaining short range contributions.
Note that the ${\bar d}_{n,p}^\mathrm{sr}$ are linear combinations of the ${\bar d}_{0,1}$ given in Eqn.~(\ref{eq:pinn}). Computations of the ${\bar d}_{n,p}^\mathrm{sr}$ and $\gpbi$ in terms of the fundamental CP-violating interactions are reviewed by~\textcite{Engel:2013lsa}, \textcite{Shindler:2015aqa}, \textcite{Bhattacharya:2015esa}, \textcite{Seng:2016pfd},  and \textcite{Bouchard:2016heu}.
%where $F_\pi=92.4$ is the pion decay constant and $g_A=1.27$ is the nucleon axial vector coupling.
%Theoretical efforts have focused, in particular, on relating the contribution of the QCD parameter $\bar\theta$ to the neutron EDM in chiral-perturbation theory~\cite{Pospelov:1999ha,rf:Crewther1979} and recently thorugh lattice-QCD calculations~\cite{Shindler:2015aqa}, which concur that 
%In particular, we point out that the QCD parameter $\theta$ contributes to $\gpbz$:  
%\be
%\gpbz\approx (0.005 {\rm -}0.04)\bar\theta+\dots,
%\label{eq:gpi0thetabar}
%\ee
%where the range repesents the present theoretical uncertainty and where the ellipses indicates  BSM contributions~\cite{Engel:2013lsa}.

In particular, we point out that the QCD parameter $\theta$ contributes to $\gpbz$:  
\be
\gpbz\approx (0.015\pm 0.003)\bar\theta+\dots,
\label{eq:gpi0thetabar}
\ee
 where the ellipses indicate  BSM contributions~\cite{deVries:2015una}. %This result updates the value with larger theoretical uncertainty given in  Ref.~\cite{Engel:2013lsa}.


\vskip 0.1in

\noindent{\em Light Nuclei:}

\vskip 0.1in

Experimental approaches to storage-ring measurements of the EDMs of the deuteron  and helion  are discussed in Sec.~\ref{sec:StorageRingEDMs}. For the deuteron, the EDM has contributions from the nucleon moments as well as the pion-exchange contribution, leading to
\bea
d_D&=&d_n+d_p\nonumber\\
&+&[(0.0028\pm 0.0003)g_\pi^0+(0.18\pm 0.02)\gpbo]\  e\ \text{fm}.\nonumber\\
\eea
For the helion ($^3$He$^{++}$), the proton spins are nearly completely paired~\cite{rf:FriarPayneGibson} and the neutron EDM dominates the one-body contribution:
\bea
d_h&=&0.9d_n-0.05d_p \nonumber\\
&+&[(0.10\pm0.03)\gpbz+(0.14\pm 0.03)\gpbo]\ e\ \text{fm}.\nonumber\\
\eea
Note that the contributions from the four-nucleon contact interactions have not been included here ~\cite{rf:deVriesDEDM} (see also \cite{Stetcu:2008vt,deVries:2011an,Song:2012yh,Wirzba:2016saz,Yamanaka:2016umw}. 

\vskip 0.2in

\noindent{\em Paramagnetic systems:}

\vskip 0.1in

In paramagnetic systems with one or more unpaired electrons, there is a net electric field $\vec{E}_\mathrm{eff}$ at the electron's average position that is generally much greater than a laboratory electric field (many V/\AA\ or GV/cm).
Consequently the EDMs  of paramagnetic atoms and P-odd/T-odd observables in polar molecules are dominated by the electron EDM and the nuclear-spin-independent electron-nucleon interaction, which couples to a scalar (S) component of the nucleus current.
%The EDM interaction for an elementary fermion is
%\be
%\label{eq:edmdef}
%\mathcal{L}^\mathrm{EDM} = -i\sum_f\ \frac{e d_f^\gamma}{2\Lambda} 
%{\bar f}\sigma^{\mu\nu} \gamma_5 f\ F_{\mu\nu}\ \ \ .
%\mathcal{L}^\mathrm{EDM} = -i\sum_f\ \frac{ d_f}{2} 
%{\bar f}\sigma^{\mu\nu} \gamma_5 f\ F_{\mu\nu} \ ,
%\ee
%where $F_{\mu\nu}$ is the electromagnetic field strength.
%In the non-relativistic limit, Eqn.~(\ref{eq:edmdef}) contains  the P-odd/T-odd interaction with the electric field ${\vec E}$,
%\be
%\label{eq:fedm}
%\mathcal{L}^\mathrm{EDM}\rightarrow \sum_f {d_f}\ 
%\chi^\dag_f {\vec\sigma}\chi_f \cdot {\vec E} \ ,
%\ee
%where $\chi_f$ is the Pauli spinor for fermion $f$ and ${\vec\sigma}$ is the 
%vector of Pauli matrices.
%The  Lagrangian  has the form 
Taking the nuclear matrix element of the interactions given in Eqn.~(\ref{eq:NSID}) and assuming non-relativistic nucleons lead to the atomic Hamiltonian
\be
{\hat H}_S = \frac{i G_F}{\sqrt{2}}\, \delta({\vec r})\, \left [(Z+N)C_S^{(0)}+ (Z-N)C_S^{(1)}\right]\gamma_0\gamma_5,
\ee
The resulting atomic EDM $d_A$ is given by
\be
%d_A = \rho_A^e d_e - \kappa_S^{(0)}\, \left[ C_S^{(0)} +\left( \frac{Z-N}{Z+N}\right) C_S^{(1)}\right]\ \ \ ,
d_A^\mathrm{para} = \rho_A^e d_e - \kappa_S^{(0)}\, C_S,
\ee
where
\be
C_S\equiv C_S^{(0)} +\left( \frac{Z-N}{Z+N}\right) C_S^{(1)},
\ee
and  $\rho_A^e$ and $\kappa_S^{(0)}$ are obtained from atomic and hadronic computations. 

For polar molecules, the effective Hamiltonian is
\be
{\hat H}_\mathrm{mol} = \left[W_d\, d_e +W_S\,  (Z+N)C_S\right] {\vec S}\cdot{\hat n}+\cdots\ \ \ ,
\ee
%where 
%\be
% {\bar C}_S \equiv (Z+N) C_S^{(0)} + (Z-N)C_S^{(1)} \equiv (Z+N)C_S
%\ee
where ${\vec S}$ and ${\hat n}$ denote the unpaired electron spin and the unit vector along the intermolecular axis, respectively. The quantities $W_d\propto E_\mathrm{eff}$ and $W_S$ that give the sensitivities of the molecular energy to the electron EDM and electron-quark interaction are obtained from molecular structure calculations~\cite{Ginges:2003qt,Meyer:2008gc,rf:Skripnikov2013,kozlov98,Petrov2007,Fleig2013,Skripnikov2017}. The resulting ground state matrix element in the presence of an external electric field ${\vec E}_\mathrm{ext}$ is
\be
\bra{\mathrm{g.s.}} {\hat H}_\mathrm{mol} \ket{\mathrm{g.s.}} = \left[W_d\, d_e +W_S\, (Z+N)C_S\right] \, \eta(E_\mathrm{ext}),
\ee
with 
\be
\eta(E_\mathrm{ext}) = \bra{\mathrm{g.s.}}{\vec S}\cdot{\hat n}  \ket{\mathrm{g.s.}} _{E_\mathrm{ext}}\ \ \ .
\ee
This takes into account the orientation of the internuclear axis and the internal electric field with respect to the external field, {\it i.e.} the electric polarizability of the molecule. This leads to the observable, a  P-odd/T-odd  frequency shift measured in molecular experiments discussed in Sec.~\ref{sec:PolarMolecules}.

%\subsubsection{Diamagnetic atoms and nucleons}

\vskip 0.1in

\noindent{\em Diamagnetic atoms and molecules:}

\vskip 0.1in

The EDMs of diamagnetic atoms of present experimental interest arise from the nuclear Schiff moment  and the nuclear-spin-dependent electron-nucleon interaction, which couples to the tensor (T) nuclear current. 
The Schiff moment, accounting for both contributions from the EDM's of unpaired nucleons  and the long-range pion-nucleon coupling, can be written
\begin{equation}
S=s_{N} d_{N} + \frac{m_N g_A}{F_\pi}\left[a_0 \gpbz + a_1 \gpbo	+	a_2\gpbt\right] \ \ ,
\label{eq:SchiffMoment}
\end{equation} 
where contributions from the unpaired nucleon EDM's are given by $s({d_{N}})=s_n d_n+s_p d_p$  \cite{Dzuba1985,dmitriev03,ban10,Yoshinaga_doi:10.1143/PTP.124.1115}. 
%$g_A\approx 1.27$  is the nucleon axial-vector coupling and $F_\pi=92.2$ MeV
%{\color{magenta} check norm - FROM OUR PRC - Tim}  is the pion decay constant. 
% $ \frac{m_N g_A}{F_\pi} = 13.5$, for an effective nucleon mass of 980 MeV. 
Values of $a_{0,1,2}$ from Eqn.~\ref{eq:SchiffMoment} for $^{199}$Hg, $^{129}$Xe, $^{225}$Ra, and TlF are presented in Table~\ref{tb:SchiffCoef}. These  depend on the details of the assumed nucleon-nucleon interaction. However note that there is no single consistent approach for all nuclei of interest. %A relevant selection of the theoretical results is presented in table~\ref{tb:SchiffCoef}, and a thorough review of the theoretical status of calculations of $a_{0,1,2}$ is provided in ref~\cite{Engel:2013lsa}; 
As discussed above, each isospin component may be particularly sensitive to a subset of the possible CP-violating interactions. For example,  the QCD parameter $\bar\theta$ contributes most strongly to $\gpbz$, while the effect of $W_L$-$W_R$ mixing in the Left-Right Symmetric Model shows up most strongly in $\gpbo$. % Note that $a_2$ is included even though the sensitivity of $\gpbt$ to the underlying CP-violating interactions is generally suppressed.

The nucleon EDM  long-range and short-range contributions to the Schiff moment can be separated using Eqn.~\ref{eq:dnfull} to write
\bea
S=   s_N \bar d_N^{sr}+\left[\frac{m_N g_A}{F_\pi} a_0\right.&+&\left.s_N\alpha_{n\gpiz}\right] \gpiz \nonumber\\
&+&  \left[\frac{m_N g_A}{F_\pi} a_1+ s_N\alpha_{n\gpio}\right] \gpio , \nonumber\\
\eea
 where the coefficients $\alpha_{N\bar g_\pi^{(0,1)}}$, given in Table~\ref{tb:diamagnetics}, are the factors multiplying $\gpiz$ and $\gpio$ in Eqn.~\ref{eq:dnfull} , and  the smaller $\gpbt$ pion-nucleon contribution to $S$ has been dropped.  


Contributions from the electron-nucleus interaction are revealed in the Hamiltonian resulting from Eqn.~(\ref{eq:NSD}):
\be
{\hat H}_T = \frac{2i G_F}{\sqrt{2}}\, \delta({\vec r})\, \left [C_T^{(0)}+ C_T^{(1)}\tau_3\right]\, {\vec\sigma}_N\cdot{\vec\gamma}\ \ \ ,
\ee
where the sum over all nucleons is again implicit;  $\tau_3$ is the nucleon isospin Pauli matrix, ${\vec\sigma}_N$ is the nucleon spin Pauli matrix, and ${\vec\gamma}$ acts on the electron wave function. Including the effect of ${\hat H}_T$, the individual nucleon EDMs $d_N$, and the nuclear Schiff moment $S$ (Eqn.~\ref{eq:SchiffDef}), one has
\be
\label{eq:diamag}
d_A(\mathrm{dia}) = %\sum_{N=p,n} \rho_Z^N d_N + 
\kappa_S S - \left[k_{C_T}^{(0)} C_T^{(0)} + k_{C_T}^{(1)} C_T^{(1)}\right]\ \ \ ,
\ee
where $\kappa_S$ and  $k_{C_T}^{(0,1)}$ give the sensitivities of the $d_A^\mathrm{dia}$ to the Schiff moment and the isoscalar and isovector electron-quark tensor interactions and are provided in Tables~\ref{tb:diamagnetics} and~\ref{tb:SchiffCoef}. %A compilation of the $\rho_Z^N$, $\kappa_S$, and $k_T^{(0,1)}$ can be found in Ref.~\cite{Engel:2013lsa}\footnote{We note that the values for  the $\kappa_S$ given in that work should be multiplied by an overall factor of $-1$ given the convention used there and in Eqn.~(\ref{eq:diamag}).}.
As indicated in Eqn.~\ref{eq:CSi}, the isoscalar and isovector tensor couplings depend on the same Wilson coefficient $\mathrm{Im}\ C_{\ell e qu}^{(3)}$, so their values differ only due to the different nucleon tensor form factors. 
%, and 
% \red{INTEPRETATION - sec~|ref{sec:GlobalAnalysis} At this level of interpretation, a meaningful \red{fit} 
%can be combined into a single parameter rather than two distinct and independent tensor couplings. 
Until recently, there has been limited information on the nucleon tensor form factors $g_T^{(0,1)}$. Computations using lattice QCD have now obtained $g_T^{(1)}=0.49 (03)$ \cite{Bhattacharya:2016zcn} and $g_T^{(0)} = 0.27 (03)$~\cite{Bhattacharya:2015wna}\footnote{Note that the numerical values of the  tensor couplings given in these references are two times larger than quoted here, owing to differences in normalization conventions.}. For the diamagnetic atoms of experimental interest (Hg, Xe, Ra) the nuclear matrix elements are dominated by the contribution from a single, unpaired neutron while for TlF, to a good approximation the proton is unpaired. We therefore replace the last term in brackets in Eqn.~(\ref{eq:diamag}) with  $k_T^{n} C_T^{n}$ or $k_T^{p} C_T^{p}$, for an unpaired neutron or proton, respectively, where
\bea
C_T^n &=& - \left[ g_T^{(0)} - g_T^{(1)}\right]\, \left(\frac{v}{\Lambda}\right)^2\,\mathrm{Im}\, C_{\ell e q u}^{(3)} \approx - 0.76 C_T^{(0)}\nonumber \\ 
C_T^p &=& - \left[ g_T^{(0)} + g_T^{(1)}\right]\, \left(\frac{v}{\Lambda}\right)^2\, \mathrm{Im}\, C_{\ell e q u}^{(3)}\approx +2.45 C_T^{(0)}. \nonumber \\ 
\label{eq:CTs}
\eea
 Table~\ref{tb:diamagnetics} provides the coefficients for the dependence of $d_A^{dia}$ on $\gpiz,\ \gpio$, and $\bar d_n^{sr}$. 

%For $^{199}\mathrm{Hg}$, $k_T^n= 4\times 10^{-20}$ \ecm.

%{\color{magenta} We now have the $C_T^{(0,1)}$ which did not exist when we did our earlier global analysis. }

%The nuclear Schiff moment arises from a P-odd/T-odd nucleon-nucleon interaction generated by the pion exchange similar to FIG.~\ref{fg:MDMEDM}, where one of the pion-nucleon vertices is the strong pion-nucleon coupling and the other is the P-odd/T-odd pion-nucleon interaction:

 %V. F. DMITRIEV, R. A. SEN�KOV, AND N. AUERBACH  PHYSICAL REVIEW C 71, 035501 (2005)
%[23] J. H. de Jesus and J. Engel, Phys. Rev. C 72, 045503 (2005).





%$\bar g_{ CP}^0\approx 0.027\;\theta_{QCD}$~\cite{rf:Crewther1979,rf:Crewther1980}, and quark EDMs contribute to the isovector component.
%For example, in reference~\cite{rf:deJesus05}   $a_0$ ranges from 0.002 to 0.01 for $^{199}$Hg. 
% In calculating the $a$'s, the pion�exchange P� and T�violating interaction, collective effects that are known to renormalize strength distributions of Schiff�like operators, pairing at the mean�field level, self�consistency, along with a number of possible Skyrme interactions were included. For $^{199}$Hg for example,  $a_0$ ranges from 0.002 to 0.01~\cite{rf:deJesus05}. 
%R. J. Crewther, P. Di Vecchia, and G. Veneziano, Phys.Lett. 88B, 123 (1979), 91B, 487(E), 1980.
%The Schiff moment could also arise from the proton or neutron EDM~\cite{rf:Dmitriev2003}. %V. F. Dmitriev and R. A. Sen�kov, Phys. Rev. Lett. 91 212303 (2003)
%$S_{\rm A}=s_p d_p+s_n d_n.$
%As noted, the isotensor coupling $\gpbt$ requires isospin breaking and is generically suppressed with respect to $\gpbz$ and $\gpbo$. The nuclear Schiff moment can then be expressed as
%\be
%S \approx  \frac{m_N g_A}{F_\pi} \left[a_0\gpbz + a_1\gpbo\right].
%\ee



%As discussed in detail in Ref.~\cite{Engel:2013lsa}, there exists considerable uncertainty in the nuclear Schiff moment calculations, so we will adopt the \lq\lq best values" and theoretical ranges for the $a_{0,1}$ given in that work. 
%{\bf At present, we do not possess an up-to-date, consistent set of $\rho_Z^N$ for all of the diamagnetic atoms of interest here. Rather than introduce an additional set of associated nuclear theory uncertainties, we thus do not include these terms in our fit. Looking to the future, additional nuclear theory work in this regard would be advantageous since, for example, the sensitivity of the present $^{199}$Hg result to $d_n$ is not too different from the limit obtained in Ref.~\cite{Baker:2006ts}.  }

%To date, no complete computations of the  $\rho_Z^N$ for different nuclei have been reported in the literature {\color{red} MJRM: check and complete statement}. Consequently, we do not include these terms in our analysis of diamagnetic atom EDMs. 

\begin{figure}[tb]
\vskip -0.75 truein
\centerline{\includegraphics[width= 4.25 truein]{MDMEDM1stgen}}
\vskip -0.5 truein
\caption{\label{fg:MDMEDM} Representative long-range, pion-exchange contributions to the neutron EDM.  The cross represents the CP-violating vertex, while the closed circle is the CP-conserving vertex. Adapted from \textcite{Pospelov:1999mv}.}
\end{figure}



In summary, contributions to the EDMs in systems accessible to experiment can be expressed in terms of the following set of low-energy parameters:
\begin{enumerate}

\item The lepton EDMs; the electron EDM $d_e$ contributes in first order to the EDMs of paramagetic atoms and molecules. 

\item  Two isospin components of the nuclear-spin-independent $eN$ coupling $C_S^{0,1}$. Since most of the heavy-atom systems have a roughly equal ratio of neutrons to protons this can be reduced to a single average $\bar C_S=C_S^0-\frac{(N-Z)}{A}C_S^1\approx C_S^0$. 


\item The nuclear-spin-dependent $eN$ coupling labeled by $C_T^{(0,1)}$, most important in diamagnetic atoms and molecules.



\item The short-range contribution to the nucleon EDMs $\bar d_{n,p}^{sr}$.


\item The pion-nucleon couplings labeled $\gpbi$ that contribute to the nucleon and nuclear EDMs and to the Schiff moments of nuclei. Given that the sensitivity of $\gpbt$ to the CP-violating interactions  is highly suppressed, we will omit it in the following.


\end {enumerate}


We therefore separate paramagnetic atoms and molecules from diamagnetic systems and also separate nucleon and  fundamental-fermion EDMs, 
%so that the EDM of  %observable in storage ring or solid-state  
as follows:

\medskip
\noindent
\centerline
{\bf Paramagnetic atoms}
\begin{equation}
d_A(\mathrm{para})=\eta_{d_e}d_e + k_{C_S} \bar C_S
\label{eq:ParamagneticEDMs}
\end{equation}

\noindent
\centerline
{\bf Polar molecules}
\begin{equation}
\Delta\omega^{\not P\not T}= \frac{-d_e E_{eff}}{\hbar} + k^\omega_{C_S} \bar C_S
\label{PolarMoleculeDeltaOmega}
\end{equation}

\noindent
\centerline
{\bf Diamagnetic atoms} 
\begin{equation}
d_A(\mathrm{dia})=\kappa_S S(\bar g_\pi^{0,1},d_N) + k_{C_T^{(0)}} C_T^{(0)} + ...
\label{eq:DiamagneticAtoms}
\end{equation}

\noindent
\centerline
{\bf Nucleons}
\begin{equation}
d_{n,p}=d_{n,p}^{lr}(\bar g_\pi^{0,1} )+{\bar d}_{n,p}^{sr}
\end{equation}

\noindent
\centerline
{\bf Charged leptons}
\begin{equation}
d_e, d_{\mu}, (d_\tau)
\end{equation}

\noindent 
The coefficients $\eta$, $k$ and $\kappa$ are presented in Tables~\ref{tb:paramagnetics},~\ref{tb:diamagnetics} and~\ref{tb:SchiffCoef}.


Note that the other contributions enter the atomic and molecular systems at higher order, but are less important.  However due to the exquisite sensitivity of the $^{199}$Hg EDM measurement, the higher order contribution of the electron EDM $d_e$ does have an impact. % in constraining $d_e$ as discussed in section~\ref{sec:GlobalAnalysis}.
Additionally, experiments in paramagnetic solid-state systems with quasi-free electrons are directly sensitive to $d_e$.
%We do not consider a possible future proton EDM search, since in this context it would introduce one additional parameter (${\bar d}_p$) and would not substantially impact the fit {\color{red} check if this is true for a $10^{-29}$ \ecm~ measurement}.


%\subsection{Theory Goals (constrain, mass scale...) (atomic and nuclear theory???)}






%We do not consider a possible future proton EDM search, since in this context it would introduce one additional parameter (${\bar d}_p$) and would not substantially impact the fit {\color{red} check if this is true for a $10^{-29}$ \ecm~ measurement}.


%\subsection{Theory Goals (constrain, mass scale...) (atomic and nuclear theory???)}



\begin{table*}[h]
%\scriptsize
\centering
\begin{tabular}{|c|c|c|c|c|}
\hline\hline
System & $\alpha_{d_e}=\eta_e$  & $\alpha_{C_S}=W_S$ & $\alpha_{C_S}/\alpha_{d_e}$ & ref.\\
\hline
Cs  & 123  & $7.1\times 10^{-19}$ \ecm &  $5.8\times 10^{-21}$    (\ecm) & $a$, $b$,$c$ \\
&$(100$-$138)$ &  $(7.0$-$7.2)$  &  $(0.6$-$0.7)\times 10^{-20}$  &   \\
\hline
Tl  & -573 &  $-7\times 10^{-18}$ \ecm &  $1.2\times 10^{-20}$   (\ecm) &  $a$, $b$ \\
& $-(562$-$716)$ &  $-(5$-$9)$      &  $(1.1$-$1.2)\times 10^{-20}$   &\\
 \hline
YbF  & $-3.5\times 10^{25}$ $\frac{\rm rad/s}{\ecm}$ & $-2.9\times 10^5$ rad/s &  $8.6\times 10^{-21}$   (\ecm) & $d$ \\
 &$-(2.9$-$3.8)$& $-(4.6$-$6.8)$    &  $(8.0$-$9.0)\times 10^{-21}$   &  \\
\hline
ThO &$-1.6\times 10^{26}$ $\frac{\rm rad/s}{\ecm}$ &$-2.1\times 10^6$ rad/s &  $1.3\times 10^{-20}$  (\ecm)  &$e$, $f$, $g$  \\
 & $-(1.3$-$1.6)$ & $-(1.4$-$2.1)$  &  $(1.2$-$1.3)\times 10^{-20}$  &   \\
\hline
HfF$^+$ & $3.5\times 10^{25}$ $\frac{\rm rad/s}{\ecm}$ &$3.2\times 10^5$ rad/s &  $8.9\times 10^{-21}$  (\ecm)   &$h$, $i$, $j$  \\
 & $-(3.4$-$3.6)$ &  $(3.0$-$3.3)$ &$(8.3$-$9.7) $  &  \\
\hline\hline
\end{tabular}
\vskip -0.1 truein
\caption{Sensitivity  to $d_e$ ($\alpha_{d_e}$) and $C_S$ ($\alpha_{C_S}$) and the ratio $\alpha_{C_S}/\alpha_{d_e}$  for observables in paramagnetic systems  based on atomic theory calculations.  Ranges (bottom entry) for coefficients $\alpha_{ij}$ representing the contribution of each of the T-odd/P-odd parameters to the observed EDM of each system. For atomic systems, the atom EDM is measured, whereas for molecular systems the P-odd/Todd frequency is measured, from which $d_e$ and $C_S$ are determined from the tabulated $\alpha's$. (Note that for YbF and ThO, $\alpha_{d_e}=eE_{eff}/\hbar=\pi W_d$, with $W_d$ given in \cite{Engel:2013lsa}; for HfF$^+$, $\alpha_{d_e}=eE_{eff}/\hbar$~\cite{Cairncross:2017fip} and $\alpha_{C_S}=W_S=W_{T,P}\frac{Z+N}{Z}$ with $W_{T,P}$ given by~\cite{Skripnikov2017}.) References: $a$~\cite{Ginges:2003qt}; $b$~\cite{Engel:2013lsa}; $c$~\cite{Nataraj2008}; $d$~\cite{rf:Dzuba2011,rf:Dzuba2011-erratum}, $e$~\cite{Meyer:2008gc}; $f$~\cite{rf:Dzuba2011,rf:Dzuba2011-erratum}; $g$~\cite{rf:Skripnikov2013}; $h$~\cite{Petrov2007}; $i$~\cite{Fleig2013}; $j$~\cite{Skripnikov2017}. \label{tb:paramagnetics}}
\end{table*}

  
 \begin{table*}[h]
\centering \renewcommand{\arraystretch}{1.5}
\begin{tabular}{|c|c|c|c|c|c|c|c|}
\hline\hline
System & $\partial d^{exp}/ \partial d_e$  & $\partial d^{exp}/\partial C_S$ ({\it e}~cm)  & $\partial d^{exp}/\partial C_T^{(0)}$ ({\it e}~cm)  & $\partial d^{exp}/\partial \gpiz$ ({\it e}~cm) &   $\partial d^{exp}/\partial \gpio$ (\ecm) &
 $\partial d^{exp}/\partial \bar d_n^{sr}$\\
\hline
neutron    &  & & &  $1.5\times 10^{-14}$  & $1.4\times 10^{-16}$ & $1$  \\
\hline
$^{129}$Xe  & -0.0008  & $-4.4\times 10^{-23}$  & $-6.1\times 10^{-21}$   & $-0.4\times 10^{-19}$  & $-2.2\times 10^{-19}$ & $1.7\times 10^{-5}$  \\
 &    &    $-4.4$-$(-5.6)$                           & $-6.1$-$(-9.1)$   &    $-23.4$-$(1.8)$  & $-19$-$ (-1.1)$  & $1.7$-$2.4$  \\
\hline 
$^{199}$Hg & -0.014  & $-5.9\times 10^{-22}$  & $3.0\times 10^{-20}$   & $-11.8\times 10^{-18}$  & 0 & $-5.3\times 10^{-4}$ \\
 &    $-0.014$-$0.012$ &  & $3.0$-$9.0$   &   $-38$-$(-9.9)$  & $(-4.9$-$1.6)\times 10^{-17}$ & $-7.7$-$(-5.2)$  \\
\hline
$^{225}$Ra  &   & $ $  & $5.3\times 10^{-20}$   & $1.7\times 10^{-15}$  & $-6.9\times 10^{-15}$ &   \\
 &    &                               &  5.3-10.0 &  $6.9$-$0.9$     &   $-27.5$-$(-3.8)$ & $(-1.6$-$0)\times 10^{-3}$ \\
\hline
TlF  & 81  &   $2.9\times 10^{-18}$  & $2.7\times 10^{-16}$  & $1.9\times 10^{-14}$  & $-1.6\times 10^{-13}$ & $0.46$ \\
 &    &                                 & & $0.5$-$2$ &  & $-0.5$-$0.5$ \\
%%%%%
%neutron    &  & & &  $1.5\times 10^{-14}$  & $1.4\times 10^{-16}$  \\
 %&    &                                 & & &  \\
%\hline
%$^{129}$Xe  & -0.0008  & $-4.4\times 10^{-23}$  & $4\times 10^{-21}$   & $-2.9\times 10^{-19}$  & $-2.2\times 10^{-19}$  \\
% &    &    $(-4.4-(-5.6))$                           & $(4-6)$   &    $(-26-(-1.8))$  & $(-19- (-1.1))$   \\
%\hline 
%$^{199}$Hg & -0.014  & $-5.9\times 10^{-22}$  & $-2\times 10^{-20}$   & $-3.8\times 10^{-18}$  & 0 \\
% &    -0.014 - (+0.012) &  & $(-5.9-(-2.0))$   &   $(-27-(-1.9))$  & $(-4.9 - 1.6)\times 10^{-17}$   \\
%\hline
%$^{225}$Ra  &   & $ $  & $-3.5\times 10^{-20}$   & $1.7\times 10^{-15}$  & $-6.9\times 10^{-15}$  \\
% &    &                               &    &     &    \\
%\hline
%TlF  & 81  &   $2.9\times 10^{-18}$  & $1.1\times 10^{-16}$  & $1.2\times 10^{-14}$  & $-1.6\times 10^{-13}$ \\
% &    &                                 & & &  \\
%%%%%%
 \hline\hline
\end{tabular}
\vskip -0.1 truein
\caption{Coefficients for P-odd/T-odd parameter contributions to EDMs  for diamagnetic systems and the neutron. The second line for each entry is the reasonable range for each coefficient.   The $\partial d^{exp}/ \partial d_e$  and  $\partial d^{exp}/ \partial C_S$ are from~\cite{Ginges:2003qt} and are based on~\cite{rf:Martensson1985} and~\cite{rf:Martensson1987} for $^{129}$Xe, and $^{199}$Hg. Also see~\cite{Fleig:2018bsf} for $^{199}$Hg. The $\partial d^{exp}/ \partial d_e$ and  $\partial d^{exp}/ \partial C_S$   for TlF are compiled in~\cite{rf:Cho1991}. The $\partial d^{exp}/ \partial C_T^{(0)}$ are adjusted for the unpaired neutron in $^{129}$Xe, $^{199}$Hg and $^{225}$Ra using $k_T$ from ~\cite{Ginges:2003qt} and is consistent with~\cite{Sahoo:2016zvr}.  For $^{225}$Ra $\partial d^{exp}/ \partial C_T^{(0)}$ is from~\cite{Dzuba2009,Singh:2015aba}. The $\gpbz$, $\gpbo$ and $\bar d_n^{sr}$ coefficients for atoms and molecules are based on data provided in Table~\ref{tb:SchiffCoef}; the range for $^{225}$Ra corresponds to $0\le s_n\le 2$ fm$^2$. 
For TlF, the unpaired neutron is replaced by an unpaired proton  and the ``best value'' assumes $\bar d_p^{sr}=-\bar d_n^{sr}$, {\it i.e.} mostly isovector in analogy to the anomalous magnetic moment, while the range is defined by $|\bar d_p^{sr}|\le |\bar d_n^{sr}|$ . \label{tb:diamagnetics}}
\end{table*}

\begin{table*}[h]
\begin{center}
\begin{tabular}{|c|c|c|c|c|c|}
\hline\hline
System & $\kappa_S=\frac{d}{S}$ (cm/fm$^3$)  & $a_0=\frac{S}{13.5\gpbz}$ ($e$-fm$^3$) & $a_1=\frac{S}{13.5\gpbo}$ ($e$-fm$^3$) & $a_2=\frac{S}{13.5\bar g_\pi^(2)}$ ($e$-fm$^3$) & $s_N$ (fm$^2$) \\
\hline
$^{129}$Xe &	   $0.27\times 10^{-17}$  (0.27-0.38) & $-0.008 (-0.005$-$(-0.05))$ & $-0.006 (-0.003$-$(-0.05))$ & $-0.009 (-0.005$-$(-0.1))$ & 0.63 \\
\hline
$^{199}$Hg &   $-2.8\times  10^{-17}$( $-4.0$-$(-2.8)$) & 0.01 $(0.005$-$0.05)$ & $\pm$0.02 $(-0.03$-$0.09)$ & $0.02 (0.01$-$0.06)$ & $1.895\pm 0.035$\\
\hline
$^{225}$Ra &   $-8.5\times 10^{-17}$ ($-8.5$-$(-6.8)$)  & $-1.5\ (-6$-$(-1))$ & $+6.0$  $(4$-$24)$ & $-4.0\ (-15$-$(-3))$ & \\
\hline
TlF &  $-7.4\times 10^{-14}$  & -0.0124 & 0.1612 & -0.0248 & $0.62$\\
%\hline
%Rn &   $2.0/3.3\times 10^{-17}$~\cite{rf:SpevakPRCv56p1357a1997,rf:DzubaPRAv60p02111a2002} & -62 & 62 & -100 \\
\hline\hline
\end{tabular}
\end{center}
\vskip -0.1 truein
\caption{\label{tb:SchiffCoef} Ranges and ``best values'' used in~\textcite{Chupp:2014gka} for atomic EDM sensitivity to the Schiff-moment and dependence of the Schiff moments on $\gpbz$ and $\gpbo$;  $\kappa_S$ and $s_N$.  References:  TlF: \cite{rf:TlFTheory}; Hg:\cite{rf:DzubaPRAv60p02111a2002,rf:FlbmKhrNuclPhysAp449a1986,Singh:2014jca}; Xe: \cite{rf:DzubaPLBv154p93a1985,rf:DzubaPRAv60p02111a2002,Teruya:2017don}; Ra: \cite{rf:SpevakPRCv56p1357a1997,rf:DzubaPRAv60p02111a2002,Singh:2015aba}. Values for $a_0$, $a_1$ and $a_2$ are compiled in \cite{Engel:2013lsa}. The value of $s_n$  is from~\cite{Dzuba1985} for $^{129}$Xe and from~\cite{dmitriev03} for $^{199}$Hg; there is no available calculation of $s_n$ for $^{225}$Ra. The value for $s_p$ for TlF is derived from~\cite{rf:Cho1991}.
}
\end{table*}



\section{Experimental techniques}
\label{sec:Techniques}
 
%
The crux of any EDM measurement is  to measure the effect of the coupling to an electric field in the background of much larger magnetic effects using the unique P-odd/T-odd signature.
Most EDM experiments using beams or cells are magnetic resonance approaches that measure the energy or, more commonly, frequency given in Eqn.~\ref{eq:EDMFreqEquation1} of transitions between magnetic sublevels in the presence of a well-controlled magnetic field $\vec B$, and electric field $\vec E$ aligned either parallel or anti-parallel to $\vec B$. Storage-ring experiments with charged particles measure the result of the additional torque on the spin due to $\vec d\times\vec E$, where $\vec E$ may arise in part from the motional field $\vec v\times\vec B$. In solid-state electron-EDM experiments, the observable is proportional to $\vec B\cdot\vec E$, where only one field is applied and the other measured - for example a strong electric field $\vec E_{\rm applied}$ would polarize electron spins in the material giving rise to an observable magnetic field $\vec B_{\rm observed}$.
 
Because every system of interest has a magnetic moment, the magnetic environment is crucial and the magnetic field must be characterized in space and time. Magnetic shielding, magnetic sensors external to the EDM volume, and comagnetometers that monitor the magnetic field within the EDM volume during the EDM measurement are all essential elements of past and future experiments. Comagnetometer species are chosen because they are less sensitive to P-odd/T-odd effects than the key species. For example a $^{199}$Hg comagnetometer was used for the neutron-EDM experiment~\cite{Baker:2006ts}, and Na was used as a comagnetometer for Cs~\cite{rf:Weisskopf1968} and Tl~\cite{Regan:2002ta}. The measurement of the $^{129}$Xe EDM utilized $^3$He as the comagnetometer species~\cite{rf:Rosenberry2001}. In the case of polar molecules discussed in Sec~\ref{sec:PolarMolecules}, the comagnetometer can be effected with one molecular species using combinations of transitions~\cite{Hudson:2002az,Baron:2013eja,Cairncross:2017fip}. 
 
%Modern approaches also take advantage of enhancements of P-odd/T-odd effects due to the enhanced polarizability of molecules and atoms with octuple nuclei.
 
EDM measurements in many systems require determining frequency differences with precision of nHz ($10^{-9}$ Hz) or less with measurement times much less than $10^9$ seconds.
Another interesting feature for stored atoms and neutrons  is the need to correct for the Earth's rotation as well as the accumulated quantum phase or Berry's phase that arises due to the combination of motional magnetic field $(\vec v\times\vec E)/c^2$ with magnetic-field gradients.
 
\subsection{Magnetic shielding}
\label{sec:MagneticShielding}
A critical component of EDM experiments is the magnetic shielding, which mitigates electromagnetic distortions in time and space. Neutron and proton EDM experiments in particular require large volumes with stringent magnetic shielding requirements. For example, improving the current limit on the neutron EDM in the next generation experiments 
% of 2.9$\cdot$10$^{-26}$~ecm \cite{Baker:2006ts} 
by two orders of magnitude requires magnetic-field gradients less than nT$/$m (see sec.~\ref{sec:systematics}). The temporal stability of the magnetic gradient must be better than 100~fTm$^{-1}$s$^{-1}$ over the  ~100-300~s neutron-storage time. This requires  strong damping  of external perturbations at extremely low frequencies, {\it i.e.} between 1-100~mHz. It is also crucial to reverse the magnetic field orientation in the lab, which requires magnetic shields that can tolerate large changes of the field {\em inside} the shield.
%
Recent advances in active and passive magnetic shielding techniques for  next-generation room-temperature EDM  experiments are reported by~\textcite{rf:JAPv117p233903y2015,Altarev:2014wka,rf:SunDynamicShielding}. Cryogenic  magnetically shielded environments have  been demonstrated, {\it e.g.} by \textcite{Cabrera:1989qr,ref:romalis_ferrite}, and these concepts have been extended to larger volumes for the SNS cryogenic nEDM experiment~\cite{Slutsky:2017mbn}. 
%These have been  demonstrated that the goals of large-scale next-generation EDM measurements can be achieved.
 
%Generally shielding a large-volume EDM experiment entails both active and passive magnetic shields along with RF shielding and thick electrical conductors for eddy-current shielding. A dynamic active shield is effected by feedback of a combination of measurements of the magnetic field at a set of positions over a volume to a set of coils~\cite{LinsThesis2016}.
 
 
Passive magnetic shielding is based on surrounding the volume of interest with a high-magnetic-permeability  material generally called mu-metal or permalloy. Permalloy is applied in thicknesses of order 1-4 mm rolled into welded cylinders or cones~\cite{rf:BudkerMSRCones} or assembled as sheets.
A passive  shield is best characterized by  the reduction in the amplitude of magnetic field variations {\it i.e.} the frequency-dependent damping factor or shielding factor $SF(f)$. %This is commonly called the shielding factor  or more recently the damping factor. 
A second crucial characteristic is  the residual field and gradient inside the shield, which is affected by a procedure called degaussing or equilibration.
%~\cite{rf:SunDynamicShielding,rf:FThielRSIv78p035106y2007,rf:JAPv117p233903y2015}.
 
 
It is useful to provide analytic approximations for static fields as guidelines for cylindrical shields composed of permalloy cylinders with permeability $\mu$, thickness $t$ and radius $R$~\cite{SumnerMagneticShielding1987}, with the caveat that actual results for damping factors at low frequencies, in particular for multi-layer shields, may differ significantly.  The transverse damping factor for a single shield is 
\begin{equation}
SF^T\approx 1+\frac{\mu t}{2R}.
\end{equation}
The transverse shielding factor for multiple layers is a product of damping factors;  for $n$ layers the transverse shielding factor is approximately
\be
SF^T\approx SF^T_n\prod_{i=1}^{n-1} SF_i^T \left(1-(\frac{R_i}{R_{i+1}})^2\right).
\ee 
The air-gaps between layers can be optimized for a given material, thickness, and number of layers.
Axial shielding is generally less effective and depends on the ratio of the length of the cylinder to the radius $a=L/R$ and the empirically determined distribution of magnetic flux over the cylinder end caps and walls, which is characterized, respectively, by quantities  $\alpha$ and $\beta$  ($\alpha\le 1$ and $1\le\beta\le 2$): 
\be
SF^A \approx 1+\left(\frac{2\kappa(a)}{1+a+\alpha a^2/3}\right) SF^T,
\ee
where
\begin{eqnarray}
\kappa(a)&=&\left(1+\frac{1}{4a^3}\right)\beta-\frac{1}{a}\nonumber\\
&+&2\alpha\left\{\ln(a+\sqrt{1+a^2})-2\big(\sqrt{1+\frac{1}{a^2}}-\frac{1}{a}\big)\right\}.\nonumber\\ 
\end{eqnarray}
While these expressions are illustrative, in general, effective design of shields  is aided by simulations of Maxwell's equations using finite-element approaches.%, often using commercial products ({\it e.g.} Opera, Comsol, {\it etc.}).
 
%Experience has shown that overlapping at edges improves performance.
 
 
 
 
 
 
%Residual-field shielding factor and equilibration.
 
 
 
%
%\subsubsection{Neutron EDM shielding}
%Improving the current limit on the neutron EDM of 2.9$\cdot$10$^{-26}$~ecm \cite{Baker:2006ts} by two orders of magnitude requires magnetic-field gradients less than 0.3~nT$/$m (see sec.~\ref{}sec:systematics) with temporal stability of the magnetic gradient must be better than 100~fTm$^{-1}$s$^{-1}$ over the  300~s neutron-storage time.
%
%This requires a strong damping factor (commonly called "shielding factor", SF) of the magnetically shielded environment at extremely low frequencies, {\it i.e.} between 1-100~mHz. It is also crucial to reverse the magnetic field in space, which requires shields that can be used with large changes of the field {\em inside} the shield.
%
%The state of the art in 2018, shown in FIG.~\ref{fig:tumshield}, is a magnetic shielded box with inside dimensions 1.54~m $\times$ 1.54~m $\times$ 2.2~m   with the following advances~\cite{Altarev:2014wka}:


%$\bullet$ Increased thickness of the permalloy sheets (1-1.6~mm thickness are typical maximum commercially available dimensions);

%$\bullet$ Fewer material joints, effected with large rectangular sheets  of 3000 mm $\times$ 750~mm mounted in pairs with perpendicular orientation and connected at many places to form one closed shell of shielding material with 5 cm overlap at the edges;

%$\bullet$  Smaller  gaps of 8~cm between shield layers, which  reduces the overall shield size without degrading performance. This differs from previous design criteria, which suggested that a significantly larger spacing of the layers helps obtain good shielding performance;

%$\bullet$ A thinner inner-shield layer of 2 mm of  permalloy Krupp Magnifer\footnote{Krupp Magnifer is a commercial product from VDM Metals GmbH, Plettenberger Strasse 2, 58791 Werdohl, Germany} balances the tradeoff of a small residual field and large shielding factor.%

%$\bullet$ The  layers are arranged with outer layers of 2 mm Krupp Magnifer, 8 mm of aluminum, and 2 mm of Magnifer. The three inner layers are, respectively  2 mm, 4 mm and 2 mm of Krupp Magnifer (see FIG.~\ref{fig:tumshield}).\\
%its shielding performance vastly exceeds any other large scale shield ever built, even with more shield layers.
 %
Additional considerations for passive shields include penetrations (holes), the temperature dependence of the magnetic properties of the shielding material,  and the applied internal field, which couples to the shield and may cause a temperature dependence of fields and gradients for the experiment~\cite{ANDALIB2017139}.  Holes up to 130 mm  do not notably change the damping factor. %Residual fields at the center of the shields and distortions close to the walls are not appreciably affected by using overlapping layers at the corners~\cite{Altarev:2014wka}.
 Also,  any conductor close to the experiment produces Johnson current noise, which in turn causes magnetic-field noise~\cite{LeeSKRomalis2008}. This includes the permalloy and is also a consideration for the non-magnetic ({\it e.g.} aluminum)  RF-shielding layer.
Temperature differences also cause slowly changing magnetic fields in many conductors, which put additional constraints on the design of the experimental apparatus located inside shields.

%
%
%\begin{figure*}[tb]
%\includegraphics[width= 7 truein]{5layer_shield}
%\vskip -0.8 truein
%\caption{\label{fig:tumshield}  (Color online) Cut through the magnetic shield of the TU M\"unchen EDM experiment. The outer magnetically shielded room (MSR) is described in detail in ref.~\cite{rf:JAPv117p233903y2015}. Item (1) is the large door of the MSR, (2) the outer magnetic shield made from 2 mm Magnifier, (3) 2 mm Magnifier shell together with the  8 mm aluminum shell for radio-frequency shielding. Inside this shell, a rail system is mounted to bring in the insert with 3 shells and the experiment. Item (4) is the outer  2 mm Magnifier shell of the insert, (5) the 4 mm middle shell and  (6) the 2 mm inner  shell of the insert; item (7) is the insert door. Item (8), inside the insert is the NMR magnet described in Fig~\ref{fig:B0Coil}, and (9) the chamber containing the EDM experiment.}
%\end{figure*}
%
%The TU-M\"unchen shielded box has a residual field of a few-hundred~pT, and a gradient of $|\nabla B|=$ 100~pT/m. An inner shield constructed of a single-layer permalloy cylinder with 1.4 m diameter and 2.1 m length mounted inside this box had an internal magnetic field of $|B|<$~100~pT over a volume of greater than 0.5~m$^3$. An alternative  inner shield consisted of a single-layer 1.8 m-cubic box with residual field $|B|<$~100~pT and gradient $|\nabla B|<$ 25~pT/m over a 0.8~m$^3$ volume.  The damping factor inside the box for low-amplitude variations of the external field  ($\sim$ 1~$\mu$T) exceeds 10$^6$ at the 0.001 Hz  low-frequency limit of the measurements, reducing   effects caused by typical external distortions (crane operation, people with tools, doors, cars and other machinery within tens of meters of the shield and major transportation such as subway trains or electric buses hundreds of meters away) to below 1~pT.
%
%Typically, ambient magnetic field drifts are up to 1000-times smaller at night, when drifts of $\sim$~10~fT and gradient  $<$~100~fT$/$m have been recorded  at all positions inside the shields.


%
A comparison of  leading magnetic-shield installations, is presented in Table~\ref{fig:sf} and Fig.~\ref{fg:ShieldComparison}. The Boston Medical Center shield (Boston), used for biomagnetism research, is composed of three permalloy layers and three aluminum layers resulting in a large damping factor at the relatively high freqeuency of 1 Hz~\cite{ref:boston}. The Berlin-Magnetically-Shielded Room (BMSR-II) is a large-scale/walk-in user facility with small residual fields~\cite{ref:bmsr2}.  The TU-M\"unchen shield (TUM-Shield), consists of six layers plus a 1 cm thick aluminum layer for RF shielding.The outer layers form a rectangular box and the inner most layer is a permalloy cylinder~\cite{Altarev:2014wka}. 
%Note that direct comparison of shields is difficult due to the unknown and different measurement conditions.

 
\begin{table}
\begin{ruledtabular}
\begin{tabular}{|l|r|c|r|}
\textrm{Shield} &      \textrm{$f$~$[ Hz ]$} &  \textrm{B$_{ext.}$~$[ \mu T_{rms} ]$} &  $\textrm{SF}(f)$ \\
\colrule
%TUM MSR both shells        &   0.01        &                               2  (peak-to-peak)                  &      279              \\
%\colrule
%TUM Insert       &   0.01        &                               2   (pp)                 &      $\sim$ 50,000        \\
\colrule
BMSR-II      &                                0.01     &   1                                   &           75000          \\
BMSR-II       &                                1     &       1                                   &           2,000,000          \\
%TUM MSR $+$ Insert      &   0.003        &             3.2 (pp)                   &   $\sim$1,400,000            \\
%TUM MSR $+$ Insert $+$ Cylinder  & 0.003  &   6.4 (pp)  &  $>$ 6,000,000\footnote{transverse shielding factor} \\
\colrule
Boston    &                                0.01     &       1                                &           1630          \\
Boston       &                                1     &         1                                  &           200,000          \\
\colrule
%TUM MSR $+$ Insert      &   0.333        &             32 (pp)                   &     $\sim$8,000,000        \\
TUM MSR $+$ Insert      &   0.01        &             4.5                 &     $\sim$2,000,000          \\
TUM MSR $+$ Insert       &   1.25        &             22                   &     $>$ 16,700,000          \\
\end{tabular}
\end{ruledtabular}
\caption{\label{fig:sf} Measured  damping factor $SF(f)$ of three shield-installations  for different external excitation strengths B$_{ext.}$(either peak-to-peak or root mean square) and frequency $f$. The BOSTON shield ~\cite{ref:boston}, BMSR-II is the Berlin Magnetically-Shielded Room at PTB~\cite{ref:bmsr2} were developed primarily for biomagnetism and magneto-medicine. The TUM shield at the Technical University of M\"unchen, was developed for neutron EDM measurements~\cite{Altarev:2014wka,rf:JAPv117p233903y2015}. 
}
\end{table}
 
 
%

Magnetic equilibration is a procedure based on commonly known degaussing techniques developed to achieve extremely small residual fields and gradients~\cite{rf:FThielRSIv78p035106y2007,rf:JAPv117p233903y2015,rf:SunDynamicShielding,10.2478/mms-2013-0021}. 
%
Magnetic equilibration based on the developments of~\textcite{10.2478/mms-2013-0021} used coils wound around the edges of each shell of a cuboid shield to generate strong magnetic flux with a sinusoidal AC current of several Hz frequency and linearly or exponentially decreasing amplitude. 
 The result of the procedure is a superposition of damped fields from outside the shield, residual magnetization of the shield material, distortions caused by holes and other imperfections, and applied internal fields. A residual field gradient of order 1~nT$/$m over about 1~m$^3$ was reached after applying the equilibration procedure in all three directions for typically 100~s in each direction.
%It is found that holes up to 130~mm and overlaps of the shielding layers do not notably change the damping factor, the residual fields at the center of the shields, and do not cause increased distortions close to the walls.~\cite{Altarev:2014wka}. 
%
%
%
Recently~\textcite{rf:JAPv117p233903y2015} implemented the L-shaped coil arrangement shown in FIG.~\ref{fg:LShapedCoils}, which enabled magnetic saturation of all shells at once in a much shorter total time  ($<$~50~s) with similarly small residual fields inside the shield. The speed and reproducibility are a benefit to EDM measurements, because it is necessary  to reverse the direction of the magnetic holding field and equilibrate the shields regularly  during an experiment to control systematic effects.
Numerical simulations have been compared to measurements showing that equilibration can be successfully fully modeled~\cite{rf:SunDynamicShielding}.
%
%
%
%
%Therefore, the equilibrium magnetization of the shield changes, which requires new magnetic equilibration after each field reversal, typically on the order of 10-100~times per day.
%
%In addition, equilibration coils can be mounted along the walls of the shield layers, instead of along the edges of the cuboid shells.
%
%The arrangement of the magnetic equilibration coils has been chosen in contradiction to previous knowledge not along the edges of the shield but rather along the panels of the shield, without notable issues visible in the residual field.
%
%
%Numerical modeling also provides for optimum placement of equilibration coils~\cite{rf:SunDynamicShielding}.
%
%A detailed description of all relevant aspects to achieve ultimate-quality magnetic shielding is in preparation \cite{ref:stuiber}.
 
%\subsubsection*{Proton EDM shielding}
%


\begin{figure*}[tb]
\center
\includegraphics[width= 7 truein]{LShapedCoils}
\vskip -2 truein
\caption{\label{fg:LShapedCoils}  (Color online)
Magnetization of a permalloy box with equilibration coils. Left: Original L-shaped configuration with coils at the corners. Right: Distributed L-shaped configuration with coils arranged over the surfaces. The coils are solid lines. The color scale shows the magnetization in the permalloy in arbitrary units. Figure provided by Z. Sun. 
}
\end{figure*}

 
A proposed experiment to measure the EDM of the proton, described in Sec.~\ref{sec:StorageRingEDMs}, requires magnetic shielding of an 800~m circumference electrostatic storage ring with  magnetic less than a few nT at any point~\cite{Anastassopoulos:2015ura}. % presents a new set of challenges~\cite{Anastassopoulos:2015ura}. 
% due to the need to  control of the evolution of the proton spin.
%To achieve proton-EDM sensitivity below 10$^{-29}$\ecm, the magnetic field must be less than a few nT at any point, and the integrated transverse magnetic field along the path of the proton orbit must be less than $\sim$~25~pT for static fields. 
%With counter-rotating proton beams, the important frequencies are 1-10 kHz, and the low-frequency shielding factor is therefore less crucial.  
An effective solution is a toroidal shield made up of individual three meter long cylinders illustrated in FIG.~\ref{fig:pedmshield}. 
%With maximum acceptable distortions  of 0.1~$\mu$T at 0.01~Hz (slower day-night drifts can usually be compensated  or removed via frequent magnetic equilibration), the calculated damping at 0.01~Hz is  approximately 1000.
%Because the magnitude of the magnetic field at any point inside this volume is the limiting field, gradients are less critical for the storage-ring-proton-EDM experiment, and to $\sim$~0.5~nT$/$m in transverse direction can be tolerated.
%Precession of the spin around fields transverse to the propagation of the beam is most dangerous.
%
%
Magnetic equilibration of a shield of this length would require  degaussing individual cylindrical sections, but
residual magnetization may build up near the equilibration coils due to non-uniformity of the magnetic flux at the ends of the cylinders. This can be compensated by the short ring of permalloy  placed inside the shield,.% which corrects the magnetic distortions caused by the equilibration coils.
%
%
%A detailed discussion of the concept of a proton-EDM measurement is found in sec.~\ref{she:perm}
%
%{\bf The magnetic field can only be measured all 4 meters along the ring. ???}
%
 
%
%The beam size itself is on the order of $\pm$~1.5cm.
%
%
%The field quality must be good enough that also all second-order effects from beam oscillations off the symmetry axis are removed.
%
%
%Therefore, at > 1 kHz any outside distortions must be damped strongly to avoid changes in the fields inside above few pT.
 
%Significantly below this frequency, changes can in principle be measured.
%
%However, the order in field multi-polarity is limited.
%
%Therefore, the absolute number of static field values along the path should not exceed about 100~pT at any place along the beam.
%
%With maximum acceptable distortions from the outside of 0.1~$\mu$T at 0.01~Hz (slower day-night drifts can usually be compensated if required or removed via frequent magnetic equilibration), the SF at 0.01~Hz is $\sim$~1000.
%
%Typically, external drifts at this frequency are lower than this anyway.
%
%
%
\begin{figure}[tb]
\vskip -0.5 truein
\includegraphics[width=3.5 truein]{long_shield}
\vskip -0.5 truein
\caption{\label{fig:pedmshield}  (Color online)
Shielding designed for a proton EDM experiment: (a) Permalloy and compensation coil configurations. The outer layers (1,3)  are connected in segments. A correction ring (2) compensates for the magnetic distortions caused by the magnetic equilibration coils (4). The compensation ring equilibration coils are labeled (5).  (b) Simulation of magnetic distortions caused by  outer-cylinder equilibration coils mitigated by the compensation ring. 
%Basic concept for shielding of long experiments. For extremely large magnetic shields, magnetic equilibration is done in independent short sections with closed magnetic flux rings at the position of joints of the shield sections. Item (1) is one of the two shield cylinders in the figure; (2) a magnetic equilibration coil wound around (1), (3) is the joint between the cylinders and (4) a ring made from shielding material, which corrects for the distortions from magnetic equilibration. This ring can be independently magnetically equilibrated after the cylinder.
}
\end{figure}
 
 
 
 
% As an additional note, there are also flexible permalloy foils available, which can be bent or wound around experimental apparatus for shielding purposes. Such materials are thin and are applied in many layers. The shielding factors obtained with such configurations tend to show a stronger directional dependence depending on the assembly. Also, the effective permeability of the material, once assembled, is significantly smaller than that of the raw material due to mechanical tension and air gaps.

\subsection*{Cryogenic shields}
 
Cryogenic shielding, based on the Meissner effect with Type-1 superconductors  {\it e.g.} Pb, has been envisioned for low-temperature EDM experiments. In contrast to permalloy-based passive shields, cryogenic shields stabilize both external perturbations and instabilities in the applied magnetic field. Also, the residual magnetic field inside a cryogenic shield is frozen during the transition to superconductivity. \textcite{Slutsky:2017mbn} developed a prototype cylindrical cryogenic shield 4 m long with 1.2 m diameter, which  provided a gradient less than 1 nT/m  over over a 0.1 m$^3$ volume.
An additional consideration for  cryogenic shielding is that the magnetic field cannot be reversed without warming up the shield.
%Must deal with Thermo-electric B-fields
 
In operation, a superconducting shield is surrounded by a room-temperature shield so that it can be cooled below the critical temperature in a small external field.  Another approach is a  combination of room-temperature and superconducting materials, for example a cylinder wrapped with METGLAS\footnote{METGLAS is commercial product: Metglas�, Inc.
440 Allied Drive, 
Conway, SC 29526-8202,
www.metglas.com.}~\cite{PG11,Slutsky:2017mbn}. Magnetic-field noise less than 100 fT/$\sqrt{\rm Hz}$ at $f=0.01$ Hz has been achieved in hybrid configurations measured by~\textcite{rf:burghoff_kreuth_presentation}. Figure~\ref{fg:ShieldComparison} provides a comparison of damping factors for several shield configurations including the cryogenic shield placed inside the Berlin MSR-II.



%Shields for cold-atom magnetometers with $< $0.5 m typical dimensions reach larger damping factors
%but worse residual fields and gradients. Shields used for atomic fountain interferometers
%have both much lower damping factors and larger residual fields~\cite{Dickerson2012}.

\begin{figure}[tb]
\includegraphics[width=3.75 truein]{noise_burghoff}
\vskip -0.25 truein
\caption{\label{fg:ShieldComparison}  (Color online)
Comparison of noise in different shields measured with SQUIDs. Supercon. refers to a superconducting cylinder cooled below the transition temperature inside the low magnetic field environment BMSR-II. The Zuse shield is also at PTB-Berlin~\cite{10.2478/mms-2013-0021}. %Note that the noise of the SC shield and the cubic shields of the TUM MSR both show comparable performance. 
At very low frequencies, the intrinsic noise of the (different) SQUID systems combined and the integration time to record data for this plot dominate the performance; at high frequencies, the noise level is dominated by the experimental setup to perform the measurement. Figure provided with permission by M. Burghoff.
}
\end{figure}

 
\subsection{Magnetometers}
\label{sec:Magnetometry}
%
%The determination of the magnetic field over the volume of the EDM experiment and during and EDM measurement at the fT level  is crucial.%to (i)  determine geometric phases and (ii) find spurious leakage currents due to the HV applied across the measurement cell.
%
%For a neutron EDM measurement, the magnetic field must be mapped over the volume of $\sim$~0.5~m$^3$ and monitored during the entire few-hundred second duration of the measurement at the fT scale. This requires specific arrangements of  scalar and vector magnetometers to determine gradients and the sources of distortions.
%
%
Any uncompensated change in the magnetic field between two subsequent or spatially separated  measurements with opposite electric field will  appear as a false EDM:
\begin{equation}
d_\mathrm{false} = \frac{\mu \Delta B }{ E}.
\label{eq:sigma_b}
\end{equation}
For a random-noise spectrum of magnetic field variations, this effect will be reduced in proportion to $\sqrt{M}$ for $M$ subsequent electric-field reversals. However, any correlation of the electric and magnetic field, for example due to leakage currents across the storage cell, will not average towards zero. %
In either case, monitoring the magnetic field is essential to a successful measurement. 
%A crucial parameter is the ratio of EDM sensitivity to magnetometer sensitivity given by $\sigma_dE_{eff}/\mu \sigma_B$.
%
As an example, for a measurement precision of $\sigma_{d} = 10^{-28}$\ecm, an electric field $E = 10^4$~V/cm, and $M$ = 10$^4$, the required effective magnetic field measurement precision is $\delta B\approx 2$ fT.
%
%This represents a significant challenge for magnetometry techniques. 
%
The magnetic field can  be measured (i) with magnetometers surrounding the experiment to estimate the full flux entering and leaving the experiment at the time of the measurement, or (ii) directly at the position of the measurement and at the same time, with a comagnetometer. External magnetometer measurements can be used to estimate the magnetic flux through the EDM experiment volume,  {\it i.e.} 4$\pi$ magnetometry~\cite{Nouri:2014epa,Nouri201492,Nouri:2015xva,LinsThesis2016} without the complications of injecting the comagnetometer species into the EDM measurement chambers. However external magnetometry requires that the magnetic environment inside  the EDM chamber be well characterized and that any changes, for example magnetized spots generated by sparks  can be monitored. 


%Magnetometers measure the magnetic field at specific positions or averaged over specific volumes near the EDM experiment, providing vector or scalar magnetic-field information. 
%Different sensors avearge over  at sensor-specific bandwidths. 
%Magnetometers are classified either as  comagnetometers that occupy the same volume at the same time as the EDM measurement or external magnetometers positioned outside the EDM measurement volume. 
% e.g. due to high voltage discharges between electrodes.
%
%Reconstruction of the shape of the field is complementary to the determination of the average field in the chamber via a comagnetometer.
%
 %A comagnetometer makes use of  a companion species that occupies the storage volume at the same time as the EDM-measurement.
% While the precision of individual  systems can be fT/$\sqrt{\rm Hz}$, the accuracy and long-term stability are typically limited by the absolute calibration and effects like the magnetic field caused by a sensor itself, typically on the scale of pT.


% vs. the required sensitivity to also measure comparably fast 
Comagnetometers were first deployed in the electron-EDM Cs beam measurement by~\textcite{rf:Weisskopf1968} and the Tl beam measurement by~\textcite{Regan:2002ta}. For atomic-EDM measurements, the concept relies on comparable magnetic moments but very different EDMs due to the $Z$ dependence  contributions due to the electron EDM, electron-nucleus couplings and  the Schiff moment.  Generally, these scale approximately as $Z^2$ for diamagnetic atoms and $Z^3$ for paramagnetic atoms. In principle, however, an experiment really measures the {\em difference} of the species' EDMs. In contrast to external magnetometers, comagnetometers should have coherence times comparable to the storage/interrogation times of milliseconds for atomic beams to  several hundred seconds for UCN to thousands of seconds for the $^{129}$Xe EDM experiments.
%Recent and ongoing developments show that the spin-precession measured with a comagnetometer contains more complex information on the fields inside the chamber than previously anticipated \cite{rf:Bales2016}. 

%where the electronic EDM of the Tl-atom was investigated using the Ramsey-method with spatially separated spin-flip coils and counter-propagating atomic beams of atomic thallium. (Sandars)
%
%As a cohabitating substance, a Na beam was used in the same experiment. 
%
%The comagnetometer also monitors sources of magnetic distortions inside the experiment,%, whereas in (ii) such sources are assumed to be stable in time and monitored before or after the experiment.
%
%Nuclear spin magnetometers operate at a low Larmor frequency and integrate magnetic fields over very long times, on the order of 100~s.
%
%This results in an extremely high sensitivity for long time constants, but faster changes in the field cannot be well tracked.
%
%An inevitable advantage of the comagentometer is that sources at the inside of the chamber, 
%e.g. localized dipole-like spots that may develop at the electrodes  during electrostatic breakdowns are measured.
%
%The comagnetometer should be able to monitor the field over the course of the entire EDM measurement. 
 
 
%
A comparison of the sensitivity and accuracy of commonly used magnetometers is presented in FIG.~\ref{fig:magnbandwidths}. Sensitivity characterizes the smallest change in the magnetic field that can be detected and generally improves with the measurement time, at least for short times. Sensitivity is clearly important for frequency stability and for monitoring systematic effects such as leakage currents. Accuracy, which is essentially  calibration stability, is required for two (or more) separated magnetometers used, for example, to determine static and time-changing magnetic-field gradients.

\subsection*{Rb and Cs magnetometers}
Alkali-metal  magnetometers (usually Rb or Cs) have been developed and implemented since the inception of optical pumping, and their sensitivity and stability has been improved and optimized for a variety of experiments. Recently, Cs has been the main focus of magnetometers for EDM measurements because sufficient vapor density is attainable at low temperatures and due to the availability of optical components including diode lasers and optical fibers~\cite{AfachBisonWeisPSIMagnetometry}. Typically a Cs optical magnetometer uses glass cells with spin polarized or aligned vapor. The magnetic field is determined from the frequency of a resonance or free-precession signal read out via transmission of resonant polarized light or optical rotation of off-resonant light.
%

Several magnetometer schemes are feasible, and the most common are called $M_x$ magnetometers~\cite{rf:BloomOriginalMangetometer,PhysRevLett.6.280} and NMOR sensors~\cite{ref:pustelny}.  $M_x$ sensors monitor the Larmor frequency of atomic spins in a static field, {\it e.g.} along $\hat z$, using a perpendicular  oscillating field along $\hat  x$ tuned to the magnetic-resonance frequency of the atomic species. 
 %
Though $M_x$ magnetometers generally have simpler design and better stability over longer times, the  RF magnetic fields may lead to cross talk among multiple sensors placed in close proximity to each other~\cite{rf:AleksandrovCsMx}. Metal cans are used as Faraday shields to mitigate these cross talk effects, but introduce magnetic-field  Johnson noise. %Recently, a light-shift driven $M_x$ scheme has also been demonstrated \cite{ref_lightshift_mx}.
%Magnetometers without RF coils, where AC Stark shift of a modulated laser beam is used for excitation have also been demonstrated.
%
NMOR  refers to Nonlinear Magneto-Optical Rotation of linear polarization, which can be used for magnetometry when the light is modulated in frequency or intensity at the Larmor frequency.
NMOR is a fully optical technique, and sensors can be built without any metallic components. Additionally, the atoms can be prepared with an alignment, a distribution of magnetic sub-levels with a magnetic quadrupole moment but no magnetic dipole moment, and  magnetic cross talk to other sensors is significantly reduced.
%W. Bell and A. Bloom, �Optically driven spin precession,� Phys. Rev. Lett., vol. 6, no. 6, pp. 280�281, Mar. 1961.
%

For Cs magnetometers, the frequencies for a typical magnetic field of 1 $\mu$T are 3.5 and 7~kHz for $M_x$ and NMOR modes respectively.  %\cite{ref:csgyromagn}.
%
Typical response times are on the order of 10-100~ms. Operation modes include continuous pumping at the resonance frequency, self-oscillation, and free precession decay, depending on the type of information needed~\cite{magnetometry_modes}.
%
%
 %For a comprehensive overview see e.g. ref.~\cite{magnetometry_modes}.
%
In particular, free precession decay is systematically cleaner due to smaller interactions with the pump laser, whereas forced oscillation or self-oscillation may be affected by light shifts, a modification of the atomic Hamiltonian in the presence of the near-resonant light, which takes the form of an effective magnetic field~\cite{ref:cohen-tann}. 
%
For example, a drift of the laser power, frequency, or polarization would change the light shift leading to instability of the magnetometer~\cite{Grujfa2015}.
%
A quantitative study of light shifts for Cs magnetometers has been undertaken by~\textcite{brian_lightshifts}.
%
%

The walls of the evacuated alkali metal vapor cells are generally coated with paraffin or other materials to improve wall-relaxation times~\cite{SingCsCoating1972}. Transverse spin-coherence times $T_2^*$ of 1-2~s, were observed for paraffin~\cite{ref:alexandrov,ref:alexandrov-erratum}, and   as long as 60~s in alkene-coated cells~\cite{ref:balabas}.\footnote{In this review, we characterize longitudinal relaxation due to wall interactions, magnetic field gradients, collisions ({\it e.g.} dipole-dipole), and weakly bound molecules as $T_1$. Observed decay of spin coherence or transverse relaxation, which may be due to longitudinal effects as well as magnetic field gradients and collisions, as $T_2^*$.}   Spin exchange between atoms is a fundamental limitation that has been suppressed by increasing the alkali-metal density or attaining very high polarization in spin-exchange relaxation-free or SERF magnetometers~\cite{RomalisSERFNature}.  Even the longest observed $T_2^*$'s are much less than the duration of a typical EDM measurement, and many independent measurements are in effect  added incoherently to obtain the magnetic-field ($B$) sensitivity for a measurement time $\tau$:
\be
\sigma_B\approx  \frac{1}{2\pi\gamma}\sqrt{\frac{1}{N_A T_2^*\tau}},
\label{eq:CsSensitivity}
\ee
%
where the gyromagnetic ratio is $\gamma=3.5$ kHz/$\mu$T for Cs and $N_A$ is the effective  number of spins observed in the time $T_2^*$.  For observation times $\tau$ up to about 10 s, the typically observed sensitivity of alkali-metal magnetometers is~\cite{ref:csintrinsicsensitivity}
\be
\sigma_B\sim~1-10~\frac{\rm fT}{\sqrt{\rm Hz}}\times\frac{1}{\sqrt{\tau}}.
\ee 
The dependence on $\tau$ of Eqn.~\ref{eq:CsSensitivity} does not hold for times greater than about 10-20 seconds due to many sources of instability, {\it i.e.} drifts in Cs density,  laser intensity, light polarization, fibers and electronics as well as, for example temperature dependent interactions of the Cs spins with the vapor cell walls.
%

Magnetometers that measure a single frequency proportional to the magnitude of a magnetic field provide information that is intrinsically a scalar.
%
Several techniques to extract vector information have been developed by  \textcite{ref:pustelny} and \textcite{Afach:15}, among others. In addition, approaches that use light shifts along different directions to modulate the vector information are very promising~\cite{ref:lightshift_cs}.
%
Laser-driven Cs atomic magnetometers can also be  operated in an array, driven by the same laser which can mitigate common-mode noise and drifts, though the possibility of cross talk between magnetometers requires care in the deployment.

\subsection*{Nuclear-spin magnetometers: $^3$He and $^{199}$Hg}

Optically pumped external and internal magnetometers with either $^3$He or $^{199}$Hg have also been proposed for neutron EDM measurements by~\textcite{rf:Ramsey1984} and studied by \textcite{GREEN1998381} and~\textcite{BORISOV2000483}. Several  planned future room-temperature neutron EDM experiments also plan to use  $^{199}$Hg as a comagnetometer, and the SNS nEDM experiment will use $^3$He as a comagnetometer as well as the detector, as described in Sec.~\ref{sec:NeutronEDM}.
 % at some stage of the development.

 For nuclear-spin magnetometers, the spins are prepared using optical pumping techniques and then set to precess freely in the magnetic field. For $^{199}$Hg, the free precession is monitored by the transmission of linearly-polarized or circularly-polarized light~\cite{GREEN1998381}. For $^3$He, the precessing magnetization at $\approx$300 $\mu$T has been monitored by inductive pickup~\cite{rf:Rosenberry2001}. For the lower fields used for neutron EDM measurements non-inductive sensors  such as SQUID magnetometers~\cite{Allmendinger:2013eya,Kuchler:2016eik} or Cs magnetometers~\cite{ref:heil3he} have been used to monitor $^3$He precession. 
%
%The magnetometers are based on work with discharge lamps by ref.~\cite{ref:cohen-tann} and on work with with lasers by ref.~\cite{rf:Romalis2001}.
%
%For $^3$He and $^{199}$Hg nuclear spins, the transverse polarization lifetimes $T_2^*$  on the order of the duration of a neutron EDM measurement or longer
%

%
The practical sensitivity limit of $^{199}$Hg magnetometers is a few fT for a 100~s integration time using both linearly and circularly polarized light. For $^{199}$Hg, the Larmor frequency is 7.79~Hz$/$$\mu$T, % \cite{ref:hggyromagn},
 which limits the bandwidth for monitoring magnetic field variations to about 1 Hz. %Details of the $^{199}$Hg comagnetometer for the ILL neutron-EDM measurements are provided in Sec.~\ref{sec:nEDMcoMag}.
%\subsubsection*{Magnetometers for Neutron EDM Experiments}
%
%
In the ILL-Sussex-Rutherford neutron EDM  $^{199}$Hg,   the $^{199}$Hg  coherence time $T_2^*$ was observed to drop  to about 60~s when the high voltage was applied~\cite{GREEN1998381,nEDMHarrisPhysRevLett.82.904}. 
Cleaning the walls with an oxygen discharge at 1~Torr 
 with regular reversal of E-field  restored the $^{199}$Hg $T_2^*$ to 400 s~\cite{GREEN1998381}. 
Other ways to improve this behavior are being investigated, including adding helium as a buffer gas to reduce the Hg mean free path which reduces the rate of depolarizing wall collisions. Unfortunately, introducing helium gas also reduces the high voltage breakdown strength of the storage cell. %Other ways to improve this behavior are being investigated,  
%\cite{ref:hg_hv_improvements}, 
%including adding helium as buffer gas to reduce the Hg mean free path. Introducing helium gas also affects the high voltage breakdown strength of the storage cell. Another advantage of helium buffer gas is that the rate of depolarizing wall collisions of Hg atoms is reduced by the shorter mean free path.
%
%
%
%
The use of a $^{199}$Hg comagnetometer further reduces the choice of wall coatings for neutron storage, as any contact with metallic surfaces will generally cause loss of $^{199}$Hg polarization. %The presence of Hg also appeared to lead to increased breakdowns for voltages above about 150 kV.
%
%
%
Light shifts during readout can also affect $^{199}$Hg magnetometers. For readout using a resonance lamp, the width of the emission spectrum averages over the light shift. For laser readout,  light-shift effects for Hg can be mitigated in several ways including reducing the readout laser power, free precession or relaxation in the dark, for which the laser interrogates the $^{199}$Hg phase for short times ({\it e.g.} 15 s) at the beginning and end of the free-precession period, or detuning the laser to the  zero light shift point~\cite{Griffith:2009zz}.
\label{sec:LightShift}

 
%
%The $^{199}$Hg EDM experiment \cite{rf:Romalis2001} also used the correlation of a stack of Hg cells  is extremely valuable to determine the field configuration and temporal stability.
%
%A further aspect is the variable size of the cells, as well as the variable pressure inside the cells, which enables operation also in a comparably large gradient.
%
%
%Hg magnetometers also have been extensively studied for its use as a comagnetometer \cite{Green:1998rg}, see Sec.~\ref{sec:NeutronEDM}
%{$^3$He cells are discussed in combination with Cs or SQUID sensors to measure the flux on top and bottom of a vertical stack of EDM chambers \cite{ref:heil3he}
%
For $^3$He transverse-spin lifetimes $T_2^*\sim 10^3$~s are easily possible and have been demonstrated at a few mbar pressure even for cells with volume comparable to the dimensions of a neutron EDM chamber~\cite{ref:heil3HeT2}. Higher $^3$He pressure is generally necessary for  detection of the polarization by a SQUID sensor or Cs magnetometer. 
%
%The pressure of magnetization of $^3$He can affect the experiment. 
For neutron EDM sensitivity of 10$^{-28}$\ecm,  $^3$He pressure of few mbar is a possible compromise. 
In the planned SNS neutron EDM experiment, the density of highly polarized $^3\mathrm{He}$ is much less, as discussed in Sec. IV.A.
%However in the planned SNS neutron EDM experiment, the density of highly polarized $^3$He optimized for the neutron storage time is much less, as discussed in Sec.~\ref{sec:NeutronEDM}. 
%
%
%With a $^3$He cell, a wide variety of cell shapes and sizes can be chosen due to the low wall relaxation rates.
%
%With several He cells, the issue of cross talk between the cells is relevant.


%


Two-photon magnetometry was originally suggested for $^{129}$Xe as an alternative to $^{199}$Hg due to the smaller neutron-absorption cross section~\cite{SkylerThesis}. Since the 147 nm single photon transition in Xe is too far in the UV, two 256 nm photons could be used for the free-precession measurement~\cite{SkylerThesis}. For $^{129}$Xe magnetometry, the spin polarized gas sample would be prepared by SEOP~\cite{rf:RosenRSI}. 
%Subsequently the proponents of the KEK/TRIUMF neutron-EDM experiment (not related to the TRIUMF RadonEDM experiment) recognized that a major systematic uncertainty due to the phase accumulated by the spins as they move in the presence of the electric field and magnetic field gradients (the so-called geometric phase~\cite{rf:PendelberryGP}) could be tuned by changing the xenon pressure and therefore the self-diffusion constant~\cite{rf:TRIUMFEDM}. 
Ultrafast two-photon spectroscopy~\cite{FreqComb} has the advantages of high peak intensity (two-photon absorption is proportional to the intensity squared) and coherence, which allows two photons of different frequencies to combine thus probing all atoms within the Doppler profile. Homogeneous broadening due to buffer gas collisions also helps cover the Doppler profile. 
%We have set up a proof-of-principle experiment using an Yb  atomic beam, which provides 2-photon transitions in the near IR and a simpler laser setup. 
Both the magnetometry signal and the efficiency for harmonic generation of the UV light depend on the laser's peak intensity, resulting in a considerable enhancement for an ultra-fast pulsed laser compared to a continuous-wave laser of the same average power. Picosecond pulses are a good compromise between high peak intensity and lower damage to crystals and optics since the dominant damage mechanisms scale with average intensity~\cite{SkylerThesis}. 
%This led to a collaboration with Michigan Professor Aaron Leanhardt who had developed a program with Yb, a two-electron atom which has atomic structure similar to Xe, with much more accessible laser wavelengths (808 nm for Yb vs 256 nm for Xe), though Yb is chemically quite different. This also led to our collaboration with the Fierlinger group in Munich to develop this concept for the nEDM efforts. We have been working to adapt a laser built for $^{199}$Hg magnetometry at 254 nm to 256 nm. As of the submission date of this proposal, we have made significant progress on a proof-of-principle experiment with Yb as well as designing the laser that could be used for both Xe (256 nm) and Rn (257 nm). We plan to submit the proof of principle paper to the arXiv as soon as possible.
One particularly important feature of two-photon-magnetometry is the spatial resolution possible due to the quadratic dependence of the scattering rate on laser intensity for two-photon transitions.  %which requires  when atom diffusion is restricted by the buffer gas. %\begin{equation}
%\label{e:scatteringrate}
%$    \mathcal{R}(\delta)  \approx \rho V \frac{|\mu_{01}|^2 |\mu_{12}|^2}{\hbar^4 \Delta^2 \Gamma} \left(\frac{I}{\epsilon_0 c}\right)^2 \frac{1}{1 + \left(\frac{\delta}{\Gamma}\right)^2}, $
%\label{e:refractiveindex}    
 %   n(\delta)  \approx  1 + \frac{\rho}{\epsilon_0} \frac{|\mu_{01}|^2 |\mu_{12}|^2}{\hbar^3 \Delta^2 \Gamma} \frac{I}{\epsilon_0 c} \frac{\frac{\delta}{\Gamma}}{1 + \left(\frac{\delta}{\Gamma}\right)^2},
%\end{equation}
%where $\rho$ is the atomic number density in the probe volume $V$, $\mu_{01}$ and $\mu_{12}$ are reduced $E1$ matrix elements for the $J=0 \rightarrow J'=1$ and $J'=1 \rightarrow J''=2$ transitions, respectively, $\Gamma$ is the decay rate of the $J''=2$ state, $\Delta$ is the detuning of the laser from the single-photon $J=0 \rightarrow J'=1$ resonance, $\delta$ is the generally small detuning of the laser from the two-photon where $\delta$ is the detuning of the two-photon energy from the transition energy, $\Delta$ is the detuning of {\it half} the energy from an intermediate opposite-parity ($p$) state, $\Gamma$ is the inverse lifetime of the upper-energy star $\mu$'s are $E!$ matrix elements and I is the intensity. 
Since the laser beam can be focused to a small waist and high intensity along the propagation direction, the scattering rate is highly position dependent and can be used to map out a magnetic field with resolution on the order of 1 mm.
 
\begin{figure}[tb]
\includegraphics[width=3.5 truein]{magnetometers}
%\includegraphics[width=3.5 truein]{sensitivity_accuracy}
\caption{\label{fig:magnbandwidths} % (Color online)
Sensitivity vs. accuracy of sensors used for low-field magnetometry in EDM measurements in typical state-of-the-art configurations: flux-gate (FG), $^3$He, alkali-metal magnetometers with Cs and K, spin-exchange-relaxation-free (SERF) mangetometers and combinations of $^3$He and Cs or SQUID readout. The term ``sensitivity'' is used to describe the response of the sensor to changes in the magnetic field, whereas ``accuracy'' is used here to describe the  absolute accuracy for measuring a fields, which is a measure of the stability, which is crucial for EDM experiments.  %Different magnetometers relevant for EDM measurements and other low-field magnetic resonance applications. In the upper part of the graph, the capability of measuring changes of the magnetic field is shown vs. the capability of measuring absolute fields, where we assume "typical" realizations of such magnetometers; the lower part shows the typical frequency ranges, where the precision is obtained.
}
\end{figure}
 
 
\subsection{Magnetic field-coil design and current sources}
 
The stability and uniformity of the applied magnetic field is essential, and %. Sufficiently uniform fields based on well-known techniques are practical, but 
 EDM experimenters have brought new innovations to the design of magnetic field coils and current sources. 
 %Stability requires both magnetic shielding discussed above, stability of the current generating the applied field, and magnetometry. 
 The basic cylindrical coil is often called the cosine-theta coil, which is based on the principle that the surface current density for a uniformly magnetized cylinder with magnetization $\vec M$ is $\vec J=\vec M\times \hat n$, {\it e.g.} for $\vec M=M\hat y$,  $|J_z|\propto|\hat y\times \hat n|=\sin(\frac{\pi}{2}-\theta)=\cos(\theta)$ ($\theta$ is the angle measured from the $x$-axis). For a finite length cylinder, the end-cap wires can be positioned to reduce end effects as determined by calculations. Often permalloy is used as a flux return, for example as shown in Fig~\ref{fig:B0Coil}, which has the feature that the field can be aligned to $\approx$~0.1 mrad for $\mu$T fields~\cite{PatentIAltarevetal}.
\begin{figure}[tb]
\includegraphics[width=3.6 truein]{B0Coil} 
\caption{\label{fig:B0Coil}   (Color online) $B_0$ coil from the panEDM apparatus. The coil consists of a closed box of mu-metal with a mu-metal cylinder 
inside the shield. The main coil is wound around the cylinder (see text), which induces azimuthal magnetization in the cylindrical shell; 
correction coils at the ends of the cylinder compensate for the finite length of the windings and for return current paths. Additional correction coils indicated account for imperfections in the geometry.
}
\end{figure}
%
 
Another very interesting approach to coil design for a source free volume ($\vec\nabla \times \vec H=0$) is provided by solving Laplace's equation for a magnetic scalar potential $\phi_B$ that satisfies the requirement, for example, of a uniform field $\vec B=B_0\hat z$ within a cylindrical volume. The currents on the surface of the volume correspond to equipotential lines, {\it i.e.} the current-carrying wires should run along equally spaced equipotential contours. For the cylinder with uniform field, this of course also corresponds to the cosine-theta configuration; however this approach is particularly useful for different shapes of coil form, {\it e.g.} a rectangular box and other field profiles~\cite{2017NIMPA.854..127M,rf:ChrisCrawfordPC}.

 Ultra-stable current sources designed for very specific currents and loads  have been developed using standard techniques of PID (proportional-integral-differential) or PI feedback. Current sensing with low-temperature-coefficient resistors compared to ultra-stable voltage references, selecting  discrete components to optimize offset drifts, tuning the gain-bandwidth product for different stages of the circuit and temperature stabilization of crucial components can provide routine performance of $10^{-7}$ and better. Feedback from magnetometers has been used to effectively stabilize magnetic fields to one part in 10$^{11}$ over 1000 second time scales~\cite{rf:Rosenberry2001}.

 
%As a side note, the characterization of field coils suitable for next-generation measurements requires  a vector map over an extended volume with a angular resolution of 100 $\mu$rad. The precision of most components that meet the requirements of low magnetic impact is typically limited to  an angular precision of few $ \times 10^{-3}$ rad. Applying time dependent fields in different spatial directions may be used to help calibrate the probes, but the procedures  remain difficult.
 
\subsection{Ultra-cold neutron sources}
\label{sec:UCNSources}
Ultra-cold neutrons (UCN), introduced by \textcite{ref:originalucn,ref:originalucn-ru} have velocities less than about 7 m/s, corresponding to kinetic energies $E_\mathrm{kin}< \sim$ 260~neV, temperatures of mK and wavelengths of tens of nm~\cite{book:golub_ucn}. As a result of the long wavelength, the interaction of UCN with material surfaces is characterized by a potential energy called the Fermi energy $V_F$ that is positive for most materials. Neutrons with kinetic energy less than $V_F$  are repelled from the chamber walls for any angle of incidence and thus can be stored in a bottle or cell. In Table~\ref{tb:UCNProperties} we list UCN properties. The gravitational and magnetic potential energies  for maximum UCN kinetic energy correspond to 2.5 meters height and magnetic field 4.3 Tesla, respectively. Thus, for example, neutrons can be stored in a gravitational bottle a few meters deep and can be polarized by reflecting one spin state from magnetic field barrier of $\approx$ 5 T.
%Table~\ref{tb:UCNProperties} summarizes properties of UCNs relevant to EDM measurements. 
%A more comprehensive discussion of the properties of UCN can be found in ref.~\cite{book:golub_ucn}.
%
%
%The upper limit in kinetic energy corresponds to an effective potential of materials $V_F$, which in most cases repells neutrons from surfaces.
%
%
%This shows that UCNs can be gravitationally confined in an  approximately 2.5 m deep gravitational well and by a 4.33 T magnetic bottle.%Both properties are close to the typical kinetic energies for experiments with technically realizable dimensions or magnetic fields.

\begin{table}
\begin{centering}
\begin{tabular}{|l|c|c|c|}
\hline\hline
UCN Velocity & $v_\mathrm{UCN}$ &$<$ 7 &m/s \\
\hline
UCN Kinetic Energy & KE & $<$ 260 & neV  \\
%\hline
%Neutron mass & $m_n$ & $1.6798$ kg=939.565 MeV \\
\hline
Gravitational energy& $m_n g$ & 102 & neV$/$m \\
\hline
Magnetic moment & $\mu_n$ & $-60.4$& neV/T \\
\hline
Gyromagnetic ratio & $\gamma_n=\frac{2\mu_n}{\hbar}$ & $2\pi\times$ 29.16 & MHz/T  \\
\hline
\hline
 \end{tabular}
\end{centering}
\caption{\label{tb:UCNProperties} Properties of UCN and neutrons relevant to the neutron EDM experiments. The negative magnetic moment indicates that the strong magnetic field seeking neutrons are ``spin down" with respect to the magnetic field.}
\end{table}


 
UCN are in principle present in any moderated neutron source. In thermal equilibrium, the neutron-density spectrum is
\begin{equation}
\frac{d\rho(v_n)}{ dv_n}= 2\Phi_0\frac{v_n^2}{v_T^4}\exp(-v_n^2/v_T^2)
\label{eq:thermalflux}
\end{equation}
where $v_T = \sqrt{2 k_B T / m_n}$.
Integrating the density up to the velocity corresponding to the Fermi energy for $v_F<<v_T$  results in a UCN density of
\begin{equation}
\rho_\mathrm{UCN}=\int\limits_0^{v_{F}}d\rho(v_n) \approx \frac{2}{3}\frac{\Phi_0}{v_T}\left(\frac{v_F}{v_T}\right)^{3},
\label{eq:inteflux}
\end{equation}
for $v_F=7$ m/s.
For a thermal neutron source $T=300$K, the core flux is $\Phi_0 \sim 10^{15}$~cm$^{-2}$~s$^{-1}$, and $\rho_\mathrm{UCN}\approx 100$ cm$^{-3}$.  
For cold moderators, typically at 20K, $\Phi_0$ decreases by about an order of magnitude and the UCN in the moderator is increased to 1000-2000~cm$^{-3}$. 
%
%
%While a comprehensive historical overview and a listing of pioneering early developments can be found e.g. in \cite{book:golub_ucn}, here the focus will be the operational or concepts currently being realized.

The world's best thermal UCN source is   illustrated in FIG.~\ref{fg:ILLUCNTurbine} \cite{steyerl1986}. 
%The UCNs, from the low-energy tail of the  cold moderator spectrum   up to $v_n\approx 20$ m/s gravitationally and with a recoiling turbine.  
Neutrons with $v_n < 20$ m/s are extracted from a $T=$20K liquid D$_2$ moderator through a vertically mounted neutron guide, slowing as they rise 17 m in the gravitational potential. The neutron guide is a tube that transports neutrons using total reflection from the surface, with the criterion being the normal component of the velocity satisfies $v_\bot \leq \sqrt{2V_F/m_n}$. At the end of the guide, the neutrons reflect from the receding copper blades of a``turbine." The turbine blades act as moving mirrors that shift the neutrons to lower velocity. After several recoils, neutrons exit the turbine. Densities up to 10 UCN~cm$^{-3}$ are available for experiments.%\cite{rf:ILLYellowBook}. 
%and densities of few trapped UCN cm$^{-3}$ for the EDM apparatus have been achieved.
 

 
\begin{figure}[tb]
\vskip -.35 truein
\centerline{\includegraphics[width=5truein]{PF2NEW}}
\vskip -.35 truein
\caption{\label{fg:ILLUCNTurbine}  (Color online) UCN source at PF2 at the Instiut Laue Langevin in Grenoble, France (ILL). Neutrons from the low-energy tail of the cold-neutron spectrum are guided upward and lose energy in the gravitational potential. The turbine further shifts the  spectrum to longer wavelengths to produce UCN that are provided to a number of experiments including the EDM. Figure provided by ILL~\cite{rf:ILLYellowBook}.
}
\end{figure}


Most modern UCN sources  are based on
superthermal conversion introduced by~\textcite{ref:golub_pendlebury_superthermal}. This achieves higher phase-space density than thermal sources  using a medium that is not in thermal equilibrium with the neutrons.
%
For example, FIG.~\ref{fig:4hedispersion} shows the dispersion curves for a free neutron and for thermal excitations in superfluid-helium (SF-He), a typical choice for a superthermal source material. The curves cross at two points: $E_0=0$ and $E_1=E_0 + \Delta$, where $\Delta\approx 1$ meV corresponding to neutron wavelength $\lambda_0= 8.9$ \AA~\cite{book:golub_ucn}. This is effectively a two-state system, and neutrons at $E_1$ can resonantly transfer their energy to the SF-He resulting in final UCN energy $E_{UCN}\approx E_0$ with a small spread  due to the width of the excitations. The process $E_\mathrm{UCN} +\Delta  \rightarrow E_\mathrm{UCN}$, called ``down-scattering," is effectively independent of SF-He temperature $T$. The reverse process - a UCN absorbing energy $\Delta$ from the SF-He thermal bath (``up-scattering")  is exponentially suppressed for $\Delta \geq k_BT$ according to  %The idea is best illustrated using a medium with two energy states , a ground state at energy E$_0$ and a state at an energy E$_1$ $=$ E$_0 + \Delta$ elevated from the ground state, exposed to a thermal or cold neutron (CN) flux.
%After being predicted by \cite{golub_helium} and experimental demonstration \cite{golube_he_source}, this concept has been further advanced to actually produce UCN.
the principle of detailed balance, which gives the ratio of the up-scattering and down-scattering cross sections~\cite{book:golub_ucn}:
%\begin{equation}
\be
\frac{\sigma(E_\mathrm{UCN} \rightarrow E_\mathrm{UCN} +\Delta)} {\sigma(E_\mathrm{UCN} +\Delta  \rightarrow E_\mathrm{UCN})}= 
\frac{E_\mathrm{UCN}+\Delta}{E_\mathrm{UCN}}e^{-\Delta/k_\mathrm{B}T}.%\sigma(E_{UCN} +\Delta  \rightarrow E_{UCN})
\label{eq:detaieledbalance}
\ee
%\end{equation}
%
The accumulation of UCN and increase of the phase space density of the neutrons does not violate Liouville's theorem, because the UCN and excitations in SF-He are both part of the same thermal system.
Producing SF-He requires temperatures $T<2.17$~K. However due to up-scattering $T\leq$~0.6~K is optimal for UCN production.  
%
%
%It can be seen that the momentum of a free neutron crosses the excitation curve at 8.9~\AA. After being predicted by \cite{golub_helium} and experimental demonstration \cite{golube_he_source}, this concept has been further advanced to actually produce UCN.
%
\begin{figure}[tb]
\begin{tabular}{l}
\hskip-0.5 truein
\includegraphics[width=4 truein]{phonon2}\\
\end{tabular}
\vskip -0.55 truein
\caption{\label{fig:4hedispersion}  (Color online) Single phonon dispersion curve for SF-He with a minimum at ~$k\approx$ 2 nm$^{-1}$ and the free neutron $E_n=(\hbar k)^2/2m_n$. The two curves intersect at $E_n=0$ and $E_n$=1 meV corresponding to $k$=0.7~\AA$^{-1}$ or $\lambda_0=8.9$ \AA.
}
\end{figure}
%



The equilbrium UCN density for a given UCN lifetime $\tau_\mathrm{tot}$ inside an SF-He source is $\rho_\mathrm{UCN}= \tau_\mathrm{tot} R_I$, where the UCN production rate per unit volume for SF-He density $\rho_\mathrm{SF}$ and incident cold-neutron differential flux  $\frac{d\Phi_0}{d\lambda}$  is 
\begin{equation}
R_I=\rho_\mathrm{SF} 
\int\ \frac{d\Phi_0}{d\lambda} \sigma \left ( \lambda \rightarrow \lambda_\mathrm{UCN} \right ) \ d\lambda.
\end{equation}
For incident neutron wavelength near $\lambda_0$=8.9~\AA, and assuming a chamber with $V_F$= 252 neV ({\it i.e.} Be), the theoretical UCN production rate based on the combined calculations of \textcite{ref:golub_pendlebury_superthermal} and \textcite{YOSHIKI2003} is $R_I = \left(4.55 \pm 0.25 \right) \!\times\! 10^{-8} d\Phi_0 /d\lambda |_{\lambda_0}$ cm$^{-3}$ s$^{-1}$, for $d\Phi_0 /d\lambda$  in units of  neutrons cm$^{-2}$ s$^{-1}$ \AA$^{-1}$. 
\textcite{baker_phys_lett_a_308_67_2003} measured the production rate for a narrow-band neutron beam near 9~\AA, which, when combined with the measured incident flux, is interpreted as $R_I = \left ( 3.48 \pm 0.53 \right ) \!\times\!10^{-8}d\Phi_0 /d\lambda |_{\lambda_0}$ cm$^{-3}$ s$^{-1}$.


%The UCN density for a UCN lifetime $\tau_{tot}$ inside a SF-He source is 
%%\red{can we write this in terms of $\lambda$ - need to write $\Phi_0=2.62\times 10^7$ etc.}
%\begin{equation}
%\rho_\mathrm{UCN} = \tau_\mathrm{tot}\ N_\mathrm{SF} \int{   \frac{d\Phi_0}{d\lambda} \sigma(E_0%{UCN} +\Delta
%  \rightarrow E_{UCN}) d\lambda},
%\label{eq:densityinhelium}
%\end{equation}
%where  $N_\mathrm{SF}$ 
%%\red{need subscript?  $N_{\rm SF}=$, also can we use $\rho_{\rm SF}$} 
%is the SF-He density. 
%The specific production rate $N_\mathrm{SF}  \sigma=%(E_0%{UCN} +\Delta
 % \rightarrow E_\mathrm{UCN})$ 
  %been calculated by~\textcite{ref:golub_pendlebury_superthermal} to be $N_\mathrm{SF}\sigma=(4.55\pm0.25) \times 10^{-8} \frac{d\Phi_0}{d\lambda}$ at $E_0=8.9\ $\AA, compared to 
%  (3.47\pm0.13) \times 10^{-8} \frac{d\Phi_0}{d\lambda}$ was
 %  measured by~\textcite{baker_phys_lett_a_308_67_2003}. 
%\red {The lower measured value can be explained by  taking into account the wall potential of helium (V$_{F,He}$~19~neV) and an upper limit for the stored energy given by a highly reflective  Be surface, providing a total corrected UCN production rate of $(6.9\pm 1.7)$ cm$^{-3}$ s$^{-1}$~\cite{Sun2PRC}.}
%
 %%%%%%%%%%%%%%%%%%%%%%%%%%%%%%%%%%%%%%%
 %
 
 
Minimizing losses during UCN production is critical for achieving high UCN densities.
%
The  loss rate is ultimately limited by the neutron lifetime $\tau_n$ with additional contributions for a total loss rate:
\begin{equation}
\frac{1}{\tau_\mathrm{tot}} = \frac{1}{\tau_\mathrm{up}} + \frac{1}{\tau_\mathrm{walls}} + \frac{1}{\tau_\mathrm{slits}} + \frac{1}{\tau_\mathrm{abs}} + \frac{1}{\tau_{n}}
\label{eq:helium_loss_rate}
\end{equation}

\begin{enumerate}
\item The thermal up-scattering rate $1/\tau_\mathrm{up}$ is small at the typical operating temperature of about 0.8~K (see Eqn.~\ref{eq:detaieledbalance}), and  below 0.6~K performance does not further improve~\cite{piegsa2014}.\\
%
\item The wall collision losses $1/\tau_\mathrm{walls}  = \mu \nu$ are defined by the energy dependent parameters, loss probability per wall collision $\mu \approx 10^{-4}$  and the wall collision frequency in the trap $\nu \sim 10-50$ s$^{-1}$,  given in Table~\ref{tb:UCNMaterials}.\\
%
\item $1/\tau_\mathrm{slits}$ is related to the mechanical precision of the trap, which defines the leakage of UCN out of the trap, which has been reduced with low-temperature Fomblin oil (a fluorinated, hydrogen free fluid with low-neutron-absorption)~\cite{Serebrov2008}. \\
\item $\tau_\mathrm{abs}$ is the UCN absorption, due mostly to $^3$He in the trap. Less than part-per-billion $^{3}$He contamination in the superfluid is essential due to the strong absorption of neutrons on $^3$He.
\\
%
\end{enumerate}
In a demonstration by \textcite{piegsa2014},  total storage times of greater than 150 s and UCN densities of $\mathrm{220\ cm^{-3}}$ have been achieved by using low temperature Fomblin oil to close gaps and slits in the trap volume. The Fermi-potential $V_F\approx$ 100 neV for Fomblin oil results in a lower UCN energy spectrum. The loss of higher energy UCN is balanced by potential advantages, because losses, depolarization, and some velocity-dependent systematic effects may be smaller.

Many factors limit UCN-source performance including low Fermi potential of wall materials,  other contributions to storage lifetime, heat input due to the geometry of the incident neutron beam extraction guides, source volume dimensions, beam size, and beam divergence. 
%Longer buildup
UCN densities can be increased to 10$^3$~cm$^{-3}$ with an optimized 8.9~\AA~neutron beam, larger storage volumes, and surface-coating  improvements.

Extraction of the UCN from the source to the experiment~\cite{piegsa2014,Masuda:2002dy}
requires either a window or a vertical exit from the superfluid He chamber such as indicated in FIG.~\ref{fg:SUN1ASchematic}.
%
For a superfluid He source, the UCN are accumulated in the source volume during production and filling of the experimental volume, which  reduces the density in proportion to the ratio of the source volume relative to the total volume of the source, guides, and experiment. The experiment should therefore be placed close to the source, and the ratio of source volume to experiment volume should be as large as possible.

A second-generation SF-He  source at ILL (SUN2) is shown in FIG.~\ref{fg:SUN1ASchematic}~\cite{Sun2PRC}. The SF-He is held in  a container with inner surfaces that have a  large  Fermi-potential {\it e.g.} for beryllium $V_F= 252$ neV, or with the help of magnetically enhanced confinement \cite{Zimmer:2013xja}. The cryogenics package (not shown) dissipates 60 mW at 0.6K. 
%Less than part-per-billion $^{3}$He contamination in the superfluid is essential due to the strong absorption of neutrons on $^3$He.
Neutrons with 8.9~\AA~wavelength  enter the SF-He container through one of the UCN reflecting walls.
Neutron beams available for such sources are typically cold beams that are guided to experiments 10's of meters away from the neutron source.
%
Neutron fluxes at ILL for the H172A beamline are  $\Phi_0=2.62 \times 10^7$~neutrons cm$^{-2}$\,s$^{-1}${\AA}$^{-1}$~\cite{piegsa2014}.
%\red{a four times higher flux was available at this wavelength after a {\bf monochromator}.?} For the PF1B beam the intensity near 8.9~\AA~is estimated to be $7 \times 10^8$~neutrons cm$^{-2}$s$^{-1}$\AA\ $^{-1}$~\cite{Sun2PRC}.
 %
\begin{figure}[tb]
\vskip 0.25 truein
\includegraphics[width=3.5 truein]{UCNProductionvolume}
\caption{\label{fg:SUN1ASchematic}  (Color online) Schematic diagram of the ILL  SUN2 source described in the text. Figure adapted from~\textcite{Sun2PRC} by  K. Leung and O. Zimmer.}
\end{figure}


UCN production is a coherent phenomenon in superfluid helium. Thus the polarization of an incident neutron beam is preserved, and it is possible to produce polarized UCN from a polarized cold-neutron beam or with a magnetic reflector surrounding the UCN production volume~\cite{Zimmer:2013xja}.
%
The alternative - polarizing UCN after extraction from the source  rejects at least half the neutrons.
%, and  absorption losses from adsorbed and residual-gas hydrogen,  t polarizer foils or long and thin neutron guides for polarizer magnets.
%
 
The possibility of directly performing a UCN storage experiment inside the source has led to the SNS nEDM experimental concept described in Sec.~\ref{sec:NeutronEDM}. This would also enable the use of higher electrostatic voltages due to the  dielectric properties of the SF-He. Breakdown strengths exceeding 100~kV$/$cm have recently been demonstrated in a 1 cm electrode gap~\cite{ref:los_alamos_hvtests}.
%
%The given time-constants for accumulation of UCN in liquid helium also favor certain classes of experiments, which deploy trapped UCN and several 100~s storage times.
%
 
 
  
\begin{table*}
\begin{centering}
\begin{tabular}{|c|c||c|c||c|c|}
\hline\hline
%    Material    &V (neV)	&\multicolumn{2}{c|}{Loss per bounce} 		&\multicolumn{2}{c|}{Depolarization}  \\
    Material    &V (neV)	&Loss per bounce  & Ref		& Depolarization & Ref  \\
\hline
DPe (300K)        		  &214 		&$1.3\times 10^{-4}$ & $a$  &$4\times 10^{-6}$ & $b$ \\
\hline
DLC on Al substrate (70K)  &270 		&$1.7\times 10^{-4}$ &	$c$	&$0.7\times 10^{-6}$ & $c$\\
\hline
DLC on Al substrate (300K) &270 		&$3.5\times 10^{-4}$ &	$c$	&$3\times 10^{-6}$& $c$\\
\hline
DLC on PET substrate (70K)&242 		&$1.6\times 10^{-4}$ &	$c$	&$15\pm \times 10^{-6}$& $c$\\
\hline
DLC on PET substrate (300K)&242 		&$5.8\times 10^{-4}$ &	$c$	&$(14\pm 1)\times 10^{-6}$& $c$\\
\hline
Fomblin 300K   			      &106.5    	&$2.2\times 10^{-5}$ & $d$ &$1\times 10^{-5}$& $e$\\
\hline
%LTF 110K				      &106.5	&$2.2\times 10^{-6}$ & $d$ &$1\times 10^{-5}$& $e$\\
%\hline
Be (10 K) 				      &252 		&$3\times 10^{-5}$    & $d$ 	    &$1.1\times 10^{-5}$& $e$\\
\hline
Be (300K) 			      &252 		&$(4-10)\times 10^{-5}$& $d$  &$1.1\times 10^{-5}$& $e$\\
\hline
NiP 			      &213 		& $1.3\times 10^{-4}$ & $f$  & $< 7\times 10^{-6}$ & $g$\\
\hline
$^{58}$Ni 			      &335		&& $h$ & strong  &\\
\hline
Fe/steel/stainless			      & 180-190	&	&  $h$ & strong  &\\
\hline\hline
\end{tabular}
\caption{Properties of materials for UCN production, storage and transport showing loss per bounce and depolarization per bounce. DPe is deuterated polyethylene; PET is Polyethylene Terephthalate; DLC is diamond-like carbon. $^{58}$Ni and steel alloys are magnetic, resulting in strong depolarization of UCN. References $a$: \cite{BrennerAPL2015}; $b$:\cite{Ito2015}; $c$ \cite{Atchison2007}; $d$: \cite{Serebrov2005}; $e$: \cite{Serebrov2003}; $f$:\cite{Pat17}; $g$: \cite{Tan16}; $h$ \cite{book:golub_ucn}.\label{tb:UCNMaterials}}
\end{centering}
\end{table*} 
 
 
\begin{table*}
\begin{centering}
\begin{tabular}{|l|l|l|rl|l|}
\hline\hline
Source & Type  & Converter & &\hskip -0.17 truein UCN/cm$^{3}$  & Ref\\
\hline
%ESS     & spallation & under discussion & - & - & future\\
%\hline
%\multicolumn{6} {|l|}{{operating}}\\
\hline
ILL PF2 & reactor cold source   &   (turbine) & 2 & polarized UCN based on detected UCN  & $a$\\
\hline
LANL & spallation & sD$_2$  & 40  &  polarized UCN observed in a test chamber & $b$\\
\hline
PSI & spallation & sD$_2$ & 22 & unpolarized UCN in a ``standard'' storage bottle & $c$\\
\hline
TRIGA Mainz & pulsed reactor & sD$_2$  & 10 & unpolarized & $d$\\
\hline
ILL SUN-II & reactor cold neutron beam & SF-He & 10 & polarized UCN source production, dilution, and polarization & $e$\\
\hline
J-PARC & spallation-VCN & rotating mirror & 1.4 & unpolarized, at source & $f$\\
\hline\hline
%\multicolumn{6} {|l|}{{planned}}\\
%\hline
%\multicolumn{6}{|l|}{\bf Planned} \\%&   &  & UCN/cm$^3$ at EDM experiment & \\
%\hline
%ILL SuperSUN & reactor cold neutron beam & SF-He  & & $>$ 50-1000/cm$^3$ (note 2) & $f$\\
%\hline
%NCSU & reactor & sD$_2$  & &$>$ 1000$/$cm$^3$ & $g$\\
%\hline
%PIK & reactor & SF-He  & & & $h$\\
%\hline
%TUM & reactor & sD$_2$  & &$\sim$1000$/$cm$^3$ & $i$\\
%\hline
%KEK/TRIUMF & spallation & SF-He & &$>$ 1000/cm$^3$  & $j$\\
%\hline
%WWRM & reactor & SF-He  & &$10^4$/cm$^3$ & $k$\\
%\hline
%SNS$^*$ & spallation & SF-He  & &500$/$cm$^3$ & $l$\\
%\hline\hline
\end{tabular}
\caption{Currently operating UCN sources with UCN densities relevant to EDM experiments based on data reported in the cited references. For unpolarized UCN, the reported densities should be decreased by  a factor of at least two  
 to make a consistent comparison. %Note that the numbers are difficult to compare and relate strongly to the actual experiment. 
References: $a$~\cite{Baker:2006ts}; $b$~\cite{Ito:2017ywc}; $c$~\cite{Becker201520} ; $d$~\cite{LauerTh2013}; $e$~\cite{Sun2PRC}; $f$~\cite{doi:10.1093/ptep/ptv177}.
%Legend:  ILL PF-2 is the UCN source shown in FIG.~\ref{fg:ILLUCNTurbine} and Sun-II/SuperSun are SF-He sources at ILL; LANL refers to the Los Alamos S-D$_2$ source, NCSU is the sources at the North Carolina State University pulsed reactor, PIK is the Gatchina reactor, PSI the S-D$_2$ UCN source, SNS is the spallation neutron source with UCN production incorporated into the EDM experiment, TUM is the Technical University of M\"unchen FRM-II reactor source, TRIUMF is the SF-He source at the TRIUMF proton cyclotron, TRIGA is a pulsed reactor at Mainz University and WWRM is at PNPI is St. Petersburg.
\label{tb:UCNSourceComparisonTable}}
\end{centering}
\end{table*}
 
 
 Alternative superthermal converters include solid deuterium~\cite{Golub1983}, deuterated methane or oxygen, as well as other (para)magnetic/spin polarized substances.~\cite{superthermal_materials}.  The choice of materials for superthermal  UCN production is based on increasing the phase space density and modes of UCN extraction and are discussed in~\textcite{ref:ZimmerCooling,nesvishevski,nesvishevski-ru}.
 % \red{Is the spectrum different for SF-He and S-D2 eg.?}
 
Solid deuterium (SD$_2$) molecules consisting of two spin-one deuterons combine to total nuclear spin $I=0$ or 2 for ortho-deuterium ($o$D$_2$) and $I=1$ for para-deuterium ($p$D$_2$). The  rotational ground-state for  spin-one bosons is $o$D$_2$.(For H$_2$, the ground state is parahydrogen with $I=0$.)
%
Rotational excitations in SD$_2$ for rotational quantum number $J$ are given by:
\begin{equation}
E_J = hcB J(J+1)=3.75\ {\rm meV}\times J(J+1),
\label{eq:sd2_rot_levels}
\end{equation}
where $B$ is the rotational constant~\cite{sd2-rotational-states}. For low-lying excitations, the rotational energy is 
comparable to cold neutron energies. 
%
At low temperature, the six $I=0,2$ $o$D$_2$  states and the three $I=1$ $p$D$_2$ states are approximately equally populated, so the relative population of ortho and para states is 2-to-1. For UCN production, an $o$D$_2$ concentration $\gtrsim$~95$\%$ is necessary to suppress upscattering of UCN, absorbing energy from $o$D$_2$ to $p$D$_2$ transitions. This requires low temperatures as well as a magnetic converter to enhance  the spin-flip rate for deuteron spins in a separate apparatus outside the source. One possible converter material is hydrous iron (III) oxide, cooled to below the triple point of deuterium~\cite{ref:sd2_ortho_para_converter}. 
%
The converted $o$D$_2$ is then transferred to the UCN source in a gaseous state without losing the ortho-state configuration due to very long thermal relaxation times in the absence of a converter.
%

The conversion cross section  per unit volume from cold neutrons to UCN for a deuterium density $\rho_{\rm H_2}$ is given by~\cite{AtchisonPRL2005,Atchison2005,ATCHISON2009252}
\be
\sigma^{CN \rightarrow UCN}_{{\rm sD}_2,8K} \rho_{\rm H_2} \sim \left( 1.11\pm 0.23\right) \times 10^{-8}~{\rm cm}^{-1},
\ee
The momentum acceptance for cold neutrons is much greater for sD$_2$ than for SF-He, leading to a higher production rate in the source.  The UCN lifetime in the sD$_2$ source is given by
\begin{equation}
\frac{1}{\tau_\mathrm{UCN,sD_2}}= \frac{1}{\tau_\mathrm{up}} + \frac{1}{\tau_{o/p}} + \frac{1}{\tau_{\rm D-abs}} + \frac{1}{\tau_{\rm H-abs}} + \frac{1}{\tau_{\rm cryst}},
\label{eq:sd2ucndensity}
\end{equation}
%
where $1 / \tau_{up}$  accounts for the losses due to thermal up-scattering (elastic scattering of UCN with mK temperature on nuclei at few K temperatures in the source)~\cite{Liu2000R3581},  $1 / \tau_{o/p}$ accounts for  losses due to scattering on $p$D$_2$, $1/\tau_{\rm D-abs}$, $1/\tau_{\rm H-abs}$, and $1/\tau_{\rm cryst}$ accounts for absorption losses on deuterium, hydrogen and scattering in the crystal, respectively~\cite{Yu1986137}.
%
Most loss channels can be controlled to better than  $(150\ \text{ms})^{-1}$, resulting in an overall UCN lifetime of $\tau_\mathrm{UCN,sD_2} \approx 75\ \mathrm{ms}$ in the perfect
 $o$D$_2$ crystal  at 4~K.
%

UCN densities of 10$^4$~cm$^{-3}$ are in principle feasible inside the source. In contrast to the SF-He source, which requires accumulation in the source for several hundred seconds followed by dilution into the guides and experimental volume,  sD$_2$ provides an effectively continuous source of UCN, provided  efficient  extraction to the experiment  within a time $\tau_{\rm UCN, sD_2}$. The optimum source thickness based on diffusion of UCN in this time is $\sim$~1~cm, not taking into account pre-moderation or thermal engineering issues that may dictate increased  thickness.  Also, UCN extracted from the source should be prevented from  diffusing back into the sD$_2$, for example by closing off the source with a valve.
%
 
 
 
%
%The strongest neutron fluxes are available at reactors, where a continuous flux of neutrons is available, e.g. the PF2 instrument at ILL. Here, three experiments share a UCN beam, in addition to a continuous beam for test purposes.
%


%There is a number of different UCN source approaches developed for EDM experiments with different features and tradeoffs.
%
%A  SF-He source provides a fixed number of UCN after accumulation inside the helium, which is then diluted when the source is connected to the experiment. 
An sD$_2$ source can provide a steady flux of UCN from the source and can, in principle, be interfaced directly to the neutron-guide vacuum. However, any UCN that diffuse back to the D$_2$ will be lost due to the short UCN storage time.
%
On the other hand, the UCN density provided by a continuous sD$_2$ source does not depend on the size of the experiment (to first order), and large experiment volumes can be filled with UCN.
%
%Typical UCN storage experiments like a neutron EDM measurement use a chamber, which is filled for 10's of seconds, and the UCNs are stored for 100's of seconds.
%
For EDM experiments, the duty cycle of the source can be less than 10\%, as the filling times of UCN storage chambers are typically 10-100~s and the cycle time $\approx$~1000~s,
%
%Therefore, also a source that provides UCN for a fraction of the whole time is suitable for an EDM measurement.
%
which is important for accelerator-based spallation neutron sources that share the primary accelerator beam.%

%
At the Los Alamos National Lab, spallation  neutrons produced by a pulsed 800-MeV proton beam striking a tungsten target are thermalized by beryllium and graphite moderators at ambient temperature and further cooled by a cold moderator that consists of cooled polyethylene beads~\cite{Sau13}. The cold neutrons are converted to UCN by down-scattering in an sD$_2$ crystal at 5 K.  UCN are vertically extracted  in a 1 meter long $^{58}$Ni guide that compensates for the 100-neV boost that UCN receive when leaving the sD$_2$.  A valve between the source and vertical guide is closed except when the proton beam pulse is incident in order to keep the UCN in the guide from returning to the sD$_2$ and being absorbed.
When in production, the peak proton current from the accelerator was typically 12 mA, delivered in bursts of 10 pulses each 625~$
\mu$s long at 20 Hz, with a 5 s delay between bursts; the average current delivered to the spallation target is 9 $\mu$A. 
%The total charge delivered per burst was ?45 ?C in 0.45 s. The time averaged current delivered to the target was ?9 ?A.
After recent upgrades, the UCN density measured at the end of a 6 meter long stainless steel guide was $\left(184\pm32\right)$  cm$^{-3}$, and a measurement made with {\em polarized} UCN in a bottle, similar to what would be used for an EDM measurement, yielded  $\left(39\pm 7\right)$ UCN cm$^{-3}$~\cite{Ito:2017ywc}.
%A density of $52\pm9$~UCN/cm$^3$ has been seen at a port outside of the biological shield~\cite{Saunders2012}, and an upgrade providing significant improvement has been reported~\cite{Ito:2017ywc}.
SF-He sources are also planned at PNPI in St. Petersburg, which calculations suggest may produce 10$^4$ UCN/cm$^3$~~\cite{Serebrov:2017gea}.

At PSI~\cite{Anghel2009272}, the cyclotron provides  a 2.2 mA,  590 MeV proton beam incident on a heavy metal spallation target for 4~s with a duty cycle of 1$\%$. Measurements with a ``standard'' storage bottle showed 22 UCN/cm$^3$ of unpolarized UCN~\cite{Becker201520}.
%to produce neutrons close to the super-thermal SD$_2$ UCN converter .
%
%The ILL-Sussex-Rutherford EDM apparatus was moved to PSI to use neutrons from the super-thermal sD$_2$  converter with reported performance comparable to the previous densities achieved at ILL~\cite{Becker201520}.
%
 Another pulsed sD$_2$ source is installed at the pulsed TRIGA reactor in Mainz, where a 200~MW-20~ms neutron pulse produces UCN, which leave the source quickly and can - with proper timing - be captured in an experiment using mechanical shutters. A density of order 10~UCN-cm$^{-3}$ in a few-10's of liter volume was achieved~\cite{LauerTh2013}.

%\subsubsection*{Scaling the sources to high neutron flux}

Though the typical lifetimes in sD$_2$ and the energy dependent production rate have been well determined experimentally, the  performance of solid deuterium based sources is typically lower than expected for operating sources (see Table~\ref{tb:UCNSourceComparisonTable}). Several issues related to the $\mathrm{sD_2}$ crystals have been considered in an effort to explain this discrepancy. For example, in practical sources, shrinkage of the crystal during the cool down may cause cracks and may also reduce thermal contact to the cooling apparatus.
%
Cracks in the crystal also change the neutron mean-free path in the source and thus may affect the extraction efficiency.
%
Thermal stress and geometrical alignment of the source may also require preparation of the crystal from the gas phase instead of freezing from the liquid state.
%
 Another important consideration is guiding the neutrons from the sD$_2$ source to an experiment.
Until recently, UCN transmission in long guides had been a limitation, but ~\textcite{ref:zechlau}   measured transmission greater than $>$~50$\%$ for   80 mm ID,  22 m long guides coated with NiMo (in the ratio 85:15) and including three 90$^\circ$ bends.

 
In Table~\ref{tb:UCNSourceComparisonTable}, we list the currently operating UCN sources and an estimate of the performance relevant to the neutron EDM experiments based on measurements reported in the papers cited. 
%These quantities should take into account losses of UCN during extraction from the source and during transport to the experiment. 
In some cases, the number of polarized neutrons was reported, as noted; for unpolarized UCN, the reported densities should be decreased by at least a factor of two to account for the loss of one spin state.   


A number of additional sources based on both SF-He and sD$_2$ are under construction or planned. The  operating SF-He sources are in principle working at the level of theoretical estimates. For these sources, scaling to larger neutron flux would result in larger heating from the primary beam and require increased cryogenic capacity, but  is not expected to significantly change the performance. At ILL, the SuperSUN source has significantly larger converter volume, cryogenic capacity, and will incorporate the magnetic reflector~\cite{Zimmer:2013xja} with the expectation of up to 1000 polarized UCN/cm$^3$. 
 At KEK, a SF-He based source using proton-spallation production of cold neutrons provide 0.7 unpolarized UCN/cm$^3$ with 78 Watts of proton power~\cite{Masuda:2002dy}. This source has been moved to TRIUMF, where orders of magnitude higher proton-beam power will be available. Upgrading the cryogenics to handle higher power is predicted to provide 100's of UCN/cm$^3$ for experiments.  
 The concept of the nEDM experiment under development for the Spallation Neutron Source at Oak Ridge National Lab is to produce the UCN within the EDM experiment as demonstrated by~\textcite{OShaughnessy:2009amn} for a neutron-lifetime measurements as discussed in detail in Sec.~\ref{sc:nEDMprospects}.
 
Plans to adapt the converter design of the sD$_2$ source at the Mainz TRIGA reactor to the strong core-flux from the FRM-II reactor at the Technical University of M\"unchen FRM-II, are expected to produce~1000 UCN/cm$^3$.  The TRIGA reactor operates at a lower average power; therefore scaling to higher power and radiation dose needs to be explored. An sD$_2$ source is under development at the  PULSTAR reactor at North Carolina State University that is predicted to provide 40 UCN/cm$^3$ for 1 MW~\cite{KOROBKINA2014169}.   %Larger incident neutron flux has apparently produced radiation damage in the crystal \cite{ref:wignereffect}. The 
%\red{The source at PSI is not yet performing as expected and an explanation has not yet been published. BE MORE OPTIMISTIC/DIPLOMATIC} 
%The recent upgrade of the LANL source demonstrates that the initial issues of these sources are mostly technical and can be mitigated~\cite{Ito:2017ywc}. 

%All future sources have the common aspiration of providing a density of 10$^3$~UCN cm$^{-3}$ or even higher inside experiments attached to the source.
%
%The time scales of all projects slipped significantly in the past, but approximately half of these projects have time-scales of less than three year
 The next major neutron source will be the European Spallation Source (ESS). The ESS will provide an opportunity for new UCN source possibilities with even greater densities for future experiments, and a number of UCN source concepts have been developed~\cite{PENDLEBURY201478,Lychagin:2015spa,Klinkby:2014ria,Nesvizhevsky:2014dya,ZIMMER201485}. %for $n$-$\bar n$ with super-mirror horn that  cold neutrons into a the source with high efficiency.
   The goal of these proposals is 10$^{3}$-10$^{4}$ UCN/cm$^{3}$.
  
%  The European Spallation source (ESS), expected to be commissioned in 2019, will provide relatively long pulses of neutrons at high intensity, and a number of  UCN source concepts have been considered, for example by~\textcite{PENDLEBURY201478}. One possibility for a SF-He source envisions focusing a cold-neutron beam from a large axis beam port to transport cold neutrons into  the source with high efficiency. Calculations suggest that 103 UCN per cm?3 would be delivered to an EDM experiment~\cite{ZIMMER201485}. 


 


\section{Experiments}

In this section we review the current status and prospects of EDM experiments. Our goal is to describe in some detail  technical aspects of the experiments, systematic errors, near-term improvements, and new concepts under development. Guided in part by the historical procession of experiments discussed in the introduction, we begin with the neutron and move onto Cs and other paramagnetic atom and molecule experiments. Diamagnetic atoms and molecules including octupole-enhanced nuclei follow. Finally we describe the development of storage ring experiments to measure EDMs of light nuclei.
% In some cases we expand on the relevant physics of the systems under study, {\it i.e. energy-level structure}. 
The results so far are summarized in Table~\ref{tb:EDMResults}.


% efforts for the neutron, paramagnetic atoms and molecules, diamagnetic atoms $^{199}$Hg and $^{129}$Xe as well as octupole deformed $^{225}$Ra and $^{221/223}$Rn, solid-state electron EDM measurements and future storage-ring measurements of proton and light nuclei EDMs. 
%
%The Hamiltonian of such a system is
%\begin{equation}
%\hbar \omega_L = - \frac{\vec{S}}{|S|} ( \mu B -  d E ),
%\end{equation}
%with $\mu$ the magnetic moment and $d$ the electric dipole moment.
%
%The frequency difference $\Delta \omega$ between measurements with parallel and anti-parallel orientations of $B$ and $E$ is
%\begin{equation}
%\hbar \Delta \omega = \omega_{\parallel}-\omega_{\not\parallel}= 4dE.
%\end{equation}
%\begin{equation}
%\sigma_d\approx {1\over 2 E}{h\over \tau SNR }
%\end{equation}
%\label{eq:EDMSigmaEquation1}
%where $E$ is the magnitude of the effective electric field.


%
%The reference can be either a measurement with inverted or zero $E$.
% 
%In many cases, an additional reference with a suppressed EDM is of used.
%
%As the frequency to be measured can be nano-Hertz or less, the duration of the measurement can be long. 
%
%With long measurement times any drifts of the environment during this time and in between either subsequent or spatially separated measurements with different conditions will limit the precision.
%
%Thus, EDM measurements can be considered 'low-background' experiments.
%

%\subsection{The generic experiment}


%\subsection{Neutron EDM}

\subsection{The Neutron }
\label{sec:NeutronEDM}


Here we discuss the neutron EDM experiment from the original beam measurements to the UCN experiments. The dominant systematic errors are discussed as well as brief descriptions of the several efforts underway to improve the sensitivity by one to two orders of magnitude.

The first experiments to search for the neutron EDM used a cold neutron beam~\cite{rf:Smith1957}, but % and Ramsey's method of separated oscillatory fields.
%
by 1980, the beam measurements became systematics limited due to complications arising from beam divergence and motional ($\vec v\times \vec E$) effects~\cite{rf:Dress1977}. These systematic issues led to experiments using stored UCN, trading higher neutron density  for  longer observation times and reduced velocity~\cite{Altarev1980}. % due to their long deBroglie wavelenths.
%
%Trapped UCN experiments have low statistics, compensated by long observation times. 
%
%UCNs provide the additional advantages that magnetic fields that are smaller and well defined over the rather small strorage volume and the low velocities, which mitigate $\vec v\times \vec E$ effects.
%
Two recent stored-UCN experiment 
configurations are shown in FIG.~\ref{fig:nedm_history}.  
%Typically, the UCN are stored in a chamber or bottle with high-voltage (HV) electrodes to produce the electric field. 
For the ILL-Sussex-Rutherford experiment (FIG.~\ref{fig:nedm_history}a)~\cite{Baker:2006ts} a single chamber is used along with a comagnetometer, while the Gatchina experiment (FIG.~\ref{fig:nedm_history}b)~\cite{Altarev1992,Altarev:1996xs} employs a pair of chambers with opposite electric fields and a common magnetic field. In the latter scheme, common-mode time dependent magnetic field variations are rejected to the degree to which the common magnetic field is uniform.
%
%The electrodes and chambers are placed in a homogeneous $B$ field.
%
%In (a), a single chamber is used, together with a comagentometer.
%
The ILL experiment is analyzed as a spin-clock comparison using a variation of Ramsey's separated oscillatory field technique to measure  two spin-polarized species (neutrons and a $^{199}$Hg comagnetometer) in the same volume at the same time. % to cancel magnetic field drifts. %and also produced the current best limit on the neutron EDM \cite{Baker:2006ts}.
%
%Systematic effects are nevertheless present, due to the issue that the two species behave differently in a combined $B$ and $E$ field.
%
%In (b), the Gatchina EDM experiment is shown \cite{ref:gatchinaexp}. Here, a double chamber experiment is used to cancel homogeneous magnetic field drifts.
%
%
%
%Homogeneous $B$ field drifts, as well as several $E$ field related issues are in first order canceled with the double chamber arrangement, where parallel and anti-parallel alignments of $B$ and $E$ are measured simultaneously. 
%
In the Gatchina approach, both chambers have very similar systematics and the velocities of the UCN are small enough that many systematic effects are negligible at the 10$^{-26}$\ecm~ level.
%
\begin{figure}[tb]
\includegraphics[width=3.6 truein]{edm_history}
\caption{\label{fig:nedm_history}  (Color online) Neutron EDM apparatus from (a) the ILL-Sussex-Rutherford experiment with one chamber for UCN and a comagnetometer~\cite{Baker:2006ts}; (b) the Gatchina apparatus, as set up at ILL, with two neutron storage chambers so that parallel and anti-parallel $E$ and $B$ field orientations are measured simultaneously~\cite{Serebrov:2015idv}. Both experiments are run with the storage chamber in vacuum at room temperature. Figures used with permission.
}
\end{figure}
%
%The results for both experiments are statistics limited. A detailed discussion of systematic effects is provided in below.


%The standard technique is Ramsey's technique of separated oscillatory fields \cite{ref:ramsey_technique}.
%
In Ramsey's technique~\cite{RevModPhys.62.541}, an interferometer in time is realized by comparing the phase from a spin-clock with frequency $\omega_L$, the Larmor precession frequency, with the phase of a reference clock  with frequency $\omega_R$ after a fixed measurement time $\tau$ as illustrated in FIG.~\ref{fig:ramseytechnique}.
%
\begin{figure}[tb]
\includegraphics[width=3.5 truein]{Ramsey_2}
\vskip -1 truein
\caption{\label{fig:ramseytechnique}  (Color online) Ramsey's technique of separated oscillatory fields. The experiment starts out with polarized particles in a stable and uniform magnetic field B$_0$, with a stable external oscillator at frequency $\omega_R$ near the Larmor frequency $\omega_L$ of the particles in the B$_0$ field. First a $\pi/2$ pulse of oscillating magnetic field ($B_1$) rotates the polarization into the plane normal to B$_0$, creating a superposition of spin-up and spin-down states. The spins and external clock evolve independently until a second $\pi / 2$ pulse is applied. The second pulse measures the phase difference between the oscillator and precessing spins that accumulates during the free precession interval. Time evolves from top to bottom in the figure.
}
\end{figure}
%
%
The optimal observation time $\tau$ is based on the UCN storage time and  the polarization lifetimes ($T_2^*$) of the UCN and the comagnetometer.
% ({\bf what is the optimium time? make a graph?})
%
%With  detuned clock frequency $\omega_R\ne\omega_L$, %: oscillations of the polarization as function of detuning, analogous to the double-slit experiment with light.
%
The  phases of the spins and the clock evolve at the different frequencies, and the phase difference after a time $\tau$ is $\Delta=(\omega_L-\omega_R)\tau$. This is read out using the polarization $P_z$, {\it i.e.} the projection of the spin along the $B_0$ field  after the second pulse is applied. In terms of the number of neutrons detected with spin  parallel ($N_\uparrow$) and antiparallel ($N_\downarrow$) to $B_0$ at the end of the free-precession cycle, the polarization is: 
\begin{equation}
P_{z} = \frac{N_\uparrow - N_\downarrow}{N_\uparrow + N_\downarrow}.
\label{eq:polarization}
\end{equation}
For $\Delta=0, \pi/2, \pi$, $P_z/P_0=-1, 0, 1$, respectively, where $P_0$ is the maximum magnitude of the polarization.
%
To maximize sensitivity to a change of frequency, $ \Delta=\pm \pi/2$ is chosen to provide the maximum slope of the fringes:
\begin{equation}
%\frac{\partial P_z}{\partial \omega}\arrowvert_{\Delta=\pi/2} \approx P_0\tau.
\left. \frac{\partial P}{\partial \omega} \right|_{\Delta = \pi/2} = \tau P_0. 
\label{eq:ramseyphase}
\end{equation}
The EDM frequency shift leads to a polarization shift



\be
\delta P_z = \frac{2d_nE}{\hbar} P_0 \tau.
\ee
%
%
The statistical precision of the Ramsey measurement is:
\begin{equation}
\sigma_{d_n}= \frac{\hbar\sigma_{P_z}}{2EP_0\tau},
\label{eq:sigmad1}
\end{equation}
%
%with the error of the frequency
%\begin{equation}
%\sigma(\delta \Delta) = \frac{\sigma(P_z)}{T},
%\label{eq:sigmad2}
%\end{equation}
%
where the uncertainty of each  $P_z$ measurement, for a small frequency shift, is 
\be
\sigma_{P_z}\approx 1/\sqrt{N_\uparrow+N_\downarrow}.
\ee
%

%The statistical uncertainty for a measurement time $\tau$ is given in Eqn.~\ref{eq:EDMSigmaEquation1}. With $(SNR)=\alpha\sqrt{N}$ where $N$ is number of detected spins,  $\alpha$ is the overall contrast or polarization for the measurement, including the initial polarization, polarization loss during the precession time $\tau$ and efficiency of polarization analysis. For a measurement repeated  $M$ times, the statistical uncertainty is reduced by $\frac{1}{\sqrt{M}}$ resulting in EDM sensitivity for $j=1/2$
%\begin{equation}
%\sigma_d = \frac{\sqrt{12}\hbar}{2 \alpha E \tau \sqrt{NM}}= \frac{\sqrt{12}\hbar}{2 \alpha E \sqrt{NT\tau}},
%\label{eq:edmsensitivity}
%\end{equation}
%
%where $T=M\tau$.


%Improvements that change the sensitivity linearly are the electric field $E$, the contrast of the measurement $P_0$ and  $\tau$, though UCN loses lead to reduced $N_{\uparrow \downarrow}$ ar $\tau$ incrases much beyond the stoarge time constant. 
%
%The number of detected neutrons only slowly improves the sensitvity.
%
%\red {CLARIFY It should be noted that for next generation experiments, instead of the determination of the zero transition, a fit through the central fringes with an optimized distribution of detuned measurements will be necessary.}

For the ILL-Sussex-Rutherford experiment, 
%spin polarized 
$^{199}$Hg was chosen as the comagnetometer for several reasons: the upper limit on the $^{199}$Hg  EDM is much smaller than the neutron-EDM sensitivity (see Sec.~\ref{sec:XeHg}); the coherence time of $^{199}$Hg can be 100 s or more; the coherence time combined with the large signal-to-noise ratio of the optically pumped atoms provides high sensitivity to magnetic field variations~\cite{GREEN1998381} . %
%The measured atomic EDM of $^{199}$Hg is $\sigma(d_{Hg})< 3.1 \cdot 10^{-29}$~ecm, much less than the sensitivity to the neutron EDM and thus $^{199}$Hg can be used as a reference substance for the neutron EDM measurement.
%
The $^{199}$Hg vapor was polarized with 254~nm light from a discharge lamp in a pre-polarizing chamber adjacent to the neutron EDM apparatus. 
%
After the UCN were loaded, the polarized Hg vapor was added to the neutron storage volume at a pressure of  about $10^{-4}$~mbar, the vapor pressure of Hg at room temperature. At this low pressure, the macroscopic magnetic field due to the $^{199}$Hg did not appreciably shift the spin precession of the neutrons. Spin-precession was initiated with a  resonant $\pi$/2 pulse that rotated the Hg spins into the plane transverse to $\vec B_0$. The $^{199}$Hg precession frequency was determined by the time-dependence of the absorption of a weak probe beam of circularly polarized 254 nm light directed through the storage chamber continuously during the UCN measurement. %, eliminating light shifts during the free-precession interval. 
The polarization and readout light intensities were balanced to optimize the performance of the magnetometer. With a mean velocity of $v_\mathrm{rms}\approx 193$ m/s, the $^{199}$Hg atoms sampled the entire storage volume in a time short compared to a  Larmor cycle of 128  ms.% effecting motional narrowing \cite{ref:motional_narrowing}.

A reanalysis of the~\textcite{Baker:2006ts} experiment  by~\textcite{Afach:2015sja} led to the final  result
 \be
d_n=(-0.21\pm 1.82)\!\times\! 10^{-26}\ \text {\ecm}.
\ee
For the Gatchina experiment, the combination of results from PNPI~\cite{Altarev:1996xs} and ILL~\cite{Serebrov:2015idv} are presented as a sensitivity of $3\times 10^{-26}$ \ecm and an upper limit at 90\% confidence level of $5.5\times 10^{-26}$ \ecm~\cite{Serebrov:2015idv}.
Major systematic effects are discussed in the next section. 



%With typical electric fields of 10 kV/cm, the current upper limit on the neutron EDM, $2.9\times 10^{-26}$ \ecm, corresponds to a frequency shift less than about $2\times 10^{-7}$Hz and effective magnetic-field stability averaged over the measurement of $7\times 10^{-15}$ T. The most sensitive EDM measurement of any system is $^{199}$Hg~\cite{Griffith:2009zz}, which sets an EDM limit about 3-orders of magnitude smaller, i.e. $2\times 10^{-29}$ Hz, corresponding to sub-nanoherz frequency measurements. 



%\subsection{Precision}

\subsection*{Neutron-EDM systematics}
%{Discussion of frequency shifts - geom. phase, Bloch Siegert....} TIM
\label{sec:neutron_systematics}
\label{sec:systematics}
%


Improved sensitivity of EDM experiments led to the discovery of new systematic effects that have been incorporated by~\textcite{Afach:2015sja} into the analysis of the~\textcite{Baker:2006ts} result. 
%
%We group limitations of the quality of EDM measurements in statistical and frequency measurement precision limiting effects, as well as frequency shifting effects.
%
%Whereas the first group of effects is best treated with increasing the number of detected spins and the signal $/$ noise ratio, as well as satbilizing all subsystems that cause noise and fluctuations in the measurement, the latter group are real systematic effects.
%
%For the determination of the EDM, all effects that shift the measured frequency proportional with the applied field $E$ are critical.
%
These include both direct false effects, such as stray magnetic fields correlated with the electric field polarity due to leakage currents generated by the high voltage apparatus near the EDM measurement cells, and indirect false effects, such as the geometric phase effect discussed below. 
%These include (i) direct false effects, e.g. leakage currents caused by the applied high-voltage  anywhere close to the EDM measurement setup that produce stray magnetic fields;  and(ii) indirect false effects, such as the geometric phase effect discussed below. %
%
To put this in perspective, with an applied electric field of 10 kV/cm, an EDM of 10$^{-26}$ \ecm~corresponds to a frequency shift of $\Delta\omega\approx 2\pi\!\times\! 50$ nHz, equivalent to a magnetic field change of 2 fT. 

A leakage current due to the high voltage applied across the storage cell would be correlated with the electric field and would produce a false EDM signal. For example,  the storage chamber used by~\textcite{Baker:2006ts} has radius $\approx 0.2$~m, so a  leakage current of 1  nA making a full turn around the cell would produce a 3 fT field. 
%For smaller diameter chambers, the leakage-current requirements are more stringent.)  
In principle, this would be compensated or monitored by the comagnetometer or external magnetometers. However, as noted below, the UCN and $^{199}$Hg positions are separated by about 3 mm and the non-uniform magnetic field due to the leakage current  would not be perfectly compensated.  For a storage cell of height of 20 cm the cancellation to first order was estimated to be 0.3/20 resulting in a false EDM of  $d_\mathrm{false}^\mathrm{leak}\approx~2\times 10^{-28}$ \ecm~ for $E$= 10 kV/cm.
% Magnetic fields due to leakage currents below 1~nA from electrodes at several 100~kV, as well as from ripple and noise of the high voltage supply will also contribute.

%Thermally induced currents (Johnson noise) in the electrodes with stationary patterns  can cause non-uniform magnetic fields easily exceeding 1 nT near                                                                                                                                                                                                                                                                                                                                                                                                                                                                                                                                                                                                                                   the surface~\cite{ref:thermalcurrents_allardprivcom}. Ideally, future experiments will use non-metallic electrodes and, in general, the use of non-metallic materials. 



%Further, the homogeneity of the magnetic field is critical: any transverse magnetic fields will cause a geometric phase of the particles moving in an inhomogeneous magnetic field with transverse components with imperfect symmetry.
%

%... shown in FIG. 14. ...: "left panel" -> FIG. 14 (a) "right panel" -> fig 14 (b)


The geometric phase or $\vec E\times\vec v$ effect~\cite{rf:ComminsAJPBerryPhase,rf:PendleburyGP,rf:GolubLamoreauxGP} is the extra phase accumulated by a quantum system due to the rotation of the quantization axis, in this case due to the combination of motional  fields and gradients of the magnetic field. To illustrate the origin of this effect, following~\textcite{rf:PendleburyGP},  consider the situation shown in  FIG.~\ref{fg:nGeomPhase} for a particle moving in a nearly circular trajectory near the wall of the radius $R$ chamber with an axial magnetic field $\vec B_0$ nominally directed out of the page ($\vec B_0\approx B_z\hat z$) and a magnetic field gradient, which produces radial components. With $\vec E\times\vec v=-E_z v_\phi \hat \rho$, and assuming cylindrical symmetry
\begin{equation}
 B_{\rho} = - \frac{\partial B_{z}}{\partial z} \frac{R}{2}-\frac{v_\phi E_z}{c^2},
\label{eq:perpfield}
\end{equation}
where the azimuthal velocity component $v_\phi$ is positive for counter-clockwise and negative for clockwise rotation. Note that $B_\rho$ changes magnitude when $v_\phi$ changes sign.
In the frame of the particle, the transverse magnetic field of magnitude $B_\rho$ {\em rotating} at a frequency $\omega_\mathrm{Rot}=v_\phi/R$ causes a shift of the Larmor frequency analogous to the Bloch-Siegert shift of NMR~\cite{BS40} as generalized by Ramsey~\cite{rf:RBSShift}: %\begin{eqnarray}
\begin{equation}
\Delta \omega = \omega_L - \omega_0 \\
= \sqrt{(\omega_0-\omega_\mathrm{Rot})^2 + \omega_\rho^2} - (\omega_0-\omega_\mathrm{Rot}).
%\end{eqnarray}
\label{eq:geomphase0}
\end{equation}
Substituting $\omega_0=\gamma B_z$ and $\omega_\rho = \gamma B_\rho$, where $\gamma$ is the particle's gyromagnetic ratio, and taking  $v_\phi$ positive, the first order shift
($\omega_\rho^2\ll(\omega_0-\omega_\mathrm{Rot})^2$ and $\omega_0>\omega_\mathrm{Rot}$) is%
%
%\begin{equation}
%\Delta \omega\approx \frac{(\gamma B_\rho)^2}{2(\gamma B_z -v_\phi/R)}. \\
%\end{eqnarray}
%\label{eq:geomphase1}
%\end{equation}
%
%Note that the sum of the transverse fields $B_m$ and $B_{rho}$ enters quadratically, resulting in interference terms linear in $E$:
%\begin{equation}
%\Delta \omega \propto \omega_{xy}^2 = \frac{1}{\gamma^2} B_\rho^2 + B_m^2 + 2 B_\rho B_m.
%\label{eq:geomphase}
%\end{equation}
% To first order, the shift for positive $v_\phi$ is
\begin{equation}
\Delta\omega_{v_\phi}=\frac{(\gamma B_\rho)^2}{2(\gamma B_z-v_\phi/R)}.
\label{eq:RBSShift}
\end{equation}
Because $B_\rho$ changes magnitude when $v_\phi$ changes sign, this shift does not average to zero for the two signs of $v_\phi$. Combining Eqns.~\ref{eq:perpfield} and~\ref{eq:RBSShift} the average shift for a positive and negative $v_\phi$ and a trajectory of radius $R$ is
\begin{equation}
\Delta\omega_{avg}=
\frac{1}{2}\frac{\gamma B_z[(\gamma\frac{\partial B_{z}}{\partial z}\frac{R}{2})^2+(\frac{\gamma v_\phi E_z}{c^2})^2]
+\gamma^2\frac{\partial B_{z}}{\partial z} v_\phi^2 \frac{E_z}{c^2}}
{(\gamma B_z)^2-(v_\phi/R)^2}.
\end{equation}
When $E_z$ is reversed, the last term in the numerator changes sign,  producing a false-EDM signal
\begin{equation}
d_\mathrm{false}^\mathrm{GP}\approx    \frac{\hbar\gamma^2\frac{\partial B_{z}}{\partial z} v_\phi^2R^2/c^2}{4(v_L^2-v_\phi^2)},
\label{eq:falseD_GP}
\end{equation}
where the ``Larmor'' velocity is $v_L=\gamma R B_z$, the size of the trajectory is $R$, and the effective velocity $v_\phi$. Note that the denominator in Eqn.~\ref{eq:falseD_GP}  goes to zero as $v_\phi\rightarrow v_L$, which has led to characterization of an {\em adiabatic} regime  with $|v_L|>>|v_\phi|$  and a non-adiabatic regime, e.g. when $|v_L|\approx |v_\phi|$. In the adiabatic regime, the spins track the magnetic field in their frame.  

As a numerical example, we take $v_\phi= 7$ m/s and $v_L\approx 200$ m/s for UCN and $v_\phi= 200$ m/s and $v_L\approx 50$ m/s for room-temperature $^{199}$Hg.  For a gradient  $\frac{\partial B_{z}}{\partial z}=0.3$ nT/m, and $R=$ 0.2~m, this results in $d_\mathrm{false}^\mathrm{GP} \approx 1\times 10^{-28}$\ecm~for UCN and $d_\mathrm{false}^\mathrm{GP} \approx  1\times 10^{-26}$\ecm~for $^{199}$Hg.
Consequently the geometric phase effect is actually much more significant for the $^{199}$Hg comagnetometer~\cite{Baker:2006ts}. More refined modeling~\cite{pignol,steyerl3} as well as numerical studies~\cite{rf:Bales2016} result in similar estimates of this effect for more realistic trajectories and field maps.


Approaches 
% that have been considered 
to reduce the geometric phase effect for future experiments include reducing the magnetic field gradients, making the chamber smaller, and manipulating the effective velocity and radius of the trajectories, for example with a buffer gas that would change the mean-free path. The residual field gradient in the  TUM\"unchen magnetic shield is $<100$ pT/m, sufficiently small to suppress the geometric phase  effect to about $1\times 10^{-28}$\ecm. However the ultimate limitation will likely be due to distortions from magnetized components of the EDM apparatus, for example  the valves for the UCN and $^{199}$Hg,  magnetic contamination of the surface, or magnetization of electrodes resulting from HV discharges.
% during the experiment will be critical. %For a sensitivity of a UCN trap like the Sussex experiment with similar $E$ and $B$ field, 
A magnetic dipole inside the chamber producing a  10~pT field at 2~cm is estimated to produce a false EDM  of $10^{-28}$\ecm. A thorough treatment of magnetic dipole sources and their effect on the geometric phase can be found in discussions by \textcite{pignol,harris,rf:GolubLamoreauxGP,steyerl2,steyerl3}. %This  represents a significant challenge to the design of next-generation experiments.


\textcite{Baker:2006ts}  developed a scheme to mitigate the geometric-phase effect with an applied gradient $\partial B_z / \partial z|_\mathrm{appl}$, taking advantage of the fact that the average position of the UCN's  is lower than the room-temperature $^{199}$Hg by about 2.8~mm.
%; moreover the energy distribution of the neutrons in the trap changed during the storage interval due to \red{ inelastic REALLY?} wall collisions, which further changed the UCN center of mass. 
%For an applied gradient $\partial B_z / \partial z|_{appl}$ and measuring the change in the respective frequencies~ \cite{Baker:2006ts}.
%For a gradient $\partial B_z/\partial z$, and assuming the EDM of $^{199}$Hg is negligible and the geometric phase for  the neutron is negligible, the $^{199}$Hg frequencies (Hz) are
%\bea
%\omega_n&=&\gamma_n(B_z+\frac{\partial B_z}{\partial z}z_n) + \frac{2d_nE_z}{\hbar}\nonumber\\
%\omega_{\rm Hg}&=&\gamma_{\rm Hg}(B_z+\frac{\partial B_z}{\partial z}z_{\rm Hg}) + \beta^{geo}_{\rm Hg}\frac{\partial B_z}{\partial z}E_z,\nonumber\\
%\eea
%where 
%\be
%\beta_{\rm Hg}^{geo}\approx \frac{\gamma_{\rm Hg}^2 v_\phi^2R^2/c^2}{2(v_L^2-v_\phi^2)},
%\ee
%with $v_\phi$ the average $^{199}$Hg velocity and $v_L=R\gamma_{\rm Hg} B_z$. Note that the frequencies can be positive or negative in the lab frame once a convention is established depending on the sign of the gyromagnetic ratios - negative of neutrons and positive for $^{199}$Hg - and the orientation of $B_z$, however generally the magnitudes of the frequencies are measured and the sign of the EDM relative to the magnetic moment would be determined based on whether the frequency increased or decreased when $E_z$ was reversed with respect to $B_z$.
%Assuming a perfect magnetic field with no gradient, the neutron EDM would be extracted from the difference of the two frequencies normalized to the gyromagnetic ratios:
%\bea
%``|d_n|''&=&\frac{\hbar}{2|E_z|}(|\omega_n|-\frac{|\gamma_n|}{|\gamma_{Hg}|}|\omega_{\rm Hg}|)\nonumber\\
%&=& d_n +  \frac{\hbar}{2|E_z|} (\gamma_n\frac{\partial B_z}{\partial z}\Delta z-\frac{|\gamma_n|}{|\gamma_{Hg}|} \beta^{geo}_{\rm Hg}\frac{\partial B_z}{\partial z}E_z)
%\nonumber\\
%\eea
% and thus the ratio of frequencies.
%
%
%
%Analysis of the experiment studied the ratio of neutron and $^{199}$Hg frequencies $R_a=|\frac{f_n}{f_{Hg}}\frac{\gamma_{Hg}}{\gamma_n}|$ as a function of the applied magnetic-field gradient $\frac{\partial B_{z} / \partial z }{B_{z}}$. 
%Assuming that the gradient causes frequencies to shift due to the difference of heights of neutrons and $^{199}$Hg and due to the $E_z$-odd geometric phase effect for $^{199}$Hg given by the schematic-model result in Eqn.~\ref{eq:falseD_GP}
%
%\begin{eqnarray}
%R_a = \frac{f_n}{f_{Hg}}\frac{\gamma_{Hg}}{\gamma_n} = 1 + \frac{\partial B_{0,z}}{ \partial z }{|B_{0,z} |}\Delta z ,
%R_a-1 =  &\pm &\frac{\partial B_{z} }{ \partial z }\frac {\Delta z}{B_{z}}\nonumber\\
%& \mp& 2\frac{\partial B_{z} }{ \partial z }\frac{\gamma_{\rm Hg}^2 v_\phi^2 R^2/c^2}{v_L^2-v_\phi^2} \frac{E_z}{hf_{\rm Hg}}|\frac{\gamma_{\rm Hg}}{\gamma_n}|
%\nonumber \\
%& \mp &  \frac{2 d_n E_z}{hf_{\rm Hg}}|\frac{\gamma_{\rm Hg}}{\gamma_n}|\nonumber\\
%&+&... 
%\label{eq:nhgratio}
%5\label{eq:edm1}
%\end{eqnarray}
%where the plus and minus refer to opposite orientations({\it i.e.} up and down) of the $B_0$ field, and $\Delta z=\langle z\rangle_n-\langle z\rangle_{\rm Hg}$ is the difference in vertical position of the center of mass of the two species and $E_z$ is positive/negative for $B_0$ and $E$ parallel/antiparallel~\cite{Afach:2014fha}.
% and $d_n^*$ , the measured EDM signal, which combined the EDM and any false-EDM that caused frequency shifts proportional to $\vec E$.
%
%The sign changed with the direction of $B_0$.
%
%To extract $d_n^*$, the frequency of the neutron $f_n$ with $E$ parallel $B_0$ compared to $f_{Hg}$ was
%\begin{equation}
%\frac{f_n(E)}{f_{Hg}} = | \frac{\gamma_{Hg}}{\gamma_n} |  + \Delta z \frac{\partial B_{0,z} / \partial z }{B_{0,z}} | \frac{\gamma_{Hg}}{\gamma_n} |  -  \frac{2 d_n^* E}{hf_{Hg}},
%\label{eq:edm1}
%\end{equation}
%
%Reversing projects $d_n^*$.
%
%The terms in the second and third lines are the false-$d$ due to the geometric phase and the true EDM, respectively and are determined by reversing $E_z$ for a specific value of the applied gradient. 
A pair of frequency measurements with $\vec E$ and $\vec B$ oriented parallel and antiparallel can be combined into an $E_z$-even geometric phase and the $E_z$-odd EDM signal. The geometric phase signal is linear in the gradient, while the EDM signal  is independent of the first-order magnetic field gradient. 
%For the applied magnetic field $B_z>0$ and t
Taking the frequencies to be positive quantities, the EDM signal is the $E_z$-odd combination of neutron and comagnetometer frequencies
\bea
d^{meas}&=&\frac{\hbar}{4|E_z|} [(\omega_n^{\uparrow\uparrow} - \omega_n^{\downarrow\uparrow})- (\omega_{\rm Hg}^{\uparrow\uparrow} - \omega_{\rm Hg}^{\downarrow\uparrow})\bigl |\frac{\gamma_n}{\gamma_{\rm Hg}}\bigr |]\nonumber\\
&\approx& d_n+d_{\rm Hg}^{GP}+\dots,
\label{eq:dnmeas}
\eea
where  ($\uparrow\uparrow$) and ($\uparrow\downarrow$)  refer to $\vec E$ and $\vec B$ parallel and antiparallel, respectively, and the $+\cdots$ indicates additional $E_z$-odd false-EDM effects. %From the schematic model in Eqn.~\ref{eq:falseD_GP},  $d_{\rm Hg}^{GP}$,  is linear in $\frac{\partial B_z}{\partial z}$.
%The $E_z$-even combination of frequencies is a measure of the average magnetic field for each species. 
Assuming a linear gradient and taking $B_z^0$ as the magnetic field at $\langle z_{\rm Hg}\rangle=0$
\bea
\frac{1}{2} (\omega_{\rm Hg}^{\uparrow\uparrow} + \omega_{\rm Hg}^{\downarrow\uparrow})&=&\bigl | \gamma_{\rm Hg}\bigr | B_z^0\nonumber \\
\frac{1}{2}(\omega_n^{\uparrow\uparrow} + \omega_n^{\downarrow\uparrow})\ &=&\bigl | \gamma_n\bigr | \bigl [ B_z^0+ \frac{\partial B_z}{\partial z} (\langle z_n\rangle - \langle z_{\rm Hg}\rangle)\bigr ]
%& &(\omega_n^{\uparrow\uparrow} + \omega_n^{\downarrow\uparrow})- (\omega_{\rm Hg}^{\uparrow\uparrow} + \omega_{\rm Hg}^{\downarrow\uparrow})\bigl | \frac{\gamma_n}{\gamma_{\rm Hg}}\bigr |
% \frac{1}{(\omega_{\rm Hg}^{\uparrow\uparrow} + \omega_{\rm Hg}^{\downarrow\uparrow}) \big |\frac{\gamma_n}{\gamma_{\rm Hg}}\bigr |}
%\nonumber\\
%& &=  \bigl |\gamma_n\bigr | \frac{\partial B_z}{\partial z} (\langle z_n\rangle - \langle z_{\rm Hg}\rangle).
\eea
These can be combined into the ratio
%\be
%R_a=\big | \frac{\omega_n^{\uparrow\uparrow}+ \omega_n^{\downarrow\uparrow}}{\omega_{\rm Hg}^{\uparrow\uparrow} + \omega_{\rm Hg}^{\downarrow\uparrow}}\big | \big |\frac{\gamma_{\rm Hg}}{\gamma_n}\bigr |=\frac{B_z^0+\frac{\partial B_z}{\partial z}\langle z_n\rangle}{B_z^0+\frac{\partial B_z}{\partial z}\langle z_{\rm Hg}\rangle},
%\label{eq:Rprime}
%\ee
%where $B_z^0$ is the value of $B_z$ at $\langle z_{\rm Hg}\rangle=0$,
\bea
 R^\prime&=& \frac{\omega_n^{\uparrow\uparrow}+ \omega_n^{\downarrow\uparrow}}{\omega_{\rm Hg}^{\uparrow\uparrow} + \omega_{\rm Hg}^{\downarrow\uparrow}} \big |\frac{\gamma_{\rm Hg}}{\gamma_n}\bigr |-1\nonumber\\
 %R_a-1 \approx  \frac{1}{B_z^0} \frac{\partial B_z}{\partial z} (\langle z_n\rangle - \langle z_{\rm Hg}\rangle).
%R^\prime&=&\frac{(\omega_n^{\uparrow\uparrow} + \omega_n^{\downarrow\uparrow})}{\bigl |\gamma_n\bigr |}- \frac{(\omega_{\rm Hg}^{\uparrow\uparrow} + \omega_{\rm Hg}^{\downarrow\uparrow})}{\bigl |\gamma_{\rm Hg}\bigr |}
% \frac{1}{(\omega_{\rm Hg}^{\uparrow\uparrow} + \omega_{\rm Hg}^{\downarrow\uparrow}) \big |\frac{\gamma_n}{\gamma_{\rm Hg}}\bigr |}
%\nonumber\\
&\approx&  \frac{1}{B_z^0} \frac{\partial B_z}{\partial z} (\langle z_n\rangle - \langle z_{\rm Hg}\rangle).
\label{eq:Rprime}
\eea
Here $B_z^0$ can be  positive or negative, and $\langle z_n\rangle<\langle z_{\rm Hg}\rangle$. Noting that $d_\mathrm{false}^\mathrm{GP}$ is propotional to $\frac{\partial B_z}{\partial z}$ while $d_n$ is independent of the gradient,  Eqns.~\ref{eq:falseD_GP},~\ref{eq:dnmeas}, and~\ref{eq:Rprime} are combined into:
\be
d^{meas}=d_n^\pm \pm kR^\prime,
\label{eq:dmeas}
\ee
where  $+/-$ refer to $B_z$ up/down, $kR^\prime$ is $d_\mathrm{false}^\mathrm{GP}$,  and the slope $k$ is  assumed to have the same magnitude for $B_z$ up and down. 
In FIG.~\ref{fig:baker_edm}, $d_n^{meas}$ is plotted vs $R^\prime$ for $B_z$ positive and negative. These data were fit to determine $d_n^+$, $d_n^-$, and $k$. The average $\frac{1}{2}(d_n^++d_n^-)$, corrected for additional systematic effects, is the final $d_n$ result.%
\begin{figure}[tb]
\centerline{\includegraphics[width=4.25 truein]{EDMAnalysisPRAClip}}
\vskip -0.4 truein
\caption{\label{fig:baker_edm} (Color online)  Dependence of the measured $E_z$-odd  EDM signal  on the magnetic field gradient, represented by $R\,^\prime$, for two $B_z$ orientations. The $y$-axis labeled ``EDM" refers to  $d^{meas}$ in Eqn.~\ref{eq:dnmeas}. The data points with error bars are from a typical  data run. The solid red lines are linear fits to Eqn.~\ref{eq:dmeas} for  the data set  with $B$-up and $B$-down. Figure from ~\textcite{Baker:2006ts}. 
}
\end{figure}

\begin{figure}[tb]
\includegraphics[width=2.9 truein]{GPNonUniformBandCellTC}%{GeometricPhasePicture}
\\
\vskip -0.25 truein
\includegraphics[width=3.5 truein]{GeomPhaseTC}%{GeometricPhasePicture}
\vskip -0.2 truein
\caption{\label{fg:nGeomPhase} Contributions to the geometric phase effect. Top: a non-uniform magnetic field that leads to radial magnetic field components.
%components $-\frac{\partial B_z}{\partial z}\frac{R}{2}$. 
Bottom: illustration of particle trajectories and contributions to $B_\rho$ from the gradient and motional magnetic field. The left side shows a positive azimuthal velocity, and the right side shows a negative azimuthal velocity. The average frequency shift for the two counter-propagating trajectories is given by Eqn.~\ref{eq:falseD_GP}. Adapted from~\textcite{rf:PendleburyGP}.}
\end{figure}



{% atoms the peripheral trajectory shown in ~\ref{fg:nGeomPhase} is possible, but for particles contrained by difussion, the orbits are much smaller and these effects are less important. Moreover the effect is signficantly more important for the $^{199}$Hg due to it's velocity of $\approx$ 200 m/s at room temperature compared to the UCN mean velocity of 8 m/s.
%For  particles with random trajectories that sample the storage volume uniformly over the measurement time  the net transverse magnetic field averages to zero. This averaging depends on the mean-free path of the particle, i.e. if $\lambda$ is much smaller than the dimension $L$ of the storage volume, the trajectories are diffusive and complete averaging requires $T>>D/L^2$, where $D=1/3\lambda v$ is the diffusion constant. For $\lambda$ greater than $L$, $T>>L/v$ is sufficient. With an applied electric field, however, the motional mangetic field  $B_m$ (\ref{eq:vxe} has components also perpendicular to $B_\perp$ that add in quadrature and thus the magnitude of the combined transvers magnetic field and motional magnetic field changes when the electric field (or magnetic field) is reversed



% the field does not average away any more.
%
%Similar to the Bloch-Siegert shift \cite{ref:blochsiegert}, an oscillating field (in the lab-frame) causes a frequency difference between the measured Larmor frequency $\omega_L$ and the undistorted frequency $\omega_0 = \gamma_i B_{0,z}$ in the (homogeneous) $B_0$ field $\Delta \omega = \omega_L - \omega_0$ \cite{ref:pendlebury, ref:lamoreaux, ref:golub}.
%
%Since the particles are moving inside the box with a characteristic angular frequency $\omega_r \sim v_{xy} / R$, the transverse fields acutally appear as rotating fields in the frame of the particle:
%
%
%Issues related to the part of the frequnecy shift linear with $E$ are: 
%\begin{itemize}
%
%\item{The shift can only be removed if gradients are small. For a next generation neutron EDM measurement based on the Sussex-chamber, a gradient of 0.3~nT$/$m averaged over the voulme of a UCN storage chamber with radius = 0.23~m and height = 0.12~m will result in $\sim 1\cdot 10^{-28}$~ecm for UCN. }
%
%\item{The shift depends on the particle velocity. Thus, any cohabitiating magnetometer is much stronger affected than the UCN.}
%\item{Any magnetic distortions or fields inside the chamber also contribute to a gradient in the chamber. In particular, dipole sources that originate from magnetic contaminations of the surface or magnetizations of electrodes that occur due to discharges during the experiment will be critical. %For a sensitivity of a UCN trap like the Sussex experiment with similar $E$ and $B$ field, 
%A magnetic dipole on the surface with a field of 10~pT at 2~cm distance from the surface (inside the chamber) already consumes the whole error budget. A thorough treatment of dipole-like contaminations and their effect on the a geometric phase can be found in \cite{pignol,harris,steyerl1,pignol,harris,steyerl2,pignol,harris,steyerl3}. This also represents a significant challenge to the design of next-generation experiments.}
%
%\item{A comagentometer is anyway strongly affected by the geometric phase due to the high particle velocity. For $^{199}$Hg in the Sussex experiment, the geometric phase due to the combination of the light-shift of the 254~nm readout beam \cite{ref:cohen-tann} already causes a frequency shift of $3.5 \cdot 10^{-27}$~ecm. The issue of light-shifts is well understood and has also been investigated in detail with a laser-based system \cite{ref:hg_edm}, as used for future cohabitiating magnetometers, the effect can be scanned and also be set ot exactly zero at a detuning off the resonance by 8~MHz.}
%\item{Any currents flowing in a metallic electrode causes magnetic fields. Thermally induced currents with stationaly patterns are well known issues \cite{ref:thermalcurrents_allardprivcom}, with fields easily exceeding a nT. A consequence of this could be the requirement for non-metallic electrodes and in general the use of non-metallic materials. }
%\item{Magnetic fields due to leakage currents below 1~nA from electrodes at several 100~kV, as well as from ripple and noise of the high voltage supply will also contribute.}
%\end{itemize}
%
%
%Whereas direct effects are changing the precession frequency of the trapped UCN e.g. by altering the magnitude of the $B_0$ field, and can be measured using correlations of $B_0$ and $E$ alignments, and are thus measurable, the treatment of such indirect effects will deserve strong attention, as they appear much more subte and are eventually invisible without high voltage.
%

%

%Additionally a non-uniform light-shift can produce an effective magnetic field gradient and contribution to the geometric-phase false EDM an amount estimated to be as large as $3.5 \cdot 10^{-27}$\ecm. The issue of light shifts is well understood and has also been investigated in detail with a laser-based system \cite{Griffith:2009zz}. For future co-habitating magnetometers, the light-shift effect can be scanned and also be set to exactly zero by detuning off the resonance by 8~MHz. 
%Optical magnetometers can measure the magnetic field continuously in contrast to UCNs, for which only the total accumulated phase is measured, making it more difficult to control these effects  during the measurement.

%Recently, a new class of systematic effects previously unknown in spin physics has been discovered, which is based on non-equilibrium spin transport in a non-dissipative system~ \cite{rf:Bales2016}. This results in deviations from Gaussian distribution of spins' accumulated phase,  which are best modeled with so-called Tsallis distributions~\cite{rf:Bales2016}. Effects on the distribution as well as shifts that would arise from skewness of the distribution may modify the accuracy of systematic effects estimated in next-generation EDM and other spin-precession experiments, {\it e.g.} the muon $g-2$ measurements~\cite{,Nouri:2016sko}. %Also any skewness in the spin distribution may have new consequences, in particular, frequency shifts steming from geometric phases and well known other systematic effects causing frequency shifts also appear (in principle) in other precision experiments with spins, an example being g-2 of the muon but also a future proton EDM experiment.\red{SAY MORE ABOUT THIS}


Additional  systematic effects are the light shift of $^{199}$Hg, discussed in Sec.~\ref{sec:LightShift} and   %\red{? UCNs precess in the dark, in contrast to many optical magnetometers at comparable precision, making it more difficult to control these effects  during the measurement. } 
 the changing energy distribution of the trapped UCN sample during the Ramsey measurement coupled with the energy dependence of UCN losses on the walls. %For example,  gravitational effects are coupled to the performance of the apparatus, but also gravitational shifts and geometric phases as well as detection efficiency cause systematic issues that will be significant for the next generation of experiments.
A thorough investigation of systematic effects for the ILL-Sussex-Rutherford experiment has  been reported by~\textcite{Afach:2015sja}.

% However, the understanding of systematic effects was matched with the statistical precision of the experiment and reduction, control of and study of potentially smaller effects will be essential. Moreover the stability of all magnetic fields on the 1-1000 second time scale must be controlled at the level of few fT including magnetic fields due to the leakage currents. 
%
The control of systematics for future experiments will rely on  intentionally varying and/or amplifying the effects, for example varying gradients, changing temperatures to affect the effective velocities, or changing buffer gas pressures.
 Ideas include a double chamber EDM experiment with a surrounding magnetometer array requiring that the magnetic gradient measured by the difference of neutron or comagnetometer frequencies in the two chambers  be consistent with the gradient measured with a surrounding (4-pi) magnetometer. Any inconsistency would be due to internal magnetization, {\it e.g.} magnetization caused by a spark of the high voltage. We also anticipate that most future EDM experiments will introduce blind analysis techniques.
%~\cite{rf:DataBlinding}\red{Do we need a reference?}. 
%
%Examples for the control of systematic effects for different systems are discussed below:

%Peter:  neutron velocity, no handle on neutron systematics neutron experiment is performed "blind" monitoring the HV




\subsection*{Neutron-EDM prospects}
\label{sc:nEDMprospects}

%A list of next-generation neutron EDM experiments is presented in Table~\ref{tab:nedms}. 
Several neutron EDM experiments are under development with the prospects of improving the sensitivity by one and eventually two orders of magnitude. In addition to the novel idea of probing P-odd/T-odd neutron scattering in a crystal, several experiments use UCN from a variety of different sources listed in Sec.~\ref{sec:UCNSources}. In the case of the SNS EDM experiment, the experiment is the source. %These ongoing efforts are described in the order of anticipated results.%UCN sources are discussed in Sec.~\ref{sec:UCNSources}.
%Apporaches with different technological concepts are listed below:

%
%\begin{table*}[t]
%\begin{tabular}{|l|l|c|rl|l|}
%\hline\hline

%\textrm{Experiment}   &  \textrm{Source}  &  \textrm{Cell/medium}      & $\sigma_d$ &( $10^{-28}$\ecm) & ref \\
%\hline
%ILL Crystal  EDM  &          cold beam     &      solid      &     $<$100  & & $a$   \\
%\hline
%NIST crystal &          cold beam     &      solid      &     $<$\,  10 &  \\
%\hline
%PNPI-ILL EDM &          Turbine    &      vacuum        &    $\sim$100 & (stage 1)    & $b$  \\
%                &                           &                        &      $<$ \ \ 5 & (stage 2)\\
%\hline
%PSI EDM  (Sussex) &          sD$_2$    &      vacuum        &  $\sim$ 50 & (stage 1)  \\
%PSI EDM  (``n2EDM")               &          sD$_2$    &      vacuum         &   $<$ \ \ 5 & (stage 2)  & $a$     \\
%\hline
%FRM-IIEDM (FRM-II) &          s$D_2$     &      vacuum      &    $<$100 &(stage 1)     \\
%PanEDM (ILL) &          SF-He   &      vacuum     &   $<$\, 10  &(stage 2)   & $a$  \\
%\red{PanEDM (ILL) } &          4-He   &     cryo      &  &  3. $1$     \\
%\hline
%LANL Room Temp.        &            sD$_2$     &       vacuum          &    \quad\  10 &    & $a$     \\
%\hline
%JPARC          &          sD$_2$     &      vacuum      &     $<$\, 10 &   & $a$    \\
%\hline
%TRIUMF/RPNC EDM               &            SF-He     &       vacuum       &       $<$\,  10     &  & $a$   \\
%\hline
%SNS EDM                                &            SF-He    &        4-He       &       $<$\ \ \  5    &  & $a$    \\
%\hline\hline
%\end{tabular}
%\caption{\label{tab:nedms}  Ongoing and planned neutron-EDM efforts worldwide (2017) as described in the text.}
%\end{table*}
%


%EDM sensitivity vs. source intensity; syst vs. statistics

%\begin{figure}[tb]
%\includegraphics[scale = 0.37]{nedm_sensitivity.jpg}
%\caption{\label{fig:edm_sensitivity} Statistical sensitivity of a neutron EDM measurement as function of cycle time and UCN source strength. For simplicity, we assume a double-chamber setup with cell dimensions similar to the Sussex experiment and $E$ = 20~kV$/$cm, a quality factor $\alpha$ = 0.8, as well as a turn-around time between Ramsey cycles of 100~s and 200 days measuring time.
%}
%\end{figure}


\noindent{\bf The PSI experiment}

For the neutron EDM program at the Paul-Scherrer Institute (PSI) near Zurich a large collaboration will use the UCN source described in Sec.~\ref{sec:UCNSources}. In a first generation effort, the original ILL-Sussex-Rutherford apparatus~\cite{Baker:2006ts} was moved and rebuilt with significant improvements to the neutron storage bottle lifetime, neutron polarization detection,  magnetic shielding, Hg comagnetometer and  the addition of Cs magnetometers. With these improvements, a result with sensitivity at the $1\times 10^{-26}$\ecm~ level is expected. Plans are underway for an improved experiment with a double-EDM chamber that is expected to extend the sensitivity to as low as $10^{-27}$\ecm.

\noindent{\bf The PNPI-ILL-PNPI experiment}

The Gatchina EDM experiment, shown in FIG.~\ref{fig:nedm_history}, has been running at ILL, where it is expected to improve on the result of~\textcite{Serebrov:2015idv}. Over the longer term, the UCN source at the PNPI research reactor WWRM is expected to provide up to 10$^4$ UCN/cm$^3$~\cite{Serebrov:2017gea} and  could extend the sensitivity to $2\times 10^{-28}$ \ecm.
%~with a variation of this apparatus~\cite{Serebrov2011}.

\noindent{\bf The FRM-II/PanEDM experiments}
The EDM experiment, originally developed at the FRM-II reactor \cite{Altarev:2012uy}, has been moved to ILL to couple to the SuperSun UCN source and renamed PanEDM.
%with contributions from UC Berkeley, ILL, LANL, Mainz, MSU, PTB Berlin, RAL, TU M\"unchen, University of Michigan, and Yale, is shown in FIG.~\ref{fig:frm_exp}.
%
It is a modular concept with UCN confined in two adjacent room-temperature chambers with opposite electric field.  Ramsey's technique will be applied to both cells simultaneously, and the EDM phase shift will have the opposite sign in the two chambers.
%
%n a later stage,  cryogenic chambers with increased electric fields are envisaged.
%
Magnetometry will be effected by  two $^{199}$Hg magnetometers above and below the EDM chambers. In addition, an array of Cs atomic-vapor magnetometers will be placed near the EDM chambers.  Plans call for two phases: Phase 1 will use the SuperSun source without the magnetic reflector with the sensitivity goal of 10$^{-27}$ \ecm; Phase 2 will employ the magnetic reflector, and a factor of 3-4 improvement is expected. With no comagnetometer, the requirements on magnetic shielding and external magnetometry are more stringent, but it is expected that the electric field can be increased by 50\% or more compared to what could be applied with a Hg comagnetometer.
%
%
%The chamber walls are coated with deuterated polyethylene, a material with $V_F\approx$ 220~neV~\cite{ref:gerd_dpe}. The measured loss probability  of $\mu < 10^{-4}$ per wall collision and depolarization is low~\cite{ref:gerd_dpe}. 
%The EDM chambers are held under vaccum. 
%\red {???Another new feature is that no additional vacuum chamber is required surrounding the chambers due to the use of a high-voltage silicone gel with extremely large bulk resistance and breakdown strength as an insulator.}
%The experiment will move for its initial stage to ILL within the PanEDM collaboration (including  PNPI) to perform  measurements using the SuperSUN UCN source at the former cryo-EDM beam line.
%
%
%\begin{figure}[tb]
%\includegraphics[width=3.4 truein]{tum_edm}
%\caption{\label{fig:frm_exp}  (Color online) The concept of the EDM experiment based at FRM-II: Item (1) are the cylindrical chambers, where UCN are trapped in vacuum with the chambers at room temperature. The chambers are formed by the high voltage electrode in the center (4), an insulator ring and a ground electrode; (2) and (5) are 199-Hg magnetometer cells; (3) the access channels for additional magnetometers surrounding the EDM chambers (initially an array of 32 Cs sensors).}
%\end{figure}
%
%
%The experiment is being performed in stages, with 


%staged approaches match statistics with systematic

% In many cases a field of $B_0 \sim 1 \mu$T ($\omega_L \sim 30$~Hz) is chosen. 
%


%Data blinding

%correlations with more configurations simultaneosly but at different positions, or sequentially at different times

% redundancy of magnetometers

% anfälligkeit der einzelnen edm experimente auf verschiedene systematiken


\noindent{\bf The LANL Room Temperature EDM experiment}

The recent four-fold increase in the LANL UCN source performance~\cite{Ito:2017ywc} has provided motivation to develop a  nEDM experiment. The concept is a double cell with a Hg comagnetometer and external magnetometry. There is also the possibility to study  $^3$He as a comagnetometer monitored with either a SQUID or by detection of UCN capture on $^3$He by detecting recoil protons with scintillators.  The plan is to deploy state of the art  magnetic shielding and to optimize storage volumes and UCN guide volumes to the LANL UCN source. The sensitivity goal is 1-3$\times 10^{-27}$ \ecm.



\noindent {\bf The TRIUMF/KEK experiment}

The TRIUMF/KEK experiment is based on a SF-He source coupled to the high-intensity proton beam from the TRIUMF cyclotron~\cite{Masuda:2002dy}. Due to the dilution of UCN when such a source is coupled to the experiment volume, the EDM chamber will be smaller than in other experiments.
A further option discussed for this room-temperature neutron EDM experiment is the use of a low-pressure $^{129}$Xe-based comagnetometer~\cite{rf:TRIUMFEDM}  with  two-photon readout~\cite{ref:xenon_twophoton}, introduced in Sec.~\ref{sec:Magnetometry}, as well as  a $^{199}$Hg comagnetometer designed to address the geometric-phase and leakage-current systematic effects.
%
%The technical realization of pumping and readout using a two-photon transition  is still under development.
%
%Also, the stability of the magnetometers under breakdowns that may occur in strong electric fields is still being investigated.


\noindent {\bf The SNS experiment}

A major effort underway in the US is the cryogenic SNS-nEDM experiment. The concept introduces  a number of novel features based on the proposal by~\textcite{rf:GolubLamoreaux}. The plan is to produce UCN in SF-He from a cold neutron beam within the EDM-storage volumes. Cryogenic superconducting shielding and the high dielectric strength of SF-He are expected to provide reduced systematics and higher electric fields than room-temperature vacuum experiments.  
A small amount of highly polarized $^3$He will be introduced into the EDM chambers to serve both as a comagnetometer and a spin analyzer. UCN are captured by $^3$He in a spin-singlet state due to an unbound resonance in $n$+$^3$He with emission of a proton  and triton with a total energy of  764 keV ($n+^3{\rm He}\rightarrow\ ^1{\rm H}+^3{\rm H}$).  Thus, effectively, only neutrons with spin opposite the $^3$He spin are captured,  and the rate of emission of the  proton and triton   will  measure  the projection of the relative spin orientation of the neutron and $^3$He. The proton and triton will be detected by scintillation in the SF-He due to the formation and decay of He$^*$ molecules and emission of an 84~nm photons. 
The UV photons are converted to longer wavelengths for detection by photomultipliers after absorption on the surface coating of  deuterated tetraphenyl-butadiene (dTPB)  polymer matrix, chosen due to the low loss of UCN on the deuterium). 
%The $^3$He also has the feature that the absorption is essentially completely spin dependent - the absorption cross section for neutorn on nuclear spin parallel is $\sigma_a^{\uparrow \uparrow}=2\sigma_a$=5300 b ($\frac{v_0}{v}$) due to a spin-zero resonance in the $n$-$^3$He (unbound resonance in $^4$He). 
As the UCN and $^3$He spins precess the absorption is modulated at the difference of precession rates. The change of modulation frequency of the scintillation light with electric field would signal a difference of the two species' EDMs and, assuming the $^3$He atomic EDM is much smaller  due to Schiff screening~\cite{Dzuba:2007zz}, the neutron EDM. 

The precession frequency of $^3$He is 32.43~MHz/T, compared to 29.16~MHz/T for the neutron, so any systematic effect (false EDM) due to a change of magnetic field correlated with the electric field is suppressed by a factor of about 10. However it is possible to shift the precession frequencies of both species with an oscillating magnetic field via the Ramsey-Bloch-Siegert effect (Eqn.~\ref{eq:geomphase0}), with $\omega_{Rot}$ set between the two free-precession frequencies to ``dress'' the spins~\cite{rf:CohenTannoudjiDressing}. This results in potentially orders of magnitude more sensitive comagnetometry.
Alternatively, the $^3$He spin-precession can be independently monitored with SQUID sensors to signal any changes in the magnetic field. 
The precessing $^3$He magnetization can be monitored with SQUID magnetometers~\cite{IEEE6905771KimClayton}. The geometric phase  will affect $^3$He, and the mean free path of the $^3$He atoms depends on temperature, which is nominally 0.4-0.5 K, but can be adjusted to scan the geometric-phase effect.
%
After each measurement, the  depolarized $^3$He must be removed from the superfluid helium using a heat-flush/phonon wind  technique~\cite{rf:HeatFlushRefs}.
%



%For the future neutron EDM measurement at the SNS (sec.~\ref{sec:sns_edm}), the use of $^3$He as a comagentometer is envisaged \cite{ref:3hecomag_idea}.
%
%Such a comagnetometer is very interesting: the experiment is operating inside superfluid helium ($^4$He). Polarized $^3$He can be added to the volume. 
%
%Here, the amount is very small compared to a gas-phase magnetometer. 
%
%
%There is a strong spin dependence of neutron aborption on $^3$He. 
%
%After neutron capture, a triton and a proton are produced, which cause scintillation in the helium volume due to the decay of He$^*$ molecules, which only exist in the exited state and emit 84~nm light during decay. 
%%
%The light is converted to longer wavelength for extraction.




%\subsubsection
\noindent {\bf Possible EDM experiments at the European Spallation Source}



A fundamental-neutron-physics program has been proposed for the European Spallation Source~\cite{Pignol20143}. In addition to possible UCN EDM experiments, a reprised beam experiment has been proposed~\cite{Piegsa2013}.
  In this concept, the pulsed structure of the ESS would provide velocity discrimination and could be used to monitor $\vec v\times \vec E$ effects
with the much greater statistical power provided by the cold-neutron beam.


%

\noindent{\bf Crystal EDM}

P-odd/T-odd rotation of the neutron spin in Laue diffraction of polarized neutrons incident on a  crystal is sensitive to the neutron EDM interaction with the strong interplanar electric field. The first experiment on CdS was undertaken by~\textcite{ShullCrystalEDM}, who found $d_n=(2.4\pm3.9)\times 10^{-22}$\ecm.  \textcite{Fedorov:2010sj} carried out an experiment at the ILL cold neutron beam facility PF1B measuring the spin-rotation of  monochromatic polarized neutrons incident on a quartz crystal. The effective electric field was estimated to be of order $10^8$ V/cm. The final neutron spin directions were analyzed for different incident neutron spin to separate the EDM effect from the Mott-Schwinger interaction with atomic electrons in the crystal. The  result from about one week of data was $d_n=(2.5 \pm6.5\ {\rm (stat)}  \pm5.5\ {\rm (sys)})\times 10^{-24}$\ecm. Prospects for an improved setup suggest that the sensitivity can be improved to $2\times 10^{-26}$\ecm~ for 100 days of data taking~\cite{Fedorov:2010sj}.


\subsection{Paramagnetic atoms: Cs and Tl}%atomic and molecular EDMs: $d_e$ and $C_S$}




%\sububsection{Paramagnetic systems}

%As noted above, 
The EDM of a paramagnetic system is most sensitive to the electron EDM $d_e$ and the strength $C_S$ of a nuclear-spin independent electron-nucleus coupling corresponding to a scalar nuclear current. The tensor nuclear current contribution is several orders of magnitude smaller, and the pseudoscalar contribution vanishes in the limit of zero velocity ({\it i.e} infinite nuclear mass).
% could arise from a combination of sources including an electron EDM, T and P violation in the nucleus and a T-violating mixing of opposite parity states of a scalar or tensor nature. For paramagnetic systems, the scalar contributions are likely to be several orders of magnitude stronger than tensor and pseudoscalar contributions, given comparable strength of the intrinsic couplings~\cite{rf:Ginges2004}. As discussed in section~\ref{sec:Hg} the nuclear contributions are strongly constrained by other measurements,  thus  for the cesium atom, one can write $d_{\rm Cs} \approx \eta_e d_e + k C_S$~\cite{rf:Carrico1968,rf:Bouchiat1975,rf:Ginges2004}. A variety of calculations place  $\eta_e$ in the range +100 to +140, and $k_{C_S}\approx 7\times 10^{-19}$ \ecm~ for cesium~\cite{rf:Ginges2004,rf:Nataraj2008}. Most authors use the single EDM result to set limits on the individual contributions assuming all other contributions are negligible, {\it e.g.} if $C_S=0$, then the result of reference ~\cite{rf:Weisskopf1968} could be interpreted as a 90\% limit on the electron EDM of  $d_e<3.3\times 10^{-24}$ \ecm~ (see footnote 1 for a discussion of upper limits). There is, however no substantial reason to assume the contribution to the cesium atomic EDM due to scalar interactions is much smaller than the electron EDM contribution. Thus measurements in more than one system sensitive to both $d_e$ and $C_S$ would be required to truly constrain possible sources of an atomic EDM. We note also that an atomic EDM can arise due to higher-order T-odd and P-odd nuclear moments when the nucleus has spin $\ge 1$. The magnetic quadrupole moment, a P-odd and T-odd distribution of currents in the nucleus would induce an atomic EDM by coupling to an unpaired electron~\cite{rf:Derevianko2005}. 
%(The nuclear anapole moment is another observable that is T-{\it even} but P-odd). 
%A nuclear magnetic quadrupole moment effect could induce an atomic EDM in $^{133}$Cs, which has nuclear spin 7/2~\cite{rf:Dmitriev1994}.
%V.F. Dmitriev, I.B. Khriplovich, V.B. Telitsin, Phys. Rev. C 50, 2358 (1994)

%\subsection{Early atomic beam measurements}
%\subsection{Paramagnetic atoms: cesium, thallium and metastable xenon}
\label{sec:CsTl}

The first direct atomic EDM experiment - measurement  of the frequency shift of the cesium atom EDM in an atomic beam with  a modulated electric field - was undertaken by ~\textcite{rf:Sandars1964} and collaborators.  There are many challenges to such a measurement that have led to the techniques applied to contemporary undertakings. First, the atomic beam, traveling at several-hundred m/s,  transited an apparatus of length less than a meter so that linewidths of several kHz were observed (Ramsey's separated oscillatory field technique was used). Second, due to the unpaired electron, the cesium atom has a large magnetic moment  that couples both to external magnetic fields and to the motional magnetic field $\vec B_m=\vec v\times \vec E/c^2$. The magnetic field produced by any  leakage currents  would change with the modulation of the electric field and could provide a false EDM signal. Misalignment of the applied magnetic and electric fields  also produces a false signal.  By determining the center of a resonance line to a precision better than the linewidth, {\it i.e.} line splitting  by more than 10,000, the frequency shift sensitivity was of order 0.1 Hz. The result is $d_{\rm Cs}=(2.2\pm 0.1)\times 10^{-19}$ \ecm~ with an electric field up to 60 kV/cm. The  error is statistical only, and the finite EDM signal is attributed to the motional effect due to a misalignment of 10 mrad~\cite{rf:Sandars1964}. Subsequent work using other alkali-metal species with lower $Z$ and less sensitivity to T-odd/P-odd interactions to monitor magnetic-field effects - now called a comagnetometer - led to the result $d_{\rm Cs}=(5.1\pm 4.4)\times 10^{-20}$ \ecm~\cite{rf:Carrico1968}. Shortly after that publication, ~\textcite{rf:Weisskopf1968} and collaborators presented a significantly improved result based on a longer interaction region, correspondingly narrower resonance lines and a sodium comagnetometer: $d_{\rm Cs}=(0.8\pm 1.8)\times 10^{-22}$ \ecm. 


The cesium atomic beam EDM experiments were ultimately limited by the linewidths,  count-rate limitations, and by systematic errors due to motional magnetic field effects, though the atomic beam machines provided the capability to use other, lighter alkali-metal species, {\it i.e.} a comagnetometer, significantly reducing the motional-field systematic errors~\cite{rf:Weisskopf1968}. 
Another approach was the vapor cell experiment developed by Hunter and collaborators~\cite{Murthy:1989zz}. The confined atoms provided much narrower resonance linewidths  ($\approx$50 Hz), and also greatly mitigated motional field effects. Though a comagnetometer was not practical in the vapor cell, the leakage currents were directly measured and set the systematic uncertainty in the final result 
\be
d_{\rm Cs}=(-1.8\pm 6.7\ (\rm{stat}) \pm 1.8\ ({\rm sys}))\times 10^{-24}\ \text{\ecm}.
\ee
%Atomic beam experiments in systems with significantly enhanced sensitivity to T-odd/P-odd interactions relative to cesium included a metastable atomic state in xenon and thalllium. %Molecular beams of polar molecules and molecular ions are discussed in Sec.~\ref{sec:PolarMolecules}.

Sandar's group performed an atomic beam experiment to search for an electric dipole moment
in the $^3P_2$ metastable state of xenon with a comagnetometer beam of krypton~\cite{rf:Player1970,HPS71}.
%M. A.	Player	and	P. G. H.	Sandars,	J.	Phys.	B	3,	1620 (1970).
 In the strong applied electric field,  the parity-allowed splittings are proportional to $m_J^2E^2$ and to the magnetic field component along  $\vec E$. The EDM signal would be a splitting linear in $\vec E$; however transitions that change the magnetic quantum  number $m_J$ are not practical due to the sensitivity of the $E^2$ term to a change of the magnitude of the electric field. Thus the $\Delta |m_J|=0$ transition $m_J=-1 \rightarrow m_J=+1$ was measured. With the xenon-krypton comparison, the difference of EDMs was found to be
$ |d_{\rm Xe}-d_{\rm Kr}|=(0.7\pm 2.2)\times 10^{-22}\ e{-\rm cm}$, where the errors are 90\% c.l.
% This was used to set a limit on the electron EDM using the estimates for xenon and krypton, $\eta_{d_e}\approx 130$ and $\eta_{d_e}\approx 20$.

% The first uncertainty is statistical and the second is systematic. 


% As discussed in section~\ref{sec:Hg} the nuclear contributions are strongly constrained by other measurements,  thus  for a paramagnetic atom or molecule, one can write~\cite{rf:Carrico1968,rf:Bouchiat1975,rf:Ginges2004}
%\begin{equation}
%d_{\rm Cs} \approx \eta_e d_e + k C_S.
%\label{eq:CsEDM}
%\end{equation}
%Where $C_S$ is a combination of scalar couplings of the valence electron to the proton and the neutron; for cesium,  $C_S=0.41 c_{sp}+0.59 c_{sn}$ ~\cite{rf:Bouchiat1975}. A variety of calculations place  $\eta_e$ in the range +100 to +140, and $k_{C_S}\approx 7\times 10^{-19}$ \ecm~ for cesium~\cite{rf:Ginges2004,rf:Nataraj2008}. Most authors use the single EDM result to set limits on the individual contributions assuming all other contributions are negligible, {\it e.g.} if $C_S=0$, then the result of reference ~\cite{rf:Weisskopf1968} could be interpreted as a 90\% limit on the electron EDM of  $d_e<3.3\times 10^{-24}$ \ecm~ (see footnote 1 for a discussion of upper limits). There is, however no substantial reason to assume the contribution to the cesium atomic EDM due to scalar interactions is much smaller than the electron EDM contribution. Thus measurements in more than one system sensitive to both $d_e$ and $C_S$ would be required to truly constrain possible sources of an atomic EDM. We note also that an atomic EDM can arise due to higher-order T-odd and P-odd nuclear moments when the nucleus has spin $\ge 1$. The magnetic quadrupole moment, a P-odd and T-odd distribution of currents in the nucleus would induce an atomic EDM by coupling to an unpaired electron~\cite{rf:Derevianko2005}. 
%(The nuclear anapole moment is another observable that is T-{\it even} but P-odd). 
%A nuclear magnetic quadrupole moment effect could induce an atomic EDM in $^{133}$Cs, which has nuclear spin 7/2~\cite{rf:Dmitriev1994}.
%V.F. Dmitriev, I.B. Khriplovich, V.B. Telitsin, Phys. Rev. C 50, 2358 (1994)




~\textcite{rf:Commins1994}  developed a vertical counter-propagating  thallium atomic beam and subsequently added sodium beams as a comagnetometer~\cite{rf:Regan2002}. These experiments pioneered a new understanding of some of the most important systematic effects for EDM experiments, including those mitigated by the comagnetometer and the geometric phase effect~\cite{rf:ComminsAJPBerryPhase,rf:PendleburyGP,rf:Barabanov2006}. The most recent result can be interpreted as 
\be
d_{\rm Tl}=(-4.0\pm 4.3)\times 10^{-25}\ e{-\rm cm}.
\ee
 
% The results for $d_{\rm Cs}$ and $d_{\rm Tl}$ can be combined using equation~\ref{eq:CsEDM} to constrain $d_e$ and $C_S$. We use $\eta_e=133$ and $k_{C_S}=7\times 10^{-19}$ \ecm~ for cesium and\ $\eta_e=-585$ and $k_{C_S}=-5\times 10^{-18}$ \ecm~ for thallium~\cite{rf:Ginges2004}. The result is $d_e=(-0.5\pm 1.4)\times 10^{-25}\ e$-cm, and $C_S= (0.6\pm 1.6)\times 10^{-5}$, with
% The cesium and thallium results are summarized in Figure~\ref{fg:CsTlEDM}, where  $d_e$  is plotted as a function of  $C_S$ given the most recent and precise results~\cite{rf:Murthy1989,rf:Regan2002}. a  limit  on $d_e$ of $2.7\times 10^{-25}$ \ecm~. The result reported in reference~\cite{rf:Regan2002}, $|d_e|<1.6\times 10^{-27}$ \ecm,  assumes $C_S=0$, which can be considered model dependent in the sense that this assumption would only be valid in the context of specific models of T- and P-violating interactions. The tensor interaction of electron and nucleus and the nuclear Schiff moment also contribute to atomic EDMs; however these are suppressed in paramagnetic atoms compared to diamagnetic atoms and molecules, and are therefore discussed in more detail in  sections ~\ref{sec:TlF} and ~\ref{sec:XeHg}.

%\begin{figure}
%\hskip-0.7 truein
 %\includegraphics[width=5.5in,angle=0]{CsTlEDM}
 %\vskip -4 truein
  %\caption{Constraints on $d_e$ and $C_S$ from cesium~\cite{rf:Murthy1989} and thallium~\cite{rf:Regan2002}. For cesium, we use $\eta_e=133$ and $k_{C_S}=7\times 10^{-19}$ $e$~cm. For thallium, we use $\eta_e=-585$ and $k_{C_S}=-5\times 10^{-18}$ $e$~cm. The solid lines are the 68\% confidence intervals and the dashed lines are the 90\% confidence intervals. The solid and dashed boxes suggest the 68\% and 90\% ranges  on $d_e$ and $C_S$ when the cesium and thallium results are combined, {\it i.e.} if $C_S$ is not assumed to be zero.}
%\label{fg:CsTlEDM}
%\end{figure}

%For cesium and thallium: $\eta_{\rm Cs}=115$, $\eta_{\rm Tl}= -585$, $k_{\rm Cs}=0.7\times 10^{-18}$ and  $k_{\rm Tl}=-5.1\times 10^{-18}$ \ecm~\cite{rf:GingesPhysRep}. 




%The most precise result for the Schiff moment from TlF based on the ratio $d_{\rm Tl}/S$ from reference~\cite{rf:CoveneySandarsTlF} is $S=(2.3\pm 3.9)\times 10^{-10}$ e-fm$^3$~\cite{rf:HindsTlF2}. 
%so that $d_p=(-3.7\pm 6.3)\times 10^{-23}$ \ecm. 


% two different approaches were introduced for very different reasons: experiments on cesium~\cite{rf:HunterCsEDM} , xenon~\cite{rf:Vold,rf:Rosenberry} and mercury{rf:Jacobs,rf:Lamoreaux,rfLRomalis,rf:Griffith} in atomic-vapor cells and a molecular beam with TlF. The vapor cell experiments provided much narrower resonance linewidths - 100 Hz or smaller for cesium and mHz or even less for mercury and xenon and also greatly mitigated motional field effects, though the applied electric fields were necessarily an order of magnitude or more smaller. 




%first with the Tl-F molecule~\cite{rf:TlFRamsey,TlFSandars}.  (Though the Oak Ridge neutron beams could be considered atomic beams and used the same magnetic resonance techniques.) 
 
%Experiments have measured EDMs of the neutron and atoms with closed shells and a single unpaired electron and in molecules. 

\subsection*{Prospects for alkali-metal atoms}

Laser cooling and trapping of  cesium and francium offer promising new directions, and several approaches are being pursued.
Cesium atomic fountain clocks based on launching atoms from a laser cooled or trapped sample  have moved to the forefront of time-keeping. 
% A. Clarion, P.Laurent, G. Santarelli, S. Ghezali, S. N. Lea, and M. Bahoura, �A cesium fountain frequency standard: Recent results,� IEEE Transactions on Instrumentation and Measurement, vol. 44, pp. 128�131, April 1995.
Narrow linewidths ($\tau\approx$ 1 s) are attained as the atoms move up and then down through a resonance region. While the $^{133}$Cs atomic frequency standard uses the $\Delta m_F=0$ transition, which is insensitive to small magnetic fields in first order, an EDM measurement must use $\Delta m_F\ge 1$. From equation~\ref{eq:EDMFreqEquation1},  with $T=1$ s, $\tau\gg T$  and $N=10^6$, the expected uncertainty  on $\omega$ is expected to be about $\delta_\omega\approx 10^{-3}$ Hz, which is consistent with observations of the Allan variance representing the short-term instability of cesium fountain clocks  ($ \sigma_\omega/\omega \approx 10^{-13}$ for $\omega=2\pi\times$ 9.2 GHz~\cite{rf:Weyers2009}). 
%S. Weyers et al., PHYSICAL REVIEW A 79, 031803R2009
 For an EDM measurement with an electric field of 100 kV/cm, which may be feasible, 
  each 1 second shot would have a sensitivity of $6\times 10^{-24}$ \ecm, comparable to the sensitivities of both the cesium and thallium measurements. 
  % presented in Table~\ref{tb:EDMResults}. 
  Thus significant improvement is possible, and a demonstration experiment with about 1000 atoms per shot and $E=60$ kV/cm was reported as a measurement of $d_e$ by \textcite{rf:Amini2006}. The result can be interpreted  as  $d_{Cs}=(-0.57 \pm 1.6) \times 10^{-20}$ \ecm~ (the authors use $\eta_{e}=114$ for cesium).
 %H. Gould, International Journal of Modern Physics D Vol. 16, No. 12B (2007) 2337�2342
%Jason M. Amini, Charles T. Munger Jr., Harvey Gould,	arXiv:physics/0602011v2 [physics.atom-ph]
The major limitation in this demonstration was the necessity to map out the entire resonance-line shape spectrum, which is the combination of transitions among the nine hyperfine sub-levels and inhomogeneities of the applied magnetic field in the resonance region.
  This subjected the measurement to slow magnetic field drifts that would need to be monitored or compensated. 
  If these problems are solved, the statistical sensitivity could be significantly improved with several orders of magnitude more atoms, higher electric field and duty-factor improvements. However
  the major systematic effect  due to $\vec v\times\vec E$ was about $2\times 10^{-22}$ \ecm~\cite{rf:Amini2006}. This could ultimately limit the sensitivity of a single species fountain measurement.
%H. Gould, International Journal of Modern Physics D Vol. 16, No. 12B (2007) 2337�2342
%Jason M. Amini, Charles T. Munger Jr., Harvey Gould,	arXiv:physics/0602011v2 [physics.atom-ph]
  
For francium, $\eta_{d_e}$, the ratio of atomic EDM to electron EDM from equation~\ref{eq:ParamagneticEDMs}, is in the range 900 to 1200~\cite{Ginges:2003qt}, and $k_{C_S}$ should be similarly enhanced for francium. Francium can be produced in significant quantities in isotope-separator rare-isotope production facilities, and $^{210}$Fr has been produced, laser cooled and trapped in a magneto-optical trap (MOT)~\cite{rf:Gomez2006}. The experiment has been moved to the isotope-separator facility (ISAC) at TRIUMF in Vancouver, Canada. A parallel effort is underway at Tohoku University Cyclotron and Radioisotope Center (CYRIC)~\cite{SakemiFrEDM1742-6596-302-1-012051,Kawamura-FrEDM}.
Francium isotopes have half-lives of 20 minutes (for $^{212}$Fr) or less, and any experiment would need to be ``on-line,'' that is the EDM apparatus would be at the site of the rare-isotope production facility. The applying the cesium-fountain approach to francium may lead the way to a future program  at a  the Facility for Rare Isotope Beams (FRIB) at Michigan State University.

The fountain concept allows linewidths on the order of 1 Hz, limited by the time for the cold atoms with vertical velocity of a few m/s to rise and fall about 1 meter. Another idea being pursued by D. Weiss and collaborators is to stop and cool alkali-metal atoms in optical molasses near the apogee of their trajectory and trap them in an optical lattice formed in a build-up cavity~\cite{rf:Fang2009}. 
 %Fang Fang and David S. Weiss Vol. 34, No. 2 / OPTICS LETTERS  2009 
Storage times in the lattice could be many seconds.  The lattice would be loaded with multiple launches, filling lattice sites that extend over 5-10 cm, and  $10^{8}$ or more  atoms could be used for the EDM measurement. After loading the lattice, the atoms would be optically pumped to maximum polarization and then the population transferred to the $m_F=0$ state by a series of microwave pulses. A large electric field ({\it e.g.} 150 kV/cm) would define the quantization axis in nominally zero magnetic field, and the energies would be  proportional to $m_F^2$ due to the parity-allowed interaction.  

In another planned innovation,  a Ramsey separated-field approach would be used with the free-precession interval  initiated by pulses that transfer atoms to a superposition of $m_F=F$ and $m_F=-F$ states and terminated by a set of pulses coherent with the initial  pulses. The relative populations transferred back to the $m_F=0$ state would be probed by optical fluorescence  that could be imaged with about 1 mm spatial resolution~\cite{rf:Zhu2013}.   With the large size of the lattice, the superposition of  stretched levels ($m_F=\pm F$) would amplify the sensitivity by a factor of $F$ relative to experiments that monitor $\Delta m_F=1$ transitions~\cite{rf:Xu1999}. In a measurement time $\tau=3$ s, and $N=2\times 10^8$, an EDM sensitivity of $6.5\times 10^{-26}$ \ecm~ for the cesium atom is expected. The optical lattice can also trap rubidium, which could be used as a comagnetometer. 

%Cold atom techniques have been developed over the past 25 years by a large number of groups, and appear to be able to provide statistical power and ways to monitor systematic effects. The continuing advances in the technology of fountain clocks provides encouragement, though an EDM measurement provides a set of distinctly separate systematic effects. An EDM measurement in a lattice would allow measurements with two or more species, thus providing a comagnetometer to monitor leakage-current and other effects.




\subsection{Paramagnetic polar molecules: YbF, ThO and $^{180}$Hf$^{19}$F$^+$}
\label{sec:PolarMolecules}

Molecular-beam EDM experiments exploit several features of diatomic polar molecules, usually one light and one heavy atom;
 most significantly the strong interatomic electric field with characteristic strength of 10-100 GV/cm. The pioneering molecular-beam EDM approach of~\cite{rf:SandarsTlF} and~\cite{rf:RamseyTlF} used thallium-fluoride (TlF) which is {\it diamagnetic} and is discussed in the next section. A number of groups have followed the lead of~\cite{Hudson:2002az} and investigated paramagnetic molecules. The most sensitive measurements in paramagnetic systems are from YbF~\cite{Hudson:2011zz},  ThO~\cite{Baron:2013eja,Baron:2016obh} and HfF$^+$~\cite{Cairncross:2017fip}.  In addition, efforts using the molecular ion  ThF$^+$, which is similar in electronic structure, are underway by~\textcite{Loh1220} .

Experimentally, the internal electric dipole moment $\vec D$ is oriented parallel or antiparallel to a relatively modest applied electric field in the lab $\vec E_{\rm lab}=E_{\rm lab}\hat z$. In essence the molecule's electric polarizability is very large resulting in an effective molecular dipole moment $D_z\propto E_{\rm lab}$.   In the atomic beam, the average orientation of $D_z$ can be expressed as a polarization ${\mathcal P}_z=D_z/|\vec D|$, for example the dependence of ${\mathcal P}_z$ for YbF is shown in FIG.~\ref{fg:YbFPz}. For ThO and HfF$^+$, which are effectively fully electrically polarized by a relatively small laboratory electric field, the estimated effective internal electric fields are $E_{\rm eff}\approx$ 84 GV/cm and $E_{\rm eff}\approx$ 23 GV/cm, respectively, which can be found from Table~\ref{tb:paramagnetics} using $E_{\rm eff}=\hbar\alpha_{d_e}/e$.

\begin{figure}[ht]
%\vskip-1.5 truein
%\hskip -1.6 truein
 \includegraphics[width=3.5truein,angle=0]{YbFPz}
 %\vskip -1. truein
 \caption{The effective electric field $E_{\rm eff}$ in YbF as a function of the applied electric field ($E_{\rm lab}$ in the text). The fully saturated $E_{\rm eff}$ corresponds to ${\mathcal P}_z=1$. Figure from~\textcite{Hudson:2002az}. 
 }
\label{fg:YbFPz}
\end{figure}


The electronic energy level structure of a $^\Sigma\Lambda_\Omega=^3\!\!\Delta_1$, {\it e.g.} $F=1$ state in the polar molecule ThO, is shown in FIG.~\ref{fg:MolecularStructure}. The quantum numbers labled by capital greek letters indicate angular momentum projected on the internuclear axis: ($\Sigma=3$ indicates the electron-spin triplet state, orbital angular momentum  $\Lambda=2$ is labeled by $\Delta$, and the total of spin and orbital angular momentum and rotation is $\Omega=1$.) The $^3\Delta_1$ state also results in a relatively small magnetic moment as the spin and orbital moments cancel. As shown, the laboratory electric field electrically polarizes the molecule, providing two directions of the internal electric field with different splittings due to the EDM, but the same magnetic field splitting. Thus the two orientations of the molecular dipole, which can be separately probed, for example by tuning a probe laser, provide an effective {\em internal} comagnetometer.

\begin{figure}[hb]
%\vskip-1.5 truein
%\hskip -1.6 truein
 \includegraphics[width=3.5truein,angle=0]{MolecularStructure}
 %\vskip -1. truein
\caption{Level structure of molecule in a  $^3\Delta_1$, $J=1$ state.   The laboratory electric field $E_{lab}$ splits the states by $\pm D_{el}E_{lab}$ into molecular dipole up and down. This results in  the two orientations of the effective internal field $E_{\rm eff}$, which is directed towards the lighter atom. The specific structure for the ThO experiment is shown; for HfF$^+$   $J=3/2$ and the stretched states ($m=\pm 3/2$) move up and down similarly. In a magnetic field $\vec B$ parallel or antiparallel to $\vec E_{lab}$, the $m=\pm 1$ states are further split by $\mu B$ as shown. The EDM $d$ further splits the $m=\pm 1$ states with $E_{eff}>0$, but reduces the splitting with $E_{\rm eff}<0$ antiparallel to $B$. }
\label{fg:MolecularStructure}
\end{figure}

The YbF experiment used ground-state $^2\Sigma_{1/2}^+$ molecules, where the $+$ indicates positive reflection-symmetry along a plane containing the internuclear axis~\cite{Hudson:2002az}. The resulting P-odd/T-odd  frequency shift is reported as
\begin{equation}
\omega^{EDM}({\rm YbF})=(5.3\pm 12.6\ {\rm (stat)}\pm 3.3\ {\rm (sys)})\ {\rm mrad/s},
\end{equation}
which can be interpreted as an electron EDM assuming $E_{\rm eff}=14.5$ GV/cm and $C_S=0$:
\begin{equation}
d_e({\rm YbF})=(-2.4\pm 5.7\ {\rm (stat)}\pm 1.5\ {\rm (sys)})\times 10^{-28}\ \text{\ecm}\ (C_S=0).
\end{equation}


The setup of the ACME ThO experiment is shown in FIG.~\ref{fg:ThOAcmeSetup} .The experiment used molecular-beam resonance methods based on the Ramsey separated-oscillatory-fields technique. For ThO, the $^3\Delta_1$ state is a metastable state originally populated from the ground state by the 943 nm optical pumping light.  In this case, $2D_{\rm el}E_{\rm lab}/h\approx$ 100 Mhz, and 1090 nm optical transitions  to an excited state with two opposite parity levels separated by 10 MHz are used to prepare and probe the orientations of a superposition of the $m=\pm1$ levels. If the state-preparation light is polarized along $\hat x$, then the initial electron spin is along $\hat y$. The EDM signal is a rotation around $\hat z$, which is detected by the component of the electron spin along $\hat x$ that reverses with the sign of $\vec E_{lab}\cdot \vec B$. The EDM frequency shift $\Delta\omega^{\rm EDM}$ is the rotation angle divided by $\tau\approx$1.1~ms, the transit time from pump to probe positions. ($\tau$ is determined from the magnetic precession angle $\phi^B=-\mu | B_z|\tau/\hbar$, where $\mu$ is the molecular magnetic moment, which is relatively small due to cancellation of spin and orbital effects.) A number of additional experimental parameters are changed to separate background and false-EDM signals including the molecular orientation ($E_{\rm eff}>0$ or $E_{\rm eff}<0$), the magnetic field  direction and magnitude, the electric field magnitude, the readout  laser polarization direction and various exaggerated imperfections. In all more than 40 parameters were varied.  Systematics were evaluated through a combination of anticipated effects informed by earlier experiments in YbF~\cite{Hudson:2011zz}  and PbO~\cite{Eckel:2013lsa}. The dominant systematic effects are generally the combination of two small effects, {\it e.g.} the AC stark shift caused by detuning the pump and probe lasers along with misalignments, gradients of the circular polarization and imperfect reversal of the electric field. The final result of the first-generation ThO experiment is a P-odd/T-odd precession frequency
\begin{equation}
\omega^{EDM}({\rm ThO})=(2.6\pm 4.8\ {\rm (stat)}\pm 3.2\ {\rm (sys)})\ {\rm mrad/s}.
\end{equation}


From Eqn.~\ref{PolarMoleculeDeltaOmega}, these  can be interpreted as the combination of contributions from the electron EDM $d_e$ and from a nuclear-spin independent (scalar) coupling labeled $C_S$; however adopting the ``sole-source'' approach (see Sec.~\ref{sec:TheoreticalInterpretation}), this can be interpreted as
\begin{eqnarray}
d_e({\rm ThO})&=&(-2.1\pm 4.5)\times 10^{-29}\ \text{\ecm}\quad (C_S=0),
\nonumber
\\
C_S({\rm ThO})&=&(-1.3 \pm 3.0)\times 10^{-9}\quad\quad\quad\quad (d_e=0).
\nonumber
\\
\end{eqnarray}
%The 95\% C.L. upper limits for the sole-source approach are $d_e\le 1.2\times 10^{-28}$ \ecm~ and $6.4\times 10^{-9}$ respectively. 
Further interpretation is provided in Sec.~\ref{sec:GlobalAnalysis}. 
This is considered a ``first-generation'' ThO effort by the experimenters, and upgrades to cold molecular beam promise improved statistical uncertainty.

Paramagnetic HfF$^+$molecular ions  in the metastable $^3\Delta_1$, $F=3/2$ state were confined in a radio frequency  trap to measure the P-odd/T-odd energy shifts by ~\textcite{Cairncross:2017fip}. (The specific isotopes were $^{180}$Hf and $^{19}$F.) 
%In this case, the effective electric field for a fully electrically polarized molecule in the lab frame was $|E_{eff}|\approx$23 GV/cm~\cite{Petrov2007,Fleig2013,Skripnikov2017},
%and $\alpha_{C_S}\approx 2.0\times 10^6$ rad/s~\cite{Skripnikov2017}. 
The molecular ions were produced by laser ablation of Hf metal in a supersonic jet with a mixture of  argon and SF$_6$ gas, which produced neutral ground state HfF molecules. The cold argon gas from the supersonic jet cooled the rotational and vibrational degrees of freedom of the molecules. UV lasers ionized the HfF, and the ions were initially  trapped by an axial static field and a 50 kHz  radial (quadrupole) field and then confined by a uniform electric field rotating at 250kHz. The result was that the ions rotated in a circle of about 1 mm diameter.  An axial magnetic field gradient created the bias field in the rest frame of the ions. The Ramsey-style EDM measurement consisted of laser preparation of a polarized spin state  followed by a $\pi/2$ pulse, which created a superposition of $m_F=\pm3/2$, a free precession time of about 700 ms, and a second $\pi/2$ pulse. The final phase was read out by selective laser depopulation of alternating $m_F=\pm3/2$ levels followed by laser ionization and detection of the Hf$^+$ and background ions by a microchannel plate. Systematic effects are studied by observing frequency shifts in channels that are not sensitive to the P-odd/T-odd effects. The important systematic effects included non-ideal reversal of the rotating magnetic field combined with the different gyromagnetic ratios of the upper (labeled $E_{\rm eff}>0$) and lower (labeled $E_{\rm eff}<0$) pairs of states (doublets) due to Stark mixing with $J=2$ states, geometric phases, and background. The difference of frequencies for two flips that project the EDM - flipping the bias magnetic field and flipping $E_{\rm eff}$ by selecting the upper to lower pair of states -- was reported: 
\begin{equation}
\omega^{BD}({\rm HfF^+})=2\pi(0.1\pm 0.87\ {\rm (stat)}\pm 0.2\ {\rm (sys)})\ {\rm mrad/s}.
\end{equation}
Assuming $C_S=0$, the resulting sole-source electron EDM is
\begin{equation}
d_e({\rm HfF^+})=(0.9\pm 7.7\ {\rm (stat)}\pm 1.7\ {\rm (sys)})\times 10^{-29}\ {\rm \ecm}.
\end{equation}
%for $E_{eff}=23$ GV/cm.
The authors did not provide a soul-source limit on $C_S$; however \textcite{Skripnikov2017} has calculated   $\alpha_{C_S}\approx 2.0\times 10^6$ rad/s, which is used in the analysis presented in Sec.~\ref{sec:GlobalAnalysis}, which includes results from ThO and $^{180}$Hf$^{19}$F$^+$  to constrain $d_e$ and $C_S$ simultaneously.



A second generation ion trap that may confine ten times more HfF$^+$ ions in a larger volume combined with  improved electrode design is expected to provide an order of magnitude higher sensitivity~\cite{Cairncross:2017fip}. The JILA group also intends to perform an experiment on ThF$^+$ ($E_{\rm eff}\approx$ 36 GV/cm), 
for which the ground state is $^3\Delta_1$ providing for coherence times that are not limited by the lifetime of an excited state. It has also been pointed out that an experiment with the isotope $^{177}$Hf  (18.6\% abundance)  with nuclear spin $I=7/2$ would be sensitive to the P-odd/T-odd magnetic-quadruople moment~\cite{Skripnikov2017MQM}.  
% 


\begin{figure}
%\vskip-1.5 truein
%\hskip -1.6 truein
 \includegraphics[width=3.5truein,angle=0]{ACMEThOSetup}
 %\vskip -1. truein
  \caption{  (Color online) The experimental layout of the ACME ThO experiment from~\textcite{Baron:2013eja}. 
 }
\label{fg:ThOAcmeSetup}
\end{figure}

A recent proposal to study the orientation-dependent hyperfine structure of polar molecules in a rare-gas matrix, which is sensitive to the electron EDM has been presented by~\textcite{edm3-2}.
Another promising idea  is to store paramagnetic molecular ions or other particles in an electrostatic storage ring of a few meters diameter, used as a large ion trap.\footnote{A ring with in principle suitable parameters exists~\cite{MPIfurKernphysikHeidelberg}. } Such a configuration enables the storage of molecular ions of all possible configurations of states. 
In TaO, for example, the  ground state structure is $^\Sigma\Lambda_\Omega=^3\!\!\Delta_1$, and ions could be trapped electrostatically for several hours in bunches of  up to 10${^7}$  ions with kinetic energies of the order 100 keV. Preparation and readout of the molecular states relevant for EDM measurements would be done with lasers. 
Due to the sub-kHz angular frequency of the particles, the small molecular magnetic dipole moment, and the eddy-current and RF shielded environment provided by the vacuum housing of the storage ring, no compensation of the ambient magnetic fields is necessary. 
The long storage times in the ring allow for a large number of repetitions of the experiment for each 
configuration, and the large number of ions stored in the ring may enable  up to six orders of magnitude greater sensitivity to $d_e$. At this level, EDMs and Majorana neutrinos have model dependent connections, thus enabling a new path to access physics beyond the SM \cite{Archambault:2004td,Ng:1995cs}. Additionally, radium or radon ions could be stored, taking advantage of the octupole enhanced Schiff moment or nuclear EDM discussed below.  
%Interestingly, such systems with a connection to the $\theta$ term at the same time enable the search for oscillating EDMs in the context of axion-like dark matter \cite{graham}. Here, the oscillation frequency of the background field is matched with frequency of the particle in the storage ring to match a resonant condition for detection. 



\subsection{Solid-state systems}

The electron EDM can also be measured in special ferro-electric and paramagnetic solid-state systems with quasi-free electron spins that can be subjected to applied electric and magnetic fields. Advantages of such a system are
\begin{enumerate}[i.]
\item a high number density of unpaired electrons ($10^{22}$ cm$^{-3}$), providing signal amplification;
\item confinement of the electrons, mitigating such effects as motional fields; 
\item features of solid-state samples including collective effects, {\it e.g.} for ferro-electric systems,  a large electric field spin-polarizes the electrons resulting in a magnetization that reverses with the electric field;
\item minimal magnetic order to mitigate spurious magnetic effects. 
\end{enumerate}
A cryogenic experiment increases the electron polarization and provides for detection of the resulting magnetization  by SQUID magnetometers. 
The desired properties of an ideal material follows from consideration of the specific requirements of the EDM search~\cite{Ignatovich:1969tv,
Ignatovich:1969tv-ru,rf:Liu2004,rf:Shapiro1968,rf:Buhmann2002,rf:Sushkov2009,rf:Sushkov2010}. 

The polycrystal Gd$_3$Ga$_5$O$_{12}$ (gadolinium-gallium-garnet) provides seven unpaired atomic electrons, high resistivity ($10^{14}$ $\Omega$-m ) and high dielectric strength (1 GV/m). Enhancement of the electron EDM leads to an atomic EDM of Gd$^{3+}$  atoms in the lattice  $d_{Gd^{3+}}\approx 20d_e$, and the result $d_e=(-5.57\pm 7.98_{stat}\pm 0.12_{syst})\times 10^{-25}$ \ecm~ was reached with 5 days of data~\cite{PhysRevD.91.102004}. An experiment in the paramagnetic ferroelectric Eu$_{0.5}$Ba$_{0.5}$TiO$_{3}$ measured $d_e=(-1.07\pm 3.06_{stat}\pm 1.74_{syst})\times 10^{-25}$ \ecm. \cite{Eckel:2012aw}. Though this result is several orders of magnitude short of the sensitivity of paramagnetic molecules, improvements to magnetic noise and shielding 
can improve sensitivity. Other materials under consideration include SrTiO$_3$ doped with Eu$^{2+}$~\cite{PhysRevB.19.3593,PhysRevB.50.601}.
%Buhmann, Dzuba, Sushkov, Phys. Rev. A 66, 042109 (2002).
%Dzuba, Shushkov, Johnson, Safronava, Phys. Rev. A 66, 032105 (2002). 
%Kuenzi, Sushkov, Dzuba, Cadogan, Phys. Rev. A 66, 032111 (2002).
%Mukhamedjanov, Dzuba, Sushkov, Phys. Rev. A 68, 042103 (2003).
Another approach in paramagnetic ferroelectrics,  would detect the electric field produced by the electron EDM that would be magnetically aligned by polarized spins~\cite{rf:Heidenreich2005}. 


\subsection{Diamagnetic atoms and molecules}
\label{sec:XeHg}

Diamagnetic atoms have the experimentally attractive feature that they can be contained in room-temperature cells with long polarization and spin-coherence lifetimes $T_1$ and $T_2^*$, because the nuclear spin is well shielded by the closed electron shell. Diamagnetic atoms can also be spin polarized using optical-pumping techniques, providing the largest possible signal-to-noise ratios and optimal statistical precision. Combined with techniques to carefully monitor and control systematic effects, measurements with $^{129}$Xe~\cite{rf:Vold1984}, with $^{129}$Xe/$^3$He~\cite{rf:Rosenberry2001} and the series of measurements with $^{199}$Hg~\cite{rf:Romalis2001,Griffith:2009zz,Graner:2016ses} are the most sensitive EDM measurements to date. The most recent $^{199}$Hg result stands alone in its sensitivity to various sources of CP violation~\cite{Graner:2016ses-erratum}. The diamagnetic molecule TlF was used in pioneering work by~\textcite{,rf:Hinds1980a} and~\textcite{rf:Wilkening1984}. The most recent and most precise TlF result was reported by~\textcite{rf:Cho1991}, and a new effort to greatly improve the sensitivity is underway~\cite{PhysRevA.95.062506}.

%Diamagnetic atoms %, {\it i.e.} atoms with closed electron shells,  acquire EDMs from several possible sources. Sensitivity to the Schiff moment of the nucleus increases approximately as $Z^2$ due to the electron momenta, relativistic effects, and the size of the nucleus ~\cite{rf:Dzuba2002}.
% V. A. Dzuba, V. V. Flambaum, J. S. M. Ginges, andM. G. Kozlov, Phys. Rev. A 66, 012111 (2002). Diamagnetic atoms are also sensitive to T-odd/P-odd neutral current interactions between the electrons and the nucleus (Tensor, Scalar or Pseudoscalar) ~\cite{rf:Ginges2004,rf:Martensson1985}. The EDM of a diamagnetic atom can also be induced, at higher order, by the electron's EDM ~\cite{rf:Flambaum1985,rf:Martensson1987}.
%V. V. Flambaum and I. B. Khriplovich, Sov. Phys. JETP 62, 872 (1985). [21] A.-M. Martensson-Pendrill and P. O? ster, Phys. Scr. 36, 444 (1987).  In general, one can write
% \begin{equation}
% d_{A}=   \kappa_{S} S   + \eta_e d_e + ( k_T C_T + k_S C_S+ k_P C_P)  .
 %\label{eq:DiamagneticEDM} 
 %\end{equation}
%Here $C_T$, $C_S$ and $C_P$ refer to P-odd and T-odd couplings of the electron to a current-density of the nucleus that transforms as a tensor, scalar and pseudoscalar, respectively. For an infinitely heavy nucleus, the recoil velocity is zero and $C_P$ would vanish. 
 
As discussed in Sec.~\ref{sec:EFTParameters} and in Eqn.~\ref{eq:DiamagneticAtoms}, the dominant contributions to the atomic EDM in diamagnetic atoms is the Schiff moment of the nucleus and the nuclear-spin-dependent electron-nucleus force with coefficient $C_T^{(0)}$. The Schiff moment  itself can arise from T-odd/P-odd NN interactions and from the EDMs of the individual nucleons  (both $^{129}$Xe and $^{199}$Hg have an unpaired neutron). However these sources can be related, depending on the nature of the P-odd/T-odd interactions as discussed in Sec.~\ref{sec:Theory}.
% The P-odd/T-odd contributions to the Schiff moment are generally separated into isospin contributions as indicated in Eqn.~\ref{eq:SchiffMoment} 




%\subsubsection*{Xenon}
\noindent{\bf Xenon}
\label{sec:Xe}

Xenon is the heaviest stable noble gas, and $^{129}$Xe is a spin-1/2 isotope. Spin-1/2 atoms in cells have the advantage that only magnetic dipole interactions with external fields, with other atoms, and with the cell walls are allowed. This leads to longer spin-coherence times and narrow linewidths compared to atoms with nuclear spin $K>1/2$, which are subject, for example, to electric quadrupole interactions, in particular with the cell walls~\cite{rf:Wu1990,rf:Chupp1990}. Spin relaxation times of several 100's of  seconds and longer are observed for free-induction decay.  In natural xenon, the abundance of $^{129}$Xe is 26\%; however isotopically enriched gas is available. Polarization of $^{129}$Xe generally of greater than 10\%, and approaching 100\%, is possible using spin-exchange, mediated by the hyperfine interaction, with laser-optically-pumped alkali-metal vapor~\cite{rf:Zeng1985}. Spin exchange also makes it possible to use the alkali-metal vapor to monitor the free-precession of $^{129}$Xe polarization.

The first EDM measurement in $^{129}$Xe by Fortson and collaborators~\cite{rf:Vold1984} used spin exchange with laser-optically-pumped rubidium to polarize $^{129}$Xe in a stack of three cylindrical cells with electric fields of magnitude 3.2 to 4.9 kV/cm applied parallel and antiparallel to a uniform and well shielded 10$\mu$T  magnetic field. The stack of cells, treated as magnetometers, allows sums and differences of the free-precession frequencies to be used to determine the average magnetic field, and the average magnetic field gradient. A third combination of the three frequencies is the EDM signal. %The magnitude of the applied electric field is notably much lower than in the atomic/molecular beam and neutron EDM experiments due to the compactness of the cells, the materials and the buffer gas. 
%The cells contained 0.2 torr (at room temperature) of natural xenon and 220 torr of N$_2$ along with a small amount of rubidium metal. The relatively low xenon pressure points out a feature of spin-exchange optical pumping: the noble gas is a strong relaxation mechanism for the rubidium spin, which was produced by optical pumping. The experiment could be optimized, or at least a favorable trade-off established, by adjusting the noble gas density, laser power and the rubidium vapor pressure. The rubidium vapor pressure was maintained by controlling the cells' temperature at  65$^\circ$ C.
 %The xenon was polarized in a relatively strong field of 10 mG oriented transverse to the cell axes and electric field and then the field pumping field was rapidly reduced leaving the spin to precess freely about the cell axes at about 0.1 s$^{-1}$. The precessing $^{129}$Xe spin was detected by the rubidium using a magnetometer technique originally developed by CohenTannoudji~\cite{rf:CohenTannoudjiMagnetometer} and adapted by Volk and Grover~\cite{rf:VolkGrover}. 
One potential systematic error for such a system was the effective magnetic field due to the hyperfine interaction, caused by any rubidium polarization projection along the electric field axis that somehow changed when the electric fields were changed. One successful approach was to ``quench'' the polarization of the two rubidium isotopes with resonance RF magnetic fields~\cite{OteizaDissertation}. Another concern was any change in the leakage currents that flowed across the cells due to the applied voltages that was different for different cells. Both effects were studied and found to be small compared to the statistical error of the measurement. The EDM of $^{129}$Xe was measured to be  $d_{\rm Xe}=(-0.3\pm 1.1)\times 10^{-26}$ \ecm, where the error is statistical only.

Another approach to measure the $^{129}$Xe EDM used  spin-exchange pumped noble-gas masers of $^{129}$Xe and $^{3}$He~\cite{rf:ChuppMaser1,Stoner:1996zz,Bear:1998zz}. Spin-exchange optical pumping is practical, in principle, for any odd~$A$ noble gas, and a population inversion can be pumped in multiple species with the same sign of the magnetic moment. The two species have very different sensitivity to the Schiff moment and to other P-odd/T-odd interactions, which are approximately proportional to $Z^2$, but similar sensitivity to magnetic field effects, particularly those produced by leakage currents that can change when the electric field is changed. Thus the $^3$He served as a comagnetometer occupying nearly the same volume as the $^{129}$Xe in a single measurement cell~\cite{rf:Chupp88}. 
%The spin-exchange cell, which required a large rubidium density to allow both the $^3$He and $^{129}$Xe masers to operate effectively, was separated from the EDM measurement cell by a transfer tube through which the spin-polarized atoms diffuse. In the measurement cell, radiation damping,  due to the coupling of the precessing spins to a tuned coil,  increased the projection of the magnetization vector onto the plane transverse to the applied magnetic field at rate $\Gamma_{RD}$; while the spin diffusion  increased the projection along the applied magnetic field ($z$-axis) at a rate $\gamma_z$; and relaxation (at the rate  $1/T_1$)  reduced the longitudinal component. (The magnetic moments of $^3$He and $^{129}$Xe are both negative so the high-energy Zeeman state has spin parallel to the applied magnetic field.) Decoherence of the ensemble at the rate $1/T_2^*$, mostly due to magnetic field inhomogeneity  tended to reduce the magnitude of the transverse component of magnetization along the $x$-axis. The combination of these  effects, illustrated in the rotating frame depiction shown in Figure~\ref{fg:Maser}, leads to a steady state equilibrium magnetization for each species  in the frame rotating at the respective Larmor frequency. For the EDM measurement,  the $^{129}$Xe precession frequency was phase locked to a local oscillator that controlled the magnetic field. Thus any change in the frequency, when the electric field was reversed,  changed the magnetic field and change the $^3$He frequency, which is about 3-times larger than the $^{129}$Xe frequency shift. 
The result reported by~\textcite{rf:Rosenberry2001} was
 \begin{equation}
d_A(^{129}{\rm Xe}) = (0.7\ \pm 3.3\ ({\rm stat})\ \pm 0.1\ ({\rm sys}))\times 10^{-27}\  e{-\rm cm}.
\end{equation}

%\begin{figure}
%\vskip-1.5 truein
%\hskip -1.6 truein
 %\includegraphics[width=5truein,angle=0]{Maser}
 %\vskip -1. truein
 % \caption{Principle torques on the magnetization vector, $\vec M$, in the noble-gas maser depicted in the rotating frame. The magnetic field is along $z$ and the gain and signal are proportional to the transverse projection along $x$.
% }
%\label{fg:Maser}
%\end{figure}
%was statistics limited by instability of the masers that  arose due to a number of factors including a
%a change of one of the torques on the magnetization vector illustrated in Figure~\ref{fg:Maser}. Specifically, any
%change of the rate at which spin diffuses into the measurement cell due, the polarization and the tuning of the resonant circuits, due to fluctuations of laser power, temperature,  magnetic field {\it etc}.  This led to small oscillations around the equilibrium magnetization and thus a change in the magnetization projection along the $z$-axis. The resulting change in magnetization effectively changed the magnetic field around which the spins precessed and introduced frequency noise and fluctuations. This would vanish in perfectly spherical cells, but the maser-cell configuratio was far from spherical. These effects were also mitigated as the projection of the magnetization onto the transverse plane approaches a maximum ($\theta=\pi/2$).  An active maser based on amplified feedback has also been developed~\cite{rf:Yoshimi2002}. 

Several experimental efforts to improve the $^{129}$Xe EDM sensitivity by 2-3 orders of magnitude are underway, including the active maser~\cite{rf:Yoshimi2002}, highly polarized liquid $^{129}$Xe detected with SQUID magnetometers~\cite{rf:RomalisLiqXe,LedbetterDissertation} and gas-phase experiments with $^3$He comagnetometry and SQUID-magnetometer detection~\cite{rf:FloHeXePaper,rf:MainzHeXePaper}. The SQUID-magnetometer experiments have demonstrated signal and noise that suggest one to three orders of magnitude improvement in sensitivity to the $^{129}$Xe EDM is possible in the near future.
%A prototype experiment with a polarizer cell and Si-electrode EDM cell was recently performed in the Munich Shielded Room. The $^{129}$Xe/$^3$He mixture was polarized in a polarizer valved cell outside of the shielded room and transferred into the room through teflon tubing to a GE180 EDM cell with Si electrodes similar to that shown in fig~\ref{fg:CellValvePhotos}. Signals from spin-precession measurements with a SQUID magnetometer array with and without high voltage showed coherence times ($T_2^*$)  of  2400 s and 3600 s, respectively for $^{129}$Xe and $^3$He  (fig~\ref{fg:HeXeFID}). The gas transfer used hand-actuated valves which will be replaced by automated valves in the final system. Details of preparation of the glass cells and the bonding technique are described in papers currently in preparation.%~\cite{rf:FloEDMProspectPaper,rf:SkylerEDMCells}. %Contemporary efforts using polarized liquid $^{129}$Xe   are discussed in section~\ref{sec:XenonLiq}

%\begin{equation}
 %d_{\rm Xe}=  ( 0.4\times10^{-20}  C_T)  e\ {\rm cm} + 2.7\times 10^{-18} S_{Xe}  {{\rm cm}\over {\rm fm^3} }.
 %\end{equation}
 %Dzuba1985,Martensson-Pendrill


%In contrast to the $^{199}$Hg based comagnetometer, it should be noted that various comagnetometers have previously been realized. 
%
%In particular, $^{129}$Xe and $^3$He have been deployed to search for small effects. 
%
%The different pressures and conditions have significant impact on cross talk, mean free path, geometric phase-like phase frequency shifts and detection possibilities, see sec.~\ref{sec:xenon}.
%



%\subsubsection*{Mercury}
\noindent{\bf Mercury}
\label{sec:Hg}

 
The $^{199}$Hg experiments undertaken by Fortson's group~\cite{rf:Romalis2001,Griffith:2009zz,Graner:2016ses} built on the ideas used in their $^{129}$Xe buffer-gas cell experiment~\cite{rf:Vold1984}. However there are two crucial differences with mercury: it is more chemically reactive, resulting in shorter coherence times, and it is heavier and thus generally more sensitive to sources of T and P violation. The most recent experiment~\cite{Graner:2016ses} used a stack of four cells sealed with sulfur-free %Lesker KL-5
 vacuum sealant and directly pumped and probed the $^{199}$Hg with a 254 nm laser~\cite{rf:Harber2000} as illustrated in FIG.~\ref{fg:HgSetup}. 
%D. M. Harber and M. V. Romalis, Phys. Rev. A 63, 013402 (2000).
The outer two of the four cells have no electric field and the inner two have electric fields in opposite directions so that a difference of the free-precession frequencies for the two inner cells is an EDM signal. An EDM-like difference of the outer cell frequencies was attributed to spurious effects such as non-uniform leakage currents correlated with the electric field reversals and were therefore scaled and subtracted from the inner-cell frequency difference to determine the EDM frequency shift. The magnitudes of the leakage currents were also monitored directly and used to set a maximum E-field correlated frequency shift that contributed to the systematic error estimate. Other systematic error sources explored included effects of high-voltage sparks on the EDM signals and a number of possible correlations of experimentally monitored parameters ({\it e.g.} laser power and magnetic field fluctuations outside the magnetic shields). There were no apparent correlations, and the leakage current ($\pm$ 0.5 pA) was so small that only upper limits on the systematic errors could be estimated. The most recent result is~\cite{Graner:2016ses-erratum} 
\begin{equation}
d_A(^{199}{\rm Hg}) = (2.20\ \pm 2.75\ ({\rm stat}) \  \pm 1.48\ ({\rm sys})  )\times 10^{-30} e{-\rm cm}. 
\end{equation}


\begin{figure}
%\vskip-1.5 truein
%\hskip -1.6 truein
 \includegraphics[width=3.5truein,angle=0]{HgSetup}
 %\vskip -1. truein
  \caption{ (Color online) The experimental layout of the Seattle $^{199}$Hg experiment from~\textcite{Graner:2016ses}. 
 }
\label{fg:HgSetup}
\end{figure}



%To further explore some of the implications of this null result, we rewrite equation~\ref{eq:DiamagneticEDM} for $^{199}$Hg:
%\begin{equation}
 %d_{\rm Hg}=  ( 2\times10^{-20}  C_T +6\times10^{-22} C_S+ 6\times10^{-23} C_P)  e{-\rm cm} -2.8\times 10^{-17} S_{Hg}  {{\rm cm}\over {\rm fm^3} } +0.01 d_e.
 %\label{eq:Hg}
%\end{equation}
 %The contributions to $S_{Hg}$ include T-odd and P-odd  NN interactions and intrinsic EDMs of the proton or neutron. Comparing these two contributions:
%\begin{equation}
%S_{\rm Hg}^{\theta_{QCD}}=  a_0g_{\pi NN}\bar g_{ CP}^0 =   3.6\times 10^{-3}\theta_{QCD} \quad ({\rm NN\ interactions});
%\end{equation}
 % estimated in reference~\cite{rf:deJesus05}.
%\begin{equation}
%S_{\rm Hg}^{\theta_{QCD}}=  s_n d_n =  6.8\times 10^{-3} \theta_{QCD}  \quad \quad \quad  \quad (d_n\ {\rm from}\ S_{\rm Hg}).
%\end{equation}
%For the NN interaction contribution, we use $\bar g_{ CP}^0=0.027\theta_{QCD}$~\cite{rf:Crewther1980} and $a_0=0.01$, from the (SkO) potential in~\cite{rf:deJesus2005}. For the neutron-EDM contribution, we take $S_{\rm A}=s_p d_p+s_n d_n$, with $s_p=0.2$ fm$^2$ and  $s_n=1.9$ fm$^2$; the uncertainties could be as high as 30\%~\cite{rf:Dmitriev2003}.   Note that the T-odd and P-odd NN interaction and $d_n$  may be related to each other through more fundamental interactions. For example $\theta_{QCD}$ contributes to the neutron EDM  ($d_n=3.6 \times  10^{-16}\theta_{QCD}\ e$-cm) as well as to $\bar g_{ CP}^0$.  Interestingly, if we assume all other contributions in equation~\ref{eq:Hg} are zero, this result provides an indirect measurement of the neutron EDM:  $d_n(^{199}{\rm Hg})=(-0.9\pm 2.8)\times 10^{-26}$. This is comparable to the current sensitivity of direct neutron-EDM measurements. Similarly, the strongly model-dependent result for the electron EDM was $d_e(^{199}{\rm Hg})=(0.49\pm 1.5)\times 10^{-27}$.
% and suggest that the EDM of $^{199}$Hg alone should not be use as a measure of $\theta_{QCD}$ and $d_n$ simultaneously.  Intrinsic and induced EDMs of the quarks  also show up in both the  neutron EDM and Schiff moment~\cite{rf:Pospelov2001,rf:Pospelov2005}, and a comparison is provided in~\cite{rf:Griffith2009}.
%M. Pospelov and A. RItz, PHYSICAL REVIEW D, VOLUME 63, 073015
 
 


 



 
% In calculating the $a$'s, the pion�exchange P� and T�violating interaction, collective effects that are known to renormalize strength distributions of Schiff�like operators, pairing at the mean�field level, self�consistency, along with a number of possible Skyrme interactions were included. 
%Where $g_{\pi\rm NN}=13.5$ is the strong $\pi$NN coupling constant and $g_{\bar CP}^{0,1,2}$ are  isoscalar, isovector and isotensor-CP violating $\pi$NN couplings and the coefficients $a_{0,1,2}$ depend on the details of the assumed nucleon-nucleon interaction.


%\subsubsection*{TlF}
\noindent{\bf TlF}
\label{sec:TlF}

Molecular beam experiments using TlF were pursued by Sandars~\cite{rf:Harrison1969,rf:Harrison1969-erratum,rf:Hinds1980a}, by Ramsey~\cite{rf:Wilkening1984} and  by Hinds~\cite{rf:Schropp1987,rf:Cho1991}. For molecular beams, the  systematic errors associated with the $\vec v\times \vec E$ and leakage current effects are mitigated by using a relatively small applied electric field to align the intermolecular axis as is the case with polar molecules discussed in~\ref{sec:PolarMolecules}. This results in a large internal electric field at the thallium nucleus~\cite{rf:Coveney1983}.  The experiment is set up to detect  an alignment of a spin or angular momentum along the electric field by detecting precession around the internuclear axis, {\it i.e.} the frequency shift when the relative orientation of  applied electric and magnetic fields are reversed. 
When the average projection of the thallium nuclear spin on the internuclear axis is taken into account ($\langle\cos\theta_{\sigma\lambda}\rangle=0.524$), a frequency shift for full electric polarization is determined to be $d=(-0.13\pm 0.22)\times 10^{-3}$ Hz. With the applied electric field of 29.5 kV/cm, the most recent result~\cite{rf:Cho1991} is interpreted as a permanent dipole moment of the thallium molecule of  
\begin{equation}
d_{\rm TlF}=(-1.7\pm 2.9)\times 10^{-23}e\ {\rm cm}.
\end{equation}

For TlF, the electron spins form a singlet, but both stable  isotopes of thallium ($^{203}$Tl and $^{205}$Tl) have nuclear spin $J^\pi=1/2^+$, and the dipole distribution in the nucleus would be aligned with  the spin through T and P violation. This gives rise to the Schiff moment.
An alternative (and the original) interpretation is based on the observation that in the odd-A thallium isotopes, one proton  remains unpaired and can induce the molecular EDM through both the  Schiff moment (see Eqns.~(\ref{eq:dndp}-\ref{eq:etanetap})) and through magnetic interactions~\cite{rf:Coveney1983}. Separating these, the proton EDM would produce a magnetic contribution to a molecular EDM  of $d_{\rm TlF}^{p-mag}=0.13\ d_p$, and a contribution to the Schiff moment that would produce a molecular EDM estimated to be $d_{\rm TlF}^{p-vol}=0.46\ d_p$. The TlF molecular EDM can also arise from the electron EDM and from  P- and T-violating scalar and tensor electron-hadron interactions. However paramagnetic systems are more sensitive to $C_S^{(0,1)}$ and diamagnetic systems such as TlF are more sensitive to $C_T^{(0,1)}$. 
%Combining these sources 
%\begin{equation}
% d_{\rm TlF}= 0.13 {d_p}+ 81 {d_e}+ ( 1.1\times10^{-16}  C_T +2.8\times10^{-18} C_S)\ e{-\rm cm} - 7.4\times 10^{-14}  S_{\rm Tl} {{\rm cm}\over {\rm fm^3} }.
 %d_{\rm TlF}= 0.13 {d_p}+ 81 {d_e}+ ( 1.1 C_T +0.03 C_S)\times10^{-16}\ e{-\rm cm} -(7.4\times 10^{-14} S_{\rm Tl} ) {{\rm cm}\over {\rm fm^3} }.
 %\label{eq:TlF}
 %\end{equation}
% The proton-EDM contribution to the Schiff moment is not explicitly written in equation~\ref{eq:TlF}, rather it is part of $S_{\rm Tl}$. From section~\ref{sec:CsTl}, $d_e$ is less than $2.7\times 10^{-25}$ \ecm, and the $^{199}$Hg result constrains the Schiff moment to less than $10^{-12}$ $e$ fm$^3$.
% If $C_T$ and $C_S$ are comparable in magnitude (a model dependent assumption), the sensitivity to tensor coupling is dominant and also tightly constrained by $^{199}$Hg. 
Thus this measurement has been interpreted as a (model dependent) measurement of the proton EDM: $d_p=(-3.7\pm 6.3)\times 10^{-23}$ \ecm.

A new effort is underway to measure CP violation in TlF using cooled molecules, which is based on work by~\textcite{PhysRevA.85.012511,PhysRevA.95.062506}. This would combine advantages of longer free-precession times and greater statistical power, which could provide several orders of magnitude greater sensitivity compared to the result of~\textcite{rf:Cho1991}.
 
\subsection{Octupole collectivity in diamagnetic systems}
\begin{figure}\vskip-1.7 truein
\hskip -1.1 truein
\includegraphics[width=4.4 truein,angle=0]{Dobaczewskixxxxxs0103}
\vskip -2 truein
  \caption{ (Color online) Estimates of intrinsic Schiff moments for octupole deformed nuclei from ~\textcite{Dobaczewski:2018nim}. Figure used with permission from the author. 
 }
\label{fg:OctSchiffDobaczewski}
\end{figure}

Recently, experimental efforts have focused on exploiting the enhanced Schiff moment in nuclei with closely spaced ground state parity doublets and strong E1 matrix elements, characteristic of  isotopes with nuclear octupole  collectivity. For  proton and neutron numbers in the range $Z$ or $N\approx 34,\ 56,\ 88$ and $N\approx 134$, nulceons near the Fermi surface populate states of opposite parity separated by total angular momentum 3$\hbar$, corresponding to reflection asymmetric states that lead to permanent octupole deformation or  octupole vibration. In either case the intrinsic dipole moment is polarized along the nuclear-spin by P-odd/T-odd interactions leading to the nuclear Schiff moment, which is enhanced by the electric polarizability of the nucleus, in analogy to polar molecules (see Sec.~\ref{sec:PolarMolecules}). In the two-state approximation, the resulting Schiff moment can be parameterized as
\begin{equation}
S\propto \eta e\frac{\beta_2\beta_3^2ZA^{2/3}r_0^3}{E_+-E_-},
\label{eq:SchiffMomentOctupole}
\end{equation} %\approx 7\times 10^{-6}\eta\ e\ {\rm fm}^3,
where $\eta$ represents the strength of the P-odd/T-odd interaction, $\beta_2$ and $\beta_3$ are the quadrupole and octupole deformation parameters and $E_\pm$  are the energies of the opposite-parity states~\cite{rf:Auerbach1997,rf:Auerbach1996,spevak95}. 
Note that the octupole deformation parameter enters $S$ quadratically, which means that both octupole vibrations and permanent deformation are equally effective~\cite{PhysRevC.78.014310}. Permanent deformation is indicative of (and indicated by) closely spaced parity doublets: $(E_+-E_-)\approx$ 50-100 keV. In $^{229}$Pa, the splitting was originally reported to be as small as 0.22 keV~\cite{Pa229AhmadPRL1982}, and evidence of strong octupole correlations support a ground-state parity doublet with $I=5/2$~\cite{Pa220AhmadPhysRevC.92.024313}, which was also predicted theoretically by~\textcite{CHASMAN19807}. In fact the evidence of the closely-spaced doublet in $^{229}$Pa provided motivation for the suggestion of enhanced T-nonconserving nuclear moments by~\textcite{Haxton:1983dq}, which followed suggestions in the earlier the work of~\textcite{rf:Feinberg1977}. 


%Octupole collectivity, {\it i.e.} significant $\beta_3$,  can arise when nucleons near the Fermi surface occupy opposite parity states with $\Delta L=3$ and $\Delta J=3$. 
There is strong  evidence of octupole collectivity for nuclei with $A\approx 200-226$, including interleaved even/odd parity states in even-A nuclei~\cite{rf:NaturePaperRef7},   parity doublets in odd-A nuclei~\cite{rf:NaturePaperRef8} and enhanced electric-dipole (E1) transition moments~\cite{rf:NaturePaperRef9}.   The strongest direct evidence for octupole collectivity  has come from recent measurement of E3 strength using Coulomb excitation of radioactive beams of $^{220}$Rn and $^{224}$Ra at ISOLDE~\cite{rf:NaturePaper}. 
%An overview of the experiment and theory is given in ref~\cite{rf:NaturePaperRef21}. 
$\beta_2$ and $\beta_3$ for  $^{220}$Rn and $^{224}$Ra 
 are quite similar. However the larger moment $Q_3$ in $^{224/226}$Ra compared to  $^{220}$Rn %as well as $^{208}$Pb, $^{230/232}$Th and $^{234}$U 
 suggests that the deformation is permanent in $^{224}$Ra, while $^{220}$Rn is a vibrator~\cite{rf:NaturePaper}. Recently
 ~\textcite{Dobaczewski:2018nim} has estimated the intrinsic Schiff moments of nuclei in this region based on $Q_3$'s extracted from $^{224}$Ra~\cite{rf:NaturePaperRef9} and from $^{226}$Ra~\cite{rf:NaturePaperRef7}, which are shown in FIG.~\ref{fg:OctSchiffDobaczewski}. To estimate the observable Schiff moment further requires calculation of the P-odd/T-odd matrix elements arising from these intrinsic moments (see also~\cite{dobaczewski05}). 
 
 An intriguing possibility for future experimental efforts is to use a molecule with an ocutpole-enhanced nucleus, for example $^{225}$RaO~\cite{FlambaumRaO}. Though experimentally very challenging and potentially limited by production of appropriate molecules, this would take advantage of both the possible octupole-enhanced Schiff moment and the very large internal electric fields in the molecule. 
 
 %\begin{table}[ht]
 %\centering
 %\begin{tabular}{|c|c|c|} 
 %\hline
 %Isotope &  $|\beta_2|$ & $|\beta_3|$ \\
 %\hline
 %$^{220}$Rn & 0.119 & 0.095 \\
 %\hline
 %$^{224}$Ra & 0.154 & 0.097  \\
 %\hline
 %\end{tabular}
 %\caption{\footnotesize \label{tb:Beta2Beta3}Deformation parameters as reported in ref~\cite{rf:NaturePaper}.}
 %\end{table}
 
%Alternatively, the Schiff moment can be expressed in terms of the isospin components of a P-odd/T-odd pion-nucleon interaction~\cite{rf:Dmitriev2005,rf:deJesus2005}:
%$
%S= {13.5}(a_0 \bar g_{ \pi}^0 + a_1 \bar g_{ \pi} ^1	+	a_2\bar g_{ \pi}^2).
%\label{eq:SchiffMoment}
%$
%Each isospin contribution may isolate specific physics, for example quark EDMs contribute to the isovector component, and  $\bar\theta$ contributes to $\bar g_\pi^0$~\cite{rf:Crewther1979,rf:Crewther1980}.  The coefficients $a_{0,1,2}$ have been calculated  for several important cases in the context of Skyrme-models~\cite{rf:EngelOctupoledMainRef}, while Woods-Saxon and Nilsson potentials have been used for a range of nuclei by Auerbach and collaborators~\cite{rf:Spevak1997}. Note that the $\bar g_\pi^2$ contribution requires isospin breaking and is expected to be suppressed~\cite{rf:EngelEDMTheoryReview} so that the dominant relevant parameters are $\bar g_\pi^0$ and $\bar g_\pi^1$. 

%\subsection*{$^{225}$Ra}

\bigskip
\noindent{\bf $^{225}$Ra}

\begin{figure}
%\vskip-1.5 truein
%\hskip -0.5 truein
 \includegraphics[width=3.25truein,angle=0]{ra-setup}%{RadiumSetUp}
 %\vskip -1. truein
  \caption{  (Color online) The experimental layout of the $^{225}$Ra experiment from~\textcite{rf:Parker2015,PhysRevC.94.025501}. 
}
\label{fg:RadiumSetUp}
\end{figure}

An ongoing effort at Argonne National Lab uses cold-atom techniques to measure the EDM of the $^{225}$Ra atom~\cite{rf:Parker2015,PhysRevC.94.025501}, which may have 2-3 orders of magnitude greater sensitivity to the P-odd/T-odd pion-nucleon couplings than $^{199}$Hg based on atomic~\cite{rf:DzubaPRAv60p02111a2002} and nuclear physics calculations~\cite{rf:Auerbach1996,rf:Auerbach1997,rf:EngelFriarHayes,rf:deJesus}. 

The apparatus is shown in FIG.~\ref{fg:RadiumSetUp}, and the atomic level structure of radium is shown in FIG.~\ref{fg:RadiumAtomLevels}. The key components of the apparatus were the radium source, the Zeeman slower loading the magneto-optical trap (MOT) and the optical-dipole trap (ODT). The radium was provided by sources of up to 9 mCi, which provide both $^{225}$Ra ($t_{1/2}$=14 d) and significantly greater quantities of $^{226}$Ra ($t_{1/2}$=1600 y), which has no spin or EDM, but which is useful for diagnostics and tuning the optical traps.
The Zeeman slower used the momentum transferred from photons in a counter-propagating laser beam which was kept close to the 714 nm intercombination-line resonance by the Zeeman shift in a spatially-varying magnetic field. The slowed atoms then entered the MOT operating on the same transition.
Atoms that leak to the $7s6d^3$ $^3D_1$ level were repumped to the ground state with a 1429 nm laser. The trapping efficiency of the  MOT was approximately $10^{-6}$, largely limited by the low scattering rate. Radium atoms were accumulated for about 40 s and cooled to $\approx 40$ $\mu$K with typically $10^5$ $^{226}$Ra atoms or $10^3$ $^{225}$Ra atoms trapped in the MOT. From the MOT, atoms were transferred with high efficiency (80\%) to the ODT, effected by a 40 W, 1550 nm, laser beam focused with an $f$= 2 m lens.  By translating the lens, the ODT-trapped atoms were moved one meter into a separate chamber within a cylindrical-multilayer magnetic shield, where a standing-wave 1550 nm ODT held the atoms for the EDM measurement. The ODT holding time-constant of $\approx40$ s was limited by collisions with residual gas atoms.  A pair of copper electrodes separated by 2.3 mm provide an electric field of 67 kV/cm.



\begin{figure*}
%\vskip-1.5 truein
%\hskip -0.5 truein
 \includegraphics[width=6.5truein,angle=0]{ra-energy-levels}%{RadiumAtomLevels}
 \vskip -0.35 truein
  \caption{  (Color online) Left: Radium energy-level diagram. Right: isotope shifts for $^{225}$Ra ($I=1/2$) and $^{226}$Ra ($I=0$) relative to the isotopic average; panels (a). (b) and  (c) show the specific levels of interest. Figure from~\textcite{rf:Parker2015,PhysRevC.94.025501}.
}
\label{fg:RadiumAtomLevels}
\end{figure*}


The EDM measurement was based on a 100 second  cycle consisting of 60 seconds to cool, trap and transfer the atoms into the ODT and two approximately 20 second free precession periods to extract the EDM signal. 
The Ramsey separated-oscillatory-field measurement consisted of state preparation with a circularly polarized laser beam (483 nm) followed by a nuclear-spin precession period of $(20+\delta)$ seconds and measurement of atoms of the opposite polarization detected by absorption of the laser light imaged onto a CCD camera. By varying $\delta$, the change in accumulated phase over the free-precession period, shown in FIG.~\ref{fg:RadiumEDMData}, was converted to an EDM induced frequency shift.

A number of systematic effects were considered, the most important of which were Stark-shift related $E^2$  effects, correlations with drifts in the magnetic field between subsequent  electric field flips, correlations with the ODT laser power, and Stark interference.
\begin{figure}
%\vskip-1.5 truein
%\hskip -0.5 truein
 \includegraphics[width=3.25truein,angle=0]{ra-2016-data}%{RadiumEDMData}
 %\vskip -1. truein
  \caption{ (Color online) Phase-shift data showing the fraction of spin-down $^{225}$Ra nuclei after the nuclear-spin precession cycle of $20+\delta$ seconds. Figure from~\textcite{rf:Parker2015,PhysRevC.94.025501}.
}
\label{fg:RadiumEDMData}
\end{figure}


The most recent  $^{225}$Ra EDM result is
\begin{equation}
d(^{225}{\rm Ra})=(4\pm 6\ {\rm (stat)}\pm 0.2\ {\rm(sys)})\times 10^{-24}\ {\rm \ecm}.
\end{equation}
This was  a 36-fold improvement over the first run and corresponds to an upper limit of $1.4\times 10^{-23}$ \ecm~ (95\%)

A disadvantage of the imaging technique for probing the spin-state population is that only a small fraction of the atoms absorb the probe light so that the contrast is low and the statistical error is dominated by photon counting statistics and not the number of $^{225}$Ra atoms. This can be mitigated by essentially counting atoms using the STIRAP technique~\cite{rf:STIRAP}
Estimated production rates at FRIB, based on fragmentation models for light targets, may provide significantly  more $^{225}$Ra for future experiments.


%\subsection*{$^{221/223}$Rn}
\bigskip 
\noindent{\bf $^{221/223}$Rn}

Radon isotopes present the possibility of using techniques developed for the $^{129}$Xe EDM measurements and exploiting potential enhancements  due to octupole collectivity in $^{221/223}$Rn, which have half-lives on the order of 20-30 minutes. A program to develop an on-line EDM experiment at TRIUMF has been underway, and the prospects of producing and harvesting significantly greater quantities of atoms from the beam dump  at FRIB is very promising~\cite{Pen201462}. 
The on-line EDM measurement consists of the following elements:
1.)  on-line collection and transfer of radon isotopes, 
2.) optical pumping polarization by spin exchange (SEOP),
3.) polarized gas transport  to the EDM cell and gas recovery,
4.) EDM cells and high voltage,
5.) magnetic field, magnetic shielding and monitoring,
6.) spin-precession monitoring.

%\begin{figure}[htbp]
%\captionsetup{width=0.85\textwidth}
%\begin{center}
%\vspace{.2in}
%\vskip -0.5 truein
%\includegraphics[height=4.5 truein]{OverviewSchematic.pdf}
%\vskip -0.75 truein
%\caption{\footnotesize Schematic layout of the RadonEDM experiment.
%, which is similar to the HeXeEDM layout and a possible layout at FRIB.
% \label{fg:RadonEDMOverview}}
%\end{center}
%\vskip -0.25 truein
%\end{figure}


For collection and transfer, a 40 kV rare-gas isotope beam was incident on a tantalum foil~\cite{rf:Warner} for typically two half-lives and then transferred to a LN$_2$-cooled cold-finger, from which the gas was then released by rapidly warming the cold finger and pushed into the SEOP cell by a piston of N$_2$ gas. The N$_2$ gas also served as a buffer gas for noble gas polarization and as an insulating gas for the high voltage applied to the EDM cells.  Nearly 100\% collection efficiency was demonstrated~\cite{rf:NussWarren,rf:Warner}. SEOP produced $^{209}$Rn and $^{221}$Rn polarization of  $\approx 10\%$~\cite{rf:Tardiff2008,rf:Kitano1988} and spin relaxation times ($T_1\approx 15\ s$)~\cite{rf:Tardiff2008} were measured.
%\begin{figure}[htbp]
%\begin{center}
%\vspace{.2in}
%\vskip -1 truein
%\includegraphics[height=2. truein]{LaserSchematic.pdf}
%\includegraphics[height=2 truein]{LaserTuning.pdf}
%\includegraphics[height=2 truein]{LinewidthvsPower.pdf}
%\vskip -1 truein
%\caption{\footnotesize SEOP laser. The nLIGHT C1-795-100 bar  is cooled with deionized water from a stand-alone chiller. The laser light directly generated by the diode bar has a linewidth of 2-3 nm.  Left: photo, right: laser spectral profile. At 3 atmosphere total pressure (N$_2$ the Rb absorption line is Lorenzian with FWHM 0.12 nm centered at 794.8 nm ).\label{fg:Laser}}
%\end{center}
%\end{figure}
%\medskip
%\noindent
%3.) Once polarized, the gas mixture including Rn isotopes is transported by expansion from the polarizer cell to the EDM measurement cell. A customized glass  valve (FIG.~\ref{fg:CellValvePhotos}), which minimizes gas outside the EDM cell and minimizes spin relaxation,  is incorporated into the EDM cell. Like all valves that see polarized gas or are located within the magnetic shields, this valve will be air-actuated with no moving metal parts. 
%One concern is that turbulence of the expanding gas will lead to loss of polarization due to cooling with magnetic-field gradients. FEM calculation of the flow patterns for the expanding gases into our cell are shown in Fig~\ref{fg:GasExpansion} and are being incorporated into a full simulation. Any anticipated problem can be mitigated by minimizing field gradients and adjusting volume ratios. 
Each EDM measurement is limited by the spin-coherence time to about one minute, whereas the half-life of the Rn isotopes of interest is about 30 min. The Rn/N$_2$ mixture can be recycled into the polarizer cell by a circulating system.
%\begin{figure}[htbp]
%\begin{center}
%\vspace{.2in}
%\vskip -1 truein
%\includegraphics[height=2. truein]{GasExpansion.pdf}
%\vskip -1 truein
%\caption{\footnotesize Calculation of gas-flow into EDM cell from optical pumping cell.).\label{fg:GasExpansion}}
%\end{center}
%\end{figure}
%\medskip
%\noindent
%4.) The EDM cell must sustain the high voltage and provide suitable surfaces and minimal magnetic/paramagnetic contamination for the longest possible polarization-relaxation times. We have measured leakage currents as a function of temperature for several materials and found that fused silica is the best candidate for Rn; however for $^{129}$Xe and especially $^3$He, alumino-silicate glass, specifically GE180, has been the glass of choice for several years~\cite{rf:Chuppnpdg,rf:BabcockGE180}. 
%For the $^{129}$Xe measurement we used Mo electrodes epoxied to a GE180 glass cylinder~\cite{rf:Rosenberry}. Electrodes made of glass coated with a conductive layer of thin-film InO$_3$-SnO$_2$ (ITO), are commonly used~\cite{rf:Griffiths,rf:Hunter}. For the HeXeEDM experiment, we have developed silicon electrodes bonded to glass, to make cells similar to those originally developed for $^3$He-spin-filter neutron polarizers because of the neutron transmission properties of silicon.
%At the University of Michigan Lurie Nanofabrication Facility, we have developed EDM cells consisting of silicon wafer electrodes, sealed to reblown GE180 glass tubing via a silicate-assisted hydroxide catalysis bonding~\cite{rf:elliffe2005hydroxide}
% (see fig~\ref{fg:CellValvePhotos}). 
%The glass edges are hand-polished to 2500 grit (0.1$\mu$m particle size); the finish is specular with no visible defects. Prior to bonding, the polished glass and silicon are cleaned in a ``piranha'' solution of 3 parts 30\% hydrogen peroxide and 7 parts sulfuric acid. The bonding solution (1 part aqueous sodium silicate and 6 parts deionized water) is prepared in a clean room or fume hood, and applied to glass in the amount of approximately 1$\mu\text{l}/\text{cm}^2$. A silicon wafer is pressed onto the glass until the joint visibly whets, and the cell is cured for 3 days before bonding the second wafer; full strength is reached in a matter of weeks. Leaks are detected on a vacuum station and sealed with a low-outgassing or optical epoxy.
 %The bonding process is a combination of anodic bonding using sodium silicate and a vacuum-epoxy filler for any voids or leaks. 
A  measurement cycle will be initiated by a $\pi/2$  pulse after which the nuclear spins  freely precess in the $x$-$y$ plane ($\vec E$ and $\vec B$ are along $\hat z$). The radon-isotope free precession can be monitored by the gamma-ray anisotropy technique or by two-photon magnetometry~\cite{ref:xenon_twophoton}. %These approaches are described in more detail in the context of EDM sensitivity in the following section. 

Estimates of the $^{221/223}$Rn  production rates are based on measured rates at ISOLDE and TRIUMF. At  ISOLDE with a 1.6 $\mu$A 1.4 GeV proton beam incident on a thick uranium-carbide target, $1.4\times 10^7$ $^{220}$Rn-s$^{-1}$ were delivered to the low-energy end of the REX-ISOLDE accelerator~\cite{rf:NaturePaper}. With a 10 $\mu$A proton beam, $10^{8}$ $^{221}$Rn-s$^{-1}$ and $2\times 10^{7}$ $^{223}$Rn-s$^{-1}$ are expected. This would provide about $3\times 10^{10}$ nuclei for each 1-hour collection cycle. Estimated production rates at FRIB, based on fragmentation models on light targets {\it i.e.} the water in the FRIB beam dump, are up to 100$\times$ greater; however developing schemes for extraction of a large fraction of this remains a challenge. %The current FRIB beam-dump design has piping that will allow filter circuits to be added to extract the noble gases, which will be the most easily and efficiently harvested species~\cite{Pen201462}. 
%We are part of the FRIB ``Applications/Harvesting'' working group, which was established to study and make progress on isotope harvesting for applications including rare isotopes of medical interest and the RadonEDM and $^{225}$Ra-EDM experiments.





%\noindent{\bf Gamma Anisotropy}
%\label{sec:GammaAnisotropy}
%Our first RadonEDM measurements will use a set of HPGe detectors, for example elements of the GRIFFIN array at ISAC. The gamma-ray emission probability following $\beta$-decay of a nucleus with $J>1/2$ is proportional to a sum of $l$-even  $P_l(\cos\theta)$, where $\theta$ is the angle between the gamma-ray momentum and the nuclear spin axis, thus for $l=2$ the rate in each detector for a specific gamma-ray will be modulated at $2\omega$ ($\omega$ is given in Eqn.~\ref{eq:hbaromega}). A full GEANT simulation for an array of detectors {\it outside} the inner magnetic shield, which would significantly attenuate low-energy gammas, was undertaken for a Masters Thesis by Evan Rand of Guelph University~\cite{rf:EvanRandThesis} (Fig~\ref{fg:GammaAnisotropy}). The expected EDM precision based on $E=$10 kV/cm and count rates limited by the HPGe detectors, is $3\times 10^{-25}$ \ecm~ for each one-day measurment. Field gradients due to  magnetic materials in the detectors are expected to have negligible effects given the relatively short $T_2^*$ of the Rn.



Radon spin precession can be measured by gamma-ray anisotropy or by direct optical detection. The gamma-anisotropy is a $P_2(\cos\theta)$ distribution of photons emitted after a polarized nucleus decays to the excited states of the daughter. For a precessing-polarized sample, the detection rate for photons in a detector at a specific azimuthal position is modulated at twice the precession frequency ($2\omega$). The statistical power of the gamma-ray anisotropy technique for the Radon-EDM measurement is limited by the intrinsic photo-peak count rate limit for typical high-purity germanium gamma-ray detectors.  The projection of the nuclear spin is monitored by detecting the transmission or fluorescence of circularly polarized light or by the optical rotation of linearly polarized light. For radon, the single and two-photon transitions correspond to wavelengths of 178 nm and 257 nm respectively, and the two-photon magnetometry techniques discussed in Sec.~\ref{sec:Magnetometry} can be applied. % (see Fig~\ref{fg:Radon2Photon}). 
In the transmission/fluorescence case, photons and angular momentum are absorbed by the atoms and the measurement is destructive in the sense that the initial state of an atom is changed and thus affects the spin-coherence time $\tau$. Therefore the intensity of the light is adjusted to optimize the measurement. It is also possible to use the Ramsey separated oscillatory field technique to allow the spins to ``precess in the dark.'' 
%, {\it i.e.} to maximize $\tau\sqrt {N_\gamma}$.
With a fluorescence collection efficiency of 10\%,  a photon-statistics-limited EDM sensitivity of $3\times 10^{-26}$~\ecm~ could be achieved in one day assuming a 50\% duty cycle, $E=10$ kV/cm and $\tau=15$ sec.  
%(Recall that each shot of Rn can be recycled through the polarizer for up to about two half-lives, i.e. 1 hour.)  
With anticipated FRIB production rates, a sensitivity of $3\times 10^{-28}$ \ecm~ in 100~days is a reasonable running-time scenario for an  off-line experiment using isotopes harvested from the beam dump~\cite{Pen201462}. 
%An experiment can be set up to probe the atoms at the start and end of a free-precession cycle or to continuously probe with optimal laser power during free-precession, so that for the TRIUMF measurement, 10$^{10}$ photons scatter from  $^{221}$Rn and each 
The octupole enhancement also provides for different isotopes of the same atomic species as comagnetometers, {\it e.g.} $^{211}$Rn and $^{209}$Rn.
%We should be able to produce several orders of magnitude more laser power at 257 nm. Sufficient power at 178 nm may be possible with significant development effort. 
%The projected statistical sensitivities for TRIUMF and FRIB are summarized in table~\ref{tb:StatisticalSummary}.





%\subsection{Summary of Systematic Effects for Atomic EDM Experiments}

%\red{In Table~\ref{tb:AtomicSystematics}, we compile the systematic effects of completed atomic and molecular EDM experiments in order to suggest the challenges of future efforts to improve or launch new experiments. }

%\begin{table*}
%\begin{centering}
%\begin{tabular}{|l|c|c|c|c|}

%\hline\hline
%\multicolumn{5}{c|}{Atomic/Molecular Beam}\\
%\hline
%Effect & Cs & Th &  & \\
%Reference &  & & & \\
%\hline
%Motional $E$-filed &  & & & \\
 

%\end{tabular}
%\caption{\label{tb:AtomicSystematics} Systematic effects in completed atomic and molecular EDM experiments.}
%\end{centering}
%\end{table*}


\subsection{Storage ring EDMs}
\label{sec:StorageRingEDMs}

Over the last two decades, EDM measurement techniques using storage rings have been developed using inspiration from the muon ``g-2'' experiment at Brookhaven National Laboratory~\cite{Bennett:2006fi}. %Following sec. 11.11 of ~\textcite{rf:Jackson1998pp564565}, f
The concept, introduced by~\cite{Farley04}, is based on the evolution of the momentum and spin of a charged particle  in the presence of magnetic and electric fields. 


For a relativistic particle of charge $q$ and rest mass $m$, the equation of motion in the lab frame is
\begin{equation}
\frac{d\vec{p}}{dt} = q \left ( \vec{v} \times \vec{B} + \vec{E} \right).
\end{equation}
where $\vec{B}$ and $\vec{E}$ are the static and uniform magnetic and electric fields in the lab frame, $\gamma=(1-v^2/c^2)^{-1/2}$ is the Lorentz factor,   % given by
%begin{equation}
%\gamma = \left ( 1- \frac{\vec{v} \cdot \vec{v}}{c^2} \right )^{-\frac{1}{2}} = \frac{dt}{d\tau},  
%\end{equation}
$t=\gamma\tau$ is the time measured in the lab frame, and $\tau$ is the proper time in the particle rest frame. %, $c$ is the speed of light, , 
The acceleration in the lab frame is \cite{rf:Jackson1998pp564565}
\begin{eqnarray}
\vec{a} = \frac{d\vec{v}}{dt} & = & \frac{ q}{m \gamma} \left [ \vec{v} \times \vec{B} + \vec{E}   - \left (\frac{\vec{v} \cdot \vec{E}}{c^2} \right ) \vec{v}\right ] \nonumber\\
& = &  \vec{\omega}_v \times  \vec{v} +\frac{ q}{m \gamma}\frac{1}{\gamma^2-1}\left ( \frac{\vec{v} \cdot \vec{E}}{c^2} \right ) \vec{v},\nonumber\\
%& = & \frac{ q}{m \gamma} \left [ \vec{\omega}_v \times  \vec{v} + \frac{1}{\gamma^2-1}\left ( \frac{\vec{v} \cdot \vec{E}}{c^2} \right ) \vec{v}\right ]\nonumber\\
\end{eqnarray}
where the rotation of the velocity in the lab frame is 
\begin{equation}
\vec{\omega}_v = \frac{q}{m\gamma}\left [ \frac{\gamma^2}{\gamma^2-1} \left ( \frac{\vec{v}\times\vec{E}}{c^2} \right ) - \vec{B}\right ].
%\vec{\omega}_v = \frac{q}{m}\left [ \frac{\gamma^2}{\gamma^2-1} \left ( \frac{\vec{v}\times\vec{E}}{c^2} \right ) - \vec{B}\right ].
\end{equation} 

The torque on the particle's spin $\vec{s}$ is 
\begin{equation}
\frac{d\vec{s}}{d\tau} = \gamma \left[ \frac{d\vec{s}}{dt} \right ] = \vec{\mu} \times \vec{B}' + \vec{d}\times \vec{E}' ,
\end{equation}
where the magnetic moment $\vec \mu$ and the EDM $\vec d$ are
 \be
 \vec{\mu} = g (q/2m) \vec{s}  \hskip 0.5 truein  \vec{d} = \eta (q/2mc) \vec{s},
 \ee
and  fields coupling to $\vec\mu$ and $\vec d$ in the particle rest frame are 
\begin{eqnarray}
\vec{B}' & = & \gamma \left ( \vec{B} - \frac{\vec{v} \times \vec{E}}{c^2} \right ) - \frac{\gamma^2}{\gamma+1}\frac{\vec{v}\cdot (\vec{v} \cdot \vec{B})}{c^2} \nonumber \\
\vec{E}' & = & \gamma \  \  \left (\vec{E} + \vec{v} \times \vec{B} \right ) - \frac{\gamma^2}{\gamma+1} \frac{\vec{v}\cdot (\vec{v} \cdot \vec{E})}{c^2} .\nonumber \\
\end{eqnarray}
For $v\approx c$,  $\vec{v} \times \hat{B} \approx 3000\ \mathrm{kV/cm/T}$. For the Brookhaven muon $g-2$ experiment, $B=1.45$ T, and the motional electric field is about two orders of magnitude larger than a typical laboratory electric field. In the lab frame, accounting for the Thomas precession \cite{doi:10.1080/14786440108564170}, the equation of motion for the spin is
\begin{equation}
	 \frac{d\vec{s}}{dt}  =  \left [ \frac{d\vec{s}}{dt} \right ]_\mathrm{rest} + \frac{\gamma^2}{\gamma+1}   \frac{\vec{s} \times(\vec{v} \times \vec{a})}{c^2}  
\end{equation}
Combining the preceding equations, the evolution of the spin in the lab frame is
%\begin{eqnarray}
%\frac{d\vec{s}}{dt}   &=&   \vec{\omega}_s\times \vec{s} \nonumber\\
%&=&	 \big (\frac{aq}{m}\big )\ \vec s\times  \bigl [  \left (\frac{1}{a\gamma}+ 1 \right ) \vec{B} 
%	- \left (1 + \frac{1}{a(\gamma + 1)}\right ) \frac{\vec{v} \times \vec{E}}{c^2} \nonumber\\
%	& & \hskip 2 truein - \frac{\gamma}{\gamma+1} \frac{\vec{v}(\vec{v}\cdot\vec{B})}{c^2}  \bigr ]  \nonumber \\ 
%	& &\nonumber \\
%	& +&\big ( \frac{\eta q}{2mc} \big )  \ \vec{s}\times\left [ \vec{E} + \vec{v} \times \vec{B} -\frac{\gamma}{\gamma+1} \frac{\vec{v}(\vec{v}\cdot\vec{E})}{c^2}  \right ]  \nonumber\\
%	 \nonumber \\ 
%\end{eqnarray}
\begin{eqnarray}
\frac{d\vec{s}}{dt}  &=&   \vec{\omega}_s\times \vec{s} \nonumber\\
         &=&   \big (\frac{q}{m\gamma}\big )\ \ \vec s\ \times\  \left [  \vec{B}  -   \frac{\gamma}{(\gamma + 1)} \frac{\vec{v} \times \vec{E}}{c^2} \right ]\nonumber\\
	 &+&\  \big (\frac{aq}{m}\big )\ \ \vec s\ \times\  \left [  \vec{B} -  \frac{\vec{v} \times \vec{E}}{c^2}	 - \frac{\gamma}{\gamma+1} \frac{\vec{v}(\vec{v}\cdot\vec{B})}{c^2}  \right ]  \nonumber \\ 
	 &+&\big ( \frac{\eta q}{2mc} \big )  \ \vec{s}\  \times\ \left [ \vec{E} + \vec{v} \times \vec{B} -\frac{\gamma}{\gamma+1} \frac{\vec{v}(\vec{v}\cdot\vec{E})}{c^2}  \right ] . \nonumber\\
	 \nonumber \\ 
\end{eqnarray}
%\begin{eqnarray}
%\frac {d \vec {s}} {dt}  &=&\frac {q}{m\gamma}  [(\vec {s}\times  \vec B)+a\gamma(\vec {s}\times  \vec B)\nonumber\\
%&-& \frac{ \vec s\times (\vec {v} \times \vec {E})}{c^2} (a\gamma +\frac {\gamma} {\gamma + 1} ) 
%-\frac{a\gamma^2}{\gamma+1}\frac{(\vec v\cdot \vec B)(\vec s\times \vec v)}{c^2}]\nonumber\\
%&+& \frac {\eta q} {2mc}\vec s\times  [  \vec {E} +  \vec v \times \vec {B}-\frac{\gamma} { \gamma+1}\frac{\vec v(\vec v\cdot \vec E)}{c^2} ].
%\end{eqnarray}
The first two terms are the non-covariant form of the Bargmann-Michel-Telegedi or BMT equation~\cite{PhysRevLett.2.435} and give the torque on the spin in the lab frame due to the magnetic moment, where $a$ is the magnetic moment anomaly. The third term in square brackets, proportional to $\eta$, is due to the EDM. 

To illustrate the principle of the EDM measurement, consider the rotation of the spin with respect to the velocity in the lab frame $\vec \omega_s-\vec\omega_v$. This can be separated into two terms: $\vec \omega_a$ and $\vec\omega_d$, where
\begin{eqnarray}
	\vec{\omega}_a   = & -&\ \frac{a q}{m} \left [ \vec{B} 
	+\!\! \left ( \frac{1}{a(\gamma^2 - 1) }-1\right )\! \frac{\vec{v} \times \vec{E}}{c^2}\! -\! \frac{\gamma}{\gamma+1}  \frac{\vec{v}(\vec{v}\cdot\vec{B})}{c^2} \right ] \nonumber\\
\vec{\omega}_d	 = & -&\! \frac{d}{\hbar J} \left [\vec{v} \times \vec{B} + \vec{E} -\frac{\gamma}{\gamma+1} \frac{\vec{v}\cdot\vec{E}}{c^2}  \vec{v} \right ]  
= -\frac{d \vec{E}'}{\hbar J \gamma}.
	\end{eqnarray}
%For a relativistic particle of charge $q$, the equation of motion is 
%\begin{equation}
% \frac{d\vec p}{dt}=q(\vec v\times\vec B+\vec E)
% \end{equation}
%and the acceleration is 
%\begin{equation}
%\vec a=\frac{d\vec v}{dt}=\frac{q}{m\gamma}\Big (\vec v\times\vec B+\vec E-\frac{\vec v(\vec v\cdot \vec E)}{c^2}\Big ).
%\label{eq:acc}
%\end{equation}
%Here $\vec B$ and $\vec E$ are the fields in the lab frame.
%The torque on the spin $\vec s$  in the particle rest frame due to the magnetic moment $\vec {\mu} = g(q/2m) \vec {s}$ and an EDM $\vec {d} = \eta (q/2mc) \vec {s}$  is
%with $\eta$ is analagous to $g$. For a relativistic particle, 
%\begin{equation}
%\frac {d\vec{s}} {d\tau }  =   \vec {\mu} \times \vec {B^\prime} + \vec {d} \times \vec {E^\prime}, 
%\end{equation} 
%where $\tau=t\over \gamma$ is the proper time, and
%\begin{eqnarray}
%\vec B^\prime &=& \gamma(\vec B-\frac{\vec v\times\vec E}{c^2})-\frac{\gamma^2}{\gamma+1}\frac{(\vec v\cdot\vec B)\vec v}{c^2},\nonumber\\
%\vec E^\prime &=& \gamma(\vec E+\vec v\times\vec B)-\frac{\gamma^2}{\gamma+1}\frac{(\vec v\cdot\vec E)\vec v}{c^2}.
%\end{eqnarray}
%Here $\vec \beta=\frac{\vec v}{c}$ and $\gamma=(1-\beta^2)^{-1}$ are the familiar reletavistic factors.
%For magnetic storage rings, $\vec v \times \hat{B}\approx$ 3 MV/cm/T, which is about 100-times larger than a typical laboratory electric field $\vec{E}$ for typical magnetic fields ({\it e.g.} $B%$=1.45 T for the Brookhaven $g-2$ experiment.
%In the lab frame, accounting for the Thomas precession~\cite{rf:Thomas1927}, the equation of motion for the spin is
%\begin{equation}
%\frac{d\vec s}{dt}=\frac {d\vec s} {d\tau } +\frac{\gamma^2}{\gamma+1}\frac{\vec s\times(\vec v\times\vec a)}{c^2}.
%\label{eq:Thomas}
%\end{equation}
%Combining Eqs.~\ref{eq:acc} and~\ref{eq:Thomas}, the spin precession rate in the lab frame is 
%\begin{eqnarray}
%\frac {d \vec {s}} {dt}  &=&\frac {q}{m\gamma}  [(\vec {s}\times  \vec B)+a\gamma(\vec {s}\times  \vec B)\nonumber\\
%&-& \frac{ \vec s\times (\vec {v} \times \vec {E})}{c^2} (a +\frac {1} {\gamma + 1} ) 
%-\frac{a\gamma}{\gamma+1}\frac{(\vec v\cdot \vec B)(\vec s\times \vec v)}{c^2}]\nonumber\\
%&+&\frac {\eta} {2|s|} \frac {q} {mc}  [  \vec {E} +  \vec v \times \vec {B}-\frac{\vec v(\vec v\cdot \vec E)}{c^2} ].
% -  \frac{\gamma} {1 + \gamma} \frac  {\vec {\beta} \cdot \vec {E}} {c}   ].
%  - \vec {s}\times  \vec {\beta}  (\frac {g} {2} -1 )    \frac {\gamma} {\gamma + 1} ( \vec {\beta} \cdot \vec {B} ) 
%\end{eqnarray}
%The first term in square brackets, due to the magnetic moment is the non-covariant Bargmann-Michel-Telegdi or BMT equation~\cite{BMT59} evaluated in the lab frame. The second term, %proportional to $\eta$, is due to the EDM. 
%The difference of the particle's  spin and velocity rotation rates is the anamolous precession rate, which can be separated into two pieces:
%\begin{eqnarray}
%\vec\omega_a=&-&\frac{aq}{m}[\vec B+(\frac{1}{a(\gamma^2-1)}-1)\frac{\vec v\times \vec E}{c^2}\nonumber\\
%&-&\frac{a\gamma}{\gamma+1}\frac{(\vec v\cdot \vec B)(\vec s\times \vec v)}{c^2}],\\
%\vec\omega_d=&-&\frac{\eta q}{2mc}(\vec v\times\vec B+\vec E-\frac{\vec v(\vec v\cdot \vec E)}{c^2} ]),
%\vec\omega_d=&-&\frac{2d}{\hbar j}(\vec v\times\vec B+\vec E-\frac{\vec v(\vec v\cdot \vec E)}{c^2} ]),
%\end{eqnarray}
%where $a$  is the  magnetic moment anomaly. For leptons 
%\begin{equation}
%a=\frac{g-2}{2},
%\end{equation}
 %and for bare nuclei, 
 %\begin{equation}
 %a = (\mu/\mu_N)/(2 j Z/A) - 1=\frac{\kappa}{Z},
 %\end{equation}
 % where $\mu/\mu_N$ is the nuclear magnetic moment in units of the nuclear magneton, $j$ is the nuclear spin, and $Z/A$ is the nuclear charge to mass ratio.  For bare nuclei, the customary %anomalous magnetic moment $\kappa$ is related to the magnetic moment anomaly by $a = \kappa/Z$~\cite{Khriplovich98}.
The magnetic-moment anomaly % $\vec{s} = \hbar \vec{J}$, and $\vec J$ is the spin of the particle. 
for leptons is
\begin{equation}
a = \frac{g-2}{2} = \left ( \frac{\mu}{\mu_B} \right ) \left ( \frac{e}{q} \right ) \left ( \frac{m}{m_e} \right )- 1
\end{equation}
and for bare nuclei is~\cite{Khriplovich98}
\begin{equation}
a = \frac{g-2}{2} = \frac{1}{2J}\left ( \frac{\mu}{\mu_N} \right ) \left ( \frac{A}{Z} \right ) - 1= \frac{\kappa}{Z},
\end{equation}
where $\mu_{N(B)}$ is the nuclear (Bohr) magneton, $Z=q/e$, $e$ is the elementary charge, $A=m/m_p$, $m_e$ is the mass of the electron, $m_{p}$ is the mass of the proton, and $\kappa$ is the customary anomalous magnetic moment.
%Measurement of  $\vec \omega_a$ is used to determine the magnetic-moment anomaly of the particle, which is also of significant interest, for example the muon $g-2$ experimental result currently has a greater than 3-$\sigma$ discrepancy with the SM calculation~\cite{Bennett:2006fi}. 
%Note that the EDM precession $\vec\omega_d$ is perpendicular to $\vec\omega_a$. 
%, and in the presence of appropriately chosen static laboratory electric and magnetic fields, the spin polarization of the particle will slowly develop a transverse component proportional to $\eta$. The torque that pulls the spin out of the orbital plane is generated by some combination of static and motional electric fields felt by the particle in its rest frame. 
%After a suitably long circulation time, the transverse component of the particle's spin polarization vector is determined by measuring the polarization-dependent asymmetry of either beta decay or elastic scattering.

The EDM experiments envision the charged particle of a carefully chosen energy with initially only longitudinal spin polarization, {\it i.e.}  $\vec s$ parallel to $\vec p$, injected into a storage ring.
In the presence of appropriately chosen static laboratory electric and magnetic fields, the spin polarization of the particle will slowly develop a spin component linearly proportional to $\eta$, which is  transverse to its velocity,  {\it i.e.} pointing out of the storage ring plane. 
%After a suitably long circulation time, t
The direction of the particle's spin polarization vector can be determined, for example  in the case of the muon, by measuring the polarization-dependent decay asymmetry or, for stable nuclei, by spin-dependent elastic scattering.
The statistical uncertainty follows Eqn.~\ref{eq:EDMSigmaEquation1} for a total of $T/\tau$  EDM measurements of duration $\tau$ by substituting  for $E$, the electric field in the particle rest frame  $E^\prime/\gamma$: 
\be
\sigma_d=\frac{\gamma\hbar J}{2E^\prime P A}\frac{1}{\sqrt{N T\tau}},
\ee
where  the particle polarization is  $P$ and the experimental analyzing power is $A$, which are both $\le 1$, and $N$ is the number of particles detected for each measurement.% JAIDEEP - CHECK FACTORS OF 2 and K.
	
The key insight of storage ring EDM  measurements  is choosing the electric and magnetic fields as well as the particle's momentum such that the $\omega_a$ is  suppressed~\cite{Farley04}.  This 
%\cite{Farley04}
 is accomplished by first making $\hat E$, $\hat B$ and the velocity all mutually orthogonal and then either choosing a radial electric field that cancels $\omega_a$ ($E_r=aBv\gamma^2/[1-av^2\gamma^2/c^2]$) or by setting  $\vec B\approx 0$ and storing particles with momentum $p\approx mc/\sqrt{a}$, {\it i.e.} purely electric confinement. A critical challenge is to minimize undesired radial magnetic fields due to misalignments and fringe fields, which would result in the transverse polarization due to the normal ``g-2'' spin precession and would mimic the EDM signal. Injection of simultaneous counter-propagating beams into the storage ring has been proposed to control these effects~\cite{Anastassopoulos:2015ura}. Unlike the EDM signal, the effect of the radial field depends on the  beam propagation direction, thus providing a way to disentangle the two sources of transverse polarization.
Other false effects, for example due to the non-orthogonality of the electric and magnetic fields, would be cancelled by summing over detectors separated by $180^\circ$ in azimuth around the ring.


%The BMT equations
% \begin{eqnarray}
%\vec{B}_a & = & \vec{B} 
%- \vec{\beta} \times \vec{E} \left [1  - \frac{1}{a(\gamma^2 -1)} \right ]  
%- \vec{\beta} \frac{\gamma(\vec{\beta} \cdot \vec{B})}{\gamma + 1}\nonumber  \\
%\vec{E}_d & = & \vec{E} + \vec{\beta} \times \vec{B} - \vec{\beta} \frac{\gamma(\vec{\beta} \cdot \vec{E})}{\gamma + 1}
%\end{eqnarray}
%are the BMT equations \cite{BMT59}, 
%$\vec{B}$ ($\vec{E}$) is laboratory magnetic (electric) field, $\vec{\beta}$ is the velocity of the particle relative to $c$, $\gamma = 1/(1-\beta^2)^2$ is the Lorentz factor, 
%$e$ is the elementary charge, $m$ is the mass of the particle, $a =  g/2-1 = (A \mu)/(Z 2 J) - 1$ is the anomalous magnetic moment of the particle \cite{Khriplovich98}, $g$ is the $g$-factor,
%$A$ is the atomic mass, $Z$ is the nuclear charge, $J$ is the nuclear spin, $d$ is the EDM of the particle, and $\hbar = c = 1$.
%This requirement implies making $B_a$ as close to zero as possible.
%In all cases, it is useful to apply electric and magnetic fields perpendicular to the velocity of the particle such that $\vec{\beta}\cdot\vec{E} = \vec{\beta}\cdot\vec{B} = 0$.
%This results in:
%\begin{eqnarray}
%\vec{B}_a & = & \vec{B} - \vec{\beta} \times \vec{E} \left [1  - \frac{1}{a(\gamma^2 -1)} \right ]  \\
%\vec{E}_d & = & \vec{E} + \vec{\beta} \times \vec{B}
%\end{eqnarray}
%The primary challenge of this technique is to minimize undesired field components that which would shorten spin coherence times and lead to false systematic effects.
%For example, a slight radial magnetic field would also produce a transverse spin polarization mimicking the EDM signal.
%The solution proposed to control this effect is to inject counter-propagating beams into the storage ring simultaneously~\cite{Anastassopoulos:2015ura}.

%While generic proposals have been made to search for EDMs in charged ions whose nuclei undergo beta decay,  there are significant efforts to develop a storage ring EDM approach only for muons, protons, deuterons, and He-3 nuclei.
Generic proposals have been made to search for EDMs with unstable charged ions using the beta-decay asymmetry for polarimetry similar to the muon $g-2$ concept~\cite{Khriplovich98,Khriplovich00HI,Khriplovich00NPA} and with
highly charged ions \cite{HCI-PR}. Current efforts to develop storage ring EDM experiments for muons, protons, deuterons, and $^{3}\mathrm{He}$  nuclei are summarized in 
Table~\ref{tab:srpar}.%The current muon EDM limit is derived from ancillary measurements of the muon decay asymmetry taken during a precision measurement of the muon anomalous magnetic moment \cite{muonEDM09}. Since the apparatus was designed to be maximally sensitive to the spin precession due to the muon anomalous magnetic moment,  the spin precession due to the muon EDM alternated directions causing it to be averaged nearly to zero. Because the averaging was not perfect, a muon EDM limit was derived.
 
 
\begin{table*}
\begin{tabular}{|c|c|c|c|c|c|c|c|c|c|c|}
\hline\hline
particle & $J$ & $a$ & $|\vec{p}|$ & $\gamma$ & $|\vec{B}|$ & $|\vec{E}|$ & $|\vec{E}'|/\gamma$ & $R$  & $\sigma^\mathrm{goal}_d$  & Ref. \\
(units) &  &  & $(\mathrm{GeV}/c)$ &  & $(\mathrm{T})$ & $(\mathrm{kV/cm})$ & $(\mathrm{kV/cm})$ & $(\mathrm{m})$  & (\!\ecm) &  \\
\hline
{$\mu^\pm$} & {$1/2$} & {$+0.00117$} & $3.094$ & $29.3$ & $1.45$ & $0$ & $4300$ & $7.11$ & $10^{-21}$ & E989 \\
 &  &  & $0.3$ & $3.0$ & $3.0$ & $0$ & $8500$ & $0.333$ & $10^{-21}$ & E34 \\
 &  &  & $0.5$ & $5.0$ & $0.25$ & $22$ & $760$ & $7$ & $10^{-24}$ & srEDM \\
 &  &  & $0.125$ & $1.57$ & $1$ & $6.7$ & $2300$ & $0.42$ & $10^{-24}$ & PSI \\ 
\hline
{$p^+$} & {$1/2$} & {$+1.79285$} & $0.7007$ & $1.248$ & $0$ & $80$ & $80$ & $52.3$ & $10^{-29}$ & srEDM \\
 &  &  & $0.7007$ & $1.248$ & $0$ & $140$ & $140$ & $30$ & $10^{-29}$ & JEDI \\
\hline
{$d^+$} & {$1$} & {$-0.14299$} & $1.0$ & $1.13$ & $0.5$ & $120$ & $580$ & $8.4$ & $10^{-29}$ & srEDM \\
 & &  &  $1.000$& $1.13$ & $0.135$ & $33$ & $160$ & $30$ &  $10^{-29}$ & JEDI \\
\hline
$^3\mathrm{He^{++}}$ & $1/2$ & $-4.18415$ & $1.211$ & $1.09$ & $0.042$ & $140$ & $89$ & $30$ &  $10^{-29}$ & JEDI \\
\hline\hline
\end{tabular}
\caption{\label{tab:srpar}Relevant parameters for proposed storage ring EDM searches.  The present muon EDM limit is ${1.8}\times 10^{-19}$\ecm~and the indirect limit on the proton EDM derived from the atomic EDM limit of $^{199}\mathrm{Hg}$ is
${2}\times 10^{-25}$\ecm. The magnetic moment anomaly is calculated using values for the unshielded magnetic moments of the particles from CODATA 2014 \cite{RevModPhys.88.035009}.
The sign convention for  positively charged particles is such that the magnetic field is vertical and the particles are circulating clockwise.
References are 
E989: Muon $g\!-\!2$ experiment at Fermilab \cite{Gorringe201573};
E34: Muon $g\!-\!2$ experiment at JPARC \cite{Gorringe201573};
srEDM: Muon EDM at JPARC \cite{KANDA2014212},  ``All-Electric'' Proton EDM at Brookhaven \cite{Anastassopoulos:2015ura}, Deuteron EDM at JPARC \cite{Morse2011}; 
PSI: Compact Muon EDM  \cite{compactMU10};
JEDI: ``All-In-One'' Proton, Deuteron, and Helion EDM at COSY \cite{JEDI}.}
\end{table*}  
 
%\begin{table*}
%\begin{centering}
%\begin{tabular}{|c|c|c|c|c|c|c|c|c|c|c|c|c|}
%\hline\hline
 %&  &  &  &   &    &   &   &   &  & \multicolumn{2}{c|}{$\sigma_d$ (\ecm)}  & \\
%particle & J & Z & A& a & m/c$^2$ (GeV) & KE (GeV) & R (m) & B (T) & cB/E &  \multicolumn{2}{r|}{current\quad\quad projected}  & ref\\
%\hline									
%$\mu$ & 1/2 & $\pm$1.0 & 0.1 & 0.00117 & 0.106 & 0.400 & 17 & 1.45 &  & $9\times 10^{-20}$ & $1\times 10^{-21}$& $a$ \\
%\hline
%proton &1/2 & 1.0 & 1.0 & 1.79285 & 0.938 & 0.233 & 30 & 0 & 0 & $3\times 10^{-26 }$ ($n$) & $10^{-28}$& $b$\\
%\hline
%deuteron & 1 & 1.0 & 2.0 & -0.14299 & 1.876 & 0.250 & 30 & 0.16 & 	12 & - & $10^{-28}$ & $c$ \\
%\hline
%helion  & 1/2 & 2.0 & 3.0 & -4.18415 & 2.808 & 0.280 & 30 & 0.05 & 0.9 & -  & $10^{-28}$ &$d$\\
%\hline\hline
%\end{tabular}
%\caption{Proposed EDM storage-ring parameters and project sensitivity. Refernces are: $a$ \cite{Logashenko:2015xab}, $b$ \cite{Anastassopoulos:2015ura}, $c$ \cite{Bagdasarian14etal}, $d$ %\red{JEDI Reference.} \label{tb:StorageRings}}
%\end{centering}
%\end{table*}
  
\textcite{Farley04} presented the first storage ring proposal for a dedicated EDM measurement, which focused on a muon EDM. Their proposal suggested technically feasible values for $\vec{E}$ and $\beta=v/c$  to make $\vec{B}_a$ equal to zero. 
%The EDM-sensitive spin precession is driven mainly by the motional electric field.
%The primary method to control for systematic effects due to undesired field components is to inject counter-propagating beams into the storage ring simultaneously.
%Unlike the EDM signal, the sign of the systematic effect would depend on the direction of beam propagation providing a way to disentangle the two sources of transverse polarization.
The current muon EDM limit $|d_\mu|\le 1.8\times 10^{-19}$ \ecm~ is derived from ancillary measurements of the muon decay asymmetry taken during a precision measurement of the muon anomalous magnetic moment \cite{Bennett:2008dy}.
The sensitivity of this measurement was limited by the fact that the apparatus was designed to be maximally sensitive to $\omega_a$.
For dedicated muon EDM experiments, under development at JPARC \cite{KANDA2014212} and PSI \cite{compactMU10}, $\vec{E}$ and $\gamma$ are chosen to make  $\omega_a=0$. 
%In this case, the motional electric field, $\vec{\beta} \times \vec{B}$, is about $10^3$ larger than typical laboratory electric field $\vec{E}$.
The spin coherence time $\tau$ in this case is limited by the muon lifetime in the lab frame ($\gamma\times$2.2~$\mu$s).
An alternative muon EDM approach using lower energy muons and a smaller and more compact storage ring is being developed at  PSI.
A proposal for injecting muons into such a compact storage ring as well as an evaluation of the systematic effects due specifically to the lower muon energy is presented by \textcite{compactMU10}.


The two main differences between between an experiment designed for muons and one designed for light nuclei
are the need for more careful control of the beam properties to preserve the spin coherence and, of course, a different spin polarimetry scheme. 
A spread in the beam position and momentum smears the cancellation of the ``g-2'' spin precession which would, after many cycles, result in decoherence of the beam.
Since the muon spin coherence time is limited by the finite muon lifetime, this is not as critical for the muon EDM experiment.
%For the case of a proton EDM search, $\vec{B}$ is chosen to be zero and choosing $\beta=1/\sqrt{a+1}$ result in a vanishing $\vec{\beta}\times\vec{E}$ term. 
For the case of a proton EDM search,  choosing $\vec{B}=0$ and   $\beta=1/\sqrt{a+1}$  supresses the $\vec{\beta}\times\vec{E}$ term~\cite{Anastassopoulos:2015ura}. This requires effective magnetic shielding, such as that discussed in sec.~\ref{sec:MagneticShielding}. 
%The corresponding momentum, known as the ``magic momentum,'' is selected for the muon anomalous magnetic moment experiment to suppress sensitivity to the quadrupole electric fields required for weak focusing of the muons. 
The electric storage ring with bending radius $R=(m/e)/(E\sqrt(a(a+1))$  is generally only possible for particles with positive magnetic moment anomalies ($a > 0$).
With $E=10^6$ V/m, a  bending radius of $R\approx10$ m is required for protons.
%A significant challenge is designing an electrostatic storage ring which preserves long spin coherence times and minimizes radial magnetic fields.
%Magnetic-shielding challenges are addressed is Sec.~\ref{sec:MagneticShielding}.
Progress has been made in describing the challenging problem of orbital and spin dynamics inside electrostatic rings \cite{Mane08,Mane12,Mane14a,HS14,Mane15a,Metodiev15, Mane14b,Mane14c,Mane15b,Mane15c}, developing simulation code for electrostatic rings \cite{Talman15a,Talman15b}, and calculating the fringe fields for different plate geometries \cite{Metodiev14}.
 %Since the proton EDM experiment requires exactly zero magnetic field, even a tiny 10's of attoTesla radial magnetic field would limit the sensitivity to $10^{-29}\ e\cdot\mathrm{cm}$.
%Providing adequate magnetic shielding is a critical challenge for this experiment and efforts alongs these lines is presented in an earlier section.
%Such a small residual magnetic field is essentially immeasurable using presently available state-of-the-art magnetometer technologies.
To achieve sensitivity of 10$^{-29}$ \ecm,  impractically small residual magnetic fields would be required, thus two counter propagating beams within the same storage ring are envisioned, for which a vertical 
separation would develop in the presence of a radial magnetic field. 
After several cycles around the ring, this vertical separation would be large enough to measure using SQUID magnetometers as precision beam position monitors (BPMs).
 The development of an electric storage ring experiment dedicated to measurement of the proton EDM is being pursued by the Storage Ring EDM collaboration srEDM~\cite{Rathmann:2013rqa}.

%Efforts are underway to develop an electrostatic proton EDM storage ring at Brookhaven National Lab in Upton, NY, USA that would fit inside the tunnel of and along side the Alternating Gradient Synchrotron (AGS).

A magnetic storage ring could also be used to measure the $J=1$ deuteron EDM using a similar technique. The deuteron polarization would be analyzed by the asymmetry in elastic scattering from a carbon target~\cite{Brantjes12}.
The goal for the deuteron EDM experiment is to maintain the spin coherence for at least as long as the vacuum-limited ion storage time which is about $10^3$ seconds for a vacuum of $10^{-10}\ \mathrm{Torr}$, which has been demonstrated at COSY~\cite{COSYDSpinLifetimePhysRevLett.117.054801}. The theory of spin evolution for a $J=1$ particle  in electromagnetic fields has been developed by \textcite{Silenko15}. 

%The muon and deuteron EDM experiments,



The J\"{u}lich Electric Dipole moment Investigations (JEDI) collaboration in  Germany is undertaking precursor experiments while developing  long term plans  to measure the EDMs of the proton, deuteron, and $^{3}\mathrm{He}$ using an ``all-in-one'' electirc and magnetic storage ring~\cite{Rathmann:2013rqa}.
An intermediate step is  direct measurement of the proton and deuteron EDMs with lower statistical sensitivity using the presently available magnetostatic Cooler Synchrotron (COSY) storage ring with some modifications.
The main challenge is to introduce beam-line elements that prevent the spin precession due to the magnetic moment anomaly from washing out the  torque on the spin generated by the presence of an EDM. 
One suggestion is to synchronize the EDM torque to the magnetic moment anomaly spin precession \cite{Orlov06}, however the approach being developed for COSY by the JEDI collaboration is to partially ``freeze'' or lock the spin to the momentum using a beam element called a ``magic''  RF Wien filter \cite{Morse13}.
If the parameters of the Wien filter are carefully chosen,  one component of the particle's spin does not undergo the usual magnetic moment anomaly spin precession, which would allow the EDM torque to build up a transverse polarization.

Spin polarimetry is critical for both measuring the EDM signal as well as for diagnosing and improving the spin coherence time.
Significant progress has been made towards controlling systematics related to spin polarimetry for deuterons.
Results indicate that precision polarimetry for both deuterons and protons is feasible at the ppm level, which is required for a $10^{-29}\ e \cdot \mathrm{cm}$ EDM sensitivity.
Preliminary efforts to measure and improve spin coherence times of deuterons using the COSY storage ring have also been reported \cite{Benati12etal,Benati13etal,Bagdasarian14etal}.
High precision ($10^{-10}$) control and monitoring of the spin motion of deuterons at COSY has also been demonstrated \cite{Eversmann15}. Plans are also underway to develop an ion source and polarimetry for $^{3}\mathrm{He}$ by the JEDI collaboration.
Although significant effort is still required to perform storage ring EDM experiments, they would provide the most direct and clean measurements of the EDM of light ions and muons and would improve the limits on their EDMs by several orders of magnitude.
%The statistical sensitivity of these types of searches is given by usual equation  where the efficiency is given by 
%\begin{equation}
%\epsilon = f A^2 P^2
%\end{equation}
%$f$ is the particle detection efficiency, $A$ is the analyzing power of the detectors, $P$ is the polarization of the particles, and $E=E_d$ is the electric field in the rest frame of the particle.




%relativistic proton beams with a magic momentum of 0.7 GeV/c


%\subsection{BAU; baryogenesis, leptogenesis - for dummies}

\section{Interpretations of current and prospective experiments}
\label{sec:Interpretation}
\label{sec:GlobalAnalysis}

In general there are many possible contributions to the EDM of any system accessible to experiment, for example the neutron EDM may arise due to a number of sources including short range, {\it e.g.} quark EDMs, and long range pion-nucleon couplings characterized by $\gpiz$ and $\gpio$. One approach to putting EDM results in context has been to use the upper limit from an experiment to set limits on individual phenomenological parameters by making use of theoretical calculations that establish the dependence on the individual parameters. This is the conventional approach, and is based on the reasoning that if the measured EDM is small then either all the contributions to the EDM (all the $\alpha_i C_i$) are small as well or large contributions must effectively cancel, that is have opposite signs and similar magnitudes. 
 While such a cancellation would be fortuitous, it may be ``required" in the sense that any underlying source of CP violation generally contributes CP violation in more than one way. Take, for example, a Left-Right Symmetric model, which contributes to both $\gpio$ and the short-range part of the neutron EDM, ${\bar d}_n^\mathrm{sr}$ as given in equations~\ref{eq:LRSM2},~\ref{eq:LRSM3} and \ref{eq:dnfull}. A cancellation would require a value of $\sin\xi$ less than $2\times 10^{-6}$. Thus in this model, either the phase $\alpha$  is very small or the mixing angle is very small, or both.
 %- meaning that the physical left-handed and right-handed bosons do not mix appreciably 
% IS THERE A BETTER EXAMPLE OF A POSSIBLE CANCELLATION?}


\subsection{Sole source}

Sole-source limits on the low-energy parameters are presented in Table~\ref{tb:SoleSource} along with the system that sets the limits. The most conservative upper limit is derived using the smallest $|\alpha_{ij}|$ from the ranges presented in Tables~\ref{tb:paramagnetics} and \ref{tb:diamagnetics}. The sole source short-range neutron contribution assumes $\gpbz=\gpbo=0$. For the short-range proton contribution, the model of~\textcite{rf:SandarsTlF} is used for TlF and from~\textcite{dmitriev03} for $^{199}$Hg. The combination of light quark EDMs $d_d-1/3d_u$ is derived from the limit on $d_n$. The parameter $\bar \theta$ and the combination of CEDMs $\tilde d_d-\tilde d_u$ are derived from the sole-source limits on  $\gpbz$ and $\gpbo$, respectively.

\begin{table}[ht]
\begin{centering}
\begin{tabular}{|c|c|c|}
\hline\hline
LE Parameter &  system & 95\% u.l. \\
\hline
$d_e$ &  ThO & $9.2\times 10^{-29}$ \ecm\\
\hline
$C_S$ &   ThO & $8.6\times 10^{-9}$ \\
\hline
$C_T$ &   $^{199}$Hg &  $3.6\times 10^{-10}$\\
\hline
$\gpbz$ &  $^{199}$Hg & $3.8\times 10^{-12}$\\
\hline
%$\gpbz$ &  neutron & $2.2\times 10^{-12}$\\
%\hline
$\gpbo$ &  $^{199}$Hg & $3.8\times 10^{-13}$\\ 
\hline
%$\gpbo$ &  TlF& $4.1\times 10^{-10}$\\ 
%\hline
$\bar g_\pi^{(2)}$ &  $^{199}$Hg & $2.6\times 10^{-11}$\\ 
\hline
$\bar d_n^{sr}$ & neutron & $3.3\times 10^{-26}$ \ecm\\
\hline
$\bar d_p^{sr}$ & TlF & $8.7\times 10^{-23}$ \ecm\\
\hline
$\bar d_p^{sr}$ & $^{199}$Hg & $2.0\times 10^{-25}$ \ecm\\
\hline\hline
\multicolumn{3}{|c|}{Other parameters}  \\
\hline
$d_d$ & $\approx 3/4 d_n$ & $2.5\times 10^{-26}$ \ecm \\
\hline
$\bar\theta$ & $\approx \gpbz/(0.015)$ & $2.5\times 10^{-10}$\\
\hline
$\tilde d_d-\tilde d_u$ & $5\times 10^{-15}\gpbo$ \ecm & $2\times 10^{-27}$ \ecm  \\
\hline\hline
\end{tabular}
\caption{Sole-source limits (95\% c.l.) on the absolute value of the low energy (LE) parameters presented in Sec.~\ref{sec:LowEnergyParameters}  for several experimental systems assuming a single contribution to the EDM or, for molecules, the P-odd/T-odd observable. The lower part of the table presents limits on other parameters derived from the best limits on the low energy parameters. \label{tb:SoleSource}}
\end{centering}
\end{table}
\subsection{Global analysis}

A global analysis of EDM results has been introduced by~\textcite{Chupp:2014gka} and is updated here. In this approach simultaneous limits are set on six low-energy parameters: $d_e$, $C_S$, $C_T$, $\gpbz$, $\gpbo$ and the short-range component of the neutron EDM $d_n^{sr}$. New results from HfF$^+$, $^{199}$Hg and $^{225}$Ra along with clarifications of the isospin dependence of $C_T$ are included in the analysis presented below.
%Paramagnetic systems are mostly sensitive to $d_e$ and $C_S$ and results in four hadronic systems (the neutron, TlF, $^{129}$Xe, $^{199}$Hg) can be used to simultaneously constrain the remaining four parameters.

%  As noted in section~\ref{sec:LEParameters}  the expected magnitude of $\gpbt$ relative to $\gpbz$ and $\gpbo$ is always suppressed by a factor $\lesssim 0.01$ associated with isospin breaking and only the CPV $\pi$-$NN$ coupling constants $\gpbz$ and $\gpbo$ are included in our analysis.


%For the Schiff moment measured in diamagnetic atoms, the nuclear theory uncertainties associated with the $\gpbo$ contribution to the $^{199}$Hg  particularly large: the sign of $a_1$ is undetermined, and it is possible that  its magnitude may be vanishingly small\cite{Engel:2013lsa}. In contrast, the computations of $a_1$ for other diamagnetic systems,  appear to be on firmer ground.

%{\bf At present, we do not possess an up-to-date, consistent set of $\rho_Z^N$ for all of the diamagnetic atoms of interest here. Rather than introduce an additional set of associated nuclear theory uncertainties, we thus do not include these terms in our fit. Looking to the future, additional nuclear theory work in this regard would be advantageous since, for example, the sensitivity of the present $^{199}$Hg result to $d_n$ is not too different from the limit obtained in Ref.~\cite{Baker:2006ts}.  }


%We observe that there exist more CPV sources than independent low-energy observables. Restricting one's attention to interactions of mass dimension six or less involving only the first generation fermions and massless gauge bosons, one finds thirteen independent operators. For the paramagnetic systems, the situation is somewhat simplified, as there exist only three relevant operators: the electron EDM and the two scalar (quark) $\times $ pseudscalar (electron) interactions. For the systems of experimental interest, the electron EDM and $C_S^{(0)}$ operators dominate. For the diamagnetic systems, on the other hand, there exist ten underlying CPV sources that may give rise to the quantities  $\gpbz$, $\gpbo$, ${\bar d}_n^\mathrm{sr}$, and $C_T^{(0,1)}$. Even with the possible addition of a future proton EDM constraint, thereby adding one additional low-energy parameter ${\bar d}_p^\mathrm{sr}$, it would not be possible to disentangle all ten sources from the experimentally accessible quantities. Future searches for the EDMs of light nuclei may provide additional handles (see, {\em e.g.} Ref.~\cite{Engel:2013lsa} and references therein), but an analysis of the prospects goes beyond the scope of the present study. Instead, we concentrate on the present and prospective constraints on the dominant low-energy parameters $d_e$, $C_S$, $\gpbz$, $\gpbo$, ${\bar d}_n^\mathrm{sr}$, and $C_T^{(0,1)}$. 

%From the arguments presented above, there are seven dominant effective-field-theory parameters: $d_e$, $C_S$, $C_T$, $\gpbz$, $\gpbo$, and the two isospin components of the short-range hadronic contributions to the neutron and proton EDMs, which we isolate as ${\bar d}_n^\mathrm{sr}$ and ${\bar d}_p^\mathrm{sr}$ in Eqn.~\ref{eq:dpfull}. We, thus, write  the EDM of a particular system as
%\begin{equation}
%d=\alpha_{d_e} d_e + \alpha_{C_S} C_S + \alpha_{C_T} C_T + \alpha_{\bar d_n^\mathrm{sr}} \bar d_n^\mathrm{sr} + \alpha_{\bar d_p^\mathrm{sr}} \bar d_p^\mathrm{sr}  + \alpha_{g_\pi^0} \bar g_\pi^0 + \alpha_{g_\pi^1}  \bar g_\pi^1 ,
%\end{equation}
%where $\alpha_{d_e}=\partial d/\partial d_e$, {\it etc}. This is compactly written in  Eqn.~\ref{eq:d_i}.
%The coefficients $\alpha_{ij}$ are provided by atomic and nuclear theory calculations and are listed in Tables~\ref{tb:paramagnetics} and~\ref{tb:diamagnetics} for diamagnetic and paramagnetic systems, respectively.
%The sensitivity of the EDM for each experimental system to the parameters presented as a best value and a reasonable range as set forth in Ref.~\cite{Engel:2013lsa}. 
%For the paramagnetic systems, the experimenters generally report their results as an electron-EDM measurement, so the $\alpha_{d_e}$ in table~\ref{tab:ExperimenmtalEDMs} are listed as unity and $\alpha_{C_S}$ as the ratio of sensitivities for $C_S$ and $d_e$~\cite{rf:DzubaPHYSICALREVIEWA84,052108(2011)}.%, with the range representing the theoretical uncertainties of both coefficients, though the uncertainties are quite small for the paramagnetic systems.
 
%\begin{table}
%\centering
%\begin{tabular}{|c|c|lc|}
%\hline\hline
%System & Year/ref &\multicolumn{2}{c|}{Result}  \\
%\hline
%\multicolumn{4}{|c|}{Paramagnetic systems}\\
%\hline
%Cs  & 1989~\cite{Murthy:1989zz} & $d_A=(-1.8\pm6.9)\times 10^{-24}$ & e\,cm \\
%&& $d_e=(-1.5\pm 5.6)\times 10^{-26}$ & e\,cm    \\
%\hline
%Tl  & 2002~\cite{Regan:2002ta}  &$d_A=(-4.0\pm 4.3)\times 10^{-25}$ & e\,cm \\
%& &  $d_e=(\quad 6.9\pm 7.4)\times 10^{-28}$ &  e\,cm  \\
% \hline
%YbF & 2011~\cite{Hudson:2011zz}  &  $d_e=(-2.4\pm 5.9)\times 10^{-28}$  &  e\,cm \\
%&&&\\
%\hline
%ThO & 2014~\cite{Baron:2013eja} &  $\omega^{\mathcal NE}=2.6\pm 5.8$ & mrad/s     \\
%  &   &  $d_e=(-2.1\pm 4.5)\times 10^{-29}$ & e\,cm    \\
%  &   &  $C_S=(-1.3\pm 3.0)\times 10^{-9}$ &   \\  
%\hline
%\multicolumn{4}{|c|}{Diamagnetic systems}\\
%\hline 
%$^{199}$Hg & 2009~\cite{Griffith:2009zz} & $d_A=(0.49\pm 1.5)\times 10^{-29}$ & e\,cm \\
%\hline
%$^{129}$Xe &2001~\cite{rf:Rosenberry} &  $d_A=(0.7\pm 3)\times 10^{-27}$ & e\,cm\\
%\hline
%TlF & 2000~\cite{rf:Cho} & $d_{\rm\ }=(-1.7\pm 2.9)\times 10^{-23}$ & e\,cm \\
%\hline
%neutron & 2006~\cite{Baker:2006ts} & $d_n=(0.2\pm1.7)\times 10^{-26}$   & e\,cm\\
%\hline
%\hline 
%\end{tabular}
%\caption{EDM results used in our analysis as presented by the authors. When $d_e$ is presented, the assumption is $C_S=0$, and for ThO, the $C_S$ result assumes $d_e=0$. We have combined statistical and systematic errors in quadrature for cases where they are separately reported by the experimenters.   \label{tb:EDMResults}}
%\end{table}


%The starting point for our analysis is the set of low-energy atomic and hadronic interactions most directly related to the EDM measurements. We distinguish two classes of systems: paramagnetic systems, namely, those having an unpaired electron spin, and diamagnetic systems, or those having no unpaired electron (including the neutron).





\subsection*{Paramagnetic systems: limits on $d_e$ and $C_S$}


 
Results are listed in Table~\ref{tb:EDMResults} for paramagnetic systems  Cs, Tl, YbF, ThO and HfF$^+$. Following~\textcite{rf:Dzuba2011,rf:Dzuba2011-erratum} we take the electron EDM result reported by each author to be the combination
 \begin{equation}
 d^\mathrm{exp}_j= d_e +\bigl(\frac{\alpha_{C_S}}{ \alpha_{d_e}}\bigr)_j C_S.
 \label{eq:deEquation}
 \end{equation} 
 The $\alpha_{C_S}/ \alpha_{d_e}$ are listed in Table~\ref{tb:paramagnetics}. 
%This can be inverted to find the electron EDM
 %\be
% d_e=d^\mathrm{exp}_\mathrm{para}-\frac {\alpha_{C_S}}{ \alpha_{d_e}} C_S.
 %\ee
As pointed out by \textcite{rf:Dzuba2011,rf:Dzuba2011-erratum}, though there is a significant range of $\alpha_{d_e}$ and $\alpha_{C_S}$ from different authors for several cases, there is much less dispersion in the ratio ${\alpha_{C_S}/\alpha_{d_e}}$.% {\it i.e.} from $(0.6-1.5)\times 10^{-20}$ e\,cm for the paramagnetic systems and from $(3-5)\times 10^{-20}$ for Hg, Xe and TlF.  

In  Figure~\ref{fig:Paramagnetics}, we plot $d_e$ vs $C_S$ for the   $d^{exp}_\mathrm{para}$ for ThO and HfF$^+$ along with 68\% and 95\% confidence-level contours for $\chi^2$ on the $d_e$-$C_S$ space, where
\begin{equation}
\chi^2=\sum_{i} \frac{\bigl[ d^\mathrm{exp}_i-d_e -\bigl(\frac {\alpha_{C_S}}{ \alpha_{d_e}}\bigr)_i C_S\bigr]^2}{\sigma_i^2}. 
\end{equation}
Here $i$ sums over Cs, Tl, YbF, ThO and HfF$^+$, but only ThO and HfF$^+$ have significant impact. The range of $\bigl(\frac {\alpha_{C_S}}{ \alpha_{d_e}}\bigr)_j$ expressed in Table~\ref{tb:paramagnetics}, about 10\%, is accommodated by adding in quadrature to the total experimental uncertainty for each system. 
The resulting constraints from all paramagnetic systems on $d_e$ and $C_S$ at 68\%\ {\rm c.l.} are
\begin{equation}
d_e=(0.8\pm 4.2)\times 10^{-28}\ {\rm e\ cm} \quad C_S=(-0.9\pm 3.7)\times 10^{-8}.
\label{eq:deGlobal68}
\end{equation}
The upper limits at 95\% confidence level are
\begin{equation}
|d_e|<8.4\times 10^{-28}\ {\rm e\ cm} \quad |C_S|<7.5\times 10^{-8}\quad (95\%\ {\rm c.l.}).
\end{equation}
%Note that the constraint on $C_S$ is actually better than that from ThO alone. 


\begin{figure}[ht]
\centerline{\includegraphics[width=3.5in]{deCsLimitsThOHfFRev}}%deCSFinal.pdf}}
\caption{(Color online) Electron EDM $d_e$ as a function of $C_S$ from the experimental results in ThO and HfF$^+$ with 1$\sigma$ experimental error bars. Also shown are 68\% and 95\% $\chi^2$ contours for all paramagnetic systems including Cs, Tl, YbF.  The top and right axes show the corresponding dimensionless Wilson coefficients $\delta_e$ and $\mathrm{Im}\, C_{eq}^{(-)}$ normalized to the squared scale ratio $(v/\Lambda)^2$.}
\label{fig:Paramagnetics}
\end{figure}






%To set constraints, the four results are fit to the form Eqn.~(\ref{eq:deEquation}) with $d_e$ and $C_S$ left as free parameters The results are 
%\begin{equation}
%d_e=(-0.1\pm 1.9/6.1)\times 10^{-27}\ {\rm e\,cm}
%\quad
%C_S=(0.1\pm 1.3/4.4)\times 10^{-7}\quad (68\%/95\%)\  {\rm CL\ Best\ coefficient \ values}. 
%d_e=(-0.4\pm 2.2)\times 10^{-27}\ {\rm e\,cm}
%\quad
%C_S=(0.3\pm 1.7)\times 10^{-7}. 
%\end{equation}
%In order to account for the variation of atomic theory results we vary $\alpha_{C_S}/\alpha_{d_e}$ over the ranges presented in Table~\ref{tb:paramagnetics} and find that when the $\alpha_{C_S}/\alpha_{d_e}$ are most similar,
%\begin{equation}
%d_e=(-0.8\pm 2.9/9.5)\times 10^{-27}\ {\rm e\,cm}
%\quad
%C_S=(0.7\pm 2.4/8.0)\times 10^{-7}\quad (68\%/95\%)\ {\rm CL\ Reasonable\ coefficient\ ranges}. 
%d_e=(-0.3\pm 3.0)\times 10^{-27}\ {\rm e\,cm}
%\quad
%C_S=(0.2\pm 2.5)\times 10^{-7}.
%\label{eq:deCSresult}
%\end{equation}

%Slightly tighter limits were found by including  $^{199}$Hg, through an iterative procedure that first determined the $C_T$ and $\bar g_\p[i^{0,1}$ contributions to $d_A(^{199}{\rm Hg})$.When $^{199}$Hg is included, the limits on $d_e$ and $C_S$ improve slightly due to the lever arm provided by the significantly different ${\alpha_{C_S}/ \alpha_{d_e}}$ compared to the paramagnetic systems with the result:
%\begin{equation}
%d_e=(-0.8\pm 2.3/7.0)\times 10^{-27}\ {\rm e\,cm}
%\quad
%C_S=(0.7\pm 2.2/5.9)\times 10^{-7}\quad (68\%/95\%)\ {\rm CL\ including\ ^{199}Hg}. 
%d_e=(-0.3\pm 2.7)\times 10^{-27}\ {\rm e\,cm}
%\quad
%C_S=(0.2\pm 2.3)\times 10^{-7}\\
%\end{equation}
%The 68\%/95\% upper limits are
%\begin{equation}
%d_e=(-0.8\pm 2.3/7.0)\times 10^{-27}\ {\rm e\,cm}
%\quad
%C_S=(0.7\pm 2.2/5.9)\times 10^{-7}\quad (68\%/95\%)\ {\rm CL\ including\ ^{199}Hg}. 
%|d_e|=<(2.7/5.4)\times 10^{-27}\ {\rm e\,cm}
%\quad 
%|C_S|<(2.3/4.5)\times 10^{-7}
%\nonumber
%\end{equation}
%Error ellipses representing 68\% and 95\% confidence interval for the two parameters $d_e$ and $C_S$ % based on paramagnetic systems and $^{199}$Hg 
%are presented in Figure~\ref{fig:Paramagnetics}.
%Also shown are error ellipses representing  the range of $d_e$ vs $C_S$ allowed by  the 1-$\sigma$ results for Tl, YbF, ThO for the best values of $\alpha_{C_S}\over \alpha_{d_e}$ from table~\ref{tb:Coefficients}. 
Corresponding 95\% c.l. constraints on $\delta_e(v/\Lambda)^2$ and $\mathrm{Im}\, C_{eq}^{(-)} (v/\Lambda)^2$, obtained from those for $d_e$ and $C_S$ by dividing by $-3.2\times 10^{-22}$ e\,cm and $-12.7$, respectively are
\be
|\delta_e(v/\Lambda)^2|<2.6\times 10^{-6}\quad \mathrm{Im}\, C_{eq}^{(-)} (v/\Lambda)^2<5.9\times 10^{-9}.
\ee
The $^{199}$Hg result is not included in the constraints of Eq~\ref{eq:deGlobal68}, but has been used to  constrain $d_e$ and $C_S$, for example by~\textcite{Chupp:2014gka,Jung:2013mg,Fleig:2018bsf}. Particularly notable is that the limits on the scalar quark-pseudoscalar electron interaction may place a lower bound on the mass scale $\Lambda$ of roughly $10^3$ TeV or more. This has been applied in the context of leptoquark models by~\textcite{Fuyuto:2018scm}.



% In order to highlight the recent baker_edm of EDM results in paramagnetic systems, in particular the recent results from paramagnetic molecules YbF~\cite{rf:Hudson2012} and ThO~\cite{rf:ThO}, in table~\ref{tb:deResults} we list the 95\% confidence interval upper limit on $d_e$ and $C_S$ based on including the results from the systems shown. From this we conclude that $d_e<4.5\times 10^{-27}$ e\,cm and $C_S<3.2\times 10^{-7}$ at 95\% confidence.

\subsection*{Hadronic parameters and $C_T$}

%Diamagnetic atom EDMs are most sensitive to the hadronic parameters $\gpbz$ and $\gpbo$ and the electron-nucleon contribution $C_T$. As noted above, $d_e$ and $C_S$ contribute to diamagnetic systems in higher order.  Given that $d_e$ and $C_S$ are effectively constrained by the paramagnetic systems, constraints on the four free parameters $C_T$, $\gpbz$, $\gpbo$ and ${\bar d}_n^\mathrm{sr}$ are provided by four experimental results from  TlF, $^{129}$Xe and $^{199}$Hg and the neutron.  
Since the introduction of our global analysis, there have been advances in hadronic and atomic theory along with three significant experimental developments in the diamagnetic/hadronic systems: 
\begin{enumerate}[i.]
\item
the four-times more sensitive result for $^{199}$Hg~\cite{Graner:2016ses-erratum}
\item  reanalysis of the neutron-EDM  which increased the uncertainty and moved the centroid by about 1/4\ $\sigma$~\cite{Afach:2015sja}, 
\item results from the octupole deformed $^{225}$Ra~\cite{PhysRevC.94.025501} %{rf:Parker2015}
\end{enumerate}
There are experimental results in five systems and four parameters $d_n^{sr}$, $C_T$,  $\gpbz$ and $\gpbo$, which are fully contstrained once $d_e$ and $C_S$ are fixed from the paramagnetic-systems results. In order to provide estimates of the allowed ranges of the  four parameters,  $\chi^2$ is defined as
 \begin{equation}
 \chi^2({\bf C_j})=\sum_i \frac{(d_i^\mathrm{exp}-d_i)^2}{\sigma_{d_i^\mathrm{exp}}^2},
 \end{equation}
 where $d_i$ have the form given in equation~\ref{eq:d_i}.
The four parameters ${\bf C_j}$ are varied to determine $\chi^2$ contours for a specific set of $\alpha_{ij}$. For 68\% confidence and four parameters, $(\chi^2-\chi^2_{min})<4.7$. 
%(Note that $\chi^2_{min}=0$ for four parameters with four constraints.)
The  $\alpha_{ij}$ are varied over the ranges presented in Table~\ref{tb:diamagnetics} to reflect the hadronic-theory uncertainties.
Estimates of the constraints are presented as ranges in Table~\ref{tab:gpiCTdn}, which has been updated from~\textcite{Chupp:2014gka}. The significant improvement in limits on $C_T$ is largely due to the change in sensitivity estimates ($\alpha_{i C_T^{0}}$) due to the recent calculations of the tensor form factors (see Eqn.~\ref{eq:CTs})\cite{Bhattacharya:2016zcn,Bhattacharya:2015wna}. Limits on $\gpiz$ also improve by about 50\% while limits on $\gpio$ and $\bar d_n^{sr}$ are  about 50\% less stringent.


%In effect $^{199}$Hg is most important in determining $g_\pi^0$ while TlF is most important in determining $g_\pi^1$ and the addition of the $^{129}$Xe result allows us to constrain $C_T$. We note that the neutron 


Since $\gpiz$ and $\gpio$ also contribute to the neutron EDM,  the short-range neutron contribution $\bar d_n^{sr}$ is notably much less constrained than the experimental limit on the neutron EDM itself.  The anticipated improved sensitivity in the next few years for the diamagnetic systems $^{199}$Hg, $^{129}$Xe, $^{225}$Ra and TlF will provide tighter constraints on $\gpiz$ and $\gpio$; however the constraints do have significant correlations. The correlations of pairs of parameters are illustrated  in FIG.~\ref{fg:dnsrvsgpis}, which shows the 68\% contour on plots of allowed values of $\bar d_n^{sr}$ vs $\gpiz$, $\gpio$ and $C_T^{(0)}$ as well $\gpio$ vs $\gpiz$.


\begin{figure*}[h]
\centerline{\includegraphics[width=4truein]{dnsrvsgpi068Contourgpi1Hgg1p6em17}\includegraphics[width=4truein]{dnsrvsgpi168Contourgpi1Hgg1p6em17}}%deCSFinal.pdf}}
\centerline{\includegraphics[width=4truein]{gpi1vsgpi068Contourgpi1Hgg1p6em17}\includegraphics[width=4truein]{dnsrvsCT68Contourgpi1Hgg1p6em17}}%deCSFinal.pdf}}
\caption{Combinations of hadronic parameters allowed by experimental results for the best values for $\alpha_{ij}$ in Table~\ref{tb:diamagnetics} with $\alpha_{{\rm Hg},\gpbo}=1.6\times 10^{-17}$ and $\alpha_{{\rm Ra},\bar d_n^{sr}}=-8\times 10^{-4}$. The allowed values at 68\% c.l. are contained within the ellipses for each pair of parameters.}
\label{fg:dnsrvsgpis}
\end{figure*}

%\begin{sidewaystable}
\begin{table*}[h]%\centering
%\begin{tabular}{||c|c|c|c|c|c|c|c|c|c||}
\begin{tabular}{||c|c|c|c|c||}
\hline\hline
%& & & & & & $C_T$ &$g_\pi^0$ & $g_\pi^1$& $d_n$ (e\,cm) \\
&&&&\\
&${\bar d}_n^\mathrm{sr}$ (e\,cm) &$\gpbz$ & $\gpbo$& $C_T^{(0)}$  \\
\hline
%Exact solution & $1.265$ &$-6.687\times 10^{-10}$ & $1.4308\times 10^{-10}$& $9.878\times 10^{-24}$ \\
%9.87852e-24
%1.2652e-07
%-6.68686e-10
%1.43081e-10
%\hline
Range from best values &&&&\\
with $\alpha_{g_\pi^1}(\mathrm{Hg})=+1.6\times 10^{-17}$  & $(-4.8$-$9.8)\times 10^{-23}$ &$(-6.6$-$3.2)\times 10^{-9}$ & $(-1.0$-$0.5)\times 10^{-9}$& $(-3.5$-$1.6)\times 10^{-7}$ \\
\hline
Range from best values &&&&\\
with $\alpha_{g_\pi^1}(\mathrm{Hg})=0$ & $(-4.3$-$3.4)\times 10^{-23}$ &$(-2.3$-$2.9)\times 10^{-9}$ & $(-0.6$-$1.3)\times 10^{-9}$& $(-3.2$-$4.0)\times 10^{-7}$\\
\hline
Range from best values &&&&\\
with $\alpha_{g_\pi^1}(\mathrm{Hg})=-4.9\times 10^{-17}$  & $(-9.3$-$2.6)\times 10^{-23}$ &$(-1.8$-$6.3)\times 10^{-9}$ & $(-1.2$-$0.4)\times 10^{-9}$& $(-11$-$3.8)\times 10^{-7}$\\
\hline
Range from full variation of $\alpha_{ij}$  &   $(-12$-$12)\times 10^{-23}$ &$(-7.9$-$7.8)\times 10^{-9}$ & $(-1.3$-$1.1)\times 10^{-9}$& $(-6.6$-$4.6)\times 10^{-7}$\\
\hline
Upper limits (95\% c.l.) & $2.4\times 10^{-22}$ & $1.5\times 10^{-8}$  & $2.4\times 10^{-9}$ & $1.1\times 10^{-6}$ \\
\hline\hline
\end{tabular}
\caption{\label{tab:gpiCTdn} Revised values and ranges for coefficients for diamagnetic systems and the neutron. The first three rows give the 68\% c.l. range allowed by experiment combined with the best values of the coefficients $\alpha_{ij}$ covering the reasonable range of $\alpha_{{\rm Hg},\gpio}$ with  $\alpha_{{\rm Ra},\bar d_n^{sr}}=-8\times 10^{-4}$; the fourth row gives ranges of coefficients for the entire reasonable ranges of the coefficients $\alpha_{ij}$ given in Table~\ref{tb:diamagnetics}, and the bottom row presents the 95\% c.l. upper limits on the coefficients for the full reasonable ranges of the coefficients.}
\end{table*}



%What is needed to separate weak and strong CPV once a signal is found, and to distinguish different models of weak CPV?

In this global analysis approach, the constraints on each parameter depend on all experiments, the sensitivity of the EDM results, and the range of theoretical uncertainties of the $\alpha_{ij}$  given in Table~\ref{tb:diamagnetics}. To illustrate the dependence of the four dominant hadronic parameters on the experimental results  this we choose four of five experiments: $^{199}$Hg, $^{129}$Xe, $^{225}$Ra, and the neutron. The inverse of the matrix $\alpha_{ij}$ from Eqn.~\ref{eq:d_i} is
\begin{eqnarray}
& &\mbox{\footnotesize $
\left[ 
\begin{array}{c}
\bar d_n^{\rm sr} \\ \gpbz  \\ \gpbo\\ C_T^{(0)}
\end{array}
\right] $} = \nonumber\\
& & 
\mbox{\footnotesize $
\begin{bmatrix}
% n  Xe  Hg  Ra  //  dn  gpi0  gpi1  CT
5.2 &\ \, 4.7\times 10^{4}&\ \, 9.5\times 10^{3} & 21\\
-2.8\times 10^{14} &-3.1\times 10^{18}&-6.3\times 10^{17}&-1.4\times 10^{15}\\
-7.0\times 10^{13} &-7.7\times 10^{17}&-1.6\times 10^{17}&-4.8\times 10^{14}\\
\ \ 1.9\times 10^{16} &\ \  1.4\times 10^{19}&\ \  3.6\times 10^{19}&\ \  8.4\times 10^{16}\\
%-1.48E+19&1.83E+20&-2.52E+14&0\\
%-1.85E+17&-9.64E+17&1.32E+12&0\\
%-2.41E+16&5.36E+16&-6.32E+12&0\\
%2.78E+03&1.44E+04&-1.90E-02&1\\
\end{bmatrix}
\left[
\begin{array}{c}
 d_n \\ d_{\rm Xe}  \\d_{\rm Hg}\\ d_{\rm  Ra} 
\end{array}
\right]
$},
\nonumber\\
\end{eqnarray}
for the best values from Table~\ref{tb:diamagnetics} with  $\alpha_{{\rm Hg},\gpbo}=1.6\times 10^{-17}$ and $\alpha_{{\rm Ra},\bar d_n^{sr}}=-8\times 10^{-4}$ . For example
\be
 {\bar d_n^{sr}}=5.2 {d_n} + 4.7\times 10^{4} {d_{\rm Xe}} + 9.5\times 10^{3}{d_{\rm Hg}}+21    {d_{\rm Ra}}\nonumber \\
\ee
This combined with the results from  Table~\ref{tb:EDMResults}   shows that the $^{129}$Xe and $^{225}$Ra results have comparable contributions to the constraints and that improving each by a factor of about 500 would make their impact similar to that of $^{199}$Hg in the context of this global analysis.

\section{Summary and Conclusions}

We live in exciting times for EDMs. The observation and explanations of the baryon asymmetry call for BSM sources of CP violation that produce EDMs  that may be discovered in the next generation of experiments in a variety of systems. Experiment has marched forward with greater sensitivity recently achieved for the neutron and $^{199}$Hg, the tremendous advance in complexity and sensitivity for ThO and HfF$^+$ polar-molecule experiments sensitive to the electron EDM, and with new techniques providing results from the octupole deformed $^{225}$Ra. The next generation of experiments on the neutron will take advantage of new ideas and techniques incorporated into UCN sources and EDM techniques at a number of laboratories around the world; and new approaches to magnetic shielding and magnetometry/comagnetometry along with deeper understanding of systematic effects will be essential to achieving the next step in sensitivity in all systems. Storage rings and rare isotopes are expected to be new approaches that move forward in the coming years.

Interpretations of EDM limits and eventually finite results continue to advance with more quantitative connections to baryogenesis and clarification of effective-field theory approaches that connect fundamental quantum field theory to low-energy parameters relevant to the structure of nucleons, nuclei, atoms and molecules. The theory of EDMs brings together theoretical approaches at each of these scales, however the nucleus is a particularly difficult system for calculations and introduces the largest uncertainties in connecting experiment to theory. 
The best experimental result - in $^{199}$Hg -  is challenged by significant nuclear theory uncertainties.
%s unfortunately subject to the large nuclear-theory uncertainties {\it i.e.} the best experiment has the worst theory.  
 With the increasing interest in EDMs due to their role in connecting cosmology, particle physics and nuclear/atomic and molecular physics, the motivations for tackling these problems in hadronic theory become stronger. 

Even in light of current uncertainties, interpretation of EDM results from the sole-source perspective or in the context of a global analysis show that CP violating parameters are surprisingly small. In the case of the QCD parameter $\bar\theta$ this leads to the strong-CP problem and its potential solution via the axion hypothesis, which may also provide an explanation of non-baryonic dark matter. In a generic approach to CP violation consistent with current limits, combined with an assumption that the phases are of order unity, the mass scale probed is tens of TeV or greater, emphasizing the complementarity of EDMs and the LHC as well as future higher-energy colliders. In the context of models that introduce new phases, such as SUSY variants and Left-Right Symmetric Models, either the phases appear to be far less than naturally  expected or the mass scale of CP violation is quite large, which introduces challenges with the connection to Electroweak Baryogenesis. 

The definitive observation of an EDM in any system will be a tremendous achievement, but a single system alone may not clarify the questions arising in the connections to fundamental theory and to cosmology, for example separating weak and strong CP-violation. We therefore conclude by calling for efforts in several systems - paramagnetic systems most sensitive to the electron EDM and electron-spin-dependent CP violating interactions as well as diamagnetic atoms/molecules, nucleons and nuclear systems where hadronic CP violation is dominant. We also call for advanced theory efforts, in particular nuclear theory, which must improve to sharpen interpretation of EDM results in all systems.

\begin{acknowledgments}

The authors would like to thank everyone in the community who has provided input and advice as well as encouragement. In particular we are grateful to Martin Burghoff, Will Cairncross, Vincenzo Cirigliano, Skyler Degenkolb,  Matthew Dietrich, Peter Geltenbort, Takeyasu Ito, Martin Jung, Zheng-Tian Lu, Kent Leung, Kia Boon Ng, Natasha Sachdeva, Z Sun,  Yan Zhou, and Oliver Zimmer. The authors also acknowledge the Excellence Cluster Universe, MIAPP, the Munich Institute for Astronomy and Astrophysics and MITP, the Mainz Institute for Theoretical Physics for hosting and providing
 the opportunity to collaborate.  TC has been supported by US Department of Energy  grant No. DE-FG0204ER41331; PF is supported by the Deutsche Forschungsgemeinschaft (DFG) Priority Program SPP 1491 �Precision Experiments with Cold and Ultra-Cold Neutrons� ; MJRM is supported by US Department of Energy Grant DE-SC0011095; JS is supported by Michigan State Univeristy.
 \end{acknowledgments}
%%%%%%%%%%%%%%%%%%%%%%%%%%%%%%%%%%%%%%%%%%%%%
%\bibliographystyle{apsrmp4-1}
%\bibliography{ChFierMJRMEDMReviewRefs}
\bibliography{rmp-handc}



%%%%%%%%%%%%%%%%%%%%%%%%%%%%%%%%%%%%%%%%%%%%


\end{document}

