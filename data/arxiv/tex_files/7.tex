\section{Classroom Response Systems} \label{sec:crs}
%\color{blue}

CRSs started with Classtalk~\cite{beatty2005transforming}, which used graphing calculators to allow students communicate. In addition to dedicated ‘clicker’ devices, current solutions include “bring your own device” (BYOD), and solutions based in image-processing. Table~\ref{tab:1} presents and compares those solutions.

Vickrey et at. literature review~\cite{vickrey2015PI} concludes PI can be effectively implemented with clickers or any other tools --- suggesting image-processing clickers solutions are as effective as any other polling methods.

% Dimensions analyzed:
% Cost
% Availability
% Complexity of use
% Anonimity for Classmates
% Anonimity for Instructor
% Flexibility of Answer

\begin{table*}[t]
    \begin{center}
        \begin{tabular}{l l l}
            \hline
             Method &  Advantages &  Disadvantages\\
            \hline
            \makecell[l]{“Low-tech” alternatives\\(show of hands, color cards)}&
            \makecell[l]{
                Very low-cost\\
                Available immediately everywhere\\
                Very easy to use
            }&
            \makecell[l]{
                Classmates tend to “follow the majority”\\
                Individual answers unrecoverable\\
                Only multiple-choice answers possible
            }\\

            \hline
            \makecell[l]{Dedicated hardware\\(‘clickers’)}&
            \makecell[l]{
                Wide commercial availability\\
                Classmates cannot see answers\\
                Instructor recovers individual answers\\
                Moderate to complex answers possible
            }&
            \makecell[l]{
                High direct and indirect costs\\
                Complex training required for teachers
            }\\

            \hline
            \makecell[l]{Software on students' devices\\(BYOD)}&
            \makecell[l]{
                Good commercial availability\\
                Classmates cannot see answers\\
                Instructor recovers individual answers\\
                Very complex answers (e.g. drawings) possible\\
                Low-cost for institutions
            }&
            \makecell[l]{
                High-cost for students\\
                Devices can be distracting\\
                Requires reliable network infrastructure\\
                Training required for teachers and students
            }\\

            \hline
            \makecell[l]{Software on teachers' device +\\cards with barcodes for students\\(image processing)}&
            \makecell[l]{
                Low-cost for students and institution\\
                Classmates cannot see answers\\
                Instructor recovers individual answers\\
                Simple training required for teachers,\\virtually no training for students
            }&
            \makecell[l]{
                Few (mostly experimental) solutions\\
                Only multiple-choice answers possible\\
                Requires line of sight to each student
            }\\

            \hline
        \end{tabular}
    \end{center}
%    \setlength{\abovecaptionskip}{5pt plus 3pt minus 2pt}
    \caption{Summary of classroom response system technologies. Image-processing CRSs --- like paperclickers --- are the only ones at the intersection of low cost, simplicity, anonymity to classmates, and trackability of answers by instructors. }
    \label{tab:1}
\end{table*}


\subsection{Clickers}
In their most usual form, CRSs use ‘clickers’, one small infra-red or radio-frequency transmitter per student, and one receiver per classroom\cite{caldwell2007clickers}. When the instructor poses a question, the students use the transmitters to answer it, and the receiver instantly  records and tabulates those answers. The instructor can display the results for immediate action, save the data for later analysis, or both~\cite{beatty2005transforming}.

‘Clicker’ solutions count with wide commercial availability and professional support. If correctly implemented and well-maintained, they allow CRSs to run smoothly and non-intrusively~\cite{caldwell2007clickers}. The commonest devices allow only multiple-choice or yes/no answers, but recent (and more expensive) versions allow numbers, words, or short phrases.

Those solutions, however, are expensive. Ideally, each student should own their transmitter (to avoid the time-consuming hassle of distributing and collecting the devices at each class), and receivers must be available for each classroom or, if they are portable, teacher. Teachers have to be trained to use the transmitters~\cite{beatty2005transforming}, and support must be readily available to solve problems~\cite{caldwell2007clickers}, minimizing class disruption. The system needs continual maintenance to work optimally; in particular, it is critical to establish who will be responsible for the devices' batteries (students, teachers, or technicians), and ensure that  responsibility is taken seriously.

Due to those costs and inconveniences, CRSs based on ‘clickers’ are very challenging to implement in developing countries.

\subsection{Bring Your Own Device}

“Bring Your Own Device” (BYOD) --- a model in which students use their own smartphones, tablets, or even laptop computers --- replace hardware ‘clickers’ by software applications that transmit the answers through WiFi or mobile data~\cite{stavert2013byod}. There is a good offer of commercial BYOD software, sometimes from the same vendors of hardware ‘clickers’, even allowing hybrid solutions where both are supported at once.

% Although these solutions work similarly to the original "clickers", some differences can be highlighted: (a) The students use their personal device and require less training to run the CRS application, access the CRS specific social network or web page; (b) The students' answers can be sent by simple short messages (SMS) or through Internet access using school WiFi or mobile data (3G/4G) networks.

CRS software usually has modest hardware requirements, allowing to apply BYOD to a wide range of devices, including obsolete/donated hardware. BYOD also allows very flexible forms of polling, that no other solution provides, including mini-essays, points and graphs in Cartesian coordinates, or even freehand drawings.

Teachers and students still need to be trained to use the system, and there still needs to be technical support, especially to solve connectivity issues. Indeed, network quickly becomes a bottleneck --- both when using public (mobile data) or local (WiFi) infrastructure --- as it is challenging to support many simultaneous connections. Ensuring the usage of the devices remains on purpose and productive is an additional challenge, as students may turn to leisure texting or web browsing.

BYOD assumes there is good network infrastructure, and that each student owns at least a smartphone, and that they feel safe to bring it to school~\cite{stavert2013byod}.  Those assumptions are far from obvious in developing countries, especially in schools serving disfavored communities. For example, the Brazilian Internet Steering Committee shows~\cite{CGIbr2014ITCeducation} that as recently as 2014, even urban schools in Brazil had connectivity issues: although 93\% of those schools had some access to the Internet, only 41\% of them granted access to students. In addition, connectivity was often low-speed and unreliable.


\subsection{Image Processing}

Image-processing solutions minimize costs by giving the students passive devices, usually cards with especial colors or codes, and keeping all active processing into a single device, which remains with the teacher. Most often, the students rotate use a card printed with a special 2D barcode, which serves both as a location and orientation marker, and as a unique ID for each student. The students can answer multiple-choice questions by rotating the cards.

Amy and Amy patented a low-cost optical polling framework~\cite{nolan:2011:biblatex} with a generic computing element that recognizes the orientation of fiducial marks on printed cards. The proposed solution is available as a smartphone app --- Plickers\footnote{https://plickers.com/}, which currently accommodates up to 63 students, which must enroll on a web-based system. Fiducial markers on printed cards had already appeared on previous works, e.g., the augmented reality system ARTag~\cite{fiala2005artag}. Amy and Amy innovate by exploiting them for low-cost CRSs. %; however, as a commercial solution, it requires Internet connectivity to be used --- even for the smartphone application sign in --- which might be a problem on no-connectivity scenarios. Also, the proposed solution does not describe the challenges related to recognizing fiducial markers in a classroom environment --- which indeed created size and encoding power restrictions to the student cards.

Cross et al.~\cite{cross2012low} also proposed a system that recognized the orientation of printed cards with unique IDs for each student, which they called  ‘qCards’. The teacher captures the answers with an off-the-shelf webcam mounted on a laptop, with software to recognize, tabulate, and display the results. The authors ran initial trials on secondary schools in Bangalore, India, with 99.8\% recognition accuracy, and 97\% captured responses in a 25-student classroom.

Miura and Nakada presented similar work~\cite{miura2012device}: they used printed cards with fiducial markers as codes, and a similar setup of camera, and PC containing the software. Their system recognizes three rotational parameters for each cards (roll, pitch and yaw), allowing students to select one of many possible multiple-choice answers in a screen. A preliminary experiment, with 19 students succeeded in tracking 18 markers.

The solution proposed by Gain~\cite{gain2013using} approaches CRSs by colored blocks printed on cardboards and a camera-phone to capture images. Students select answers by picking different colors. They report 85\% recognition accuracy in a medium-size class (up to 125 students). Although the system forgoes peer anonymity, and the possibility to track responses to individual students for later analysis, it is the only image-processing system tested in classes that big.

Finally, Ito and Miura~\cite{ito2015portable} experimented a portable version of that previous system, recognizing the same fiducial marks in an tablet computer, including the capability of detecting the response printed card bending amount as an additional input mechanism, which could encode the student mood.

As far as we known, paperclickers is the only existing image processing CRS solution at the intersection of being an academic work, having its entire source code publicly released, and being available for download on user's devices for actual, practical use. Being on that intersection allows future contributors to easily test hypotheses and add improvements to paperclickers, with real-world impact to users.