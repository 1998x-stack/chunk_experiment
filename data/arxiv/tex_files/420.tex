%%%%%%%%%%%%%%%%%%%%%%%%%%%%%%%%%%%%%%%%%%%%%%%%%
% reductions 
%%%%%%%%%%%%%%%%%%%%%%%%%%%%%%%%%%%%%%%%%%%%%%%%%

\section{Proof of the Orbit Theorem}
\label{sec:reductions}

In this section we prove the Orbit Theorem assuming the Elementary Swap Theorem (Theorem~\ref{thm:elementary-swap-2}, proved in Section~\ref{sec:topology}), and assuming the following two results on elementary swaps.
The first result shows that 
every elementary swap can be realized by a relatively short flip sequence that can be found efficiently,
and the second result gives us a way to combine elementary swaps so that, after moving $e$'s label to $f$, we can get $f$'s label back to $e$.
These lemmas will be proved in Section~\ref{sec:bounds}. 

\begin{lemma}
\label{lemma:elem-swap}
If there is an elementary swap between two edges in a triangulation $\cal T$ then there is a flip sequence of length $O(n^6)$ to realize the elementary swap, and, furthermore, this sequence can be found in polynomial time.
\end{lemma}


\begin{lemma} 
\label{lemma:elem-swap-seq}
Let $\cal T$ be a labelled triangulation containing two edges $e$ and $f$.   If there is a sequence of elementary swaps on $\cal T$ that takes the label of edge $e$ to edge $f$, then there is 
an elementary swap of $e$ and $f$ in $\cal T$.
\end{lemma}

%\comment{
As we show in Section~\ref{sec:bounds}, 
a simple group-theoretic argument suffices to prove a weaker version of Lemma~\ref{lemma:elem-swap-seq}, namely, that under the stated assumptions, there is a sequence of elementary swaps exchanging the labels of $e$ and $f$ in $\cal T$.  
Proving the stronger version, which we need for our bounds on the length of flip sequences, requires using the properties of elementary swaps.
%proving a weaker version of Lemma~\ref{lemma:elem-swap-seq}, namely, that under the stated assumptions, there is a sequence of elementary swaps exchanging labels of $e$ and $f$ in $\cal T$, boils down to a simple group-theoretic argument. 
%Proving, in addition, that this can be done with a single elementary swap requires using the properties of elementary swaps.
%}

We prove the Orbit Theorem in stages, first Theorem~\ref{thm:swap} (the case of swapping two labels in a triangulation),
then the more general case of permuting edge labels in a triangulation, and finally the full result.

\begin{proof}[Proof of Theorem~\ref{thm:swap}] 
The ``if'' direction is clear, so we address the ``only if'' direction.
Suppose that ${\cal T} = (T, \ell)$ is the given edge-labelled triangulation and that 
$e$ and $f$ are edges of $T$ that are in the same orbit.  
%Suppose that edges $e$ and $f$ are in the same orbit.  
Then
there is a sequence of flips that changes ${\cal T}$ to an edge-labelled triangulation ${\cal T}' = (T', \ell')$ where $T'$  contains $f$ and $\ell'(f) = \ell(e)$.  
We now apply the result that any constrained triangulation of a point set can be flipped to any other.
Fix edge $f$ and flip $T'$ to $T$.  
Applying the same flip sequence to the labelled triangulation ${\cal T}'$ yields an edge-labelling of triangulation $T$ in which edge $f$ has the label $\ell(e)$.  Thus we have a sequence of flips that permutes the labels of $\cal T$ and moves the label of $e$ to $f$.

By the Elementary Swap Theorem (Theorem~\ref{thm:elementary-swap-2}) there is a sequence of elementary swaps whose effect is to move the label of edge $e$ to edge $f$.  
By Lemma~\ref{lemma:elem-swap-seq} there is an elementary swap of $e$ and $f$ in $\cal T$. 
 \end{proof}

%%%%%%%%%%%%%%%%%%%%%%%%%%%%%%%%%%%%%%%%%%%


\begin{theorem}[Edge Label Permutation Theorem]
\label{thm:strong-permutation}
Let $T$ be a triangulation of a point set with two edge-labellings $\ell_1$ and $\ell_2$ such that for each label $l$, the edge with label $l$ in $\ell_1$ and the edge with label $l$ in $\ell_2$ are in the same orbit.  Then there is a sequence of $O(n)$ elementary swaps to transform the first labelling to the second.  
Such a sequence can be realized via a sequence of $O(n^7)$ flips, which can be found in polynomial time.
\end{theorem}
\begin{proof}
The idea is to effect the permutation as a sequence of swaps. 
If every edge has the same label in $\ell_1$ and $\ell_2$ we are done.   
So consider a label $l$ that is attached to a different edge in $\ell_1$ and in $\ell_2$.
Suppose $\ell_1(e) = l$ and $\ell_2(f) = l$, with $e \ne f$.
By hypothesis, $e$ and $f$ are in the same orbit.  
By Theorem~\ref{thm:swap} there is an  elementary swap of $e$ and $f$ in $(T, \ell_1)$ which results in a new labelling $\ell_1'$ that matches $\ell_2$ in one more edge (namely the edge $f$) and still has the property that for every label $l$, the edge with label $l$ in  $\ell_1'$ and the edge with label $l$ in $\ell_2$ are in the same orbit.  
Thus we can continue this process until all edge labels match those of $\ell_2$.
In total we use $O(n)$ elementary swaps. These can be realized via a sequence of $O(n^7)$ flips by Lemma~\ref{lemma:elem-swap}.   Furthermore, the sequence can be found in polynomial time.  
\end{proof}

%\changed{Should we mention the corollary that the group of label permutations of $\cal T$ that can be realized by flips is a product of symmetric groups, one for each orbit (restricted to edges of $\cal T$)?}

%%%%%%%%%%%%%%%%%%%%%%%%%%%%%%%%%%%%%%%%%%%
We can now prove the Orbit Theorem.

\begin{proof}[Proof of Theorem~\ref{thm:orbit}]
The necessity of the condition is clear, and we can test it in polynomial time by finding all the orbits, so we address sufficiency.
The idea
%\comment{(from~\cite{bose2013flipping})}
 is to reconfigure ${\cal T}_1$ to have the same underlying unlabelled triangulation as ${\cal T}_2$ and then apply the previous theorem.  The details are as follows. 
Let ${\cal T}_1 = (T_1, \ell_1)$ and ${\cal T}_2 = (T_2, \ell_2)$. 
There is a sequence $\sigma$ of $O(n^2)$ flips to reconfigure the unlabelled triangulation $T_1$ to $T_2$, and $\sigma$ can be found in polynomial time.
Applying $\sigma$ to the labelled triangulation ${\cal T}_1$ yields a labelled triangulation ${\cal T}_3 = (T_2, \ell_3)$.  
Note that for every label $l$, the edges of ${\cal T}_1$ and ${\cal T}_3$ having label $l$ belong to the same orbit.
This is because flips preserve orbits (by definition of orbits).
Thus by Theorem~\ref{thm:strong-permutation} there is a flip sequence $\tau$ that reconfigures ${\cal T}_3$ to ${\cal T}_2$, and this flip sequence can be found in polynomial time and has length $O(n^7)$.
The concatenation of the two flip sequences, $\sigma \tau$, reconfigures ${\cal T}_1$ to ${\cal T}_2$, has length $O(n^7)$, and can be found in polynomial time.  
\end{proof}
