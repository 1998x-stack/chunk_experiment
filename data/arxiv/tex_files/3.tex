\section{Introduction}
\label{sec:Introduction}
 
Many of the choices that drive hardware and software component designs in emerging extreme-scale high-performance computing (HPC) systems are made to deliver maximum application performance, but are also subject to the constraints of cost, power and reliability. While HPC system architectures have evolved significantly over the past decade, these constraints are expected to force further dramatic changes to the system stack to achieve exascale performance.  
Recent system architectures have emphasized increasing on-chip and node-level parallelism in addition to complex memory architectures consisting of deeper hierarchies and diverse technologies \cite{Shalf:2010}. The software infrastructure, including the system software, middleware and tools, has continued to evolve to keep up with these changes to the system architectures to drive application performance on these extreme-scale computers.  

The reliability and availability of the recent generation of HPC systems have been degrading in comparison to their predecessors \cite{Geist:2016}. This trend is projected to cause future extreme-scale systems to experience unprecedented rates of faults, which will make it difficult to accomplish productive work. The increasingly complex, multicomponent hardware and software environment only makes the challenge of detection of faults in a timely manner, containment of error propagation and mitigation of the impact of error and failure events more difficult.
Resilience solutions must protect the correctness of HPC applications in the presence of faults, errors and failures arising from a multitude of sources, including the system environment, the interactions between platform hardware and system software components and applications, and variability in behavior of hardware components, while seeking to limit the performance and power overhead they impose on the system.  

To navigate the complexities of this emerging landscape of HPC design, we proposed a structured approach to designing HPC resilience solutions based on the concept of design patterns \cite{Hukerikar:2017}. In general, a design pattern is a general reusable solution to a commonly occurring problem within a given context in any design discipline. A pattern provides a description or template for how to solve a problem that may be adapted to specific context. Resilience patterns describe solutions to confront faults and their consequences. The patterns describe techniques for detection, containment and mitigation of faults, errors and failure events. They can be instantiated at any layer of the system stack. The resilience design patterns serve as building block elements for designing complete solutions, and are useful for the exploration of design alternatives for a target HPC system environment and application workload. Section \ref{sec:Background} describes the concept of patterns and summarizes the different types of resilience patterns that are organized in a catalog.  

The development of resilience solutions through composition of various design patterns lends structure to the design and implementation process by compelling designers to consider the key issues of protection coverage, fault model, handling capability, etc. However, objectively selecting pattern solutions that have been examined and utilized successfully in a specific context with the intention of adapting them to a new architecture or software environment of a future system requires criteria based on a quantitative foundation. 
Mathematical models of hardware or software components, or even entire HPC systems, which are solved either analytically or through discrete event simulation, are useful to HPC designers for predicting resilient behavior of the system in the presence of various fault, error and failure events, without having to build the component or system. 
This paper develops models for analytical evaluation of reliability and performance measures of the various resilience design patterns in our pattern catalog. These models are presented in Section \ref{sec:Models}. 
The models are designed to capture the interaction between the resilient behavior and the performance overhead incurred by instantiating a specific pattern. 
Section \ref{sec:Evaluation} discusses approaches to calculate reliability and performance of a solution built by combining several patterns.  


