\section{Evaluation}
\label{sec-eval}

The skeleton-based grasp planning approach is evaluated by applying the algorithm on a wide variety of objects of the Yale-CMU-Berkeley (YCB) Object and Model Set \cite{Calli2015} and the SecondHands subset of the KIT object model database \cite{Kasper12}. All object models have been generated by processing point cloud scans of real-world objects. By using such realistic object models, we show that the grasp planning approach can be used under real-world conditions, i.e. without depending on perfectly modeled object meshes.

We compare the results of our approach with a randomized grasp planner \cite{Vahrenkamp12b} which generates power grasps by aligning the approach direction of the end effector to surface normals and evaluates the success by closing the fingers, determining the contacts and evaluating force closure and the grasp quality with the grasp wrench space approach. The planner generates similar results as the approach used in \cite{Diankov2010} and \cite{Kappler2015}.

In total, we use 83 objects, 69 from the YCB data set (we exclude all objects with degenerated meshes) and 14 from the SecondHands subset of the KIT object database.
The evaluation is performed with three different robot hand models: The ARMAR-III hand, the Schunk Dexterous Hand, and the Shadow Dexterous Hand (see \autoref{fig:results-hands}). For all hands we define a precision and power preshape with corresponding GCP information. In total, we generated around 4000 valid grasps for each hand.

\begin{figure*}[t!]%
\centering
\includegraphics[width=0.75\textwidth]{fig/grasp-planning/grasps-hands.png}
\caption{Excerpt of the generated grasps for the three hand models that were used in the evaluation.}%
\label{fig:results-hands}%
\end{figure*}

\subsection{Efficiency and Force-Closure Rate}

The first evaluation concerns the performance of the developed approach in terms of efficiency and force closure rate. In \autoref{tab:eval-time-fc}, the results of the reference planning approach and the skeleton-based grasp planner are shown for the ARMAR-III hand, the Schunk Dexterous Hand, and the Shadow Dexterous Hand. 
The table shows the measured mean values together with the standard deviation over all grasps that were generated on all 83 considered objects. 
The time that is needed for generating a valid grasp directly depends on the complexity of the involved 3D models of the hand since many collision and distance calculations need to be performed when closing the fingers for contact detection.
%For this reason the performance of the Schunk Dexterous Hand is lower compared to the other hand models since we use a very detailed mesh model.
%It can be seen, that the skeleton-based grasp planning approach is capable of efficiently generating valid grasps for a wide variety of objects.
The comparison shows that the skeleton-based approach outperforms the surface-based planner in terms of efficiency (time needed to generate a grasping pose) and force-closure rate (number of force-closure grasps in relation to all generated grasps). 

When looking at the resulting grasping poses in \autoref{fig:results-hands}, it can be seen that the planned grasping poses are of high quality in terms of what a human would expect how a robot should grasp an object.
Although we cannot provide any numbers on the \textit{human-likeness} of the generated grasps, the underlying algorithm produces good grasps in this sense since the robot hand is aligned with the structure of the object's shape.

%Since we evaluated these numbers on three different hand models on a wide variety of objects, we can infer that the approach is suitable for automatic grasp generation for robotic hands 


\begin{table}[h!]%
\begin{center}
\begin{tabular}{ |c|c|c|c| } 
 \hline
												& Avg. Time & Force-Closure & Robustness \\ 
												& per Grasp & Rate 			& Score $r$\\ 
 \hline
\multicolumn{4}{|l|}{\textbf{Surface-based Grasp Planner}}\\
 \hline
 ARMAR-III Hand 			& $32.29ms$ 		& $57.80\%$  	& $57.17\%$\\   % niko 
							& $\pm 23.87ms$ 	& $\pm 28.54\%$ & $\pm 16.18\%$\\
 Schunk Dext. Hand 			& $67.43ms$ 		& $70.78\%$  	& $61.94\%$\\   % niko 
							& $\pm 23.22ms$ 	& $\pm 21.72\%$ & $\pm 27.39\%$\\
 Shadow Dext. Hand 			& $90.88ms$ 		& $43.55\%$  	& $55.34\%$\\   % niko
							& $\pm 34.05ms$ 	& $\pm 28.32\%$ & $\pm 18.83\%$\\
 \hline
\multicolumn{4}{|l|}{\textbf{Skeleton-based Grasp Planner}}\\
 \hline
 ARMAR-III Hand 			& $12.19ms$ 		& $95.74\%$  	& $94.23\%$\\   % niko
							& $\pm 3.47ms$ 		& $\pm 11.33\%$ & $\pm 9.52\%$\\
 Schunk Dext. Hand 			& $12.98ms$ 		& $86.69\%$  	& $94.87\%$\\   % niko
							& $\pm 2.72ms$ 		& $\pm 20.50\%$ & $\pm 9.74\%$\\
 Shadow Dext. Hand 			& $28.68ms$ 		& $85.53\%$  	& $76.46\%$\\   % niko
							& $\pm 11.34ms$ 	& $\pm 25.40\%$ & $\pm 20.36\%$\\
\hline
\end{tabular}
\caption{Results of the evaluation.}
%reference approach and the skeleton-based grasp planner. The grasp generation time together with the percentage of valid grasping hypotheses and the overall robustness rate is listed for different robotic hands.}
\label{tab:eval-time-fc}%
\end{center}
\end{table}
\vspace{-1cm}
\subsection{Robustness}

We evaluate the robustness of the generated grasping information by investigating how inaccuracies in hand positioning would affect the success rate of the grasping process. 
Related to the approach in \cite{weisz2012pose}, we compute a robustness score $r$ for each grasp which indicates how many pose variances within a certain distribution result in a force-closure grasp. 
We create erroneous variances of the grasping pose $p$ by applying a random position and orientation offset to the pose of the grasp. As proposed in \cite{weisz2012pose}, the offset is applied w.r.t. the center of all contacts. The resulting pose $p'$ is then evaluated by moving the hand to $p'$, closing the fingers, detecting the contacts and evaluating if the pose would result in a force-closure grasp. If $p'$ results in an initial collision between hand and object, we count this pose as a failed sample, although there might be numerous grasps which could be executed. To get more detailed information, such situations could be further investigated by applying a physical simulation and considering the approach movement.

%If $p'$ results in an initial collision between hand and object, weignore this sample since it is difficult to determine the effect of such poses without a physical simulation environment. 

The robustness score $r$ is then generated by determining the percentage of force-closure grasps of the total number of displaced samples. In all our experiments we draw 100 samples of displaced grasping poses with a normal distributed error (standard deviation: $10$mm and $5$ degree).

In \autoref{fig:eval-armar3-robustness}, a histogram of the robustness scores for all generated grasps on all considered objects for the hand of ARMAR-III is shown. The histogram bins on the x axis cover 5\% each and the y axis indicates the absolute percentage of each histogram bin. 
The same histogram is shown in \autoref{fig:eval-schunk-robustness} and in \autoref{fig:eval-shadow-robustness} for the Schunk Dexterous Hand and the Shadow Dexterous Hand respectively.
%It can be seen that the skeleton-based grasp planner produces high quality grasps which are robust to positioning errors. 
It can be seen, that the majority of the planned grasps of the skeleton-based approach are robust to disturbances. 

As shown in \autoref{tab:eval-time-fc} in the right column, the robustness score $r$ could be considerably increased for all three hand models when using the skeleton-based grasp planning approach.
It is above $94\%$ for the ARMAR-III and the Schunk Dexterous Hand, which means that $94\%$ of the investigated ill-positioned grasps were leading to force-closure configurations. 
The value for the Shadow Dexterous Hand is lower, which seems to be mainly caused by the fact that the hand kinematics is more complex and the corresponding preshapes provide less room for re-positioning.

The execution of these high-quality grasps with a real robot manipulator would result in higher grasping performance since inaccuracies in perception and/or gripper positioning would still result in a successful grasp.




%grasp planner produces high quality grasps which are robust to positioning errors. 

\begin{figure}%
\begin{tikzpicture} 
\begin{axis}[ 
ybar, 
height=5 cm, 
bar width=0.1cm, % bar width
x = 0.38cm,	% space between bars
enlarge x limits={abs=0.3cm}, % center between first bar and legend
ymin=0,ymax=0.9,
%legend style={at={(0.5,-0.2)}, anchor=north,legend columns=-1}, 
%ylabel={percentage}, 
symbolic x coords={5,10,15,20,25,30,35,40,45,50,55,60,65,70,75,80,85,90,95,100},
xticklabel=$\pgfmathprintnumber{\tick}$\,\%, 
%xtick=data, 
xtick={5, 25, 50, 75, 100},
tick label style={/pgf/number format/fixed}
%nodes near coords, 
%nodes near coords align={vertical}, 
%x tick label style={},%rotate=45,anchor=east}, 
] 
\addplot[red,fill={red!30!white}] table[col sep=comma,header=false] {./data/plots/niko/armar3-generic-robustness-with-collision.csv}; 
\addplot[blue!80!white,fill={blue!30!white}] table[col sep=comma,header=false] {./data/plots/niko/armar3-robustness-with-collision.csv}; 
\node[] (image) at (rel axis cs:0.2,0.71) {\includegraphics[width=2cm]{./fig/grasp-planning/gcp_power.png}};
\end{axis} 
\end{tikzpicture}
\caption{Robustness histograms for the reference grasp planning approach (red) and the skeleton-based grasp planner (blue) with the ARMAR-III hand. }%
\label{fig:eval-armar3-robustness}%
\end{figure}

\begin{figure}%
\begin{tikzpicture} 
\begin{axis}[ 
ybar,
height=5 cm, 
bar width=0.1cm, % bar width
x = 0.38cm,	% space between bars
enlarge x limits={abs=0.3cm}, % center between first bar and legend
ymin=0,ymax=0.8,
%legend style={at={(0.5,-0.2)}, anchor=north,legend columns=-1}, 
%ylabel={percentage}, 
symbolic x coords={5,10,15,20,25,30,35,40,45,50,55,60,65,70,75,80,85,90,95,100},
xticklabel=$\pgfmathprintnumber{\tick}$\,\%, 
%xtick=data, 
xtick={5, 25, 50, 75, 100},
tick label style={/pgf/number format/fixed}
%nodes near coords, 
%nodes near coords align={vertical}, 
%x tick label style={},%rotate=45,anchor=east}, 
] 
\addplot[red,fill={red!30!white}] table[col sep=comma,header=false] {./data/plots/niko/schunk-generic-robustness-with-collision.csv}; 
\addplot[blue!80!white,fill={blue!30!white}] table[col sep=comma,header=false] {./data/plots/niko/schunk-robustness-with-collision.csv}; 
\node[] (image) at (rel axis cs:0.2,0.75) {\includegraphics[width=2cm]{./fig/sdh2.png}};
\end{axis} 
\end{tikzpicture}
%\caption{Robustness histograms for the reference grasp planning approach (red) and the skeleton-based grasp planner (blue) with the Schunk Dexterous Hand.}%
\caption{Robustness histograms for the Schunk Dexterous Hand.}%
\label{fig:eval-schunk-robustness}%
\end{figure}



\begin{figure}[t!]%
\begin{tikzpicture} 
\begin{axis}[ 
ybar,
height=5 cm, 
bar width=0.1cm, % bar width
x = 0.38cm,	% space between bars
enlarge x limits={abs=0.3cm}, % center between first bar and legend
ymin=0,ymax=0.4,
%legend style={at={(0.5,-0.2)}, anchor=north,legend columns=-1}, 
%ylabel={percentage}, 
symbolic x coords={5,10,15,20,25,30,35,40,45,50,55,60,65,70,75,80,85,90,95,100},
xticklabel=$\pgfmathprintnumber{\tick}$\,\%, 
%xtick=data, 
xtick={5, 25, 50, 75, 100},
tick label style={/pgf/number format/fixed}
%nodes near coords, 
%nodes near coords align={vertical}, 
%x tick label style={},%rotate=45,anchor=east}, 
] 
\addplot[red,fill={red!30!white}] table[col sep=comma,header=false] {./data/plots/niko/shadow-generic-robustness-with-collision.csv}; 
\addplot[blue!80!white,fill={blue!30!white}] table[col sep=comma,header=false] {./data/plots/niko/shadow-robustness-with-collision.csv}; 
\node[] (image) at (rel axis cs:0.2,0.75) {\includegraphics[width=2cm]{./fig/shadowhand2.png}};
\end{axis} 
\end{tikzpicture}
%\caption{Robustness histograms for the reference grasp planning approach (red) and the skeleton-based grasp planner (blue) with the Shadow Dexterous Hand.}%
\caption{Robustness histograms for the Shadow Dexterous Hand.}%
\label{fig:eval-shadow-robustness}%
\end{figure}
