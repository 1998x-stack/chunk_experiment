\section{Preliminaries and Related Work}

Let $G=(V, E)$ be an undirected, unweighted graph with $n = |V|$ vertices
and $m = |E|$ edges.
$G$ can contain self-loops and parallel (duplicate) edges, and is not necessarily connected.
We assume a global ordering of the vertices to break ties when necessary.
If the degree of every vertex is bounded by a constant, we say the graph
is \defn{bounded-degree}.
%
We use standard definitions of \emph{spanning tree}, \emph{spanning
  forest}, \emph{connected component}, \emph{biconnected component},
\emph{articulation points}, \emph{bridge}, and
\emph{$k$-edge-connectivity} on a graph, and
\emph{lowest-common-ancestor} (LCA) query on a tree (as summarized in
Appendix~\ref{sec:appendix-prelim}).
%
Let $[n] = \{1, 2, \cdots, n\}$ where $n$ is a positive integer.

%An event happens with high probability (\whp{}) if it happens with the probability of $1-n^{-\Omega(1)}$.

\myparagraph{Computation models}
Sequential algorithms are analyzed using the \defn{\seqmodel}
model~\cite{blelloch2016efficient}, comprised of an infinitely large
asymmetric memory and a small \local{} memory.  The cost of writing to
the large memory is $\wcost$, and all other operations have unit cost.
This models practical settings in which there is a small amount of
standard symmetric memory (e.g., a cache) in addition to the asymmetric memory.

For parallel algorithms, we use the \defn{\ourmodellong\ (NP)} 
model~\cite{BBFGGMS16}, which is designed to characterize both parallelism
and memory read-write asymmetry. In the model, a
computation is represented as a (dynamically unfolding) directed acyclic graph (DAG) of tasks that
begins and ends with a single task called the root. A task consists of
a sequence of instructions that must be executed in order. Tasks can
also call the Fork instruction, which creates child tasks that can be
run in parallel with each other. The memory in the \ourmodel\ Model
consists of (i) an infinitely large
\emph{asymmetric} memory accessible to all tasks and (ii) a small
task-specific \emph{\local} memory accessible only to a task
and its children.  The cost of writing to large memory is
$\wcost$, and all other operations have unit cost.
%
%The primary metrics for algorithms in the \ourmodel\ Model are work
%and \depth. 
The \defn{work} $W$ of a computation is the sum of the costs of the
operations in its DAG and the \defn{\depth{}} $D$ is the
cost of the DAG's most expensive path.  Under mild assumptions,
Ben-David et al.~\cite{BBFGGMS16} showed that a work-stealing scheduler
can execute an algorithm whose \ourmodel{} complexity is
work $W$ and \depth{} $D$ in $O(W / P + \wcost D)$ expected time
on $P$ processors.

In both models,
%Because the emphasis in this paper is on reducing the number of writes
%to the asymmetric memory, we will often explicitly report a bound on
the number of \defn{writes} refers only to the writes to the
asymmetric memory, ignoring any writes to \local{}
memory.  All reads and writes are to words of size $\Theta(\log n)$
for input size $n$.  The size of the \local{} memory is measured in
words.

\myparagraph{Related Work}
Read-write asymmetries have been studied in the context of NAND Flash
chips~\cite{BT06, Eppstein14, Gal05, ParkS09}, focusing on how to
balance the writes across the chip to avoid uneven wear-out of
locations.  Targeting instead the new memory technologies, read-write
asymmetries have been an active area of research in the
systems/database/architecture communities (e.g., \cite{Arulraj17sigmod,
  Bausch2012damon, Chen11, Chen2015pvldb, ChoL09, LeeIMB09,
  Oukid2016sigmod, Viglas12, Viglas14, wang2013wade, yang:iscas07,
  zhou2012writeback, ZhouZYZ09, ZWT13}).  In the algorithms community,
Blelloch et al.~\cite{BFGGS15} defined several sequential and parallel
computation models that take asymmetric read-write costs into account,
and analyzed and designed sorting algorithms under these models.
Their follow-up paper~\cite{blelloch2016efficient} presented
sequential algorithms for various problems that do better than their
classic counterparts under asymmetric read-write costs, as well as
several lower bounds.  Carson et al.~\cite{carson2016write} presented
write-efficient sequential algorithms for a similar model, as well as
write-efficient parallel algorithms (and lower bounds) on a
distributed memory model with asymmetric read-write costs, focusing on
linear algebra problems and direct N-body methods.  Ben-David et
al.~\cite{BBFGGMS16} proposed a nested-parallel model with asymmetric
read-write costs and presented write-efficient, work-efficient, low
\depth{} (span) parallel algorithms for reduce, list contraction, tree
contraction, breadth-first search, ordered filter, and planar convex
hull, as well as a write-efficient, low-\depth{} minimum spanning tree
algorithm that is nearly work-efficient.  Jacob and
Sitchinava~\cite{jacob2017} showed lower bounds for an asymmetric
external memory model.  In each of these models, there is a small
amount of \local{} memory that can be used to help minimize the number
of writes to the large asymmetric memory.

Although graph decompositions with various properties have been shown
to be quite useful in a large variety of applications (e.g.,
\cite{abraham2012using, awerbuch1985complexity, awerbuch1992low, awerbuch1989network, blelloch2014nearly, linial1991decomposing, miller2013parallel}),
to our knowledge none of the prior
algorithms provide the necessary conditions for processing graphs
with a sublinear number of writes in order to answer connectivity/biconnectivity
queries (targeting instead other decomposition properties
that are unnecessary in our setting, such as few edges between
clusters).  For example, Miller et al.'s~\cite{miller2013parallel}
parallel low-diameter decomposition algorithm requires at least
$\Omega(n)$ writes (even if a write-efficient BFS~\cite{BBFGGMS16} is
used), and provides no guarantees on the partition sizes.  Similarly,
algorithms for size-balanced graph partitioning (e.g.,
\cite{andreev2004balanced}) require $\Omega(n)$ writes.  Our
\implicit{} construction is reminiscient of sublinear time algorithms
for estimating the number of connected
components~\cite{Berenbrink2014ipl, chazelle2005siam} in its use of
BFS from a sample of the vertices.  However, their BFS is used for a
completely different purpose (approximate counting of $1/n_u$, the
inverse of the size of the connected component containing a sampled
node $u$), does not provide a partitioning of the nodes into clusters
(two BFS from different nodes can overlap), and cannot be used for
connectivity or biconnectivity queries (two BFS from the same
connected component may be truncated before intersecting).


