\section{The EASIER detection setup, GIGAS61 and GIGADuck detectors}
\label{sec:setup}
EASIER is  a novel radio-detector  concept composed of a  radio sensor
and of  an envelope  detection electronics embedded  in the SD  of the
Pierre  Auger  Observatory.  This  concept  was  implemented in  three
bandwidths: the VHF  band (30-80~MHz), the L band  (1-1.5~GHz) and the
C-band (3.4-4.2~GHz).  We  focus in this article on  the L- and C-band
only. The EASIER  experiment is one of the  three experiments deployed
at the Pierre  Auger Observatory to search for the  MBR emitted by the
ionization electrons left  in the atmosphere after the  passage of the
shower. In  contrast to the two other  ones, namely AMBER~\cite{amber}
and  MIDAS~\cite{midas},  which  instrument  an  array  of  feed  horn
antennas  illuminated  by  a  parabolic  dish, EASIER  relies  on  the
observation  of the shower  from the  ground level  with a  wide angle
antenna  pointing directly  to the  sky.  In  2011, a  first set  of 7
antennas was  deployed, followed by  54 additional in 2012  making the
\mbox{GIGAS61}  array.   The  analysis  of  \mbox{GIGAS61}  data  has  revealed  the
observation    of   radio    signals   emitted    by   EAS    in   the
C-band\cite{amber}.  However,  such detection  occurred  only for  air
showers at  distances less than around \unit[200]{m}  from the \mbox{GIGAS61}
antenna and could  be also explained by other  emission processes than
the   MBR.   Furthermore,   new  estimations   of  the   expected  MBR
intensity~\cite{imen2015,imen2016} led  to the development  of two new
versions of  EASIER, called \mbox{GIGADuck-C}  (installed in March  2015) and
GIGADuck-L (installed in December  2016), with an enhanced sensitivity
to search for signal from larger distances hence fainter.

\subsection{The electromagnetic background at the Pierre Auger Observatory}
Radio measurement are often hindered  by man-made noise.  Prior to the
installation, the electromagnetic background  was measured on the site
of the Pierre  Auger Observatory located in the  Pampa Amarilla in the
province of  Mendoza in Argentina.   Figure~\ref{fig:Pampa_spec} shows
the  power spectrum between  \unit[2.6 and  4.6]{GHz} measured  with a
C-band LNBf (Low Noise Block feed).  The gain of the amplifier used is
roughly \unit[60]{dB} between \unit[3.4 and 4.2]{GHz}. With a recorded
power in the sensitive  bandwidth of \mbox{\unit[-55]{dBm / 3MHz}} the
noise  floor is  thus about  \unit[-180]{dBm/Hz}.  No  strong  peak is
observed above this level in  the tens of recorded spectra making this
band   adequate   for   our   experiment.\\   In   the   L-band   (see
Figure~\ref{fig:Pampa_spec}  (right)),   peaks  can  be   noticed,  in
particular  around \unit[900]{MHz}  where a  strong  intermittent peak
could  be observed.   It originates  from the  Auger  SD communication
system  and  the  mobile  phone  band  and  can  be  reduced  with  an
appropriate  filtering.   Other  peaks  are also  present  inside  the
frequency  band of interest  between \unit[1  and 1.4]{GHz}  but their
amplitude remains acceptable.
\begin{figure}[!t]
  \center {
    \subfigure{\includegraphics[width=0.49\linewidth]{Pampa_spec}}
    \subfigure{\includegraphics[width=0.49\linewidth]{Lbandspectrum.pdf}}
    \caption{\small{Frequency spectra in  Pampa Amarilla in the C-band
        recorded (left)  and in the  L-band (right). The  band between
        the  dashed line  is the  frequency  band once  the filter  is
        applied.}
      \label{fig:Pampa_spec}
    }
}
\end{figure}


\subsection{The experimental setup}  
The EASIER detector is embedded in a sub-array of the surface detector
(SD) of the Pierre Auger Observatory. The SD is composed of 1660 water
Cherenkov  detectors (WCD)  arranged in  a triangular  grid  of 1500~m
spacing. Each WCD is equipped with three Photo Multiplier Tubes (PMT),
a     local     acquisition     and    a     communication     system,
see~\cite{augerdetector}  for  a   detailed  description.   An  EASIER
detector unit is designed to be  integrated into a WCD. It is composed
radio sensor installed on top of the SD station and an electronics box
located  below  the  hatch  box  on  top  of  the  SD  electronics(see
Fig.~\ref{fig:easierscheme}). The three setups described in this paper
share  common elements presented  rightafter, their  specificities are
addressed in the following paragraphs.\\ The sensor is an antenna with
a main  lobe of  30 to 45$^\circ$  depending on the  considered setup.
Since the  expected radiation is unpolarized, there  is no requirement
on the polarization type of the antenna.  The sensor is followed by an
amplification and  a filtering stage.   The radio frequency  signal is
then  transformed into  a power  envelope by  a  logarithmic amplifier
(Analog Device  AD8318) which delivers  a voltage proportional  to the
logarithm of the RF input power.   This model was chosen for its large
frequency bandwidth  and its  fast time response  of a  few \unit[tens
  of]{ns}.  The output voltage is in  turn adapted to the front end of
the WCD electronics which is originally built to accept PMT's negative
voltage   between  0  and   -2V  (see   Fig.~\ref{fig:diagram}).   The
adaptation is performed through an amplification that sets the dynamic
range  to \unit[20]{dB}  and an  offset  used to  adjust the  baseline
level.  The EASIER analogic signal replaces one of the six channels of
the  WCD front  end electronics.   The final  part of  the acquisition
includes   an   antialiasing    filter   cutting   frequencies   above
\unit[20]{MHz}  and  the  FADC  (Flash Analog  to  Digital  Convertor)
digitizer.  The recorded waveform is \unit[19.2]{$\mu$s} long acquired
with a \unit[40]{MS/s} rate and has an amplitude sampled over 1024 ADC
units (refereed  as ADCu in  the following)~\cite{augerdetector}.  The
data  stream  is  then  sent   to  the  central  acquisition  and  the
reconstruction  of the  EAS event  is performed  independently  of the
radio signals.   As a consequence,  no separate trigger for  the radio
signal is needed and the EASIER data are simply part of the regular SD
data stream.  As an additional  benefit, the radio detector is powered
by  the station battery  and is  also integrated  into the  SD station
monitoring system.
\begin{figure}[!ht]
  \centering
  \hspace*{-3ex}  
\subfigure{\includegraphics[width=0.59\linewidth]{blockdiagram.pdf}}  
  \caption{Block diagram of an EASIER detector unit.}
 \label{fig:diagram}  
\end{figure}
\begin{figure}[!ht]
  \centering
  \hspace*{-3ex}  
    \subfigure{\includegraphics[width=0.69\linewidth]{schemeeasier.pdf}}  
  \caption{EASIER general  scheme. One of the three  antennas shown on
    the left  hand side is installed on  a pole that sits  on the WCD.
    The antenna  is vertical  in the case  of \mbox{GIGAS61}  and 6 out  of 7
    antennas are  tilted by  an angle $\alpha$  = 20$^{\circ}$  in the
    GIGADuck hexagon design (see text). The RF signal is amplified and
    transferred to the GIGAS box to be transformed in its envelope and
    acquired in the SD acquisition.}
 \label{fig:easierscheme}  
\end{figure}
\subsubsection*{GIGAS61}
The    \mbox{GIGAS61}    antenna     is    a    commercial    horn    antenna
(Fig.~\ref{fig:easierscheme}) made of a cylindrical feed and a quarter
wave length monopole at its  bottom. The metallic ring around the feed
reduces the backlobe and widens the main lobe.  A hemispherical radome
is glued  to the ring to  protect the antenna from  rain.  The antenna
has a gain  of around \unit[9]{dB}. It points to the  Zenith and has a
half-power beamwidth (HPBW)  of $90^\circ$.  It is tuned  at a central
frequency of 3.8~GHz and a bandwidth of approximately \unit[500]{MHz}.
It  is associated with  a low  noise block  (LNB) which  amplifies the
signal  by approximately  \unit[60]{dB}  and lowers  down the  central
frequency  to  \unit[1.35]{GHz}.  The  antenna  and  the  LNB will  be
referred to LNBf  hereafter. A bias tee is inserted  after the LNBf to
both distribute the power supply to the LNB and transmit the RF signal
on  a \unit[75]{$\Omega$}  line.   The line  impedance  is adapted  to
\unit[50]{$\Omega$} by  a resistor bridge.  The  low-frequency part of
the spectrum  is filtered out  by a \unit[900]{MHz}  high-pass filter.
The adaptation electronics of \mbox{GIGAS61} is made partly with commercially
available device.  The power detector used, the Minicircuit ZX47-50 is
the encapsulated version of the  Analog Device AD8318. The rest of the
adaptation is carried out with  a custom made board.\\A first array of
seven detectors was installed at the Pierre Auger Observatory in April
2011.  The smooth operation and the results of this first test bed led
to an  extension by  54 more detectors  covering a  total instrumented
surface of  \unit[93]{$\rm km^2$}.   The first seven  LNBf are  of the
model GI301 made by Global Intersat  and the 54 units of the extension
are from WSInternational, model  DMX241.  The \mbox{GIGAS61} array is located
in the South-West part of the Pierre Auger Observatory.  Its footprint
is shown  in Fig.~\ref{fig:detector}-left.  Even if the  MBR signal is
expected unpolarized,  we fixed the polarization of  each antenna. Out
of  the 61 antennas,  33 have  a North-South  polarization, and  28 an
East-West one.\\ Several radio signals  in the C-band were detected in
coincidence  with Auger  EAS events  with  \mbox{GIGAS61}\cite{amber}.  These
detections validated the concept of the coincident radio detection and
were the  first detections  of EAS in  the C-band.  However,  such EAS
emissions in the microwave band  may have a different origin than MBR.
In  particular, the  signals were  detected  at distances  to the  air
shower axis of a few hundred meters only.  This feature is in favor of
the hypothesis of a beamed emission over an isotropic one as origin of
these signals.
\begin{figure}[!t]
  \centering
  \hspace*{-3ex}  
  \subfigure{\includegraphics[width=0.45\linewidth]{layoutGHz.png}}
  \subfigure{\includegraphics[width=0.54\linewidth]{GD.pdf}}
  \caption{Left: GIGAS61, and GIGADuck arrays layout within the Pierre
    Auger SD.  Middle and right: Top  view of GIGADuck  array and side
    view of one detector.}
  \label{fig:detector}  
\end{figure}

\subsubsection*{GIGADuck}
The need to  improve our sensitivity at large  distance and to collect
more data led to the  design and installation of two optimized arrays,
in  the C-band  and  the L-band,  with  a higher  antenna  gain and  a
modified antenna orientation.  The array  is now composed of a central
detector pointing  to the Zenith with six  peripheral detectors tilted
by  20$\rm ^{\circ} $  in zenith  and with  azimuth adjusted  to point
towards  the   central  detector  (Fig.~\ref{fig:detector}-middle  and
right).   This configuration  increases the  overlap of  the detectors
field  of view  and enhance  the  probability to  obtain a  coincident
detection.   Indeed,   this  configuration  was   chosen  because  the
observation of a coincidence between two radio detectors would support
the  hypothesis of  an isotropic  emission.   \\As an  example of  the
improved performance  of GIGADuck, the simulation of  the radio signal
power produced by MBR emitted by  a vertical shower and detected by an
antenna  belonging  either  to  \mbox{GIGAS61}  or to  GIGADuck  at  a
distance of  \unit[750]{m} is shown  in Fig.~\ref{fig:simexample}.  In
the case of  this particular configuration of distance  and angle, the
signal collected by the tilted \mbox{GIGADuck-C} antenna is around ten
times larger, due  mainly to the higher gain and  the direction of the
main lobe.  In  the L-band, the signal is  increased by another factor
10,  due to  the quadratic  dependence of  the effective  area  of the
antenna  with  the   wavelength  (see  Eq.~\eqref{eq:aeff}).   Further
comparisons  are  shown   in  the  section~\ref{sec:simulation},  they
include calibrated value of  detector noise and realistic distribution
of the shower energies and arrival direction.

\begin{figure}[!t]
  \centering
  \hspace*{-3ex}  
  \subfigure{\includegraphics[width=0.65\linewidth]{simulationexample.png}}
  \caption{Simulation  of the power  received from  a vertical  10 EeV
    shower  as a  function  of time  for  GIGAS and  \mbox{GIGADuck-C}
    detectors.}
  \label{fig:simexample}  
\end{figure}

\paragraph{GIGADuck-C} 
In the C-band,  the antenna is a pyramidal  horn of \unit[15]{dB} gain
from the  A-Info company. It  increases the maximum  antenna effective
area by a  factor of six with respect to the  antennas of \mbox{GIGAS61}.  It
is protected  by a  thin radome  in plexiglass.  The  LNB is  a Norsat
8115F.   It was  chosen for  its low  noise figure  and has  a flatter
response in frequency with respect to \mbox{GIGAS61} horns, and thus a larger
bandwidth.

\paragraph{GIGADuck-L}
In  the L-band,  the sensor  is a  helicoidal antenna  with  a conical
copper grid  at its  base. It is  tuned to  be sensitive to  a central
frequency of  \unit[1.4]{GHz} and a gain of  \unit[15]{dB}. The sensor
is directly followed  by an electric surge protection  and a band-pass
filter (from \unit[1.1 to 1.4]{GHz})  to decrease the amplitude of the
signal  at  \unit[900]{MHz}  caused  by   the  GSM  band  and  the  SD
communication  system.  The  choice  of placing  a  filter before  the
amplifier is not optimal in terms  of noise figure but is necessary to
prevent the amplifier saturation.  The amplification board is composed
of two separated  amplifiers from the Avago company,  the MGA633P8 and
the MGA13116.  They are combined to  obtain a gain of around 50~dB. In
both GIGADuck  versions the adaptation  electronics was made  on custom
made board with discrete components. 
