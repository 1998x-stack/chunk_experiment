\section{Related Work}
\label{sec-realted-work}

Napier divided hand movements in humans into \textit{prehensive movements} and \textit{non-prehensive movements} to distinguish between suitable and non-suitable grasping movements \cite{napier1956prehensile}. 
He showed that prehensive grasping movements of the hand consist of two basic patterns which he termed precision grasp and power grasp. 
Based on this work Cutkosky developed a grasp taxonomy which distinguishes between 16 different grasping types \cite{cutkosky1989}. 
The transfer of grasping strategies in humans to robotic applications is usually done by reducing the complexity, e.g. by considering a low number of grasp types (i.e. power and precision grasp) or by using the Eigengrasp approach which is capable of approximating human grasping movements with low degree of freedom spaces \cite{CiocarlieA09}.

Approaches for grasp synthesis in the literature are usually divided into analytical and empirical or data-driven algorithms \cite{bohg2014, sahbani2012overview}.
Analytical approaches are based on geometrical properties and/or kinematic or dynamic formulations, whereas data driven approaches rely on simulation. The complexity of the problem is often reduced by utilizing randomized methods and by considering simplified contact models.
Data-driven approaches also make use of simulation environments such as \textit{Graspit!}~\cite{Miller2004}, OpenRave~\cite{Diankov2010}, and Simox~\cite{Vahrenkamp12b} to generate and evaluate grasping poses. Many works generate grasping hypotheses and evaluate them by applying the Grasp Wrench Space approach which allows to determine if a grasp is \textit{force-closure} (i.e. valid) and to compute $\epsilon$ or \textit{Volume} quality values indicating how much the grasping configuration can resist to external forces \cite{ferrari1992planning, Roa2015}.

The execution of such grasps with real robot hands is challenging since small disturbances, as they appear during real-world execution, could quickly lead to unstable grasping configurations as shown in \cite{weisz2012pose}. It has been shown that such potential inaccuracies in execution can be considered during grasp planning by analyzing small variations of the generated grasping pose.

In part-based grasp planning there exists several approaches which consider parts of the object for grasp synthesis.
In \cite{miller2003} the object is approximated by a set of primitive shapes (box, cylinder, sphere, cone) to which manually defined hand configurations can be assigned. 
A similar approach has been presented in \cite{goldfeder2007}. The object is decomposed in superquadrics and several heuristics are used to generate the grasping information. 
In \cite{huebner2008} the object is decomposed based on Minimal Volume Bounding Boxes. grasping information is synthesised on the resulting bounding boxes. An extension of this work is presented in \cite{geidenstam2009} where neural networks are used to combine 3D and 2D grasping strategies. 
The grasp planner presented in \cite{aleotti2010grasp} is operating on a topological object segmentation which is computed based on the Reeb Graph formalism. The resulting object segments are used for randomized grasp generation. 
Independent Contact Regions can be used to determine surface areas that are suitable for grasping \cite{Roa07}.
An voxelized object representation is used by \cite{Song2016} to plan feasible grasps via hand-object geometric fitting strategies. 

Our previous work described in \cite{pryb2010, przybylski2012} is related to the work we present here since it uses the medial axis transform as a simplified object representation on which grasp planning is performed. The medial axis representation is computed on point clouds or mesh data and provides information about the object structure and object symmetries. Before grasp planning can be performed, the medial axis transform is processed via cluster algorithms and transferred to grid-based data structures on which grasp synthesis is realized based on several heuristics for typical grid structures. In contrast to \cite{przybylski2012}, we use mean curvature skeletons \cite{tagliasacchi2012mean} to represent the object structure.  Compared to the medial axis approach, this representation results in a reduced complexity while preserving full object surface information (see Section~\ref{sec-object-skeleton}). We exploit the skeleton structure for object segmentation and in Section~\ref{sec-planning} we show how the object segments can be analyzed for part-based grasp planning. The approach is evaluated in Section~\ref{sec-eval} on several object data sets by investigating the force closure rate and the robustness of the generated grasps according to \cite{weisz2012pose}. In addition, we compare the results to a randomized grasp planning algorithm which aligns the approach direction to the object surface \cite{Vahrenkamp12b}, similar to the approach used in \cite{Diankov2010} and \cite{Kappler2015}.



