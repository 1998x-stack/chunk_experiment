%section 1


\section{Preliminaries}

\subsection{Elliptic surfaces}

\subsubsection{Generalities}

As for basic references about elliptic surfaces, we refer  to \cite{kodaira, miranda-basic}.  We also
refers to \cite{shioda90} for general facts on the Mordell Weil lattices. In particular, 
for those on rational elliptic surfaces, we refer to \cite{oguiso-shioda}.
We also use the notation and terminology used in \cite{bannai-tokunaga, tokunaga14} freely.
In this article, by an {\it elliptic surface}, we always mean a smooth projective surface $S$ 
  with a fibration $\varphi : S \to C$ over a smooth projective curve $C$ as follows:
 \begin{enumerate}
 
  \item[(i)]  $\varphi$ has a section
 $O : C \to S$ (we identify $O$ with its image). 
  \item[(ii)] There exists a non-empty finite subset, $\Sing(\varphi)$, of $C$ such
  that $\varphi^{-1}(v)$ is a smooth curve of genus $1$ (resp. a singular curve ) for $v \in C\setminus \Sing(\varphi)$ (resp. $v \in \Sing(\varphi))$. Note that there exist no multiple fibers since $\varphi$ has the section $O$.

 \item[(iii)]  $\varphi$ is minimal, i.e., there is no exceptional
 curve of the first kind in any fiber. 
 \end{enumerate}
 
 For $v \in \Sing(\varphi)$, we put $F_v = \varphi^{-1}(v)$. 
 We denote its irreducible decomposition by 
 \[
 F_v = \Theta_{v, 0} + \sum_{i=1}^{m_v-1} a_{v,i}\Theta_{v,i}, 
 \]
 where $m_v$ is the number of irreducible components of $F_v$ and $\Theta_{v,0}$ is the
unique irreducible component with $\Theta_{v,0}O = 1$. We call $\Theta_{v,0}$ the {\it identity
 component}.  The classification  of singular fibers is well-known (\cite{kodaira}). We use the Kodaira Notation to denote the types of singular fibers. 
  We also denote the subset  of $\Sing(\varphi)$  consisting of points giving reducible singular fibers by
 $\Red(\varphi) := \{v \in \Sing(\varphi)\mid \mbox{$F_v$ is reducible}\}$. 
Let $\MW(S)$ be the set of sections of $\varphi : S \to C$.  $\MW(S) \neq \emptyset$ as $O \in \MW(S)$.
By \cite[Theorem 9.1]{kodaira}, $\MW(S)$  is an abelian group with the zero
element $O$. We call
$\MW(S)$  the Mordell-Weil group.   
We also denote the multiplication-by-$m$ map ($m \in \ZZ$) on
$\MW(S)$ by $[m]s$ for $s \in \MW(S)$. 
 On the other hand, the generic fiber $E:= S_{\eta}$ of
$S$ is as a curve of genus $1$ over $\CC(C)$, the rational function field of $C$. The restriction of $O$
to $E$ gives rise to a ${\mathbb C}(C)$-rational point of $E$, and one can regard $E$
as an elliptic curve over ${\mathbb C}(C)$, $O$ being the zero element. $\MW(S)$ can be identified
with the set of ${\mathbb C}(C)$-rational points $E(\CC(C))$ canonically. 
For $s \in \MW(S)$, we denote the corresponding rational point by $P_s$. 
Conversely,
for an element $P \in E(\CC(C))$,  we denote the corresponding  section by $s_P$.
  The abelian group $G_{\Sing(\varphi)}$, and the homomorphsim
$\gamma : \MW(S) \to G_{\Sing(\varphi)}$   are those defined in \cite[p. 83]{tokunaga12}.
For $s \in \MW(S)$, $\gamma(s)$ describes at which irreducible component $s$ meets on 
$F_v$. 


%
Let $\NS(S)$ be the N\'eron-Severi group of $S$ and let $T_{\varphi}$ be the 
subgroup of $\NS(S)$ generated by $O$, a fiber $F$ and  $\Theta_{v,i}$ $(v \in \Red(\varphi)$,
$1 \le i \le m_v-1)$. Then we have the following theorems:

\begin{thm}\label{thm:shioda-basic0}(\cite[Theorem~1.2, 1.3]{shioda90})
 {Under our assumptions,

\begin{enumerate}
\item[(i)] $\NS(S)$ is torsion free, and

\item[(ii)] there is a natural map $\tilde{\psi} : \NS(S) \to \MW(S)$ which induces an isomorphism of 
groups
\[
\psi : \NS(S)/T_{\varphi} \cong \MW(S) (\cong E(\CC(C))).
\]
In particular, $\MW(S)$ is a finitely generated abelian group.
\end{enumerate}
}
\end{thm}

\begin{thm}\label{thm:shioda-basic}{(\cite[Theorem~1.3]{shioda90}) Under our assumptions,
there is a natural map $\tilde{\psi} : \NS(S) \to \MW(S)$ which induces an isomorphism of 
groups
\[
\psi : \NS(S)/T_{\varphi} \cong \MW(S).
\]
In particular, $\MW(S)$ is a finitely generated abelian group.
}
\end{thm}

For a divisor on $S$, we  put $s(D) = \tilde\psi(D)$. 

In \cite{shioda90}, a lattice structure on $E(\CC(C))/E(\CC(C))_{\tor}$ is defined by using
the intersection pairing on $S$ through $\psi$. We use the terminologies, notation and results in \cite{shioda90}, freely.
In particular, $\langle \, , \, \rangle$ denotes the height pairing and $\mbox{\rm Contr}_v$ denotes the contribution term
given in \cite{shioda90} in order to compute $\langle \, , \, \rangle$.


%Then we have
%
%\begin{lem}\label{lem:fund-relation}{(\cite[Lemma~5.1]{shioda90})
%$D$ is uniquely written in the form:
%\[
%D \approx s(D) + (d-1)O + nF + \sum_{v\in \Red(\varphi)}\sum_{i=1}^{m_v-1}b_{v,i}\Theta_{v,i},
%\]
%where $\approx$ denotes the algebraic equivalence of divisors, and $d, n$ and $b_{v,i}$
%are integers defined as follows:
%\[
%d = DF \qquad n = (d-1)\chi({\mathcal O}_S) + OD - s(D)O,
%\]
%and
%\[
%\left [ \begin{array}{c}
%          b_{v,1} \\
%          \vdots \\
%          b_{v, m_v-1} \end{array} \right ] = A_v^{-1}\left [\begin{array}{c}
%                                                                          D\Theta_{v,1} - s(D)\Theta_{v,1} \\
%                                                                          \vdots \\
%                                                                      D\Theta_{v,m_v-1} - s(D)\Theta_{v, m_v-1}
%                                                                      \end{array} \right ].
%\]
%Here $A_v$ is the intersection matrix $(\Theta_{v,i}\Theta_{v, j})_{1\le i, j \le m_v-1}$.  
%}
%\end{lem}
%
%For a proof, see \cite{shioda90}.
%
%\medskip
%
% Also,  in \cite{shioda90},  a $\QQ$-valued bilinear form $\langle \, , \, \rangle$ on
% $\MW(S)$  is defined by using the intersection pairing on $\NS(S)$.  
% As for an explicit formula for $\langle s_1,
%s_2\rangle$ ($s_1, s_2 \in \MW(S)$, see \cite[Theorem 8.6]{shioda90}. Note that
%$\langle s_1, s_2\rangle$ is given 
% Here are two basic properties of $\langle \, , \, \rangle$:
%\begin{itemize}
%
%\item $\langle s, \, s \rangle \ge 0$ for $\forall s \in \MW(S)$ and the equality holds if and 
%only if $s$ is an element of finite order in $\MW(S)$. 

%\item 
%
%is given as follows:
%\[
%\langle s_1, s_2 \rangle = \chi({\mathcal O}_S) + s_1O + s_2O - s_1s_2 - \sum_{v \in \Red(\varphi)}
%\mbox{Contr}_v(s_1, s_2),
%\]
%where $\mbox{Contr}_v(s_1, s_2)$ is given by
%\[
%\mbox{Contr}_v(s_1, s_2) = (s_1\Theta_{v,1}, \ldots, s_1\Theta_{v, m_v-1})(-A_v)^{-1}
%\left ( \begin{array}{c}
%        s_2\Theta_{v,1} \\
%        \vdots \\
%        s_2\Theta_{v, m_v-1}
%        \end{array} \right ).
%\]
% As for explicit values of 
%$\mbox{Contr}_v(s_1, s_2)$, we refer to \cite[(8.16)]{shioda90}.
%\item Let $\MW(S)^0$ be the subgroup of $\MW(S)$ given by 
%\[
%\MW(S)^0 := \{ s \in \MW(S) \mid \mbox{$s$ meets $\Theta_{v, 0}$ for $\forall v \in \Red(\varphi)$}\}.
%\]
%$\MW(S)^0$ is called the {\it narrow part} of $\MW(S)$. By \cite[Theorem 8.6]{shioda90},
%$(\MW(S)^0, \, \langle \, , \, \rangle)$ is a positive definite even integral lattice.
%\end{itemize}

%  The abelian group $G_{\Sing(\varphi)}$, and the homomorphsim
%$\gamma : \MW(S) \to G_{\Sing(\varphi)}$   are those defined in \cite[p. 83]{tokunaga12}.
%For $s \in \MW(S)$, $\gamma(s)$ describes at which irreducible component $s$ meets on 
%$F_v$. 


\subsubsection{Double cover construction of an elliptic surface}\label{subsec:double-cover}

We refer to \cite[Lectures III and IV]{miranda-basic} for details.
Let $\varphi: S \to \PP^1$ be an elliptic surface over a smooth projective curve $\PP^1$. 
%Under our assumptions, we can regard the generic fiber $E$
%of $\varphi$ as  an elliptic curve over $\CC(\PP^1)(\cong \CC(t))$, the rational function field of $\PP^1$.
As we see in \cite{bannai-tokunaga, tokunaga14}, $S$ can be represented as the minimal resolution of a double cover of a Hirzebruch surface $\Sigma_d$
as follows. The inversion of $E$ with respect to
the group law  induces an involution $[-1]_{\varphi}$ on $S$. 
Let $S/\langle [-1]_{\varphi}
\rangle$ be the quotient by $[-1]_{\varphi}$.  It is known that  $S/\langle[-1]_{\varphi}\rangle$ 
is smooth and we can  blow down  $S/\langle [-1]_{\varphi} \rangle$ to its relatively minimal
model $W$. We denote the morphisms involved by
      \begin{itemize}
       \item $f: S \to S/\langle [-1]_{\varphi}\rangle$: the quotient morphism, 
       \item $q: S/\langle [-1]_{\varphi}\rangle \to W$: the blow down, and 
      \item $S \miya{{\mu}} S' \miya{{f'}} W$: the Stein factorization of $q\circ f$. 
      \end{itemize}
      Then (i)  $W$ is   $\Sigma_d$, where $d = 2\chi(\mcO_S)$ and (ii)
 the branch locus $\Delta_{f'}$ of $f'$ is of the form $\Delta_0 + \mcT$, where
 $\Delta_0$ is a section with $\Delta_0^2 = -d$ and $\mcT \sim 3(\Delta_0 + d{\mathfrak f})$, 
 ${\mathfrak f}$ being a fiber of the ruling $\Sigma_d \to {\mathbb P}^1$. Moreover, 
 singularities of $\mcT$ are at most simple singularities (see \cite[Chapter II, \S 8]{bpv} for simple singularities and their notation).  
 
%\end{enumerate}

Conversely,  if $\Delta_0$ and $\mcT$ on $\Sigma_d$, $d$: even,  satisfy the above conditions, we
obtain an elliptic surface $\varphi : S \to \PP^1$, as the canonical 
resolution of a double cover $f' : S' \to \Sigma_d$ with $\Delta_{f'} = \Delta_0
+ \mcT$, and the following diagram (see \cite{horikawa} for the canonical resolution):
\[
\begin{CD}
S' @<{\mu}<< S \\
@V{f'}VV                 @VV{f}V \\
\Sigma_d@<<{q}< \widehat{\Sigma}_d.
\end{CD}
\]
Here, $q$ is a composition of blowing-ups so that $\widehat{\Sigma}_d
= S/\langle [-1]_{\varphi}\rangle$. 
Hence any elliptic surface  is obtained in this way. In the case when $S$ is rational, $d = 2$.
In the following, we call the diagram above
{\it the double cover diagram for $S$}.


Moreover, if $S$ is rational and has a reducible singular fiber, $\widehat{\Sigma}_2$ can be blown
down to $\PP^2$, as we remark in \cite[1.2.2]{bannai-tokunaga}.  $\mcT$ is mapped to
a reduced quartic $\mcQ$, which in not concurrent four lines,  and $\Delta_0$ is mapped to a smooth point $z_o$ on
 $\mcQ$.  Conversely, given a reduced quartic $\mcQ$ ($\neq$ concurrent four lines) and a point 
 $z_o \in \mcQ$, we obtain $S$ as above, which we denote by $S_{\mcQ, z_o}$, which is nothing but the surface described in the introduction. The induced
 generically $2$-$1$ morphism from $S_{\mcQ, z_o}$ to $\PP^2$ is $f_{\mcQ, z_o}$. 
% and the elliptic fibration  by
 %$\varphi_{\mcQ, z_o} : S_{\mcQ, z_o} \to \PP^1$.
 
 
 
 \subsubsection {$S_{\mcQ, z_o}$ for the case when $\mcQ$ is $\mcE + \mcL_o$ in the combinatorics 1 and 2}
 
 Let $z_{o}$ be a smooth point on $\mcQ$. The tangent line $l_{z_{o}}$ to $\mcQ$ at $z_{o}$ gives rise to a singular fiber of ${\varphi}_{\mcQ , z_{o}}$ whose type is determined by how $l_{z_{o}}$ intersects with $\mcQ$ as follows:

\begin{center}
\begin{tabular}{|c|c|l|} \hline
(i) & $\I_{2}$ & $l_{z_{o}}$ meets $\mcQ$ with two other distinct points. \\ \hline
(ii) & $\III$ & $z_{o}$ is an inflection point of $\mcE$. \\ \hline
(iii) & $\I_{0}^{\ast}$ & $l_{z_{o}} = {\mcL}_{o}$. \\ \hline
(iv) & $\I_{4}$ & $l_{z_{0}}$ passes through a  point in  $\mcQ\cap \mcL_o$. \\ \hline
\end{tabular}
\end{center}

%In order to know the structure of $E_{\mcQ, z_o}$, we need to know the configurations of reducible singular fibers 
%of $\varphi_{\mcQ, z_o}$.

 Hence by  \cite[Table 6.2]{miranda-persson},  possible configurations of reducible singular fibers of $S_{\mcQ, z_o}$ are
 as follows:

\noindent{\sl Case 1: $\mcE$ is a smooth cubic}
 
% From \cite[Table 6.2]{miranda-persson} and  above table, we have the following table for possible configurations 
% of reducible singular fibers of $S_{\mcQ, z_{o}}$:

\begin{center}
\begin{tabular}{|c|c|} \hline
 & singular fibers \\ \hline
(i) , (ii)& $\{ a \I_{2}, b\III \}, a+ b = 4, a, b \ge 0, b \neq 3$ \\ \hline
(iii) & $\{ I_0^{\ast} \}$ \\ \hline
(iv) & $\{ \I_{4}, 2\I_{2} \}$ \\ \hline

\end{tabular}
\end{center}

%Since 
%the difference between $\mathrm{III}$ type and $\mathrm{I}_{2}$ type do not affect the structure of $E_{\mcQ, z_o}(\CC(t))$ and 
%$\mcL_o$ gives rise to a $2$-torsion section, by \cite{oguiso-shioda}, we infer that $E_{\mcQ, z_o}(\CC(t)) \cong D_4^{\ast} \oplus
%\ZZ/2\ZZ$.
%For our proof of Theorem~\ref{thm:comb-1}, we choose  $z_o$ satisfying  (i) or (ii). Hence we omit the remaining cases.
% 
 
\noindent  {\sl Case 2: $\mcE$ is a nodal cubic}

\begin{center}
\begin{tabular}{|c|c|} \hline
case & configration of singular fibers \\ \hline
(i), (ii)  & $\{a\I_2, b\III\}, a+ b = 5, 0 \le b \le 2$ \\ \hline
(iii) & $\{\I_0^*, \I_2\}, \{\I_0^*, \III\}$ \\ \hline
(iv) &  $\{\I_4, 3\I_2\}$ \\ \hline
%$4$ & $\{ 2 \, \mathrm{I}_{2} , \, \mathrm{I}_{0}^{\ast} \} , \{ \mathrm{III}, \mathrm{I}_{2} , \mathrm{I}_{0}^{\ast} \}$ \\ \hline
\end{tabular}
\end{center}
%%%%%%%%%%%%図を入れる

For our proof of Theorem~\ref{thm:comb-1}, we choose  $z_o$ satisfying  (i) or (ii).
In these cases, since 
the difference between fibers of type $\mathrm{III}$ and $\mathrm{I}_{2}$ do not affect the structure of $E_{\mcQ, z_o}(\CC(t))$ and since  
$\mcL_o$ gives rise to a $2$-torsion section, by \cite{oguiso-shioda}, we infer that
 $E_{\mcQ, z_o}(\CC(t)) \cong D_4^{\ast} \oplus
\ZZ/2\ZZ$ (resp. 
 $(A_1^*)^{\oplus 3} \oplus\ZZ/2\ZZ$) for Case 1 (resp.  Case 2).

% Hence, by \cite{oguiso-shioda},  we have
%$E_{\mathcal{Q}, z_{0}}(\mathbb{C}(t)) = {A_{1}^{\ast}}^{\oplus 3} \oplus \mathbb{Z}/2 \mathbb{Z}$.


%structures of $E_{\mathcal{Q}, z_{0}}( \mathbb{C}(t))$ with respect to the height pairing $\langle \ , \ \rangle$ is given by Gram-matrices as follows:

%\begin{enumerate}
%\item[(I)] For the configurations of singular fibers of cases $1$ and $2$, the structure of
%\begin{align*}
%E_{\mathcal{Q}, z_{0}}(\mathbb{C}(t)) = {A_{1}^{\ast}}^{\oplus 3} \oplus \mathbb{Z}/2 \mathbb{Z}.
%\end{align*}
%\item[(II)] For the configurations of singular fibers of cases $3$, the structure of
%\begin{align*}
%E_{\mcQ, z_{0}}(\CC(t)) = ({A_{1}^{\ast}})^{\oplus 2} \oplus \ZZ/2 \ZZ.
%\end{align*}
%
%\end{enumerate}
%



\subsection{Galois covers}

For the notation and terminology  on Galois covers, we use those in \cite{act, bannai-tokunaga, tokunaga94} freely.

\subsubsection{$D_{2p}$-covers}

We here introduce notation for dihedral covers which we use frequently. For details, see \cite{tokunaga94}.
Let $D_{2p}$ be the dihedral group of order $2p$, where $p$ is an  odd prime. 
 In order to present $D_{2p}$, we use
 the notation
 \[
 D_{2p} = \langle \sigma, \tau \mid \sigma^2 = \tau^p = (\sigma\tau)^2 = 1\rangle.
 \]
 Given a $D_{2p}$-cover, we obtain a double cover $D(X/Y)$ of $Y$ canonically by considering the
 $\CC(X)^{\tau}$-normalization of $Y$, where $\CC(X)^{\tau}$ denotes the fixed field 
 of the subgroup of $D_{2p}$ generated by $\tau$.  Then $X$ is a $p$-fold cyclic cover of $D(X/Y)$ and
 we denote the covering morphisms by
 $\beta_1(\pi) : D(X/Y) \to Y$ and $\beta_2(\pi) : X \to D(X/Y)$, respectively.  

\subsubsection{Elliptic $D_{2p}$-covers}

Let $\varphi : S \to \PP^1$ be a rational elliptic surface and let $f : S \to \widehat{\Sigma}_d$ denote the one 
int the double cover diagram.  For our criterion to distinguish  the topology of plane curves, we make use  of the existence/non-existence
 of $D_{2p}$-covers $\pi_p : X_p \to \widehat{\Sigma}_d$ of $\widehat{\Sigma}_d$ satisfying  (i) $D(X_p/\widehat{\Sigma}_d) = 
 S$ and (ii) $\beta_1(\pi_p) = f$.  Following \cite{tokunaga14}, we call such a $D_{2p}$-cover  an
 elliptic $D_{2p}$-cover. We denote the covering transformation of $f$ by $\sigma_f$. 
 As we remark in \cite[\S 3]{tokunaga14}, the branch locus $\Delta_{\beta_2(\pi_p)}$ of 
 $\beta_2(\pi_p)$ is the form
 \[
 \mcD + \sigma_f^*\mcD + \Xi + \sigma_f^*\Xi,
 \]
 where
 \begin{enumerate}
  
  \item[(i)]  no irreducible component of $\mcD$ and $\sigma_f^*\mcD$ is contained in any fiber (we call such a divisor are horizontal), and
  there exist no common components between $\mcD$ and $\sigma_f^*\mcD$,  and
 
 \item[(ii)] all irreducible components $\Xi$ and $\sigma_f^*\Xi$ are fiber components of $\varphi$ and 
 there exist no common components between $\Xi$ and $\sigma_f^*\Xi$.
 
 \end{enumerate}
 
 \begin{rem}\label{rem:components} { \rm Possible irreducible components of $\Xi$ and $\sigma_f^*\Xi$ can be determined by
  \cite[Remark 3.1]{tokunaga14}. In particular, if singular fibers of $\varphi$ are of types $\I_1, \I_2, \II, \III$ only,
  $\Xi_f = \emptyset$.
 }
 \end{rem}
 
% To prove our statement, we need a slightly modified version of \cite[Theorem 3.2]{tokunaga14} as follows:
% 
% \begin{thm}\label{thm:criterion}{Let $p$ be an odd prime. Choose $s_1, \ldots, s_r \in \MW(S)$ such that
% $\sigma_f^*s_i \not\in \{s_1, \ldots, s_r\}$ for every $i$. Then $(I)$ and $(II)$ in the below are equivalent:
% 
% \begin{enumerate}
% 
% \item[(I)] Put $\mcD = \sum_i s_i$. There exists a $D_{2p}$-cover $\pi_p : X_p \to \widehat{\Sigma}_d$ 
% such that 
%      \begin{itemize}
%        \item $D(X_p/\widehat{\Sigma}_d) = S$ and $\beta_1(\pi_p) = f$,
%        
%        \item $\Delta_{\beta_2(\pi_p)} = \mcD + \sigma_f^*\mcD + \Xi + \sigma_f^*\Xi$ for some effective
%        divisor $\Xi$ on $S$ such that
%        all irreducible components $\Xi$ and $\sigma_f^*\Xi$ are fiber components of $\varphi$ and 
%         there exist no common components between $\Xi$ and $\sigma_f^*\Xi$.
%             \end{itemize}
%  
% \item[(II)] There exist integers $a_i$ ($i = 1, \ldots, r$) such that
%      \begin{itemize}
%        \item $1 \le a_i < p$ ($i = 1, \ldots, r$), and 
%       \item $\displaystyle{\sum_{i=1}^s[a_i]s_i}$ is $p$-divisible in $\MW(S)$, i.e., 
%       there exists $s \in \MW(S)$ 
%       such that 
%       \[
%      \sum_{i=1}^s[a_i]s_i = [p]s.
%      \]
%     \end{itemize}
%       
% \end{enumerate}
% 
% 
% }
% \end{thm}








