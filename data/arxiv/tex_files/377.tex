\section{Optimal Solution without D2D} \label{sec:optimal_no_d2d}
In this section, we describe how to compute the minimum spectrum
without D2D, i.e., $F^{\textsf{ND}}$. Since there are no D2D links,
we can calculate the required minimum spectrum for each BS separately.
Let us denote $F_b^{\textsf{ND}}$ as the minimum spectrum
of BS $b$ to deliver all its own traffic demands, i.e., $\mathcal{J}_b$.
Then the total minimum spectrum without D2D is\footnote{Here for simplicity, we assume that
all BSs use orthogonal spectrum.
%In practice, the same spectrum can be spatially reused by multiple BSs separated by enough distances.
We discuss how to extend our results
to the practical case of spectrum reuse in Sec.~\ref{sec:towards_spectrum_reduction}.}
$
F^{\textsf{ND}} = \sum_{b \in \mathcal{B}} F_b^{\textsf{ND}}.
$


\subsection{Problem Formulation}
For each BS $b \in \mathcal{B}$, we formulate the problem of
minimizing the spectrum to deliver all demands in cell $b$ without D2D,
named as $\textsf{Min-Spectrum-ND}_b$,
\bse
\bee
%&  \nnb \\
& \min_{x^{j}_{u_j,b}(t), \gamma_b(t), F_b \in \mathbb{R}^+}  \quad F_b  \label{equ:nd_up_peak_obj}\\
\text{s.t.}
%& \quad \eqref{equ:nd_up_traffic_cons1}, \eqref{equ:nd_up_traffic_cons2}, \nnb \\
& \quad \sum_{t=s_j}^{e_j} x^{j}_{u_j,b}(t)R_{u_j,b} = r_j, \forall j \in \mathcal{J}_b
\label{equ:nd_up_traffic_cons1} \\
& \quad \sum_{j \in \mathcal{J}_b: t \in [s_j,e_j]} x^{j}_{u_j,b}(t) = \gamma_b(t), \forall t \in [T]
\label{equ:nd_up_peak_cons1} \\
& \quad \gamma_b(t) \le F_b,
\forall t \in [T]
\label{equ:nd_up_peak_cons2} \\
& \quad x^{j}_{u_j,b}(t) \ge 0, \forall j \in \mathcal{J}_b, t \in [s_j, e_j]
\label{equ:nd_up_traffic_cons2}
\eee
\ese
where $x^{j}_{u_j,b}(t)$ is the allocated spectrum (unit: Hz)
for transmitting demand $j$ from user $u_j$ to BS $b$ at slot $t$,
the auxiliary variable $\gamma_b(t)$ is the total used spectrum from users to BS $b$ at slot $t$,
and $F_b$ is the allocated (peak) spectrum to BS $b$,

Our objective is to minimize the total allocated spectrum of BS $b$, as shown in \eqref{equ:nd_up_peak_obj}.
Without D2D, users can only be served by its own BS.
Equation \eqref{equ:nd_up_traffic_cons1} shows the volume requirement for any
traffic demand $j$, i.e., the total traffic volume $r_j$ needs to be delivered from user $u_j$ to BS $b$ during its lifetime.
Equation \eqref{equ:nd_up_peak_cons1} depicts the total needed spectrum of cell $b$ (i.e., $\gamma_b(t)$) in slot $t$,
which is the summation of allocated spectrum for all active jobs in slot $t$.
Inequality \eqref{equ:nd_up_peak_cons2} shows that the total needed spectrum of cell $b$ in any slot $t$
cannot exceed the total allocated spectrum of BS $b$.
Finally, inequality \eqref{equ:nd_up_traffic_cons2} means that the allocated spectrum for a job in any slot is non-negative.



Let us denote $d_{\max} \triangleq \max_{j \in \mathcal{J}} (e_j-s_j+1)$ as
the maximum delay among all demands. Then the number of variables in
$\textsf{Min-Spectrum-ND}_b$ is $O(|\mathcal{J}_b| \cdot d_{\max}+T)$
and the number of constraints in $\textsf{Min-Spectrum-ND}_b$ is also $O(|\mathcal{J}_b| \cdot d_{\max}+T)$.

\subsection{Characterizing the Optimal Solution}
To solve $\textsf{Min-Spectrum-ND}_b$, we can use standard linear programming (LP) solvers.
However, LP solvers cannot exploit the structure of this problem. We next propose a combinatorial algorithm that exploits the problem structure and
achieves lower complexity than general LP algorithms.

We note that $\textsf{Min-Spectrum-ND}_b$ resembles a uniprocessor scheduling problem for preemptive tasks
with hard deadlines \cite{Buttazzo97}. Indeed, we can attach each task $j \in \mathcal{J}_b$ with an
arrival time $s_j$ and a hard deadline $e_j$ and the requested service time
$\frac{r_j}{R_{u_j,b}}$. Then for a given amount of allocated spectrum $F_b$
(which resembles the maximum speed of the processor), we can use
the earliest-deadline-first (EDF) scheduling algorithm  \cite{EDF73} to check its feasibility. Since we
can easily get an upper bound for the minimum spectrum,
%\footnote{For instance,
%we can construct a feasible solution by averaging each traffic demand
%within its lifetime, i.e., $x^{j}_{u_j,b}(t)=\frac{{r_j}/{R_{u_j,b}}}{e_j-s_j+1},
%\forall t \in [s_j, e_j].$ Then $F_b= \max_{t\in [T]}\sum_{j \in \mathcal{J}_b: t \in [s_j,e_j]} x^{j}_{u_j,b}(t)$
%serves as an upper bound for $F_b^{\textsf{ND}}$.},
we can use binary search to find the minimum spectrum $F_b^{\textsf{ND}}$,
supported by the EDF feasibility-check subroutine.

More interestingly, we can even get a semi-closed form for $F_b^{\textsf{ND}}$, inspired by \cite[Theorem 1]{YDS95}.
Specifically, let us define the \emph{intensity} \cite{YDS95} of an interval $I = [z,z']$
to be
\be
g_b(I) \triangleq \frac{\sum\limits_{j \in \mathcal{A}_b(I)} \frac{r_j}{R_{u_j,b}}}{z'-z + 1}
\label{equ:intensity_nd_b}
\ee
where
$
\mathcal{A}_b(I) \triangleq \{j \in \mathcal{J}_b:
[s_j, e_j] \subset [z,z']\}
$
is the set of all active traffic demands whose lifetime is within the interval $I=[z,z']$.
Then we have the following theorem.


\begin{theorem}
$F_b^{\textsf{ND}} = \max\limits_{I \subset [T]} g_b(I)$.
\label{the:YDS}
\end{theorem}

\begin{IEEEproof}
Since the proof of Theorem 1 was omitted in \cite{YDS95} and
the theorem is not directly mapped to the minimum spectrum problem,
we give a  proof
\ifx \ISTR \undefined
in our technical report \cite{TR} for completeness.
\else
in Appendix \ref{app:proof_YDS} for completeness.
%in the supplementary materials for completeness.
\fi
\end{IEEEproof}

Theorem~\ref{the:YDS} shows that $F_b^{\textsf{ND}}$ is the maximum intensity over all intervals.
To obtain the interval with maximum intensity (and hence $F_b^{\textsf{ND}}$),
we adapt the  algorithm originally developed for solving the job scheduling problem
in \cite{YDS95}, which is called YDS algorithm named after the authors, to our spectrum minimization problem.
%In the rest of this paper, we will call Theorem \ref{the:YDS} the YDS algorithm.
The time complexity
of the YDS algorithm is related to the total number of possible intervals. Clearly
the optimal interval can only begin from the generation time of a demand and end
at the deadline of a demand. So the total number of intervals needed to be checked is $O(|\mathcal{J}_b|^2)$.
Thus the time complexity of our adaptive YDS algorithm is $O(|\mathcal{J}_b|^2)$ \cite{YDS95}.
But the complexity of general LP algorithms is $O((|\mathcal{J}_b| \cdot d_{\max}+T)^4L)$
where $L$ is a parameter determined by the coefficients of the LP \cite{khachiyan1980polynomial}.
Thus, our combinatorial algorithm has much lower complexity than general LP algorithms.
