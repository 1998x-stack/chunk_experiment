


\section{A Low-Complexity Heuristic Algorithm for \textsf{Min-Spectrum-D2D}}  \label{sec:heuristic}
Our proposed LP formulation for \textsf{Min-Spectrum-D2D} has high complexity
due to the size of input traffic demand and cellular network.
To reduce the complexity, in this section, we propose a heuristic algorithm which
can significantly reduce the number of traffic demands that is needed to be considered. Moreover,
our algorithm has a parameter (which is $\lambda$ defined shortly)
such that we can balance the complexity and
the performance.

Our proposed algorithm has three steps.
\par\noindent\rule{0.5\textwidth}{0.4pt}

\emph{Step I.}
We solve \textsf{Min-Spectrum-ND}$_b$ for each BS $b \in \mathcal{B}$, and get
the optimal solution $\{x^{j}_{u_j,b}(t), \gamma_b(t), F_b\}$.

\emph{Step II.}
For each BS $b$ with the spectrum profile $\gamma_b(t)$, we consider the following set,
\be
T_b(\lambda) \triangleq \{t \in [T]: \gamma_b(t) > \lambda F_b \},
\label{equ:T-b-lambda}
\ee
where parameter $\lambda \in [0,1]$ controls the split level.
Now we divide all cell-$b$ traffic demands $\mathcal{J}_b$ into two demand sets
\be
\resizebox{0.89\linewidth}{!}{$\mathcal{J}_b^{\textsf{D2D}}(\lambda) \triangleq \{j \in \mathcal{J}_b:
 \exists t \in [s_j, e_j] \cap T_b(\lambda) \text{ s.t. } x_{u_j,b}^{j}(t) > 0 \},$}
\label{equ:D2D-demand-set}
\ee
and
\be
\resizebox{0.89\linewidth}{!}{$\mathcal{J}_b^{\textsf{ND}}(\lambda) \triangleq \{j \in \mathcal{J}_b: x_{u_j,b}^{j}(t) = 0, \forall t \in [s_j, e_j] \cap T_b(\lambda)\}.$}
\label{equ:ND-demand-set}
\ee

For all traffic demand in $\mathcal{J}_b^{\textsf{ND}}(\lambda)$, we  schedule them according to $\{x^{j}_{u_j,b}(t)\}$ without D2D,
which results in at most $\gamma_b(t)$ spectrum requirement for BS $b$ at slot $t$.
Note that no demand in $\mathcal{J}_b^{\textsf{ND}}(\lambda)$
is served in slot set $T_b(\lambda)$. We thud denote $\tilde{\gamma}_b(t)$ as
the already allocated spectrum spectrum for demand set $\mathcal{J}_b^{\textsf{ND}}(\lambda)$ for BS $b$ at slot $b$,
which satisfies $\tilde{\gamma}_b(t) \le \gamma_b(t)$ when $t \notin T_b(\lambda)$
and $\tilde{\gamma}_b(t) =0$ when $t \in T_b(\lambda)$.


\emph{Step III.}  We solve the D2D load balancing problem with traffic demands
$\mathcal{J}^{\textsf{D2D}}(\lambda) \triangleq \{\mathcal{J}_b^{\textsf{D2D}}(\lambda): b \in \mathcal{B}\}$,
according to the following LP, which adaptes \textsf{Min-Spectrum-D2D} in \eqref{equ:d2d-lp} by considering the already allocated spectrum $\{\tilde{\gamma}_b(t)\}$,
\bse \label{equ:d2d-lp-heuristic}
\bee
%& \textsf{PEAK-D2D}  \nnb \\
& \min_{x^{j}_{u,v}(t), \alpha_b(t),  \beta_b(t), F_b \in \mathbb{R}^+}  \quad \sum_{b \in \mathcal{B}} F_b \\
%\text{subject to} & \quad \eqref{equ:d2d_up_traffic_cons1}, \eqref{equ:d2d_up_traffic_cons2},
%\eqref{equ:d2d_up_traffic_cons3}, \eqref{equ:d2d_up_traffic_cons4}, \nnb \\
\text{s.t.} & \quad \eqref{equ_demand}, \eqref{equ_reach},
\eqref{equ_conservation}, \eqref{equ:y_nonnegative}, \eqref{equ:y_nonnegative_selflink} \nnb \\
& \quad \sum_{v \in \mathcal{U}_b} \sum_{j \in \mathcal{J}^{\textsf{D2D}}(\lambda): t \in [s_j, e_j]} x^{j}_{v,b}(t) = \alpha_b(t),  \forall b \in \mathcal{B}, t \in [T]
\label{equ:d2d_peak_cons1-heuristic}\\
& \quad \sum_{u \in \mathcal{U}_b} \sum_{v \in \text{in}\left( u \right)
\backslash \left\{ u \right\}} \sum_{j \in \mathcal{J}^{\textsf{D2D}}(\lambda): t \in [s_j, e_j]} x^{j}_{v,u}(t) = \beta_b(t),  \nnb \\
& \qquad \qquad \forall b \in \mathcal{B}, t \in [T]
\label{equ:d2d_peak_cons2-heuristic}\\
& \quad \alpha_b(t)+\beta_b(t) + \tilde{\gamma}_b(t) \le F_b,\forall b \in \mathcal{B}, t \in [T]
\label{equ:d2d_peak_cons3-heuristic}
\eee
\ese


Similar to the overhead minimization problem \textsf{Min-Overhead} in \eqref{equ:overhead-lp},
given the optimal spectrum requirement of \eqref{equ:d2d-lp-heuristic}, denoted as, $F^{\textsf{Heuristic}}(\lambda)$,
we next minimize the overhead by solving the following LP,
\bse \label{equ:d2d-lp-heuristic-overhead}
\bee
%& \textsf{PEAK-D2D}  \nnb \\
& \min_{\substack{x^{j}_{u,v}(t), \alpha_b(t), \\ \beta_b(t), F_b \in \mathbb{R}^+}}  \quad  \sum_{t=1}^{T} \sum_{j \in \mathcal{J}^{\textsf{D2D}}: t \in [s_j, e_j-1]}
\sum_{u \in \mathcal{U}} \sum_{\substack{v:v \in \mathcal{U}, \\ (u,v) \in \mathcal{E}}} x^{j}_{u,v}(t) {R_{u,v}} \\
%\text{subject to} & \quad \eqref{equ:d2d_up_traffic_cons1}, \eqref{equ:d2d_up_traffic_cons2},
%\eqref{equ:d2d_up_traffic_cons3}, \eqref{equ:d2d_up_traffic_cons4}, \nnb \\
\text{s.t.} & \quad \eqref{equ_demand}, \eqref{equ_reach},
\eqref{equ_conservation}, \eqref{equ:y_nonnegative}, \eqref{equ:y_nonnegative_selflink} \nnb \\
& \quad \sum_{v \in \mathcal{U}_b} \sum_{j \in \mathcal{J}^{\textsf{D2D}}(\lambda): t \in [s_j, e_j]} x^{j}_{v,b}(t) = \alpha_b(t),  \forall b \in \mathcal{B}, t \in [T]
\label{equ:d2d_peak_cons1-heuristic-overhead}\\
& \quad \sum_{u \in \mathcal{U}_b} \sum_{v \in \text{in}\left( u \right)
\backslash \left\{ u \right\}} \sum_{j \in \mathcal{J}^{\textsf{D2D}}(\lambda): t \in [s_j, e_j]} x^{j}_{v,u}(t) = \beta_b(t),  \nnb \\
& \qquad \qquad \forall b \in \mathcal{B}, t \in [T]
\label{equ:d2d_peak_cons2-heuristic-overhead}\\
& \quad \alpha_b(t)+\beta_b(t) + \tilde{\gamma}_b(t) \le F_b,  \forall b \in \mathcal{B}, t \in [T]
\label{equ:d2d_peak_cons3-heuristic-overhead} \\
& \quad \sum_{b \in \mathcal{B}} F_b \le F^{\textsf{Heuristic}}(\lambda) \label{equ:sum-spectrum-requirement-overhead}
\eee
\ese

\par\noindent\rule{0.5\textwidth}{0.4pt}

Note that in \eqref{equ:d2d-lp-heuristic}/\eqref{equ:d2d-lp-heuristic-overhead}, all variables, $x^{j}_{u,v}(t), \alpha_b(t),  \beta_b(t), F_b$, have
the same meanings of those in \eqref{equ:d2d-lp}/\eqref{equ:overhead-lp}. There are two differences between \eqref{equ:d2d-lp-heuristic}/\eqref{equ:d2d-lp-heuristic-overhead}
and \eqref{equ:d2d-lp}/\eqref{equ:overhead-lp}. First, the traffic demand set in \eqref{equ:d2d-lp-heuristic}/\eqref{equ:d2d-lp-heuristic-overhead} is
$\mathcal{J}^{\textsf{D2D}}(\lambda)$ while that in \eqref{equ:d2d-lp}/\eqref{equ:overhead-lp} is $\mathcal{J}$.
Likewise, the traffic scheduling policy characterized by \eqref{equ_demand}, \eqref{equ_reach},
\eqref{equ_conservation}, \eqref{equ:y_nonnegative}, \eqref{equ:y_nonnegative_selflink} in \eqref{equ:d2d-lp-heuristic}/\eqref{equ:d2d-lp-heuristic-overhead}  is  for the traffic demand set
$\mathcal{J}^{\textsf{D2D}}(\lambda)$ while that in \eqref{equ:d2d-lp}/\eqref{equ:overhead-lp} is for the traffic demand set $\mathcal{J}$.
Second, constraint \eqref{equ:d2d_peak_cons3-heuristic}/\eqref{equ:d2d_peak_cons3-heuristic-overhead}
is different from constraint \eqref{equ:d2d_peak_cons3}/\eqref{equ:overhead_peak_cons3} in that \eqref{equ:d2d_peak_cons3-heuristic}/\eqref{equ:d2d_peak_cons3-heuristic-overhead} considers the already allocated spectrum $\{\tilde{\gamma}_b(t)\}$.
Namely, the spectrum requirement for BS $b$ at slot $t$ includes the already allocated spectrum
$\tilde{\gamma}_b(t)$ to serve the traffic demand $\mathcal{J}_b^{\textsf{ND}}$
and the new allocated spectrum $(\alpha_b(t)+\beta_b(t))$ to serve the traffic demand
$\mathcal{J}_b^{\textsf{D2D}}(\lambda)$.




Obviously, if the number of traffic demand in $\mathcal{J}^{\textsf{D2D}}(\lambda)$ is much less than the total number of traffic demands in $\mathcal{J}$, which is indeed the case according to our empirical study in Sec.~\ref{sec:simulation},
we can significantly reduce the number of variables and constraints in
\eqref{equ:d2d-lp-heuristic}/\eqref{equ:d2d-lp-heuristic-overhead}
in Step III as compared to the LP problem \textsf{Min-Spectrum-D2D}/\textsf{Min-Overhead} in \eqref{equ:d2d-lp}/\eqref{equ:overhead-lp}.
After these three steps, the total spectrum is given by the objective value of \eqref{equ:d2d-lp-heuristic}
and the corresponding overhead is given by the objective value of \eqref{equ:d2d-lp-heuristic-overhead}.
An example of our heuristic algorithm is shown
\ifx \ISTR \undefined
in our technical report \cite{TR}.
\else
in Appendix~\ref{app:an-example-for-heuristic}.
% in the supplementary materials.
\fi




We denote the spectrum reduction of our heuristic algorithm as
\be
\rho^{\textsf{Heuristic}}(\lambda) \triangleq \frac{F^{\textsf{ND}}-F^{\textsf{Heuristic}}(\lambda)}{F^{\textsf{ND}}}.
\ee
Similarly, we denote $\eta^{\textsf{Heuristic}}(\lambda)$ as the overhead ratio of our heuristic algorithm.
We next show that the performance guarantee of our heuristic algorithm.

First, for the spectrum we reduction, we have,
\begin{theorem} \label{thm:performance-heuristic}
$(1-\lambda) \rho \le \rho^{\textsf{Heuristic}}(\lambda) \le \rho.$
\end{theorem}
\begin{IEEEproof}
\ifx \ISTR \undefined
Please see Appendix~\ref{app:proof-performance-heuristic}.
\else
Please see our technical report \cite{TR}.
\fi
\end{IEEEproof}





Theorem \ref{thm:performance-heuristic} shows that when $\lambda=0$, we have
$\rho^{\textsf{Heuristic}}(0) = \rho$. This is because when $\lambda=0$, we have
$\mathcal{J}^{\textsf{D2D}}(0)=\mathcal{J}$, i.e., all demands  participate
in D2D load balancing in our heuristic algorithm when $\lambda=0$ and thus
the objective value of \eqref{equ:d2d-lp-heuristic} when $\lambda=0$ is exactly $F^\textsf{D2D}$.
When $\lambda=1$, since $\mathcal{J}^{\textsf{D2D}}(1)=\emptyset$, all traffic demands are served locally without D2D and
therefore the objective value of \eqref{equ:d2d-lp-heuristic} when $\lambda=1$ is exactly $F^\textsf{ND}$.
Thus, the lower bound $(1-\lambda) \rho = 0$  is tight.
Further, the lower bound $(1-\lambda) \rho$,
decreases as $\lambda$ increases, but the computational complexity
decreases as $\lambda$ increases. Thus, this lower bound illustrates
the tradeoff between the performance and the complexity of our heuristic algorithm.

Second, we give an upper bound for the overhead ratio\footnote{Recall that $d_{\max}$ is the maximum demand delay.}.
\begin{theorem} \label{the:overhead-upper-bound-heuristic}
$\eta^{\textsf{Heuristic}}(\lambda) \le \frac{ (d_{\max}-1) \sum\limits_{j \in \mathcal{J}^{\textsf{D2D}  }(\lambda)} r_j }{
(d_{\max}-1) \sum\limits_{j \in \mathcal{J}^{\textsf{D2D}  }(\lambda)} r_j + \sum\limits_{j \in \mathcal{J}} r_j} \le \frac{d_{\max}-1}{d_{\max}}$.
\end{theorem}
\begin{IEEEproof}
\ifx \ISTR \undefined
Please see our technical report \cite{TR}.
\else
Please see Appendix \ref{app:proof-of-overhead-upper-bound-heuristic}.
% in the supplementary materials.
\fi
\end{IEEEproof}
We can see that the upper bound of the overhead ratio is 0 when $\lambda=1$ because $\mathcal{J}^{\textsf{D2D}}(1)=\emptyset$, i.e.,
all traffic demands are served locally without D2D. Moreover, when $\lambda$ increases, the upper bound decreases because
less traffic demands participate in D2D load balancing.

Overall, our heuristic algorithm
reduce the complexity of our global LP approach and has performance guarantee.
Moreover, our proposed heuristic algorithm has a controllable parameter $\lambda$ to
balance the benefit in terms of spectrum reduction, the cost in terms of overhead ratio, and
the computational complexity for our D2D load balancing scheme.
