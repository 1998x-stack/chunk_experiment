%% ==============================================================
%% PREAMBLE Journal article
%% =============================================================

%% ==============================================================
%% ENCODING
%% ==============================================================
%\usepackage[utf8]{inputenc}
%\usepackage[T1]{fontenc}
%% ==============================================================

%% ==============================================================
%% MATH SYMBOLS/FONTS
%% ==============================================================
%\usepackage{amsfonts}
\usepackage{amsmath}
\usepackage{amsthm}
\usepackage{amssymb}
\usepackage{mathrsfs} % Ralph Smith’s Formal Script Font : mathscr{A}
%\xrightarrow[T]{n\pm i-1}
%\usepackage{siunitx} % physics units
%% MODIF ND
\newcommand{\si}[1]{#1}
\newcommand{\second}{$\mathrm{s}$}
\newcommand{\nano}{$\mathrm{n}$}
\newcommand{\metre}{$\mathrm{m}$}
\newcommand{\SIrange}[3]{$#1$#3 to $#2$#3}


%\usepackage{bbold} % fonction indicatrice
\usepackage{dsfont} % \mathds{1} pour indicatrice
% \usepackage{bbm} % \mathbbm{1} pour autre version de l'indicatrice
\usepackage{mathtools}
\usepackage{stmaryrd}  % \llbracket et \rrbracket %sinon : $[\![$ et $]\!]$
\usepackage{enumitem} % personnalisation des enumerate
%% ==============================================================

%% ==============================================================
%% ALGORITHM
%% ==============================================================
\usepackage[ruled,vlined]{algorithm2e} % package environnement algorithme
\newcommand{\forcond}[2]{#1 \KwTo #2}
%\usepackage{framed}
%% ==============================================================

%% ==============================================================
%% FIGURES/TABLES/SUBFIGURES
%% ==============================================================
\usepackage{booktabs}
%\usepackage[caption=false,font=footnotesize]{subfig} % subfig altère la mise en page IEEE pour table/figures si l'option caption=false n'est pas choisie

%\usepackage{caption}
%\usepackage{subcaption}
\usepackage{multirow} % cellule sur plusieurs lignes
%\usepackage{subcaption}
%\usepackage[labelformat=simple]{subcaption}
%\captionsetup[subtable]{position=top}
% refaire subfig, abandonner subtables
\usepackage[figuresleft]{rotating} % rotation des figures/tableaux avec les environnements spécifiques
% \begin{sidewaysfigure} \end{sidewaysfigure}
% \begin{sidewaystable} \end{sidewaystable}

% Remove the trailing whitespace from citations
\usepackage[noadjust]{cite}
\usepackage{graphicx}
\DeclareGraphicsExtensions{.pdf,.jpg,.png}
%\usepackage{listings}
%\vspace*{\stretch{1}} % espaces élastiques !
\usepackage{textcomp} % degree symbol \textdegree

% Useful for tables
%\resizebox{0.48\textwidth}{!}{%
%		\begin{tabular}{@{}llccccc@{}} ...
% }
%%


%% ==============================================================

%% ==============================================================
%% LINKS/HYPERTEXT
%% ==============================================================
\usepackage{url}
\usepackage{hyperref}  %affichage/création des liens hypertextes
\hypersetup{colorlinks,%
            citecolor=red,%
            filecolor=black,%
            linkcolor=blue,%
            urlcolor=blue}
%% ==============================================================

%% ==============================================================
%% NEW COMMANDS
%% ==============================================================
% Remark : spacing in math mode
% \, thin space		\: medium space		\; large space
\newcommand{\wrt}{with respect to}
\def\defequal{\stackrel{\mbox{\footnotesize def}}{=}}
% Résultats de simulation
\newcommand{\best}[1]{\textcolor[rgb]{0.00,0.00,1.00}{\mathbf{#1}}}
%\newcommand{\second}[1]{\textcolor{blue}{#1}}
%% ==============================================================

%% ==============================================================
%% THEOREMS
%% ==============================================================
%\newtheorem{theorem}{Theorem}[section]
\newtheorem{theorem}{Theorem}
%\newtheorem{lemma}[theorem]{Lemma}
\newtheorem{lemma}{Lemma}
%\newtheorem{proposition}[theorem]{Proposition}
\newtheorem{proposition}{Proposition}
\newtheorem*{proposition*}{Proposition}
\newtheorem{corollary}{Corollary}[theorem]
\theoremstyle{definition}
\newtheorem*{defn}{Definition}
\newtheorem{remark}{Remark}
\newtheorem*{remark*}{Remark}
\newtheorem{assumption}{Assumption}
%% ==============================================================


\newcommand{\nd}[1]{\textcolor[rgb]{1.00,0.00,0.00}{#1}}
