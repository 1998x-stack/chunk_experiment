%%%%%%%%%%%%%%%%%%%%%%%%%%%%%%%%%%%%%%%%%%%%%%%%%
% topology
%%%%%%%%%%%%%%%%%%%%%%%%%%%%%%%%%%%%%%%%%%%%%%%%%

\section{Proof of the Elementary Swap Theorem}
\label{sec:topology}

%\comment{
As mentioned in the introduction, we prove the Elementary Swap Theorem using topological properties of the \emph{flip complex}, whose 1-skeleton (i.e.~vertices and edges) is the flip graph. 
In fact, we will only need the 2-cells of the flip complex, not any higher-dimensional structure. 
We will show that 2-cells of the flip complex correspond to $4$- and $5$-cycles in the flip graph.
%We will show that any 2-cell of the flip complex corresponds to a $4$-cycle or $5$-cycle in the flip graph.

The basic idea is as follows.  We will translate the Elementary Swap Theorem to a statement about walks in the flip graph.  
The hypothesis of the Elementary Swap Theorem is that we have a sequence of flips that permutes the edge labels of a triangulation $T$.  In the flip graph, this sequence corresponds to a closed walk $w$ that starts and ends at triangulation $T$.   Our main topological result 
is that the flip complex has a trivial fundamental group, which will imply that such a closed walk $w$ can be decomposed into simpler \emph{elementary walks}.  Each elementary walk starts at $T$, traces a path in the flip graph, then traverses the edges of a 2-cell, then retraces the path back to $T$. 
The edge-label permutation induced by an elementary walk depends on the 2-cell.
If the 2-cell is a $4$-cycle, the permutation is the identity; and if the 2-cell is a $5$-cycle, then the permutation is a transposition, and the elementary walk corresponds to an elementary swap.  Altogether, this implies that the permutation induced by the closed walk $w$ can be expressed as a composition of elementary swaps, which proves the Elementary Swap Theorem.

%%%%%%%Uli's original version
%To prove the Elementary Swap Theorem, we consider closed walks in the flip graph---flip sequences starting and ending at the same triangulation $T$---and we analyze which permutations of the edges of $T$ (equivalently, of labellings of $T$) are induced by such walks.
%
%In a nutshell, the basic topological fact that we will use is the following: we can enlarge the flip graph 
%to a $2$-dimensional complex by attaching $2$-dimensional faces (quadrilaterals and pentagons)
%to \emph{elementary cycles} of length $4$ and $5$ in the flip graph, and the resulting $2$-dimensional 
%cell complex is \emph{simply connected}, i.e., has trivial fundamental group.
%
%This can be interpreted combinatorially and allows us to decompose an arbitrary closed walk in the
%flip graph into simpler \emph{elementary walks}
%each of which traverses such an elementary cycle exactly once. 
%In terms of edge-labels in a triangulation, elementary quadrilateral walks correspond to a trivial permutation and elementary pentagonal walks correspond to elementary swaps. From this, we can then deduce that every permutation that can be realized by a closed walk in the flip graph
%can be expressed as a composition of elementary swaps, and the Elementary Swap Theorem follows.
%%%%%%%%%%%%%

Before stating our main topological theorem, we first define the special cycles that will be shown to correspond to $2$-cells of the flip complex. %graph.
In the same way that an edge of the flip complex corresponds to two triangulations that differ on one edge, every 2-cell of the flip complex corresponds to a set of triangulations that differ on two edges.  
Define an \emph{elementary $4$-cycle} to be a cycle of the flip graph obtained in the following way.  Take a triangulation $T$ and two edges $e,f \in T$ whose removal leaves two internally disjoint convex quadrilaterals in $T$.  Each quadrilateral can be triangulated in two ways, which results in four triangulations that contain  $F:=T\setminus \{e,f\}$.  These four triangulations form a $4$-cycle in the flip graph, as shown in Figure~\ref{4-5cycle}(a).  Observe that a traversal of the cycle corresponds to a  sequence of flips that returns edge-labels to their original positions. 

Define an \emph{elementary $5$-cycle} to be a cycle of the flip graph obtained in the following way.  Take a triangulation $T$ and two edges $e,f \in T$ whose removal leaves a convex pentagon in $T$.  There are five triangulations that contain  $F:=T\setminus \{e,f\}$, and they form a $5$-cycle in the flip graph, as shown in Figure~\ref{4-5cycle}(b). Observe that the sequence of flips around such a cycle permutes labels of $e$ and $f$ as shown in Figure~\ref{fig:basic-pentagon-swap}.

\begin{figure}
\begin{centering}
\includegraphics[width=.9\linewidth]{4-5cycle}
\par\end{centering}
\protect\caption{(a) Triangulations that differ in the diagonals of two internally
disjoint quadrilaterals form an \emph{elementary $4$-cycle} in the flip graph.  The cycle does not permute the labels (shown as red and blue).  (b)  Triangulations that differ in the diagonals of a convex pentagon form an \emph{elementary 5-cycle} in the flip graph. 
This cycle permutes labels as shown in Figure~\ref{fig:basic-pentagon-swap}.}
\label{4-5cycle}
\end{figure}


% Anna's notes 
%Consider a triangulation $T$ and two distinct flippable edges $e,f\in T$, and consider removing both $e$ and $f$ from $T$.   We distinguish two cases depending on whether $e$ and $f$ are contained in a common triangle of $T$.  If they are not, then removing $e$ and $f$ creates two internally disjoint convex quadrilaterals $Q_e$ and $Q_f$.  In this case there are four
%triangulations of $P$ containing $F:=T\setminus \{e,f\}$ (corresponding to the four choices of pairs of diagonals of $Q_e$ and $Q_f$), and these form a $4$-cycle in the flip graph, which we will refer to as an \emph{elementary $4$-cycle}, see Figure~\ref{4-5cycle}~(a).
%In the second case, $e$ and $f$ are contained in a common triangle of $T$.   
%and that the union of the three triangles of $T$ containing them is a convex pentagon. In this case there are five triangulations of $P$ containing the edges in $F:=T\setminus\{e,f\}$, and these triangulations form a $5$-cycle in the flip graph, which we will refer to as an \emph{elementary $5$-cycle}, see Figure~\ref{4-5cycle}~(b). 

As a side remark, note that it can be shown that, in fact, any cycle in the flip graph of length less than 6 is an elementary 4- or 5-cycle. However, we will not need this in what follows.



Our main topological theorem is the following.
%The key to proving the Elementary Swap Theorem is the following topological result.

\begin{theorem}
\label{thm:flip-complex}
Let $P$ be a set of $n$ points in general position in the plane.
There is a high-dimensional cell complex $\fcomplex=\fcomplex(P)$, which we call the \emph{flip complex}, 
such that:
%with the following properties:
\begin{enumerate}
\item The 1-skeleton of $\fcomplex$ is the flip graph of $P$;
\item There is a one-to-one correspondence between the 2-cells of $\fcomplex$ and the elementary 4-cycles and elementary 5-cycles of the flip graph of $P$;
%The 2-cells of $\fcomplex$ correspond to the elementary 4-cycles and elementary 5-cycles of the flip graph;  
\item $\fcomplex$ has the topology of (i.e.,~is homotopy equivalent to) a high-dimensional ball; therefore its \emph{fundamental group}, $\pi_1(\fcomplex)$, is trivial.
\end{enumerate}
\end{theorem}
%}

In what follows, we will use a number of notions from combinatorial topology; some of these 
we will recall along the way, but others we will only describe informally or leave undefined 
and instead refer the reader to standard textbooks for further background (in particular, we refer 
the reader to \cite[Appendix~4.7]{Bjorner:Oriented-matroids-1999} and \cite{Hudson:Piecewise-linear-topology-1969} for background on \emph{regular cell complexes}, \emph{shellability}, and \emph{piecewise linear balls and spheres}, to \cite{Stillwell:Classical-topology-and-combinatorial-group-1993} for background on the fundamental group of cell complexes, and to 
\cite{Hudson:Piecewise-linear-topology-1969,Munkres:Elements-of-algebraic-topology-1984}
for background on \emph{dual complexes}; we will provide more detailed references for specific 
results below).

Theorem~\ref{thm:flip-complex} follows from a result of Orden and Santos~\cite{Orden:The-polytope-of-non-crossing-graphs-on-a-planar-2005}; we are grateful to F.~Santos for bringing this reference to our attention. In fact, Orden and Santos show something stronger: There exists a simple polytope $\ospoly=\ospoly(P)$ and a face $F$ of $\ospoly$ such that $\fcomplex$ can be taken to be the complement of the star of $F$ in $\ospoly$.

Before becoming aware of the work of Orden and Santos, we found a different proof
of Theorem~\ref{thm:flip-complex} that starts out by considering  the simplicial complex 
$\tcomplex=\tcomplex(P)$ whose faces are the sets of pairwise non-crossing edges (line segments) spanned by $P$. This complex $\tcomplex$ is shown to be a \emph{shellable simplicial ball} (by an argument based on constrained Delaunay triangulations), and $\fcomplex$ is then constructed as the \emph{dual complex} of $\tcomplex$. We hope that this alternative proof of Theorem~\ref{thm:flip-complex} is of some independent interest and present it in Sections~\ref{sec:tcomplex} and~\ref{sec:dual-complex} below.  Before that, in Section~\ref{sec:top-to-swap}, we show how to derive the Elementary Swap Theorem from Theorem~\ref{thm:flip-complex}. 



%\subsection{(Closed) Walks in the Flip Graph and Permutations of Labels}
\subsection{From Topology to the Elementary Swap Theorem}
\label{sec:top-to-swap}


%\comment{
In this section we use Theorem~\ref{thm:flip-complex} to prove the Elementary Swap Theorem.
We begin by defining elementary walks.
A \emph{walk} in the flip graph is a sequence $T_0,T_1,\dots,T_k$ of triangulations (possibly with repetitions) such that $T_{i-1}$ and $T_i$ differ by a flip. We will refer to $T_0$ and $T_k$ as the start and the end of the walk, respectively.
%, and call $k$ the length of the walk. 
A walk is \emph{closed} if it starts and ends at the same triangulation.  
If $w_1$ and $w_2$ are walks such that the end of $w_1$ equals the start of $w_2$ then 
we can define their \emph{composition} $w_1w_2$ in the obvious way. 
Furthermore, if  $w=(T=T_0,T_1,\ldots,T_k)$ is a walk, we will use the notation 
$w^{-1}=(T_k,T_{k-1},\dots,T_0)$ for the \emph{inverse walk}.

Fix a triangulation $T_0$. An \emph{elementary quadrilateral walk} is a closed walk of the form 
$wzw^{-1}$, where $z$ is an elementary $4$-cycle in the flip graph, and $w$ is a walk 
from $T_0$ to some triangulation on $z$. An \emph{elementary pentagonal walk} is defined analogously, with $z$ an elementary $5$-cycle.
%}


It is straightforward to check the effect of these elementary walks on labellings:
\begin{lemma}
\label{lem:elementary-walk-permutations} 
Let $(T_0,\ell)$ be a labelled triangulation.
%\begin{enumerate}
%\item 
An elementary quadrilateral walk does not permute the labels.
%\item 
An elementary pentagonal walk swaps the labels of two edges ($e$ 
and $f$ in Figure~\ref{4-5cycle}(b))  and leaves all other labels fixed; this corresponds exactly 
to the notion of an elementary swap introduced earlier.
%\end{enumerate}
\end{lemma}

Another operation that does not affect the permutation of labels induced by a closed walk is the following. A \emph{spur} $ww^{-1}$ starting and ending at $T$ is an arbitrary walk $w$ starting 
at $T$, immediately followed by the \emph{inverse walk}. If $w_1$ and $w_2$ are walks in the flip graph such that $w_1$ ends at a triangulation $T$ and $w_2$ 
starts there, and if $s$ is a spur at $T$, 
then we say that the walk $w_1 s w_2$ 
differs from $w_1 w_2$ by a \emph{spur insertion}
The inverse operation is called a \emph{spur deletion}.

\begin{lemma}\label{lem-spurs}
If two closed walks $w$ and $w'$ in the flip graph differ only by a finite number of spur insertions and deletions then they yield the same permutation of edge labels.
\end{lemma}
\begin{proof}
A flip immediately followed by its inverse flip has no effect on labels. The lemma follows by induction on the length of a spur and 
%on 
the number of spur insertions and deletions.
\end{proof}


By Lemmas~\ref{lem:elementary-walk-permutations} and \ref{lem-spurs}, the Elementary Swap Theorem directly reduces to the following, which we prove using 
Theorem~\ref{thm:flip-complex}:

\begin{proposition}
\label{topological EST}
Let $w$ be a closed walk in the flip graph starting and ending at $T_0$. 
Then, %$w$ is spur equivalent to the composition of finitely many elementary walks.
up to a finite number of spur insertions and deletions, $w$ can be written as the composition of 
finitely many elementary walks.
\end{proposition}

%\remove{ % moved earlier
%The key to proving Proposition~\ref{topological EST} is the following topological result.
%In what follows, we will use a number of notions from combinatorial topology; some of these 
%we will recall along the way, but others we will only describe informally or leave undefined 
%and instead refer the reader to standard textbooks for further background (in particular, we refer 
%the reader to \cite[Appendix~4.7]{Bjorner:Oriented-matroids-1999} and \cite{Hudson:Piecewise-linear-topology-1969} for background on \emph{regular cell complexes}, \emph{shellability}, and \emph{piecewise linear balls and spheres}, to \cite{Stillwell:Classical-topology-and-combinatorial-group-1993} for background on the fundamental group of cell complexes, and to 
%\cite{Hudson:Piecewise-linear-topology-1969,Munkres:Elements-of-algebraic-topology-1984}
%for background on \emph{dual complexes}; we will provide more detailed references for specific 
%results below).
%\begin{theorem}
%\label{thm:flip-complex}
%Let $P$ be a set of $n$ points in the plane.
%There is a regular cell complex $\fcomplex=\fcomplex(P)$, which we call the \emph{flip complex}, 
%with the following properties:
%\begin{enumerate}
%\item The $1$-skeleton $\skel_1(\fcomplex)$ of $\fcomplex$ equals the flip graph of $P$ (i.e., the vertices of $\fcomplex$ correspond to the triangulations of $P$, and the edges of $\fcomplex$ to flips);
%\item The fundamental group $\pi_1(\fcomplex)$ is trivial. 
%%$\fcomplex$ is contractible; as a consequence, the \emph{fundamental group} 
%%$\pi_1(\fcomplex)$ is trivial.
%\end{enumerate}
%\end{theorem}
%
%Theorem~\ref{thm:flip-complex} follows from a result of Orden and Santos~\cite{Orden:The-polytope-of-non-crossing-graphs-on-a-planar-2005}; we are grateful to F.~Santos for bringing this reference to our attention. In fact, Orden and Santos show something stronger: There exist a simple polytope $\ospoly=\ospoly(P)$ and a face $F$ of $\ospoly$ such that $\fcomplex$ can be taken to be the complement of the star of $F$ in $\ospoly$.
%
%Before becoming aware of the work of Orden and Santos, we found a different proof
%of Theorem~\ref{thm:flip-complex} that starts out by considering  the simplicial complex 
%$\tcomplex=\tcomplex(T)$ whose faces are the sets of pairwise non-crossing edges (line segments) spanned by $P$. This complex $\tcomplex$ is shown to be a \emph{shellable simplicial ball} (by an argument based on constrained Delaunay triangulations), and $\fcomplex$ is then constructed as the \emph{dual complex} of $\tcomplex$. We hope that this alternative proof of Theorem~\ref{thm:flip-complex} is of some independent interest and present it in Section~\ref{sec:dual-complex} below.
%
%To derive Proposition~\ref{topological EST} from Theorem~\ref{thm:flip-complex}, we 
%need the following characterization of the $2$-dimensional faces of the flip 
%complex $\fcomplex$, which we will also prove in Section~\ref{sec:dual-complex}.
% 
%\begin{lemma}
%\label{lemma-flip-complex-2-faces}
%The $2$-dimensional faces of $\fcomplex$ are quadrilaterals and pentagons corresponding to the elementary cycles in the flip graph: 
%For every elementary cycle $z$ (of length $4$ or $5$) in the flip graph, $\fcomplex$
%contains a unique $2$-dimensional cell (a quadrilateral or a pentagon) whose boundary is $z$.
%\end{lemma}
%} 

\begin{proof}%[Proof of Prop.~\ref{topological EST}, using Thm.~\ref{thm:flip-complex} and Lem.~\ref{lemma-flip-complex-2-faces}]
%\comment{
We use the well-known fact that the fundamental group of a cell complex can be defined \emph{combinatorially} in terms of closed walks in the $1$-skeleton and this definition is equivalent to the usual topological definition in terms of continuous loops, see \cite[Chap.~7]{Seifert:A-Textbook-of-Topology-1980} or \cite[Chap.~4]{Stillwell:Classical-topology-and-combinatorial-group-1993}. In particular, in a cell complex with trivial fundamental group any two closed walks in the $1$-skeleton starting at the same vertex are related by a finite number of spur insertions, deletions and so-called 2-cell relations.  

We describe the combinatorial definition of the fundamental group of the flip complex $\fcomplex$ in detail. By Theorem~\ref{thm:flip-complex}, the $1$-skeleton of $\fcomplex$ is the flip graph of $P$. Fix a \emph{base triangulation} $T_0$, and, for every triangulation $T$, fix a walk $p_T$ from $T_0$ to $T$. Given two triangulations $T_1,T_2$ that differ by a flip, we form the closed walk $w_{T_1,T_2}$ in the flip graph, called a \emph{generating walk}, that goes from $T_0$ to $T_1$ along $p_{T_1}$, then flips to $T_2$, and then returns to $T_0$ along $p^{-1}_{T_2}$.  It is easy to see that, up to a finite number of spur insertions and deletions, every closed walk starting and ending at $T_0$ can be written as a composition of generating walks.

We say that walks $w$ and $w'$ are \emph{$2$-cell related} if we can express them as
$w=w_1w_2$ and $w'=w_1z w_2$, where $z$ is a closed walk traversing the boundary 
of a $2$-cell (an elementary cycle) exactly once in either orientation. 
%A priori, this is not a symmetric relation, but $w_1w_2$ and $w_1zz^{-1}w_2$ differ only by the spur $zz^{-1}$.
Notice that $w_1w_2$ and $w_1zz^{-1}w_2$ differ only by the spur $zz^{-1}$, hence, up to spur insertion and deletion, being $2$-cell related is symmetric. 
 
Also, notice the \emph{precomposition property}: if $w$ and $w'$ are $2$-cell related as above and if $w$ is precomposed with the
closed walk $w_1z w_1^{-1}$ then the result $w''=(w_1z w_1^{-1})w=w_1z(w_1^{-1}w_1)w_2$ differs from $w'$ only by the spur $w_1^{-1}w_1$. By Theorem~\ref{thm:flip-complex}, a boundary of a $2$-cell is an elementary $4$- or $5$-cycle and so the walk $w_1z w_1^{-1}$ above is an elementary walk.

Two walks in the flip graph are called equivalent if they differ by a finite number of spur insertion and/or deletions and by applying a finite number of $2$-cell relations. It is not hard to check that this defines an equivalence relation, and the fundamental group $\pi_1(\fcomplex)$ is given as the set of equivalence classes of closed walks starting and ending at $T_0$. 

%Triviality of the fundamental group
By Theorem~\ref{thm:flip-complex}, the fundamental group of the flip complex $\fcomplex$ is trivial. This translates into the fact that every closed walk starting and ending at $T_0$ is equivalent to the trivial walk. By 
%the preceding discussion, 
the precomposition property, this means that, 
up to a finite number of spur insertions and deletions, every closed walk is a composition of finitely many elementary walks.
%}
\end{proof}

