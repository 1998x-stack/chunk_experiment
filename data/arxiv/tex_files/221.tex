% !TEX root = bottom.tex

%%%%%%%%%%%%%%%%%%%%%%%%%%%%%%%%%%%%%%%%%%%%%%%%%%%%%%
\section{Conclusions and outlook}
\label{sec:con}
%%%%%%%%%%%%%%%%%%%%%%%%%%%%%%%%%%%%%%%%%%%%%%%%%%%%%%

By combining a lattice QCD vetted in-medium heavy-quark potential with a realistic anisotropic hydrodynamics simulation for the bulk matter created in a heavy-ion collision, we have studied the nuclear modification factor $R_{AA}$ for the bottomonium ground and first excited states. 
The functional form and temperature dependence of the lattice-vetted heavy quark potential lead to a significantly reduced dependence of the Upsilon $R_{AA}$ on the values of the shear to entropy ratio of the QGP. From a phenomenological point of view this is a welcome finding, since it strengthens the position of bottomonium as a genuine dynamical thermometer of the QGP, as originally envisioned.

We have found that our estimates for the suppression of primordial bottomonium show the best agreement with experimental results at RHIC energies, where they reproduce the measured data within errors. The larger imaginary part in the lattice-vetted potential compared to previously used model potentials leads to a stronger suppression, which in the case of RHIC energies is the reason for the excellent reproduction of the experimental data. At LHC run1 we find, on the other hand, hints of a tendency to overestimate the suppression compared to the data, while at 5.02 TeV we clearly see too much suppression when using the lattice-vetted potential. Two reasons for this behavior immediately come to mind. The first is related to the physics mechanism we assume underlies the imaginary part of the potential, the second one is related to the fact that no regeneration component has been included in our current study.

Our calculation uses a simple implementation of bottomonium suppression as discussed in Sec.~\ref{sec:dynamics}. By using the imaginary part of the potential directly in the Schr\"odinger equation we have assumed that it arises solely from gluo-dissociation of the heavy-quarkonium. Studies of bottomonium in perturbative pNRQCD have shown however that the imaginary part contains both contributions from gluo-dissociation and Landau damping, which leads to the excitation of bottomonium states without decay of the $Q\bar{Q}$ pair. Eventually we will need to understand how to disentangle these two contributions, since the latter, as shown in exploratory studies in the context of open-quantum systems  \cite{Rothkopf:2013kya,Kajimoto:2017rel} leads to a weaker suppression. I.e. our inability to disentangle the underlying mechanisms of the imaginary part have led us to assign it fully to a loss channel which may overestimate the actual suppression.

One possible way for future improvement of the proper potential based real-time description of the heavy quarkonium evolution, would be to utilize the framework of open-quantum systems. As was proposed e.g.~in \cite{Akamatsu:2011se}, the imaginary part of the heavy-quark potential, which arises in the description of the unequal time correlation functions of Eq.~\eqref{Eq:ForwProp} is unraveled into either a stochastic dynamics on the level of the Schr\"odinger equation or equivalently into a master equation for the density matrix of quarkonium states. Recent developments in this field include a formulation of a Lindblad type coupled dynamics for color singlet and octet degrees of freedom \cite{Akamatsu:2014qsa,Akamatsu:2015kaa,DeBoni:2017ocl}, which has been evaluated in a perturbative setting in \cite{Brambilla:2016wgg}. A first attempt to describe bottomonium dynamics was presented in \cite{Rothkopf:2013kya}, however, the medium evolution used in this paper was not yet realistic.

In addition, we have estimated the suppression of bottomonium states without including a possibility for regeneration. While for charmonium there are clear indications for the presence of a regeneration components in the observed yields \cite{Rapp:2017chc}, no such signals have been firmly established for bottomonium yet. One hint towards the onset of regeneration could be the small change in the bottomonium suppression when going from 2.76 TeV to 5.02 TeV. While the lattice-vetted potential predicts suppression that increases with beam energy, the data appears to not decrease as rapidly as suggested by our calculations.  This would allow for the addition of a regeneration component to approach the data from below, however, we are not yet in a position to quantitatively estimate the magnitude of this effect.

In the end it will be necessary to explore both paths to come to a robust phenomenological understanding of the observed yields. The model of bottomonium suppression needs to be made more flexible in order to accommodate the different underlying physics mechanisms inherent in the imaginary part, e.g. via a stochastic Sch\"odinger equation and the inclusion of regeneration via a rate equation framework is desirable.


\acknowledgements

A.R.~acknowledges fruitful discussions with Y. Akamatsu. A.R.~was supported by the DFG Collaborative Research Centre SFB
1225 (ISOQUANT).  B.K.~and~M.S. were supported by the U.S. Department of Energy, Office of Science, under Award No. DE-SC0013470.