% !TEX root = bottom.tex

%%%%%%%%%%%%%%%%%%%%%%%%%%%%%%%%%%%%%%%%%%%%%%%%%%%%%%
\section{Non-equilibrium corrections, static Schr\"odinger equation solution, and real-time QGP evolution}
\label{sec:dynamics}
%%%%%%%%%%%%%%%%%%%%%%%%%%%%%%%%%%%%%%%%%%%%%%%%%%%%%%

With the real and imaginary parts of the potential specified, we can now turn to the method used to fold the dynamical evolution of the full three-dimensional QGP evolution together with information about the real and imaginary parts of the resulting binding energies.  In order to do so, however, we will must first make an extension of the result presented in the previous section to the case of a QGP with momentum-space anisotropies. This is necessary for a realistic model of QGP evolution and quarkonia suppression.  The simplest form for the soft-particle distribution function that can be used to take into account QGP momentum-space anisotropies is a generalization of an isotropic phase-space distribution which is squeezed or stretched along one direction in momentum space, defined by $\hat{\bf n}$, with a parameter $-1 < \xi < \infty$.  In a heavy-ion collision the direction $\hat{\bf n}$ can be identified with the beam line direction, $\hat{\bf n} = \hat{\bf z}$.  The resulting one-particle distribution function is given by the following spheroidal ``Romatschke-Strickland'' form \cite{Romatschke:2003ms,Romatschke:2004jh}
%
\begin{equation}
f({\bf p},\xi,\Lambda)\equiv f_{\rm iso}(\sqrt{{\bf p}^2+\xi({\bf p\cdot \hat{n}})^2}/\Lambda) \, ,
\label{distansatz}
\end{equation}
%
where $\Lambda$ is the transverse temperature scale and $f_{\rm iso}$ is the isotropic thermal distribution function associated with the soft degrees of freedom in the QGP.

%%%%%%%%%%%%%%%%%%%%%%%%%%%%%%%%%%%%%%%%%%%%%%%%%%%%%%
\subsection{Anisotropic Debye mass}
\label{ssec:amd}
%%%%%%%%%%%%%%%%%%%%%%%%%%%%%%%%%%%%%%%%%%%%%%%%%%%%%%

The Debye mass, which we use to evaluate the heavy-quark potential in our approach, is self consistently computed from the dynamical evolution of the soft bulk matter. The conventional isotropic $m_D$ is defined via an integral over the isotropic distribution function
%
\be \label{eq:Debye}
m_D^2 = -\frac{g^2}{2\pi^2} \int_0^\infty d p \,
  p^2 \, \frac{d f_{\rm iso}}{d p} ~,
\ee
% 
where $p^2 \equiv {\bf p}^2 = p_\perp^2 +p_z^2$. Since lattice studies are restricted to a thermal and isotropic state, we have to use additional input to incorporate the effects of a QGP momentum-space anisotropy on the potential itself.  For this purpose, we use the results contained in Sec.~2.2 of Ref.~\cite{Strickland:2011aa} where it was shown that, empirically, one can incorporate the momentum-space anisotropy parameter $\xi$ into the potential by making the Debye mass depend on the angle with respect to the beam line direction.  The result obtained was
%
\be
\left(\frac{\mu}{m_D}\right)^{-4} =  
1 + \xi\left(a - \frac{2^b(a-1)+(1+\xi)^{1/8}}{(3+\xi)^b}\right) 
\left(1 + \frac{c(\theta) (1+\xi)^d}{(1+ e\xi^2)} \right) \, ,
\label{eq:muparam}
\ee
%
with $a=16/\pi^2$, $b = 1/2$, $d=3/2$, $e=1/3$, and
%
\be
c(\theta) = \frac{3 \pi ^2 \cos (2 \theta)+\left(9+4 \sqrt{3}-4 \sqrt{6}\right) \pi ^2+64 \left(\sqrt{6}-3\right)}{4
   \left(\sqrt{3} \left(\sqrt{2}-1\right) \pi ^2-16 \left(\sqrt{6}-3\right)\right)} \, .
\ee
%
With these values in hand, we simply tabulate the real and imaginary parts of the potential functions with respect to $m_D$ and $r$, and then replace $m_D \rightarrow \mu$ in the isotropic potential to obtain the corresponding anisotropically modified potential. The parameter $\xi$ sets the degree of momentum-space anisotropy, with $-1 < \xi < 0$ corresponding to a prolate distribution, with more momentum along the beam line direction than transverse to it and $\xi >0$ corresponding to an oblate distribution.  The anisotropy parameter is dynamically tied to the bulk QGP evolution using anisotropic hydrodynamics (aHydro) which we will discuss shortly \cite{Martinez:2010sc,Florkowski:2010cf,Martinez:2010sd,Ryblewski:2010bs,Ryblewski:2011aq,Martinez:2012tu,Ryblewski:2012rr,Strickland:2014pga,Nopoush:2014pfa,Alqahtani:2015qja,Alqahtani:2016rth,Alqahtani:2017jwl,Alqahtani:2017tnq}.

%%%%%%%%%%%%%%%%%%%%%%%%%%%%%%%%%%%%%%%%%%%%%%%%%%%%%%
\subsection{Solving the Schr\"odinger equation with a complex potential}
\label{ssec:schrod}
%%%%%%%%%%%%%%%%%%%%%%%%%%%%%%%%%%%%%%%%%%%%%%%%%%%%%%

Since the potential no longer has full rotational symmetry one has to go beyond a single (radial) dimension when solving the Schr\"odinger equation.  With the assumed form for the one-particle distribution function, only one symmetry direction is broken (spheroidal symmetry), and one can, in principle, simply use a two-dimensional solver. Here we have chosen to make use of a full three-dimensional solution since codes for this are already available even for complex-valued potentials \cite{Strickland:2009ft,Margotta:2011ta}.  For this purpose, the real-time Schr\"odinger equation is solved in imaginary time. The grid spacing for $\Upsilon(nS)$ states is chosen as $a=0.15$ fm on $N=256^3$ regularly-spaced grid points. Due to the larger size of p-wave bottomonia, our grid spacing for the $\chi_{b}(mP)$ states is set to $a=0.175$ fm. Starting from a randomized three-dimensional wavefunction, we evolve in imaginary time until the ground state wavefunction converges to within a given tolerance.  Using ``snapshots'' of the wavefunction stored during the imaginary-time evolution, we can project out the low-lying excited states \cite{Strickland:2009ft}.  Additionally, by fixing the symmetry (symmetric vs anti-symmetric) of the initial random wavefunction we can select the s-wave or p-wave states independently \cite{Strickland:2009ft}.  In this way we can obtain the wavefunctions of the 1s, 2s, 3s, 1p, and 2p Upsilon states and in turn we can compute the real and imaginary parts of their respective energies. For more details concerning the numerical method, we refer the reader to Refs.~\cite{Strickland:2009ft,Margotta:2011ta}.


%%%%%%%%%%%%%%%%%%%%%%%%%%%%%%%%%%%%%%%%%%%%%%%%%%%%%%
\subsection{Anisotropic hydrodynamics equations and initial conditions}
\label{ssec:3p1}
%%%%%%%%%%%%%%%%%%%%%%%%%%%%%%%%%%%%%%%%%%%%%%%%%%%%%%

To proceed, we assume that the underlying bulk one-particle distribution function is well approximated by Eq.~(\ref{distansatz}) at all points in spacetime. In addition, we assume that the bulk evolution is well described using hydrodynamical degrees of freedom, such as energy density, pressures, and viscous corrections.  This assumption has been tested by comparing the predictions of anisotropic hydrodynamics to exact solutions of the Boltzmann equation~\cite{Florkowski:2013lza,Florkowski:2013lya,Bazow:2013ifa,Florkowski:2014sfa,Florkowski:2014sda,Denicol:2014xca,Denicol:2014tha,Nopoush:2014qba,Molnar:2016gwq,Martinez:2017ibh} (see also \cite{Damodaran:2017ior} for more comparisons to kinetic theory) where it has been shown to reproduce the exact kinetic evolution even far from equilibrium.  The equations used herein are obtained using the zeroth and first moments of the Boltzmann equation in the relaxation-time approximation.  For details about the dynamical equations used and their physical content, we refer the reader to Sec.~IV of Ref.~\cite{Ryblewski:2015hea}.

In order to solve the aHydro dynamical equations, one has to make a reasonable assumption about the initial conditions at the initial longitudinal proper-time for the hydrodynamic evolution, $\tau = \tau_0$.  For our initial conditions, we take the system to be isotropic in momentum space ($\xi=0$) with zero transverse flow and Bjorken flow in the longitudinal direction.  In the transverse plane, the initial energy density is computed from a linear combination of smooth Glauber wounded-nucleon and binary-collision profiles with a binary mixing factor of $\alpha = 0.15$.  In the longitudinal direction, we used a ``tilted'' profile with a central plateau and Gaussian tails resulting in a profile function of the form $\rho(\varsigma) \equiv \exp \left[ - (\varsigma - \Delta \varsigma)^2/(2 \sigma_\varsigma^2) \, \Theta (|\varsigma| - \Delta \varsigma) \right]$, with \mbox{$\varsigma = \arctanh(z/t)$} being spatial rapidity.  The parameters entering the longitudinal profile function were fitted to the pseudorapidity distribution of charged hadrons with the results being $\Delta\varsigma = 2.3$ and $\sigma_{\varsigma} = 1.6$.  The first quantity sets the width of the central plateau and the second sets the width of the Gaussian ``wings''.  The resulting initial energy density at a given transverse position ${\bf x}_\perp$ and spatial rapidity $\varsigma$ was computed using ${\cal E} \propto (1-\alpha) \rho(\varsigma) \left[ W_A({\bf x}_\perp) g(\varsigma) + W_B({\bf x}_\perp) g(-\varsigma)\right] + \alpha \rho(\varsigma) C({\bf x}_\perp)$, where $W_{A,B}({\bf x}_\perp)$ is the wounded nucleon density for nuclei $A$ and $B$, $C({\bf x}_\perp)$ is the binary collision density, and $g(\varsigma)$ is the ``tilt function''.  The tilt function $g(\varsigma) = 0$ if $\varsigma < -y_N$, $g(\varsigma) = (\varsigma+y_N)/(2y_N)$ if $-y_N \leq \varsigma \leq y_N$, and $g(\varsigma)=1$ if $\varsigma > y_N$ where $y_N = \log(2\sqrt{s_{NN}}/(m_p + m_n))$ is the nucleon momentum rapidity \cite{Bozek:2010bi}.

%%%%%%%%%%%%%%%%%%%%%%%%%%%%%%%%%%%%%%%%%%%%%%%%%%%%%%
\subsection{Computing the suppression}
\label{ssec:sup}
%%%%%%%%%%%%%%%%%%%%%%%%%%%%%%%%%%%%%%%%%%%%%%%%%%%%%%

The solution of the 3+1d aHydro dynamical equations gives us hard momentum scale $\Lambda$, the momentum-anisotropy parameter $\xi$, and, consequently, the anisotropic Debye mass $\mu$ as a function of proper time, transverse coordinate 
${\bf x}_\perp$, and spatial rapidity $\varsigma$.  Solving the Schr\"odinger equation in addition provides the real and 
imaginary parts of the binding energy of a given state as a function of $\Lambda$ and $\xi$.  Folding these together 
gives us the real and imaginary parts of the binding energy 
as a function of proper time, transverse coordinate ${\bf x}_\perp$, and 
spatial rapidity $\varsigma$: $\Re[E_{\rm bind}(\tau,{\bf x}_\perp,\varsigma)]$ and 
$\Im[E_{\rm bind}(\tau,{\bf x}_\perp,\varsigma)]$, respectively.  Full details of the method used can be found in Ref.~\cite{Strickland:2011aa}.  Here we summarize the important points.

We use the imaginary part of the binding energy to provide information about the decay rate of a given state and the real part of the binding energy to tell us when the state becomes completely unbound.  The imaginary part of the binding energy can be related to the decay rate, $\Gamma$, using $\partial_\tau n_i = 
-\Gamma_i n_i$, where $i$ indexes the state in question, giving $\Gamma_i = -2 \Im[E_i]$ \cite{Strickland:2011aa}.
Finally, using the fact $\Im[E_{\rm bind}] = - \Im[E]$ we obtain 
%
\be
\Gamma(\tau,{\bf x}_\perp,\varsigma) = 
\left\{
\begin{array}{ll}
2 \Im[E_{\rm bind}(\tau,{\bf x}_\perp,\varsigma)]  & \;\;\;\;\; \Re[E_{\rm bind}(\tau,{\bf x}_\perp,\varsigma)] >0 \\
10\;{\rm GeV}  & \;\;\;\;\; \Re[E_{\rm bind}(\tau,{\bf x}_\perp,\varsigma)] \le 0 \\
\end{array}
\right.
\ee
%
The value of 10 GeV in the second case was chosen in order to quickly suppress states which are
fully unbound and, in practice, the results do not depend significantly on this value as long as it is large enough to quickly disassociate the state under consideration.

We then integrate the instantaneous decay rate, $\Gamma$, obtained in this manner
over proper-time to extract the dimensionless logarithmic suppression factor 
%
\be
\zeta(p_T,{\bf x}_\perp,\varsigma) \equiv \Theta(\tau_f-\tau_{\rm form}(p_T)) \int_{{\rm max}(\tau_{\rm form}(p_T),\tau_0)}^{\tau_f} 
d\tau\,\Gamma(\tau,{\bf x}_\perp,\varsigma) \, ,
\label{eq:zeta}
\ee
%
where $\tau_{\rm form}(p_T)$ is the lab-frame formation time of the state in question.
The formation time of a state in its local rest frame 
can be estimated by the inverse of its vacuum binding energy~\cite{Karsch:1987uk}.
For the formation times for the $\Upsilon(1S)$, $\Upsilon(2S)$, $\Upsilon(3S)$, $\chi_{b}(1P)$, $\chi_{b}(2P)$, $\chi_{b}(3P)$ and states we take $\tau_{\rm form}^0$ = 0.2 fm/c, 0.4 fm/c, 0.6 fm/c, 0.4 fm/c, 0.6 fm/c, and 0.6 fm/c, respectively.

Our choice for the initial proper time $\tau_0$ for plasma evolution is $\tau_0 =$ 0.3 fm/c at both RHIC and LHC energies.
The final time, $\tau_f$, is defined to be the proper time when the local effective temperature drops below $T_d = 192$ MeV. At this energy density, plasma screening effects are assumed to decrease
rapidly due to the transition to the hadronic phase and the widths of the states will become
approximately equal to their vacuum widths.  

From $\zeta$ obtained via Eq.~(\ref{eq:zeta}) one can directly compute the suppression factor $R_{AA}$ using
%
\be
R_{AA}(p_T,{\bf x}_\perp,\varsigma) = e^{-\zeta(p_T,{\bf x}_\perp,\varsigma)} \, .
\ee
%
Next, we must average over transverse momenta, implementing the appropriate cuts.
For this purpose, we assume that 
all states have an approximately $1/E_T^4$ spectrum.  Integrating over transverse momentum given $p_T$-cuts $p_{T,\rm min}$ and
$p_{T,\rm max}$ we obtain the $p_T$-cut suppression factor
%
\be
R_{AA}({\bf x}_\perp,\varsigma) \equiv \frac{\int_{p_{T,\rm min}}^{p_{T,\rm max}} 
dp_T^2 \, R_{AA}(p_T,{\bf x}_\perp,\varsigma) /(p_T^2 + M^2)^2}{\int_{p_{T,\rm min}}^{p_{T,\rm max}}dp_T^2/(p_T^2 + M^2)^2} \, .
\ee
%
For implementing cuts in centrality we compute $R_{AA}$ for finite impact parameter $b$ and map centrality
to impact parameter in the standard manner. Finally, to compare with experimental observations we average $R_{AA}({\bf x}_\perp,\varsigma)$
over ${\bf x}_\perp$.  For this operation, we use a production probability distribution which proportional to the overlap density~\cite{Strickland:2011aa}
%
\be
\langle R_{AA}(\varsigma) \rangle \equiv 
\frac{\int_{{\bf x}_\perp} \! d{\bf x}_\perp \, n_{AA}({\bf x}_\perp)\,%
R_{AA}({\bf x}_\perp,\varsigma)}{\int_{{\bf x}_\perp} \! d{\bf x}_\perp \,% 
n_{AA}({\bf x}_\perp)} \, .
\label{eq:geoaverage}
\ee
