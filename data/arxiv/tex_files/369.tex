%%%%
%% Load the class. Put any options that you want here (see the documentation
%% for the list of options). The following are samples for each type of
%% thesis:
%%
%% Note: you can also specify any of the following options:
%%  logo: put a University of Edinburgh logo onto the title page
%%  frontabs: put the abstract onto the title page
%%  deptreport: produce a title page that fits into a Computer Science
%%      departmental cover [not sure if this actually works]
%%  singlespacing, fullspacing, doublespacing: choose line spacing
%%  oneside, twoside: specify a one-sided or two-sided thesis
%%  10pt, 11pt, 12pt: choose a font size
%%  centrechapter, leftchapter, rightchapter: alignment of chapter headings
%%  sansheadings, normalheadings: headings and captions in sans-serif
%%      (default) or in the same font as the rest of the thesis
%%  [no]listsintoc: put list of figures/tables in table of contents (default:
%%      not)
%%  romanprepages, plainprepages: number the preliminary pages with Roman
%%      numerals (default) or consecutively with the rest of the thesis
%%  parskip: don't indent paragraphs, put a blank line between instead
%%  abbrevs: define a list of useful abbreviations (see documentation)
%%  draft: produce a single-spaced, double-sided thesis with narrow margins
%%
%% For a PhD thesis -- you must also specify a research institute:
\documentclass[phd,lfcs,twoside,logo]{infthesis}

%%% PACKAGES %%%

%% FOR INDEXING %%
\usepackage{makeidx}
\makeindex
%% FOR INDEXING %%

\usepackage[ddmmyyyy]{datetime}
\usepackage[numbers]{natbib}
\usepackage{color}
\usepackage[usenames,dvipsnames]{xcolor}
\usepackage{hyperref}
\usepackage{amssymb}
\usepackage{amsmath}
\usepackage{amsthm}
\usepackage{proof}
\usepackage{stmaryrd}
\usepackage{verbatim}
\usepackage[all]{xy}
\usepackage{multirow}
\usepackage{bigints}
\usepackage{enumerate}
\usepackage{nccmath}
\usepackage{csquotes}

%%% PACKAGES %%%

%%% MACROS %%%

\newtheorem{theorem}{Theorem}[section]
\newtheorem{proposition}[theorem]{Proposition}
\newtheorem{corollary}[theorem]{Corollary}
\newtheorem{lemma}[theorem]{Lemma}
\newtheorem{example}[theorem]{Example}
\newtheorem{conjecture}[theorem]{Conjecture}
\theoremstyle{definition}
\newtheorem{definition}[theorem]{Definition}

\newcommand{\dcomment}[1]{\colorbox{lightgray}{{\fontsize{0.32cm}{0.32cm}$#1$}}}
\newcommand{\dhide}[1]{\colorbox{white}{{\fontsize{0.33cm}{0.32cm}$#1$}}}
\newcommand{\dscomment}[1]{\colorbox{lightgray}{{\fontsize{0.2cm}{0.32cm}$#1$}}}
\newcommand{\dshide}[1]{\colorbox{white}{{\fontsize{0.22cm}{0.32cm}$#1$}}}

\newcommand{\vertbar}{\,\vert\,}

\newcommand{\SCCompC}{SCCompC}

\newcommand{\mkrule}[2]{\infer{#1}{#2}}
\newcommand{\mkrulelabel}[3]{\infer[#3]{#1}{#2}}

\newcommand{\defeq}{\stackrel{\mbox{\tiny def}}=}

\newcommand{\return}[1]{\mathtt{return~}#1}
\newcommand{\doto}[4]{#1 \mathtt{~to~} #2 \mathtt{~in}_{#3} \mathtt{~} #4}

\newcommand{\inl}[2]{\mathtt{inl}_{#1} \mathtt{~} #2}
\newcommand{\inr}[2]{\mathtt{inr}_{#1} \mathtt{~} #2}
\newcommand{\seminl}[2]{\mathsf{inl}_{#1} \mathtt{~} #2}
\newcommand{\seminr}[2]{\mathsf{inr}_{#1} \mathtt{~} #2}
\newcommand{\seminj}[2]{\mathsf{inj}_{#1} \mathtt{~} #2}
\newcommand{\inj}[2]{\mathtt{inj}_{#1} \mathtt{~} #2}
\newcommand{\absurd}[2]{\mathtt{case~} #2 \mathtt{~of}_{#1} \mathtt{~} ()}
\newcommand{\absurdsf}[2]{\mathsf{case~} #2 \mathsf{~of}_{#1} \mathsf{~} ()}
\newcommand{\case}[4]{\mathtt{case~} #1 \mathtt{~of}_{#2} \mathtt{~} (#3,#4)}
\newcommand{\casesf}[4]{\mathsf{case~} #1 \mathsf{~of}_{#2} \mathsf{~} (#3,#4)}

\newcommand{\algop}{\mathtt{op}}
\newcommand{\sigalgop}{\mathsf{op}}

\newcommand{\SemNat}{\mathbb{N}}
\newcommand{\Nat}{\mathsf{Nat}}

\newcommand{\Exception}{\mathsf{Exc}}
\newcommand{\State}{\mathsf{St}}
\newcommand{\Location}{\mathsf{Loc}}
\newcommand{\Value}{\mathsf{Val}}
\newcommand{\Character}{\mathsf{Chr}}
\newcommand{\Updates}{\mathsf{Upd}}

\newcommand{\U}{\mathsf{U}}
\newcommand{\F}{\mathsf{F}}

\newcommand{\zero}{\mathtt{zero}}
\newcommand{\suc}[1]{\mathtt{succ~} #1}
\newcommand{\succc}{\mathtt{succ}}
\newcommand{\natrec}[4]{\mathtt{nat\text{-}elim}_{#1}(#2, #3,#4)}

\newcommand{\Bool}{\mathsf{Bool}}
\newcommand{\true}{\mathsf{true}}
\newcommand{\false}{\mathsf{false}}

\newcommand{\IdType}[3]{\mathsf{Id}_{#1}(#2,#3)}
\newcommand{\refl}[2]{\mathtt{refl}_{} \mathtt{~} #2}
\newcommand{\pathind}[6]{\mathtt{eq\text{-}elim}_{{#1}}(#2,#3,#4,#5,#6)}
\newcommand{\pathindeff}[5]{\mathtt{eq\text{-}elim}_{{#1}}(#2,#3,#4,#5)}

\newcommand{\semrefl}[1]{\mathtt{r}_{#1}}
\newcommand{\semind}[2]{\mathtt{i}_{#1,#2}}

\newcommand{\funsection}{\mathsf{s}}

\newcommand{\thunk}[1]{\mathtt{thunk~} #1}
\newcommand{\thunka}[2]{\mathtt{thunk}_{#1} \mathtt{~} #2}
\newcommand{\force}[2]{\mathtt{force}_{#1} \mathtt{~} #2}

\newcommand{\dtensorlet}[3]{#2 \mathtt{~to~} #1 \mathtt{~in~} #3}
\newcommand{\dtensorleta}[4]{#2 \mathtt{~to~} #1 \mathtt{~in}_{#4} \mathtt{~} #3}

\newcommand{\lj}[2]{#1\vdash #2}
\newcommand{\zj}[3]{#1\vdash #2:#3}
\newcommand{\vj}[3]{#1\vdash #2:#3}
\newcommand{\cj}[3]{#1\vdash #2: #3}
\newcommand{\hj}[4]{#1 \vertbar #2 \vdash #3:#4}

\newcommand{\ljeq}[3]{#1\vdash #2 = #3}
\newcommand{\zeq}[4]{#1 \vdash #2 = #3 : #4}
\newcommand{\veq}[4]{#1 \vdash #2 = #3:#4}
\newcommand{\ceq}[4]{#1 \vdash #2 = #3:#4}
\newcommand{\heq}[5]{#1 \vertbar #2 \vdash #3 = #4:#5}

\newcommand{\unit}{\star}
\newcommand{\pair}[2]{\langle #1,#2 \rangle}
\newcommand{\proj}[2]{\mathsf{proj}_{#1}\,#2}
\newcommand{\pmatch}[4]{\mathtt{pm~} #1 \mathtt{~as~} #2 \mathtt{~in}_{#3} \mathtt{~} #4}
\newcommand{\pmatchsf}[4]{\mathsf{pm~} #1 \mathsf{~as~} #2 \mathsf{~in}_{#3} \mathsf{~} #4}
\newcommand{\hompmatch}[5]{\mathtt{pm}_{#1} \mathtt{~} #2 \mathtt{~as~} #3 \mathtt{~in}_{#4} \mathtt{~} #5}
\newcommand{\runas}[4]{#1~\mathtt{as}~#2~\mathtt{in}_{#3}~#4}

\newcommand{\id}{\mathsf{id}} 
\newcommand{\comp}{\circ}

\newcommand{\Id}{\mathsf{Id}}

\newcommand{\sem}[1]{\llbracket #1 \rrbracket}
\newcommand{\vsem}[1]{\llbracket #1 \rrbracket}
\newcommand{\csem}[1]{\llbracket #1 \rrbracket}
\newcommand{\hsem}[1]{\llbracket #1 \rrbracket}

\newcommand{\efftrans}[2]{\llparenthesis \, #1 \, \rrparenthesis_{#2}}

\newcommand{\fst}[1]{\mathtt{fst}\,#1}
\newcommand{\snd}[1]{\mathtt{snd}\,#1}
\newcommand{\semfst}[1]{\mathsf{fst}\,#1}
\newcommand{\semsnd}[1]{\mathsf{snd}\,#1}

\newcommand{\Set}{\mathsf{Set}}
\newcommand{\Alg}{\mathsf{Alg}}
\newcommand{\Mod}{\mathsf{Mod}}
\newcommand{\CPO}{\mathsf{CPO}}
\newcommand{\Law}{\mathsf{Law_c}}

\newcommand{\Fam}{\mathsf{Fam}}
\newcommand{\CFam}{\mathsf{CFam}}

\newcommand{\ul}[1]{\underline{#1}}

\newcommand{\ia}[1]{\{#1\}}


\newcommand{\sproj}[4]{\mathsf{proj}_{#1;#2 : #3;#4}}
\newcommand{\ssubst}[5]{\mathsf{subst}_{#1;#2 : #3;#4;#5}}

\newcommand{\pl}[1]{{\sc{#1}}}

%% FOR INDEX %%
\setlength{\columnsep}{1cm}
%% FOR INDEX %%

\addtocontents{toc}{\protect\setcounter{tocdepth}{2}}

%%% MACROS %%%

%% Information about the title, etc.
\title{
Fibred Computational Effects
}
\author{Danel Ahman}

%% If the year of submission is not the current year, uncomment this line and 
%% specify it here:
%\submityear{2017}

%% Optionally, specify the graduation month and year:
% \graduationdate{February 1786}

%% Specify the abstract here.
\abstract{%
%
We study the interplay between \emph{dependent types} and \emph{computational effects}, 
two important areas of modern programming language research. On the one hand, dependent types underlie proof assistants such as Coq and functional programming languages such as Agda, Idris, and F*, providing programmers a means for encoding detailed specifications of program behaviour using types. On the other hand, computational effects, such as exceptions, nondeterminism, state, I/O, probability, etc., are integral to all widely-used programming languages, ranging from imperative languages, such as C, to functional languages, such as ML and Haskell. Separately, dependent types and computational effects both come with rigorous mathematical foundations, dependent types in the effect-free setting and computational effects in the simply typed setting. Their \emph{combination}, however, has received much less attention and no similarly exhaustive theory has been developed. 
%
In this thesis we address this shortcoming by providing a comprehensive treatment of the combination of  these two fields, and demonstrating that they admit a mathematically elegant and natural combination.

Specifically, we develop a core effectful dependently typed language, eMLTT, based on Martin-L\"{o}f's intensional type theory and a clear separation between (effect-free) values and (possibly effectful) computations familiar from simply typed languages such as Levy's Call-By-Push-Value and Egger et al.'s Enriched Effect Calculus. 
 A novel feature of our language is the \emph{computational $\Sigma$-type}, which we use to give a uniform treatment of type-dependency in sequential composition.
%
In addition, we define and study a class of category-theoretic models, called \emph{fibred adjunction models}, that are suitable for defining a sound and complete interpretation of eMLTT. Specifically, fibred adjunction models naturally combine standard category-theoretic models of dependent types (split closed comprehension categories) with those of computational effects (adjunctions). We discuss and study various examples of these models, including a domain-theoretic model so as to extend eMLTT with general recursion.

We also investigate a dependently typed generalisation of the algebraic treatment of computational effects by showing how to extend eMLTT with \emph{fibred algebraic effects} and their \emph{handlers}.
In particular, we specify fibred algebraic effects using a dependently typed generalisation of Plotkin and Pretnar's effect theories, enabling us to capture precise notions of computation such as state with location-dependent store types and dependently typed update monads.
For handlers, we observe that their conventional term-level definition leads to unsound program equivalences becoming derivable in languages that include a notion of homomorphism, such as eMLTT. To solve this problem, we propose a novel type-based treatment of handlers via a new computation type,  the \emph{user-defined algebra type}, which pairs a value type (the carrier) with a family of value terms (the operations). This type internalises Plotkin and Pretnar's insight that handlers denote algebras for a given equational theory of computational effects. We demonstrate the generality of our type-based treatment of handlers by showing that their conventional term-level presentation can be routinely derived, and this treatment provides a useful mechanism for reasoning about effectful computations. Finally, we show that these extensions of eMLTT  can be soundly interpreted in a fibred adjunction model based on the families of sets fibration and models of Lawvere theories.
}

%% Now we start with the actual document.
\begin{document}

%% First, the preliminary pages
\begin{preliminary}

%% This creates the title page
\maketitle

%% Lay summary
\begin{laysummary}
%
\emph{Dependent types} provide a lightweight and modular means to integrate programming and formal program verification. In particular, the types of programs written in dependently typed programming languages (Agda, Idris, F*, etc.) can be used to express specifications of program correctness. These specifications can vary from being as simple as requiring the divisor in the division function to be non-zero, to as complex as specifying the correctness of compilers of industrial-strength languages. Successful compilation of a program then guarantees that it satisfies its type-based specification.

While dependent types allow many runtime errors to be eliminated by rejecting erroneous programs at compile-time, dependently typed languages are yet to gain popularity in the wider programming community. One reason for this is their limited support for \emph{computational effects}, an integral part of all widely used programming languages, ranging from imperative languages, such as C, to functional languages, such as ML and Haskell.
For example, in addition to simply turning their inputs to outputs, programs written in these programming languages can raise exceptions, access computer's memory, communicate over a network, render images on a screen, etc.

Therefore, if 
dependently typed programming languages are to truly live up to their promise of seamlessly integrating  programming and formal program verification, we must first understand how to properly account for computational effects in such languages.
While there already exists work on this topic, ingredients needed for a comprehensive theory are generally missing. For example, foundations are often not settled; available effects may be limited; or effects may not be treated systematically.

In this thesis we address these shortcomings by providing a comprehensive treatment of the combination of dependent types and general computational effects. Specifically, we i) define a core effectful dependently typed programming language; ii) study its category-theoretic denotational semantics;
and iii) demonstrate how to extend the algebraic treatment of computational effects (including the handlers of algebraic effects) to the dependently typed setting, enabling us to uniformly specify a wide range of computational effects in terms of operations and equations. 
In particular, in this thesis we demonstrate that  dependent types and computational effects admit a mathematically natural combination, in which well-known concepts and results from the simply typed setting can be reused and adapted, but which also reveals new and interesting programming language features and corresponding mathematical structures. 

\end{laysummary}

%% Acknowledgements
\begin{acknowledgements}
I would like to thank the many people who have been important  
to my PhD studies. This long journey would not have been 
possible without their support and guidance.

First and foremost, I would like to express my sincerest gratitude to 
my supervisor Gordon Plotkin for all the guidance, support, and
encouragement he provided during my time in Edinburgh. 
Although we often found ourselves working in different geographic locations 
and timezones, he always found time to comment on my work and 
come up with useful suggestions, and provide a great deal of invaluable feedback.

I would also like to thank my second supervisor Alex Simpson for discussions 
and suggestions concerning my research in the earlier stages of my PhD studies. 
I am also grateful to Ian Stark for taking over the second supervisor's duties when 
Alex moved to Ljubljana. I would also like to thank Phil Wadler for agreeing to be 
on my yearly review meeting panels, and for all the feedback and constructive criticism he 
provided. 

I am also grateful to Paul Levy and James Cheney for agreeing to 
be my examiners, for spending many hours of their time carefully reading through a thesis of this length, 
and for all their feedback that helped to greatly improve 
the final version of this thesis.

I am very grateful to LFCS and its members for providing an excellent research 
environment. In particular, I would like to thank my fellow PhD students of IF 5.32, past and present, for many 
interesting discussions about work and life in general: 
Alyssa Alcorn, Simon Fowler, Weili Fu, Jiansen He, Ben Kavanagh, Craig McLaughlin, Fabian Nagel, 
Shayan Najd, Jack Williams, and Jakub Zalewski. I am also grateful for the support and friendship of many other LFCS students: Daniel Hillerstr\"{o}m, Theodoros Kapourniotis, Karoliina Lehtinen, Kristjan Liiva, Einar Pius, Panagiotis Stratis, and Marcin Szymczak.
I would also like to thank the occupants of IF 5.28, past and present, for having time 
to discuss various aspects of research, both mine and theirs: Bob Atkey, Brian Campbell, Sam Lindley, James McKinna, and Garrett Morris.

I would also like to thank the members of the MSP group at the University of Strathclyde for many interesting visits, seminars, and reading groups; these provided a useful and much needed distraction  from my studies. In particular, I would like to thank Neil Ghani with whom I and Gordon co-authored the FoSSaCS'16 paper on dependent types and computational effects which became the basis of the work presented in this thesis. I would also like to thank Fredrik Forsberg and Conor McBride for many interesting discussions about dependent types, both conventional and cubical. More generally, I would like to thank all the various Scottish programming languages and semantics research groups for making Scotland such a wonderful research environment, in particular, by organising events such as SPLS, ScotCats, and CLAP Scotland.

I am also grateful to the past and present members of the Logic and Semantics Group at the 
Institute of Cybernetics (now at the Department of 
Software Science) at the Tallinn University of Technology for hosting my visits, and for organising great events such as the Estonian Winter School in Computer Science and the Estonian Computer Science Theory Days. Specifically, I would like to thank James Chapman and Tarmo Uustalu for their encouragement and support regarding my PhD research, and also for our continued  collaborations on directed containers and related topics.

I would also like to thank Mihai Budiu and Nikhil Swamy for inviting me to do internships 
at Microsoft Research, and Gordon for putting me in contact with them in the 
first place. During these two internships, I learnt a lot about conducting research and working in a large   corporate environment. Both internships also greatly broadened my knowledge about the more practical aspects of programming language research.

I am also grateful to Ohad Kammar, Justus Matthiesen, and Kayvan Memarian for many 
interesting discussions about programming language research, hiking, cycling, and life in general, 
and for accommodating me during my visits to Cambridge.

Special thanks are reserved to the many people I shared the Bruntsfield Gardens flat during my four and half year stay in Edinburgh, and who helped to make it a true home away from home: Barbara Balazs, Krzysztof Geras, Michaela Keil, Stephen McGroarty, Zs\'{o}fia Nem\'{e}nyi, Toomas Remmelg, and Michael Wilson. 

I am also forever indebted to Ege Ello and Veiko Vostrjakov who have offered immense emotional support over the past years, provided me with a place to stay when visiting Estonia, and more generally have taken me in as a member of their family.

Finally, I would like to acknowledge the financial support of the University of Edinburgh (through the Principal's Career Development PhD Scholarship) and the Archimedes Foundation (in collaboration with the Estonian Ministry of Education and Research, through the scholarship program Kristjan Jaak). I am also grateful for the travel funding provided by the LFCS, the Archimedes Foundation, the ACM SIGPLAN PAC, the ERDF funded Coinduction project, and the EUTypes Cost Action. I am also grateful for C\u{a}t\u{a}lin Hri\c{t}cu and the ERC SECOMP project for funding me while this thesis was under examination  and during the preparation of its final version.
\end{acknowledgements}

%% Next we need to have the declaration.
\standarddeclaration

%% Finally, a dedication (this is optional -- uncomment the following line if
%% you want one).
% \dedication{To my mummy.}

%% Create the table of contents
\tableofcontents

%% If you want a list of figures or tables, uncomment the appropriate line(s)
% \listoffigures
% \listoftables

\end{preliminary}

%%%%%%%%
%% Include your chapter files here. See the sample chapter file for the basic
%% format.

%\include{chap1}
% \include{chap2}
%% ... etc ...

%%% INTRODUCTION %%%
\include{chapter1}

%%% PRELIMINARIES %%%
\include{chapter2}

%%% CORE LANGUAGE %%%
\include{chapter3}

%%% FIBRED ADJUNCTION MODELS %%%
\include{chapter4}

%%% SOUNDNESS AND COMPLETENESS %%%
\include{chapter5}

%%% ALGEBRAIC EFFECTS %%%
\include{chapter6}

%%% HANDLERS OF ALGEBRAIC EFFECTS %%%
\include{chapter7}

%%% CONCLUSIONS AND FUTURE WORK %%%
\include{chapter11}

%%%%%%%%
%% Any appendices should go here. The appendix files should look just like the
%% chapter files.
\appendix

\include{chapter1_appendix}
\include{chapter4_appendix}
\include{chapter5_appendix}
\include{chapter6_appendix}

%% Choose your favourite bibliography style here.
\bibliographystyle{abbrv}

%% If you want the bibliography single-spaced (which is allowed), uncomment
%% the next line.
% \singlespace

%% Specify the bibliography file. Default is thesis.bib.
\bibliography{references}

%\newpage

%%% SUBJECT INDEX %%%

\renewcommand\indexname{Notation and Subject Index}

\cleardoublepage
\phantomsection
\addcontentsline{toc}{chapter}{Notation and Subject Index}
\printindex

%% ... that's all, folks!
\end{document}
