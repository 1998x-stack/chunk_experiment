% !TEX root = bottom.tex

%%%%%%%%%%%%%%%%%%%%%%%%%%%%%%%%%%%%%%%%%%%%%%%%%%%%%%
\section{Results}
\label{sec:results}
%%%%%%%%%%%%%%%%%%%%%%%%%%%%%%%%%%%%%%%%%%%%%%%%%%%%%%

In this section, we present our numerical results and compare them to data from RHIC and LHC. The background dynamics discussed in the previous section have been determined for three different values of the shear viscosity to entropy ratio varied around the proposed quantum bound $\eta/s = 1/4\pi$, with the initial temperature tuned so that the soft particle multiplicity is held fixed. The corresponding initial temperature values for the appropriate beam energies are listed in Table \ref{table1} and lead to hydrodynamic evolution, which reproduces light particle spectra and azimuthal flow well. %We assume an initially isotropic plasma in momentum space which is then taken to be anisotropic once time begins evolving. (This should go into the previous section)

%============================================================================
\begin{table}[t!]
\begin{tabular}{|c|c|c|c|}
\hline
\backslashbox{$\;4\pi\eta/s$\Big.}{\Big.$\sqrt{s_{NN}}\;$} & \ 0.2 TeV\ \ & \  2.76 TeV \ \ & \  5.02 TeV\ \  \\
\hline
\hline
1 & 0.442 & 0.552 & 0.641\\
\hline
2 & 0.440 & 0.546 & 0.632\\
\hline
3 & 0.439 & 0.544 & 0.629\\
\hline
\end{tabular}
\caption{Values of initial temperature $T_0$ in GeV used for RHIC $0.2$ TeV, LHC run1 $2.76$ TeV, and LHC run2 $5.02$ TeV for the different values of the shear viscosity over entropy ratio considered herein.  In the aHydro simulations used, the QGP was assumed to be initially isotropic in momentum space.}
\label{table1}
\end{table}
%============================================================================

The survival probabilities of the bottomonium states are folded with the aHydro background evolution to obtain the primordial $R_{AA}$ for each state. The resulting values for LHC run1 are shown in Fig. \ref{fig:rawRK}. Due to the large suppression from the lattice-vetted heavy-quark potential, the curves for the $\chi_{b}(2P)$, $\Upsilon(3S)$, and $\chi_{b}(3P)$ states fall on top of each other.  For all of these states, they are completly suppressed in the interior of the fireball, however, there are always states at the edge of the plasma where the temperature/density is low where the states are largely unsuppressed.  This results in a kind of universal ``halo" survival probability which is related to the geometry of the fireball and its temperature profile.

%============================================================================
\begin{figure*}[t!]
\includegraphics[width=0.55\linewidth]{RawRAARK}
\caption{
(Color online) Primordial $R_{AA}$ for each bottomonium state as a function of the number of participants for the parameter sets of LHC run1.
}
\label{fig:rawRK}
\end{figure*}
%============================================================================

Since it is not the primordial bottomonium states themselves that are measured in the detector but instead the decay di-leptons, feed down must be factored to calculate the inclusive $R_{AA}$ for each state observed in experiment as described in the previous section. The relevant feed down fractions are obtained from averaging over $p_{T}$ from experimental feed down yields \cite{Woeri:2015hq} and are listed in Table \ref{feeddown}.

%============================================================================
\begin{table}[t!]
\begin{tabular}{|c|c|c|c|}
\hline
\multicolumn{4}{|c|}{Feed down fractions}\\
\hline
\hline
$\Upsilon(1S) \rightarrow \Upsilon(1S)$ & 0.668 & - & -\\
\hline
$\Upsilon(2S) \rightarrow \Upsilon(1S)$ & 0.086 & $\Upsilon(2S) \rightarrow \Upsilon(2S)$ & 0.604\\
\hline
$\Upsilon(3S) \rightarrow \Upsilon(1S)$ & 0.010 & $\Upsilon(3S) \rightarrow \Upsilon(2S)$ & 0.043\\
\hline
$\chi_{b}(1P) \rightarrow \Upsilon(1S)$ & 0.170 & - & -\\
\hline
$\chi_{b}(2P) \rightarrow \Upsilon(1S)$ & 0.051 & $\chi_{b}(2P) \rightarrow \Upsilon(2S)$ & 0.309\\
\hline
$\chi_{b}(3P) \rightarrow \Upsilon(1S)$ & 0.015 & $\chi_{b}(3P) \rightarrow \Upsilon(2S)$ & 0.044\\
\hline
\end{tabular}
\caption{Feed down fractions to the $\Upsilon(1S)$ state used in the determination of the final measured yields.}
\label{feeddown}
\end{table}
%============================================================================

%============================================================================
\begin{figure*}[h]
\centerline{
\includegraphics[width=0.65\linewidth]{PotentialCompare1S5020}
}
\caption{
(Color online) bottomonium suppression for the LHC run2 parameter set based on the perturbative model of \cite{Strickland:2009ft} (top three curves) and the lattice-vetted heavy-quark potential (bottom three curves).
Note that due to the large imaginary part in the lattice-vetted potential the suppression is consistently larger in than in the perturbative model potential. Interestingly we find that virtually no dependence on the chosen values for the shear viscosity to entropy ratio is observed with the lattice-vetted potential. This behavior is found both at RHIC and LHC energies.
}
\label{fig:inclusivecompare}
\end{figure*}
%============================================================================

The inclusive $\Upsilon(1S)$ $R_{AA}$ is presented in Fig. \ref{fig:inclusivecompare} both computed using the lattice-vetted heavy-quark potential (solid lines) as well as with the perturbative Bazow-Strickland model potential of Ref.\cite{Strickland:2011aa} (dashed lines). Two observations can be immediately made: First, the $R_{AA}$ obtained with the lattice-vetted potential lies consistently below the values obtained from the perturbative potential. The reason is that the former features a stronger imaginary part and thus the bottomonium states are more easily dissociated. Secondly we find that that $R_{AA}$ computed with the lattice-vetted heavy-quark potential is virtually independent of the $\eta/s$ parameter of the aHydro background evolution. This behavior is consistently observed both at RHIC and LHC energies and has important consequences for the role bottomonium can play as a probe of the QGP. The less the suppression depends on parameters other than the temperature, the more bottomonium can be used as a as dynamical thermometer of nuclear matter under extreme conditions.

In order to meaningfully compare to experimental results, we need to quantify the uncertainty in the used potential. When vetting the potential with lattice QCD simulations it was observed that the lattice potential values could be fitted with a Debye mass with around 10-20\% error. We, therefore, have repeated our calculations including a modest $\pm 15\%$ variation of $m_{D}$ which leads to an error estimate on the $R_{AA}$ theory curve quantifying the potential uncertainty. Since our results are essentially independent of the shear viscosity parameter, we set $4\pi\eta/s=2$ consistent with recent particle spectra fits \cite{Alqahtani:2017jwl}. 

%============================================================================
\begin{figure*}[h]
\includegraphics[width=0.55\linewidth]{RK_Npart_200_Compare}
\caption{
(Color online) $R_{AA}$ as a function of the number of participants compared to STAR data taken at RHIC as a function of $N_{\text{part}}$ for 0.2 TeV Au-Au collisions. We show both the previous estimates based on a model potential (light gray lines with grey error band) and our new results obtained using the lattice-vetted heavy-quark potential (solid line with blue error band). The error-band around our new central value corresponds to a $\pm 15\%$ variation in $m_{D}$ used to estimate the uncertainty in the determination of the potential. Due to the stronger imaginary part present in the lattice-vetted potential, the new estimates move to lower values and are in very good agreement with experimental observations.  %(Need the previous estimated results in here too with dashed lines)
}
\label{fig:STAR}
\end{figure*}
%============================================================================

We start the explicit comparisons of our computed yields to experiment with data obtained at RHIC by the STAR collaboration in Fig.~\ref{fig:STAR}. We show both the previous estimates based on a perturbative model potential (light gray lines with grey error band) and based on the lattice-vetted heavy quark potential (solid line with blue error band). As is expected from the behavior found in Fig.~\ref{fig:inclusivecompare} our new results lie systematically below those coming from the perturbative model potential. Interestingly, the shift to lower values brings the values for $R_{AA}$ into very good agreement with the measured RHIC data.

%============================================================================
\begin{figure*}[h]
\centerline{
\includegraphics[width=0.47\linewidth]{RK_Npart_2760}
\includegraphics[width=0.47\linewidth]{RK_Npart_5020}
}
\caption{
(Color online) $R_{AA}$ as a function of the number of participants $N_{\text{part}}$ compared to CMS data taken at the LHC for 2.76 TeV (left) and 5.02 TeV (right) Pb-Pb collisions. At 2.76 TeV (left) we find that within the uncertainty of the calculation we reproduce the ground state $R_{AA}$ with estimated having a slight tendency to take on lower values. The excited state $\Upsilon^\prime$ on the other hand is excellently captured. At 5.02 TeV (right) our estimates consistently lie below the experimental observations both for the ground state and the excited state, the deviation increasing with increasing centrality of the collisions.
}
\label{fig:RKNpart}
\end{figure*}
%============================================================================

Turning to LHC energies, in Fig.~\ref{fig:RKNpart} we compare our new results with $R_{AA}$ data obtained by the CMS collaboration at (left) 2.76 TeV and (right) 5.02 TeV as function of centrality. Previous estimates of the $\Upsilon(1S)$ ground state suppression for LHC run1, based on the model potential, reproduced the data points best when selecting values of $4\pi\eta/s=2$ with stronger shear leading to systematically lower values \cite{Krouppa:2015yoa}. A similar conclusion was reached for the $\Upsilon(2S)$ suppression. Now with the lattice-vetted heavy quark potential the dependence on the assumed value of the shear viscosity is essentially absent and the stronger imaginary part in the lattice-vetted potential induces slightly stronger suppression. For $\Upsilon(1S)$ our new estimates agree with the data within the still relatively large error bars but are slightly lower than the experimental data. On the other hand, the trend in the excited state data points is excellently reproduced, touching also the point at the lowest centrality bin, providing a better description overall than the perturbative model results (see Ref.~\cite{Krouppa:2017lsw} for a compilation of the prior results).

That being said, when moving to the higher energy of run2 at the LHC, we find that the trend of stronger suppression continues in our estimates of bottomonium suppression. At this energy the lattice-vetted model overpredicts the amount of suppression of both the $\Upsilon(1S)$ and $\Upsilon(2S)$. This means that now our $R_{AA}$ systematically overestimates the suppression for both states. The discrepancy is larger for more central collisions while for smaller $N_{\rm part}$ we still find reasonable agreement with the data.

%============================================================================
\begin{figure*}[h]
\centerline{
\includegraphics[width=0.47\linewidth]{ptUpsilons2760CMS}
\includegraphics[width=0.47\linewidth]{ptUpsilons5020CMS}
}
\caption{
(Color online) $R_{AA}$ as a function of transverse momentum $p_{T}$ compared to CMS data taken at the LHC for 2.76 TeV (left) and 5.02 TeV (right) Pb-Pb collisions. Also here at LHC run1 energies we find good agreement with the $\Upsilon(1S)$ data and an excellent reproduction of the $\Upsilon(2S)$ suppression. At 5.02 TeV the experimentally determined suppression appears slightly weaker than what our calculation predicts, with good agreement for $\Upsilon(1S)$ at small $p_T$ and the largest discrepancies around 8-10GeV.}
\label{fig:RKpT}
\end{figure*}
%============================================================================

In Fig. \ref{fig:RKpT} we plot the nuclear modification factor as function of $p_{T}$ integrated over all centrality classes at (left) 2.76 TeV and (right) 5.02 TeV. Similarly to our findings in terms of centrality at LHC run1 energies, the agreement here is best for $\Upsilon(2S)$ and acceptable for $\Upsilon(1S)$. There is a tendency visible to slightly overestimate the suppression when using the lattice-vetted potential in the computation.

%============================================================================
\begin{figure*}[h]
\centerline{
\includegraphics[width=0.47\linewidth]{RK_Rapidity2760}
\includegraphics[width=0.47\linewidth]{RK_Rapidity5020}
}
\caption{
(Color online) $R_{AA}$ as a function of spatial rapidity $y$ compared to CMS data taken at the LHC for 2.76 TeV (left) and 5.02 TeV (right) Pb-Pb collisions. Similar to the $p_T$ plots, at LHC run1 energies we observe good agreement with the $\Upsilon(1S)$ data and the $\Upsilon(2S)$ suppression is very well reproduced. At 5.02 TeV the experimentally determined suppression appears slightly weaker than what our calculation predicts.
}
\label{fig:RKrap}
\end{figure*}
%============================================================================

Finally, in Fig.~\ref{fig:RKrap} we show $R_{AA}$ as a function of spatial rapidity using the CMS cuts at 2.76 TeV (left) and 5.02 TeV (right) collisions at the LHC. The outcome of our calculation as can now be expected from the above discussion agrees well for the $\Upsilon(1S)$ at LHC run1 energies and gives an excellent account of the $\Upsilon(2S)$ suppression. At 5.02 TeV the trend of a overestimation of the suppression manifests itself again.

