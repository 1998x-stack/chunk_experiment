\section{Conclusion}
%
After more than fifty years, a coherent picture connecting the Roper resonance with the nucleon's first radial excitation has become visible.  Completing this portrait only became possible following
%
the acquisition and analysis of a vast amount of high-precision nucleon-resonance electroproduction data with single- and double-pion final states on a large kinematic domain of energy and momentum-transfer,
%
development of a sophisticated dynamical reaction theory capable of simultaneously describing all partial waves extracted from available, reliable data,
%
formulation and wide-ranging application of a Poincar\'e covariant approach to the continuum bound state problem in relativistic quantum field theory that expresses diverse local and global impacts of DCSB in QCD,
%
and the refinement of constituent quark models so that they, too, qualitatively incorporate these aspects of strong QCD.
%
In this picture:
\begin{itemize}
\setlength\itemsep{0em}
\item the Roper resonance is, at heart, the first radial excitation of the nucleon.
\item It consists of a well-defined dressed-quark core, which plays a role in determining the system's properties at all length-scales, but exerts a dominant influence on probes with $Q^2\gtrsim m_N^2$, where $m_N$ is the nucleon mass;
\item and this core is augmented by a meson cloud, which both reduces the Roper's core mass by approximately 20\%, thereby solving the mass problem that was such a puzzle in constituent-quark model treatments, and, at low-$Q^2$, contributes an amount to the electroproduction transition form factors that is comparable in magnitude with that of the dressed-quark core, but vanishes rapidly as $Q^2$ is increased beyond $m_N^2$.
\end{itemize}

These fifty years of experience with the Roper resonance have delivered lessons that cannot be emphasized too strongly. Namely, in attempting to predict and explain the QCD spectrum, one must: fully consider the impact of meson-baryon final-state interactions (MB\,FSIs), and the couplings between channels and states that they generate; and look beyond merely locating the poles in the $S$-matrix, which themselves reveal little structural information, to also consider the $Q^2$-dependences of the residues, which serve as a penetrating scale-dependent probe of resonance composition.

Moreover, the Roper resonance is not unusual.  Indeed, in essence, the picture drawn here is also applicable to the $\Delta$-baryon; and an accumulating body of experiment and theory indicates that almost all baryon resonances can be viewed the same way, \emph{viz}.\ as systems possessing a three-body dressed-quark bound-state core that is supplemented by a meson cloud, whose importance varies from state to state and whose observable manifestations disappear rapidly as the resolving power of the probe is increased.  In this connection, it is important to highlight that CLAS12 at the newly upgraded JLab will be capable of determining the electrocouplings of most prominent nucleon resonances at unprecedented photon virtualities: $Q^2\in [6,12]\,$GeV$^2$ \cite{E12-09-003, E12-06-108A}.  Consequently, the associated experimental program will be a powerful means of validating the perspective described herein.

Assuming the picture we've drawn is correct, then CLAS12 will deliver empirical information that can address a wide range of issues that are critical to our understanding of strong interactions, \emph{e.g}.: is there an environment sensitivity of DCSB; and are quark-quark correlations an essential element in the structure of all baryons?  As reviewed herein, existing experiment-theory feedback suggests that there is no environment sensitivity for the nucleon, $\Delta$-baryon and Roper resonance: DCSB in these systems is expressed in ways that can readily be predicted once its manifestation is understood in the pion, and this includes the generation of diquark correlations with the same character in each of these baryons.
%
Resonances in other channels, however, may contain additional diquark correlations, with different quantum numbers, and potentially be influenced in new ways by MB\,FSIs.  Therefore, these channels, and higher excitations, open new windows on sQCD and its emergent phenomena whose vistas must be explored and mapped if the most difficult part of the Standard Model is finally to be solved. 