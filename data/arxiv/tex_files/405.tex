
\subsection{Problem Formulation}
Consider a problem of detecting the presence of a point-like target
using an uniform linear array of N antennas.
For the detection, reflection %from $N$ pulses
is collected at the cell under test (CUT) and surrounding range cells, in which
data is assumed to compose of only noise
and referred to as secondary data.
%each from
%one antenna, is sampled and arranged in an $N \times (K+1)$ array whose column
%$\boldsymbol{z}_t$
%consists samples of range cell $t$, with $t = 0,1,\ldots,K$.
%Target's presence is sought at range cell $t = 0$, while the remaining data
%are from $K$ surrounding range cells and is assumed to compose of only noise.
%That data is referred to in a sequel as secondary data.
Target's return
%If a potential target appears, its
%return
 in an equivalent baseband form
is represented as
$\alpha \boldsymbol{p}$, where $\alpha$ is a complex scalar accounting for the combined effect of
a target's reflection and channel propagation
and $\boldsymbol{p}$ is an $N \times 1$ steering vector departing
%We assume that, for some reasons, $\boldsymbol{p}$ departs
from the nominal steering vector $\boldsymbol{s}$,
%an angle $\theta$
i.e.,
$\boldsymbol{s} = \left[1,\exp(j\theta),\ldots,\exp(j(N-1)\theta)\right]^T $
and
$\boldsymbol{p} = \left[1,\exp(j\phi),\ldots,\exp(j(N-1)\phi)\right]^T $, 
where
$(\theta - \phi)$ is unknown but 
 $(\theta - \phi) \in [-\beta,\beta]$ with $\beta$
  %$\theta$ is the constant phase shifting of $\boldsymbol{s}$ and
  % $\beta \in [-\pi/2,\pi/2]$
  is a known quantity expressing
  the discrepancy of $\boldsymbol{p}$ from $\boldsymbol{s}$.
  % how far the actual steering
%vector $\boldsymbol{p}$ departs from the nominal one $\boldsymbol{s}$.
%Regarding noise,
%superposition of all undesired echoes, called clutter, at one range cell is modelled as
Clutter at a range cell $\boldsymbol{c}_t$ is modelled as
a complex SIRP
with index $ t = 0$ indicating the CUT and
$t = 1,2,\ldots,K$ indicating surrounding range cells.
%We assume that clutter at the cell under test and surrounding range cells
% represented by
%$\boldsymbol{c}_t$ where $t = 0,1,2,\ldots,K$.
From the definition of a complex SIRP \cite{Conte87}, we have
\begin{equation}\label{noise_model}
      \boldsymbol{c}_t \thinspace=\thinspace s_t \boldsymbol{g}_t, \qquad t = 0,1,2,\ldots, K.
\end{equation}
where $s_t$, the
\textit{texture} component of $\boldsymbol{c}_t$,
is a real non-negative random variable with some 
distribution $f_s(s_t)$ and
$\boldsymbol{g}_t$, the
\textit{speckle} component of $\boldsymbol{c}_t$,
is an $N \times 1$ vector of zero mean and multivariate complex normal distribution
with normalized covariance matrix $\boldsymbol{M}$, i.e.,
% usually normalized according to
$\tr{\boldsymbol{M}} = N$.
We also assume that $\left \{ \boldsymbol{g}_0,\boldsymbol{g}_1,\ldots,\boldsymbol{g}_K  \right \}$
 are independent identically distributed
 %with Hermitian positive definite
 %covariance matrix and the
 and possess the circular symmetric property.
Note that $s_t$ and $\boldsymbol{g}_t$ are independent, 
$\left \{ s_t\right\}_{t = 0}^{t = K}$
may be correlated.
In practice, the distribution $f_s(s_t)$ is \textit{priori} unknown,
making it impossible to derive the probability density function (pdf)
of $\boldsymbol{c}_t$.
However,
if the illumination time is much shorter than
the de-correlation time of the \textit{texture} $s_t$
%it is well-known that
%the de-correlation time of
%\textit{texture} component is much longer than that of
% \textit{speckle} component \cite{Ward90};
%therefore,
we can consider $s_t$ as, as in this paper,
unknown deterministic parameters \cite{Conte_Aug02}.
%if the illumination time is shorter than the
%\textit{texture}'s correlation time.
% Modelling $s_t$ as random variables
% was found in detection problems when the illumination time
% is long \cite{Conte95}\cite{Conte98} or when
% there is great correlation in
% reflected signals observed at successive range cells, for example
% sea clutter  \cite{Watt90}.
% In this paper, we choose to consider $s_t$ as deterministic parameters.
 This assumption then leads to the independence of clutter
 at all range cells.

The detection problem can now be stated as a problem of binary hypotheses
\begin{equation}\label{hypotheses}
 \left \{
 \begin{array}{ll}
   H_0 : & \boldsymbol{z} \thinspace=\thinspace
\boldsymbol{c}_0,  \\
   H_1 : &  \boldsymbol{z} \thinspace=\thinspace
\alpha \boldsymbol{p} + \boldsymbol{c}_0,
 \end{array}
  \right.\
\end{equation}
where the null hypothesis $H_0$ and alternative hypothesis $H_1$ denote the cases of
clutter-only and signal plus clutter, respectively, and $\boldsymbol{z}$ denotes the equivalent baseband of received signal at the CUT.
The pdf of the observed data $\boldsymbol{z}$
can be expressed as
\begin{equation}\label{null_hp}
f_0\left(\boldsymbol{z}
 \right) \thinspace=\thinspace
 \frac{1}{\pi^N s_0^{2N}|\boldsymbol{C}|}
 \exp \left \{ -\boldsymbol{z}^H \frac{1}{s_0^2} \boldsymbol{C}^{-1}\boldsymbol{z} \right \}
\end{equation}
and
\begin{multline}\label{alt_hp}
f_1\left(\boldsymbol{z}
 \right) \thinspace=\thinspace \\
 \frac{1}{\pi^N s_0^{2N}|\boldsymbol{C}|}
 \exp \left \{ -\left(\boldsymbol{z} - \alpha \boldsymbol{p}\right)^H \frac{1}{s_0^2}
 \boldsymbol{C}^{-1}\left(\boldsymbol{z} - \alpha \boldsymbol{p}\right) \right \},
\end{multline}
where $\boldsymbol{C} = E[\boldsymbol{c}_t \boldsymbol{c}_t^H]$ the covariance matrix of radar clutter.
It's easy to see that $\boldsymbol{C} = E[s^2]\times \boldsymbol{M}$.
To maximise the detection probability
given a predetermined false alarm rate,
we employ the Neyman-Pearson criterion.
Due to the ignorance of
the clutter covariance matrix $\boldsymbol{C}$, \textit{texture} $s_0$, the steering vector
$\boldsymbol{p}$, and $\alpha$, we resort to a GLRT scheme,
replacing these nuisance parameters with their MLEs under each hypothesis
\begin{equation}\label{glrt}
  \frac{\max \limits_{\boldsymbol{p}} \max \limits_{\alpha} \max \limits_{s_0} \max \limits_{\boldsymbol{C}}
   f_1\left(\boldsymbol{z} \right)}
  {\max \limits_{s_0} \max \limits_{\boldsymbol{C}} f_0\left(\boldsymbol{z}\right)}
  \thinspace\mathop{\gtrless}_{H_0}^{H_1}\thinspace G_2,
\end{equation}
where $G_2$ is a threshold set for a predetermined false alarm rate.
For detection, the next logical step is to replace $\boldsymbol{C}$ with its MLE.
However, it is proved in \cite{Gini99} that
a closed-form of the
MLE of covariance matrix $\boldsymbol{C}$
does not exist.
%Another estimators for $\boldsymbol{C}$ then have been proposed,
%yielding detectors possessing CFAR w.r.t
%$f_s(s_t)$ or $\boldsymbol{M}$.
Hence, we
assumed known $\boldsymbol{C}$
in the following development
 and derive
a detector.
Later, $\boldsymbol{C}$ is replaced by some estimators
and properties of the resulting detectors
will be discussed. 