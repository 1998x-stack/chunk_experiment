%Examples

\section{Examples}

We end this paper by giving explicit examples for Combinatorics $1$,  $2$ and $3$. 
Let us begin with explaining our method to construct explicit bisections briefly introduced in 
\cite{bannai-tokunaga}. We here use notation and terminology in \cite[2.2.3]{bannai-tokunaga} freely.

Let $U_i \cong \CC^2, \, (i = 1, 2)$ be affine open sets of $\Sigma_2$ with coordinates
$(t, x)$ $(i = 1)$ and $(s, x')$ $(i = 2)$ such that $t = 1/s, \, x = x'/s^2$. Suppose that $E_{\mcQ, z_o}$ is
given by a Weierstrass equation
\[
E_{\mcQ, z_o}: y^2  = f_{\mcT}(t, x), \quad f_{\mcT}(t, x) = x^3 + b_2(t) x^2 + b_3(t)x + b_4(t),
\]
where $b_i \in \CC[t]$, $\deg b_i \le i$ and $f_{\mcT}$ defines the trisection $\mcT$ on $U_i$. Let
$P = (x_P(t), y_P(t)) \in E_{\mcQ, z_o}(\CC(t))$. Consider the line in $\bbA^2_{\CC(t)}$ through
$P$ defined by
\[
L_P : y = l_P(t, x), \quad l_P(t, x) = r(t)(x - x_P(t)) + y_P(t), \, r(t) \in \CC(t).
\]
Then $f_{\mcT}(t, x) - l_P(t,x)^2$ factors into of the form
$(x - x_P(t))g(t, x)$, $g(t, x) \in \CC(t)[x]$. Suppose that $x_P(t), y_P(t) \in \CC[t]$ and we  choose
$r(t) \in \CC[t]$ such that $g(t, x) \in \CC[t,x]$ is irreducible and  the total degree of $g$ is $2$. Then the conic
$C(r(t), P)$ given by $g(t, x) = 0$ is a contact conic to $\mcQ$. Moreover, if we put
\[
f^*_{\mcQ, z_o}C(r(t), P) = C^+ + C^-,
\]
we have (i) $C^{\pm}$ are bisections and (ii) (if we choose $\pm$ suitably) $s(C^+)  = - s_P$.
We construct the bisections $D_i$ in the previous sections in this way. Now we go on to construct 
examples for each combinatorics.



\begin{exmple}[{\bf Combinatorics 1}]\label{ex:comb1} \rm

Let $[T, X, Z]$ be homogeneous coordinates of $\PP^2$ and let
 $(t, x):=(T/Z, X/Z)$ be affine coordinates for $\CC^{2} = \PP^{2} \setminus \{ Z = 0 \} $.

\medskip

%\begin{enumerate}

%\item[(1-a)] We put the affine equations of $\mcE$ and ${\mcL}_{o}$ as follows:
\underline{{\sl Combinatorics 1-(a):}}  Consider $\mcE$ and $\mcL_o$ given by the affine equations:
%\begin{align*}
\[
\mcE : x^{2} - t^{3} - 3{t}^{2} - 2t = 0, \,\,
{\mcL}_{o} : x = 0.
\]
$E_{\mcQ, z_o}$ is given by a Weierstarss equation
\[
E_{\mcQ, z_o} : y^2 = x(x^{2} - t^{3} - 3{t}^{2} - 2t)
\]

%\end{align*}
 %$\mcE$ is smooth.
% Put $\mcQ = \mcE + {\mcL}_{o}$ and
Put $p_{1}=[0,0,1]$, $p_{2}=[-1, 0, 1]$, and $p_{3}=[-2, 0, 1]$.  Choose $[0, 1, 0]$ as $z_{o}$. Then 
$\varphi_{\mcQ, z_o}$ has $4$ singular fibers of type $\III$ and $E_{\mcQ , z_{o}}(\CC (t)) \cong D_{4}^{\ast} \oplus \ZZ /2 \ZZ$. $P_{\tau}$ is given by $(0, 0)$. We choose  $P_{0}, P_{1}, P_{2}, P_{3}$ for a basis of $D_{4}^{\ast}$-part  as follows:
 \begin{itemize}
 \item $\displaystyle {P_{{0}}\, := \,\left [ \left( \frac{-1-i}4 \right)  \left( -t-1+i \right) ^{2},
 \frac {\sqrt{2-2\,i}}8\, \left( -t-1+i \right)  \left( t+1+i \right) ^{2} \right]}$,

% \item ${\mcL}_{o}:x = 0$\\
%$\displaystyle P_{o} = \left( 0, 0 \right)$ \\
%\vspace{1mm}

\item  $\displaystyle{ P_{{1}}\, := \, \left [- \left(  \sqrt{2}-1 \right) t,t \left(  \sqrt{2}+t \right)  \sqrt{ \sqrt{2}-1}\right]}$.
\item $\displaystyle {P_{{2}}\, := \,\left [ \left( 1+i \right)  \left( t+1 \right) , \sqrt{-1-i} \left( t+1 \right) 
\mbox{} \left( -t-1+i \right) \right ]}$,

\item  $\displaystyle{ P_{{3}}\, := \, \left [-i \left(  \sqrt{2}+1 \right)  \left( t+2 \right) , \sqrt{i \left(  \sqrt{2}+1 \right) }
 \left( t+2+ \sqrt{2} \right)  \left( t+2 \right) \right ]}$,

%\item  $\displaystyle{ P_{{3}}\, := \, \left [- \left(  \sqrt{2}-1 \right) t,t \left(  \sqrt{2}+t \right)  \sqrt{ \sqrt{2}-1}\right]}$.

\end{itemize}
% For $P \in E_{\mcQ, z_o}(\CC(t))$, let $x_{P}(t)$ and $y_{P}(t)$ denote the $x$- and $y$- coordinates of $P$. 
 By straightforward 
 computation, we see that $\mcL_i: x - x_{P_i}(t) = 0$ are tangent lines for $\mcE$ through $p_i$ $(i = 1, 2, 3)$, while
 $x - x_{P_0}(t) = 0$ is a contact conic to $\mcQ$ through $z_o$,  and we infer that
 $\langle P_0, \, P_0 \rangle = 2$ and $\langle P_i , \, P_i \rangle = 1$. 
 

 
 
 Also by explicit computation we have
 \[
 P_4: = P_2 \dot{-}P_3 \dot{+} P_{\tau} =
 \left [ -(\sqrt {2}+1)t, \, -\frac{\sqrt{-1-i}}{2} (i +1 +  {i}\sqrt{2})(-\sqrt{2}+t)t \right ].
 \]
  Let $\mcL_4: x - x_{P_4}(t) = 0$. Then since $\mcL_4$ is a tangent line, $\langle P_4, P_4\rangle=1$ from Remark \ref{rem:obs-1}, which implies $\langle P_2, P_3\rangle=\frac{1}{2}$. We can compute the height pairing for $ \langle P_i, P_j\rangle$ ($\{i,j\}\subset \{0,1,2,3\})$ in a similar way
 and we infer that $[\langle P_i, P_j \rangle ]$ is the $D_4^*$ given in \S 2. 
 
 
 Then we see both $\mcQ + \sum_{i=1}^3\mcL_i$ and
$\mcQ + \sum_{i=2}^4$ have Combinatorics (1-a).

\medskip

 \begin{rem} {\rm The generators as above are computed by Ms. Emiko Yorisaki in her master's thesis 
 (\cite{yorisaki}).
} 
 \end{rem}

%\item[(1-b)] We put the affine equations of $0}\mcE$ and ${\mcL}_{o}$ as follows:

{\sl Combinatorics 1-(b)} Consider 
\begin{align*}
\mcE &: x^{2} - t^{3} - {t}^{2} = 0, \\
{\mcL}_{o} &: 2x - 3t - 3 = 0.
\end{align*}
Note that $\mcE$ has the node at $[0, 0, 1]$. Put $p_{1}=[-1, 0, 1]$, $p_{2}=[-3/4, 3/8, 1]$, and $p_{3}=[3, 6, 1]$, then $\mcE \cap {\mcL}_{o} = \{ p_{1}, p_{2}, p_{3} \}$. We put $\mcQ = \mcE + {\mcL}_{o}$ and choose $[0, 1, 0]$ as the distinguished point $z_{o}$. As we discussed in Section \ref{sec:3}, $E_{\mcQ , z_{o}}(\CC (t)) \cong (A_{1}^{\ast})^{\oplus 3}\oplus \ZZ /2 \ZZ$ and we can assume lines $\mcL_{i} \, (i=1, 2, 3)$ connecting the node of $\mcE$ and $p_{i} \, (i=1, 2, 3)$ are generators of $(A_{1}^{\ast})^{\oplus 3}$-part. Then we have the following coordinates:
%\begin{itemize}
\[
\begin{array}{cccc}
P_{\tau} = \left[ \frac{3t+3}{2}, 0 \right],  & 
 P_{1} = \left[ 0, \frac{\sqrt{6}t(t+1)}{2} \right],  &
 P_{2} =\left[ - \frac{t}{2}, \frac{\sqrt{2}t(4t+3)}{4} \right],  &
 P_{3} = \left[ 2t, \frac{\sqrt{-2}t(t-3)}{2} \right] 
 \end{array}
 \]
%\end{itemize} 
%\begin{center}
%\begin{tabular}{ll}
%Equations & Element of $E_{\mcQ , z_{o}}(\CC (t))$ \\ \hline
%${\mcL}_{o}:2x - 3t - 3 = 0$ &  \\
%$\mcL _{1}:x=0$ & $\displaystyle P_{1} = \left( 0, \frac{\sqrt{6}t(t+1)}{2} \right)$ \\
%$\mcL _{2}:2x+t=0$ & $\displaystyle P_{2} =\left( - \frac{t}{2}, \frac{\sqrt{2}t(4t+3)}{4} \right)$ \\
%$\mcL _{3}:x-2t=0$ & $\displaystyle P_{3} = \left( 2t, \frac{\sqrt{-2}t(t-3)}{2} \right)$ \\
%\end{tabular}
%\end{center}

Furthermore, by computation on $E_{\mcQ, z_{o}}(\CC(t))$, we have $P_{4}$, $P_{5}$, $P_{6}$, and $P_{7}$ as follows:
\begin{align*}
P_4 & := P_3 \dot{+} P_1 \dot{+} P_{\tau} = \left[ {\frac { \left( 1-2 \sqrt {-3} \right)  \left( 26t+
21+3 \sqrt {-3} \right) }{52}}, \frac{\sqrt{2}(\sqrt{3}+\sqrt{-1})(4t+3)(2t+3+\sqrt{-3})}{16} \right] \\
%
P_5 & := P_1 \dot{+} P_2 \dot{+} P_{\tau} = \left[ -(1+\sqrt {3})( 2\,t-3-3\,\sqrt {3}),  \frac{\left( 2+\sqrt {3} \right)  \left( t-3 \right)  \left( t-6-4\,\sqrt {3} \right)
}{\sqrt{2}} \right] \\
%
P_6 & := P_2 \dot{+} P_3 \dot{+} P_{\tau} = \left[ -2 \sqrt{-1}(t+1), \frac{(2+ \sqrt{-1})(t+1)(t+2)}{\sqrt{2}} \right]\\
%
P_7 & := P_3 \dot{-} P_1 \dot{+} P_{\tau} = \left[ {\frac { \left( 1+2 \sqrt {-3} \right)  \left( 26t+
21-3 \sqrt {-3} \right) }{52}}, - \frac{\sqrt{2}(\sqrt{3}-\sqrt{-1})(4t+3)(2t+3-\sqrt{-3})}{16} \right]
\end{align*}
%Let $f_{\mcQ, z_o}$ be an induced generically $2$-to-$1$ morphism from $S_{\mcQ, z_o}$ to $\PP^2$. Then we see that $ f_{\mcQ , z_{o}}(s_{P_{i}}) \, (i=4, 5, 6, 7)$ are tangent lines to $\mcE$. In fact, the affine equation of each $f_{\mcQ , z_{o}}(s_{P_{i}})$ corresponds to $x$ coordinate of $P_{i}$. 
Let $f_{\mcQ, z_o}$ be as before and $\mcL_i := f_{\mcQ , z_{o}}(s_{P_{i}}) \, (i=4, 5, 6, 7)$.
Put
%\begin{align*}
\[
\mcB^1 := \mcQ + \sum_{i=4}^6 \mcL_i,\,\,\, 
%f_{\mcQ , z_{o}}(s_{P_{4}}) + f_{\mcQ , z_{o}}(s_{P_{5}}) + f_{\mcQ , z_{o}}(s_{P_{6}}), \\
\mcB^2 := \mcQ + \sum_{i=5}^7 \mcL_i.
\]
%f_{\mcQ , z_{o}}(s_{P_{5}}) + f_{\mcQ , z_{o}}(s_{P_{6}}) + f_{\mcQ , z_{o}}(s_{P_{7}}).
%\end{align*}
Then we have a Zariski pair $(\mcB^1 , \mcB^2 )$.
%\end{enumerate}

%ex2

\end{exmple}
\begin{exmple}[{\bf Combinatorics 2, 3-(b)}]\rm
We keep the same affine equations of $\mcE$, ${\mcL}_{o}$, ${\mcL}_{1}$, ${\mcL}_{2}$, and ${\mcL}_{3}$ as Example \ref{ex:comb1} 1-(b). Choose $[0, 1, 0]$ as the distinguished point $z_{o}$. As we discussed in Section \ref{sec:3}, we have a contact conic $\mcC$ to $\mcQ$ such that (i) $\mcC:= f_{\mcQ, z_o}(C^+)$ and (ii) $P_{C^+} = [2]P_1$. 

We put ${\mcB}^{i} := \mcQ + \mcC + {\mcL}_{i} \, (i = 1, 2, 3)$. From \ref{thm:comb-1}, we see that both of $({\mcB}^{1}, {\mcB}^{2})$ and $({\mcB}^{1}, {\mcB}^{3})$ are Zariski pairs having Combinatorics 2.

Next we will give explicit example of Zariski triple with Combinatorics 3-(b). Using the same basis and coordinates given  in Example \ref{ex:comb1} 1-(b), we obtain $\mcQ_1, \mcQ_2, \mcQ_3$ by explicit calculations as follows:
\begin{itemize}
\item $\mcQ_1=\displaystyle[2]P_{1}=\left[ \frac{{t}^{2}+9t+9}{6}, \frac{\sqrt {6}t \left( {t}^{2}-9\,t-9 \right)}{36}  \right]$
\item $\mcQ_2=\displaystyle[2]P_{2}=\left[\frac{1}{8}\left(t^2+16t+16 \right), \frac{\sqrt{2}}{32}\left(t+2\right)\left(t^2-16t-16\right) \right]$
\item $\mcQ_3=\displaystyle[2]P_{3}=\left[-\frac{1}{2}(t^2+t+1), \frac{\sqrt{2}\,i}{4} (t+2)(t^2-t-1)\right]$
\end{itemize}
%\begin{center}
%\begin{tabular}{ll}
%Equations & Element of $E_{\mcQ , z_{o}}(\CC (t))$ \\ \hline
%${\mcC}:{t}^{2}+9t-6x+9=0$ & $\displaystyle[2]P_{1}=\left( \frac{{t}^{2}+9t+9}{6}, \frac{\sqrt {6}t \left( {t}^{2}-9\,t-9 \right)}{36}  \right)$
%\end{tabular}
%\end{center}

We use the method given at the beginning of this section and construct bisections $D_1, \ldots, D_5$ and contact conics $\mcC_1, \ldots, \mcC_5$ corresponding to $-s_{\mcQ_1}$, $-s_{\mcQ_2}$, $-s_{\mcQ_3}$. 
The equations  can be calculated by using the data in the following table:
\begin{center}
\begin{tabular}{c|c|c}
conic & rational point & $r(t)$\\
\hline
$\mcC_j, (j=1,2,3)$ & $\mcQ_1$ & $\frac{1}{\sqrt{6}}t+b_j$\\
$\mcC_4$ & $\mcQ_2$ & $\frac{\sqrt{2}}{4}t+b_4$\\
$\mcC_5$ & $\mcQ_3$ & $-\frac{i}{\sqrt{2}}t+b_5$
\end{tabular}
\end{center}

The explicit equations for $C_{j} \, (j = 1, 2, 3), C_{4}$, and $C_{5}$ to $\mcE + {\mcL}_{o}$ become as follows:
\begin{align*}
\mcC _{j} : & {b_{j}}\,({b_{j}} + 6 \sqrt{6})\, t^2 -2 \sqrt{6}\, {b_{j}}\, tx + 6\, x^2 + 3\, {b_{j}}\, (3\,{b_{j}}+2 \sqrt{6})\, t - 6\, {b_{j}}^2 x + 9\, {b_{j}}^2=0, \\
\mcC _{4} : & (2\, {b_{4}}^2 +  30 \sqrt{2}\, {b_{4}} + 33)\, t^2 - 8 ( \sqrt{2}\, {b_{4}} -1)\, tx \\
&+ 16\, x^2 + 16 ( 2\, {b_{4}}^2 + 4 \sqrt{2}\, {b_{4}} + 3 )\, t - 8 ( 2\, {b_{4}}^2 -1)\, x + 16 ( 2\, {b_{4}}^2 + 2 \sqrt{2}\, {b_{4}} + 1 )=0, \\
\mcC _{5} : & ( {b_{5}}^2 - 6 )\, t^2 - 2 ( \sqrt{-2}\, {b_{5}} -2)\, tx - 2\, x^2 + ( {b_{5}}^2 - 4 \sqrt{-2}\, {b_{5}} - 6 )\, t \\
&+ 2 ( {b_{5}}^2 +2)\, x + {b_{5}}^2 - 2 \sqrt{-2}\, {b_{5}} -2=0.
\end{align*}
We put
%\begin{align*}
\[
{\mcB}^{1}:= {\mcE} + {\mcL}_{o} + {\mcC}_{1} + {\mcC}_{2} + {\mcC}_{3}, \,
{\mcB}^{2}:= {\mcE} + {\mcL}_{o} + {\mcC}_{1} + {\mcC}_{2} + {\mcC}_{4}, \,
{\mcB}^{3}:= {\mcE} + {\mcL}_{o} + {\mcC}_{1} + {\mcC}_{4} + {\mcC}_{5}.
\]
%\end{align*}
If we choose general $b_{1}, b_{2}, b_{3}, b_{4}, b_{5} \in \CC$, then we have a Zariski triple $({\mcB}^{1}, {\mcB}^{2}, {\mcB}^{3})$.
\end{exmple}

\begin{exmple}[{\bf Combinatorics  3-(a)}] \label{ex:combi-3a}\rm

Let $P_0, P_1, P_2$ and $P_3 \in E_{\mcQ, z_o}(\CC(t))$ be the basis of $D^*_4$-part considered in Example~\ref{ex:comb1}, 1-(a).
Based on these points, we construct $Q_0, Q_1, Q_2, Q_3$ as in \S 3.  For $\mcQ_0, \mcQ_1, \mcQ_2$, we have explicit coordinates
\begin{align*}
&\bullet \displaystyle Q_{{0}}\, := \, \left[  - \left( \frac{\sqrt {2}+1}2 \right)  \left( 6-2\,t\sqrt {2}+{t}^{2}-4\,\sqrt {2}+4\,t \right)\right. , \\
 &\quad\quad\left.-\frac{1}{8}\,\sqrt {1-i} \left( -2+4\,i+3\,i\sqrt {2}-\sqrt {2} \right) 
 \left( 2\,t\sqrt {2}+{t}^{2}+4\,\sqrt {2}-2\,t-6 \right)  \left( -2+
\sqrt {2}-t \right) 
\right] \\
&\bullet \displaystyle Q_{{1}}\, := \, \left[  \frac{1}{4}(-1+i)\left( 2\,i+2\,it+2\,t+{t}^{2} \right) \right. ,\\
&\quad\quad\quad\left. - \frac{\sqrt{2}}{16}\left( i\sqrt {2-2\,i}+\,\sqrt {2-2\,i} \right)  \left( -i{t}^{2}+{t}^{3}-2\,it+3\,{t}^{2}+2-2\,i+4\,t \right) \right] \\
&\bullet \displaystyle Q_{{2}}\,:=\left[ \frac{1}{4}\left( 1+i\right)  \left( -t-1+i \right) ^{2}\right.,\left. \frac{1}{8}\left( -1+i\right) \sqrt {-1-i} \left( -t-1+i \right)  \left( t+1+i \right) ^{2} \right]
%&\displaystyle Q_{{3}}\, := \, \left( \left( 1/4-i/4 \right)  \left( t+1+i \right) ^{2} \right. ,\left.\left( 1/8-i/8 \right) \sqrt {1-i} \left( t+1+i \right)  \left( -t-1+i \right) ^{2}\right)
\end{align*}

We use the method given at the beginning of this section and construct bisections $D_0, \ldots, D_5$ and contact conics $\mcC_0, \ldots, \mcC_5$ corresponding to $-s_{\mcQ_0}$, $-s_{\mcQ_1}$, $-s_{\mcQ_2}$  and $-s_{\mcQ_0\dot+\mcQ_1}$. 
The equations  can be calculated by using the data in the following table. 
\begin{center}
\begin{tabular}{c|c|c}
conic & rational point & $r(t)$\\
\hline
$\mcC_0$ & $\mcQ_0$ & $\frac{1}{4}\sqrt{1-i}\left(-2i-i\sqrt{2}+\sqrt{2}\right) t+b_0$\\
$\mcC_j, (j=1,2,3)$ & $\mcQ_1$ &  $\frac{i}{2}\sqrt {1-i}\,t+b_j$\\
$\mcC_4$ & $\mcQ_2$ & $\-\frac{1}{4}\sqrt{2-2i}(i+1)t+b_4$ \\
$\mcC_5$ & $\mcQ_1\dot+\mcQ_2$ & $\frac{1}{4}(1-i)\left(i\sqrt {2}+i+\sqrt {2}\right) t+b_5$
\end{tabular}
\end{center}

For $\mcC_0$, by using $P=\mcQ_0$ and $r(t)=\left(-2i-i\sqrt{2}+\sqrt{2}\right)\frac{\sqrt{1-i}}{4} t+b_0$ as above, we have the explicit equation

\begin{align*}
&\mcC_0: \left( {t}^{2}+4\,t-2\,t\sqrt {2}+6-4\,\sqrt {2}+2\,\sqrt {2}x-2\,x
 \right) {b}^{2}\\
 &+\sqrt {1-i} \left( i\sqrt {2}{t}^{2}-i\sqrt {2}tx+4\,
i\sqrt {2}t-i{t}^{2}+2\,\sqrt {2}{t}^{2}-\sqrt {2}tx+4\,i\sqrt {2}-6\,
it+10\,t\sqrt {2}-3\,{t}^{2} \right.\\
&\left.\quad\quad+2\,tx-14-6\,i+10\,\sqrt {2}-14\,t
 \right)b \\
 &-\left( 7-5\,\sqrt {2} \right)  \left( 6\,\sqrt {2}tx-4\,{x}^{2}
\sqrt {2}+2\,t\sqrt {2}+2\,\sqrt {2}x-{t}^{2}+8\,tx-6\,{x}^{2}+2\,x-2
 \right) 
=0
%&\mcC_j: i/2{t}^{2}\sqrt {1-i}{b_j}-i{t}^{2}+ixt\sqrt {1-i}{b_j}-{t}^{2}\sqrt {1-i}{b_j}+1/
%4\,{{b_j}}^{2}{t}^{2}-3/2\,it-i/4{t}^{2}{{b_j}}^{2}-2\,t\sqrt {1-i}{b_j}+{{b_j}}^{2}t+
%{{b_j}}^{2}x+i/2{{b_j}}^{2}+i/2x-{b_j}\sqrt {1-i}+1/2\,{{b_j}}^{2}+3/4\,{t}^{2}+tx-{x}
%^{2}-i/2+t/2+x/2=0, (j=1,2,3)\\
%&\mcC_4: \sqrt {2-2\,i}{t}^{2}{b_4}-1/2\,\sqrt {2-2\,i}{b_4}tx-1/4\,{{b_4}}^{2}{t}^{2}-i/2
%\sqrt {2-2\,i}{b_4}+\sqrt {2-2\,i}t{b_4}-i/4{t}^{2}{{b_4}}^{2}+i/2{{b_4}}^{2}-{{b_4}}^{2}t
%+{{b_4}}^{2}x+1/2\,\sqrt {2-2\,i}{b_4}-i/2x\sqrt {2-2\,i}t{b_4}+i/2x-1/2\,{{b_4}}^{2}+
%3/4\,{t}^{2}-tx-{x}^{2}+i/2+t/2-x/2=0\\
%&\mcC_5: 3+6\,t-3/2\,\sqrt {1-i}\sqrt {2}{b_5}{t}^{2}+i\sqrt {1-i}t{b_5}-3\,t\sqrt {1-i
%}\sqrt {2}{b_5}+3/2\,i\sqrt {1-i}{t}^{2}{b_5}-i/2\sqrt {2}{t}^{2}{{b_5}}^{2}-it
%\sqrt {2}{{b_5}}^{2}-i\sqrt {2}tx-3\,{b_5}\sqrt {1-i}-{x}^{2}+2\,\sqrt {2}+9/2
%\,{t}^{2}+3\,t\sqrt {2}-ix+ixt\sqrt {1-i}{b_5}-2\,\sqrt {1-i}\sqrt {2}{b_5}+i
%\sqrt {1-i}{b_5}-2\,itx-2\,it{{b_5}}^{2}-i\sqrt {2}x-i/2{t}^{2}{{b_5}}^{2}-i\sqrt 
%{2}{{b_5}}^{2}-i{{b_5}}^{2}+i/2\sqrt {2}x\sqrt {1-i}t{b_5}+i\sqrt {1-i}\sqrt {2}{b_5}+
%1/2\,xt\sqrt {1-i}\sqrt {2}{b_5}+2\,it\sqrt {1-i}\sqrt {2}{b_5}+{{b_5}}^{2}x-3/2\,
%{t}^{2}\sqrt {1-i}{b_5}-5\,t\sqrt {1-i}{b_5}=0
\end{align*}
We omit the equations of the other conics as they are rather long. 
We put 
\[
%\begin{array}{ccc}
\mcB^1  :=  \mcQ + \mcC_1 + \mcC_2 + \mcC_3,  \,\mcB^2  :=   \mcQ + \mcC_0 + \mcC_1 + \mcC_2,\, 
\mcB^3  :=  \mcQ + \mcC_0 + \mcC_1 + \mcC_4, \, \mcB^4  :=   \mcQ + \mcC_0 + \mcC_1 + \mcC_5.
%\end{array}
\]
It can be checked that for a general choice of $b_0,\ldots, b_5$, 
these curves  have Combinatorics 3-(a), and we have a Zariski 4-ple.

\end{exmple}

