
We now present some results about the Brownian excursion with drift. 
We provide a construction which  is analogous to that of the drifted Brownian meander. 
We consider the sequence of conditional processes $ \{ B^\mu | \Lambda_{u,v,c} : u,c >v \}_{} $ whose
sample paths lie in sets of the form 
$\Lambda_{u,v,c} = \{ \omega \in C[0,T]  :{\min_{ 0 \leq z \leq t } \omega(z) > v },\,\, \omega(0) = u,\,\, \omega(t) = c  \}$, with $ c,u > v $. We study the distribution of these processes and of process
$ B^\mu | \Lambda_{v,v,v} $ defined as the weak limit of $  B^\mu | \Lambda_{u,v,c} $ when the starting 
point $ u $ and the endpoint $ c $ collapse onto the barrier $ v $.

The main tool used to carry out the needed calculation is the distribution of an absorbing Brownian motion  
with distribution  \eqref{eq:bm-drift-recall-new}. With this at hand we can write
\begin{align*}
	&
	P \Big\{ 
	B^\mu(s) \in \mathrm d y \Big | \min_{ 0 \leq z \leq t } B^\mu(z) > v, B^\mu(0) = u, B^\mu(t) = c
	\Big\}
	 = \numberthis \label{eq:bmd-bdg-dft}\\
	&= 
	%
		%
		%
		\frac{
			e^{- \frac{( y- \mu s -u)^2}{2s} } - 
			e^{ - 2 \mu u}
			e^{- \frac{( y + u - 2 v - \mu s )^2}{2s} }
		}{\sqrt{2\pi s}} \cdot 
		\frac{
			e^{- \frac{( c- \mu(t-s) - y)^2}{2(t-s)} } - 
			e^{ - 2 \mu y} 
			e^{- \frac{(c + y - 2 v  - \mu (t-s) )^2}{2(t-s)} }
		}{\sqrt{2\pi (t-s)}} 
		%
	%}%
	%
	% 
	%{
	\times \\
	&
	\qquad \times 
	\left( 
		\frac{
			e^{- \frac{( c- \mu t -  u)^2}{2t} } -
			e^{-2\mu u } 
			e^{- \frac{( c + u - 2 v  - \mu t )^2}{2t} }
		}{\sqrt{2 \pi t}}
	\right)^{\!-1}	
	%}
	\,  \mathrm d y \\
	%
	%
	&=
	\sqrt{\frac{t}{2 \pi s(t-s)}}	\!
	%\frac{
		%
		\Big(
		e^{- \frac{( y- u)^2}{2s} } -
		%
		e^{- \frac{( 2v - y   - u)^2}{2s} }
		\Big)
		e^{ - \frac{\mu^2 s}{2} + \mu (y-u)}
	%	
	%}
	%
	% 
	%{
		%
	%}
%	\quad \times \\
%	& \qquad \times 
	\Big(
	e^{- \frac{( c -y)^2}{2(t-s)} } - 
	% 
	e^{- \frac{( 2v - c  - y)^2}{2(t-s)} }
	\Big)
	e^{- \frac{ \mu^2(t-s)}{2} + \mu(c-y) }
	%
	\\
	& \qquad \times
	\left[ 
	\left( 
	e^{- \frac{( c - u)^2}{2t} } - 
	%
	e^{- \frac{( 2v -c - u)^2}{2t} }
	\right)
	e^{ - \frac{\mu^2 t}{2} + \mu (c-u)}
	%
	\right]^{\!-1}
	  \mathrm d y
	%
	%			
\end{align*}

It is straightforward to check that in \eqref{eq:bmd-bdg-dft} the 
resulting distribution does not depend on the value of the drift $ \mu $. This consideration holds
for any finite dimensional distribution of the drifted excursion, as stated in the following theorem.

\begin{theorem}\label{thm:exc-fin-dim}
	The finite dimensional distributions of the drifted Brownian excursion process $ B^\mu | \Lambda_{u,v,c}  $ are given by  
%
\begin{align*}
&
P \bigg\{ \bigcap_{j=1}^n \left( B^\mu(s_j) \in \mathrm d y_j \right) \,\Big \vert \min_{0\leq z \leq t} B^\mu(z)> v , B^\mu(0)=u, B^\mu(t) = c\bigg \}  = \numberthis  \label{eq:exc-fin-dim}\\ 
&   \prod_{j=1}^{n}  
%
\Bigg \{ 
\frac{ e^{ - \frac{(y_j - y _{j-1})^2}{2(s_j - s_{j-1})}  } - 
e^{ - \frac{(2v - y_j - y _{j-1})^2}{2(s_j - s_{j-1})}  } }   {\sqrt{2 \pi ( s_j - s_{j-1})} } \,\mathrm d y_j
\Bigg\} 
%
 % 
	\frac{ e^{ - \frac{ (c - y_n) ^ 2}{2(t-s_n)} }
		- e^{ - \frac{ (2v - c  - y_n) ^ 2}{2(t-s_n)}} }{\sqrt{2 \pi (t - s_{n})}}
\left(
	\frac{ e^{ - \frac{ (c - u) ^ 2}{2t} }
		- e^{ - \frac{ (2 v - c  -  u) ^ 2}{2t}} }{\sqrt{2 \pi t}}
\right)^{\!\!-1}
\end{align*}
for $ 0 = s_0 < s_1 < \cdots < s_n < t, y_0 = u $ with $ y_j,u,c > v $, and the following equality in distribution holds
\begin{equation}\label{eq:exc-no-drift-eq}
B^\mu | \Lambda_{u,v,c} \stackrel{law}{=}   B | \Lambda_{u,v,c}. 
\end{equation}
%
%

\end{theorem}

\begin{proof}
	By Markovianity we can write that
\begin{align*}
&
P \bigg\{ \bigcap_{j=1}^n \left( B^\mu(s_j) \in \mathrm d y_j \right) \,\Big \vert \min_{0\leq z \leq t} B^\mu(z)> v , B^\mu(0)=u, B^\mu(t) = c\bigg \}  = \\ 
& =  \prod_{j=i}^{n}  
%
\Bigg \{ 
\frac{ e^{ - \frac{(y_j - y _{j-1})^2}{2(s_j - s_{j-1})}  } - 
	e^{ - \frac{(2v - y_j - y _{j-1})^2}{2(s_j - s_{j-1})}  } }   {\sqrt{2 \pi ( s_j - s_{j-1})} }
e^{ - \frac{ \mu^2}{2} (s_j - s_{j-1}) +    \mu(y_j - y_{j-1})   }
\Bigg\} 
%
\times 
\\
&\qquad \times 
\frac{ e^{ - \frac{ (c - y_n) ^ 2}{2(t-s_n)} }
	- e^{ - \frac{ (2v - c  - y_n) ^ 2}{2(t-s_n)}} }{\sqrt{2 \pi (t - s_{n})}}
e^{ - \frac{ \mu^2}{2} (t - s_{n}) + \mu(c - y_{n}) }
\left(
\frac{ e^{ - \frac{ (c - u) ^ 2}{2t} }
	- e^{ - \frac{ (2 v - c  -  u) ^ 2}{2t}} }{\sqrt{2 \pi t}}
%
e^{- \frac{ \mu^2}{2} t + \mu(c - u)  }
\right)^{\!\!-1}
\end{align*}
thus the finite dimensional distributions of the excursion with drift coincide with the corresponding finite
dimensional distributions of the driftless Brownian excursion. 
\end{proof}
\begin{remark}
	
We observe that in analogy with result   \eqref{eq:exc-no-drift-eq} 
\citet{Beghin1999} showed that the distribution
\[
P\left\{  \max_{ 0 \leq s \leq t} B^\mu(s) > \beta \Big | B^\mu(t) = \eta \right\} 
=
e^{   -  \frac{ 2 \beta (\beta - \eta) }{t}} 
\]
is independent of $\mu$ for every $\eta $.
%
\end{remark}


%
%\begin{align*}
%& \prod_{j=i}^{n} \left\{   
%\left( 
%e^{ - \frac{(y_j - y _{j-1})^2}{2(s_j - s_{j-1})}  } - 
%e^{ - \frac{(y_j + y _{j-1})^2}{2(s_j - s_{j-1})}  }
%\right) 
%\frac{ e^{\mu(y_j - y_{j-1}) + \frac{ \mu^2}{2} (s_j - s_{j-1})} }{
%	\sqrt{2 \pi ( s_j - s_{j-1})} 
%}
%\right\} \times \\
%& \qquad \times 
%\left( e^{ - \frac{ (v - y_n) ^ 2}{(t-s_n)} }
%- e^{ - \frac{ (v + y_n) ^ 2}{(t-s_n)}} \right)
%\frac{ e^{  \mu(v-y_n) - \frac{\mu^2}{2} (t-s_n) } }{\sqrt{2 \pi (t - s_{n})}}
%\end{align*}
%
%
Weak convergence to the Brownian excursion has been established in \cite{durrett77}. 
\begin{theorem}\label{thm:weak-conv-exc}
	The following weak limit holds:
	\begin{equation}\label{eq:weak-conv-exc}
%	B^\mu \Big |\Big \{  \min_{ 0 \leq z \leq t } B^\mu > v, B^\mu(0) = u, B^\mu(t) = c \Big \}
	B^\mu \Big |\Lambda_{u,v,c} 
 \xRightarrow[u,c\downarrow v]{}
	B^\mu \Big | \Big \{ \inf_{ 0 < z < t } B^\mu(z) > v, B^\mu(0) = v, B^\mu(t) =c \Big \}
	\end{equation}
\end{theorem}
%\begin{proof}
%	In view of \autoref{thm:exc-fin-dim}, the first condition of \autoref{thm:weak-conv},
%	i.e. the convergence of finite dimensional distributions immediately follows 
%	by simply applying De l'H\^ opital's rule twice. In order to prove the tightness 
%	of the family of measures induced by $ B^\mu \Big |\Lambda_{u,v,c}  $, 
%	reasoning as in \autoref{thm:tight-bmd}, we must prove that
%	\begin{equation}\label{eq:exc-tight-cond-calc}
%		\lim_{\delta \to 0}
%		\lim_{u,c \to v} 
%		P \Big( \max_{0\leq z \leq \delta } | B^\mu ( z)  | < \eta  \, \Big | 
%		\min_{ 0\leq z \leq t} B^\mu ( z) > v ,  B^\mu(0) = u , B^\mu(t) = c \Big) = 1
%		\qquad \forall \eta > 0 . 
%	\end{equation}
%	Since
%	\begin{align*}
%	&
%		P \Big( \max_{0\leq z \leq \delta } | B^\mu ( z)  | < \eta  \, \Big | 
%	\min_{ 0\leq z \leq t} B^\mu ( z) > v ,  B^\mu(0) = u , B^\mu(t) = c \Big)
%	\numberthis \\
%	&=
%		\int_{v}^{u + \eta} P\left\{ v < \min_{ 0\leq z \leq \delta} B^\mu ( z )  
%		< \max_{0 \leq z <\delta } B^\mu(z) < u + \eta,
%		B^\mu(\delta ) \in \mathrm d y | B^\mu(0) = u  \right\} \times \\
%		%
%		& \qquad \times 
%		\frac{
%			P\left\{ \min_{ \delta \leq z \leq t} B^\mu( z) > v,
%			 B^\mu(t) \in \mathrm d c \Big | B^\mu(\delta ) = y \right\}
%		}{
%			P\left\{ \min_{ 0 \leq z \leq t} B^\mu( z) > v, 
%			 B^\mu(t) \in \mathrm d c \Big | B^\mu(0 ) = u \right\}
%		}
%	\end{align*}
%	it is sufficient to perform the same calculations as in \autoref{lem:tight-cond-calc} to obtain the statement 
%	\eqref{eq:exc-tight-cond-calc}. 
%\end{proof}
%

We give the explicit form of the finite-dimensional distributions of the right member of 
\eqref{eq:weak-conv-exc} in the special case when $ v =0 $. 
For $u\to 0$ the distribution \eqref{eq:exc-fin-dim} becomes
\begin{align*}
&
	P \bigg\{ \bigcap_{j=1}^n \left( B^\mu(s_j) \in \mathrm d y_j \right) \,\Big \vert \inf_{0< z < t} B^\mu(z)> 0 , B^\mu(0)=0, B^\mu(t) = c\bigg \} \\
	&=
	\lim_{ u \downarrow 0}
	P \bigg\{ \bigcap_{j=1}^n \left( B^\mu(s_j) \in \mathrm d y_j \right) \,\Big \vert \min_{0\leq z \leq t} B^\mu(z)> 0 , B^\mu(0)=u, B^\mu(t) = c\bigg \}
	\\
	& =
	\frac{y_1\, t \sqrt t }{c\, s_1 \sqrt{s_1}} e^{- \frac{y_1^2}{2 s_1} + \frac{c^2}{2t}}
	\prod_{j=2}^{n} 
	\left[
	%
	\frac{
		e^{ - \frac{  (y_j - y_{j-1})^2   }{   2(s_j - s_{j-1})   }      } 
		- 
		e^{ - \frac{  (y_j + y_{j-1})^2   }{   2(s_j - s_{j-1})   }      } 
	}
	{
		\sqrt{2\pi ( s_j - s_{j-1})}
	}
	%
	\right]
	\frac{
		e^{ - \frac{  (c - y_{n})^2   }{   2(t - s_{n})   }      } 
		- 
		e^{ - \frac{  (c + y_{n})^2   }{   2(t - s_{n})   }      } 
	}
	{
		\sqrt{2\pi ( t - s_{n})}
	}
	%
\end{align*}
%

%\begin{align*}
%	&\lim\limits_{u\to 0} \bmdbd \numberthis \label{eq:bmd-bdg-dft-0} = \\
%	&=
%	\sqrt{\frac{t}{2 \pi s(t-s)}}
%	\frac{ 
%		e^{- \frac{( y -\mu s)^2}{2s} } 
%		}{
%		e^{- \frac{( v -\mu t)^2}{2t} } 
%		}	
%	\frac{ yt}{sv}
%	\left[
%	e^{- \frac{( v- \mu(t-s) - y)^2}{2(t-s)} } - 
%	e^{ - 2 \mu y} 
%	e^{- \frac{( v  - \mu (t-s) + y)^2}{2(t-s)} }
%	\right]
%	\, \mathrm d y
%\end{align*}
%
%
%
The further limit for $c \to 0 $ yields
%
%
\begin{align*}
&
P \bigg\{ \bigcap_{j=1}^n \left( B^\mu(s_j) \in \mathrm d y_j \right) \,\Big \vert \inf_{0< z < t} B^\mu(z)> 0 , B^\mu(0)=0, B^\mu(t) = 0\bigg \} \\
&=
\lim_{ u,c \downarrow 0}
P \bigg\{ \bigcap_{j=1}^n \left( B^\mu(s_j) \in \mathrm d y_j \right) \,\Big \vert \min_{0\leq z \leq t} B^\mu(z)> 0 , B^\mu(0)=u, B^\mu(t) = c\bigg \}
\\
& =
\frac{y_1\, t \sqrt t }{s_1 \sqrt{s_1}} e^{- \frac{y_1^2}{2 s_1}}
\prod_{j=2}^{n} 
\left[
%
\frac{
	e^{ - \frac{  (y_j - y_{j-1})^2   }{   2(s_j - s_{j-1})   }      } 
	- 
	e^{ - \frac{  (y_j + y_{j-1})^2   }{   2(s_j - s_{j-1})   }      } 
}
{
	\sqrt{2\pi ( s_j - s_{j-1})}
}
%
\right]
\frac{2 y_n}{t- s_n}
\frac{
	e^{ - \frac{   y_{n}^2   }{   2(t - s_{n})   }      } 
}
{
	\sqrt{2\pi ( t - s_{n})}
}.
%
\end{align*}




The one dimensional distribution of the excursion reads

\begin{align*}
& P\Big\{ B^\mu(s) \in \mathrm d y \Big |  \inf_{ 0 < z < t } B^\mu(z) > 0, B^\mu(0) = 0, B^\mu(t) =0 \Big \} = \numberthis \label{eq:bmd-bdg-dft-0-0} \\
%
&=
\sqrt \frac{2}{\pi} y^2 \left(\frac{t}{s(t-s)}\right) ^\frac 32 
e^{- \frac{ y^2t}{ 2 s (t-s)}} \, \mathrm d y \qquad y>0 \,,\,\, s<t.
\end{align*}



%
%
%
%\subsection{Sojourn time of a conditional process. }
%
%We here study the time spent by the Brownian particle on $(0, \infty)$ up to time $t$ under the condition 
%that $ \min_{0\leq z \leq l} B(z)>0 $, for $l<t$. In symbols we want to study the distribution of
%\begin{equation}\label{eq:souj-def}
%\Gamma_{(0,t)}  = l + \Gamma_{l,t} = l + \int_{l}^{t} \mathbbm 1_{[0, \infty)}( B(s)) \mathrm d s 
%\end{equation}
%Clearly the condition that $ \min_{0\leq z \leq l} B(z)>0 $, with the assumption that $B(0) = u >0 $ 
%exerts its effect on the distribution of the sojourn time $ \Gamma_{l,t}$. 
%
%
%It is well-known that 
%\begin{equation}\label{eq:souj-old}
%P\left\{  \Gamma_t \in \mathrm d s \Big | B(0)  =B(t) = 0    \right\} = \frac{ \mathrm d s }{t}  
%\qquad 0 < s < t
%\end{equation}
%where 
%\[ 
%\Gamma_ t = \int_{0}^{t} \mathbbm 1_{[0, \infty)}( B(s)) \mathrm d s \, . 
%\]
%In the analysis of \eqref{eq:souj-def} we need to extend result \eqref{eq:souj-old}, in particular we
%must generalize it as follows 
%\begin{align}\label{eq:souj-new}
%&P\left\{  \Gamma_t \in \mathrm d s \Big | B(0)=u, B(t) = 0    \right\} =
%\\
%&=\sqrt \frac{t}{2 \pi} e^{  \frac{ u^2}{2t}} \int_{0}^{s} \frac{ u e ^ {  - \frac{  u^2}{2w}}}{ \sqrt{ w^3 (t-w)^3}} \,\mathrm d w
%\qquad 0\leq s \leq t \,\,,\,\,\, u \in \mathbbm R^+ \notag 
%\end{align}
%
%Equation \eqref{eq:souj-new} can be derived following the argument 
%\cite{BEGHIN2003291} based on an application of the conditional Feynman-Kac functional. 
%
%
%We here consider the conditional process $ B| \Lambda_{u,0}^l$ where
%\[
%\Lambda_{u,0}^l = \{  \omega \in C[0,t]:  \min_{ 0\leq z \leq l} \omega(z) > v, \omega(0) = u , \omega(t)=0\}
%\]
%
%The tightness of the family of measures $\{ \mu_{u,0}^l , u>0  \}$ defined as
%\[
%\mu_{u,0}^l(A) = P\Big(B \in A \Big | \Lambda_{u,0}^l \Big ) \qquad A \in \mathscr C, \quad u > 0
%\]
%
%can be proved following the same steps of \autoref{thm:tight-bmd}: this only requires a suitable adaptation
%of the proof of \autoref{lem:tight-cond-kolmo}. 
%Thus arguing as in \autoref{thm:weak-conv-mdr} we have that the weak convergence
%$
%\mu_{u,0}^l \Rightarrow_{u \downarrow 0} \mu_{0,0}^l
%$
%holds 
%where 
%\[
%\mu_{0,0}^l(A) = P\Big(B \in A \Big | \inf_{ 0<  z \leq l} B(z) > v, B(0) = 0 , B(t)=0\Big ) \qquad A \in \mathscr C, \quad u > 0
%\]
%
%In the next theorem we are able to give the explicit distribution of $\Gamma_{l,t}$, and the limit distribution
%of the sojourn functional when $ u $ approaches the zero level.  
%
%
%
%\begin{theorem}
%	The conditional distribution of $ \Gamma_{l,t} $ under $ \mu_{0,0}^l $ is given by
%	\begin{align*}
%	P\left\{  \Gamma_{l,t} \in \mathrm d s \Big | \inf_{0< z\leq l} B(z) > 0, B(0)=0 , B(t)=0 \right\} 
%	&=
%	\frac t2 \sqrt \frac{l } {t-l} 
%	\int_{l}^{l+s}
%	\frac{\mathrm d w} {\sqrt{w^3 (t-w)^3 }} \,\mathrm d s 
%	\\
%	&= 
%	\frac 1t \sqrt \frac{l } {t-l}
%	\left\{
%	\frac{t - 2l}{  \sqrt{  l(t - l) }} - 
%	\frac{t - 2(l + s)}{  \sqrt{  (l+s)(t - l -s) }}  
%	\right\} \mathrm d s  \, .
%	\end{align*}
%\end{theorem}
%
%\begin{proof}
%	
%	
%The conditional distribution of $ \Gamma_{l,t} $ under $ \mu_{u,0}^l $ can be computed as 
%\begin{align*}
%&P\left\{  \Gamma_{l,t} \in \mathrm d s \Big |  \min_{0\leq z\leq l} B(z) > 0, B(0)=u , B(t)=0 \right\} =
%\numberthis \label{eq:souj-cond-a} \\
%&=
%\int_{0}^{\infty} 
%P\left\{  \min_{0\leq z\leq l} B(z) > 0, B(l) \in \mathrm d y \Big \vert  B(0)=u \right\} \times \\
%& \qquad  \times 
%P\left\{  \Gamma_{l,t} \in \mathrm d s \Big |   B(l) \in \mathrm d y, B(t)=0 \right\}
%P\left\{  B(t) \in \mathrm d 0 \Big \vert  B(l)=y \right\}  
%\\
%& \qquad  \times 
%\left(\int_{0}^{\infty} 
%P\left\{  \min_{0\leq z\leq l} B(z) > 0, B(l) \in \mathrm d y \Big \vert  B(0)=u \right\}
%P\left\{  B(t) \in \mathrm d 0 \Big \vert  B(l)=y \right\} \right)^{-1} \\
%&=
%\frac{
%	{\displaystyle
%		\int_{0}^{\infty} }
%	\left [
%	\frac{e^{- \frac{ (y - u )^2}{2 l } } 
%		- e^{  - \frac{ (y + u)^2}{2 l }   } }
%	{ \sqrt{2\pi l } }  
%	\right ]
%	%
%	\left[ 
%	\sqrt \frac{ t - l }{ 2\pi } e^{ \frac{y^2}{2(t-l)} }
%	\int_0^s
%	\frac{y e^{-\frac{y^2}{2w}}} {\sqrt{ w^3 (t-l-w)^3}} 
%	\, \mathrm d w
%	\right] 
%	\frac{ e^{-\frac{y^2}{2(t-l)}}} {\sqrt{2\pi (t-l)}} \mathrm d y 
%}{
%	{\displaystyle
%		\int_{0}^{\infty} }
%	\left [
%	\frac{e^{- \frac{ (y - u )^2}{2 l } } 
%		- e^{  - \frac{ (y + u)^2}{2 l }   } }
%	{ \sqrt{2\pi l } }  
%	\right ]
%	%
%	\frac{ e^{-\frac{y^2}{2(t-l)}}} {\sqrt{2\pi (t-l)}} \mathrm d y 
%} \mathrm d s
%\end{align*}
%where we applied \eqref{eq:souj-new} in the last step. We now let  $u\to0$ in  \eqref{eq:souj-cond-a}. 
%This is justified by the weak convergence of the measures $ \mu_{u,0}^l $ and an application 
%of the continuous mapping theorem to the bounded continuous functional $ \Gamma_{l,t} $. 
%By computing the limit we have that
%%
%%
%\begin{align*}
%&P\left\{  \Gamma_{l,t} \in \mathrm d s \Big |  \inf_{0 < z\leq l} B(z) > 0, B(0)=0 , B(t)=0 \right\} 
%/ \mathrm d s= \numberthis \label{eq:souj-cond-c} \\
%&=
%\frac{
%	{\displaystyle
%		\int_{0}^{\infty} } 
%	y e^{- \frac{ y ^2}{2 l } }
%	%
%	\frac{ 1}{ \sqrt{2\pi } }
%	\int_0^s
%	\frac{y e^{-\frac{y^2}{2w}}} {\sqrt{ w^3 (t-l-w)^3}} 
%	\, \mathrm d w
%	\mathrm d y 
%}{
%	{\displaystyle
%		\int_{0}^{\infty} }
%	y e^{- \frac{ y ^2}{2 l } }
%	%
%	\frac{ e^{-\frac{y^2}{2(t-l)}}} {\sqrt{t-l}} \mathrm d y 
%}\\
%&=
%\frac{t \sqrt{(t-l)}}{l(t-l)}
%\int_0^s
%\frac{\mathrm d w} {\sqrt{ w^3 (t-l-w)^3}} 
%\int_{0}^{\infty} y^2 \frac{ e^{-\frac{y^2}{2}  \left( \frac{l+w}{lw}\right)}  } {\sqrt{2\pi}}
%\mathrm d y \\
%&=
%\frac{t }{2l\sqrt {t-l}}  
%\int_0^s
%\frac{\mathrm d w} {\sqrt{ w^3 (t-l-w)^3}}  \left(\sqrt \frac{lw}{l+w}\right)^3\\
%&=
%\frac{t \sqrt l }{2\sqrt {t-l}}  
%\int_0^s
%\frac{\mathrm d w} {\sqrt{(t-l-w)^3 (l+w)^3}}  \\
%%
%%
%%
%%
%	&=
%	\frac t2 \sqrt \frac{l } {t-l} 
%	\int_{l}^{l+s}
%	\frac{\mathrm d w} {\sqrt{w^3 (t-w)^3 }}\\
%	&=
%	\frac t2 \sqrt \frac{l } {t-l}  
%	\int_{ \arcsin \sqrt \frac lt }^{ \arcsin \sqrt \frac{l+s}s}
%	\frac{ \mathrm d \varphi}{\sin^2 \varphi \cos^2 \varphi }\\
%	&=
%	\frac 1t \sqrt \frac{l } {t-l}
%	%
%	\left\{
%	\frac{ \sqrt{   l + s }}{  \sqrt{  t - l -s  }} - 
%	\sqrt \frac{l } {t-l} -
%	\frac{ \sqrt{  t - l -s  }}{ \sqrt{ l+s}  } + 
%	\sqrt \frac{t-l}{l}
%	\right\} \\
%	&=
%	\frac 1t \sqrt \frac{l } {t-l}
%	\left\{
%	\frac{t - 2l}{  \sqrt{  l(t - l) }} - 
%	\frac{t - 2(l + s)}{  \sqrt{  (l+s)(t - l -s) }}  
%	\right\} 
%	\qquad 0 < s <t-l . 
%	\end{align*}
%
%\end{proof} 
%
%%The non-negativity of \eqref{eq:souj-cond-c} can be shown in two different ways. 
%%The first one directly from \eqref{eq:souj-cond-b} while the second one can be 
%%proved by writing the density \eqref{eq:souj-cond-c} as 
%%\[ 
%%f_{\Gamma_{l,t}} = H \cdot \left\{ h(s) - \frac{1}{h(s)} + K\right\}
%%\]
%%Thus
%%\[ 
%%\frac{\mathrm d f_{\Gamma_{l,t}}  }{ \mathrm ds} =
%%H  \left\{ h'(s) + \frac{h'(s)}{h^2(s)} \right\} = 
%%H \cdot  h'(s) \left\{ 1 + \frac{1}{h^2(s)} \right\} 
%%\]
%%where $h(s) = \frac{l+s}{t - l-s}$. Furthermore we have that 
%%$h'(s) = \frac{1}{2 h(s)} \cdot \frac{ t}{(t-l-s)^2}> 0$ and 
%%\[ 
%%\lim\limits_{s \to 0^+ } f_{\Gamma_{l,t}} = 0 \qquad 
%%\lim\limits_{s \to t-l^- } f_{\Gamma_{l,t}} = +\infty
%%\]
%%This shows that the assumption that $ \min_{0\leq s\leq t B(s) > 0} $ 
%%implies that in $[l,t]$ the Brownian particle spends on the half-line 
%%an increasingly long time.
%
%For $l = \frac 12$ the density \eqref{eq:souj-cond-c} simplifies as
%\begin{align}\label{eq:souj-cond-d}
%& P\left\{  \Gamma_{\frac t2,t} \in \mathrm d s \Big |  \min_{0\leq z\leq l} B(z) > 0, B(0)=0 , B(t)=0 \right\}
%= \\
%&= \frac{ 4s}{ t \sqrt{ t^2 - 4s^2}} \mathrm d s \qquad 0<s<t \notag.
%\end{align}
%
%We observe that for $l = 0$ we retrieve in \eqref{eq:souj-cond-c} the uniform distribution. 
%\vskip 1cm
%
%
\subsection*{Acknowledgments}
We thank both referees for their accuracy in the analysis of the first draft of this paper.
They have detected misprints and errors and their constructive criticism has substantially 
improved the paper. 
%Clearly \eqref{eq:souj-cond-d} coincides with the distribution 
%\[ P\left\{ B\left( \frac12 , 1 \right) < \frac{ 2x}{t}\right\} \]

%%
%%
%


%
%We can evaluate the mean value of $\Gamma_{l,t}$ which becomes 
%\begin{align}\label{eq:souj-exp}
%&\mathbb E \left( \Gamma_{l,t} 
%\Big |  \min_{0\leq z\leq l} B(z) > 0, B(0)= B(t)=0 \right )
%= \\
%&=
%\frac t2 \sqrt \frac{l}{t- l}
%\left(
%\frac \pi 2 - \arcsin \sqrt \frac lt 
%\right) + 
%\frac{t-2l}{2} \notag 
%\end{align}
%In view of \eqref{eq:souj-cond-c} we can write \eqref{eq:souj-exp} as
%\begin{align*}
%& \mathbb E \left( \Gamma_{l,t} 
%\Big |  \min_{0\leq z\leq l} B(z) > 0, B(0)= B(t)=0 \right )
%\\
%&= \frac t2 \sqrt \frac{l}{t- l}
%\int_{0}^{t-l} s
%\int_0^s
%\frac{\mathrm d w} {\sqrt{(t-l-w)^3 (l+w)^3}}  \,\mathrm d s =\\
%& = 
%\frac t2 \sqrt \frac{l}{t- l} 
%\int_{0}^{t-l}
%\frac{\mathrm d w} {\sqrt{(t-l-w)^3 (l+w)^3}} 
%\int_{w}^{t-l} s
%\,\mathrm d s =\\
%& = 
%\frac t4 \sqrt \frac{l}{t- l} 
%\int_{0}^{t-l}
%\frac{t-l -w} {\sqrt{(t-l-w) (l+w)^3}} \, \mathrm d w
%\\
%& = 
%\frac 12 \sqrt \frac{l}{t- l} 
%\int_{\arcsin \sqrt \frac lt}^{\frac \pi 2}
%\left\{ \frac{t-2l} {\sin^2 \varphi }  + t\right\}\, \mathrm d \varphi 
%\\
%&=
%\frac t2 \sqrt \frac{l}{t- l}
%\left(
%\frac \pi 2 - \arcsin \sqrt \frac lt 
%\right) + 
%\frac{t-2l}{2}
%\end{align*}
%For $l=0$ $\mathbb E \Gamma_{0,t} = \frac t2$ while for $l = \frac t2$
%we have $\mathbb E \Gamma_{\frac t2,t} = \frac{t\pi}8$.
%For this case the mean total amount of time spent on $(0, \infty)$ by the Brownian particle is therefore
%\[ \frac t2 \left( 1 + \frac \pi 4 \right) \]
%The distribution function of $\Gamma_{l,t}$ writes
%\begin{align*}
%&P\left\{ \Gamma_{l,t} < \bar{s}
%\Big |  \min_{0\leq z\leq l} B(z) > 0, B(0)= B(t)=0 
%\right\}  = 
%\numberthis \label{eq:souj-dist-fun} \\
%&=
%\frac 1t \sqrt \frac{l } {t-l}
%%
%\int_{0}^{\bar s}
%\left[ 
%\frac{t-2l}{\sqrt{l(t-l)}} + 
%\frac{ \sqrt{   l + s }}{  \sqrt{  t - l -s  }} - 
%\frac{ \sqrt{  t - l -s  }}{ \sqrt{ l+s}  } 
%\right]\,\mathrm  ds \\
%&=
%\frac{\bar s (t-2l)}{t(t-l)}  -2 \sqrt \frac{l } {t-l}
%\int_{ \arccos \sqrt \frac lt }^{ \arccos \sqrt \frac{l+\bar s}t}
%\cos^2 \varphi \mathrm d \varphi - 
%2 \sqrt \frac{l } {t-l}
%\int_{ \arcsin \sqrt \frac lt }^{ \arcsin \sqrt \frac{l+\bar s}t}
%\cos^2 \varphi \mathrm d \varphi \\
%&=
%\frac{\bar s (t-2l)}{t(t-l)}  - 
%\sqrt \frac{l } {t-l} 
%\left[
%\varphi + \sin \varphi \cos \varphi 
%\right]_{  \arccos \sqrt \frac lt  } ^ { \arccos \sqrt \frac{l+\bar s}t} + \\
%& \qquad - 
%\sqrt \frac{l } {t-l} 
%\left[
%\varphi + \sin \varphi \cos \varphi 
%\right]_{ \arcsin \sqrt \frac lt }^{ \arcsin \sqrt \frac{l+\bar s}t}\\
%&=	
%\frac{\bar s (t-2l)}{t(t-l)}  - 2\sqrt \frac{l } {t-l} 
%\left\{   
%\sqrt{  1 - \frac{ l+\bar s }{t}}\sqrt \frac{ l+\bar s }{t} - 
%\sqrt{  1 - \frac{ l }{t}}\sqrt \frac{ l }{t}  
%\right\}\\
%&=
%\frac{\bar s (t-2l)}{t(t-l)}  - \frac 2t \sqrt \frac{l } {t-l} 
%\sqrt{  (t - l - \bar s)(l+\bar s) } +  \frac {2l}t
%\qquad 0 \leq \bar s \leq t-l
%\end{align*}
%From \eqref{eq:souj-dist-fun} it emerges that 
%\begin{equation*}
%\lim\limits_{t\to \infty }
%P\left\{  \frac{ \Gamma_{l,t} } {t} < \bar{s} 
%\Big |  \min_{0\leq z\leq l} B(z) > 0, B(0)= B(t)=0 
%\right\} 
%= \bar s \qquad 0 < \bar s <1
%\end{equation*}
%that is the random ratio $ \frac{\Gamma_{l,t}}{t} $ is 
%asymptotically uniform on $[0,1]$.
%
%
%For $ l= \frac 12$ the distribution function becomes
%\[ P\left\{ \Gamma_{\frac t2,t} < \bar{s} \right\}  = 
%1 - \frac{ \sqrt{t^2 - 4 \bar s^2}}{t} \qquad 0 \leq \bar s \leq \frac t2 \]
%
%
%
%





%Therefore  
%\begin{align*}
%	&\bmdbd[s][y][0][t][0][0] = \\
%	&=
%	\bmdb[s][y][0][t][0][0] 
%\end{align*}
%
% 


% !TeX root = ./mdr-main.tex