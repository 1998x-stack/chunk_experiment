\documentclass[10pt, a4paper, reqno]{amsart}


\usepackage[utf8]{inputenc}

\usepackage[english]{babel}

\usepackage{amsmath}
\usepackage{amsfonts}
\usepackage{amssymb}
\usepackage{amsthm}
\usepackage[foot]{amsaddr}
\usepackage{mathtools}
\usepackage{bm, bbm}
\usepackage{bigints}
\usepackage{hyperref}
\usepackage[capitalise]{cleveref}
\usepackage{xparse}
\usepackage{graphicx}
\usepackage[percent]{overpic}
\usepackage[usenames,dvipsnames,svgnames,table]{xcolor}
\usepackage[scr=boondoxo, scrscaled=.98]{mathalfa}
\usepackage{dsfont}


%\usepackage{refcheck}

\usepackage{enumerate
}


 
\usepackage[numbers]{natbib}


\nocite{*}


\usepackage[left=3cm, right=3cm, top=3.5cm, bottom=4cm]{geometry}





% brownian meander
\DeclareDocumentCommand{\bmd}{O{s} O{y} O{0} O{t} O{u}}
{P \Big\{ B(#1) \in \mathrm d #2 \,\Big \vert \min_{#3\leq z \leq #4} B(z)> 0 , B(#3)=#5 \Big\}}


% brownian meander with drift
\DeclareDocumentCommand{\bmdd}{O{s} O{y} O{0} O{t} O{u}}
{P \Big\{ B^\mu(#1) \in \mathrm d #2 \,\Big \vert \min_{#3\leq z \leq #4} B^\mu(z)> 0 , B^\mu(#3)=#5 \Big\}}


% generalized brownian meander
\DeclareDocumentCommand{\bmdv}{O{s} O{y} O{0} O{t} O{u}}
{P \Big\{B(#1) \in \mathrm d #2 \,\Big \vert \min_{#3\leq z \leq #4} B(z)> v , B(#3)=#5 \Big\}}


% bridge of the brownian meander
\DeclareDocumentCommand{\bmdb}{O{s} O{y} O{0} O{t} O{u} O{v} }
{P \Big\{B(#1) \in \mathrm d #2 \,\Big \vert \min_{#3\leq z \leq #4} B(z)> 0 , B(#3)=#5, 
	B(#4)=#6 \Big\}}


% bridge of the brownian meander
\DeclareDocumentCommand{\bmdbd}{O{s} O{y} O{0} O{t} O{u} O{v} }
{P \Big\{B^\mu(#1) \in \mathrm d #2 \,\Big \vert \min_{#3\leq z \leq #4} B^\mu(z)> 0 , 	
	B^\mu(#3)=#5, 
	B^\mu(#4)=#6 \Big\}}




% numbering 
\newcommand\numberthis{\addtocounter{equation}{1}\tag{\theequation}}
\numberwithin{equation}{section}

%% bmd to copy
%
%
% P \Big(B(s) \in \mathrm d y \,\Big \vert \min_{0\leq z \leq t} B(z)> 0 , B(0)=u \Big)
%
%
%
%


\setlength{\parindent}{0pt}
\allowdisplaybreaks



\theoremstyle{remark}
\newtheorem{remark}{Remark}[section]



\theoremstyle{plain}
\newtheorem{theorem}{Theorem}[section]
\newtheorem{lemma}{Lemma}[section]
\newtheorem{prop}{Proposition}[section]

\theoremstyle{definition}
\newtheorem{corollary}{Corollary}[section]

\theoremstyle{definition}
\newtheorem{definition}{Definition}[section]


\newcommand{\lemmaautorefname}{Lemma}


%opening
\title{SOME RESULTS ON THE BROWNIAN MEANDER WITH DRIFT}
\author{F. Iafrate and E. Orsingher}
\address{Sapienza, University of Rome, Italy.}
\email{enzo.orsingher@uniroma1.it, francesco.iafrate@uniroma1.it}
\date{\scriptsize \texttt{\today}}





\begin{document}



\begin{abstract}
%We present some generalizations of the Brownian meander
%(also with drift) and study also the distribution of
%its maximum and first passage time . Analogous generalizations
%are presented for the bridge of the Brownian meander. 
%The last part of the paper is devoted to the sojourn time 
%$\Gamma_{l,t}$ spent on $[0,\infty)$ during the interval
%$(0,t)$ under the condition that up to time $l<t$ 
%$\min_{ 0\leq s \leq l} B(s)>0$. We obtain 
%some generalizations of the arcsine law also for the case 
%of the Brownian excursion. 
In this paper we study the drifted Brownian meander, that is a Brownian motion starting from $ u $
and subject to the condition that $ \min_{ 0\leq z \leq t} B(z)> v  $ with $  u > v $. 
The limiting process for $ u \downarrow v $ is analyzed and the sufficient conditions for its construction
are given. 
We also study the distribution of the maximum of the meander with drift and the related first-passage times. 
The representation of the meander endowed with a drift is provided and extends the well-known result of the 
driftless case. 
The last part concerns the drifted excursion process the distribution of which coincides with the driftless case. 
%The sojourn time of a process related to the bridge of the meander is also examined. 
\end{abstract}
\maketitle

\keywords{\small \textbf{Keywords}: Tightness, Weak convergence, First passage times, Absorbing drifted Brownian motion, Drifted Brownian Excursion
}

\section{Introduction}


\input{intro-new.tex}


\section{Preliminaries}\label{sec:prelim}
\input{preliminaries.tex}



\section{Weak convergence to the Brownian meander with drift}\label{sec:joint}
\input{joint-new.tex}
%\input{joint.tex}


\input{max-bmd-new.tex}

%\section{Some generalizations of the Brownian meander}
%
%\input{general-mdr.tex}


\input{repr-mdr-drift.tex}

%\section{Bridge of the Brownian meander}\label{sec:excursion}
%
%\input{bridge-bmd.tex}
%


\section{Excursion with drift}

\input{excursion.tex}



%\section{The distribution of the maximum of the Brownian meander}
%
%\input{max-bmd.tex}
%
%
%\section{On the first passage time of the Brownian meander}
%
%\input{first-psg.tex}
%
%
%
%
%
%\section{Sojourn time of the Brownian meander}
%
%\input{sojourn-time-dist-fk.tex}



\bibliographystyle{abbrvnat}
\bibliography{biblio}


\end{document}
