
\section{Congestion Control: An Adapted SFA Policy \label{sec:congestion_control}}

In this section, we describe an online congestion control policy,
and analyze its performance in terms of explicit bounds on the flow
waiting time, as stated in Theorem \ref{thm:congestion} in Section
\ref{sec:results}. We start by describing the 
store-and-forward allocation policy (SFA) that is introduced in \cite{proutiere_thesis}
and analyzed in generality in \cite{bonald2003insensitive}. We describe
its performance in Section \ref{subsec:SFA}. 
Section \ref{subsec:congestion_control} details our congestion control 
policy, which is an adaptation of the SFA policy.

\subsection{Store-and-Forward Allocation (SFA) Policy\label{subsec:SFA}}

We consider a bandwidth-sharing networking operating in continuous
time. A specific allocation policy of interest is the store-and-forward
allocation (SFA) policy. 

\medskip
\noindent\textbf{Model. }Consider a continuous-time network with a set $\text{\ensuremath{\mathcal{L}}}$
of resources. Suppose that each resource $l\in\mathcal{\mathcal{L}}$
has a capacity $C_{l}>0.$ Let $\mathcal{J}$ be the set of routes.
Suppose that $|\mathcal{L}|=L$ and $|\mathcal{J}|=J$.
Assume that a unit volume of flow on route $j$ consumes an amount
$B_{lj}\geq0$ of resource $l$ for each $l\in\mathcal{L}, $ where
$B_{lj}>0$ if $l\in j$, and $B_{lj}=0$ otherwise. The simplest
case is where $B_{lj}\in\{0,1\},$ i.e., the matrix $B=(B_{lj},\;l\in\mathcal{L},\;j\in\mathcal{J})\in\Real_{+}^{L\times J}$
is a $0$-$1$ incidence matrix. We denote by $\mathcal{\mathcal{K}}$
the set of all resource-route pairs $(l,j)$ such that route $j$
uses resource $l$, i.e., $\mathcal{K}=\left\{ (l,j):\;B_{lj}>0\right\} $;
assume that $|\mathcal{K}|=K.$

Flows arrive to route $j$ as a Poisson process of rate $\lambda_i$. For a flow of size
$x$ on route~$j$ arriving at time $t_{\text{start}},$ its completion
time $t_{\text{end}}$ satisfies 
\[
x=\int_{t_{\text{start}}}^{t_{\text{end}}}c(t)dt,
\]
where $c(t)$ is the bandwidth allocated to this flow on each link
of route $j$ at time $t.$ The size of each flow on route $j$ is
independent with distribution equal to a random variable $X_{j}$
with finite mean. Let $\mu_{j}=(\E[X_{j}])^{-1},$ and $\alpha_{j}=\frac{\lambda_{j}}{\mu_{j}}$
be the traffic intensity on route $j.$

Let $N_{j}(t)$ record the number of flows on route $j$ at time $t,$
and let $\N(t)=(N_{1}(t),N_{2}(t),\ldots,N_{J}(t)).$ Since the service
requirement follows a general distribution, the queue length process
$\N(t)$ itself is not Markovian. Nevertheless, by augmenting the
queue-size vector $\N(t)$ with the residual workloads of the set
of flows on each route, we can construct a Markov description of the
system, denoted by a process $\Z(t).$ On average, $\alpha_{j}$
units of work arrive at route~$j$ per unit time. Therefore, a necessary
condition for the Markov process $\Z(t)$ to be positive recurrent
is that 
\[
\sum_{j:\;l\in j}B_{lj}\alpha_{j}<C_{l},\quad\forall l\in\mathcal{L}.
\]
An arrival rate vector $\blambda=(\lambda_{j}:j\in\mathcal{J})$ is
said to be admissible if it satisfies the above condition. Given
an admissible arrival rate vector $\blambda,$ we can define the load
on resource $l$ as 
\[
g_{l}(\blambda)=\left(\sum_{j:\;j\in l}B_{lj}\alpha_{j}\right)/C_{l}.
\]

We focus on allocation polices such that the total bandwidth allocated
to each route only depends on the current queue-size vector $\n=(n_{j}:\;j\in\mathcal{J})\in\Int_{+}^{J},$ which represents
the number of flows on each route. We consider
a processor-sharing policy, i.e., the total bandwidth $\phi_{j}$
allocated to route $j$ is equally shared between all $n_{j}$ flows.
The bandwidth vector $\mathbf{\bphi}(\n)=(\phi_{1}(\n),\ldots,\phi_{J}(\n))$
must satisfy the capacity constraints
\begin{equation}
\sum_{j:\;j\in l}B_{lj}\phi_{j}(\n)\leq C_{l},\qquad\forall l\in\mathcal{L}.\label{eq:capacity_constraint}
\end{equation}

\smallskip
\noindent\textbf{Store-and-Forward Allocation (SFA) policy. }The SFA policy
was first considered by Massouli\'{e} (see page 63, Section 3.4.1 in~\cite{proutiere_thesis}) and
later analyzed in the thesis of Prouti\`{e}re \cite{proutiere_thesis}.
It was shown by Bonald and Prouti\`{e}re \cite{bonald2003insensitive}
that the stationary distribution induced by this policy has a product
form and is insensitive for phase-type service distributions. Later
Zachary established its insensitivity for general service distributions
\cite{zachary2007insensitivity}. Walton \cite{walton2009fairness}
and Kelly et al.\ \cite{kelly2009resource} discussed the relation
between the SFA policy, the proportionally fair allocation and multiclass
queueing networks. Due to the insensitivity property of SFA, the invariant
measure of the process $\N(t)$ only depends on the parameters $\blambda$
and $\boldsymbol{\mu}.$ 

We introduce some additional notation to describe this policy. Given
the vector $\n=(n_{j}:\;j\in\mathcal{J})\in\Int_{+}^{J},$ which represents
the number of flows on each route, we define the set 
\begin{align*}
U(\n) & =\left\{ \m=(m_{lj}:\;(l,j)\in\mathcal{K})\in\Int_{+}^{K}:\;n_{j}=\sum_{l:\;l\in j}m_{lj},\;\forall j\in\mathcal{J}\right\} .
\end{align*}
With a slight abuse of notation, for each $\m\in\Int_{+}^{K},$ we
define $m_{l}:=\sum_{j:\;l\in j}m_{lj}$ for all $l\in\mathcal{L}.$
We also define quantity
\[
{m_{l} \choose m_{lj}:\;j\ni l}=\frac{m_{l}!}{\prod_{j:\;l\in j}(m_{lj}!)}.
\]
 For $\n\in\Int_{+}^{J},$ let
\[
\Phi(\n)=\sum_{\m\in U(\n)}\prod_{l\in\mathcal{L}}\left({m_{l} \choose m_{lj}:\;j\ni l}\prod_{j:\;l\in j}\left(\frac{B_{lj}}{C_{l}}\right)^{m_{lj}}\right).
\]
 We set $\Phi(\n)=0$ if at least one of the components of $\n$ is
negative. 

The SFA policy assigns rates according to the function $\bphi:\Int_{+}^{J}\rightarrow\Real_{+}^{J},$
such that for any $\n\in\Int_{+}^{J},$ $\bphi(\n)=(\phi_{j}(\n))_{j=1}^{J},$
with 
\[
\phi_{j}(\n) = \frac{\Phi(\n-\e_{j})}{\Phi(\n)},
\]
 where $\e_{j}$ is the $j$-th unit vector in $\Int_{+}^{J}.$

\smallskip

The SFA policy described above has been shown to be feasible, i.e.,
$\bphi$ satisfies condition (\ref{eq:capacity_constraint}) \cite[Corollary 2]{kelly2009resource},
\cite[Lemma 4.1]{walton2009fairness}. Moreover, prior work has established
that the bandwidth-sharing network operating under the SFA policy
has a product-form invariant measure for the number of waiting
flows \cite{bonald2003insensitive,walton2009fairness,kelly2009resource,zachary2007insensitivity}, and the measure is insensitive to the flow size distributions~\cite{zachary2007insensitivity,walton2009fairness}. The above work is summarized in the following theorem~\cite[Theorem 4.1]{shah2014SFA}.
\begin{thm}
\label{thm:SFA} Consider a bandwidth-sharing network operating under
the SFA policy described above. If
\[
\sum_{j:\;l\in j}B_{lj}\alpha_{j}<C_{l},\quad\forall l\in\mathcal{L},
\]
then the Markov process $\Z(t)$ is positive
recurrent, and $\N(t)$ has a unique stationary distribution $\bpi$ given by
\[
\bpi(\n)=\frac{\Phi(\n)}{\Phi}\prod_{j\in\mathcal{J}}\alpha_{j}^{n_{j}},\;\n\in\Int_{+}^{J}, \label{eq:SFA_distribution}
\]
 where 
\[
\Phi=\prod_{l\in\mathcal{L}}\left(\frac{C_{l}}{C_{l}-\sum_{j:\;l\in j}B_{lj}\alpha_{j}}\right).
\]

\end{thm}
By using Theorem \ref{thm:SFA} (also see \cite[Propositions 4.2 and 4.3]{shah2014SFA}),
an explicit expression can be obtained for $\E[N_{j}],$  the expected number of flows on each route.
\begin{prop}
\label{prop:delay_BN}Consider a bandwidth-sharing network operating
under the SFA policy, with the arrival rate vector $\blambda$ satisfying
$$\sum_{j:\;l\in j}B_{lj}\alpha_j<C_{l},\;\forall l\in\mathcal{L}.$$
For each $l\in\mathcal{L},$ let $g_{l}=\left(\sum_{j:\;l\in j}B_{lj}\alpha_j\right)/C_{l}.$
Then $\N(t)$ has a unique stationary distribution. In particular,
for each $j\in\mathcal{J},$ 
\[
\E[N_{j}]=\sum_{l:\;l\in j}\frac{B_{lj}\alpha_j}{C_{l}}\frac{1}{1-g_{l}}.
\]
\end{prop}

\begin{proof}
	Define a measure $\tilde{\bpi}$ on $\Int_{+}^{K}$ as follows \cite[ Proposition 4.2 ]{shah2014SFA}:
	for each $\m\in\Int_{+}^{K},$
	\[
	\tilde{\bpi}(\m)=\frac{1}{\Phi}\prod_{l=1}^{L}\left({m_{l} \choose m_{lj}:\;j\ni l}\prod_{j:\;l\in j}\left(\frac{B_{lj}\alpha_{j}}{C_{l}}\right)^{m_{lj}}\right).
	\]
	It has been shown that $\tilde{\bpi}$ is the stationary distribution
	for a multi-class network with processor sharing queues \cite[Proposition 1]{kelly2009resource}, \cite[Proposition 2.1]{walton2009fairness}.
	Note that the marginal distribution of each queue in equilibrium is
	the same as if the queue were operating in isolation. To be precise,
	for each queue $l,$ the queue size is distributed as if the queue has Poisson
	arrival process of rate $g_{l}.$ Hence the size of queue $l,$ denoted
	by $M_{l},$ satisfies $\E_{\tilde{\bpi}}[M_{l}]=\frac{g_{l}}{1-g_{l}}.$ Let $M_{lj}$
	be the number of class $j$ customers at queue $l.$ Then by Theorem
	3.10 in \cite{kelly1979reversibility}, we have
	\begin{equation}
	\E_{\tilde{\bpi}}[M_{lj}]=\frac{B_{lj}\alpha_{j}}{\sum_{i:\;l\in i}B_{li}\alpha_{i}}\E_{\tilde{\bpi}}[M_{l}]=\frac{B_{lj}\alpha_{j}}{C_{l}}\frac{1}{1-g_{l}}.\label{eq:Exp_Mlj}
	\end{equation}
	
	
	Given the above expressions of $\tilde{\bpi}$ and $\bpi$, we can show that $\bpi(\n)=\sum_{\m\in U(\n)}\tilde{\bpi}(\m),$ $\forall\n\in\Int_{+}^{J}$.
	This fact has been shown by Bonald and Prouti\`{e}re \cite{bonald2004performance}
	and Walton \cite{walton2009fairness}. Hence we have 
	\begin{align*}
	\E_{\bpi}[N_{j}] & =\sum_{k=0}^{\infty}k\left(\sum_{\n\in\Int_{+}^{J}:n_{j}=k}\bpi(\n)\right)=\sum_{k=0}^{\infty}k\left(\sum_{\n\in\Int_{+}^{J}:n_{j}=k}\sum_{\m\in U(\n)}\tilde{\bpi}(\m)\right)\\
	& =\sum_{k=0}^{\infty}k\left(\sum_{\m\in\Int_{+}^{K}}\Indicator(\sum_{l:\;l\in j}m_{lj}=k)\tilde{\bpi}(\m)\right)\\
	& =\E_{\tilde{\bpi}} \Big[ \sum_{l:\;l\in j}M_{lj} \Big] \\
	& =\sum_{l:\;l\in j}\frac{B_{lj}\alpha_{j}}{C_{l}}\frac{1}{1-g_{l}}
	\end{align*}
	as desired.
\end{proof}

We now consider the departure processes of the bandwidth-sharing network. It is a known result that the bandwidth sharing network with the SFA policy is reversible, as summarized in the proposition below. This fact has been observed by Prouti\`{e}re~\cite{proutiere_thesis}, 
Bonald and Prouti\`{e}re~\cite{bonald2004performance}, and Walton~\cite{walton2009fairness}. 
Note that Lemma~\ref{lem:poisson} follows immediately from the reversibility property of the network and the fact that flows of each type arrive according to a Poisson process.
\begin{prop}[Proposition 4.2 in~\cite{walton2009fairness}] \label{prop:reversibility} 
Consider a bandwidth sharing network operating under SFA. If $\sum_{j:\;l\in j}B_{lj}\alpha_{j}<C_{l},\;\forall l\in\mathcal{L},$ then the network is reversible.
\end{prop}


\subsection{An SFA-based Congestion Control Scheme\label{subsec:congestion_control}}

We now describe a congestion control scheme for a general acyclic
network. As mentioned earlier, every \emph{exogenous} flow is pushed to the corresponding external buffer upon arrival. The internal buffers store flows that are either transmitted from other nodes or
moved from the external buffers. We want to emphasize that
only packets present in the internal queues are eligible
for scheduling. The congestion control mechanism determines the time
at which each external flow should be admitted into the network for
transmission, i.e., when a flow is removed from the external buffer
and pushed into the respective source internal buffer.  

The key idea of our congestion control policy is as follows. Consider
the network model described in Section \ref{sec:Model-and-Notation},
denoted by $\mathcal{N}$. Let $A$ be an $N\times J$ matrix where
\[
A_{nj}=\begin{cases}
1, & \text{if route \ensuremath{j} passes through node \ensuremath{n}}\\
0, & \text{otherwise.}
\end{cases}
\]
The corresponding admissible region $\Lambda$ given by (\ref{eq:capacity})
can be equivalently written as 
\[
\Lambda=\left\{ \blambda\in\Real_{+}^{J|\mathcal{X}|}:\;A\balpha<\boldsymbol{1}\right\} ,
\]
 where \emph{$\balpha=(\alpha_{j},\;j\in\mathcal{J})$ }with\emph{
$\alpha_{j}:=\sum_{x\in\mathcal{X}}\;x\lambda_{j,x},$ }and $\boldsymbol{1}$
is an $N$-dim vector with all elements equal to one. 

Now consider a virtual bandwidth-sharing network, denoted by \textbf{$\mathcal{N}_{B}$},\textbf{
}with $J$ routes which have a one-to-one correspondence with the
$J$ routes in $\mathcal{N}$. The network $\mathcal{N}_{B}$ has
$N$ resources, each with unit capacity. The resource-route relation
is determined precisely by the matrix $A.$ Observe that each resource
in $\mathcal{N}_{B}$ corresponds to a node in $\mathcal{N}.$ Assume
that the exogenous arrival traffic of \textbf{$\mathcal{N}_{B}$ }is
identical to that of $\mathcal{N}$. That is, each flow arrives at
the same route of $\mathcal{N}$\textbf{ }and\textbf{ $\mathcal{N}_{B}$
}at the same time. \textbf{$\mathcal{N}_{B}$ }will operate under
the insensitive SFA policy described in Section \ref{subsec:SFA}. 

\smallskip

\noindent\textbf{Emulation scheme} $E$. Flows arriving at $\mathcal{N}$ will
be admitted into the network for scheduling in the following way.
For a flow $f$ arriving at its source node $v_{s}$ at time $t_{f}$,
let $\delta_{f}$ denote its departure time from \textbf{$\mathcal{N}_{B}$}.
In the network $\mathcal{N},$ this flow will be moved from the external buffer to the internal queue of node $v_{s}$ at time $\delta_{f}.$
Then all packets of this flow will simultaneously become eligible
for scheduling. Conceptually, all flows will be first fed into the
virtual network \textbf{$\mathcal{N}_{B}$ }before being admitted
to the internal network for transmission. 

\smallskip
This completes the description of the congestion control policy. Observe that the centralized controller simulates the virtual network $\mathcal{N}_{B}$ with the SFA policy in parallel, and tracks the departure processes of flows in $\mathcal{N}_{B}$ to make congestion control decisions for $\mathcal{N}$. According to the above emulation scheme $E$, we have the following lemma.
\begin{lem}
\label{lem:delay_flow_congestion}Let $D_{B}$ denote the delay of
a flow $f$ in $\mathcal{N}_{B},$ and $D_{W}$ be the amount of time
the flow spends at the external buffer in $\mathcal{N}$. Then
$D_{W}=D_{B}$ surely. 
\end{lem}



For each flow type $(j,x)\in\mathcal{T},$ let $D_{B}^{(j,x)} (\blambda)$ denote
the sample mean of the delays over all type-$(j,x)$ flows in the bandwidth
sharing network $\mathcal{N}_{B}.$ That is, $D_{B}^{(j,x)} (\blambda) := \limsup_{k\rightarrow\infty}\frac{1}{k}\sum_{i=1}^{k}D_{B}^{(j,x),i} (\blambda),$
where $D_{B}^{(j,x),i} (\blambda)$ is the delay of the $i$-th type-$(j,x)$
flow. From Theorem~\ref{thm:SFA} and Proposition~\ref{prop:delay_BN}, we readily deduce the following
result.

\begin{prop}
\label{prop:delay_SFA}In the network $\mathcal{\mathcal{N}}_{B},$
we have 

\[
D_{B}^{(j,x)} (\blambda)=\sum_{v:v\in j}\frac{x}{1-f_{v}},\quad\text{with probability \ensuremath{1.}}
\]
\end{prop}

\begin{proof}
 Recall that in the network $\mathcal{N}_{B},$ flows arriving on each route are
further classified by the size. Each type-$(j,x)$ flow
requests $x$ amount of service, deterministically. It is not difficult to see that properties of the bandwidth-sharing network stated in Section~\ref{subsec:SFA} still hold with the refined classification. We denote by $N_{j,x}(t)$
the number of type-$(j,x)$ flows at time $t.$ A Markovian description
of the system is given by a process $\Z(t)$, whose state contains the queue-size
vector $\N(t)=(N_{j,x}(t),j\in\mathcal{J},x\in\mathcal{X})$ and the residual workloads of the set of flows
on each route. 

By construction, each resource $v$ in the network $\mathcal{N}_{B}$ corresponds to
a node $v$ in the network $\mathcal{N},$ with unit capacity. The load $ g_v $ on the resource
$v$ of $ \mathcal{N}_{B} $ and the load $ f_v $ on node $ v $ of $ \mathcal{N} $ satisfy the relationship 
\[
g_{v}=\sum_{v:v\in j}A_{vj}\alpha_{j}=f_{v}.
\]
As $\blambda\in\Lambda,$ $f_{v}<1$ for each $v\in\mathcal{V}.$
By Theorem \ref{thm:SFA}, the Markov process $\Z(t)$ is positive
recurrent, and $\N(t)$ has a unique stationary distribution. Let $N_{j,x}$
denote the number of type-$(j,x)$ flow in the bandwidth sharing network
$\mathcal{N}_{B}$, in equilibrium. Following the same argument as
in the proof of Proposition \ref{prop:delay_BN}, we have 
\[
\E[N_{j,x}]=\sum_{v:\;v\in j}\frac{x\lambda_{j,x}}{1-f_{v}}.
\]

Consider the delay of a type-$(j,x)$ flow in equilibrium, denoted
by $\bar{D}_{B}^{(j,x)} (\blambda)$. By applying Little's Law to the stationary
system concerning only type-$(j,x)$ flows, we can obtain 
\[
\E[\bar{D}_{B}^{(j,x)} (\blambda)]=\frac{\E[N_{j,x}]}{\lambda_{j,x}}=\sum_{v:\;v\in j}\frac{x}{1-f_{v}}.
\]


By the ergodicity of the network $\mathcal{N}_{B},$ we have
\[
D_{B}^{(j,x) (\blambda)}=\E[\bar{D}_{B}^{(j,x)} (\blambda)],\quad\text{with probability}\;1,
\]
thereby completing the proof of Proposition~\ref{prop:delay_SFA}.
\end{proof}

Equipped with Proposition~\ref{prop:delay_SFA} and Lemma \ref{lem:delay_flow_congestion}, we are now ready to prove Theorem \ref{thm:congestion}.
\begin{proof}[Proof of Theorem \ref{thm:congestion}] 

By the definition, we have $\frac{1}{1-f_{v}}\leq\frac{1}{1-\rho_{j}(\blambda)}$,
for each $v\in j.$ Proposition \ref{prop:delay_SFA} implies that, with probability 1,
\[
D_{B}^{(j,x)} (\blambda)=\sum_{v:v\in j}\frac{x}{1-f_{v}}\leq\frac{xd_{j}}{1-\rho_{j}(\blambda)},
\]
It follows from Lemma \ref{lem:delay_flow_congestion} that $D_{W}^{(j,x)} (\blambda) = D_{B}^{(j,x)} (\blambda).$
Therefore, we have
\[
D_{W}^{(j,x)} (\blambda) \leq\frac{xd_{j}}{1-\rho_{j}(\blambda)},\quad\text{with probability}\;1.
\] 
This completes the proof of Theorem~\ref{thm:congestion}.
\end{proof}

