\section{Detector calibration}
\label{sec:calibration}	
EASIER  detectors   are  required  to  measure   faint  and  impulsive
signals. The widely used  figure of merit of the  sensitivity for such
detectors reads as
%%\begin{linenomath*}
\begin{equation}
F  =  \frac{k_{\text{B}}  T_{\text{sys}} }{A_{\text{eff}} \sqrt{\Delta  \nu    \Delta t}},
\label{eq:sensitivity}  
\end{equation}
%%\end{linenomath*}
where  $F$ represents  the flux  resulting  from a  signal that  would
equate the noise fluctuations,  $T_{\text{sys}}$ stands for the system
noise equivalent  temperature (the sum of the  thermal noise collected
by the  antenna and  the electronics noise  added mainly by  the first
amplifier), $k_\text{B}$  is the Boltzmann  constant, $A_{\text{eff}}$
is  the  effective area  of  the antenna  (i.e.   the  portion of  the
incoming radio flux transformed into electrical power), and the square
root  term  is  the  amount   of  samples  over  which  the  noise  is
averaged. In simple cases, $\Delta \nu \Delta t$ is the product of the
bandwidth $\Delta \nu$ with a time  constant of a low pass filter, but
in cases  of transient signals,  the expected duration of  the signals
has to be used for $\Delta t$.  We detail first the calibration of the
sensor including  the measurement or  simulation of the  parameters in
Eq.~\eqref{eq:sensitivity}.   In  a   second  time  we  determine  the
calibration parameters of the adaptation stage of the signal chain.

\subsection{Sensor calibration}
\label{sec:calibrationsensor}
\subsubsection{Antenna effective area}
The effective  area for a  particular wavelength $\lambda$  is derived
from  the knowledge  of the  antenna gain  pattern  $G(\theta, \phi)$,
i.e. the gain of the antenna as a function of the direction:
\begin{eqnarray} \label{eq:aeff}
A_{\text{eff}} (\theta, \phi) = {{\lambda^2\ G(\theta, \phi)}\over{4 \pi} } 
\end{eqnarray}
The  gain pattern can  be either  measured or  simulated. It  has been
measured for the  antenna DMX241 from WS International  and for an ATM
horn  coupled to  a Norsat  LNB  in an  anechoic chamber  at the  IMEP
(Institut  de Microelectronique  Electromagnetisme  et Photonique)  at
Grenoble.   In  addition to  these  measurements,  the High  Frequency
Simulation Software (HFSS) from ANSYS~\cite{hfss} was used to simulate
the patterns of  the different antenna types, taking  into account the
setup of the sensors, such as  the presence of a radome. The simulated
effective area of the antenna used in the three setups are represented
in Fig.~\ref{fig:aeff}.
\begin{figure}[!ht]
 \centering
 \hspace*{-3ex}
 \subfigure{\includegraphics[width=0.65\linewidth]{effectiveareas.png}}
 \caption{Simulated     effective    area    for     \mbox{GIGAS61}    (DMX),
   \mbox{GIGADuck-C}(Norsat) and \mbox{GIGADuck-L} (Helix) antennas.}
 \label{fig:aeff}
\end{figure}
\subsubsection{System noise temperature}
The  system  noise  factor  is  defined  as  the  degradation  of  the
signal-to-noise ratio (SNR) along the signal chain stages  and can be
expressed with a system  noise temperature, $T_{\text{sys}}$. The main
contributions to the noise temperature are:
\begin{itemize}
\item  the antenna  temperature $T_{\text{ant}}  $: the  thermal noise
  emitted by  broad microwave  sources such as  the sky or  the ground
  collected by the antenna;
\item the electronics noise  $T_{\text{elec}}$: the noise added by the
  electronics  stage,   usually  dominated  by  the   first  stage  of
  amplification.
\end{itemize}
A radome  used to  protect an  antenna is a  source of  signal losses,
affecting  the  SNR  and  adding   up  a  contribution  in  the  noise
temperature.\\  To  estimate the  temperature  of  the three  detector
versions,  we have  applied  three different  methods.   A well  known
method to  measure a  temperature contribution from  a device  like an
amplification system  is the  $Y$ factor method.   It consists  in the
measurement of the  device output when it is  subject to two different
known  sources   of  noise.   In   the  case  of   \mbox{GIGAS61}  and
\mbox{GIGADuck-C} detectors,  the amplification system is  part of the
feed and cannot be isolated and tested separately. Hence, to apply the
$Y$  factor method and  produce a  stable noise  in the  detector, the
source has to  be a microwave emitting source that  covers most of the
antenna main lobe.  Having two references allows one to cancel out the
gain of  the system and  to extract the  noise.  Another method  is to
make use of  a natural microwave source like the  Sun as a calibration
source when it goes through the field of view of the antenna.  Lastly,
one can  also directly measure  the background radio power  and deduce
the noise temperature provided a precise knowledge of the system gain.
The $Y$  factor method was used  in a dedicated  measurement to obtain
\mbox{GIGAS61} detector  temperatures.  The second method  is used for
the  GIGADuck arrays  by  measuring  the Sun  flux  in the  monitoring
data. As it  will be described later, the Sun signal  was also used to
correct    for   the    pointing   direction    of   \mbox{GIGADuck-C}
antennas.  Finally the  direct method  is used  to measure  the L-band
setup system noise temperature.
\paragraph{GIGAS61 --}
For GIGAS61 detectors,  we apply the $Y$ factor  method to measure the
electronics  noise  temperature. The  measurement  took  place at  the
detector site.  The setup comprises the main components of the nominal
detectors,  namely  the  LNBf,  the  radome  and  the  power  detector
Minicircuit ZX47-50.   The antenna  was oriented consecutively  up and
down  and  the voltage  of  the power  detector  was  recorded with  a
portable  oscilloscope.   The   voltage  difference  between  the  two
measurements  is related  to a  difference of  power according  to the
calibration curve  of the power detector  (see Eq~(\eqref{eq:eqpd}) in
section~\ref{sec:calibrationadapt}).      The     electronics    noise
temperature $T_\text{elec}$ is computed with:
%%\begin{linenomath*}
\begin{equation}
	T_{\text{elec}} = \frac{T_{\text{hot}} - YT_{\text{cold}}}{Y-1},
\end{equation}
%%\end{linenomath*}
where    $Y   =    P_{\text{hot}}/P_{\text{cold}}$,   $T_{\text{hot}}$
($T_{\text{cold}}$) is the antenna temperature when the antenna points
toward the  ground (the sky)  and $P_{\text{hot}}$ ($P_{\text{cold}}$)
are  the   corresponding  powers.  The  antenna   temperature  is  the
brightness  temperature of  the  surrounding sources  weighted by  the
antenna gain:
%%\begin{linenomath*}
\begin{equation}
T_{\text{ant}} = \int_{\theta =  0}^{\theta = \pi}\int_{\phi = 0}^{\phi =2\pi} T_{\text{B}}(\theta, \phi) G(\theta,\phi) \sin{\theta} d\theta d\phi
\label{eq:tant}
\end{equation}
%%\end{linenomath*}
with   $T_{\text{B}}(\theta)$  the   brightness  temperature   in  the
direction  $\theta$.  We applied  the  formula~\eqref{eq:tant} with  a
brightness  temperature  profile  (found  in~\cite{otoshidanslacolle})
which ranges  from \unit[4]{K} in  the sky direction  to \unit[270]{K}
towards  the   ground.  Antenna  temperatures   of  $T_{\text{hot}}  =
\unit[260]{K}$  and $T_{\text{cold}} =  \unit[6]{K}$ are  obtained. It
yields to electronics  temperatures of $T^{\text{GSI}}_{\text{elec}} =
\unit[(114\pm    10)]{K}$    and    $T^{\text{DMX}}_{\text{elec}}    =
\unit[(97\pm  9)]{K}$.    Finally  we  add   the  antenna  temperature
$T_{\text{ant}} = T_{\text{cold}} =  \unit[6]{K}$ to obtain the system
noise temperature.
\paragraph{GIGADuck-C --}
Compared  to  GIGAS61  detectors,  GIGADuck  antennas  have  a  larger
effective  area which  make them  sensitive  to the  Sun flux.   Since
GIGADuck data are part of  the SD data stream including the monitoring
system, the radio baseline  is recorded every \unit[400]{s} with other
information  such as  the outside  temperature. We  use these  data to
search for the Sun signal and estimate the system temperature from it.
The position of the Sun in the sky is well known and the absolute flux
density in the frequency band is based on observations at the Nobeyama
Radio  Observatory  (NRO)  at \unit[3.75]{GHz}~\footnote{The  Nobeyama
  Radio  Polarimeters are  operated by  Nobeyama Radio  Observatory, a
  branch of  National Astronomical Observatory of  Japan}. Examples of
the  Sun  path  through  the   GIGADuck  C-band  array  are  shown  in
Figure~\ref{fig:sunsim}. Most  of the antennas (except  the antenna on
the stations called Juan and  Luis) have the Sun passing through their
field of view during the austral summer.  However, when the Sun is low
in  the sky  (during austral  winter time),  none of  the  antennas is
sensitive  to it.  Since all  GIGADuck antennas  point in  a different
direction, one  expects the  Sun to produce  a signal  with relatively
different  intensity and shifted  time of  maximum according  to their
orientation.  Indeed,  we use  both information to  constrain together
the  system  noise  temperature  and  the pointing  direction  of  the
GIGADuck antenna. \\
The increase of  power $P_\text{Sun}$ induced upon the  passage of the
Sun over the system noise power $P_\text{sys}$ in the antenna field of
view reads
%%\begin{linenomath*}
\begin{equation}
  \label{eq:deltaP}
  \unit[\Delta  P]{[dBm]}  =  10\log_{10}\left(\frac{P_{\text{sys}}  +
    P_{\text{Sun}}(\theta_{\text{Sun}},\phi_{\text{Sun}})
  }{P_{\text{sys}}}     \right)    =    10\log_{10}\left(1+\frac{1}{2}
  \frac{F_{\text{Sun}}A_{\text{eff}}(\theta_{\text{Sun}},\phi_{\text{Sun}})}{k_\text{B}
    T_{\text{sys}}} \right),
\end{equation}
%%\end{linenomath*}
where  $F_{\text{Sun}}$  is  the  total  solar flux  measured  by  the
Nobeyama   Radio   Observatory~\cite{nobeyama}  at   \unit[3.75]{GHz},
$A_{\text{eff}}(\theta_{\text{Sun}},\phi_{\text{Sun}})    $   is   the
antenna effective area  for the given position of the  Sun in the sky,
and the factor $1/2$ is the polarization factor.
\begin{figure}[!ht]
 \centering
 \hspace*{-3ex}
 \subfigure{\includegraphics[width=0.45\linewidth]{sunpolar.png}}
 \subfigure{\includegraphics[width=0.55\linewidth]{newfitexamp.png}}
 \caption{Left: Sun transit for the two solstices in polar coordinates
   where the radial distance  represents the zenith angle. The azimuth
   is set to  0 for the East direction.  The colored circles represent
   the field of view of  the GIGADuck antennas.  Right: Example of the
   baseline during one  day. The top plot shows  the original baseline
   and the outside  temperature along the day.  The  bottom plot shows
   the baseline  once corrected from the  temperature dependence.  The
   fit result is also shown in red.}
 \label{fig:sunsim}
\end{figure}
The radio baseline is  strongly correlated to the outside temperature,
but can  also be  affected by the  humidity in  a non trivial  way. We
first operate  a selection on the  dataset to isolate a  set of stable
days,  i.e.  the  days  when the  baseline  is not  affected by  other
parameters but the outside temperature.  \\Firstly, a time window of 8
hours around the time when the  Sun is expected to produce the highest
signal is removed temporarily.  Then  we reject the singular days when
the  baseline   RMS  is  lower   than  2~ADCu  (compared   to  20~ADCu
typically). These  low variations indicate  that the signal  chain was
faulty  at  that  time.   Days  with large  amplitude,  often  due  to
thunderstorm condition, are  removed by requiring baseline differences
over the day  lower than 200~ADCu. From this  data set, the dependence
of the  radio baseline with the  outside temperature is  fitted with a
linear function.   To improve  the selection of  stable days,  the day
with  the largest  residual is  removed and  the fitting  procedure is
repeated until  no residual larger  than 10~ADCu is found.   Then, the
time window which encompasses the  Sun contribution is restored in the
selected  days  and  the  complete  baselines are  corrected  for  the
temperature dependence.   The final step consists in  fitting the bump
induced by  the Sun flux  with a Gaussian  function and a  third order
polynomial.    An  example  of   the  radio   baseline  is   shown  in
Figure~\ref{fig:sunsim}-right before  the temperature correction (top)
and after (bottom).   \\The selection and fit procedure  are tested by
introducing fake  signals to  mimic the Sun  contribution in  the real
baselines of the  antennas oriented towards the South  (Luis and Juan)
thus  insensitive to  the  Sun.   Signals with  a  Gaussian shape  are
introduced  with  various amplitude  and  time  and are  reconstructed
according to the  method described above.  The uncertainty  due to the
limited knowledge  of the  baseline amounts to  $\pm$\unit[4]{ADCu} on
the amplitude of  the peak and to $\pm$\unit[12]{minutes}  on the time
of  maximum.  The  spread of  the result  of the  fit is  found  to be
$\pm$\unit[5]{ADCu} and $\pm$\unit[6]{minutes}.  \\The goal is to find
the best parameters  to describe the system noise  temperature and the
pointing direction  given the observed amplitude and  time of maximum.
We simulate the signal induced by  the Sun microwave flux for a system
temperature  from \unit[30 to  120]{K} with  \unit[1]{K} step  and for
angles  $\Delta \theta  \in$ [0$^{\circ}$;  20$^{\circ}$]  and $\Delta
\phi \in  $[0$^{\circ}$; 180$^{\circ}$]  around the nominal  angle for
the set  of days selected  in the aforementioned procedure.   For each
set  of  input parameters  ($T_\text{sys}$,  $\Delta \theta$,  $\Delta
\phi$), the baselines  in ADCu are computed.  The  best parameters are
found by minimizing the following $\chi^2$:
%%\begin{linenomath*}
\begin{equation}
	\chi^2   (T_\text{sys})  \rvert_{\Delta   \theta,\Delta  \phi}
        =\sum_{i}         \frac{         (t.o.m._\text{i,sim}        -
          t.o.m._\text{i,data})^2}{\sigma_\text{t.o.m}^2}             +
        \frac{(B_\text{i,sim}           -          B_\text{i,data})^2}
        {\sigma_{\text{B}^2}}
	\label{eq:chi2}
\end{equation}
%%\end{linenomath*}
where each day is labeled with  the index $i$, $t.o.m.$ stands for the
time of the maximum in data and simulation, $B_\text{{i,data}}$ is the
maximum   of   the  fitted   signal   in   ADCu   in  the   data   and
$B_\text{{i,sim}}$ the signal  in the simulation taken at  the time of
the   maximum   measured   in    the   data   (see   the   scheme   in
Figure~\ref{fig:GDtempres}-left).   The   result  are  given   in  the
table~\ref{tab:temptab}.  Angular deviations from the nominal position
are found to be at  most \unit[14]{$^\circ$} (in angular distance) and
temperatures range  from \unit[54]{K}  to \unit[61]{K}. An  example of
the temperature measured for each day  in the data set and the time of
maximum    compared   to    the    simulated   one    is   shown    in
Figure~\ref{fig:GDtempres}-right.
\begin{center}
\begin{table}[t]
\caption{Results   of  fits,   superscript  and   subscript   are  the
  statistical and systematics uncertainty respectively.}
\vspace{3mm}
\centering
\resizebox{\linewidth}{!}{
\begin{tabular}{cccc}
\toprule station name & Popey  & Orteguina & Domo \\ \midrule original
orientation $\theta  / \phi$ &  20 / 120 &  20 / 180  & 20 / 0  \\ new
orientation       $\theta^{\text{stat}}_{\text{sys}},      \phi^{\text{stat}}_{\text{sys}}$      &
25$^{+1/-1}_{+1/-1}$  /  116$^{+1/-1}_{+1/-2}$  &23$^{+1/-1}_{+3/-2}$,
170$^{+1/-1}_{+1/-1}$              &             33$^{+1/-1}_{+1/-1}$,
12$^{+5/-5}_{+6/-5}$\\ System Temperature $T^{\text{stat}}_{\text{sys}} $ & 61$^{+2/
  -2}_{+12/-10}$  &   54$^{+2/-1}_{+12/-7}$  &  58$^{+2/-3}_{+8  /-9}$
\\ \bottomrule
\end{tabular}
}
\label{tab:temptab}
\end{table}
\end{center}

\begin{figure}[!ht]
 \centering
 \hspace*{-3ex}
 \subfigure{\includegraphics[width=0.53\linewidth]{schemeparameters2.png}}
  \subfigure{\includegraphics[width=0.46\linewidth]{temp_orteguina2.png}}
  \caption{Left:  Scheme with  the parameters  used in  the $\chi^{2}$
    function in \eqref{eq:chi2}.   Right: Temperature measured for one
    GIGADuck  detector (Orteguina)  with the  Sun signal  and  time of
    maximum.   We show together  the results  for the  nominal antenna
    orientation  (in   red)  and  the  one  retrieved   from  the  Sun
    observations (in blue).}
 \label{fig:GDtempres}
\end{figure}
\paragraph{GIGADuck-L --}
The L-band sensors are also sensitive to the Solar flux.  Thirty daily
baselines are  overlaid in Fig.~\ref{fig:GDLbaseline_BW}  and exhibits
the Sun  passage (around \unit[18]{h}) but also  other modulation (for
instance around \unit[0]{h}  or \unit[7]{h}). These modulations, whose
origin  may  be the  positioning  satellite  signal,  prevent us  from
quantifying the Sun contributions in the same way as above. Hence, the
noise  temperature  is deduced  from  the  direct  measurement of  the
baseline, simply by  dividing the measured power by  the total gain of
the system. This method requires a precise calibration of the absolute
gain   of   the   detector   which   was  performed   prior   to   the
installation. The  amplifier is pre-terminated and its  gain and noise
temperature  could  be measured  respectively  with  a Vector  Network
Analyser and a Noise figure  meter.  In the C-band this measurement is
made difficult by the use  of LNBf and the impossibility to disconnect
the  amplification stage  from the  feed waveguide.  The  system noise
temperature is measured for all seven L-band detectors, it ranges from
\unit[94]{K} to \unit[145]{K}.

\subsubsection{Sensor bandwidth}
The absolute  gain of  the RF part  which includes the  amplifier, the
bias tee,  the cables  etc., does not  enter directly in  the detector
sensitivity,    but     the    frequency    bandwidth     does    (see
Eq.~\eqref{eq:sensitivity}).  The normalized gains of the LNB used for
\mbox{GIGAS61} (DMX241) and \mbox{GIGADuck-C}  and the \mbox{GIGADuck-L} are represented in
Fig.~\ref{fig:GDLbaseline_BW} and the  effective bandwidth is computed
according to:
%%\begin{linenomath*}
\begin{equation}
  \Delta \nu = \frac{1}{G_{\text{max}}} \int G(f) df
\end{equation}
%%\end{linenomath*}
The obtained effective bandwidths  for \mbox{GIGAS61} detectors are \unit[437
  $\rm \pm$  30]{MHz} and \unit[445  $\rm \pm$ 56]{MHz} for  the GI301
and the  DMX241 respectively.  As for the  \mbox{GIGADuck-C}, a  bandwidth of
\unit[750]{MHz} is measured for the Norsat LNB, and finally an average
of \unit[250]{MHz} is found for the \mbox{GIGADuck-L} LNAs.

\begin{figure}[!ht]
 \centering
 \hspace*{-3ex}
 \subfigure{\includegraphics[width=0.49\linewidth]{jorgehour}}
	 \includegraphics[width=0.49\linewidth]{normedgain.png}
 \caption{Left:    30    daily    baselines   for    Jorge    detector
   (GIGADuck-L). Right:  Normalized gain  of the three  mentioned LNBf
   after the frequency downconversion.  The thick blue and green lines
   are the  average over several  detectors. Only one  measurement was
   performed with the Norsat LNBf (in red). }
 \label{fig:GDLbaseline_BW}
\end{figure}


\subsection{Electronics calibration}
\label{sec:elec}
We describe  here the functioning of the  adaptation electronics.  The
first part of this section is  dedicated to the study of steady signal
of the  adaptation needed  to describe the  baseline level,  while the
second  part describes the  time response,  necessary to  simulate the
full signal  chain. 
\subsubsection{Response to steady signals}

\label{sec:calibrationadapt}
The adaptation electronics  is composed of the power  detector and the
adaptation  board. The power  detector output  voltage $V_{\text{pd}}$
was calibrated  in laboratory using  a noise waveform.  The  noise was
produced  using the  output of  an actual  LNBf placed  in front  of a
microwave absorber  to obtain the same  spectrum as in  the data.  The
input  power   $P_{\text{in}}$  was  varied   with  attenuators.   The
power-voltage characteristic reads:
%%\begin{linenomath*}
\begin{equation}
  V_{\text{pd}}  [\text{V}]  =  -0.0234 P_{\text{in}}  [\text{dBm}]  +
  \text{offset}_1,
\label{eq:eqpd}
\end{equation}
%%\end{linenomath*}
$\text{offset}_1 $ is  the voltage offset of the  power detector.  The
power detector voltage  is then amplified by a factor  4.2 to obtain a
final power  dynamics of \unit[20]{dB}  over the \unit[2]{V}  swing of
the SD  acquisition.  An offset was  designed to be  adjustable on the
adaptation  board to  make  up  for the  differences  of the  detector
gains. The overall conversion from the input power to the ADCu is:
%%\begin{linenomath*}
\begin{equation}
  \text{P[ADCu]} = 50.2 P_{\text{in}} [\text{dBm}] + \text{offset}_2,
\label{eq:eqcalibration}
\end{equation}
%%\end{linenomath*}
where $\text{offset}_2$ accounts for the total offset.

\subsubsection{Response to impulsive signal}
\label{sec:elecimpulsive}
\paragraph{Power detector --}
To understand the power detector response to impulsive signals, we set
a detection chain  in the laboratory composed of a  LNBf followed by a
power  detector.  An  impulsive  and  high frequency  (HF)  signal  is
produced  by  the spark  of  an  electronic  lighter.  The  signal  is
recorded simultaneously after the LNBf and after the power detector by
a  fast  oscilloscope.   An  example  of these  signals  is  shown  in
Fig.~\ref{fig:powerdetsim}.   We  can  therefore  build  a  method  to
reproduce the  power detector output from  a HF signal.   We find that
the power  detector output  is well reproduced  when one  performs the
convolution  of  the  HF  signal  in dBm  (logarithmic  unit)  and  an
exponential function with a decay constant $\tau$:
%%\begin{linenomath*}
\begin{equation}
  V_{\text{PD}}(t) = k_{1} \int_{t>0}P_{\text{dBm}}(u)\exp{\left(\frac{t-u}{\tau}\right)}du + k_{2}
  \label{eq:convolution}
\end{equation}
%%\end{linenomath*}
The   factor  $k_{1}$   is   fixed  to   the   conversion  factor   in
Eq.~\eqref{eq:eqpd},  $  k_{2}$  is  a  floating offset  and  $\tau  =
\unit[6.3]{ns}$ was found to provide the best fit to the data.
\footnote{The  first seven  detectors of  \mbox{GIGAS61} have  a  longer time
  response  $\tau_{\text{capa}} =  \unit[41.5]{ns}$ due  to  an output
  capacitor  present by  default  in the  power  detector ZX47-50  and
  removed in the following version of EASIER.}
\begin{figure}[!ht]
 \centering
 \hspace*{-3ex}
 \subfigure{\includegraphics[width=0.7\linewidth]{capa_method3.png}}
 \caption{Example of RF and  power detector waveforms.  The top pannel
   shows the  RF waveform, the  middle pannel the waveforms  after the
   power  detector: the  blue  one is  a  measurement and  the red  is
   simulated from  the waveform  on the top  pannel.  The  lower panel
   show the difference of the power detector waveforms.}
 \label{fig:powerdetsim}
\end{figure}

\paragraph{Adaptation board --}
To measure  the response  of the  adaptation board, we  add it  to the
calibration setup  described in  the previous paragraph.   We recorded
simultaneously  the input of  the board  and its  output. We  find the
board response  by measuring the transfer function $\tilde{H}(f)$  in the frequency
domain:
%%\begin{linenomath*}
\begin{equation}
  \tilde{H}(f)                                                        =
  \frac{\tilde{V}_{\text{out}}(f)}{\tilde{V}_{\text{in}}(f)}.
\end{equation}
%%\end{linenomath*}
The   gain  and   the  phase   of   the  board   are  represented   in
Fig.~\ref{fig:board}.  The  time   response  is  obtained  by  Fourier
transformation. \\The last part of  the chain, the Auger SD front end, is
simulated   with  a   low-pass   filter  with   $   f_{\text{cut}}  =
\unit[20]{MHz}$ and by sampling in time and amplitude.
\begin{figure}[!ht]
 \centering
 \hspace*{-3ex}
 \subfigure{\includegraphics[width=0.65\linewidth]{adaptationboardcarac.png}}
 \caption{Measurement and fit of the gain and phase of the adaptation board.}
 \label{fig:board}
\end{figure}
