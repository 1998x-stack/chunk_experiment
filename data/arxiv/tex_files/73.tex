\section{Discussion}


\added{In this section, we summarize the key results of our analysis 
 and provide insights into how these results can be utilized for better 
group event planning experiences. The results cover the impact of user mobility,
individual preference, and the host preference on the group event scheduling process.
We also discuss the impact of user behavior during the voting process.
\newline \textbf{User Mobility.} The analysis in Section
\ref{sec:impact-user-mobility} showed that users with higher mobility are more
likely to be active participants in group events. They are 
more active in voting for proposed event location and event time. We attribute this
behavior to the capability of these users to travel greater distances (perhaps
due to owning a car or having more available time), which gave them flexibility
in selecting location and time options for events. One way to use this
observation for better event planning would be to recommend more diverse
locations and dates for highly mobile users, as they tend to be more willing to
explore new options. On the other hand, users with less mobility should not be
overwhelmed with a large number of choices. In addition to recommending
event locations and time, this observation can be used for forming groups by matching users
with similar mobility levels in the same group, which can lead to smoother event
planning experiences.
\newline \textbf{Individual Preference.} The analysis in Section
\ref{sec:impact-individual-preference} revealed typical patterns related to
individual preferences for event time and locations. First, with regard to event 
location, users tend to arrange events at nearby locations to avoid
traveling long distances. Second, with regard to event time, on weekdays, users want to schedule
events after their working hours, while on weekends users show more
flexibility. These observations are worth considering for smarter group event
planning. For event locations, the application could suggest places such that
the mean travel distance for the group members is minimized, so as to provide a
reasonable compromise for the whole group. Likewise, suggested event time 
should occur outside of typical working hours of the group members.  
\newline \textbf{Host Preference.}  The results in
Section~\ref{sec:impact-host-preference} show that group event planning is
heavily influenced by the host who creates the invitation.  From the analysis in this section,
it is evident that the final event location is on average closer to the host's
frequented locations than that of other group members.  We also see that
the final event locations and dates are more likely to be the options voted by the
host.  However, the influence of the host can also lead to negative outcomes:
when the host chooses not to follow the group's consensus, event attendance is
reduced.  These effects point to the need to carefully consider the preferences
of the host, and how these preferences align with the preferences of the group, when
providing recommendations to event participants.  Effective communication
mechanisms between the host and the participants should also be provided.
% Hence, the application should be designed carefully so as to minimize any
% negative outcomes. For example, in the OutWithFriendz app, we could have
% removed the final decision making authority given to the host and left it to
% the group to decide the final location and time through the chat screen after
% the voting process is closed by the host.
\newline \textbf{Voting Analysis.}  The analysis in
Section~\ref{sec:voting-process} shows that the votes cast by early voters are
very likely to affect late voters.  Late voters tend to vote for fewer options,
and these options tend to match those that have already received votes from
early voters.  This phenomenon can be used to improve the event planning
experience.  For example, we could encourage users to vote early, so that their
votes will carry more weight.  We could also hide existing voting results, so as
to prevent existing votes from biasing later voters, and facilitate the voting
process by providing voting recommendations to users based on their historical
voting patterns.
}

\deleted{\textbf{Expanding event coverage.} Though we have collected more
than 300 completed group events, we hope to grow our user base through more
effective advertising, so that we may achieve viral adoption and gather data at
even larger scales.}  

% We discovered a series of factors that may impact event
% attendance rate. However, we would like to expand our data collection in
% multiple dimensions.  We would like to increase the number of events with large
% group size, and add to these pairwise groups.  We hope to grow the user base
% through more effective advertising so that we can obtain viral adoption and data
% at much larger scales.
% We also hope to be able to expand our studies to include pairwise user groups.

% The majority of events in our dataset are related to dining.  To get more
% useful data for more detailed analysis, we are actively doing market
% advertising for our OutWithFriendz application and expect to have more events
% coming in within a few months.

\deleted{\textbf{Facebook friends requirement}.  To obtain a user's friend list, OutWithFriendz currently only allows users to log in through their Facebook accounts. Users may also want to invite people who are not already a Facebook friend or do not use Facebook at all. This limits our app's ability to support larger groups. We plan to design an ``Add Friend' function which enables users to log in and connect with other users directly within the application.}

\deleted{\textbf{Different voting mechanisms.} All the polls in OutWithFriends are
currently designed to be open polls, which means late-coming voters can see
existing voting results, which may influence their vote.  In a future version of
the application, we plan to add a closed poll option. For closed polls, existing
voting results will be hidden from new voters. This functionality will allow us
to examine how a closed poll mechanism influences the group decision-making
process.}

\deleted{\textbf{Group recommendation.} 
We have seen from our work on OutWithFriendz that a number of factors influence
group decision making behavior. We believe that group context could be seen as
inhabiting a latent trait space, similar to how users inhabit a latent user
trait space in the matrix factorization framework for individual recommendation.
Furthermore, our work has revealed that host and individual influence within a
group plays a significant role in the group decision-making process. In future
work, we intend to pursue the development of a group recommendation system that
incorporates these ideas into a probabilistic model for group preference
behavior. Building a distributed system implementation of mobile group
recommendation will help us gain a better understanding of mobile group
dynamics in the real world, and provide useful suggestions for group organizers.}



