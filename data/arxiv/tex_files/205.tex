\section{Further Details on Prior Work on Asymmetirc Memory}
\label{sec:priorwork}

Read-write asymmetries have been studied in the context of NAND Flash
chips~\cite{BT06, Eppstein14, Gal05, ParkS09}, focusing on how to
balance the writes across the chip to avoid uneven wear-out of
locations.  Targeting instead the new technologies, Chen et
al.~\cite{Chen11} and Viglas~\cite{Viglas12, Viglas14} presented
write-efficient algorithms for database operators such as hash joins
and sorting.  Blelloch et al.~\cite{BFGGS15} defined several
sequential and parallel computation models that take asymmetric
read-write costs into account, and analyzed and designed sorting
algorithms under these models.  Their follow-up
paper~\cite{blelloch2016efficient} presented sequential algorithms for
various problems that do better than their classic counterparts under
asymmetric read-write costs, as well as several lower bounds.  Carson
et al.~\cite{carson2016write} presented write-efficient sequential
algorithms for a similar model, as well as write-efficient parallel
algorithms (and lower bounds) on a distributed memory model with
asymmetric read-write costs, focusing on linear algebra problems and
direct N-body methods.  Ben-David et al.~\cite{BBFGGMS16} proposed a
nested-parallel model with asymmetric read-write costs and presented
write-efficient, work-efficient, low \depth{} (span) parallel
algorithms for reduce, list contraction, tree contraction,
breadth-first search, ordered filter, and planar convex hull, as well
as a write-efficient, low-\depth{} minimum spanning tree algorithm
that is nearly work-efficient.  Jacob and Sitchinava~\cite{jacob2017}
showed lower bounds for an asymmetric external memory model.  In each
of these models, there is a small amount of \local{} memory that can
be used to help minimize the number of writes to the large asymmetric
memory.
