\section{Lessons learned} \label{sec:lessons}

The major challenge throughout development was ensuring a robust detection and decoding of the students' cards. Detecting and decoding a large number of cards in the uncontrolled environment of a classroom, while targeting low-cost computational device proved technically very challenging. Although TopCodes are very robust to distortions and noise, we had to create several additional adaptations to transpose them from their original application context (augmented reality) to ours (CRSs).

On the usability tests, the recording of the users interaction with the app --- including their ``think-aloud'' comments and  recommendations --- was the strategy that provided the most actionable information. The unstructured interviews were also interesting, but, to our surprise we found the the structured, formal survey the least useful of the instruments --- it only provided enough information to reinforce trends we had already understood --- with more confidence --- in the recordings and interviews. We believe that a survey has to be exceptionally well-designed to provide actionable information, while interviews and recording can be useful even for developers without a huge background in Human-Computer Interaction. In future projects, we will attempt to apply heuristic evaluation~\cite{nielsen1990heuristic} experiments with real users; we believe that cost-effective technique would have anticipated some of the problems found in our user trials.

Relying on storyboards for design and documentation worked very well for a small team,  designing a small-sized (less than 10-screen workflow) user-interaction driven application. Our team comprised 5 people, partially changing throughout the project --- a scenario not uncommon on academic research. We employed storyboards to elicit and document the requirements, to sketch the interaction elements, to design the navigation and dynamics of the application, etc. We also used them to image usage scenarios, which were also a crucial to design the usability tests.

