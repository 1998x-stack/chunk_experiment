%%%%%%%%%%%%%%%%%%
% PACKAGES
%%%%%%%%%%%%%%%%%%

\usepackage{amsthm,amssymb,amsfonts,amsmath}
\usepackage{amscd} %Commutative diagrams. 
\usepackage{mathrsfs}
\usepackage[numbers, sort&compress]{natbib}
\usepackage{subfigure, graphicx}
\usepackage{vmargin}
\usepackage{setspace}
\usepackage{paralist}
\usepackage{hyperref}
\usepackage{color}
\usepackage[normalem]{ulem}
\usepackage{stmaryrd} %for \bbr
%\usepackage[notcite,notref]{showkeys} % shows labels 
\usepackage[refpage,noprefix,intoc]{nomencl} %for list of notation.

%%%%%%%%%%%%%%%%% 
% MATH COMMANDS
%%%%%%%%%%%%%%%%% 

% Probability
\newcommand{\E}[1]{{\mathbf E}\left[#1\right]}
\newcommand{\e}{{\mathbf E}}
\newcommand{\V}[1]{{\mathbf{Var}}\left\{#1\right\}}
\newcommand{\va}{{\mathbf{Var}}}
\newcommand{\p}[1]{{\mathbf P}\left\{#1\right\}}
\newcommand{\psub}[2]{{\mathbf P}_{#1}\left\{#2\right\}}
\newcommand{\psup}[2]{{\mathbf P}^{#1}\left\{#2\right\}}
\newcommand{\Esup}[2]{{\mathbf E^{#1}}\left\{#2\right\}}
\newcommand{\esup}[1]{{\mathbf E^{#1}}}
\newcommand{\Esub}[2]{{\mathbf E_{#1}}\left\{#2\right\}}
\newcommand{\esub}[1]{{\mathbf E_{#1}}}
\newcommand{\I}[1]{{\mathbf 1}_{[#1]}}
\newcommand{\Cprob}[2]{\mathbf{P}\set{\left. #1 \; \right| \; #2}}
\newcommand{\probC}[2]{\mathbf{P}\set{#1 \; \left|  \; #2 \right. }}
\newcommand{\Cexp}[2]{\mathbf{E}\set{\left. #1 \; \right| \; #2}}
\newcommand{\expC}[2]{\mathbf{E}\set{#1 \; \left|  \; #2 \right. }}
\newcommand{\eqdist}{\ensuremath{\stackrel{\mathrm{d}}{=}}}
\newcommand{\convdist}{\ensuremath{\stackrel{\mathrm{d}}{\rightarrow}}}
\newcommand{\convas}{\ensuremath{\stackrel{\mathrm{a.s.}}{\rightarrow}}}
\newcommand{\aseq}{\ensuremath{\stackrel{\mathrm{a.s.}}{=}}}

% Theorem types, theorem references

\newtheorem{thm}{Theorem}
\newtheorem{lem}[thm]{Lemma}
\newtheorem{prop}[thm]{Proposition}
\newtheorem{cor}[thm]{Corollary}
\newtheorem{dfn}[thm]{Definition}
\newtheorem{definition}[thm]{Definition}
\newtheorem{conj}{Conjecture}
\newtheorem{ex}{Exercise}[section]
\newtheorem{fact}[thm]{Fact}
\newtheorem{claim}[thm]{Claim}
\newtheorem{cla}[thm]{Claim}
\newcommand{\refT}[1]{Theorem~\ref{#1}}
\newcommand{\refC}[1]{Corollary~\ref{#1}}
\newcommand{\refL}[1]{Lemma~\ref{#1}}
\newcommand{\refR}[1]{Remark~\ref{#1}}
\newcommand{\refS}[1]{Section~\ref{#1}}
\newcommand{\refP}[1]{Proposition~\ref{#1}}
\newcommand{\refE}[1]{Example~\ref{#1}}
\newcommand{\refF}[1]{Figure~\ref{#1}}
\newcommand{\refand}[2]{\ref{#1} and~\ref{#2}}

\numberwithin{equation}{section} %Numbers equations by section
\numberwithin{thm}{section}

%Natural Numbers, Integers etc.
\providecommand{\R}{}
\providecommand{\Z}{}
\providecommand{\N}{}
\providecommand{\C}{}
\providecommand{\Q}{}
\providecommand{\G}{}
\providecommand{\Lt}{}
\renewcommand{\R}{\mathbb{R}}
\renewcommand{\Z}{\mathbb{Z}}
\renewcommand{\N}{{\mathbb N}}
\renewcommand{\C}{\mathbb{C}}
\renewcommand{\Q}{\mathbb{Q}}
\renewcommand{\G}{\mathbb{G}}
\renewcommand{\Lt}{\mathbb{L}}

% Other
\newcommand{\ubar}[1]{\underline{#1}}
\providecommand{\obar}{}
\renewcommand{\obar}[1]{\overline{#1}}
\newcommand{\pran}[1]{\left(#1\right)}
\newcommand{\pcur}[1]{\left\{#1\right\}}
\providecommand{\eps}{}
\renewcommand{\eps}{\epsilon}
\providecommand{\ora}[1]{}
\renewcommand{\ora}[1]{\overrightarrow{#1}}

%%%%%%%% 
%FOR MARGINAL COMMENTS, COLOR TEXT 
%%%%%%%%%%%

%\reversemarginpar
%\marginparwidth 1.1in
%
%\newcommand\marginal[1]{\marginpar{\raggedright\parindent=0pt\tiny #1}}
%\newcommand\LAB{\marginal{LAB}}
%\newcommand\REM[1]{{\raggedright\texttt{[#1]}\par\marginal{XXX}}}
%\newcommand\rem[1]{{\raggedright\texttt{[#1]}\marginal{XXX}}}
%\definecolor{clou}{rgb}{0.8,0.25,0.5125}
%\newcommand{\lou}[1]{\textcolor{clou}{#1}}
%\newcommand{\lmar}[1]{\textcolor{clou}{\marginal{#1 \\ LAB}}}
%

%%%%%%%%%%%%%%%%
% HYPERREF SETUP
%%%%%%%%%%%%%%%%
\hypersetup{
    bookmarks=true,         % show bookmarks bar?
    unicode=false,          % non-Latin characters in Acrobat’s bookmarks
    pdftoolbar=true,        % show Acrobat’s toolbar?
    pdfmenubar=true,        % show Acrobat’s menu?
    pdffitwindow=true,      % page fit to window when opened
    pdftitle={My title},    % title
    pdfauthor={Author},     % author
    pdfsubject={Subject},   % subject of the document
    pdfnewwindow=true,      % links in new window
    pdfkeywords={keywords}, % list of keywords
    colorlinks=true,       % false: boxed links; true: colored links
    linkcolor=blue,          % color of internal links
    citecolor=blue,        % color of links to bibliography
    filecolor=blue,      % color of file links
    urlcolor=blue           % color of external links
}

%%%%%%%%%%%%%%%%
% FOR TITLE 
%%%%%%%%%%%%%%%%
\newcommand\urladdrx[1]{{\urladdr{\def~{{\tiny$\sim$}}#1}}}
% at the end of this \oclock tells the time. 
\begingroup
  \count255=\time
  \divide\count255 by 60
  \count1=\count255
  \multiply\count255 by -60
  \advance\count255 by \time
  \ifnum \count255 < 10 \xdef\oclock{\the\count1:0\the\count255}
  \else\xdef\oclock{\the\count1:\the\count255}\fi
\endgroup

%%%%%%%%%%%%%%%%%%%%%%%%%%%
%TO BUILD A LIST OF NOTATION
%%%%%%%%%%%%%%%%%%%%%%%

\renewcommand{\nomname}{List of notation and terminology}
\renewcommand*{\pagedeclaration}[1]{\unskip\dotfill\hyperpage{#1}} 
\makenomenclature							\setlength{\nomitemsep}{-\parsep}				

%%%%%%%%%%%%%%%%%%%%%%%%%%%
% SINGLE CHARACTERS IN FONTS
%%%%%%%%%%%%%%%%%%%%%%%%%%%

%Sets
\newcommand\cA{\mathcal A}
\newcommand\cB{\mathcal B}
\newcommand\cC{\mathcal C}
\newcommand\cD{\mathcal D}
\newcommand\cE{\mathcal E}
\newcommand\cF{\mathcal F}
\newcommand\cG{\mathcal G}
\newcommand\cH{\mathcal H}
\newcommand\cI{\mathcal I}
\newcommand\cJ{\mathcal J}
\newcommand\cK{\mathcal K}
\newcommand\cL{{\mathcal L}}
\newcommand\cM{\mathcal M}
\newcommand\cN{\mathcal N}
\newcommand\cO{\mathcal O}
\newcommand\cP{\mathcal P}
\newcommand\cQ{\mathcal Q}
\newcommand\cR{{\mathcal R}}
\newcommand\cS{{\mathcal S}}
\newcommand\cT{{\mathcal T}}
\newcommand\cU{{\mathcal U}}
\newcommand\cV{\mathcal V}
\newcommand\cW{\mathcal W}
\newcommand\cX{{\mathcal X}}
\newcommand\cY{{\mathcal Y}}
\newcommand\cZ{{\mathcal Z}}

%Roman characters
\newcommand{\rA}{\mathrm{A}} 
\newcommand{\rB}{\mathrm{B}} 
\newcommand{\rC}{\mathrm{C}} 
\newcommand{\rD}{\mathrm{D}} 
\newcommand{\rE}{\mathrm{E}} 
\newcommand{\rF}{\mathrm{F}} 
\newcommand{\rG}{\mathrm{G}} 
\newcommand{\rH}{\mathrm{H}} 
\newcommand{\rI}{\mathrm{I}} 
\newcommand{\rJ}{\mathrm{J}} 
\newcommand{\rK}{\mathrm{K}} 
\newcommand{\rL}{\mathrm{L}} 
\newcommand{\rM}{\mathrm{M}} 
\newcommand{\rN}{\mathrm{N}} 
\newcommand{\rO}{\mathrm{O}} 
\newcommand{\rP}{\mathrm{P}} 
\newcommand{\rQ}{\mathrm{Q}} 
\newcommand{\rR}{\mathrm{R}} 
\newcommand{\rS}{\mathrm{S}} 
\newcommand{\rT}{\mathrm{T}} 
\newcommand{\rU}{\mathrm{U}} 
\newcommand{\rV}{\mathrm{V}} 
\newcommand{\rW}{\mathrm{W}} 
\newcommand{\rX}{\mathrm{X}} 
\newcommand{\rY}{\mathrm{Y}} 
\newcommand{\rZ}{\mathrm{Z}} 

%Boldface characters
\newcommand{\bA}{\mathbf{A}} 
\newcommand{\bB}{\mathbf{B}} 
\newcommand{\bC}{\mathbf{C}} 
\newcommand{\bD}{\mathbf{D}} 
\newcommand{\bE}{\mathbf{E}} 
\newcommand{\bF}{\mathbf{F}} 
\newcommand{\bG}{\mathbf{G}} 
\newcommand{\bH}{\mathbf{H}} 
\newcommand{\bI}{\mathbf{I}} 
\newcommand{\bJ}{\mathbf{J}} 
\newcommand{\bK}{\mathbf{K}} 
\newcommand{\bL}{\mathbf{L}} 
\newcommand{\bM}{\mathbf{M}} 
\newcommand{\bN}{\mathbf{N}} 
\newcommand{\bO}{\mathbf{O}} 
\newcommand{\bP}{\mathbf{P}} 
\newcommand{\bQ}{\mathbf{Q}} 
\newcommand{\bR}{\mathbf{R}} 
\newcommand{\bS}{\mathbf{S}} 
\newcommand{\bT}{\mathbf{T}} 
\newcommand{\bU}{\mathbf{U}} 
\newcommand{\bV}{\mathbf{V}} 
\newcommand{\bW}{\mathbf{W}} 
\newcommand{\bX}{\mathbf{X}} 
\newcommand{\bY}{\mathbf{Y}} 
\newcommand{\bZ}{\mathbf{Z}} 

%For table of contents when using AMS styles. Change 5 to 4 if not using hyperref. 
\DeclareRobustCommand{\SkipTocEntry}[5]{}
