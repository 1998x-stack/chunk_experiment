\section{Constituent Quark Model Expectations}
%
Theoretical speculations on the nature of the Roper resonance followed immediately upon its discovery.  For instance, it was emphasized that the enhancement observed in experiment need not necessarily be identified with a resonant state \cite{DALITZ1965159}; but if it is a resonance, then it has structural similarities with the ground-state nucleon \cite{PhysRevLett.16.772}.
%%
%% Supermultiplet SU(6)xO(3) = spin x flavour x space
%% HO potential does not distinguish between states of different flavour, J
%% It just identifies different HO excitations: nu= 0, 1, 2, ..., where nu= number of HO quanta excited.
%% nu=0 = "vacuum" = ground-state = (56, L^P=0^+)
%% nu=1 = (70,1^-)
%% nu=2 = (56,2^+), (56,0^+), (70,2^+), (70,0^+), (20,1^+)
%% spin-spin and spin-orbit interactions will split these levels
%%

The Roper was found during a dramatic period in the development of hadron physics, which saw the
%%The next fifteen years saw
the appearance of ``color'' as a quantum number carried by ``constituent quarks'' \cite{Greenberg:1964pe}, the interpretation of baryons as bound states of three such constituents \cite{GellMann:1964nj, Zweig:1981pd}, and the development of nonrelativistic quantum mechanical models with two-body potentials between constituent quarks that were tuned to describe the baryon spectrum as it was then known \cite{Hey:1982aj}.  Owing to their mathematical properties, harmonic oscillator potentials were favored as the zeroth-order term in the associated Hamiltonian:
\begin{equation}
\label{HOHamiltonian}
H_0=T + U_0\,, \; T = \sum_{i=1}^3 \frac{p_i^2}{2 M_i} \,, \;
U_0 = \sum_{i<j=1}^3 \tfrac{1}{2} K r_{ij}^2,
\end{equation}
where $p_i$ are the constituent-quark momenta, $r_{ij}$ are the associated two-body separations, and spin-dependent interactions were treated as [perturbative] corrections.  The indices in Eq.\,\eqref{HOHamiltonian} sample the baryon's constituent-quark flavors so that, \emph{e.g}.\ in the proton, $\{1,2,3\}\equiv\{U={\rm Up},U={\rm Up},D={\rm Down}\}$, and $K$ is a common ``spring constant'' for all the constituents.  If one assumes that all three constituent-quarks have the same mass, \emph{viz}.\ $M_1=M_2=M_3$, then this Hamiltonian produces the level ordering in Fig.\,\ref{HOlevels}.  [A similar ordering of these low-lying levels is also obtained with linear two-body potentials \cite{Richard:1992uk}.]

\begin{figure}[t]
\centerline{\includegraphics[width=0.4\textwidth]{F1.pdf}}
\caption{\label{HOlevels}
Blue lines: level ordering produced by the Hamiltonian in Eq.\,\eqref{HOHamiltonian}.  The $(56^\prime,0^+)$ level represents a supermultiplet that is completed by the states in the following representations of $SU(3)\times O(3)$: $(56,2^+)$, $(20,1^+)$, $(70,2^+)$, $(70,0^+)$.
%
Green dashed lines and shaded bands: pole-mass and width of the nucleon's two lowest-lying $J=1/2$ excitations, determined in a wide ranging analysis of available data \cite{Kamano:2013iva}.
%
For the purposes of this illustration, $\hbar \omega$ is chosen so that the proton-$N(1535)\,1/2^-$ splitting associates the $N(1535)\,1/2^-$ state with the $(70,1^-)$ supermultiplet, as suggested in quantum mechanics by its spin and parity.
}
\end{figure}

It is evident in Fig.\,\ref{HOlevels} that the natural level-ordering obtained with such potential models has the first negative-parity $\Delta L=1$ angular momentum excitation of the ground state three-quark system -- the $N(1535)\,1/2^-$ -- at a lower energy than its first radial excitation.  If the Roper resonance, $N(1440)1/2^+$, is identified with that radial excitation, whose quantum numbers it shares, then there is immediately a serious conflict between experiment and theory. However, this ignores the ``perturbations'', \emph{i.e}.\ corrections to $H_0$, which might describe spin-spin, spin-orbit, and other kindred interactions, that can eliminate the degeneracies in $n\geq 2$ harmonic oscillator supermultiplets.  [There are no such degeneracies in the $n=0,1$ supermultiplets.]  In this connection it was proved \cite{Gromes:1976cr, Isgur:1978wd} that given any anharmonic perturbation of the form $\sum_{i<j}U(r_{ij})$, then at first-order in perturbation theory the $n=2$ supermultiplet is always split as depicted in Fig.\,\ref{n2splitting}, where $\Delta$ is a measure of the shape of the potential.  In practice, there is always a value of $\Delta$ for which the $(56^\prime,0^+)$ [Roper] state is shifted below the $N(1535)\,1/2^-$.  Typically, however, the value is so large that one must question the validity of first-order perturbation theory \cite{Isgur:1978wd}.
%In first-order perturbation theory, any potential $U=\sum_{i<j}U(r_{ij})$ will split the $N=2$ harmonic oscillator energies into exactly the same pattern with a shift parameter $\Delta \approx 400\,$MeV obtained in a fit.

\begin{figure}[t]
\centerline{\includegraphics[width=0.47\textwidth]{F2.pdf}}
\caption{\label{n2splitting}
If an arbitrary anharmonic potential, restricted only insofar as it can be written as the sum of two-body potentials, is added to $H_0$ in Eq.\,\eqref{HOHamiltonian}, then at first order in perturbation theory the $n=2$ harmonic oscillator supermultiplet is split as indicated here.  [$E_0$ is roughly the original $(56^\prime,0^+)$ energy and $\Delta$ is a measure of the shape of the potential].
}
\end{figure}

%In order to highlight the enormous progress that has been made in baryon spectroscopy throughout the last decade, it is worth quoting a particle physics perspective from an earlier era \cite{Hey:1982aj}: ``\emph{In the previous sections of this review, we have described in some detail both the state of the data and of the phenomenological quark potential models used to describe the baryon spectrum.  Although there may still be some weakly coupled resonances lurking around the noise level or background of partial-wave analyses, it seems clear that the major features of the spectrum are known. It is not at all clear that we will ever have much more than, at best, a rudimentary outline of charmed or bottom baryon spectroscopy and it is probable that we have now identified over 90\% of the resonant states that we shall ever disentangle from the experimental data.  Indeed, given present experimental trends, it seems probable also that little more, if any, new experimental data relevant to the baryon spectrum will be forthcoming.}''

Notwithstanding such difficulties, at this time it was not uncommon for practitioners to imagine that such models were providing a realistic picture of the baryon spectrum and, in fact, they were a ``\emph{phenomenal phenomenological success}'' \cite{Hey:1982aj}.  Indeed, there was a perception \cite{Hey:1982aj} that: ``\emph{Although there may still be some weakly coupled resonances lurking around the noise level or background of partial-wave analyses, it seems clear that the major features of the spectrum are known. It is not at all clear that we will ever have much more than, at best, a rudimentary outline of charmed or bottom baryon spectroscopy and it is probable that we have now identified over 90\% of the resonant states that we shall ever disentangle from the experimental data.  Indeed, given present experimental trends, it seems probable also that little more, if any, new experimental data relevant to the baryon spectrum will be forthcoming}.''   Such conclusions were premature, as made clear by Sec.\,\ref{Experiment} herein and also the vast array of novel experimental results from the Belle, BaBar, BESIII and LHCb collaborations \cite{BraatenFB21, ChengpingFB21, Aaij:2015tga}, which reveal states that cannot be explained by quark models.

This period of enthusiasm coincided with the ``discovery'' of QCD, \emph{i.e}.\ the high-energy physics community were convinced that a fundamental relativistic quantum field theory of the strong interaction had been found \cite{Marciano:1977su, Marciano:1979wa}.  Some of its peculiar features had been exposed on the perturbative domain \cite{Politzer:2005kc, Wilczek:2005az, Gross:2005kv}, but the spectrum of bound-states it supported could not then be determined.\footnote{
It may still be said today that the complete spectrum of bound states supported by real QCD, \emph{i.e}.\ in the presence of dynamical quarks with realistic values for their current masses, is unknown.
}
In the absence of approaches with a direct QCD connection, studies of quantum mechanical constituent quark models [CQMs] continued; and in relation with the Roper resonance it was found that within a broad class of plausible phenomenological potentials, the negative-parity orbital excitation of the three-quark ground-state is always lower in mass than the $L=0$ radial excitation \cite{Hogaasen:1982rb, Richard:1992uk}.  This means, \emph{e.g}.\  that the ordering in Fig.\,\ref{n2splitting} is an artifact of first-order perturbation theory, which is unreliable when the leading correction is comparable to the value of $\hbar \omega$ associated with $H_0$; and, moreover, that the ordering of the nucleon's low-lying excitations is incorrect in a wide array of such constituent-quark models \cite{Capstick:2000qj, Crede:2013sze, Giannini:2015zia}.

The difficulty in providing a sound theoretical explanation of the Roper resonance was now becoming apparent.  In fact, at this point it was considered plausible that the $N(1440)\,1/2^+$ might not actually be a state generated by three valence quarks.  Instead, the notion was entertained that it may be a hybrid, \emph{viz}.\ a system with a material valence-gluon component or, at least, that the Roper might contain a substantial hybrid component \cite{Barnes:1982fj, Li:1991yba, Capstick:2002wm}. %[However, as the discussion of Fig.\,\ref{A12_lowQ} explains, this possibility is eliminated by modern electroproduction data.]

The appearance of QCD refocused attention on some prominent weaknesses in the formulation of CQMs.  In particular, their treatment of constituent-quark motion within a hadron as nonrelativistic, when calculations showed $\langle p_i \rangle \sim M_i$, where $\langle p_i\rangle$ is the mean-momentum of a bound constituent-quark; and the use of nonrelativistic dynamics, \emph{e.g}.\ the omission of calculable relativistic corrections to the various potential terms, which would normally become energy-dependent.  Consequently, a relativized constituent-quark model was developed \cite{Godfrey:1985xj} and applied to the baryon spectrum \cite{Capstick:1986bm}; but these improvements did not change the ordering of the energy levels, \emph{i.e}.\ the low-lying excitations of the nucleon were still ordered as depicted in Fig.\,\ref{HOlevels}.
%
This remains true even within a relativistic field theory framework that employs instantaneous interquark interactions to compute the baryon spectrum \cite{Loring:2001kx}; namely, a three-body term expressing linear confinement of constituent-quarks and a spin-flavor dependent two-body interaction to describe spin-dependent mass splittings.
%%-reviewed in https://www.epj-conferences.org/articles/epjconf/pdf/2017/03/epjconf_sfbtr2017_02005.pdf

The QCD-inspired CQMs described above all assume that interquark dynamics derives primarily from gluon-related effects.  An alternative is to suppose that the hyperfine interaction between constituent-quarks is produced by exchange of the lightest pseudoscalar mesons \cite{Glozman:1995fu}, \emph{i.e}.\ the pseudo--Nambu-Goldstone modes: $\pi$-, $K$- and $\eta$-mesons, in which case the hyperfine interaction is flavor-dependent, in contrast to that inferred from one-gluon exchange.  Using straightforward algebraic arguments, one may demonstrate that this sort of Goldstone-boson-exchange (GBE) hyperfine interaction produces more attraction in systems whose wave functions possess higher spin-flavor symmetry.  Such dynamics can thus lead to an inversion of the excited state levels depicted in Fig.\,\ref{HOlevels}, so that the Roper resonance, viewed as the lowest radial excitation of a three constituent-quark ground state, lies below the $N(1535)\,1/2^-$, which is the first orbital excitation of that system \cite{Yang:2017}.  This inversion of levels is a positive feature of the model; and it hints that meson-like correlations should play a role in positioning states in the baryon spectrum.  [Similar conclusions may be drawn from analyses of unquenched CQMs \cite{JuliaDiaz:2006av}.]

On the other hand, a GBE picture of baryon structure can only be figurative, at best.  All mesons are composite systems, with radii that are similar in magnitude to those of baryons; and hence one-boson exchange between constituent-quarks cannot be understood literally.  Moreover, serious difficulties of interpretation arise immediately if one attempts to compute the meson spectrum using a similar Hamiltonian, \emph{e.g}.\ how are the pointlike bosons exchanged between constituent quarks to be understood in the context of the nonpointlike mesonic bound states they help produce?

A deeper class of questions is relevant to all such CQMs.  Namely, in the era of QCD: can any connection be drawn between that underlying theory and the concept of a constituent quark; can the interactions between the lightest quarks in nature veraciously be described by a potential, of any kind; and notwithstanding the challenges they face in describing the Roper resonance, do their apparent successes in other areas yield any sound insights into strong interaction phenomena?  At present, each practitioner has their own answers to these questions.  Our view will subsequently emerge, but is readily stated: used judiciously, CQMs continue to be valuable part of the sQCD toolkit.

