%%%%%%%%%%%%%%%%%%%%%%%%%%%%%%%%%%%%%%%%%%%%%%%%%%%%%%
% document class
%%%%%%%%%%%%%%%%%%%%%%%%%%%%%%%%%%%%%%%%%%%%%%%%%%%%%%

\documentclass[showpacs,aps,prd,preprint,nofootinbib,showkeys,unsortedaddress,raggedbottom]{revtex4-1}
\pdfoutput=1

%%%%%%%%%%%%%%%%%%%%%%%%%%%%%%%%%%%%%%%%%%%%%%%%%%%%%%
% packages
%%%%%%%%%%%%%%%%%%%%%%%%%%%%%%%%%%%%%%%%%%%%%%%%%%%%%%

\usepackage{lipsum}

\usepackage{bm}
\usepackage{amsmath}
\usepackage{amssymb}
\usepackage{graphicx}
\usepackage{subfigure}
\usepackage[usenames,dvipsnames]{color}
\definecolor{darkblue}{RGB}{0,0,196}
\usepackage[colorlinks=true,linkcolor=darkblue,citecolor=darkblue,urlcolor=darkblue]{hyperref}
\usepackage{diagbox}

\usepackage{setspace}
\usepackage{footmisc}
\usepackage[makeroom]{cancel}
  
%%%%%%%%%%%%%%%%%%%%%%%%%%%%%%%%%%%%%%%%%%%%%%%%%%%%%%
% custom commands
%%%%%%%%%%%%%%%%%%%%%%%%%%%%%%%%%%%%%%%%%%%%%%%%%%%%%%

\renewcommand\footnotelayout{\fontsize{9}{12}\selectfont}
\newcommand{\intdP}{\int\!dP}

\DeclareMathOperator{\sech}{sech}
\DeclareMathOperator{\arcsinh}{arcsinh}
\DeclareMathOperator{\arctanh}{arctanh}
\DeclareMathOperator{\arccoth}{arccoth}
\newcommand{\redflag}[1]{{\color{red} #1}}

\def\be{\begin{equation}}
\def\ee{\end{equation}}
\def\ba{\begin{eqnarray}}
\def\ea{\end{eqnarray}}

\newcommand{\vE}{\vec E}
\newcommand{\rh}{\hat r}
\renewcommand{\vr}{\vec r}
\newcommand{\vp}{\vec p}
\newcommand{\eps}{{\varepsilon}}
\newcommand{\rmii}[1]{{\mbox{\tiny\rm{#1}}}}
\newcommand{\mD}{m_\rmii{D}}
\renewcommand{\Re}{\rm Re}
\renewcommand{\Im}{\rm Im}

%%%%%%%%%%%%%%%%%%%%%%%%%%%%%%%%%%%%%%%%%%%%%%%%%%%%%%
% begin document
%%%%%%%%%%%%%%%%%%%%%%%%%%%%%%%%%%%%%%%%%%%%%%%%%%%%%%

\begin{document}

\title{Bottomonium suppression using a lattice QCD vetted potential}


\author{Brandon Krouppa} 
\affiliation{Department of Physics, Kent State University, Kent, OH 44242 United States}

\author{Alexander Rothkopf} 
\affiliation{Institute for Theoretical Physics, Heidelberg University, Philosophenweg 12, 69120 Heidelberg, Germany}

\author{Michael Strickland} 
\affiliation{Department of Physics, Kent State University, Kent, OH 44242 United States}

\begin{abstract}
We estimate bottomonium yields in relativistic heavy-ion collisions using a lattice QCD vetted, complex-valued, heavy-quark potential embedded in a realistic, hydrodynamically evolving medium background. We find that the lattice-vetted functional form and temperature dependence of the proper heavy-quark potential dramatically reduces the dependence of the yields on parameters other than the temperature evolution, strengthening the picture of bottomonium as QGP thermometer. Our results also show improved agreement between computed yields and experimental data produced in RHIC 200 GeV/nucleon collisions. For LHC 2.76 TeV/nucleon collisions, the excited states, whose suppression has been used as a vital sign for quark-gluon-plasma production in a heavy-ion collision, are reproduced better than previous perturbatively-motivated potential models; however, at the highest LHC energies our estimates for bottomonium suppression begin to underestimate the data.  Possible paths to remedy this situation are discussed.
\end{abstract}

\date{\today}

\pacs{12.38.Mh, 24.10.Nz, 25.75.Ld, 47.75.+f}

\keywords{Quark-gluon plasma, Relativistic heavy-ion collisions, Quarkonia suppression}

\maketitle

%%%%%%%%%%%%%%%%%%%%%%%%%%%%%%%%%%%%%%%%%%%%%%%%%%%%%%

\input{bottom_introduction.tex}

\input{bottom_potentialmodel.tex}

\input{bottom_dynamics.tex}

\input{bottom_results.tex}

\input{bottom_conclusions.tex}

%%%%%%%%%%%%%%%%%%%%%%%%%%%%%%%%%%%%%%%%%%%%%%%%%%%%%%


%%%%%%%%%%%%%%%%%%%%%%%%%%%%%%%%%%%%%%%%%%%%%%%%%%%%%%
\bibliography{bottom}
%%%%%%%%%%%%%%%%%%%%%%%%%%%%%%%%%%%%%%%%%%%%%%%%%%%%%%


\end{document}

