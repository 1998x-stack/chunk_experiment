\def\names{{04/10/2014},{06/02/2014},{09/19/2014},{11/17/2014},{04/29/2015},{10/13/2015}}

\begin{figure*}
\centering
\foreach \name[count=\j] in \names {
	\def\idx{\the\numexpr\j+3}    % ???? I want to make m=k+4
	\ifnum\j=5
		\begin{subfigure}[t]{0.15\textwidth}
			\begin{tikzpicture}
    \begin{scope}[
    node distance = 2.5mm,
        inner sep = 0pt,spy using outlines={rectangle, red, magnification=4, %connect spies
        }
                 ]
\node (img)  {\includegraphics[keepaspectratio,width=\textwidth]{\idx}};
\coordinate (F) at (0.1, 0.5);
\coordinate (G) at (-1.5, 1.5);
\spy [size=5mm] on (F) in node[below right=of G.north east];
\end{scope}
\draw[dashed,red] (tikzspyonnode.north east) -- (tikzspyinnode.north east);
\draw[dashed,red] (tikzspyonnode.south west) -- (tikzspyinnode.south west);
    \end{tikzpicture}
    		\caption{\name}
    		\label{fig:cube\j}
    	\end{subfigure}	
	\else
		\begin{subfigure}[t]{0.15\textwidth}
			\includegraphics[keepaspectratio,width=\textwidth]{\idx}
			\caption{\name}
			\label{fig:cube\j}
		\end{subfigure}	
	\fi
}
\caption{Mud lake dataset used in the MTHS experiment with the corresponding acquisition dates. The area delineated in red in Fig.~\ref{fig:cube5} highlights a region known to contain outliers (this observation results from a previous analysis led on this dataset in \cite{Thouvenin2015b}).}
\label{fig:cube}
\end{figure*} 
