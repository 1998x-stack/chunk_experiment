%************************************************;
%                                                ;
%  Name                                          ;
%    Driver_LS_double_porosity.tex               ;
%                                                ;
%  Written By                                    ;
%    Nischal Karthik Mapakshi                    ;
%    Justin Chang                                ;
%    Kalyana Babu Nakshatrala                    ;
%                                                ;
%************************************************;
\documentclass[11pt,reqno]{amsproc}
\linespread{1.1}
%
\allowdisplaybreaks
\usepackage{fullpage}
\usepackage[semicolon,square,authoryear]{natbib}
\numberwithin{equation}{section}
\usepackage{cite}
\usepackage{enumerate}
\usepackage{caption}
%
\usepackage{color}
\usepackage[section]{placeins}
\usepackage{graphicx}
\graphicspath{{Figures/}}
\usepackage{psfrag,epsfig}
\usepackage{epstopdf}
\usepackage{subfigure}
\usepackage{amsmath}
%Used in SPM to show larger text in \frac
\newcommand\ddfrac[2]{\frac{\displaystyle #1}{\displaystyle #2}}

\usepackage[debug=false, colorlinks=true, pdfstartview=FitV, linkcolor=blue, citecolor=blue, urlcolor=blue]{hyperref}

%-----------------------------------------;
%  additional packages to extend storage  ;
%-----------------------------------------;
\usepackage{morefloats}


\usepackage{multirow}

%Wrap text in tables
\usepackage{booktabs}

%subfigures
\usepackage{subfigure}
%\usepackage{subcaption}
\usepackage{placeins}

\newlength{\drop}
\definecolor{amethyst}{rgb}{0.6, 0.4, 0.8}
\definecolor{burgundy}{rgb}{0.5, 0.0, 0.13}
%-------------------------------------;
%  additional packages for theorems   ;
%-------------------------------------;
\newtheorem{theorem}{Theorem}[section]
\newtheorem{lemma}[theorem]{Lemma}
\newtheorem{proposition}[theorem]{Proposition}
\newtheorem{definition}[theorem]{Definition}
\newtheorem{statement}[theorem]{Statement}
\newtheorem{remark}[theorem]{Remark}
\newtheorem{example}[theorem]{Example}
\newtheorem{corollary}[theorem]{Corollary}

%======================;
%  Title of the paper  ;
%======================;
\title{\textbf{A scalable variational inequality 
approach for flow through porous media 
models with pressure-dependent viscosity}}

%========================;
%  Authors of the paper  ;
%========================;
\author{\textbf{N.~K.~Mapakshi}, \textbf{J.~Chang} and \textbf{K.~B.~Nakshatrala} \\
{\small Department of Civil and Environmental Engineering, 
  University of Houston. \\
  \textbf{Correspondence to:}~\textsf{knakshatrala@uh.edu}}}

\date{\today}
\begin{document}

%===========================;
%  Title page of the paper  ;
%===========================;
\begin{titlepage}
  \drop=0.1\textheight
  \centering
  \vspace*{\baselineskip}
  \rule{\textwidth}{1.6pt}\vspace*{-\baselineskip}\vspace*{2pt}
  \rule{\textwidth}{0.4pt}\\[\baselineskip]
  {\LARGE \textbf{\color{burgundy}
  A scalable variational inequality approach for  \\[0.2\baselineskip] 
  flow through porous media models with \\[0.2\baselineskip]
  pressure-dependent viscosity}}\\[0.3\baselineskip]
    \rule{\textwidth}{0.4pt}\vspace*{-\baselineskip}\vspace{3.2pt}
    \rule{\textwidth}{1.6pt}\\[0.5\baselineskip]
    \scshape
    %%
    An e-print of the paper will be made available on arXiv. \par
    %%
    \vspace*{0.5\baselineskip}
    Authored by \\[0.3\baselineskip]
    %%  
    {\Large N.~K.~Mapakshi \par}
    {\itshape Graduate Student, University of Houston. \par}
    \vspace*{0.2\baselineskip}
    %
    {\Large J.~Chang \par}
    {\itshape Postdoctoral Researcher, Rice University. \par}
    \vspace*{0.2\baselineskip}
    %
    {\Large K.~B.~Nakshatrala\par}
    {\itshape Department of Civil \& Environmental Engineering \\
    University of Houston, Houston, Texas 77204--4003. \\ 
    \textbf{phone:} +1-713-743-4418, \textbf{e-mail:} knakshatrala@uh.edu \\
    \textbf{website:} http://www.cive.uh.edu/faculty/nakshatrala\par}    
    \vspace*{0.2\baselineskip}
    %%
   \begin{figure}[h]
   	\centering
   	\subfigure[RT0 formulation]{\includegraphics[scale=0.29]{Figures/3D_Orig.png}}
   	\subfigure[Proposed VI-based formulation]{\includegraphics[scale=0.29]{Figures/3D_VI.png}}
   	\captionsetup{format=hang}
	\vspace{-0.15in}
   	\caption*{\small{This picture shows the pressure profiles 
	of a 3D reservoir with a bore hole at the top surface. The 
	left figure depicts the pressure profile obtained using the 
	lowest-order Raviart-Thomas (RT0) formulation. The 
	missing chunks represent the regions in which the 
	discrete maximum principle (DMP) is violated. The 
	right figure shows the pressure profiles under the 
	proposed VI-based formulation, and there are no 
	violations of DMP.}}
   \end{figure}
    \vfill
    %% 
    %%       
    {\scshape 2017} \\
    {\small Computational \& Applied Mechanics Laboratory} \par
\end{titlepage}

%=========================;
%  Abstract and keywords  ;
%=========================;
\begin{abstract}
Mathematical models for flow through porous media typically 
enjoy the so-called maximum principles, which place bounds 
on the pressure field. 
%
It is highly desirable to preserve these bounds on the pressure field in predictive numerical simulations, that is, one needs to satisfy discrete maximum principles (DMP). Unfortunately, many of the existing formulations for flow through porous media models do \emph{not} satisfy DMP. 
%
  This paper presents a robust, scalable numerical formulation based on variational inequalities (VI), to model non-linear flows through heterogeneous, anisotropic porous media without violating DMP. VI is an optimization technique that places bounds on the numerical solutions of partial differential equations. To crystallize the ideas, a modification to Darcy equations by taking into account pressure-dependent viscosity will be discretized using the lowest-order Raviart-Thomas (RT0) and Variational Multi-scale (VMS) finite element formulations. It will be shown that these formulations violate DMP, and, in fact, these violations increase with an increase in anisotropy. It will be shown that the proposed VI-based formulation provides a viable route to enforce DMP. Moreover, it will be shown that the proposed formulation is scalable, and can work with any numerical discretization and weak form.
  %  
  A series of numerical benchmark problems are solved to demonstrate 
  the effects of heterogeneity, anisotropy and non-linearity on DMP 
  violations under the two chosen formulations (RT0 and VMS), and 
  that of non-linearity on solver convergence for the proposed 
  VI-based formulation.
  %
  Parallel scalability on modern computational platforms will be 
  illustrated through strong-scaling studies, which will prove 
  the efficiency of the proposed formulation in a parallel setting. 
  %
  Algorithmic scalability as the problem size is scaled
  up will be demonstrated through novel static-scaling 
  studies. The performed static-scaling studies can 
  serve as a guide for users to be able to select an 
  appropriate discretization for a given problem size.
  %
\end{abstract}
%
\keywords{Variational inequalities; pressure-dependent viscosity; 
anisotropy; maximum principles; flow though porous media; 
parallel computing}
%  
\maketitle

%==================================;
%  Include all the sections below  ;
%==================================;

\input{Sections/S1_NN_Intro}

\input{Sections/S2_NN_GE}

\input{Sections/S3_NN_Mixed}

\input{Sections/S4_NN_VI}

\input{Sections/S5_NN_NR}

\input{Sections/S6_NN_CR}

%================;
%  Bibliography  ;
%================;
\bibliographystyle{plainnat}
\bibliography{References}
%%
\end{document}
