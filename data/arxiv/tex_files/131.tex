\documentclass[a4paper]{article}

%\pdfoutput=1 %may be needed for Arxiv submissions: if some graphics are in .pdf format, include this so that the .tex file is built using PDFLaTeX instead of LaTeX

\usepackage{microtype}%if unwanted, comment out or use option "draft"
\usepackage[usenames]{xcolor}
\usepackage{bbm}
\usepackage{lineno}
\usepackage{hyperref}
\usepackage{amsthm}
\usepackage{amsmath}
\usepackage{graphicx}
\usepackage{amssymb}
\usepackage{url}
\usepackage{amsopn}
\usepackage{authblk}

%%%%%%%%%%%%%%%%%%%%%%%%%%%%%%%%%%%%%%%
%%% MACROS FOR COMMENTS
%%%%%%%%%%%%%%%%%%%%%%%%%%%%%%%%%%%%%%%
%\newcommand{\note}[1]{{\color{red} #1}}
%\newcommand{\note}[1]{{}}

\newcommand{\comment}[1]{{\color{blue} #1}}
%\newcommand{\changed}[1]{{\color{blue} #1}}
% to remove the blue colour, comment the above line and uncomment the following one
\newcommand{\changed}[1]{{#1}}
\newcommand{\anna}[1]{{\color{cyan} #1}}
\newcommand{\alt}[1]{{\color{cyan} #1}}
\newcommand{\uli}[1]{{\color{red}#1}}
\newcommand{\remove}[1]{{}}

%%%%%%%%%%%%%%%%%%%%%%%%%%%%%%%%%%%%%%%
%%% MORE MACROS
%%%%%%%%%%%%%%%%%%%%%%%%%%%%%%%%%%%%%%%

%%% Triangulation complex
\newcommand{\tcomplex}{\ensuremath{\mathbbm{T}}}

%%% Flip complex
\newcommand{\fcomplex}{\ensuremath{\mathbbm{X}}}

%%% Orden-Santos Polytope
\newcommand{\ospoly}{\ensuremath{\mathbbm{Y}}}
\DeclareMathOperator{\skel}{skel}

%
\theoremstyle{plain}
\newtheorem{theorem}{Theorem}
\newtheorem{lemma}[theorem]{Lemma}
\newtheorem{corollary}[theorem]{Corollary}
\newtheorem{fact}[theorem]{Fact}
\newtheorem{proposition}[theorem]{Proposition}
\theoremstyle{definition}
\newtheorem{definition}[theorem]{Definition}
\newtheorem{example}[theorem]{Example}
\theoremstyle{remark}
\newtheorem*{remark}{Remark}

\bibliographystyle{plainurl}

\title{A Proof of the Orbit Conjecture for Flipping Edge-Labelled Triangulations}
\date{}
%\titlerunning{A Proof of the Orbit Conjecture for Flipping Edge-Labelled Triangulations}

\author[1]{Anna Lubiw}
\author[2]{Zuzana Mas\'arov\'a}
\author[2]{Uli Wagner}
\affil[1]{School of Computer Science, University of Waterloo\\
 Waterloo, ON, Canada, N2L 3G1\\
  \texttt{alubiw@uwaterloo.ca}}
\affil[2]{IST Austria\\
  Am Campus 1, 3400 Klosterneuburg, Austria\\
  \texttt{zuzana.masarova@ist.ac.at, uli@ist.ac.at}}
  
%\authorrunning{A.\, Lubiw, Z.\, Mas\'arov\'a and U.\, Wagner} %mandatory. First: Use abbreviated first/middle names. Second (only in severe cases): Use first author plus 'et. al.'
%\Copyright{Anna Lubiw, Zuzana Mas\'arov\'a and Uli Wagner}%mandatory, please use full first names. LIPIcs license is "CC-BY";  http://creativecommons.org/licenses/by/3.0/
%\subjclass{F.2.2 Nonnumerical Algorithms and Problems}% mandatory: Please choose ACM 1998 classifications from http://www.acm.org/about/class/ccs98-html . E.g., cite as "F.1.1 Models of Computation". 
%\keywords{triangulations, reconfiguration, flip, constrained triangulations, Delaunay triangulation, shellability, piecewise linear balls}% mandatory: Please provide 1-5 keywords
  

\begin{document}

\maketitle

\begin{abstract}
Given a triangulation of a point set in the plane, a \emph{flip} deletes an edge $e$ whose removal leaves a convex quadrilateral, and replaces $e$ by the opposite diagonal of the quadrilateral. 
It is well known that any triangulation of a point set can be reconfigured to any other triangulation by some sequence of flips.
We explore this question in the setting where each edge of a triangulation has a label, and a flip transfers the label of the removed edge to the new edge.
%In the case of an edge-labelled triangulation, the new edge acquires the label of the removed edge $e$.
It is not true that every labelled triangulation of a point set can be reconfigured to every other labelled triangulation via a sequence of flips, but we characterize when this is possible.
%We characterize which permutations of the edge-labels of a triangulation are achievable by sequences of flips.
There is an obvious necessary condition: for each label $l$,  
if edge $e$ has label $l$ in the first triangulation and edge $f$ has label $l$ in the second triangulation,
%if the permutation takes label $\ell$ from edge $e$ to edge $f$, 
then there must be some sequence of flips that moves label $l$ from $e$ to $f$, ignoring all other labels.   
Bose, Lubiw, Pathak and Verdonschot 
formulated the \emph{Orbit Conjecture}, which states
%conjectured 
that this necessary condition is also sufficient, i.e.~that \emph{all} labels can be simultaneously mapped to their destination if and only if \emph{each} label individually can be mapped to its destination.
We prove this conjecture. %Orbit Conjecture.
Furthermore, we give a polynomial-time algorithm to find a sequence of flips 
to reconfigure one labelled triangulation to another, if such a sequence exists, and
we prove an upper bound of $O(n^7)$ % we've reduced the bound from 8 to 7
on the length of the flip sequence.  
%The best known lower bound is $O(n^3)$.  


Our proof uses the topological % ok to add "topological"?
result %fact 
that 
the sets of pairwise non-crossing edges on a planar point set form 
a simplicial complex that 
is homeomorphic to a high-dimensional ball 
(this follows from a result of Orden and Santos; we 
give a different proof based 
on a shelling argument). 
The dual cell complex of this simplicial ball, 
called the
%which we call the 
\emph{flip complex}, has the usual flip graph as its $1$-skeleton.  We use properties 
of the $2$-skeleton of the flip complex %this dual complex 
to prove the Orbit Conjecture.

%It follows that the flip graph of (unlabelled) triangulations can be enlarged to a \emph{simply-connected} $2$-dimensonal cell complex by adding quadrilateral and pentagonal $2$-cells that correspond to triangulations minus a pair of flippable edges whose removal creates either two internally disjoint convex quadrilaterals or a convex pentagon.

\end{abstract}


%%%%%%%%%%%%%%%%%%%%%%%%%%%%%%%%%%%%%%%%%%%%%%%%%
% introduction
%%%%%%%%%%%%%%%%%%%%%%%%%%%%%%%%%%%%%%%%%%%%%%%%%
\input introduction
 
%%%%%%%%%%%%%%%%%%%%%%%%%%%%%%%%%%%%%%%%%%%%%%%%%
% reductions 
%%%%%%%%%%%%%%%%%%%%%%%%%%%%%%%%%%%%%%%%%%%%%%%%%
\input reductions.tex

%%%%%%%%%%%%%%%%%%%%%%%%%%%%%%%%%%%%%%%%%%%%%%%%%
% topology 
%%%%%%%%%%%%%%%%%%%%%%%%%%%%%%%%%%%%%%%%%%%%%%%%%
%\input topology.tex
\input topology1.tex
\input topology2.tex
\input topology3.tex

%%%%%%%%%%%%%%%%%%%%%%%%%%%%%%%%%%%%%%%%%%%%%%%%%
% bounds 
%%%%%%%%%%%%%%%%%%%%%%%%%%%%%%%%%%%%%%%%%%%%%%%%%
\input bounds.tex
 
%%%%%%%%%%%%%%%%%%%%%%%%%%%%%%%%%%%%%%%%%%%%%%%%%
% conclusions 
%%%%%%%%%%%%%%%%%%%%%%%%%%%%%%%%%%%%%%%%%%%%%%%%%
\input conclusions.tex

%%
%% Acknowledgements
%%
\subparagraph*{Acknowledgements}
%\comment{
This research was initiated at the 2016 Bellairs Workshop on Geometry and Graphs. We thank anonymous reviewers and participants of the 2017 Symposium on Computational Geometry for helpful suggestions.
%} 

%%
%% Bibliography
%%
\bibliography{references}
\newpage


\end{document}