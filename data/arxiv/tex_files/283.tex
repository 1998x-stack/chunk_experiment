\documentclass[preprint,number,sort&compress,longtitle]{elsarticle}   % version for ArXiV
%\documentclass[review,number,sort&compress,longtitle]{elsarticle}
%\documentclass[preprint,number,sort&compress,longtitle]{elsarticle}   % version for ArXiV
%\documentclass[5p,sort&compress,longtitle]{elsarticle}
%\documentclass{elsarticle}
%\documentclass[review,number,sort&compress]{elsarticle}
\pdfoutput=1 % only if pdf/png/jpg images are used -- try this for ArXiV
%\documentclass[hyper]{JINST}
%\pdfoptionpdfminorversion=6
%\usepackage{morefloats}

\journal{Nuclear Instruments and Methods in Physics Research A}

%\usepackage[square,comma,sort&compress]{natbib}

\usepackage[switch]{lineno}

\usepackage[english]{babel}
%\usepackage{mathpazo}
\usepackage{times}
\usepackage{graphicx}
\usepackage{fontenc}
\usepackage{mathptmx}
%\usepackage{pstricks}
\usepackage{subfigure}
\usepackage{amsmath,amssymb}
\usepackage{upgreek}
\usepackage{xspace}
\usepackage{color}
\definecolor{darkred}{rgb}{0.5,0,0}
\definecolor{darkblue}{rgb}{0,0,0.5}
\definecolor{firebrick}{rgb}{0.75,0.125,0.125}
\definecolor{darkgreen}{rgb}{0,0.5,0}
%\usepackage[colorlinks=true,linkcolor=firebrick,citecolor=darkgreen,urlcolor=darkblue]{hyperref}  -- exclude from ArXiV version?
\usepackage[pdftex,colorlinks=true]{hyperref}
%\usepackage{hyperref}   -- exclude from ArXiV version (?)
\usepackage{units}
\usepackage{booktabs}
\usepackage{epstopdf}
\usepackage{amsmath}

\graphicspath{{plots/}}
\pdfoptionpdfminorversion=6

\renewcommand\floatpagefraction{1}
\makeatletter
\def\elsartstyle{%
   \def\normalsize{\@setfontsize\normalsize\@xiipt{14.5}}
   \def\small{\@setfontsize\small\@xipt{13.6}}
    \let\footnotesize=\small
    \def\large{\@setfontsize\large\@xivpt{18}}
    \def\Large{\@setfontsize\Large\@xviipt{22}}
    \skip\@mpfootins = 18\p@ \@plus 2\p@
    \normalsize
}
\@ifundefined{square}{}{\let\Box\square}
\makeatother

\setcounter{topnumber}{10}
\renewcommand{\topfraction}{1.}
\setcounter{bottomnumber}{0}
\renewcommand{\bottomfraction}{0}
\setcounter{totalnumber}{10}
\renewcommand{\textfraction}{0.}
\renewcommand{\floatpagefraction}{1.}

\def\file#1{\texttt{#1}}
\def\red#1{{\color{red}#1}}
\def\Offline{\mbox{$\overline{\textrm%
{Off}}$\hspace{.05em}\protect\raisebox{.4ex}%
{$\protect\underline{\textrm{line}}$}}\xspace}

% define blank footnote "\blfootnote" command
\makeatletter
\def\blfootnote{\xdef\@thefnmark{}\@footnotetext}
\makeatother


 
\begin{document}
%%\linenumbers

\begin{frontmatter}
%\modulolinenumbers[2]


\title{GIGAS: a  set of microwave sensor  arrays to detect  molecular bremsstrahlung  radiation from extensive air shower}
\input{authors}
\input{institutions}


\begin{abstract}
  \noindent

We present  the GIGAS (Gigahertz  Identification of Giant  Air Shower)
microwave  radio sensor arrays  of the  EASIER project  (Extensive Air
Shower Identification with Electron Radiometers), deployed at the site
of the  Pierre Auger cosmic ray  observatory.  The aim  of these novel
arrays  is to  probe  the intensity  of  the molecular  bremsstrahlung
radiation expected  from the development of the  extensive air showers
produced by  the interaction of ultra  high energy cosmic  rays in the
atmosphere.  In  the designed setup,  the sensors are  embedded within
the surface detector array of the Pierre Auger observatory allowing us
to  use the  particle signals  at ground  level to  trigger  the radio
system.  A series of seven, then  61 sensors have been deployed in the
C-band, followed by a new series  of 14 higher sensitivity ones in the
C-band and the L-band.  The design, the operation, the calibration and
the sensitivity to extensive air showers of these arrays are described
in this paper.
\end{abstract}
\begin{keyword}
high energy cosmic rays \sep microwave \sep radio
\sep molecular bremsstrahlung  
\end{keyword}
\end{frontmatter}
\setcounter{footnote}{0}
%%\newpage
%%\tableofcontents
%%\thispagestyle{empty}
%%\noindent
\include{introduction}
\include{setup}
\include{calibration}
\include{performance}
\include{conclusion}

\bibliographystyle{elsarticle-num}
\bibliography{easier}

\end{document}

