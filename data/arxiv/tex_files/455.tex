\IEEEPARstart{T}{he} problem of radar detection in Gaussian clutter
has been addressed in the
pioneering work \cite{Kelly86}. Therein, the
presence of a point-like target was sought in a single vector
of the form $b \boldsymbol{s}$, where
$b$ was an unknown complex scalar accounting
for the combined effect of  a target's reflectivity and channel propagation and
$\boldsymbol{s}$, representing the radar signature, was perfectly known.
It was reported that
the detector in \cite{Kelly86} suffers a
performance loss when
the actual radar signature
departs from its nominal one, for example cases of
an imperfect array calibration.

 To increase detection probability when a mismatch occurs,
 %instead of detecting solely $b \boldsymbol{s}$
 detecting $b \boldsymbol{s}$ where $\boldsymbol{s}$ is not completely known but lies
 in an assumed range has been proposed.
Such a range where $\boldsymbol{s}$ lies in could be possibly
modelled as a known linear subspace or a cone with axis the nominal radar signature.
%In the former approach,
Subspace detectors, based on the former approach, have been proposed in
 \cite{Scharf94}--\cite{Aubry_Jul14}.
 Coordinates of the signal to detect %within the subspace
 are unknown; detection is performed
    by computing energy
of the measurement in the signal subspace \cite{Scharf94}.
However, there is no guidance on choosing an appropriate subspace to which
a signal of interest belongs.
The latter approach circumvents this drawback by
assuming a nominal radar signature as axis of a cone to which
a signal of interest belongs to \cite{Ramprashad96}.
 % introducing
% a more flexible model: cone class model, firstly proposed in ,
% assuming that actual radar signature vector lies in a cone whose axis is the nominal one.
Cone class based detectors
have been proposed in \cite{DeMaio05}--\cite{DeMaio09}:
In most cases, likelihood ratios
are obtained by numerical
methods, hence
it is difficult
to explain and investigate the detection
nature and performance.
 %since the likelihood ratios in most cases

 In the above mentioned research, the radar clutter is modelled as a
 Gaussian process, whereas
 in many circumstances, for instance under a low aspect angle,
  radar clutter is
   better characterized as a
 spherically invariant random process (SIRP)
 (compound Gaussian process) \cite{Conte87}\cite{Ward90}.
 Briefly, SIRP is
   a Gaussian process $g(t)$ (called \textit{speckle})
modulated by a temporally and spatially "more-slowly varying"
non negative random process $s(t)$ (called \textit{texture}), 
which is independent of $g(t)$ and represents the
 illumination patch's reflectivity.
 Problems of detecting a perfectly known 
 %Detection in compound Gaussian clutter when a 
 radar signature in compound Gaussian clutter%  is
 %assumed to be perfectly known
have been addressed in
 \cite{Conte95}--\cite{Rabaste_Jul14}, where detectors are called
 normalized matched filters.
 The problem of detecting in compound Gaussian clutter
  a mismatched signal, a possibility in
 some practical cases, has been not considered yet and
 is solved % the topic of
 in this paper.

 We addressed, based on the generalized likelihood ratio test (GLRT),
 the problem of detecting
 a point-like target 
 embedded in compound Gaussian clutter,
  employing
 an uniform array of antennas.
 Here, the radar signature is the steering vector that, due to some reason,
 departs from the nominal one.
 The maximum likelihood estimate (MLE) of the
 unknown steering vector lying in a cone 
% in the
% presence of a steering vector mismatch.
% $b\boldsymbol{s}$ when
% $\boldsymbol{s}$, due to some error, departs from the nominal
% steering vector.
% For detection, we employed the Neyman-Pearson criterion.
%Due to ignorance of clutter
% covariance matrix and target's response, we
% resort to the generalized likelihood ratio
% test (GLRT).
%In case
 %If
% the mismatched steering vector
% is assumed to
%  belongs to a cone, its
% maximum likelihood estimate
 %of mismatched steering vector
 then leads to
  a fractional quadratically constrained quadratic optimization problem,
 which is not easy to solve \cite{DeMaio11}.
We hence introduced a more specific constraint on the mismatched steering vector:
phase shifting % (associating to the angle of arrival)
 of the mismatched steering vector lying in a known range.
 A practical example demonstrating the rationale of this assumption is
 a case of inaccurate estimate of an arrival angle.
 The optimization problem associated to the mismatched steering vector estimate
then can be transferred in a form
solvable via a semi--definite programming (SDP) \cite{Roh06}.
In case of a perfect match, %matched radar signature,
the proposed detector provides a comparable detection probability with
that of a normalized matched filter. %previous detector.
%Via numerical analysis, robustness of the proposed detector
%is confirmed.
In the presence of a mismatch, %ed signal,
the proposed detector outperforms a normalized matched filter
%the previous detector 
even with a slight
mismatch. Additionally, 
with a maximum mismatch lying in the range designed 
numerical results showed a
detection loss of around $3$ dB.
% and compared with that of previous detectors
%assuming a perfect knowledge on the signature of signal to detect.
Factors that affect the proposed detector's performance have been investigated.
Remarkably, the proposed detector
possess CFAR w.r.t the clutter covariance matrix.\\
\indent The rest of the manuscript is organized as follows.
The problem formulation is stated and the proposed detector is derived in Section II.
Numerical results are represented in section III. Finally, conclusion is reported in section IV. \\
\indent \textit{Notation:}
We adopt the notation of using boldface lower case and upper case for vectors
and matrices, respectively. The transpose and complex conjugate transpose of a matrix
are denoted by $(\cdot)^T$ and $(\cdot)^H$, respectively.
For a square matrix,
$\tr(\cdot)$, $|\cdot|$, and $\rank(\cdot)$ respectively stand for its trace,
determinant, and rank.
$\diag(\boldsymbol{A})$ denotes a vector whose $i$-th element is the
$i$-th diagonal element of a matrix $\boldsymbol{A}$; while
$\diag(\boldsymbol{a})$ denotes
 a diagonal matrix whose diagonal elements are elements of a vector $\boldsymbol{a}$.
  $(\cdot)^{-1}$ represents the inverse of an invertible square matrix.
  $\circ$ denotes the Hadamard product.
  $\boldsymbol{A}_{nm}$ denotes the element at
  $n$-th row, $m$-th column of a matrix $\boldsymbol{A}$.
  $\mathbb{C}^{N \times N}$, $\mathbb{R}^{N \times N}$, and $\mathbb{H}^{N \times N}$
stand for the set of $N \times N$ complex, $N \times N$ real, and $N \times N$
Hermittian matrices, respectively.
For any $\boldsymbol{A} \in \mathbb{H}^{N \times N}$,
$ \boldsymbol{A} \succeq 0$ means that $\boldsymbol{A}$
is a positive semi-definite matrix.
$\mathbb{R}^{+}$ is the set of non-negative real numbers.
The real part of a complex scalar, vector, or matrix is represented by
$\Re \{\cdot \}$.
$\| \cdot \|$ is the Euclidean norm of a vector, and
$|\cdot|$ denotes the modulus of a complex number.
Finally, the letter $j$ represents the square root of $-1$ and $E[\cdot]$ denotes a statistical expectation.

