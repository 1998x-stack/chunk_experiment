\setcounter{equation}{0}
\setcounter{figure}{0}
%\section{Theoretical Approaches and Perspectives}
%
\section{Relativistic Quantum Field Theory}
\label{TheorySection}
%
\subsection{Lattice-regularized QCD}
\label{sec:lQCD}
%
An introduction to the numerical simulation of lattice-regularized QCD (lQCD) is provided elsewhere \cite{Gattringer:2010zz}; so here we simply note that this method is a nonperturbative approach to solving QCD in which the gluon and quark fields are quantized on a discrete lattice of finite extent, whose intersections each represent a point in spacetime \cite{Wilson:1974sk}.

The lQCD approach has provided a spectrum of light ground-state hadrons that agrees with experiment \cite{Durr:2008zz}, but numerous hurdles are encountered in attempting to compute properties of resonance states in this way \cite{Liu:2016rwa, Briceno:2017max}.
%%Adelaide Roper paper missed multi-particle projection operators and for this reason can not be considered as appropriate LQCD treatment of Roper.
In connection with the Roper, which in reality couples strongly to many final-state interaction [FSI] channels, as indicated in Fig.\,\ref{EBACRoper}, these include the following:
%
the challenges of computing with a realistic pion mass and developing both a fully-representative collection of interpolating fields and a valid strategy for handling all contributing final-state interaction channels, which incorporate the issue of ensuring that the nucleon's lowest excitations are properly isolated from all higher excitations; and the problem of veraciously expressing chiral symmetry and the pattern by which it is broken in both the fermion action and the algorithm used in performing the simulation.%
\footnote{QCD is asymptotically free \cite{Politzer:2005kc, Gross:2005kv, Wilczek:2005az}.  It is therefore possible to define it in the absence of a quark Lagrangian mass, \emph{viz}.\ a massless theory.  In a truly massless theory, fermions are characterized by their helicity [left or right], no interactions can distinguish between left-handed and right-handed fields, and the theory is therefore chirally symmetric.  In QCD, however, dynamical effects act to destroy this symmetry.}

\begin{figure}[t]

%\centerline{%\hspace*{2em}%
%\includegraphics[clip,width=0.5\textwidth]{zFigs/F10_V1.pdf}}

\centerline{%\hspace*{2em}%
\includegraphics[clip,width=0.42\textwidth]{F10_V1R.pdf}}

\caption{\label{lQCDRoper1}
%
Illustrative collection of lQCD results for the mass of the nucleon (lower band) and its lightest positive-parity excitation as a function of $m_\pi^2$, where $m_\pi$ is the pion mass used in the simulation.  The results depicted were obtained with different lattice formulations and varying methods for identifying the excited state, as described in the source material \cite{Mahbub:2010rm, Edwards:2011jj, Engel:2013ig, Liu:2014jua, Alexandrou:2014mka} and, in particular, \cite{Liu:2016rwa}.}
\end{figure}

Much needs to be learnt and implemented before these problems are overcome, so the current status of lQCD results for the Roper is unsettled.  This is illustrated in Fig.\,\ref{lQCDRoper1}, which provides a snapshot of recent results for the masses of the nucleon and its lowest-mass positive-parity excitation.  In this image, almost all formulations of the lQCD problem produce values that extrapolate [as $m_\pi^2$ is taken toward its empirical value] to a Roper mass of roughly $1.8\,$GeV, \emph{i.e}.\ to a mass that is 0.4\,GeV above the real part of the empirical value, \emph{viz}. $1.4\,$GeV.  However, one band appears to extrapolate to somewhere near this empirical value.  Contrary to the other formulations, the fermion action in that case \cite{Liu:2014jua} possesses good chiral symmetry properties; and its proponents argue \cite{Liu:2016rwa} that this feature enables the simulation to better incorporate aspects of the extensive dynamical channel couplings which are known to be important in explaining and understanding the spectral features of $\pi N$ scattering in the $P_{11}$ channel \cite{JuliaDiaz:2007kz, Suzuki:2009nj, Kamano:2010ud}.

As we have emphasized heretofore, computing a value [even correct] for the Roper mass is insufficient to validate a formulation of the Roper resonance problem and its solution.  An additional and far more stringent test is an explanation of the pointwise behavior of the transition form factors measured in electroproduction, Eq.\,\eqref{NRcurrents}.  The first such lQCD calculations, which used the quenched truncation of the theory, are described in \cite{Lin:2008qv}.   More recently, results were obtained with two light quarks and one strange quark [$N_f = 2 + 1$] \cite{Lin:2011da}.  They are depicted in Fig.\,\ref{lQCDelectroRoper}.  These simulations identified the Roper resonance with the first positive-parity excitation of the nucleon, whose computed mass is roughly 1.8\,GeV, and focused on the low-$Q^2$ domain.  Significantly, compared with the quenched results, the inclusion of $N_f = 2 + 1$ dynamical fermions produces a sign change in $F_2^\ast$, located in the same neighborhood as that seen in experimental data.  This difference between quenched and dynamical simulations once again suggests that meson-baryon (MB)\,FSIs are a critical part of the long-wavelength structure of the Roper.

\begin{figure}[!t]

\centerline{%
\includegraphics[width=0.84\linewidth]{F11A_V2AR.pdf}}
\vspace*{1ex}

\centerline{%
\includegraphics[width=0.84\linewidth]{F11B_V2BR.pdf}}
%%
%%\begin{center}
%%\hspace{-0.75cm}
%%    \includegraphics[width=0.47\linewidth]{zBurkert/lQCDF1pR.pdf}\hspace{0.5cm}
%%	\includegraphics[width=0.47\linewidth]{zBurkert/lQCDF2pR.pdf}
%%\end{center}
%%\vspace{-.25cm}	
%
\caption{\label{lQCDelectroRoper}
%
Existing results for the Dirac [upper panel] and Pauli [lower] proton-Roper transition form factors computed using the methods of lQCD \cite{Lin:2011da} on anisotropic lattices with pion masses [in GeV]: $0.39$ [red squares], $0.45$ [orange triangles], $0.875\,$ [green circles]; and associated spatial lengths of $3$, $2.5$, $2.5\,$fm.
%
Open circles are empirical results from the CLAS Collaboration \cite{Aznauryan:2009mx, Dugger:2009pn, Mokeev:2012vsa, Mokeev:2015lda}. }
\end{figure}

% See also conclusion of Leinweber
%(Conclusion: The large mass in the CQM and in early LQCD is a consequence of the quenched approximation and in LQCD of using too small lattice sizes.)

\subsection{Insights from Continuum Analyses}
\label{sec:continuumQCD}
%
A widely used approach to developing a solution of QCD in the continuum is provided by the Dyson-Schwinger equations (DSEs) \cite{Roberts:1994dr, Chang:2011vu, Bashir:2012fs, Roberts:2015lja, Horn:2016rip, Eichmann:2016yit}, which define a symmetry-preserving [and hence Poincar\'e covariant] framework with a traceable connection to the Lagrangian of QCD.
%
The challenge in this approach is the need to employ a truncation in order to define a tractable bound-state problem.  %That truncation might also involve an \emph{Ansatz} for the infrared behaviour of one or more coloured Schwinger functions, although that need is passing \cite{Binosi:2014aea}.
%
In this connection, much has been learnt in the past twenty years, so that one may now separate DSE predictions into three classes:
%
\emph{A}.\ model-independent statements about QCD;
%
\emph{B}.\ illustrations of such statements using well-constrained model elements and possessing a traceable connection to QCD;
%
\emph{C}.\ analyses that can fairly be described as QCD-based, but whose elements have not been computed using a truncation that preserves a systematically-improvable connection with QCD.

The treatment of a baryon as a continuum three--valence-body bound-state problem became possible following the formulation of a Poincar\'e-covariant Faddeev equation \cite{Cahill:1988dx, Burden:1988dt, Cahill:1988zi, Reinhardt:1989rw, Efimov:1990uz}, which is depicted in Fig.\,\ref{figFaddeev}.  The ensuing years have seen studies increase in breadth and sophistication; and in order to understand those developments and the current status, it is apt to begin by elucidating the nature of the individual ``bodies'' whose interactions are described by that Faddeev equation.

\begin{figure}[t]
\centerline{%
\includegraphics[clip,width=0.47\textwidth]{F12_V3.pdf}}
\caption{\label{figFaddeev}
Poincar\'e covariant Faddeev equation: a homogeneous linear integral equation for the matrix-valued function $\Psi$, being the Faddeev amplitude for a baryon of total momentum $P= p_q + p_d$, which expresses the relative momentum correlation between the dressed-quarks and -diquarks within the baryon.  The shaded rectangle demarcates the kernel of the Faddeev equation: \emph{single line}, dressed-quark propagator; $\Gamma$,  diquark correlation amplitude; and \emph{double line}, diquark propagator.  Further details are provided in Sec.\,\ref{sec:continuumQCD}.}
\end{figure}

It is worth opening with an observation, \emph{viz}.\ although it is commonly thought that the Higgs boson is the origin of mass, that is incorrect: it only gives mass to some very simple particles, accounting for just 1-2\% of the weight of more complex entities, such as atoms, molecules and everyday objects.  Instead, the vast bulk of all visible mass in the universe is generated dynamically by interactions in QCD \cite{Wilczek:2012sb}.  This remark is readily substantiated by noting that the mass-scale for the spectrum of strongly interacting matter is characterized by the proton's mass, $m_N \approx 1\,{\rm GeV} \approx 2000\,m_e$, where $m_e$ is the electron mass.  However, the only apparent scale in chromodynamics is the current-quark mass.  This is the quantity generated by the Higgs boson; but, empirically, the current-mass is just $1/250^{\rm th}$ of the scale for strong interactions, \emph{viz}.\ more-than two orders-of-magnitude smaller \cite{Olive:2016xmw}.  No amount of ``staring'' at the Lagrangian for QCD, ${\mathpzc L}_{\rm QCD}$, can reveal the source of that enormous amount of ``missing mass''.  Yet, it must be there;\footnote{This is a stark contrast to quantum electrodynamics [QED] wherein, \emph{e.g}.\ the scale in the spectrum of the hydrogen atom is set by $m_e$, which is a prominent feature of ${\mathpzc L}_{\rm QED}$ that is generated by the Higgs boson.}  and exposing the character of the Roper resonance is critical to understanding the nature of strong mass generation within the Standard Model.

One of the keys to resolving this conundrum is the phenomenon of DCSB \cite{Nambu:2011zz}, which can be exposed in QCD by solving the quark gap equation, \emph{i.e}.\ the Dyson-Schwinger equation [DSE] for the dressed-quark self-energy \cite{Roberts:1994dr}:
{\allowdisplaybreaks
\begin{subequations}
\label{gendseN}
\begin{align}
%S^{-1}(k) & = [i\gamma\cdot k + M(k^2)]/Z(k^2) \\
S^{-1}(p;\zeta) & = i\gamma\cdot p \, A(p^2;\zeta) + B(p^2;\zeta)\\
%
%S^{-1}(p)
& = Z_2 \,(i\gamma\cdot p + m^{\rm bm}) + \Sigma(p;\zeta)\,,\\
%
\Sigma(p;\zeta)& =  Z_1 \int^\Lambda_{dq}\!\! g^2 D_{\mu\nu}(p-q;\zeta)\frac{\lambda^a}{2}\gamma_\mu S(q;\zeta) \Gamma^a_\nu(q,p;\zeta) ,
\end{align}
\end{subequations}}
\hspace*{-0.5\parindent}where the dressed-gluon propagator may be expressed via
\begin{equation}
\label{DressedGluon}
 D_{\mu\nu}(k;\zeta) = \Delta(k^2;\zeta) D^0_{\mu\nu}(k) \,,
%= \left[\delta_{\mu\nu}-\frac{k_\mu k_\nu}{k^2} \right] \frac{\Delta(k^2)}{k^2}
\end{equation}
$k^2 D^0_{\mu\nu}(k^2)=\delta_{\mu\nu}- p_\mu p_\nu/p^2$;
%$D^{0}_{\mu\nu}(p)$ is the free-gauge-boson propagator;
$\Gamma_\nu^a=(\lambda^a/2) \Gamma_\nu$ is the gluon-quark vertex; $\int^\Lambda_{dq}$ indicates a Poincar\'e-invariant regularization of the integral, with regularization scale $\Lambda$; $m^{\rm bm}$ is the current-quark bare mass; and $Z_{1,2}(\zeta,\Lambda)$, respectively, are the vertex and quark wave-function renormalization constants, which also depend on the renormalization scale, $\zeta$.

The dressed-quark propagator in Eq.\,\eqref{gendseN} may be rewritten in the form:
\begin{equation}
\label{Mpdefinition}
S(p;\zeta) = Z(p^2;\zeta)/[i\gamma\cdot p + M(p^2)]\,,
\end{equation}
where $M(p^2) = B(p^2;\zeta) /A(p^2;\zeta) $ is the dressed-quark mass function, which is independent of $\zeta$.  In these terms, DCSB is the appearance of a $M(p^2) \not\equiv 0$ solution of Eq.\,\eqref{gendseN} when $m^{\rm bm}\equiv 0$, so that the quark acquires mass even in the absence of a Higgs mechanism.

Whether or not DCSB emerges in the Standard Model is decided by the structure of the gap equation's kernel.  Hence the basic question is: Just what form does that kernel take?  Owing to asymptotic freedom, the answer is known on the perturbative domain \cite{Jain:1993qh, Maris:1997tm, Maris:1999nt, Qin:2011dd, Qin:2011xq, Bloch:2002eq}, \emph{viz}.\  on ${\mathpzc A} = \{(p,q)\,|\, k^2=(p-q)^2 \simeq p^2 \simeq q^2 \gtrsim 2\,{\rm GeV}^2\}$:
\begin{equation}
\tfrac{g^2}{4\pi} D_{\mu\nu}(k) \, Z_1 \, \Gamma_\nu(q,p)
 \stackrel{{\mathpzc A}}{=} \alpha_s(k^2)\,
D^{0}_{\mu\nu}(k) \, Z_2^2 \,\gamma_\nu\,,
\label{UVmodelindependent}
\end{equation}
where $\alpha_s(k^2)$ is QCD's running coupling.   The question thus actually relates only to the infrared domain,  which is a complement of ${\mathpzc A}$, and so resides in sQCD.

The past two decades have revealed a great deal about the infrared behaviour of the running coupling, dressed-gluon propagator and dressed-gluon-quark vertex; and the current state of understanding can be traced from an array of sources \cite{Boucaud:2011ug, Binosi:2014aea, Aguilar:2015bud, Binosi:2016wcx, Binosi:2016nmeRMP}.  Of particular interest is the feature that the gluon propagator saturates at infrared momenta, \emph{i.e}.\
\begin{equation}
\Delta(k^2\simeq 0) = 1/m_g^2,
\end{equation}
which entails that the long-range propagation characteristics of gluons are dramatically affected by their self-interactions. Importantly, one may associate a renormalization-group-invariant (RGI) gluon mass-scale with this effect: $m_0 \approx 0.5\,$GeV\,$\approx m_N/2$ \cite{Binosi:2014aea, Binosi:2016nmeRMP, Cyrol:2016tym}, and summarize a large body of work, which began roughly thirty-five years ago \cite{Cornwall:1981zr}, by stating that gluons, although acting as massless degrees-of-freedom on the perturbative domain, actually possess a running mass, whose value at infrared momenta is characterised by $m_0$.

The mathematical tools that have enabled theory to arrive at this conclusion \cite{Abbott:1980hw, Abbott:1981ke, Cornwall:1981zr, Cornwall:1989gv, Pilaftsis:1996fh, Binosi:2002ft, Binosi:2003rr, Binosi:2009qm} can also be used to compute a \emph{process-independent} running-coupling for QCD, $\widehat{\alpha}_{\rm PI}(k^2)$  \cite{Binosi:2016nmeRMP}.  Depicted as the solid [blue] curve in Fig.\,\ref{FigwidehatalphaII}, this is a new type of effective charge, which is an analogue of the Gell-Mann--Low effective coupling in QED \cite{GellMann:1954fq} because it is completely determined by the gauge-boson propagator.  The result in Fig.\,\ref{FigwidehatalphaII} is a parameter-free Class-A prediction, capitalizing on analyses of QCD's gauge sector undertaken using both continuum methods and numerical simulations of lQCD.

\begin{figure}[t]

\centerline{%\hspace*{2em}%
\includegraphics[clip,width=0.42\textwidth]{F13_V4.pdf}}

\caption{\label{FigwidehatalphaII}
%
Solid [blue] curve: process-in\-de\-pen\-dent RGI running-coupling $\widehat{\alpha}_{\rm PI}(k^2)$ \cite{Binosi:2016nmeRMP}.  The shaded (blue) band bracketing this curve combines a 95\% confidence-level window based on existing lQCD results for the gluon two-point function with an error of 10\% in the continuum analysis of relevant ghost-gluon dynamics.
%
World data on the process-dependent effective coupling $\alpha_{g_1}$, defined via the Bjorken sum rule \cite{%
%%CLAS
Deur:2005cf, Deur:2008rf, Deur:2014vea,
%% Hermes
Ackerstaff:1997ws, Ackerstaff:1998ja, Airapetian:1998wi, Airapetian:2002rw, Airapetian:2006vy,
%% Fermilab
Kim:1998kia,
%% CERN:
Alexakhin:2006oza, Alekseev:2010hc, Adolph:2015saz,
%% SLAC:
Anthony:1993uf, Abe:1994cp, Abe:1995mt, Abe:1995dc, Abe:1995rn, Anthony:1996mw, Abe:1997cx, Abe:1997qk, Abe:1997dp, Abe:1998wq, Anthony:1999py, Anthony:1999rm, Anthony:2000fn, Anthony:2002hy}.
%
The shaded [yellow] band on $k>1\,$GeV represents $\alpha_{g_1}$ obtained from the Bjorken sum by using QCD evolution \cite{Gribov:1972, Altarelli:1977, Dokshitzer:1977} to extrapolate high-$k^2$ data into the depicted region \cite{Deur:2005cf, Deur:2008rf}; and, for additional context, the dashed [red] curve is the effective charge obtained in a light-front holographic model, canvassed elsewhere \cite{Deur:2016tte}.
%
%The k-axis scale is linear to the left of the vertical dashed line and logarithmic otherwise.
}
\end{figure}

The data in Fig.\,\ref{FigwidehatalphaII} represent empirical information on $\alpha_{g_1}$, a process-dependent effective-charge \cite{Grunberg:1982fw} determined from the Bjorken sum rule, one of the most basic constraints on our knowledge of nucleon spin structure.  Sound theoretical reasons underpin the almost precise agreement between $\widehat{\alpha}_{\rm PI}$ and $\alpha_{g_1}$ \cite{Binosi:2016nmeRMP}, so that the Bjorken sum may be seen as a near direct means by which to gain empirical insight into QCD's Gell-Mann--Low effective charge.
%
Given the behavior of the prediction in Fig.\,\ref{FigwidehatalphaII}, it is evident that the coupling is everywhere finite in QCD, \emph{i.e}. there is no Landau pole, and this theory possesses an infrared-stable fixed point.  Evidently, QCD is infrared finite owing to the dynamical generation of a gluon mass scale.\footnote{%
A theory is said to possess a Landau pole at $k^2_{\rm L}$ if the effective charge diverges at that point.  In QCD perturbation theory, such a Landau pole exists at $k^2_L=\Lambda_{\rm QCD}^2$.  Were such a pole to persist in a complete treatment of QCD, it would signal an infrared failure of the theory.  On the other hand, the absence of a Landau pole in QCD supports a view that QCD is unique amongst four-dimensional quantum field theories in being defined and internally consistent at all energy scales.  This might have implications for attempts to develop an understanding of physics beyond the Standard Model based upon non-Abelian gauge theories \cite{Appelquist:1996dq, Sannino:2009za, Appelquist:2009ka, Hayakawa:2010yn, Cheng:2013eu, Aoki:2013xza, DeGrand:2015zxa, Binosi:2016xxu}.}

As a unique process-independent effective charge, $\widehat{\alpha}_{\rm PI}$ appears in every one of QCD's dynamical equations of motion, setting the interaction strength in all cases, including the gap equation, Eq.\,\eqref{gendseN}.  It therefore plays a crucial role in determining the fate of chiral symmetry.

The remaining element in the gap equation is the dressed gluon-quark vertex, $\Gamma_\nu$, whose complete expression involves twelve matrix-valued tensor structures, six of which are zero in the absence of chiral symmetry breaking.  If this vertex were only weakly modified from its tree-level form, $\gamma_\nu$, then, with $\widehat{\alpha}_{\rm PI}$ in Fig.\,\ref{FigwidehatalphaII}, chiral symmetry would be preserved in Nature \cite{Binosi:2016wcx}.  It is not; and after nearly forty years of studying $\Gamma_\nu$, with numerous contributions that may be traced from an analysis of Abelian theories \cite{Ball:1980ay}, continuum and lattice efforts have revealed just how the vertex is dressed so that DCSB is unavoidable.  Namely, the smooth, infrared-finite coupling depicted in Fig.\,\ref{FigwidehatalphaII} is strong enough to force nonzero values for the six terms in $\Gamma_\nu$ that usually vanish in the chiral limit.  This seeds a powerful positive feedback chain so that chiral symmetry is not only broken, but there is a sense in which it is very difficult to keep the growth of the dressed-quark mass function, $M(p^2)$, within physically reasonable bounds \cite{Binosi:2016wcx}.  Consequently, the solution of Eq.\,\eqref{gendseN} describes a dressed-quark with a dynamically generated running mass that is large in the infrared: $M(p^2 \simeq 0) \approx 0.3\,$GeV, as illustrated in Fig.\,\ref{gluoncloud}.

\begin{figure}[t]
\centerline{\includegraphics[width=0.38\textwidth]{F14_V5R2.pdf}}
\caption{\label{gluoncloud}
Renormalization-group-invariant dressed-quark mass function, $M(p)$ in Eq.\,\eqref{Mpdefinition}: \emph{solid curves} -- gap equation results \cite{Bhagwat:2003vw,Bhagwat:2006tu}, ``data'' -- numerical simulations of lQCD \protect\cite{Bowman:2005vx}.  (\emph{N.B}.\ $m=70\,$MeV is the uppermost curve and current-quark mass decreases from top to bottom.)  The current-quark of perturbative QCD evolves into a constituent-quark as its momentum becomes smaller.  The constituent-quark mass arises from a cloud of low-momentum gluons attaching themselves to the current-quark.  This is DCSB, the essentially nonperturbative effect that generates a quark \emph{mass} \emph{from nothing}; namely, it occurs even in the chiral limit.
%
Notably, the size of $M(0)$ is a measure of the magnitude of the QCD scale anomaly in $n=1$-point Schwinger functions \cite{Roberts:2016vyn};
%
and experiments on $Q^2\in [0,12]\,$GeV$^2$ at the upgraded JLab facility will be sensitive to the momentum dependence of $M(p)$ within a domain that is here indicated approximately by the shaded region.
}
\end{figure}

It is dressed quarks characterized by the mass function in Fig.\,\ref{gluoncloud} that are the basic elements in the Faddeev equation depicted in Fig.\,\ref{figFaddeev}, whose solutions in all allowed channels both generate the baryon spectrum and play a key role in computing the transitions between ground- and excited-states.
%
As highlighted elsewhere \cite{Cloet:2013gva, Binosi:2016wcx}, this means that since quarks carry electric charge, experiments involving electron scattering from hadrons serve as a probe of the momentum dependence of this mass function and also its collateral influences.  Measurements at the upgraded JLab facility will explore a region that is indicated approximately by the shading in Fig.\,\ref{gluoncloud}, \emph{i.e}.\ the domain of transition from strong- to perturbative-QCD.

Contemporary theory indicates that DCSB is responsible for more than 98\% of the visible mass in the Universe \cite{Brodsky:2015aiaRMP}.  Simultaneously, it ensures the existence of nearly-massless pseudo--Nambu-Goldstone modes [pions], each constituted from a valence-quark and -antiquark whose individual Lagrangian current-quark masses are $<1$\% of the proton mass \cite{Maris:1997hd}.

Another important consequence of DCSB is less well known.  Namely, any interaction capable of creating pseudo--Nambu-Goldstone modes as bound-states of a light dressed-quark and -antiquark, and reproducing the measured value of their leptonic decay constants, will necessarily also generate strong colour-antitriplet correlations between any two dressed quarks contained within a nucleon. %: scalar-isoscalar, $[ud]$; and pseudovector-isovector, $\{uu\}$, $\{ud\}$, $\{dd\}$.
Although a rigorous proof within QCD cannot be claimed, this assertion is based upon an accumulated body of evidence, gathered in two decades of studying two- and three-body bound-state problems in hadron physics \cite{Segovia:2015ufa}.  No realistic counter examples are known; and the existence of such diquark correlations is also supported by lQCD \cite{Alexandrou:2006cq, Babich:2007ah}.

The properties of such diquark correlations have been charted.  As color-carrying correlations, diquarks are confined \cite{Bender:1996bb, Bender:2002as, Bhagwat:2004hn}.  Additionally, owing to properties of charge-conjugation, a diquark with spin-parity $J^P$ may be viewed as a partner to the analogous $J^{-P}$ meson \cite{Cahill:1987qr}.  It follows that the strongest diquark correlations are: scalar isospin-zero, $[ud]_{0^+}$; and pseudovector, isospin-one, $\{uu\}_{1^+}$, $\{ud\}_{1^+}$, $\{dd\}_{1^+}$.  Moreover, whilst no pole-mass exists, the following mass-scales, which express the strength and range of the correlation, may be associated with these diquarks \cite{Cahill:1987qr, Maris:2002yu, Alexandrou:2006cq, Babich:2007ah, Eichmann:2016hgl, Lu:2017cln} [in GeV]:
%$m_{[ud]_{0^+}} \approx 0.7-0.8\,$GeV, $m_{\{uu\}_{1^+}}  \approx 0.9-1.1\,$GeV,
\begin{equation}
m_{[ud]_{0^+}} \approx 0.7-0.8\,,\;
m_{\{uu\}_{1^+}}  \approx 0.9-1.1  \,,
\end{equation}
with $m_{\{dd\}_{1^+}}=m_{\{ud\}_{1^+}} = m_{\{uu\}_{1^+}}$ in the isospin symmetric limit.  The ground-state nucleon necessarily contains both scalar-isoscalar and pseudovector-isovector correlations: neither can be ignored and their presence has many observable consequences \cite{Roberts:2013mja, Segovia:2013uga}.

Realistic diquark correlations are also soft and interacting.  All carry charge, scatter electrons, and possess an electromagnetic size which is similar to that of the analogous mesonic system, \emph{e.g}.\ \cite{Maris:2004bp, Eichmann:2008ef, Roberts:2011wy}:
%%\begin{equation}
%%\label{qqradii}
$r_{[ud]_{0^+}} \gtrsim r_\pi$, $r_{\{uu\}_{1^+}} \gtrsim r_\rho$,
%%\end{equation}
with $r_{\{uu\}_{1^+}} > r_{[ud]_{0^+}}$.  As in the meson sector, these scales are set by that associated with DCSB.

It is important to emphasize that these fully dynamical diquark correlations are vastly different from the static, pointlike ``diquarks'' which featured in early attempts \cite{Lichtenberg:1967zz, Lichtenberg:1968zz} to understand the baryon spectrum and explain the so-called missing resonance problem, \emph{viz}.\ the fact that quark models predict many more baryons states than were observed in the previous millennium \cite{Burkert:2004sk}.   As we have stated, modern diquarks are soft [not pointlike].  They also enforce certain distinct interaction patterns for the singly- and doubly-represented valence-quarks within the proton, as reviewed elsewhere \cite{Roberts:2013mja, Segovia:2014aza, Roberts:2015lja, Segovia:2016zyc}.  On the other hand, the number of states in the spectrum of baryons obtained from the Faddeev equation \cite{Eichmann:2016hgl, Lu:2017cln} is similar to that found in the three-constituent quark model, just as it is in contemporary lQCD spectrum calculations \cite{Edwards:2011jj}.  [Notably, modern data and recent analyses have already reduced the number of missing resonances \cite{Ripani:2002ss, Burkert:2012ee, Kamano:2013iva, Crede:2013sze, Mokeev:2015moa,  Anisovich:2017pmi}.]

The existence of these tight correlations between two dressed quarks is the key to transforming the three valence-quark scattering problem into the simpler Faddeev equation problem illustrated in Fig.\,\ref{figFaddeev}, without loss of dynamical information \cite{Eichmann:2009qa}.   The three gluon vertex, a signature feature of QCD's non-Abelian character, is not explicitly part of the bound-state kernel in this picture.  Instead, one capitalizes on the fact that phase-space factors materially enhance two-body interactions over $n\geq 3$-body interactions and exploits the dominant role played by diquark correlations in the two-body subsystems.  Then, whilst an explicit three-body term might affect fine details of baryon structure, the dominant effect of non-Abelian multi-gluon vertices is expressed in the formation of diquark correlations.  Consequently, the active kernel here describes binding within the baryon through diquark breakup and reformation, which is mediated by exchange of a dressed-quark; and such a baryon is a compound system whose properties and interactions are largely determined by the quark$+$diquark structure evident in Fig.\,\ref{figFaddeev}.

This continuum approach to the baryon bound-state problem has been employed to calculate a wide range of nucleon-related observables \cite{Wilson:2011aa, Chang:2012cc, Roberts:2013mja, Segovia:2014aza, Roberts:2015dea, Xu:2015kta, Segovia:2016zyc, Eichmann:2016yit}.  In particular, in the computation of the mass and structure of the nucleon and its first radial excitation \cite{Segovia:2015hra}.  This Class-C analysis begins by solving the Faddeev equation, to obtain the masses and Poincar\'e-covariant wave functions for these systems,
%\footnote{In the limit of exact isospin symmetry, which is a good approximation within the strong interaction, the neutron and proton wave functions are indistinguishable, and the same is true for their excitations.}
taking each element of the equation to be as specified in \cite{Segovia:2014aza}, which provides a successful description of the properties of the nucleon and $\Delta$-baryon.  With those inputs, the masses are [in GeV]:
\begin{equation}
\label{eqMasses}
%\textstyle
%\begin{array}{l|cc}
%            & \mbox{Nucleon\,(N)} & \mbox{Roper\,(R)} \\\hline
%\mbox{mass} & 1.18 & 1.73
%\end{array}\,.
\mbox{nucleon\,(N)} = 1.18\,,\;
\mbox{nucleon-excited\,(R)} = 1.73\,.
\end{equation}

The masses in Eq.\,\eqref{eqMasses} correspond to the locations of the two lowest-magnitude $J^P=1/2^+$ poles in the three dressed-quark scattering problem.  The associated residues are the canonically-normalized Faddeev wave functions, which depend upon $(\ell^2,\ell \cdot P)$, where $\ell$ is the quark-diquark relative momentum and $P$ is the baryon's total momentum.  Figure\,\ref{figFA} depicts the zeroth Chebyshev moment of all $S$-wave components in that wave function, \emph{i.e}.\ projections of the form
\begin{equation}
%\textstyle
{\mathpzc W}(\ell^2;P^2) = \frac{2}{\pi} \int_{-1}^1 \! du\,\sqrt{1-u^2}\,
{\mathpzc W}(\ell^2,u; P^2)\,,
\end{equation}
where $u=\ell\cdot P/\sqrt{\ell^2 P^2}$.  Drawing upon experience with quantum mechanics and with excited-state mesons studied via the Bethe-Salpeter equation \cite{Holl:2004fr, Qin:2011xq, Rojas:2014aka}, the appearance of a single zero in $S$-wave components of the Faddeev wave function associated with the first excited state in the three dressed-quark scattering problem indicates that this state is a radial excitation.  Notably, one may associate a four-vector length-scale of $1/[0.4 {\rm GeV}]\approx 0.5\,$fm with the location of this zero.
% a 3-vector length-scale could be this number * 2/Sqrt[3]
[Similar conclusions have been drawn using lQCD \cite{Roberts:2013ipa}.]

\begin{figure}[!t]
%
\centerline{%
\includegraphics[width=0.68\linewidth]{F15A_V6A.pdf}}
\vspace*{1ex}

\centerline{%
\includegraphics[width=0.68\linewidth]{F15B_V6B.pdf}}
%
%%\begin{center}
%%\hspace{-0.75cm}
%%    \includegraphics[width=0.45\linewidth]{zFigs/TF7A.pdf}\hspace{0.5cm}
%%	\includegraphics[width=0.45\linewidth]{zFigs/TF7B.pdf}
%%\end{center}
%%\vspace{-.25cm}	
%
\caption{\label{figFA}
%
\emph{Upper panel}.  Zeroth Chebyshev moment of all $S$-wave components in the nucleon's Faddeev wave function, which is obtained from $\Psi$ in Fig.\,\ref{figFaddeev}, by reattaching the dressed-quark and -diquark legs.
%
\emph{Lower panel}.  Kindred functions for the first $J^P=1/2^+$ excited state.
%
Legend: $S_1$ is associated with the baryon's scalar diquark; the other two curves are associated with the axial-vector diquark; and here the normalization is chosen such that $S_1(0)=1$.}
\end{figure}

% 1357, 76 ... 1364, 105
Let us focus now on the masses in Eq.\,\eqref{eqMasses}.  As discussed in connection with Fig.\,\ref{EBACRoper}, the empirical values of the pole locations for the first two states in the nucleon channel are: $0.939\,$GeV for the nucleon; and two poles for the Roper, $1.357 - i \,0.076$, $1.364 - i \, 0.105\,$GeV.  At first glance, these values appear unrelated to those in Eq.\,\eqref{eqMasses}.  However, deeper consideration reveals \cite{Eichmann:2008ae, Eichmann:2008ef} that the kernel in Fig.\,\ref{figFaddeev} omits all those resonant contributions which may be associated with the MB\,FSIs that are resummed in dynamical coupled channels models  \cite{JuliaDiaz:2007kz, Suzuki:2009nj, Kamano:2010ud, Ronchen:2012eg, Kamano:2013iva, Doring:2014qaa} in order to transform a bare-baryon into the observed state.  The  Faddeev equation analysed to produce the results in Eq.\,\eqref{eqMasses} should therefore be understood as producing the \emph{dressed-quark core} of the bound-state, not the completely-dressed and hence observable object.

Clothing the nucleon's dressed-quark core by including resonant contributions to the kernel produces a physical nucleon whose mass is $\approx 0.2$\,GeV lower than that of the core \cite{Ishii:1998tw, Hecht:2002ej, Chang:2009ae, Sanchis-Alepuz:2014wea}.  Similarly, clothing the $\Delta$-baryon's core lowers its mass by $\approx 0.16\,$GeV \cite{JuliaDiaz:2007kz}.   It is therefore no coincidence that [in GeV] $1.18-0.2 = 0.98\approx 0.94$, \emph{i.e}.\ the nucleon mass in Eq.\,\eqref{eqMasses} is 0.2\,GeV greater than the empirical value.  A successful body of work on the baryon spectrum \cite{Lu:2017cln}, and nucleon and $\Delta$ elastic and transition form factors \cite{Segovia:2014aza, Roberts:2015dea, Segovia:2016zyc} has been built upon this knowledge of the impact of omitting resonant contributions and the magnitude of their effects.  Therefore, a comparison between the empirical value of the Roper resonance pole-position and the computed dressed-quark core mass of the nucleon's radial excitation is not the critical test.  Instead, it is that between the masses of the quark core and the value determined for the meson-undressed bare-Roper, \emph{viz}.:
\begin{equation}
\label{eqMassesA}
\begin{array}{l|c}
    & \mbox{mass/GeV} \\\hline
\mbox{R}_{{\rm core}}^{\mbox{\footnotesize \cite{Segovia:2015hra}}} & 1.73\\
\mbox{R}_{{\rm core}}^{\mbox{\footnotesize \cite{Wilson:2011aa}}}   &  1.72\\
\mbox{R}_{{\rm core}}^{\mbox{\footnotesize \cite{Lu:2017cln}}}   &  1.82\\
\mbox{R}_{\rm DCC\,bare}^{\mbox{\footnotesize \cite{Suzuki:2009nj}}} & 1.76
\end{array}\,.
\end{equation}
Evidently, as already displayed in Fig.\,\ref{EBACRoper}, the DCC bare-Roper mass agrees with the quark core results obtained using both a QCD-kindred interaction \cite{Segovia:2015hra} and refined treatments of a strictly-implemented vector$\,\otimes\,$vector contact-interaction \cite{Wilson:2011aa, Lu:2017cln}.\footnote{It is also commensurate with the value obtained in simulations of lQCD whose formulation and/or parameters suppress MB\,FSIs, Fig.\,\ref{lQCDRoper1}.}  This is notable because all these calculations are independent, with just one common feature; namely, an appreciation that observed hadrons should realistically be built from a dressed-quark core plus a meson-cloud.

The agreement in Eq.\,\eqref{eqMassesA} is suggestive but not conclusive because, plainly, the same mass is obtained from the Faddeev equation using vastly different fundamental interactions.  The mass alone, then, does not serve as a fine discriminator between theoretical pictures of the nucleon's first radial excitation and its possible identification with the Roper resonance.   Critical additional tests are imposed by requiring that the theoretical picture combine a prediction of the Roper's mass with detailed descriptions of its structure and how that structure is revealed in the momentum dependence of the proton-Roper transition form factors.  Moreover, it must combine all this with a similarly complete picture of the proton, from which the Roper resonance is produced.  As detailed in Sec.\,\ref{Experiment}, precise empirical information is now available on the proton-Roper transition form factors, reaching to momentum transfers $Q^2\approx 4.5\,$GeV$^2$.  At such scales, these form factors probe a domain whereupon hard dressed-quark degrees-of-freedom could be expected to determine their behavior.  Finally, to increase the level of confidence, one should impose an additional test, requiring that the theoretical picture also explain all related properties of the $\Delta^+$-baryon, which is typically viewed as the proton's spin-flip excitation.

With Faddeev amplitudes for the participating states in hand, computation of the form factors in Eq.\,\eqref{NRcurrents} is a straightforward numerical exercise once the electromagnetic current is specified.  That sufficient to express the interaction of a photon with a baryon generated by the Faddeev equation in Fig.\,\ref{figFaddeev} is known \cite{Oettel:1999gc, Segovia:2014aza}.  It is a sum of six terms, with the photon separately probing the quarks and diquarks in various ways, so that diverse features of quark dressing and the quark-quark correlations all play a role in determining the form factors.

In any computation of transition form factors, one must first calculate the analogous elastic form factors for the states involved because the associated values of $F_1(Q^2=0)$ fix the normalization of the transition.  These normalizations can also be used to reveal the diquark content of the bound-states \cite{Roberts:2013mja, Segovia:2014aza, Segovia:2015hra}; and the analysis which produces the first row in Eq.\,\eqref{eqMassesA} yields:
\begin{equation}
\label{Pdiquark}
\begin{array}{l|cc}
        & N    & R    \\\hline
P_{J=0 \times 0} & 62\% & 62\% \\
P_{J\neq 0 \times 0} & 38\% & 38\% \\
\end{array}\,,
\end{equation}
where $P_{J=0 \times 0}$ measures the contribution to $F_1(Q^2=0)$ from overlaps with a scalar diquark in both the initial and final state, and $P_{J\neq 0 \times 0}$ is all the rest.  This calculation predicts that the relative strength of scalar and axial-vector diquark correlations in the nucleon and its radial excitation is the same.  However, the result is sensitive to the character of the quark-quark interaction.  Hence, this is a prediction that is tested by experiment.
%
Charge radii may also be computed from the elastic form factors, with the result \cite{Segovia:2015hra}: $r^{ \Psi}_{R^+}/r_{p}^{\Psi}=1.8$, \emph{i.e}.\ a quark-core radius for the radial excitation that is 80\% larger than that of the ground-state.  In contrast, non-relativistic harmonic oscillator wave functions yield a value of 1.5 for this ratio.  The difference highlights the impact of orbital angular momentum and spin-orbit repulsion, which is introduced by relativity into the Poincar\'e-covariant Faddeev wave functions for the nucleon and its radial excitation and increases the size of both systems.
%
The ratio of magnetic radii is $1.6$.

\begin{figure}[!t]
%
\includegraphics[height=0.52\linewidth,width=0.74\linewidth]{F16A_V7A.pdf}\hspace*{0.9em}
\vspace*{1ex}

\centerline{%
\includegraphics[width=0.72\linewidth]{F16B_V7B.pdf}}
%
%%\begin{center}
%%\hspace{-0.75cm}
%%    \includegraphics[width=0.4\linewidth]{zFigs/TF8A.pdf}\hspace{0.5cm}
%%	\includegraphics[width=0.4\linewidth]{zFigs/TF8B.pdf}
%%\end{center}
%%\vspace{-.25cm}	
%
\caption{\label{figFT}
%
\emph{Upper panel} -- $F_{1}^{\ast}$ as a function of $x=Q^2/m_N^2$.
%
Legend: Gray band within black curves -- dressed-quark core contribution with up-to 20\% Faddeev amplitude renormalization from MB\,FSIs, implemented according to Eq.\,\eqref{eqMBFSI}.  The transition form factor curve with smallest magnitude at $x=1$ has the maximum renormalization.
%
Green band within green dotted curves -- inferred MB\,FSI contribution.  The band demarcates the range of uncertainty arising from $0\to 20$\% renormalization of the dressed-quark core.
%
Blue dashed curve --  least-squares fit to the data on $x \in (0,5)$.
%
Red dot-dashed curve -- contact interaction result \cite{Wilson:2011aa}.
%
\emph{Lower panel} -- $F_{2}^{\ast}(x)$ with same legend.
%
Data: circles [blue] \cite{Aznauryan:2009mx};
triangle [gold] \cite{Dugger:2009pn};
squares [purple] \cite{Mokeev:2012vsa, Mokeev:2015lda};
and star [green] \cite{Olive:2016xmw}.}
\end{figure}

The form factors predicted in \cite{Segovia:2015hra} to describe the transition between the proton and its first radial excitation are depicted in Fig.\,\ref{figFT}.
%
The upper panel depicts the Dirac transition form factor $F_{1}^{\ast}$, which vanishes at $x=Q^2/m_N^2 =0$ owing to orthogonality between the proton and its radial excitation.  The calculation [gray band] agrees quantitatively in magnitude and qualitatively in trend with the data on $x\gtrsim 2$.  Crucially, nothing was tuned to achieve these results.  Instead, the nature of the prediction owes fundamentally to the QCD-derived momentum-dependence of the propagators and vertices employed in formulating the bound-state and scattering problems.  This point is further highlighted by the contact-interaction result [red, dot-dashed]: with momentum-independent masses and vertices, the prediction disagrees both quantitatively and qualitatively with the data.  Experiment is evidently a sensitive tool with which to chart the nature of the quark-quark interaction and hence discriminate between competing theoretical hypotheses; and it is plainly settling upon an interaction that produces a momentum-dependent quark mass of the form in Fig.\,\ref{gluoncloud}, which characterises QCD.

The mismatch on $x\lesssim 2$ between data and the prediction in \cite{Segovia:2015hra} is also revealing.  As we have emphasized, that calculation yields only those form factor contributions generated by a rigorously-defined dressed-quark core whereas meson-cloud contributions are expected to be important on $x\lesssim 2$.  Thus, the difference between the prediction and data may plausibly be attributed to MB\,FSIs.  One can quantify this by recognizing that the dressed-quark core component of the baryon Faddeev amplitudes should be renormalized by inclusion of meson-baryon ``Fock-space'' components, with a maximum strength of 20\% \cite{Cloet:2008fw, Eichmann:2008ef, Bijker:2009up, Aznauryan:2016wwm}.  Naturally, since wave functions in quantum field theory evolve with resolving scale \cite{Lepage:1979zb, Lepage:1980fj, Efremov:1979qk, Raya:2015gva, Gao:2017mmp}, the magnitude of this effect is not fixed.  Instead ${\mathpzc I}_{MB}={\mathpzc I}_{MB}(Q^2)$, where $Q^2$ measures the resolving scale of any probe and ${\mathpzc I}_{MB}(Q^2) \to 0^+$ monotonically with increasing $Q^2$.  Now, form factors in QCD possess power-law behaviour, so it is appropriate to renormalize the dressed-quark core contributions via
\begin{subequations}
\begin{align}
\label{eqMBFSI}
F_{\rm core}(Q^2) & \to [1- {\mathpzc I}_{MB}(Q^2)] F_{\rm core}(Q^2)\,,\\
\quad {\mathpzc I}_{MB}(Q^2) & = [1-0.8^2]/[1+Q^2/\Lambda_{MB}^2]\,,
\end{align}
\end{subequations}
with $\Lambda_{MB}=1\,$GeV marking the midpoint of the transition between the strong and perturbative domains of QCD as measured by the behaviour of the dressed-quark mass-function in Fig.\,\ref{gluoncloud}.  Following this procedure \cite{Roberts:2016dnb}, one arrives at the estimate of MB\,FSI contributions depicted in Fig.\,\ref{figFT}.
%%which represent the best inference available today for the strength of MB-FSIs on the $N\to R$ transition form factors and helicity amplitudes.

The lower panel of Fig.\,\ref{figFT} depicts the Pauli form factor, $F_{2}^{\ast}$.  All observations made regarding $F_{1}^{\ast}$ also apply here, including those concerning the inferred meson-cloud contributions.  Importantly, the existence of a zero in $F_{2}^{\ast}$ is not influenced by meson-cloud effects, although its precise location is.

\begin{figure}[!t]
%
%\centerline{%
\includegraphics[height=0.52\linewidth,width=0.74\linewidth]{F17A_V8A.pdf}\hspace*{0.9em}
\vspace*{1ex}

\centerline{%
\includegraphics[width=0.72\linewidth]{F17B_V8B.pdf}}
%
%%\begin{center}
%%\hspace{-0.75cm}
%%    \includegraphics[width=0.42\linewidth]{zFigs/TF9A.pdf}\hspace{0.5cm}
%%	\includegraphics[width=0.41\linewidth]{zFigs/TF9B.pdf}
%%\end{center}
%%\vspace{-.25cm}	
%
\caption{\label{aznauryanLFQM}
%
$F_{1}^{\ast}$ (upper panel) and $F_{2}^{\ast}$ (lower) for the proton-Roper transition as a function of $x=Q^2/m_N^2$.
%
Legend.  Gray band within black curves, prediction in \cite{Segovia:2015hra};
%
and dashed (blue) curve, least-squares fit to the data on $x \in (0,5)$.  [Both also depicted in Fig.\,\ref{figFT}.]
%
Dotted (brown) curve, LF\,CQM result reconstructed from the helicity amplitudes in \cite{Aznauryan:2016wwm} using Eqs.\,\eqref{ThHelAmp}.
%
Data: circles [blue] \cite{Aznauryan:2009mx};
triangle [gold] \cite{Dugger:2009pn};
squares [purple] \cite{Mokeev:2012vsa, Mokeev:2015lda};
and star [green] \cite{Olive:2016xmw}.
}
\end{figure}

This is an opportune moment to review the picture of the Roper resonance that is painted by constituent quark models.  Figure\,\ref{A12_lowQ} emphasized the importance of relativity in reproducing a zero in $F_2^\ast$, which generates the zero in $A_{1/2}$; and the discussion in this subsection has highlighted that the natural degrees-of-freedom to employ when studying measurable form factors are strongly-dressed quasi-particles (and correlations between them).  It is interesting, therefore, that constituent quark models, formulated using light-front quantization (LF\,CQMs) and incorporating aspects of the QCD dressing explained herein, have been used with success to describe features of the nucleon-Roper transition \cite{Cardarelli:1996vn, Aznauryan:2012ec, Aznauryan:2016wwm}.
%% --- Cardarelli gets level ordering of Captick & Isgur = 1/2+ 1/2- 1/2+ ... so it would place Roper at same place as DSE if MB FSIs were anticipated.
In these models, the dressing effects are implemented phenomenologically, \emph{i.e}. via parametrizations chosen in order to secure a good fit to certain data; and they do not properly comply with QCD constraints at large momenta, \emph{e.g}.\ using constituent-quark electromagnetic form factors that fall too quickly with increasing momentum transfer \cite{Cardarelli:1996vn} or a dressed-quark mass function that falls too slowly \cite{Aznauryan:2012ec}.  Notwithstanding these limitations, the outcomes expressed are qualitatively significant.  This is illustrated in Fig.\,\ref{aznauryanLFQM}, which reveals a striking similarity between the DSE prediction for the dressed quark-core components of the transition form factors and those computed using a LF\,CQM that incorporates a running quark mass \cite{Aznauryan:2016wwm}.  The parameters of the LF\,CQM model were adjusted by fitting nucleon elastic form factors on $Q^2\in [0,16]\,$GeV$^2$, allowing room for MB\,FSIs and estimating their impact.   Qualitatively, therefore, despite fundamental differences in formulation, both the DSE and LF\,CQM approaches arrive at the same conclusion regarding the nature of the proton-Roper transition form factors: whilst MB\,FSIs contribute materially on $x\lesssim 2$, a dressed-quark core is exposed and probed on $x\gtrsim 2$.

It should be emphasized here that were the Roper a purely molecular meson-baryon system, in the sense defined in Sec.\,\ref{sec:DCC}, then the transition form factors would express an overlap between an initial state proton, which certainly possesses a dressed-quark core, and a much more diffuse system.  In such circumstances, $F_{1,2}^\ast$ would be far softer than anything that could be produced by a final state with a material dressed-quark core.  Consequently, the agreement between CLAS data and theory in Figs.\,\ref{figFT}, \ref{aznauryanLFQM} renders a molecular hypothesis untenable.
%% ...Form factor of molecular system ... scattering from deuteron ... difuse object ... large ... form factor has no hard part from a quark core

Finally, given the scope of agreement between experiment and theory in Figs.\,\ref{figFT}, \ref{aznauryanLFQM}, it is time to apply a final test, \emph{viz}.\ does the same perspective also deliver a consistent description of the nucleon and $\Delta$-baryon elastic form factors and the nucleon-$\Delta$ transition?  An affirmative answer is supported by an array of results \cite{Segovia:2014aza, Roberts:2015dea, Segovia:2016zyc}, from which we will highlight just one.

The $\gamma^\ast+N\to\Delta$ transition form factors excite keen interest because of their use in probing, \emph{inter} \emph{alia}, the relevance of perturbative QCD to processes involving moderate momentum transfers \cite{Carlson:1985mm, Pascalutsa:2006up, Aznauryan:2011qj}; shape deformation of hadrons \cite{Alexandrou:2012da}; and the role that resonance electroproduction experiments can play in exposing non-perturbative features of QCD \cite{Aznauryan:2012baS}.  Precise data on the dominant $\gamma^\ast+N\to\Delta$ magnetic transition form factor now reaches to $Q^2 = 6.5\,$GeV$^2$ \cite{Aznauryan:2009mx, Villano:2009sn}.  It poses both opportunities and challenges for QCD theory because this domain joins the infrared, where MB\,FSIs can be important, to the ultraviolet, where the dressed-quark core should control the transition.  The result obtained using the framework that produces the grey band in Figs.\,\ref{figFT}, \ref{aznauryanLFQM} is drawn as the solid curve in Fig.\,\ref{figNDelta}.  In this case, too, there is a mismatch between data and calculation at low-$Q^2$: both qualitatively and quantitatively, that difference can be attributed to MB\,FSIs \cite{Sato:2000jf, Burkert:2004sk, JuliaDiaz:2006xt, Pascalutsa:2006up, Crede:2013sze}.  However, on $Q^2\gtrsim 1\,$GeV$^2$ the theoretical curve agrees with the data: this is significant because, once again, no parameters were varied in order to ensure this outcome.  Importantly, a similar picture emerges when quark dressing effects are incorporated in LF\,CQMs  \cite{Aznauryan:2016wwm}.

\begin{figure}[t]
\centerline{\includegraphics[clip,width=0.35\textwidth]{F18_V9.pdf}}
\caption{\label{figNDelta}
%
Comparison between data \cite{Aznauryan:2009mx} on the magnetic $\gamma^\ast+N\to\Delta$ transition form factor and a theoretical prediction (solid curve) \cite{Segovia:2014aza, Segovia:2016zyc}.  The dashed curve shows the result that would be obtained if the interaction between quarks were momentum-independent \cite{Segovia:2013rca}.}
\end{figure}


\subsection{Light-Front Transverse Transition Charge Densities}
%
The nucleon-$\Delta$ and nucleon-Roper transition form factors have been dissected in order to reveal the relative contributions from dressed-quarks and the various diquark correlations \cite{Segovia:2016zyc}.  This analysis reveals that $F_1^\ast$ is largely determined by a process in which the virtual photon scatters from the uncorrelated $u$-quark with a $[ud]_{0^+}$ diquark as a spectator, with lesser but non-negligible contributions from other processes.  In exhibiting these properties, $F_1^\ast$ shows qualitative similarities to the proton's Dirac form factor.

Such features of the transition can also be highlighted by studying the following transition charge density: \cite{Tiator:2008kd}:
%\begin{subequations}
%\label{eqrhob}
\begin{align}
\label{eqrhob}
\rho^{pR}(|\vec{b}|)
& := \int \frac{d^2 \vec{q}_\perp }{(2\pi)^2} \,{\rm e}^{i \vec{q}_\perp \cdot \vec{b}} F_1^\ast(|\vec{q}_\perp|^2)\,,
%
%& = \int_0^\infty \frac{dQ}{2\pi} \,Q\,J_0(|\vec{b}| Q)\,F_1^\ast(Q^2)\,,
\end{align}
%\end{subequations}
where $F_1^\ast$ is the proton-Roper Dirac transition form factor, depicted in Figs.\,\ref{figFT}, \ref{aznauryanLFQM} and interpreted in a frame defined by $Q=({q}_\perp=(q_1,q_2),Q_3=0,Q_4=0)$.  Plainly, $Q^2 = |\vec{q}_\perp|^2$.  Defined in this way, $\rho^{pR}(|\vec{b}|)$ is a light-front-transverse charge-density with a straightforward quantum mechanical interpretation \cite{Miller:2007uy}.  %Notably,
%\begin{equation}
%\int d^2\vec{b}\,\rho^{pR}(|\vec{b}|)  = F_1^\ast(0) = 0\,.
%\end{equation}

\begin{figure}[t]
%\centerline{%
%\includegraphics[clip,width=0.35\linewidth]{Vertex.eps}}
\centerline{%
\includegraphics[width=0.72\linewidth]{F19A_V10A.pdf}}
%
\centerline{%
\includegraphics[width=0.72\linewidth]{F19B_V10B.pdf}}
\caption{\label{rhob}
%
$\rho^{pR}(|\vec{b}|)$ (upper panel) and $|\vec{b}| \rho^{pR}(|\vec{b}|) $ (lower) calculated using Eq.\,\eqref{eqrhob}: solid (black) curve -- dressed-quark core contribution, computed using the midpoint-result within the gray bands in the left panel of Fig.\,\ref{figFT}; and dashed (blue) curve -- empirical result, computed using the dashed (blue) curve therein.}
\end{figure}

Fig.\,\ref{rhob} depicts a comparison between the empirical result for $\rho^{pR}(|\vec{b}|)$ and the dressed-quark core component: the difference between these curves measures the impact of MB\,FSIs on the transition.
%
Within the domain displayed, both curves describe a dense positive center, which is explained by noting that the proton-Roper transition is dominated by the photon scattering from a positively-charged $u$-quark in the presence of a positively-charged $[ud]_{0^+}$ diquark spectator, as mentioned above.
%
Furthermore, both curves exhibit a zero at approximately 0.3\,-\,0.4\,fm, with that of the core lying at larger $|\vec{b}|$.  Thence, after each reaching a global minimum, the dressed-quark core contribution returns slowly to zero from below whereas the empirical result returns to pass through zero once more, although continuing to diminish in magnitude.

The long-range negative tail of the dressed-quark core contribution, evident in Fig.\,\ref{rhob}, reveals the increasing relevance of axial-vector diquark correlations at long range because the $d\{uu\}_{1^+}$ component is twice as strong as $u\{ud\}_{1^+}$ in the proton and Roper wave functions, and photon interactions with uncorrelated quarks dominate the transition.
%
Moreover, consistent with their role in reducing the nucleon and Roper quark-core masses, one sees that MB\,FSIs introduce significant attraction, working to screen the long negative tail of the quark-core contribution and thereby compressing the transition domain in the transverse space.  [The dominant long-range MB effect is $n\pi^+$, which generates a positive tail.] In fact, as measured by the rms transverse radius, the size of the empirical transition domain is just two-thirds of that associated with the dressed-quark core.

%
%%Constituent quark models formulated using light-front quantization (LF\,CQMs) have also been used to compute, \emph{inter alia}, nucleon-Roper transition form factors \cite{Cardarelli:1996vn, Aznauryan:2012ec}.

%%In these cases it is known that configuration mixing, i.e. Aznauryan:2016wwm

%%In these cases, a successful description is only possible after aspects of the dynamical dressing and binding effects described above are included.


%This is a prodigious task, but a ten-year international collaborative effort, drawing together experimentalists and theorists with a diverse array of skills, has arrived at a solution to the puzzle.  Namely,  the observed Roper resonance is at heart the proton's first radial excitation and consists of a well-defined dressed-quark core augmented by a meson cloud that reduces its mass by approximately 20\% and materially alters its electroproduction form factors on $Q^2 < m_p^2$, where $m_p$ is the proton's mass.%This colloquium will describe the experimental motivations and developments which enabled electroproduction data to be procured within a domain that is unambiguously the purview of strong-QCD, thereby providing a real challenge and opportunity for modern theory; and it will survey the developments in reaction models and QCD theory that have finally made it possible to reach this definitive conclusion about the nature of the Roper resonance.


%The proton is the core of the hydrogen atom, lies at the heart of every nucleus, and has never been observed to decay; but it is nevertheless a composite object, whose properties and interactions are determined by its valence-quark content: $u$ + $u$ + $d$, \emph{i.e}.\ two up ($u$) quarks and one down ($d$) quark.  So far as is now known \cite{Agashe:2014kda}, bound-states seeded by two valence-quarks do not exist; and the only two-body composites are those associated with a valence-quark and -antiquark, \emph{i.e}.\ mesons.  These features are supposed to derive from colour confinement. Suspected to emerge in QCD, confinement is an empirical reality; but there is no universally agreed theoretical understanding.

%Such observations lead one to a position from which the proton may be viewed as a Borromean bound-state, \emph{viz}.\ a system constituted from three bodies, no two of which can combine to produce an independent, asymptotic two-body bound-state.  In QCD the complete picture of the proton is more complicated, owing, in large part, to the loss of particle number conservation in quantum field theory and the concomitant frame- and scale-dependence of any Fock space expansion of the proton's wave function \cite{Dirac:1949cp, Keister:1991sb, Coester:1992cg, Brodsky:1997de}.  Notwithstanding that, the Borromean analogy provides an instructive perspective from which to consider both quantum mechanical models and continuum treatments of the nucleon bound-state problem in QCD.  It poses a crucial question: Whence binding between the valence quarks in the proton, \emph{i.e}.\ what holds the proton together?

%In numerical simulations of lattice-regularised QCD (lQCD) that use static sources to represent the proton's valence-quarks, a ``Y-junction'' flux-tube picture of nucleon structure is produced, \emph{e.g}.\ Ref.\,\cite{Bissey:2006bz, Bissey:2009gw}.  This might be viewed as originating in the three-gluon vertex, which signals the non-Abelian character of QCD and is the source of asymptotic freedom \cite{Politzer:2005kc, Gross:2005kv, Wilczek:2005az}.  Such results and notions would suggest a key role for the three-gluon vertex in nucleon structure \emph{if} they were equally valid in real-world QCD wherein light dynamical quarks are ubiquitous.  As will become evident, however, they are not; and so a different explanation of binding within the nucleon must be found.


