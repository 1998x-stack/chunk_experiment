\section{Object Skeletons}
\label{sec-object-skeleton}

3D mesh objects are processed in order to generate mean curvature skeletons which provide a medically centered skeleton representing the object's topology \cite{tagliasacchi2012mean}. As we show in this section, the skeleton data structure is used to segment the object according its topology. 

\subsection{Mean Curvature Skeletons}
\label{sec-skeleton}

Mean curvature skeletons are generated by a contraction-based curve skeleton extraction approach. As input, the 3D mesh of the object is processed in order to retrieve a regularized surface which results in a triangulated object data structure. The set of object surface points are denoted by $O={o_0, \ldots, o_{n-1}}$.
As described in \cite{tagliasacchi2012mean}, the skeleton is build based on iterative mesh contraction via a mean curvature flow approach. Several results are depicted in \autoref{fig:skeleton}.
\begin{figure}[tbh]%
\centering
\includegraphics[height=0.13\textheight]{fig/skeleton/teddy9b.png}%
\includegraphics[height=0.14\textheight]{fig/skeleton/hammer5_remeshed.png}\,%
\includegraphics[height=0.14\textheight]{fig/skeleton/spraybottle0_remeshed.png}%
\caption{The mean curvature skeleton of several objects.}%
\label{fig:skeleton}%
\end{figure}

\begin{figure*}[t]%
\centering
\includegraphics[height=0.17\textheight]{fig/segmentation/teddy-seg2.png}\,\,%
\includegraphics[height=0.17\textheight]{fig/segmentation/pliers.png}%
\includegraphics[height=0.17\textheight]{fig/segmentation/spraybottle-2.png}\,\,\,\,%
\includegraphics[height=0.17\textheight]{fig/segmentation/stanley-three.png}\,\,\,%
\includegraphics[height=0.17\textheight]{fig/segmentation/ycb-clamp-mustard2.png}%
\includegraphics[height=0.17\textheight]{fig/segmentation/ycb-banana.png}%
%\includegraphics[height=0.17\textheight]{fig/segmentation/segmentation-error.pdf}%
\caption{The segmentation of several objects together with the computed skeletons. The surface is colored according to the corresponding skeleton. Branching areas are colored in blue, endpoints result in red color, and the surface associated with connecting segments is visualized in yellow.}%
\label{fig:segmentation}%
\end{figure*}

A resulting skeleton is a graph $S = (V,E)$, in which each vertex $v \in V$ is connected to one or multiple neighbors via edges $e \in E$.
A vertex $v_i = \{s_i, P_i\}$ consist of the 3D position of the skeleton point $s_i \in R^3$ and a set of uniquely associated points on the object surface $P_i \subset O$.
Since all surface points of the object are uniquely associated with a skeleton point, the following relation holds $\sum{|P_i|} = |O|$.
An edge $e = \{v_a, v_b\}$ connects the two vertices $v_a$ and $v_b$.

\subsection{Mesh Segmentation}
\label{sec-segmentation}

The object is segmented based on the skeleton structure in order to generate segments which represent the object's topology. For further processing, each skeleton vertex $v$ is classified according to its connection to neighboring vertices:
\begin{itemize}
	\item \textbf{Branching Vertex:} Such vertices represent a branches or crossings in the skeleton structure. As expressed in \autoref{eq:BranchingVertex}, a vertex $v$ is a branching vertex if there exist more than two edges in the skeleton $S=(V,E)$ containing $v$.
	\begin{equation}
	|\{e \in E: v \in e\}|>2 \Leftrightarrow \text{v is a branching vertex}
	\label{eq:BranchingVertex}
	\end{equation}
	\item \textbf{Endpoint Vertex:} An endpoint vertex $v$ is connected to exactly one other vertex (see \autoref{eq:EndpointVertex}).
	\begin{equation}
	|\{e \in E: v \in e\}| = 1  \Leftrightarrow \text{v is an endpoint vertex}
	\label{eq:EndpointVertex}
	\end{equation}
	\item \textbf{Connecting Vertex:} A connecting vertex $v$ connects to exactly who neighbors as expressed in \autoref{eq:ConnectingVertex}.
	\begin{equation}
	|\{e \in E: v \in e\}| = 2  \Leftrightarrow \text{v is an connecting vertex}
	\label{eq:ConnectingVertex}
	\end{equation}
\end{itemize}

The mesh can now be easily segmented by analyzing the skeleton structure and grouping the skeleton vertices according to their connections. 
A segment $S_i \subset S$ is defined as follows:
\begin{equation}
\begin{split}
	\forall v \in S_i: \text{v is an connecting vertex}\,\,\wedge \\
	\forall e=\{v_a,v_b\} \in S_i: v_a, v_b \in S_i.
\end{split}
\label{eq:Segment}
\end{equation}

The resulting segments $S_i$, contain sub graphs of $S$ consisting of connecting vertices which are enclosed by branching or endpoint vertices.
%distinguish between object segments according to their skeleton structure.
%Segments $S_i \subset S$ are defined as sub sets of the skeleton graph $S$.
%\begin{itemize}
	%\item \textbf{Branching Segments:} Theses segments contain branching and/or crossings in the skeleton structure:\\ $\exists j: |\{e \in S_i: v_j \in e\}|>2$.
	%\item \textbf{Endpoint Segments:} Theses segments do not contain branches, but at least one endpoint:\\ $\exists j: |\{e \in S_i: v_j \in e\}| = 1$.
	%\item \textbf{Connecting Segments:} All vertices connect to exactly two neighbors:\\  $\forall v_j \in V: |\{e \in S_i: v_j \in e\}| = 2$.
%\end{itemize}
%
%In addition, we define a margin $t$ that ensures that all vertices on \textit{Connecting Segments} have a minimal distance to either end or branching verices in the graph. In our experiments we set $t$ to the half of the hand width.
Exemplary segmentation results are depicted in \autoref{fig:segmentation}.


