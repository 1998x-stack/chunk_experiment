\appendix
\section{Antisymmetric wavefunction reconstruction}\label{APXA}
\noindent In order to construct an antisymmetric wavefunction, we propose a specific decomposition of $\R^{dN}$ with {\it antisymmetric} local basis functions. In order to simplify the presentation, we will assume that i) the subdomains $\{\Omega_i\}_{i \in \{1,\cdots,L^{dN}\}}\in \R^{dN}$ are hypercubes of identical size and $L\in 2\N+1$, and ii) there is no overlap between the subdomains. We first define:
\begin{definition}
 We denote by $\sigma(i;p,q) \in \{1,\cdots,L^{dN}\}$ the subdomain index such that for $({\bf x}_1,\cdots,{\bf x}_p,\cdots,{\bf x}_q,\cdots,{\bf x}_N) \in \Omega_{i}$: $({\bf x}_1,\cdots,{\bf x}_q,\cdots,{\bf x}_p,\cdots,{\bf x}_N) \in \Omega_{\sigma(i;p,q)}$. The index $i$ refers to the subdomain index, and $(p,q)$ to the permutation coordinate indices.
\end{definition}
Notice that we naturally have $\sigma\big(\sigma(i;p,q);q,p\big)=i$ and that $\sigma\big(\sigma(i;p,q);q,p\big)$ is unique as there is no subdomain overlap. %A priori, the basis functions $\big\{v_{j}^{i}\big\}_{1\leq j\leq K_i}$ in $\Omega_i$ and $\big\{v_{j}^{\sigma(i;p,q)}\big\}_{1\leq j\leq K_{\sigma(i;p,q)}}$ in $\Omega_{\sigma(i;p,q)}$ are distinct.
Antisymmetry of the wavefunction would then occur if for all $i \in\{1,\cdots,L^{dN}\}$, the local basis $\big\{v^{i}_j\big\}_{1\leq j\leq K_i}$ coincides with the local bases $\big\{v^{\sigma(i;p,q)}_j\big\}_{1\leq j \leq K_{\sigma(i;p,q)}}$, for all $(p,q) \in\{1,\cdots,N\}^2$.  We define
\begin{eqnarray*}
\Sigma(i)=\big\{\sigma(i;p,q)\in \{1,\cdots,L^{dN}\}, \, \forall (p,q)\in \{1,\cdots,N\}^2\big\}
\end{eqnarray*}
If there is no overlap between subdomains, $\Sigma(i)$ is actually a singleton. In order to guarantee the antisymmetry of the overall wavefunction, at any time and any Schwarz iteration, we proceed as follows. \\
\noindent{\it Antisymmetric wavefunction}. For any $i\in \{1,\cdots,L^{dN}\}$ and for any $l \in \Sigma(i)$:
\begin{itemize}
\item we solve a (local) TDSE on $\Omega_i$ in the form
\begin{eqnarray}\label{statWF}
\psi^{(k)}_i(\cdot,t) = \sum_{j=1}^{K_i}c_j^{i,(k)}(t)v^i_{j}
\end{eqnarray}
\item and we deduce:
\begin{eqnarray}\label{statWF2}
\psi^{(k)}_{\Sigma(i)}(\cdot,t) = -\sum_{j=1}^{K_{\Sigma(i)}}c_j^{i,(k)}(t)v^{\Sigma(i)}_{j}
\end{eqnarray}
\end{itemize}
The global wavefunction is then reconstructed according to the algorithm derived in Section \ref{NAWF} without overlap. We deduce by construction, the following proposition.
\begin{prop}
The reconstructed wavefunction $\psi^{(k)}$ with \eqref{statWF} and \eqref{statWF2}, is antisymmetric.
\end{prop} 
\noindent{\bf Proof.} The proof is trivial. For any $({\bf x}_1,\cdots,{\bf x}_p,\cdots,{\bf x}_q,\cdots,{\bf x}_N) \in \Omega_i$, \\
we have $({\bf x}_1,\cdots,{\bf x}_q,\cdots,{\bf x}_p,\cdots,{\bf x}_N)\in \Omega_{\Sigma(i)}$, then 
\begin{eqnarray*}
\left.
\begin{array}{lcl}
\psi^{(k)}({\bf x}_1,\cdots,{\bf x}_p, \cdots,{\bf x}_q,\cdots, {\bf x}_N,t) & =& \psi_i^{(k)}({\bf x}_1,\cdots,{\bf x}_p, \cdots,{\bf x}_q,\cdots, {\bf x}_N,t)\\
\\
&  = & -\psi_{\Sigma(i)}^{(k)}({\bf x}_1,\cdots,{\bf x}_q, \cdots,{\bf x}_p,\cdots, {\bf x}_N,t)\\
\\
&  =&  -\psi^{(k)}({\bf x}_1,\cdots,{\bf x}_q, \cdots,{\bf x}_p,\cdots, {\bf x}_N,t).
\end{array}
\right.
\end{eqnarray*}
 $\Box$
\\
\\
{\bf Discrete local Hamiltonian construction with local Slater's determinants}. The construction of discrete local Hamiltonians $\widetilde{{\bf H}}_{i}$ is similar to the procedure described in Subsection \ref{subsec:SLO}, which is an application of \cite{CAM15-10}, in a DDM framework. Due to the antisymmetry constraint detailed above, we however need additional specifications. The strategy presented in Subsection \ref{subsec:SLO}, and extended to $d$ dimensions and $N$ particles allows for the construction for any subdomain $\Omega_i$, $1\leq i \leq L^{dN}$, of the SLO's $\big\{\varphi_j^{i}({\bf x})\big\}_{j}$. These SLO's are smooth, have compact support and possess orthogonality properties, which  are described at the end of Subsection \ref{subsec:SLO}. We here summarize the explicit construction of a local Hamiltonian $\widetilde{{\bf H}}_i$, $1 \leq i \leq L^{dN}$, say for $\Omega_i$. Following the notations used above, we need to compute:
\begin{eqnarray*}
\left.
\begin{array}{lcl}
I_{mpqr}^{ijkl} &= &\int_{\R^{dN}} \varphi^i_m({\bf x})\varphi^j_p({\bf x})\cfrac{\varphi^k_q({\bf s})\varphi^l_r({\bf s})}{|{\bf s}-{\bf x}|}d{\bf x}d{\bf s} \\
&= &\int_{\hbox{Supp}\varphi_m^i\cap \hbox{Supp}\varphi_p^j\cap \hbox{Supp}\varphi_q^k\cap \hbox{Supp}\varphi_r^l}\varphi^m_j({\bf x})\varphi^j_p({\bf x})\Phi_{qr}^{kl}({\bf x})d{\bf x}.
\end{array}
\right.
\end{eqnarray*} 
where 
\begin{eqnarray*}
\Phi_{qr}^{kl}({\bf x}) = \int_{\hbox{Supp}\varphi^k_q\cap \hbox{Supp}\varphi^l_r}\cfrac{\varphi_q^k({\bf s})\varphi_r^l({\bf s})}{|{\bf s}-{\bf x}|}d{\bf s}
\end{eqnarray*} 
can be achieved using mollifiers:
\begin{eqnarray*}
\Phi_{qr}^{kl}({\bf x}) \approx \int_{\hbox{Supp}\varphi^k_q\cap \hbox{Supp}\varphi^l_r}\varphi_q^k({\bf s})\varphi_r^l({\bf s})B_{\epsilon}({\bf s}-{\bf x})d{\bf s}
\end{eqnarray*} 
or alternatively, if $d=3$
\begin{eqnarray*}
-\triangle \Phi_{qr}^{kl} = 4\pi\varphi_q^k\varphi_r^l
\end{eqnarray*}
Then, for $m=l,l\pm 1$
\begin{eqnarray*}
I_{lm}^{ij} = \int_{\hbox{Supp}\varphi^i_l\cap \hbox{Supp}\varphi_m^j}\cfrac{1}{2}\varphi_l^i({\bf x})\triangle\varphi_m^j({\bf x})-\sum_{A=1}^P\cfrac{Z_A}{|{\bf x}-{\bf x}_A|}\varphi_l^i({\bf x})\varphi_m^j({\bf x})d{\bf x}.
\end{eqnarray*} 
Notice that for constructing  ${\bf Q}_i^{x,y,z}$, we need to compute
\begin{eqnarray*}
J^{ij}_{lm} = \int_{\hbox{Supp}\varphi^i_l\cap \hbox{Supp}\varphi_m^j}{\bf x}\varphi_l^i({\bf x})\varphi_m^{j}({\bf x})d{\bf x}%\int_{\Omega_i}\varphi^l_i({\bf s})\varphi^j_{m}({\bf s})d{\bf s}
\end{eqnarray*}
which does not present any additional difficulty compared to one-domain problems, see again \cite{CAM15-09}.  We then efficiently construct the local Hamiltonians using the strategy presented in \cite{CAM15-10}. Once the matrix local Hamiltonians are constructed, we can determine the LSD's and solve the time-independent and dependent Schr\"odinger equations.
% of Matrix ${\bf H}_{i}$ is described in \cite{CAM15-10}. Note that in the following, orbitals are indexed on top by the subdomain index, as the selection of 1-electron orbitals, is subdomain dependent, in fact layer independent Section \eqref{AWF}.  That is for any orbital indices, $(i,j,k,l) \in \{1,\cdots,K_i\}^4$, and any electron indices, $(s,t) \in \{1,\cdots,N\}^2$, we have to compute
%\begin{eqnarray}\label{i4}
%\left.
%\begin{array}{lcl}
%I^{i}_{jklm} & =& \int_{\Omega^s_i\times\Omega_i^t}\cfrac{\phi^i_j({\bf x})\phi^i_k({\bf x})\phi^i_l({\bf s})\phi^i_m({\bf s})}{|{\bf x}-{\bf s}|}d{\bf x}d{\bf s}\\
%\\
%&  = & \int_{\Omega^t_i}\phi^i_l({\bf s})\phi^i_m({\bf s})\Phi^i_{jk}({\bf s})d{\bf x}d{\bf s}
%\end{array}
%\right.
%\end{eqnarray}
%where 
%\begin{eqnarray*}
%\Phi^{i}_{jk}({\bf s}) = \int_{\Omega^i_s}\cfrac{\phi^i_j({\bf x})\phi^i_k({\bf x})}{|{\bf x}-{\bf s}|}d{\bf x}
%\end{eqnarray*}
%In order to use the same trick as \cite{CAM15-10}, we need to define the integrals over $\R^6$. This is possible by extending basis function to $\R^6-\Omega^s_i\times \Omega_i^t$; these extensions are denoted by $\big\{\widetilde{\phi}^i_j\big\}_{1\leq j \leq K_i}$, and read 
%\begin{eqnarray*}
%\widetilde{\phi}^i_j({\bf x}) = \left\{
%\begin{array}{ll}
%\widetilde{\phi}^i_j({\bf x}), & \hbox{ if } {\bf x} \in \Omega^{s,t}_i,\\
%0, &  \hbox{ if } {\bf x} \in \R^d - \Omega^{s,t}_i
%\end{array}
%\right.
%\end{eqnarray*}
%Note however, that $\big\{\widetilde{\phi}^i_j\big\}_j$ are only piecewise regular (PROBLEM...???). Now, we get in $d$-dimension:
%\begin{eqnarray*}
%\left.
%\begin{array}{lcl}
%I^{i}_{jklm} & =& \int_{\R^6}\cfrac{\widetilde{\phi}^i_j({\bf x})\widetilde{\phi}^i_k({\bf x})\widetilde{\phi}^i_l({\bf s})\widetilde{\phi}^i_m({\bf s})}{|{\bf x}-{\bf s}|}d{\bf x}d{\bf s}\\
%\\
%&  = & \int_{\R^d}\widetilde{\phi}^i_l({\bf s})\widetilde{\phi}^i_m({\bf s})\widetilde{\Phi}^i_{jk}({\bf s})d{\bf x}d{\bf s}
%\end{array}
%\right.
%\end{eqnarray*}
%where $\widetilde{\Phi}^{i}_{jk}({\bf s}) = \int_{\R^d}\cfrac{\widetilde{\phi}^i_j({\bf x})\widetilde{\phi}^i_k({\bf x})}{|{\bf x}-{\bf s}|}d{\bf x}$, so that in $\R^3$ (only), $\triangle \widetilde{\Phi}^{i}_{jk}({\bf x}) = 4\pi \widetilde{\phi}^i_j({\bf x})\widetilde{\phi}^i_k({\bf x})$, see \cite{CAM15-10}. We solve this Poisson equation using \cite{CAM15-10}, and then take the restriction to $\Omega^s_{i}$,  $\widetilde{\Phi}^{i}_{jk|\Omega^t_i} \approx \Phi^{i}_{jk}$ and we get \eqref{i4}.
%Other integrals are $d$-dimensional, such as, in ${\bf H}_{i}$ :
%\begin{eqnarray*}
%I^i_{jk} = \int_{\Omega^s_i}\cfrac{1}{2}\phi^i_j({\bf x})\triangle\phi^i_k({\bf x})-\sum_{A=1}^P\cfrac{Z_A}{|{\bf x}-{\bf x}_A|}d{\bf x}
%\end{eqnarray*} 
%as well as in ${\bf Q}^{x,y,z}$
%\begin{eqnarray*}
%J^{i}_{jk} = \int_{\Omega^s_i}{\bf x}^{x,y,z}\phi^i_j({\bf x})\phi^i_{k}({\bf x})d{\bf x}\int_{\Omega^t_i}\phi^i_l({\bf s})\phi^i_{m}({\bf s})d{\bf s}
%\end{eqnarray*}
%do not present any additional difficulty compared to 1-domain problems, see again \cite{CAM15-09}.
