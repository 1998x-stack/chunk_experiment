\subsection{Detection with known structure of covariance matrix $\boldsymbol{C}$}
MLEs of the \textit{texture} components under each hypothesis are given as in \cite{Conte_Apr02}
\begin{eqnarray}\label{texture_mle}
% \nonumber % Remove numbering (before each equation)
  H_0 \thinspace : \thinspace s_0^2 &=& \frac{1}{N} \boldsymbol{z}^H \boldsymbol{C}^{-1} \boldsymbol{z}, \\
  H_1 \thinspace : \thinspace s_0^2 &=& \frac{1}{N}
\left ( \boldsymbol{z} - \alpha \boldsymbol{p} \right  )^H \boldsymbol{C}^{-1}\left ( \boldsymbol{z} - \alpha \boldsymbol{p} \right  ).
\end{eqnarray}
Direct substitution of the MLEs
of $s_0$ into (\ref{glrt}) leads to
\begin{equation}\label{glrt_alphaCp}
\frac{\left(\boldsymbol{z}^H \boldsymbol{C}^{-1} \boldsymbol{z}\right)^N}
{\min \limits_{\alpha}
 \min \limits_{\boldsymbol{p}}
  \left \{ \left ( \boldsymbol{z} - \alpha \boldsymbol{p} \right  )^H \boldsymbol{C}^{-1}\left ( \boldsymbol{z} - \alpha \boldsymbol{p} \right  )\right\}^N
  }
 \thinspace\mathop{\gtrless}_{H_0}^{H_1}\thinspace G_1,
\end{equation}
where $G_1$ is a suitable modification of $G_2$.
We then proceed by replacing $\alpha$ with its MLE, which is \cite{Conte95}
%Now, the MLE of $\alpha$ is the one that minimizes denominator of (\ref{glrt_alphaCp}), which is
\begin{equation}\label{alpha_mle}
\alpha \thinspace = \thinspace
\frac{\boldsymbol{p}^H \boldsymbol{C}^{-1} \boldsymbol{z}}
{\boldsymbol{p}^H \boldsymbol{C}^{-1} \boldsymbol{p}}.
\end{equation}
into (\ref{glrt_alphaCp}).
After some manipulations, the likelihood ratio is recast as
\begin{equation}\label{glrt_equi}
\max \limits_{\boldsymbol{p}}
\frac{\left |\boldsymbol{z}^H \boldsymbol{C}^{-1}\boldsymbol{p} \right |^2}
     { \left (\boldsymbol{z}^H \boldsymbol{C}^{-1} \boldsymbol{z} \right )
   \left (\boldsymbol{p}^H \boldsymbol{C}^{-1} \boldsymbol{p} \right )}
   \thinspace\mathop{\gtrless}_{H_0}^{H_1}\thinspace G,
\end{equation}
where $G$ is a suitable modification of $G_1$.
It is easy to see,
by using the Schwatz's inequality,
 that $G \in [0,1]$.
 %since the left hand side of (\ref{glrt_equi}) lies
%in $[0,1]$ (easily derived by using Schwatz's inequality).
Also, the test in (\ref{glrt_equi}) does not change if
we substitute $\boldsymbol{C}$ with $\boldsymbol{M}$.
In addition,
if $\boldsymbol{p}$ is known,
(\ref{glrt_equi}) becomes the likelihood ratio
of the detector proposed
in \cite{Conte95},
\begin{equation}\label{asymp_lrt}
\frac{\left |\boldsymbol{z}^H \boldsymbol{M}^{-1}\boldsymbol{p} \right |^2}
     { \left (\boldsymbol{z}^H \boldsymbol{M}^{-1} \boldsymbol{z} \right )
   \left (\boldsymbol{p}^H \boldsymbol{M}^{-1} \boldsymbol{p} \right )}
   \thinspace\mathop{\gtrless}_{H_0}^{H_1}\thinspace G.
\end{equation}
The detector in \cite{Conte95}
(\ref{asymp_lrt}) is referred to as the normalized matched filter (NMF)
with known $\boldsymbol{M}$
and as an adaptive NMF (ANMF) with an estimated $\boldsymbol{M}$.
It is worth noting that (\ref{asymp_lrt}) was derived in \cite{Conte95}
to detect a coherent pulse trains with 
the number of pulses goes to infinity, given that
$s_0$ was a
random variable of a well-behaved distribution.

\indent
Now, we have to solve the maximization problem in (\ref{glrt_equi})
w.r.t $\boldsymbol{p}$ before proceeding to a decision on a target's presence.
% We hence restrict our attention to the case that
% $\phi \in [\theta - \beta,\theta + \beta ]$ as mentioned earlier in the problem formulation.
From the observation that the expression to be maximized
depends only on $\phi$,
we rewrite (\ref{glrt_equi}) as
\begin{equation}\label{glrt_Fphi}
\frac{1}{\boldsymbol{z}^H \boldsymbol{C}^{-1} \boldsymbol{z}}
\left [
\max \limits_{\phi \in [\theta - \beta,\theta + \beta]}F(\phi)
\right ]
   \thinspace\mathop{\gtrless}_{H_0}^{H_1}\thinspace G_2,
\end{equation}
where
$
F(\phi) \thinspace = \thinspace
                          \dfrac{\boldsymbol{p}^H \boldsymbol{C}^{-1}\boldsymbol{z}\boldsymbol{z}^H \boldsymbol{C}^{-1}\boldsymbol{p}}
                          {\boldsymbol{p}^H \boldsymbol{C}^{-1} \boldsymbol{p}}
$.
Note that $F(\phi)$ is always non-negative since
$\boldsymbol{C}^{-1}$
and
$\boldsymbol{C}^{-1}\boldsymbol{z}\boldsymbol{z}^H \boldsymbol{C}^{-1}$
 are semi-definite.
Expanding
%the expression of
$F(\phi)$ in term of $\phi$ and using
$\exp (-j k \phi) = \left (\cos{\phi} - j \sin{\phi}\right )^k$, we have
\begin{equation}\label{Fphi_intheta}
F(\phi) \thinspace = \thinspace  \frac
           {x_0 + 2 \Re \sum \limits_{k=1}^{N-1} x_k \exp(-jk\phi) }
           {y_0 + 2\Re \sum \limits_{k=1}^{N-1} y_k \exp(-jk\phi) }
\end{equation}
with
\begin{eqnarray}
% \nonumber % Remove numbering (before each equation)
    x_0 \thinspace &=& \thinspace \tr \left(\boldsymbol{C}^{-1}\boldsymbol{z}\boldsymbol{z}^{H}\boldsymbol{C}^{-1}\right), \label{x0_poly} \\
 y_0 \thinspace &=& \thinspace \tr \left(\boldsymbol{C}^{-1}\right), \\
  x_k \thinspace &=& \thinspace \sum\limits_{m-n=-k} \left(\boldsymbol{C}^{-1}\boldsymbol{z}\boldsymbol{z}^{H}\boldsymbol{C}^{-1}\right)_{nm}, \\
  y_k \thinspace &=& \thinspace \sum\limits_{m-n=-k} \left (\boldsymbol{C}^{-1}\right)_{nm},\label{yk_poly}
\end{eqnarray}
and $k= 1,2,\ldots,N-1$.
 Note that $x_0$ and $y_0$ are real since
$ \boldsymbol{C}^{-1}\boldsymbol{z}\boldsymbol{z}^{H}\boldsymbol{C}^{-1} $ and
$\boldsymbol{C}^{-1}$
are Hermitian.
We observe that
finding the maximum w.r.t $\phi$ of $F(\phi)$ in (\ref{Fphi_intheta}) is not straightforward
since the numerator and denominator are polynomials of
at most ($N-1$)-th degree in $\cos{\phi}$ and $\sin{\phi}$.
We then solve the maximization here by a numerical method.
Firstly, denote by $t$ the maximum value of $F(\phi)$, then $t$ is the
lowest upper bound of $F(\phi)$, i.e. $t$ is the solution of
the optimization problem
 \begin{equation}\label{Fphi_minimization}
\begin{array}{ll}
 \underset{t \in \mathbb{R}^{+}}{\text{minimize}} & t \\
                           \text{such that} &
                           g(\phi, t)\thinspace \triangleq \thinspace \geqslant 0, \\
 & \phi \in [\theta - \beta,\theta + \beta].
  \end{array}
\end{equation}
where
 $$
  g(\phi, t)\thinspace \triangleq \thinspace ty_0 - x_0 +
                           2\Re \left\{ \sum_{k=1}^{N-1}\left(ty_k - x_k\right)\exp(-jk\phi)\right\}
 $$

%% talking about non negative polynomial and SDP
We have another observation that
$g(\phi, t)$ is
a real non-negative trigonometric polynomial
over the interval
$[\theta - \beta,\theta + \beta]$, so
coefficients of $g(\phi, t)$ follow the following theorem \cite{Davidson02}.
%In (\ref{Fphi_minimization}), we
%notice that on the interval
%%$g(\phi)$ is a real non-negative trigonometric polynomial for which
 %concerning coefficients of
%a non-negative polynomial was derived in \cite{Davidson02}.
%Using that theorem, coefficients of $g(\phi)$ and hence $t$,
%can be computed employing a
%SDP
%\cite{Roh06}.
%That theorem is shown in the following;
%however,
%we firstly adopt the below definition
\begin{definition}
Let $
\boldsymbol{W}_{DFT} \in \mathbb{C}^{M \times M}
$
be the DFT matrix
\begin{equation*}
  \boldsymbol{W}_{DFT} = \left [ \boldsymbol{w}_0, \boldsymbol{w}_1,\ldots,\boldsymbol{w}_{M-1} \right ],
\end{equation*}
where
$
\boldsymbol{w}_k = \left[ 1, \exp(-jk2\pi/M),\ldots,\exp(-j(M-
\right. \linebreak \left.
1)k 2\pi/M)\right]^T
$.
\end{definition}
We define $\boldsymbol{W}$ and $\boldsymbol{W}_1$ as matrices composed of the first $N$ and $N-1$
columns of $\boldsymbol{W}_{DFT}$, respectively.
\begin{theorem}
Let $p(\phi)$ be a trigonometric polynomial in $\phi$ with degree (N-1) or less, and have the form
$$
p(\phi)\thinspace = \thinspace q_0 + 2\Re \sum \limits_{k =1}^{N-1} q_k \exp \left(-jk\phi \right),
 $$
 with $\boldsymbol{q} = [q_0, q_1,q_2,\ldots,q_{N-1}] \in \mathbb{R} \times \mathbb{C}^{N-1}$.
 $p(\phi)$ is non-negative on $[\theta - \beta,\theta + \beta]$ %with $0 < \beta < \pi $
 if and only if there exist
 $\boldsymbol{X}_1 \in \mathbb{H}^{N \times N}$ and $\boldsymbol{X}_2 \in \mathbb{H}^{(N-1)\times (N-1)}$ so that
 $ \boldsymbol{q} =
 \boldsymbol{W}^H \left [\diag \left ( \boldsymbol{W} \boldsymbol{X}_1 \boldsymbol{W}^H \right ) +
 \boldsymbol{d}\circ \diag \left ( \boldsymbol{W}_1 \boldsymbol{X}_2 \boldsymbol{W}_1^H \right )\right]$,
 where
 $\boldsymbol{X}_1 \succeq 0$, $\boldsymbol{X}_2 \succeq 0$,
  $\boldsymbol{d} \in \mathbb{R}^{M \times 1}$ has elements
 $d_k = \cos(2\pi k/M-\theta)-\cos\beta$ for $k = 0,1,\cdots, M-1$
 and $M \geq 2N - 1$.
 \end{theorem}
 For a detailed explanation of the idea underlying the above theorem as well as its applications,
 interest readers may refer \cite{Roh06}.
 Applying the theorem and
 denoting
 $\boldsymbol{y} = \left [y_0, y_1,\ldots,y_{N-1}\right ]^T$ and
 $\boldsymbol{x} = \left [x_0, x_1,\ldots,x_{N-1}\right]^T$, with
 $x_i$, $y_i$, $ i = 0,1,\ldots,N-1$ are computed as in
 (\ref{x0_poly})--(\ref{yk_poly}),
  we have
  $
   t \boldsymbol{y} - \boldsymbol{x}
   \thinspace  = \thinspace
   \boldsymbol{W}^H \left [\diag \left ( \boldsymbol{W} \boldsymbol{X}_1 \boldsymbol{W}^H \right ) + \\
 \boldsymbol{d}\circ \diag \left ( \boldsymbol{W}_1 \boldsymbol{X}_2 \boldsymbol{W}_1^H \right )\right],
 $


\indent The minimization (\ref{Fphi_minimization}) is now recast as a SDP
\begin{equation}\label{SDP_detector}
\begin{array}{ll}
                        \underset{t,\boldsymbol{X}_1,\boldsymbol{X}_2}{\text{minimize}} & t\\
 \text{s.t} & t\boldsymbol{y}-\boldsymbol{x} \thinspace=\thinspace
 \boldsymbol{W}^H [ \diag \left ( \boldsymbol{W} \boldsymbol{X}_1 \boldsymbol{W}^H \right ) + \\
            & \boldsymbol{d}\circ \diag \left ( \boldsymbol{W}_1 \boldsymbol{X}_2 \boldsymbol{W}_1^H \right ) \\
                            & t \in \mathbb{R}^{+} \\
                            & \boldsymbol{X}_1 \succeq 0,\thinspace \boldsymbol{X}_1 \in \mathbb{H}^{N \times N} \\
                              & \boldsymbol{X}_2 \succeq 0, \boldsymbol{X}_2 \in \mathbb{H}^{(N-1)\times (N-1)} ].
        \end{array}
\end{equation}
The SDP above can be solved efficiently using the interior point method.
%, yielding
%optimal value of (\ref{SDP_detector}), which is also
%the maximum of $F(\phi)$.
% However, no information on
%$\phi$ is given.
In passing, we note that
 this algorithm was applied in \cite{DeMaio10} to  detect a point-like target
in correlated Gaussian noise under an unknown direction of arrival.\\
\indent Denote by $t^{\ast}$ the optimal value %of $t$
attained from solving (\ref{SDP_detector}),
the likelihood ratio  test is written as follows
\begin{equation}\label{final_test}
\frac{1}
     {  \boldsymbol{z}^H \boldsymbol{C}^{-1} \boldsymbol{z} } t^{\ast}
   \thinspace\mathop{\gtrless}_{H_0}^{H_1}\thinspace G_2.
\end{equation}
\indent
