\section{Background: Resilience Design Patterns}
\label{sec:Background}

\subsection{Concept}
Design patterns identify the key aspects of a solution to common problem, and presents the solution in the form of an abstract description, which provides designers with guidelines on how to address the problem. Patterns capture the best-known techniques to solve a problem. We developed resilience design patterns \cite{Hukerikar:2017} to support a systematic approach to designing and implementing new resilience solutions and adapting existing solutions to future extreme-scale architectures and software environments. 

The patterns describe the design decisions and trade-offs that must be considered when applying a pattern solution to a specific context. The descriptions encourage designers to reason about the impact of applying a solution on a system's performance scalability and power consumption overhead as well as consider implementation issues. Based on the patterns, we developed a framework that enables designers to comprehensively evaluate the scope of protection domain and the handling efficiency of resilience solutions.

The basic template of a resilience design pattern is defined in an event-driven paradigm, in which each resilience design pattern consists of a {\em behavior} and a set of {\em activation} and {\em response interfaces}. The patterns present solutions to specific problems in detecting, recovering from, or masking a fault, error or failure event. The pattern descriptions are abstract and they may be implemented by HPC applications' algorithms, numerical libraries, system software, or even in the hardware architectures. We have organized the resilience design pattern as a catalog that contains detailed descriptions of the patterns. The catalog is available as a specification document \cite{RDP:Spec}, in which each resilience pattern is presented using a structured format to enable designers to quickly discover whether the pattern solution is suitable to the problem being solved. 

\subsection{Classification}
We developed a pattern classification scheme that organizes the resilience patterns in a layered hierarchy, in which each level addresses a specific aspect of the problem. The classification enables designers to separately reason about the patterns that define the scope of the protection domain and those that define the semantics of the detection, containment and mitigation. The hierarchical organization of the patterns permits system architects to work on the overall organization of the solutions by analyzing the integration of various resilience patterns across the system stack while designers of individual hardware and software components can focus on implementation of the patterns.  

Resilience in the context of HPC systems and its applications has two key dimensions: (1) forward progress of the system; (2) data consistency in the system. Based on these factors, we organize the resilience design patterns into two major categories, \textbf{state} patterns and \textbf{behavioral} patterns.
The behavioral patterns identify detection, containment, or mitigation actions that enable a system to cope with the presence of a fault, error, or failure event. These patterns are organized hierarchically and they include \textbf{strategy}, \textbf{architectural} and \textbf{structural} patterns. 

The strategy patterns define high-level polices of a resilience solution. Their descriptions are deliberately abstract to enable hardware and software architects to reason about the overall organization of the techniques used and their implications on the full system design. These patterns describe the overall structure of the solution and the key attributes of the solution and their capabilities independent of the layer of system stack and hardware/software architectural features. The architectural patterns convey specific methods necessary for the construction of a resilience solution. They explicitly convey the type of fault, error or failure event that they handle and provide detail about the key components and connectors that make up the solution. The structural patterns provide concrete descriptions of the solution rather than high-level strategies. They comprise of instructions that may be implemented in hardware/software components. While the strategy and architectural patterns serve to provide designers with a clear overall framework of a solution and the type of events that it can handle, the structural patterns express the details so they can contribute to the development of complete working solutions.

\subsection{Designing Resilience Solutions using Patterns}
Each pattern in the resilience design pattern catalog presents a solution to a specific problem in detecting, containing or mitigating a fault, error or failure event. In order to construct complete resilience solutions designers must identify patterns that provide each of these capabilities and apply them to a well-defined protection domain. Therefore, a complete solution consists of at least one state pattern (defining scope
of the protection domain), and one or more behavioral patterns (supporting a combination of
detection, containment and mitigation solutions).

For hardware and software designers to make practical use these patterns in the development of resilient versions of their designs, we have developed a design framework that a set of guidelines are necessary to combine the patterns and refine their implementations. The framework is based on design spaces that are arranged
in a hierarchy. By working through the design spaces, designers can convert initial outline of the resilience solution into a concrete implementation by considering the layer of abstraction for the pattern implementation, scalability of the solution, portability to other architectures, dependencies on any hardware/software features, flexibility to adapt the solution to accelerated fault rates, capability to handle other types of fault and error events, the performance and performance overheads. 

