\documentclass{article}
\usepackage{amsmath, amssymb}
\usepackage{amssymb,pb-diagram,pst-all,graphicx}
\usepackage{amscd}

\usepackage{fancybox}
%\usepackage{eclbkbox}
%\usepackage{jsaiac2r}
%%\usepackage{KBSv2r}
%\usepackage{latexsym}
%\usepackage{ndefinition}
%\input def.tex
%%%%%%%%%%%%%%%%%%%%%%%%%%%%%%%%%%%%%%%%
\DeclareMathOperator\irr{Irr}
%\DeclareMathOperator\Sing{Sing}
\DeclareMathOperator\ltop{\sigma_\text{top}}
\DeclareMathOperator\STop{\Sigma_\text{top}}

\def\oo{{\cal o}}
\newcommand{\RR}{\mathbb R}
\newcommand{\FF}{\mathbb F}
\newcommand{\Id}{\mathop {\rm Id}\nolimits}
\newcommand{\ZZ}{\mathbb Z}
\newcommand{\PP}{\mathbb P}
\newcommand{\QQ}{\mathbb Q}
\newcommand{\CC}{\mathbb C}
\newcommand{\NN}{\mathbb N}
\newcommand{\mcB}{\mathcal B}
\newcommand{\mcC}{\mathcal C}
\newcommand{\mcF}{\mathcal F}
\newcommand{\mcL}{\mathcal L}
\newcommand{\mcI}{\mathcal I}
\newcommand{\DD}{\mathbbl D}
\newcommand{\mcD}{\mathcal D}
\newcommand{\mcQ}{\mathcal Q}
\newcommand{\mcT}{\mathcal T}
\newcommand{\EE}{\mathbb E}
\newcommand{\calE}{{\mathcal E}}
\newcommand{\TPL}{\mathcal {TPL}}
\newcommand{\calG}{\mathcal {G}}
\newcommand{\bAlex}{\mathbf{Alex}}
\newcommand{\ucurve}{\underline{\mathrm {Curve}}}
\newcommand{\bCov}{\mathop {\rm \underline{Cov}}\nolimits}
\newcommand{\Sub}{\underline {\mathrm {Sub}}}
\newcommand{\redu}{{\mathrm {red}}}
%\newcommand{\bCov}{\mathbf{Cov}}
%\newcommand{\bCov}{\mathop {\rm \underline{Cov}}\nolimits}
\newcommand{\balpha}{\boldsymbol{\alpha}}
\newcommand{\bbA}{\mathbb A}
\newcommand{\bde}{\boldsymbol {e}}
\newcommand{\jimei}{\boldsymbol {1}}
\newcommand{\caS}{{\cal S}}
\newcommand{\gtS}{{\mathfrak S}}
\newcommand{\gtif}{{\mathfrak f}}
\newcommand{\tor}{{\mathrm {tor}}}
\newcommand{\hor}{{\mathrm {hor}}}
\newcommand{\ver}{{\mathrm {ver}}}
\newcommand{\lex}{{\mathrm {lex}}}
%\newcommand{\red}{{\mathrm {red}}}
\newcommand{\LL}{\mathop {\bf L}\nolimits}
\newcommand{\bA}{\mathop {\bf A}\nolimits}
\newcommand{\bB}{\mathop {\bf B}\nolimits}
\newcommand{\bD}{\mathop {\bf D}\nolimits}
\newcommand{\bE}{\mathop {\bf E}\nolimits}
\newcommand{\boldc}{\boldsymbol {c}}
\newcommand{\boldzero}{\boldsymbol {0}}
\newcommand{\gtA}{{\mathfrak A}}
\newcommand{\gtf}{{\mathfrak f}}
\newcommand{\gtd}{{\mathfrak d}}
\newcommand{\Cov}{\mathop {\rm Cov}\nolimits}
\newcommand{\Coker}{\mathop {\rm Coker}\nolimits}
\newcommand{\MW}{\mathop {\rm MW}\nolimits}
\newcommand{\Gal}{\mathop {\rm Gal}\nolimits}
\newcommand{\SL}{\mathop {\rm SL}\nolimits}
\newcommand{\GL}{\mathop {\rm GL}\nolimits}
\newcommand{\Hom}{\mathop {\rm Hom}\nolimits}
\newcommand{\PGL}{\mathop {\rm PGL}\nolimits}
\newcommand{\NS}{\mathop {\rm NS}\nolimits}
\newcommand{\Div}{\mathop {\rm Div}\nolimits}
\newcommand{\Fix}{\mathop {\rm Fix}\nolimits}
\newcommand{\rank}{\mathop {\rm rank}\nolimits}
\newcommand{\Red}{\mathop {\rm Red}\nolimits}
\newcommand{\Pic}{\mathop {\rm Pic}\nolimits}
\newcommand{\Proj}{\mathop {\rm Proj}\nolimits}
\newcommand{\Supp}{\mathop {\rm Supp}\nolimits}
\newcommand{\Sing}{\mathop {\rm Sing}\nolimits}
\def\hiya#1{\mathrel{\mathop{\leftarrow}\limits^#1}}
\def\miya#1{\mathrel{\mathop{\rightarrow}\limits^#1}}
\def\XS#1{X_{S_4}^#1}
\def\MS#1{M_{S_4}^#1}
\newcommand{\scS}{\mathop {\cal S}\nolimits}
\newcommand{\scA}{\mathop {\cal A}\nolimits}
\newcommand{\mcE}{\mathop {\cal E}\nolimits}
\newcommand{\OO}{\mathop {\textbf O}\nolimits}
%\newcommand{\mcC}{\mathcal C}
\newcommand{\mcO}{\mathop {\mathcal O}\nolimits}
\newcommand{\scC}{\mathop {\cal C}\nolimits}
\newcommand{\scQ}{\mathop {\cal Q}\nolimits}
\newcommand{\disc}{\mathop {\rm disc}\nolimits}
\newcommand{\codim}{\mathop {\rm codim}\nolimits}
\newcommand{\Singular}{\mathop {\rm Sing}\nolimits}
\newcommand{\mult}{\mathop {\rm mult}\nolimits}
\newcommand{\Aut}{\mathop {\rm {Aut}}\nolimits}
\newcommand{\Bir}{\mathop {\rm {Bir}}\nolimits}
\newcommand{\Cr}{\mathop{\rm {Cr}}\nolimits}
\newcommand{\Ind}{\mathop{\rm {Ind}}\nolimits}
\newcommand{\essd}{\mathop {{\rm ed}}\nolimits}
\newcommand{\dom}{\mathop {{\rm dom}}\nolimits}
\newcommand{\id}{\mathop {{\rm id}}\nolimits}
\def\tpi {\tilde {\pi}}
\def\tbeta {\tilde {\beta}}
\def\tf{\tilde f}
\def\tS{\tilde S}
\def\tC{\tilde C}
\def\tq{\tilde q}
\def\mugep{\infty^{+}}
\def\mugem{\infty^{-}}
\def\Tva{T_{\varphi_x}}
\def\trho {\tilde {\rho}}
\def\tvarpi{\tilde {\varpi}}
%\the section. \arabic {thm}
\newtheorem{thm}{Theorem}[section]
\newtheorem{cor}{Corollary}[section]
\newtheorem{prop}{Proposition}[section]
\newtheorem{lem}{Lemma}[section]
\newtheorem{defin}{Definition}[section]
\newtheorem{exmple}{Example}[section]
\newtheorem{rem}{Remark}[section]
\newtheorem{qz}{Question}[section]
\newtheorem{prbm}{Problem}[section]
\newtheorem{mthm}{Theorem}
%\renewcommand{\mthm}{\arabic{mthm}}
\newtheorem{mcor}[mthm]{Corollary}
\newtheorem{Exmple}[mthm]{Example}
\newtheorem{Cor}[mthm]{Corollary}
\newtheorem{Rem}[mthm]{Remark}
\newtheorem{Prop}[mthm]{Proposition}
\newtheorem{Definition}[mthm]{Definition}

\newcommand{\I}{\mathop {\rm I}\nolimits}
\newcommand{\II}{\mathop {\rm II}\nolimits}
\newcommand{\III}{\mathop {\rm III}\nolimits}
\newcommand{\IV}{\mathop {\rm IV}\nolimits}
%\newenvironment{theorem}{\begin{thm}\rm}{\end{thm}}
%\newenvironment{corollary}{\begin{cor}\rm }{\end{cor}}
%\newenvironment{proposition}{\begin{prop}\rm }{\end{prop}}
%\newenvironment{lemma}{\begin{lem}\rm }{\end{lem}}
%\newenvironment{definition}{\begin{defin}\rm }{\end{defin}}
\newenvironment{example}{\begin{exmple}\rm }{\end{exmple}}
\newenvironment{remark}{\begin{rem}\rm }{\end{rem}}
\newenvironment{quiz}{\begin{qz}\rm }{\end{qz}}
%\newenvironment{prbm}{\begin{prbm}\rm }{\end{prbm}}
\newcommand{\qed}{\hfill $\Box$}

\newcommand{\proof}{\noindent{\textsl {Proof}.}\hskip 3pt}
\newcommand{\proofend}{\qed \par\smallskip\noindent}

\newcommand{\afterindexspace}{\hskip 3pt}
\newcommand{\theoremnamespace}[1]{\hskip -3pt [{\bf #1}] \hskip 3pt}

\renewcommand{\thesubparagraph}{\theparagraph.\@arabic\c@subparagraph}
\addtolength{\oddsidemargin}{-10mm}
%
\addtolength{\textwidth}{20mm}

%
 
  \begin{document}
  
  \begin{center}
%{\Large \bf Geometrically good sections of elliptic surfaces \\
%and \\
%Zariski pairs for cubic-line, cubic-conic-line arrangements
%}
{\Large \bf Rational points of elliptic surfaces 
and 
Zariski $N$-ples for cubic-line, cubic-conic-line arrangements
%Zariski pairs 
%for cubic-line, cubic-conic-line arrangements
}

\bigskip



\bigskip
{\bf 
Shinzo BANNAI, Hiro-o TOKUNAGA and  Momoko YAMAMOTO
%\\
%and \\
%Emiko YORISAKI
}
%\footnote{Research partly supported by the research grant 22540052
%from JSPS}

\end{center}
\normalsize

\begin{abstract}
In this paper, we continue the study of the relation between rational points of rational elliptic surfaces and plane curves. As an application, we give first examples of Zariski pairs of cubic-line arrangements that do not involve inflectional tangent lines.
\end{abstract}



\section*{Introduction}


  In this article, we study the arithmetic of  rational points of certain rational elliptic surfaces from a geometric point of view 
  in order to construct arrangements of plane curves of low degree which give  rise to candidates for Zariski pairs.
  A pair of reduced plane curves
  $(\mcB^1, \mcB^2)$ is said to be a {\it Zariski pair}
 if it satisfies the following conditions:

\begin{enumerate}

\item[(i)] For each $i$, there exists a tubular neighborhood $T(\mcB^i)$ of $\mcB^i$ such
that $(T(\mcB^1), \mcB^1)$ is homeomorphic to $(T(\mcB^2), \mcB^2)$.


\item[(ii)] There exists no homeomorphism from 
$(\PP^2, \mcB^1)$ to $(\PP^2, \mcB^2)$.
\end{enumerate}

The first condition can be replaced by {\it the combinatorics} (or {the combinatorial type}) of $\mcB^i$. For the precise definition
of the combinatorics,  see \cite{act} (It can also be found in \cite{tokunaga14}).  Since the combinatorics is easier to
treat with, we always consider that of $\mcB^i$.
The study of Zariski pairs was originated by Zariski in \cite{zariski29}. Since the '90's there have been
a lot of results on Zariski pairs by many mathematicians via various methods (see the reference \cite{act},
for example).
As we noted in \cite{act}, there are two main ingredients in the study of Zariski pairs.
Namely,
\begin{enumerate}

\item[(I)] To find reduced curves $\mcB^1$, $\mcB^2$ with the same combinatorics so that $\mcB^1$, $\mcB^2$ have
certain different features, and

\item[(II)] To prove $(\PP^2, \mcB^1)$ is not homeomorphic to $(\PP^2, \mcB^2)$ based on the different feature as above.

\end{enumerate}

A Zariski $N$-ple is a natural generalization, where the number of curves is increased.
One of our new feature of this article concerns   (I): Construction of plane curves via geometry and arithmetic 
of sections for certain rational elliptic surfaces. This basic idea can be found in
\cite{tokunaga14} by the first autor  and in \cite{bannai-tokunaga}  by the first and second authors. 
In this article, however,
we make use of the arithmetic of sections more intensively than previous papers.

For (II), in order to distinguish $(\PP^2, \mcB^1)$ and $(\PP^2, \mcB^2)$, our tool  is Galois covers
 branched along $\mcB^i$  developed in \cite{bannai-tokunaga, tokunaga14}.  
Note that there are
various other tools, for example, 
the fundamental group $\pi_1(\PP^2\setminus\mcB^i, \ast)$, braid monodromy and Alexander invariants.
Recently two more new tools, the linking set and the connected number,  are introduced
by J.-B. Meilhan, B.~Guerville-Ball\'{e} \cite{benoit-jb} and T.~Shirane \cite{shirane16}, respectively.

 We explain our object more concretely.
  The first and second authors have studied Zariski pairs (or $N$-ples) for arrangements of curves whose
 irreducible components are of low degree, less than or equal to 4, via geometry of sections and bisections of
 rational elliptic surfaces (\cite{bannai-tokunaga, tokunaga14}).  In this article, we also continue to study such objects. More precisely, we consider
 a reducible curve whose irreducible components consist of 
 
 \begin{enumerate}
 \item[(i)] one irreducible cubic and lines, and
 
 \item[(ii)] one irreducible cubic, smooth conics and lines.
 \end{enumerate}
 %We use the notations and terminologies in \cite{bannai-tokunaga, tokunaga14}, freely.
 

In \cite{artal94}, a Zariski pair of sextics consisting of 
 a smooth cubic and its three inflectional tangents was given. This example
was also considered in \cite{tokunaga96} from a different approach. In \cite{benoit-jb}, a Zariski
pair of octics consisting of a smooth cubic and its $5$ inflectional tangents is given. In \cite{bbst},  
that of reducible curves consisting of 
a smooth cubic and  $k$ of its inflectional tangents ($k = 4, 5, 6$) are considered. Note that
there exist no Zariski pair of a smooth cubic and $k$ of its inflectional tangents for $k = 1, 2, 7, 8, 9$.
Also, in \cite{at00}, E.~Artal Bartolo and the second author studied a Zariski pair of sextics whose
irreducible components are a nodal cubic, a smooth conic and an inflectional tangent line.
All of these examples  contain inflectional tangents of a cubic as irreducible components.
As for another new feature of this article,  we focus on  Zariski pairs for cubic-line or cubic-conic-line arrangements 
{\it without inflectional tangents}. Also, since Zariski pairs for sextic curves have already
been intensively studied by many mathematicians, e.g., \cite{oka05, oka06, shimada}, we consider
the case of degree $7$ as follows:


%\medskip

{\bf Combinatorics 1.} Let $\mcE$, $\mcL_o$ and $\mcL_i$ ($i = 1, 2, 3$) be as below and
we put $\mcB = \mcE  + \mcL_o + \sum_{i=1}^3 \mcL_i$:

\begin{enumerate}

\item[(i)] $\mcE$:  a smooth or nodal cubic curve.

\item[(ii)] $\mcL_o$: a transversal line to $\mcE$ and we put $\mcE \cap \mcL_o = \{p_1, p_2, p_3\}$.

\item[(iii)] $\mcL_i$ ($i = 1, 2, 3$): a line through $p_i$ and tangent to  $\mcE$ at a different point from $p_i$.
We denote the tangent point of $\mcL_i$ by $r_i$.

\item[(iv)] $\mcL_1, \mcL_2$ and $\mcL_3$ are not concurrent.

\end{enumerate}



%\medskip


{\bf Combinatorics 2.} Let  $\mcE$, $\mcC$, $\mcL_o$ and $\mcL$ as below and
we put $\mcB = \mcE + \mcC +  \mcL_o + \mcL$\,:

\begin{enumerate}

\item[(i)] $\mcE$:  a nodal cubic curve.

\item[(ii)] $\mcL_o$: a transversal line to $\mcE$ and we put $\mcE \cap  \mcL_o = \{p_1, p_2, p_3\}$.

\item[(iii)] $\mcL_i$: a  line connecting the node of  $\mcE$  and one of $p_i, (i = 1, 2, 3)$.

\item[(iv)] $\mcC$: a contact conic  to $\mcE + \mcL_o$ and intersecting $\mcL$ transversely.

\end{enumerate}


 Here
 we call a smooth conic $\mcC$ a contact conic to a reduced plane curve $\mcB$ if the following condition is satisfied:
$(\ast)$ Let $I_x(\mcC, \mcB)$ denotes the intersection multiplicity at $x \in \mcC \cap \mcB$.
For $\forall x \in \mcC \cap \mcB$, $I_x(\mcC, \mcB)$ is even and $\mcB$ is smooth at $x$. 
%\end{enumerate}




In both combinatorics, no inflectional tangents are involved and this is the new feature compared to 
previous examples. 
Let us explain more precisely. In the following, we use
the  notation and terminology  used in \cite{bannai-tokunaga}.

% As we noted in \cite{act}, there are two main ingredients in the study of Zariski pairs.
%Namely,
%\begin{enumerate}
%
%\item[(I)] To find reduced curves $\mcB^1, \mcB^2$ with the same combinatorics so that $\mcB^1$ and $\mcB^2$ have
%certain different feature, and
%
%\item[(II)] To prove $(\PP^2, \mcB^1)$ is not homeomorphic to $(\PP^2, \mcB^2)$ based on the different feature as above.
%
%\end{enumerate}
%
%In order to distinguish $(\PP^2, \mcB^1)$ and $(\PP^2, \mcB^1)$, various tools have been used, for example, 
%the fundamental group $\pi_1(\PP^2\setminus\mcB^i, \ast)$, braid monodromy,  Alexander invariants and Galois coves
%branched along $\mcB^i$.  Recently two more new tools, the linking set and the component number,  are introduced
%by J.-B. Meilhan, B.~Guerville Ball\'{e} \cite{benoit-jb} and T.~Shirane \cite{shirane16}, respectively. 


Put $\mcQ = \mcL_o + \mcE$ and choose a smooth point $z_o \in \mcE$. 
Consider the minimal resolution $S_{\mcQ}$ of a double cover of $\PP^2$ branched along $\mcQ$ and blow up $S_{\mcQ}$  twice at the 
preimage of $z_o$. Then we obtain a rational elliptic surface 
$\varphi_{\mcQ, z_o} : S_{\mcQ, z_o} \to \PP^1$ and its generic fiber is denoted by $E_{\mcQ, z_o}$(see \S 1 for a more precise description). 
 We denote the induced generically $2$-to-$1$ morphism from $S_{\mcQ, z_o}$ to $\PP^2$ by $f_{\mcQ, z_o}$. Let  $E_{\mcQ,z_o}$ be
 the generic fiber of $\varphi_{\mcQ, z_o} : S_{\mcQ, z_o} \to \PP^1$.  The group of sections of $\varphi_{\mcQ, z_o}$, 
 $\MW(S_{\mcQ, z_o})$,  can be canonically  identified with the group of $\CC(t)$-rational points $E_{\mcQ, z_o}(\CC(t))$.
  For $s \in \MW(S_{\mcQ, z_o})$, we denote the corresponding rational  point by $P_s$. Conversely, for $P \in E_{\mcQ, z_o}(\CC(t))$,
 we denote the corresponding section by $s_P$. Now, since $s \in \MW(S_{\mcQ, z_o})$ can be considered a curve,
% on  $S_{\mcQ, z_o}$ meeting any fiber of $\varphi_{\mcQ, z_o}$ at one point, 
$f_{\mcQ, z_o}(s)$ gives rise to a plane curve in $\PP^2$. In our construction of plane curves with Combinatorics 1 and 2,
we make use of lines and conics of the form $f_{\mcQ, z_o}(s)$ for some $s \in \MW(S_{\mcQ, z_o})$. In our particular cases,  it can be
explained more explicitly as follows (We use the notation in 
\cite{oguiso-shioda} in order to describe the structure of $E_{\mcQ, z_o}(\CC(t))$):
%\begin{enumerate}
%
% \item[(i)] Take the minimal resolution $S_{\mcQ}$ of a double cover of $\PP^2$ branched along $\mcQ$.
% 
% \item[(ii)] The pencil of lines through $z_o$ induces a pencil of curves of genus $1$, $\Lambda_{z_o}$ on $S_{\mcQ}$.  By resolving
% the base points $\Lambda_{z_o}$, we obtain $S_{\mcQ, z_o}$ and $\varphi_{z_o}$ is the morphism induced by  $\Lambda_{z_o}$.
% 
% \end{enumerate}
\medskip

{\bf Combinatorics 1.}
(a) $\mcE$: a smooth cubic.  If we choose $z_o \in \mcE \setminus \{p_1, p_2, p_3\}$, 
$E_{\mcQ, z_o}(\CC(t))$ $ \cong D_4^* \oplus \ZZ/2\ZZ$.  As we see in \S 3 , 
we choose generators  $P_0, P_1, P_2, P_3$ for the $D_4^*$-part suitably and the $2$-torsion $P_{\tau}$.    We also 
put $P_4 := P_2 \dot{-}P_3 \dot{+}P_{\tau}$, where $\dot{+}, \dot{-}$ denote the addition and subtraction on $E_{\mcQ, z_o}(\CC(t))$.
 Let $s_{P_i}$ be the sections corresponding to $P_i$ ($i = 1, \ldots, 4$),
respectively.
% and let $t$ be the $2$-torsion section corresponding to $P_{\tau}$. 
Then $\mcL_i := f_{\mcQ, z_o}(s_{P_i})$, $(i = 1, \ldots, 4)$ are
tangent lines to $\mcE$ such that each of them passes through $\mcE \cap \mcL_o$.  If $p_i$  $(i = 1, 2, 3)$ are not 
inflection points and any three of them are not concurrent, then
both $\mcB^1:= \mcQ + \sum_{i=1}^3 \mcL_i$ and 
$\mcB^2:= \mcQ + \sum_{i=2}^4 \mcL_i$ have Combinatorics 1-(a).

%\medskip

(b) $\mcE$: a nodal cubic.   If we choose $z_o \in \mcE \setminus \{p_1, p_2, p_3\}$,
$E_{\mcQ, z_o}(\CC(t)) \cong (A_1^*)^{\oplus 3} \oplus \ZZ/2\ZZ$. As we see in \S 3, 
we choose generators  $P_1, P_2, P_3$  of $E_{\mcQ, z_o}$ so that the Gram matrix of
$P_i$ $(i = 1, 2, 3)$ is $[ \langle P_i, P_j \rangle ] = 1/2\delta_{ij}$  and the $2$-torsion $P_{\tau}$.  We put
\[
\begin{array}{ccc}
P_4  :=   P_3 \dot{+} P_1 \dot{+} P_\tau & &P_5  :=  P_1 \dot{+} P_2 \dot{+} P_\tau \\
P_6  :=  P_2 \dot{+} P_3 \dot{+} P_\tau  &  &P_7  :=  P_3 \dot{-} P_1 \dot{+} P_\tau .
\end{array}
\]
%\begin{eqnarray*}
%P_4 & :=  & P_3 \dot{+} P_1 \dot{+} P_\tau \\
%P_5 & := & P_1 \dot{+} P_2 \dot{+} P_\tau \\
%P_6 & := & P_2 \dot{+} P_3 \dot{+} P_\tau \\
%P_7 & := & P_3 \dot{-} P_1 \dot{+} P_\tau
%\end{eqnarray*}
Then we infer that $\mcL_i:= f_{\mcQ, z_o}(s_{P_i})\, (i = 4, 5, 6, 7)$ are tangent lines to $\mcE$ such that each of them passes
 through $\mcE \cap \mcL_o$.  If any three of them are not concurrent, then
both $\mcB^1:= \mcQ + \sum_{i=4, 5, 6} \mcL_i$ and 
$\mcB^2:= \mcQ + \sum_{i= 5, 6, 7} \mcL_i$ have Combinatorics 1-(b).

 


{\bf Combinatorics 2.} We keep our notation in  Combinatorics 1.
Our construction is similar to that in \cite{tokunaga14}.  We first note that $\mcL_i := f_{\mcQ, z_o}(s_{P_i})$ $(i = 1, 2, 3)$ are lines
connecting the node of $\mcE$ and $p_i$.  Plane curves given by  $f_{\mcQ, z_o}(s_{[2]P_i})$  $(i = 1, 2, 3)$ are
contact conics by a similar argument to that in \cite[p. 633]{tokunaga14}.
Suppose that $\mcC_i$ meets $\mcL_j$  transversely for any $i, j$. 
Then $\mcB^1 := \mcQ + \mcL_i +  \mcC_i $ and $\mcB^2:= \mcQ + \mcL_i +  \mcC_j$ ($i \neq j$) have
Combinatorics 2.

%Namely consider bisections $C^+$ on $S_{\mcQ, z_o}$ 
% such that (i) $\mcC:= f_{\mcQ, z_o}(C^+)$ is
%a contact conic to $\mcQ$ and (ii) $P_{C^+} = [2]P_1$. 

\medskip

Now our statement is 

 
\begin{thm}\label{thm:comb-1}{Let $\mcB^1$ and $\mcB^2$ be as above. Then $(\mcB^1, \mcB^2)$ is a Zariski pair.}
\end{thm}


%We also give  a Zariski triple and $4$-ple based on rational points given above.
The organization of this paper as follows:
In \S 1, we give a summary on  facts and previous results, which we need later.  
We prove Theorem~\ref{thm:comb-1} in \S 2. A Zariski triple and 4-ple for
cubic-conic-line or cubic-conic arrangements are considered in \S 3 and explicit examples are given in \S 4.

%We consider good sections (or rational points), which we use
%our construction  as above in \S 2.
\medskip



{\bf Acknowledgements.} The second author is partially supported by Grant-in-Aid for Scientific Research C (17K05205).


\input section-1b.tex

%\input section-2.tex

\input section-3b.tex


\input section-4.tex

\input section-5c.tex
%Note that the notion  of Zariski pairs are generalized to Zariski $N$-plet.  
%(Zariski N???????????????)




%
%This example is generalized to a smooth cubic and its $k$-inflectional tangents ($k = 4, 5, 6$)
%in \cite{benoit-jb, bbst}. 

     \begin{thebibliography}{99}
\bibitem{artal94}
E.~Artal~Bartolo:
\newblock \emph{Sur les couples des {Zariski}},
\newblock {\rm J.\ Algebraic Geometry}, {\bf 3} (1994) no. {\bf 2}, 223--247

   \bibitem{act} E.~Artal Bartolo, J.-I.~Cogolludo and H.~Tokunaga:
   \emph{A survey on Zariski pairs}, Adv. Stud. Pure Math., \textbf{50} (2008), 1-100.
   
   \bibitem{at00} E.~Artal Bartolo and H.~Tokuanga: \emph{Zariski pairs of index 19 and the Mordell-Weil groups 
 K3 surfaces}, Proc. London Math. Soc. {\bf 80} (2000), 127-144.
 
 \bibitem{bbst} S.~Bannai, B.~Guerville-Ball\'e, T.~Shirane and H.~Tokunaga:
  \emph{On the topology of arrangements of a cubic and its inflectional tangents}, 
  Proc. Japan Acad. Ser. A Math. Sci. {\bf 93} (2017), 50-53.
  %
  %
  %
%  
%  \bibitem{bannai16}  S.~Bannai: \emph{A note on splitting curves of plane quartics and multi-sections of rational elliptic surfaces}, Topology and its Applications {\bf 202} (2016), 428-439.
%
%   
%   \bibitem{bgst} S.~Bannai, B.Guerville-Ball\'{e}, T.~Shirane and H.~Tokunaga:
%  \emph{On the topology of arrangements of a cubic and its inflectional tangents}, arXiv:1607.07618.
%  
%  \bibitem{bannai-shirane16} S.~Bannai and T.~Shirane: \emph{Nodal curves with a contact-conic and 
%  Zariski pairs}, arXiv:1608.03760.
%  
 \bibitem{bannai-tokunaga} S.~Bannai and H.~Tokunaga:  \emph{Geometry of  bisections of elliptic surfaces  and 
 Zariski $N$-plets for conic arrangements}, Geom.  Dedicata
 {\bf  178} (2015),  219-237,  DOI 10.1007/s10711-015-0054-z. 

% \bibitem{bannai-tokunaga17} S.~Bannai and H.~Tokunaga:  \emph{Geometry of  bisections of elliptic surfaces  and 
% Zariski $N$-plets II}, Preprint 2017.
 
 
%\bibitem{bannai-kawashima-tokunaga} Bannai, S. ,  Kawashima, M. and Tokunaga, H.: \emph{On the topology of the complements of reducible plane curves via Galois covers}, arXiv:1304.0536
 
 \bibitem{bpv} W.~Barth, K.~Hulek, C.A.M.~Peters and A. Van de Ven: Compact complex surfaces, 
 Ergebnisse der Mathematik und ihrer Grenzgebiete {\bf 4} 2nd Enlarged Edition, Springer-Verlag (2004).
% 
%\bibitem{CLO} D.~Cox, J.~Little, D.~O'Shea: Ideals, varieties and algorithms. An introduction to computational algebraic geometry and commutative algebra. 3rd edition, Springer-Verlag (2007).

%\bibitem{Degtyarev-alex}
%A.~I. Degtyarev.
%\newblock \emph{Alexander polynomial of a curve of degree six},
%\newblock {\rm J. Knot Theory Ramifications}, {\bf 3} (1994), 439--454.

%\bibitem{Eisenbud-Neumann} D.~Eisenbud and W. Neumann: Three-dimensional link theory and invariants of plane
%curve singularities, Ann. of Math. Stud. {\bf 110}, Princeton Univ. Press (1985).

%\bibitem{Esnault}
%H.~Esnault:
%\newblock \emph{Fibre de {M}ilnor d'un c\^one sur une courbe plane singuli\`ere},
%\newblock {\rm Invent. Math.}, {\bf 68} (1982) no. {\bf 3}, 477--496.

%\bibitem{fulton} W.~Fulton: Intersection Theory, Springer-Verlag (1984).
%

\bibitem{benoit-jb} B.~Guerville-Ball\'e and J.-B. Meilhan: \emph{A linking invariant for algebraic curves}, Available at \texttt{arXiv:1602.04916}.

 \bibitem{horikawa} E.~Horikawa: \emph{ On deformations of quintic surfaces},
\rm Invent. Math. {\bf 31} (1975), \rm $43 - 85$.
%
%\bibitem{Kawashima2}
%M.~Kawashima and M.~Oka:
%\newblock \emph{On {A}lexander polynomials of certain {$(2,5)$} torus curves.}
%\newblock {\rm J. Math. Soc. Japan}, {\bf 62} (2010) no. {\bf 1}, 213--238.


\bibitem{kodaira} K.~Kodaira: \emph{On compact analytic surfaces II-III}, Ann. of Math. \textbf{77}
(1963), 563-626, \textbf{78} (1963), 1-40.
%\bibitem{MR85h:14017}
%A.~Libgober:
%\newblock \emph{Alexander polynomial of plane algebraic curves and cyclic multiple planes.}
%\newblock Duke Math. J. \textbf{49} (1982), no. 4, 833--851
%\bibitem{Loeser-Vaquie}
%F.~Loeser and M.~Vaqui{\'e}.
%\newblock \emph{Le polyn\^ome d'{A}lexander d'une courbe plane projective},
%\newblock {\rm Topology}, {\bf 29} (1990) no. {\bf 2}, 163--173.
  
%\bibitem{miranda} R.~Miranda: \emph{The moduli of Weierstrass fibrations over ${\mathbb P}^1$}, 
% Math. Ann. \textbf{255}(1981), 379-394.
 
\bibitem{miranda-basic} R.~Miranda: \emph{Basic theory of elliptic surfaces}, Dottorato di Ricerca in Matematica, ETS Editrice, Pisa, 1989.

%% 
\bibitem{miranda-persson} R.~Miranda and U.~Persson: \emph{
On extremal rational elliptic surfaces}, Math. Z. \textbf{193} (1986), 537-558.
%
%
%\bibitem{namba-tsuchi}
%M.~Namba and H.~Tsuchihashi, \emph{On the fundamental groups of {G}alois
%  covering spaces of the projective plane}, Geom. Dedicata~\textbf{104} (2004),
%  97--117.
%
\bibitem{oguiso-shioda}  K.~Oguiso and T.~Shioda: \emph{The Mordell-Weil lattice of a Rational
Elliptic surface}, Comment. Math. Univ. St. Pauli \textbf{40} (1991), 83-99.
%
%\bibitem{Okabook}
%M.~Oka.
%\newblock \emph{Non-degenerate complete intersection singularity}.
%\newblock Hermann, Paris, 1997.
%
%\bibitem{oka03}
%M.~Oka.
%\newblock \emph{Alexander polynomial of sextics.}
%\newblock {\rm J. Knot Theory Ramifications}, {\bf 12} (2003) no. {\bf 5}, 619--636.

\bibitem{oka05}
M.~Oka: \emph{Zariski pairs on sextics I}, Vietnam J. Math. 33 (2005), Special Issue, 81-92.



\bibitem{oka06} M.~Oka: \emph{Zariski pairs on sextics II},  Singularity theory, 837-863, World Sci. Publ., 
Hackensack, NJ, 2007. 
%


%\bibitem{OkaSurvey}
%M.~Oka.
%\newblock \emph{A survey on {Alexander} polynomials of plane curves.}
%\newblock {\rm Singularit\'es Franco-Japonaise, S\'eminaire et congr\`es},
%  {\bf 10} Soc. Math. France, Paris, 2005.
%%
%\bibitem{AEC} J.H.~Silverman: \emph{The Arithmetic of Elliptic Curves}, Graduate Texts in Mathematics, {\bf 106}
%Springer-Verlag, 1985.
%
%\bibitem{silverman} J.H.~Silverman: \emph{Advanced Topics in the Arithmetic of Elliptic Curves}, Graduate Texts in Mathematics, {\bf151} Springer-Verlag, 1994
%

\bibitem{shimada} I.~Shimada: \emph{Lattice Zariski k-ples of plane sextic curves and Z-splitting curves for double plane sextics}, 
Michigan Math. J. {\bf 59} (2010),  621-665

 \bibitem{shioda90} T.~Shioda: \emph{On the Mordell-Weil lattices}, \rm Comment. Math. Univ. St. Pauli
\textbf{39} (1990), 211-240.  

%
%\bibitem{shioda92} T.~Shioda: \emph{Existence of a rational elliptic surface with a given Mordell-Weil lattice},  
%\rm Proc. Japan Acad. Ser. A Math. Sci. {\bf 68} (1992), no. {\bf 9}, 251--255. 
%
%\bibitem{shioda93} T.~Shioda: \emph{Plane Quartics and Mordell-Weil Lattices of Type $E_7$},
%\rm Comment. Math. Univ. St. Pauli
%\textbf{42} (1993), 61--79.  
%
%
% \bibitem{shioda-usui} T.~Shioda and H.~Usui: \emph{Fundamental invariants of Weyl groups and
% excellent families of elliptic curves}, Comment. Math. Univ. St. Pauli \textbf{41} (1992),
% 169-217.
%% 
%
\bibitem{shirane16} T.~Shirane: \emph{A note on splitting numbers for
Galois covers and $\pi_1$-equivalent Zariski $k$-plets}, Proc. AMS., DOI 10.1090/proc/13298
%


\bibitem{tokunaga94} H.~Tokunaga:  \emph{On dihedral Galois coverings}, \rm Canadian J. of
Math. {\bf 46} \rm (1994),1299 - 1317.
%
\bibitem{tokunaga96} H.~Tokunaga: \emph{A remark on Artal's paper}, \rm Kodai Math. J. {\bf 19} (1996), 207-217.
%\bibitem{tokunaga97} H.~Tokunaga: \emph{Dihedral coverings branched along maximizing sextics}, \rm Math. Ann. {\bf 308}  (1997)
%633-648.
%%%
%\bibitem{tokunaga98}  H.~Tokunaga: \emph{Some examples of Zariski pairs arising from certain 
%elliptic K3 surfaces I}, \rm
%Math. Z. {\bf 227} (1998), 465-477, \emph{Some examples of Zariski pairs arising from certain 
%elliptic K3 surfaces II: Degtyarev's conjecture}, 
% Math. Z. {\bf 230} (1999), 389-400
%
%
%\bibitem{tokunaga04} H.~Tokunaga: \emph{Dihedral covers and an elementary arithmetic on elliptic surfaces}, 
%J. Math. Kyoto Univ. \textbf{44}(2004), 55-270.

%\bibitem{tokunaga10} H.~Tokunaga: \emph{Geometry of irreducible plane quartics  and  their quadratic residue conics}, 
% J. of Singularities(electric), \textbf{2} (2010), 170-190.
%
\bibitem{tokunaga12} H.~Tokunaga: \emph{Some sections on rational elliptic surfaces and certain special
conic-quartic configurations}, Kodai Math. J.\textbf{35} (2012), 78-104.
%
%
\bibitem{tokunaga14} H.~Tokunaga: \emph{
Sections of elliptic surfaces 
and
 Zariski pairs for conic-line arrangements via dihedral covers
 }, J. Math. Soc. Japan {\bf 66} (2014), 613-640.
%
%\bibitem{tumenbayar-tokunaga} K.~Tumenbayar and H.~Tokunaga: \emph{Elliptic surfaces and contact conics for a 
%$3$-nodal quartic}, to appear in Hokkaido Math. J.

%
\bibitem{yorisaki} E.~Yorisaki: \emph{Generators for the Mordell-Weil group of a certain rational elliptic surface and
its application}, Master's thesis,  Tokyo Metropolitan University (in Japanese) February 2014.
%
%%
\bibitem{zariski29} O.~Zariski: 
\emph{On the problem of existence of algebraic functions of two variables possessing a 
given branch curve}, Amer. J. Math.~\textbf{51} (1929), 305--328.
%%
%%
%\bibitem{zariski37}
%O.~Zariski:  \emph{The topological discriminant group of a {R}iemann surface of genus $p$}, 
%Amer. J. Math.~\textbf{59} (1937), 335--358.
%%
%\bibitem{zariski} O. Zariski: \emph{On the purity of the branch locus of algebraic functions}, 
%Proc. Nat. Acad. USA \textbf {44} (1958), 791-796.
%




\end{thebibliography}

\noindent Shinzo BANNAI\\
%Department of Natural Sciences\\
National Institute of Technology, Ibaraki College\\
866 Nakane, Hitachinaka-shi, Ibaraki-Ken 312-8508 JAPAN \\
{\tt sbannai@ge.ibaraki-ct.ac.jp}\\

\noindent Hiro-o TOKUNAGA, Momoko YAMAMOTO\\
Department of Mathematics and Information Sciences\\
%Graduate School of Science and Engineering,\\
Tokyo Metropolitan University\\
1-1 Minami-Ohsawa, Hachiohji 192-0397 JAPAN \\
{\tt tokunaga@tmu.ac.jp, yamamoto-momoko@ed.tmu.ac.jp}
%
%\noindent Emiko YORISAKI\\
%KONICA MINOLTA, INC. \\
%2970 Ishikawa-machi, Hachioji \\
%Tokyo 192-8505, JAPAN
%

{\tt }

\vspace{0.5cm}
      
%\noindent Masayuki KAWASHIMA\\
%      Department of Mathematics,\\
%         Tokyo University of Science,\\
%         1-3 Kagurazaka, Shinjuku-ku, 
%         Tokyo 162-8601 JAPAN\\
%{\tt kawashima@ma.kagu.tus.ac.jp}
% \input alexander1.tex     
 \end{document}


