%%%%%%%%%%%%%%%%%%%%%%%%%%%%%%%%%%%%%%%%%%%%%%%%%%%%%%%%%%%%%%%%%%%%%%%%%%%%%%%%
%2345678901234567890123456789012345678901234567890123456789012345678901234567890
%        1         2         3         4         5         6         7         8

\documentclass[letterpaper, 10 pt, conference]{ieeeconf}  % Comment this line out if you need a4paper

%\documentclass[a4paper, 10pt, conference]{ieeeconf}      % Use this line for a4 paper

\IEEEoverridecommandlockouts                              % This command is only needed if
                                                          % you want to use the \thanks command

\overrideIEEEmargins                                      % Needed to meet printer requirements.

% See the \addtolength command later in the file to balance the column lengths
% on the last page of the document

\makeatletter
\newcommand*{\rom}[1]{\expandafter\@slowromancap\romannumeral #1@}
\makeatother



\pdfminorversion=4
% \usepackage[margin=0.5in]{geometry}
% icra version
% \usepackage[bindingoffset=0.2in,%
%             left=0.67in,right=0.67in,top=0.79in,bottom=0.6in,%
%             footskip=.25in]{geometry}
% iros version
\usepackage[bindingoffset=0.2in,%
            left=0.75in,right=0.75in,top=0.75in,bottom=0.75in,%
            footskip=.25in]{geometry}
%\usepackage{siunitx}
% \usepackage{cm-super}
% The following packages can be found on http:\\www.ctan.org
\usepackage{graphics} % for pdf, bitmapped graphics files
\usepackage{epsfig} % for postscript graphics files
\usepackage{epstopdf}
\DeclareGraphicsRule{.tif}{png}{.png}{`convert #1 `dirname #1`/`basename #1 .tif`.png}
\graphicspath{ {figs} }
\usepackage{graphicx}
% \usepackage{subfig}
\usepackage{mwe}
\usepackage{caption}
\usepackage{xcolor}
\usepackage{subcaption}
\usepackage{lipsum}
\usepackage{float}
\usepackage{todonotes}
\usepackage{color,soul}
\usepackage{hyperref}
\usepackage{multirow}
\usepackage{algorithm}
\usepackage{algorithmic}

%\usepackage{booktabs}
%\usepackage{hyperref}

% \usepackage{soul}
% \usepackage{ulem}
%\usepackage{mathptmx} % assumes new font selection scheme installed
%\usepackage{times} % assumes new font selection scheme installed
\usepackage{amsmath} % assumes amsmath package installed
\DeclareMathOperator*{\argmin}{\arg\!\min}
\DeclareMathOperator*{\argmax}{\arg\!\max}
\usepackage[scientific-notation=true]{siunitx}
\listfiles
\usepackage{amssymb}  % assumes amsmath package installed
% \usepackage{bbm}
%\usepackage{algorithm2e}
\usepackage{array}
\newcolumntype{x}[1]{>{\centering\arraybackslash\hspace{0pt}}p{#1}}

% JTS I am defining a bunch of useful math notation here, please
% extend where necessary
\newcommand{\bx}{\mathbf{x}}
\newcommand{\ba}{\mathbf{a}}
\newcommand{\bs}{\mathbf{s}}
\newcommand{\bI}{\mathbf{I}}
\newcommand{\cS}{\mathcal{S}}
\newcommand{\cX}{\mathcal{X}}
\newcommand{\cA}{\mathcal{A}}
\newcommand{\cB}{\mathcal{B}}
\newcommand{\cD}{\mathcal{D}}

\newcommand{\etal}{{\em et al.}}

\definecolor{dblue}{rgb}{0,0,0.7}
\newcommand{\TODO}[2][somebody]{{\color{blue}[TODO (#1): #2]}}

\setlength{\textfloatsep}{2.0ex}
\setlength{\dbltextfloatsep}{2.0ex}


% \usepackage[showframe]{geometry}
% \usepackage{lipsum}
% \usepackage{graphicx}

% \usepackage[aboveskip=.5pt]{subcaption}
% \setlength{\abovecaptionskip}{5pt}
% \setlength{\belowcaptionskip}{-5pt}

\title{\LARGE \bf
    Socially Compliant Navigation through Raw Depth Inputs with Generative Adversarial Imitation Learning
% not the best title yet .. but better than nothing ;)
%  Towards Visual Navigation across Environments with Successor Feature based Deep Reinforcement Learning
%Successor Feature based Reinforcement Learning for Visual Navigation across Tasks and Environments
}


\author{Lei Tai$^{1}$ \ \ \
Jingwei Zhang$^{2}$ \ \ \
Ming Liu$^{1}$\ \ \
Wolfram Burgard$^{2}$% <-this % stops a space
\thanks{$^{*}$This paper is supported by Shenzhen Science, Technology and Innovation Commission (SZSTI) JCYJ20160428154842603 and JCYJ20160401100022706; also supported by the Research Grant Council of Hong Kong SAR Government, China, under Project No. 11210017 and No. 16212815 and No. 21202816 awarded to Prof. Ming Liu.}
\thanks{$^{1}$Department of Electronic and Computer Engineering, The Hong Kong University of Science and Technology; \{ltai, eelium\}@ust.hk }
\thanks{$^{2}$Department of Computer Science, Albert Ludwig University of Freiburg; \{zhang, burgard\}@informatik.uni-freiburg.de }
}

\begin{document}


\maketitle
\thispagestyle{empty}
\pagestyle{empty}


%%%%%%%%%%%%%%%%%%%%%%%%%%%%%%%%%%%%%%%%%%%%%%%%%%%%%%%%%%%%%%%%%%%%%%%%%%%%%%%%
\begin{abstract}
We present an approach for mobile robots to learn to navigate in dynamic environments with pedestrians via raw depth inputs,
in a socially compliant manner.
To achieve this, we adopt a generative adversarial imitation learning (GAIL) strategy, % for motion planning,
which improves upon a pre-trained behavior cloning policy.
Our approach overcomes the disadvantages of previous methods,
as they heavily depend on the full knowledge of the location and velocity information of nearby pedestrians,
which not only requires specific sensors,
%but also consumes much computation time for extracting such state information from raw sensor input.
but also the extraction of such state information from raw sensory input could consume much computation time. % for extracting such state information from raw sensor input.
In this paper, our proposed GAIL-based model performs directly on raw depth inputs and plans in real-time.
%With mobile robots increasingly appearing in human environments, they should navigate in pedestrian-rich environment safely and efficiently. In this paper, we present a learning-based navigation strategy in a pedestrian-rich environment. Traditional planning models take the precise localizations and velocities of nearby pedestrians as references.
%Two disadvantages of those methods are (1) dependencies on highly precise sensors and (2) the time-consuming estimation procedure to extract the related information. Here, we collect the dataset from the simulated environment to train a social-force estimation model through behavior cloning and improve the model through generative adversarial imitation learning.
%The generative adversarial model tries to discriminate the sample generated from the collected model-based dataset and the trained learning-based model in real time.
Experiments show that our GAIL-based approach greatly improves the safety and efficiency of the behavior of mobile robots from pure behavior cloning. % both safely and efficiently.
The real-world deployment also shows that our method is capable of guiding autonomous vehicles to navigate in a socially compliant manner directly through raw depth inputs.
%Additionally, through the generative adversarial imitation learning, the robot learns how to navigate in a pedestrian rich environment through raw depth visual input eventually.
In addition, we release a simulation plugin for modeling pedestrian behaviors  based on the social force model.


\end{abstract}

%%%%%%%%%%%%%%%%%%%%%%%%%%%%%%%%%%%%%%%%%%%%%%%%%%%%%%%%%%%%%%%%%%%%%%%%%%%%%%%%
\input{secs/introduction}

\input{secs/background}

\input{secs/methods}

\input{secs/experiments}

\input{secs/conclusion}

% \addtolength{\textheight}{-12cm}   % This command serves to balance the column lengths
                                  % on the last page of the document manually. It shortens
                                  % the textheight of the last page by a suitable amount.
                                  % This command does not take effect until the next page
                                  % so it should come on the page before the last. Make
                                  % sure that you do not shorten the textheight too much.

%%%%%%%%%%%%%%%%%%%%%%%%%%%%%%%%%%%%%%%%%%%%%%%%%%%%%%%%%%%%%%%%%%%%%%%%%%%%%%%%



%%%%%%%%%%%%%%%%%%%%%%%%%%%%%%%%%%%%%%%%%%%%%%%%%%%%%%%%%%%%%%%%%%%%%%%%%%%%%%%%



%%%%%%%%%%%%%%%%%%%%%%%%%%%%%%%%%%%%%%%%%%%%%%%%%%%%%%%%%%%%%%%%%%%%%%%%%%%%%%%%

%\section*{ACKNOWLEDGMENT}

%%%%%%%%%%%%%%%%%%%%%%%%%%%%%%%%%%%%%%%%%%%%%%%%%%%%%%%%%%%%%%%%%%%%%%%%%%%%%%%%

\small
% \footnotesize{tiny}
% \ragged2e
% \spaceskip 0.0em \relax
\bibliographystyle{IEEEtran}
\bibliography{tai18icra}


\end{document}
