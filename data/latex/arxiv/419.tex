\def\names{{vca},{sisal},{dsplr},{davis},{sync},{async}}
\def\materials{{Soil},{Water},{Veg.}}

\begin{figure}[t]
\centering
\foreach \name [count=\i] in \names {
	\foreach \material[count=\j] in \materials {
		\def\idx{\the\numexpr\j+(\i-1)*3+14} 
		\begin{subfigure}[t]{0.15\textwidth}	
		\includegraphics[keepaspectratio,width=\textwidth]{\idx}
		\caption{\material{} (\MakeUppercase{\name})}
		\label{fig:endm_\j_\name}
		\end{subfigure}
	} \\
}
\caption{Endmembers ($\m_r$, red lines) recovered by the different methods from the real dataset depicted in Fig.~\ref{fig:cube}. Endmembers extracted by VCA, SISAL and DSPLR show a notable sensitivity to the presence of outliers in these data, }
%\caption{Endmembers ($\m_r$, red lines) and their variants affected by variability ($\m_r + \dm_{r,t}$, blue dotted lines) recovered by the different methods from the real dataset depicted in Fig.~\ref{fig:cube}. Signatures corresponding to different time instants are represented in a single figure to better appreciate the variability recovered from the data. The spectra represented in black correspond to signatures corrupted by outliers, while those given in green represent endmembers which have been split into several components by the associated estimation procedure.}
\label{fig:real_endm}
\end{figure}