The Brownian meander is a Brownian motion 
$\{B(t)\,,\,\, t>0\}$
evolving under the condition that $ \min_{0\leq s\leq t} B(s) > 0 $. 
If the additional condition that $ B(t) = c >0  $ then we have the Brownian excursion which 
is the bridge of the Brownian meander. 


%An active research on the Brownian meander was undertaken during the 
%Seventies(\cite{durrett77}), 
%although important results can be found in the classical book by 
%\citet{ito1996diffusion}. 

Early results in this field emerged in the study of the behaviour of 
random walks conditioned to stay positive where the Brownian meander 
was obtained as the weak limit of such conditional processes (\cite{belkin70}, 
\cite{iglehart74}). In the same spirit the distribution of the maximum 
of the Brownian meander and excursion has been derived in \cite{kaigh78}. 
Further investigations about such distributions can be found in \cite{chung1976}. 
Some important results can also be found in the classical book by 
\citet{ito1996diffusion}. The notion 
of Brownian meander as a conditional Brownian motion and 
problems concerning weak convergence to such processes have been treated 
in \cite{durrett77}. 
More recently analogous results have been obtained in the general setting
of Lévy processes (see for example \cite{chaumont2005}). 

Brownian meanders emerge in path decompositions of the Brownian motion.
In particular Denisov (\cite{denisov84})  shows that a Brownian motion around a maximum 
point can be represented (in law) by means of a two-sided Brownian meander, which is constructed by gluing together 
two meanders. 

These processes also arise in several scientific fields. 
Possible applications range from SPDE's with reflection (\cite{zambotti04}) to enumeration of random graphs (see \cite{janson07} for a survey of the results in this field). 

In this paper we study the meandering process of the drifted Brownian motion $ B^\mu (t) $ and, 
in particular we start by analyzing the joint $ n- $fold distributions
\begin{equation}\label{eq:mdr-joint-intro}
P \bigg\{ \bigcap_{j=1}^n \left( B^\mu(s_j) \in \mathrm d y_j \right) \,\Big \vert \min_{0\leq z \leq t} B^\mu(z)> v , B^\mu(0)=u \bigg \} 
\end{equation}
%
for $ v < y_j,\,0< s_j \leq t,\, j= 1, \ldots n $ and $u>v $.

In order to examine the interesting case where the starting point coincides with the barrier level $ v $
we use the tools of weak convergence of probability measures. In section 2 we briefly recall the needed results and
in section 3 we show that the required conditions hold. 

We will study some sample path properties of the Brownian meander regarding the maximal oscillation around the starting point 
\begin{equation}\label{eq:lemma-cond-intro}
\lim_{\delta \to 0}
\lim_{u \to v} 
P \Big( \max_{0\leq z \leq \delta } | B^\mu ( z) - B^\mu(0)  | < \eta  \, \Big | 
\min_{ 0\leq z \leq t} B^\mu ( z) > v ,  B^\mu(0) = u \Big) = 1 \qquad \forall \eta > 0
\end{equation}
%
%
and the maximal oscillation of the sample paths at an arbitrary point $ s $
\begin{align}
& \lim_{\delta \to 0} \lim_{ u \downarrow v} P \left\{  \max_{s - \delta \leq z \leq s +\delta } |B^\mu (z) | \leq \eta \,\Big | \min_{ 0 \leq z \leq t } B^\mu(z) > v, B^\mu (0) = u\right\} 
\\
&= 
P\left\{ B^\mu (s ) \leq \eta \Big  | \inf_{ 0 < z < t } B^\mu(z) > v, B^\mu(0) = v \right\} \, .
\notag 
\end{align}

In the limit for $ u \to v $ we obtain a stochastic process with marginal distributions equal to 
%
%
%
\begin{align}\label{eq:mdr-0-dist-intro}
&
P \bigg\{  
B^\mu ( s) \in \mathrm d y \bigg |  \inf_{ 0 < z < t } B^\mu( z)  > v, B^\mu(0) = v  
\bigg \} \\
&=
\mathrm d y \left(\frac ts \right)^{\frac 32} 
\frac{
	(y - v) e^{  -\frac{(y-v)^2}{2s}  }
}{
	\int_0^{\infty} w  e^{  - \frac { w^2 } { 2t } + \mu w } \,\mathrm d w 
}
\int_0^\infty  \left( e^{ - \frac{(w-(y-v) )^2}{2 ( t-s)}}  - e^{ - \frac{(w + (y - v ) )^2}{2 ( t-s)}}  \right)
	\frac{e^{\mu w }}{\sqrt{2 \pi( t-s)}} \, \mathrm d w \notag 
\end{align}
$ s < t, \, y > v $. In particular, for $ s = t $, \eqref{eq:mdr-0-dist-intro} reduces to 
%
%
\begin{align*}
&
P \bigg\{  
B^\mu ( t) \in \mathrm d y \bigg |  \inf_{ 0 < z < t } B^\mu( z)  > v, B^\mu(0) = v  
\bigg \} 
=
\frac{
	(y - v )   e^{  - \frac{ (y-v)^2}{2t} } e^{\mu y } 
}{
	\int_v^\infty (w-v) e^{  - \frac{ (w-v)^2}{2t}   }  e^{ \mu w} \, \mathrm d w
} \,\mathrm d y 
\numberthis \label{eq:bmd-drift-dist-t-intro} \,\, , \qquad y > v\,\, .
\end{align*}
%
%
which for $ \mu = 0 $ coincides with the truncated Rayleigh distribution. 
We obtain the distribution of the maximum of the Brownian meander which, in the simplest case, has the form
%
\begin{align*}
&
P\bigg\{ \max_{0 \leq z \leq t} B^\mu(z) <x \Big |  \inf_{0 < z < t} B^\mu (z) > 0, B^\mu(0) = 0  \bigg\}
\label{eq:max-bmd-t-v0-fin-intro}
\numberthis
\\
&=
\frac{  
	\sum_{r = -\infty} ^ { + \infty}
	(-1)^r
	e^{  -  \mu r x - \frac{x^2 r^2}{2t}  } 
	+ \mu
	\int_0^\infty e^{ - \frac{w^2}{2t} } e^{ \mu (-1)^{ \left \lfloor \frac w x \right \rfloor} } \,\mathrm d w
}{
	1 + \mu \int_0^\infty 
	e^{  - \frac{y^2}{2t}  } 
	e^{ \mu y } \,\mathrm d y	
}
\end{align*}
%
%
and for $  \mu = 0 $  yields the well-known distribution 
\begin{equation}\label{eq:max-bmd-mu0-intro}
P\bigg\{ \max_{0 \leq z \leq t} B(z) <x \, \Big |  \inf_{0 < z < t} B (z) > 0, B(0) = 0  \bigg\}
=
\sum_{r = - \infty} ^ { + \infty} 
(-1)^r
e^{  - \frac{x^2 r^2 }{2t}  } \,\, .
\end{equation}
%
%
A related result concerns the first-passage time of the drifted meander. The random variable 
$ T_x = \inf \{ s < t' : B^\mu(s) = x  \} $, $ t'> t $, under the condition that $ \min_{ 0 \leq z \leq t } B^\mu (z) > v $ 
has distribution $ P(T_x > s | \min_{ 0 \leq z \leq t } B^\mu (z) > v, B^\mu(0) = u) $ with a substantially 
different structure for $ s < t  $ and $ t < s < t' $. The effect of the conditioning event in the case $ s > t $ 
corresponds to assuming a Rayleigh-distributed starting point at time $ t $.  

The fifth section of the paper is devoted to the extension to the drifted Brownian meander of the 
relationship
\begin{equation}\label{eq:mdr-repr-0-intro}
\frac{
	\Big|  B(T_0 + s (t - T_0)) \Big|
}{\sqrt{t - T_0}}
\stackrel{i.d.}{=} M(s) \qquad 0 < s < 1
\end{equation}
where $ T_0 = \sup \{s < t : B(s) = 0\} $ (see \citet{pitman99}). 
We are able to show  that 
%
%
\begin{align}\label{eq:mdr-repr-intro}
P\left\{   
\frac { 
	\Big | B^\mu\Big ( T_0^\mu + s ( t - T_0^\mu) \Big )  \Big |    
}
{
	\sqrt{t - T^\mu_0}
} \in \mathrm d y 
\right\} = 
\frac 12 
\mathbb E \left[
P\left\{ M^{- \mu \sqrt{ t - T_0^\mu}}(s) \in\mathrm d y \right\} + 
P\left\{ M^{\mu \sqrt{ t - T_0^\mu}} (s)\in\mathrm d y \right\} 
\right]
\end{align}
$ 0 < s < 1 $, where  $M^{\pm \mu \sqrt{ t - T_0^\mu}}$ is the Brownian meander starting at zero
with a time-varying drift $ \pm \mu \sqrt{ t - T_0^\mu} $ and $ T_0^\mu = \sup\{s < t: B^\mu(s) = 0\} $. 



Result \eqref{eq:mdr-repr-intro} can also be written as
\begin{equation}
	\frac{
		\Big|  B^\mu(T_0 + s (t - T^\mu_0)) \Big|
	}{\sqrt{t - T^\mu_0}}
	\stackrel{i.d.}{=} M^{\mu X \sqrt{ t - T_0^\mu} }(s) \qquad 0 < s < 1
\end{equation}
where $ X $ is a r.v. taking values $ \pm 1 $ with equal probability, independent from $ M $ and $ T^\mu_0 $ . 
%
%
In the last section we give the distribution of the Brownian excursion and show that
%
	\begin{align*}
	& 
	P\Big\{ B^\mu (s ) \in \mathrm d y \Big | \inf_{ 0 < z < t } B^\mu ( z)> 0, B^\mu ( 0) =  B^\mu ( t) = 0 \Big\}
	\numberthis \label{eq:exc-intro} \\
	&=
	\sqrt \frac{2}{\pi} y^2 \left(\frac{t}{s(t-s)}\right) ^\frac 32 
	e^{- \frac{ y^2t}{ 2 s (t-s)}} \, \mathrm d y \qquad y>0 \,,\,\, s<t
	\\
	&=
	P\Big\{  B(s ) \in \mathrm d y \Big | \inf_{ 0 < z < t } B ( z)> 0, B ( 0) =  B ( t) = 0 \Big\} 
	\end{align*}
which therefore does not depend on the drift $ \mu $.
%
%For the Brownian excursion we give the following result for the sojourn time on $ (0, \infty) $ 
%in the time interval $ (l,t] $ under the condition that $ \inf_{ 0\leq z \leq l}B(z) > 0 $: 
%
%
%\begin{equation}
%P\left\{  \Gamma_{l,t} \in \mathrm d s \Big |  \inf_{0< z\leq l} B(z) > 0, B(0)=0 , B(t)=0 \right\} 
%= 
%\mathrm d s 
%\frac t2 \sqrt \frac{l } {t-l} 
%\int_{l}^{l+s}
%\frac{\mathrm d w} {\sqrt{w^3 (t-w)^3 }} \, .
%\end{equation}


% !TeX root = ./mdr-main.tex
