%% bare_jrnl.tex
%% V1.4b
%% 2015/08/26
%% by Michael Shell
%% see http://www.michaelshell.org/
%% for current contact information.
%%
%% This is a skeleton file demonstrating the use of IEEEtran.cls
%% (requires IEEEtran.cls version 1.8b or later) with an IEEE
%% journal paper.
%%
%% Support sites:
%% http://www.michaelshell.org/tex/ieeetran/
%% http://www.ctan.org/pkg/ieeetran
%% and
%% http://www.ieee.org/

%%*************************************************************************
%% Legal Notice:
%% This code is offered as-is without any warranty either expressed or
%% implied; without even the implied warranty of MERCHANTABILITY or
%% FITNESS FOR A PARTICULAR PURPOSE! 
%% User assumes all risk.
%% In no event shall the IEEE or any contributor to this code be liable for
%% any damages or losses, including, but not limited to, incidental,
%% consequential, or any other damages, resulting from the use or misuse
%% of any information contained here.
%%
%% All comments are the opinions of their respective authors and are not
%% necessarily endorsed by the IEEE.
%%
%% This work is distributed under the LaTeX Project Public License (LPPL)
%% ( http://www.latex-project.org/ ) version 1.3, and may be freely used,
%% distributed and modified. A copy of the LPPL, version 1.3, is included
%% in the base LaTeX documentation of all distributions of LaTeX released
%% 2003/12/01 or later.
%% Retain all contribution notices and credits.
%% ** Modified files should be clearly indicated as such, including  **
%% ** renaming them and changing author support contact information. **
%%*************************************************************************


% *** Authors should verify (and, if needed, correct) their LaTeX system  ***
% *** with the testflow diagnostic prior to trusting their LaTeX platform ***
% *** with production work. The IEEE's font choices and paper sizes can   ***
% *** trigger bugs that do not appear when using other class files.       ***                          ***
% The testflow support page is at:
% http://www.michaelshell.org/tex/testflow/



\documentclass[journal]{IEEEtran}
%
% If IEEEtran.cls has not been installed into the LaTeX system files,
% manually specify the path to it like:
% \documentclass[journal]{../sty/IEEEtran}
\newcommand{\subparagraph}{}
\usepackage{titlesec}
\titlespacing{\subsection} {0pt}{6pt}{3pt}
%\titlespacing{\section} {0pt}{12pt}{3pt}



% Some very useful LaTeX packages include:
% (uncomment the ones you want to load)


% *** MISC UTILITY PACKAGES ***
%
%\usepackage{ifpdf}
% Heiko Oberdiek's ifpdf.sty is very useful if you need conditional
% compilation based on whether the output is pdf or dvi.
% usage:
% \ifpdf
%   % pdf code
% \else
%   % dvi code
% \fi
% The latest version of ifpdf.sty can be obtained from:
% http://www.ctan.org/pkg/ifpdf
% Also, note that IEEEtran.cls V1.7 and later provides a builtin
% \ifCLASSINFOpdf conditional that works the same way.
% When switching from latex to pdflatex and vice-versa, the compiler may
% have to be run twice to clear warning/error messages.






% *** CITATION PACKAGES ***
%
\usepackage{cite}
% cite.sty was written by Donald Arseneau
% V1.6 and later of IEEEtran pre-defines the format of the cite.sty package
% \cite{} output to follow that of the IEEE. Loading the cite package will
% result in citation numbers being automatically sorted and properly
% "compressed/ranged". e.g., [1], [9], [2], [7], [5], [6] without using
% cite.sty will become [1], [2], [5]--[7], [9] using cite.sty. cite.sty's
% \cite will automatically add leading space, if needed. Use cite.sty's
% noadjust option (cite.sty V3.8 and later) if you want to turn this off
% such as if a citation ever needs to be enclosed in parenthesis.
% cite.sty is already installed on most LaTeX systems. Be sure and use
% version 5.0 (2009-03-20) and later if using hyperref.sty.
% The latest version can be obtained at:
% http://www.ctan.org/pkg/cite
% The documentation is contained in the cite.sty file itself.


% *** GRAPHICS RELATED PACKAGES ***
%
\ifCLASSINFOpdf
  \usepackage[pdftex]{graphicx}
  \setlength{\textfloatsep}{0.7\baselineskip plus 0.2\baselineskip minus 0.2\baselineskip}
%\setlength{\textfloatsep}{10 pt} % use to set space between top figure and the text
 \setlength{\floatsep}{2 pt plus 5pt minus 2pt} % space b/w two top floats
 \setlength{\abovecaptionskip}{2pt plus 5pt minus 5pt}
 
  % declare the path(s) where your graphic files are
  % \graphicspath{{../pdf/}{../jpeg/}}
  % and their extensions so you won't have to specify these with
  % every instance of \includegraphics
  % \DeclareGraphicsExtensions{.pdf,.jpeg,.png}
\else
  % or other class option (dvipsone, dvipdf, if not using dvips). graphicx
  % will default to the driver specified in the system graphics.cfg if no
  % driver is specified.
  % \usepackage[dvips]{graphicx}
  % declare the path(s) where your graphic files are
  % \graphicspath{{../eps/}}
  % and their extensions so you won't have to specify these with
  % every instance of \includegraphics
  % \DeclareGraphicsExtensions{.eps}
\fi
% graphicx was written by David Carlisle and Sebastian Rahtz. It is
% required if you want graphics, photos, etc. graphicx.sty is already
% installed on most LaTeX systems. The latest version and documentation
% can be obtained at: 
% http://www.ctan.org/pkg/graphicx
% Another good source of documentation is "Using Imported Graphics in
% LaTeX2e" by Keith Reckdahl which can be found at:
% http://www.ctan.org/pkg/epslatex
%
% latex, and pdflatex in dvi mode, support graphics in encapsulated
% postscript (.eps) format. pdflatex in pdf mode supports graphics
% in .pdf, .jpeg, .png and .mps (metapost) formats. Users should ensure
% that all non-photo figures use a vector format (.eps, .pdf, .mps) and
% not a bitmapped formats (.jpeg, .png). The IEEE frowns on bitmapped formats
% which can result in "jaggedy"/blurry rendering of lines and letters as
% well as large increases in file sizes.
%
% You can find documentation about the pdfTeX application at:
% http://www.tug.org/applications/pdftex





% *** MATH PACKAGES ***
%
\usepackage{amssymb}
\usepackage{amsmath}
% A popular package from the American Mathematical Society that provides
% many useful and powerful commands for dealing with mathematics.
%
% Note that the amsmath package sets \interdisplaylinepenalty to 10000
% thus preventing page breaks from occurring within multiline equations. Use:
%\interdisplaylinepenalty=2500
% after loading amsmath to restore such page breaks as IEEEtran.cls normally
% does. amsmath.sty is already installed on most LaTeX systems. The latest
% version and documentation can be obtained at:
% http://www.ctan.org/pkg/amsmath





% *** SPECIALIZED LIST PACKAGES ***
%
\usepackage{algorithmic}
% algorithmic.sty was written by Peter Williams and Rogerio Brito.
% This package provides an algorithmic environment fo describing algorithms.
% You can use the algorithmic environment in-text or within a figure
% environment to provide for a floating algorithm. Do NOT use the algorithm
% floating environment provided by algorithm.sty (by the same authors) or
% algorithm2e.sty (by Christophe Fiorio) as the IEEE does not use dedicated
% algorithm float types and packages that provide these will not provide
% correct IEEE style captions. The latest version and documentation of
% algorithmic.sty can be obtained at:
% http://www.ctan.org/pkg/algorithms
% Also of interest may be the (relatively newer and more customizable)
% algorithmicx.sty package by Szasz Janos:
% http://www.ctan.org/pkg/algorithmicx


\usepackage{enumitem}

% *** ALIGNMENT PACKAGES ***
%
\usepackage{multirow}
\usepackage{array}
% Frank Mittelbach's and David Carlisle's array.sty patches and improves
% the standard LaTeX2e array and tabular environments to provide better
% appearance and additional user controls. As the default LaTeX2e table
% generation code is lacking to the point of almost being broken with
% respect to the quality of the end results, all users are strongly
% advised to use an enhanced (at the very least that provided by array.sty)
% set of table tools. array.sty is already installed on most systems. The
% latest version and documentation can be obtained at:
% http://www.ctan.org/pkg/array


% IEEEtran contains the IEEEeqnarray family of commands that can be used to
% generate multiline equations as well as matrices, tables, etc., of high
% quality.




% *** SUBFIGURE PACKAGES ***
%\ifCLASSOPTIONcompsoc
%  \usepackage[caption=false,font=normalsize,labelfont=sf,textfont=sf]{subfig}
%\else
%  \usepackage[caption=false,font=footnotesize]{subfig}
%\fi
% subfig.sty, written by Steven Douglas Cochran, is the modern replacement
% for subfigure.sty, the latter of which is no longer maintained and is
% incompatible with some LaTeX packages including fixltx2e. However,
% subfig.sty requires and automatically loads Axel Sommerfeldt's caption.sty
% which will override IEEEtran.cls' handling of captions and this will result
% in non-IEEE style figure/table captions. To prevent this problem, be sure
% and invoke subfig.sty's "caption=false" package option (available since
% subfig.sty version 1.3, 2005/06/28) as this is will preserve IEEEtran.cls
% handling of captions.
% Note that the Computer Society format requires a larger sans serif font
% than the serif footnote size font used in traditional IEEE formatting
% and thus the need to invoke different subfig.sty package options depending
% on whether compsoc mode has been enabled.
%
% The latest version and documentation of subfig.sty can be obtained at:
% http://www.ctan.org/pkg/subfig




% *** FLOAT PACKAGES ***
%
%\usepackage{fixltx2e}
% fixltx2e, the successor to the earlier fix2col.sty, was written by
% Frank Mittelbach and David Carlisle. This package corrects a few problems
% in the LaTeX2e kernel, the most notable of which is that in current
% LaTeX2e releases, the ordering of single and double column floats is not
% guaranteed to be preserved. Thus, an unpatched LaTeX2e can allow a
% single column figure to be placed prior to an earlier double column
% figure.
% Be aware that LaTeX2e kernels dated 2015 and later have fixltx2e.sty's
% corrections already built into the system in which case a warning will
% be issued if an attempt is made to load fixltx2e.sty as it is no longer
% needed.
% The latest version and documentation can be found at:
% http://www.ctan.org/pkg/fixltx2e


%\usepackage{stfloats}
% stfloats.sty was written by Sigitas Tolusis. This package gives LaTeX2e
% the ability to do double column floats at the bottom of the page as well
% as the top. (e.g., "\begin{figure*}[!b]" is not normally possible in
% LaTeX2e). It also provides a command:
%\fnbelowfloat
% to enable the placement of footnotes below bottom floats (the standard
% LaTeX2e kernel puts them above bottom floats). This is an invasive package
% which rewrites many portions of the LaTeX2e float routines. It may not work
% with other packages that modify the LaTeX2e float routines. The latest
% version and documentation can be obtained at:
% http://www.ctan.org/pkg/stfloats
% Do not use the stfloats baselinefloat ability as the IEEE does not allow
% \baselineskip to stretch. Authors submitting work to the IEEE should note
% that the IEEE rarely uses double column equations and that authors should try
% to avoid such use. Do not be tempted to use the cuted.sty or midfloat.sty
% packages (also by Sigitas Tolusis) as the IEEE does not format its papers in
% such ways.
% Do not attempt to use stfloats with fixltx2e as they are incompatible.
% Instead, use Morten Hogholm'a dblfloatfix which combines the features
% of both fixltx2e and stfloats:
%
% \usepackage{dblfloatfix}
% The latest version can be found at:
% http://www.ctan.org/pkg/dblfloatfix




%\ifCLASSOPTIONcaptionsoff
%  \usepackage[nomarkers]{endfloat}
% \let\MYoriglatexcaption\caption
% \renewcommand{\caption}[2][\relax]{\MYoriglatexcaption[#2]{#2}}
%\fi
% endfloat.sty was written by James Darrell McCauley, Jeff Goldberg and 
% Axel Sommerfeldt. This package may be useful when used in conjunction with 
% IEEEtran.cls'  captionsoff option. Some IEEE journals/societies require that
% submissions have lists of figures/tables at the end of the paper and that
% figures/tables without any captions are placed on a page by themselves at
% the end of the document. If needed, the draftcls IEEEtran class option or
% \CLASSINPUTbaselinestretch interface can be used to increase the line
% spacing as well. Be sure and use the nomarkers option of endfloat to
% prevent endfloat from "marking" where the figures would have been placed
% in the text. The two hack lines of code above are a slight modification of
% that suggested by in the endfloat docs (section 8.4.1) to ensure that
% the full captions always appear in the list of figures/tables - even if
% the user used the short optional argument of \caption[]{}.
% IEEE papers do not typically make use of \caption[]'s optional argument,
% so this should not be an issue. A similar trick can be used to disable
% captions of packages such as subfig.sty that lack options to turn off
% the subcaptions:
% For subfig.sty:
% \let\MYorigsubfloat\subfloat
% \renewcommand{\subfloat}[2][\relax]{\MYorigsubfloat[]{#2}}
% However, the above trick will not work if both optional arguments of
% the \subfloat command are used. Furthermore, there needs to be a
% description of each subfigure *somewhere* and endfloat does not add
% subfigure captions to its list of figures. Thus, the best approach is to
% avoid the use of subfigure captions (many IEEE journals avoid them anyway)
% and instead reference/explain all the subfigures within the main caption.
% The latest version of endfloat.sty and its documentation can obtained at:
% http://www.ctan.org/pkg/endfloat
%
% The IEEEtran \ifCLASSOPTIONcaptionsoff conditional can also be used
% later in the document, say, to conditionally put the References on a 
% page by themselves.




% *** PDF, URL AND HYPERLINK PACKAGES ***
%
%\usepackage{url}
% url.sty was written by Donald Arseneau. It provides better support for
% handling and breaking URLs. url.sty is already installed on most LaTeX
% systems. The latest version and documentation can be obtained at:
% http://www.ctan.org/pkg/url
% Basically, \url{my_url_here}.




% *** Do not adjust lengths that control margins, column widths, etc. ***
% *** Do not use packages that alter fonts (such as pslatex).         ***
% There should be no need to do such things with IEEEtran.cls V1.6 and later.
% (Unless specifically asked to do so by the journal or conference you plan
% to submit to, of course. )

\usepackage{color,soul}
\definecolor{Blue}{rgb}{0.3,0.3,0.9}
% correct bad hyphenation here
\hyphenation{op-tical net-works semi-conduc-tor}
\DeclareMathSizes{6}{6}{6}{6}
% \DeclareMathSizes{10}{10}{10}{10}
\newcommand{\squeezeup}{\vspace{-2.5mm}}
\begin{document}
%
% paper title
% Titles are generally capitalized except for words such as a, an, and, as,
% at, but, by, for, in, nor, of, on, or, the, to and up, which are usually
% not capitalized unless they are the first or last word of the title.
% Linebreaks \\ can be used within to get better formatting as desired.
% Do not put math or special symbols in the title.
\title{Real-Time Local Volt/VAR Control Under External Disturbances with High PV Penetration}
%
%
% author names and IEEE memberships
% note positions of commas and nonbreaking spaces ( ~ ) LaTeX will not break
% a structure at a ~ so this keeps an author's name from being broken across
% two lines.
% use \thanks{} to gain access to the first footnote area
% a separate \thanks must be used for each paragraph as LaTeX2e's \thanks
% was not built to handle multiple paragraphs
%

\author{A.~Singhal,~\IEEEmembership{Member,~IEEE,}
        V.~Ajjarapu,~\IEEEmembership{Fellow,~IEEE,}
         J.C.~Fuller,~\IEEEmembership{Senior~Member,~IEEE,}
        and~J.~Hansen,~\IEEEmembership{Member,~IEEE}%
\thanks{This work was supported in part by U.S. Department of Energy’s Sunshot Initiative Program DE-0006341.}        
\thanks{A. Singhal and V. Ajjarapu are with Department of Electric and Computer Engineering, Iowa State University, Ames, IA, 50010 USA (e-mail: ankit@iastate.edu, vajjarap@iastate.edu).}% <-this % stops a space
\thanks{J. C. Fuller, J. Hensen are with Pacific Northwest National Laboratory, Richland, WA, 99453 USA (e-mail: jason.fuller@pnnl.gov, jacob.hansen@pnnl.gov).}}% <-this % stops a space
%\thanks{Manuscript received April 19, 2005; revised August 26, 2015.}}

% note the % following the last \IEEEmembership and also \thanks - 
% these prevent an unwanted space from occurring between the last author name
% and the end of the author line. i.e., if you had this:
% 
% \author{....lastname \thanks{...} \thanks{...} }
%                     ^------------^------------^----Do not want these spaces!
%
% a space would be appended to the last name and could cause every name on that
% line to be shifted left slightly. This is one of those "LaTeX things". For
% instance, "\textbf{A} \textbf{B}" will typeset as "A B" not "AB". To get
% "AB" then you have to do: "\textbf{A}\textbf{B}"
% \thanks is no different in this regard, so shield the last } of each \thanks
% that ends a line with a % and do not let a space in before the next \thanks.
% Spaces after \IEEEmembership other than the last one are OK (and needed) as
% you are supposed to have spaces between the names. For what it is worth,
% this is a minor point as most people would not even notice if the said evil
% space somehow managed to creep in.



% The paper headers
 \markboth{Accepted in IEEE Transactions of Smart Grid - DOI: https://doi.org/10.1109/TSG.2018.2840965}%
 {}
% The only time the second header will appear is for the odd numbered pages
% after the title page when using the two side option.
% 
% *** Note that you probably will NOT want to include the author's ***
% *** name in the headers of peer review papers.                   ***
% You can use \ifCLASSOPTIONpeerreview for conditional compilation here if
% you desire.




% If you want to put a publisher's ID mark on the page you can do it like
% this:
%\IEEEpubid{0000--0000/00\$00.00~\copyright~2015 IEEE}
% Remember, if you use this you must call \IEEEpubidadjcol in the second
% column for its text to clear the IEEEpubid mark.



% use for special paper notices
%\IEEEspecialpapernotice{(Invited Paper)}




% make the title area
\maketitle

% As a general rule, do not put math, special symbols or citations
% in the abstract or keywords.

\begin{abstract}
Volt/var control (VVC) of smart PV inverter is becoming one of the most popular solutions to address the voltage challenges associated with high PV penetration. This work focuses on the local droop VVC recommended by the grid integration standards IEEE1547, rule21 and addresses their major challenges i.e. appropriate parameters selection under changing conditions, and the control being vulnerable to instability (or voltage oscillations) and significant steady state error (SSE). This is achieved by proposing a two-layer local real-time adaptive VVC that has two major features i.e. a) it is able to ensure both low SSE and control stability simultaneously without compromising either; and b) it dynamically adapts its parameters to ensure good performance in a wide range of external disturbances such as sudden cloud cover, cloud intermittency, and substation voltage changes. A theoretical analysis and convergence proof of the proposed control is also discussed. The proposed control is implementation friendly as it fits well within the integration standard framework and depends only on the local bus information. The performance is compared with the existing droop VVC methods in several scenarios on a large unbalanced 3-phase feeder with detailed secondary side modeling.
\end{abstract}

% Note that keywords are not normally used for peer review papers..
\vspace{-2.5mm}
\begin{IEEEkeywords}
solar photovoltaic system, smart grid, volt/var control, smart inverter, real-time control, distributed control.
\end{IEEEkeywords}
\vspace{-2mm}
% For peer review papers, you can put extra information on the cover
% page as needed:
% \ifCLASSOPTIONpeerreview
% \begin{center} \bfseries EDICS Category: 3-BBND \end{center}
% \fi
%
% For peerreview papers, this IEEEtran command inserts a page break and
% creates the second title. It will be ignored for other modes.
\IEEEpeerreviewmaketitle

\section{Introduction}
% The very first letter is a 2 line initial drop letter followed
% by the rest of the first word in caps.
% 
% form to use if the first word consists of a single letter:
% \IEEEPARstart{A}{demo} file is ....
% 
% form to use if you need the single drop letter followed by
% normal text (unknown if ever used by the IEEE):
% \IEEEPARstart{A}{}demo file is ....
% 
% Some journals put the first two words in caps:
% \IEEEPARstart{T}{his demo} file is ....
% 
% Here we have the typical use of a "T" for an initial drop letter
% and "HIS" in caps to complete the first word.

% You must have at least 2 lines in the paragraph with the drop letter
% (should never be an issue)

\IEEEPARstart{S}{olar} photovoltaic (PV) penetration is continuously rising, and is expected to be tripled in the next 5 years in the USA \cite{seia_2016}.
%was the source of the highest new capacity addition in electricity generation in USA in 2016 \cite{seia_2016}. Especially, the roof-top solar PV's technical potential is significantly increasing in USA \cite{gagnon_rooftop_2016}.
High PV penetration is being fueled by the favorable policies and significant cost reductions, nonetheless, it brings its own set of technical challenges such as voltage rise and rapid voltage fluctuations due to cloud transients which could lead to the reduced power quality \cite{coster_integration_2011,mather_high-penetration_2016}. % Under the premise that solar will keep growing as expected, several efforts are required to address the associated voltage challenges. 
%To mitigate these challenges, a few simple methods are suggested such as lowering substation voltage, increasing conductor diameter, curtailing solar generation etc. But, these methods are not economical and not adaptive to changing conditions (low solar output during peak load)\cite{masters_voltage_2002,coster_integration_2011}. %
In traditional volt/var control (VVC), voltage regulating devices such as capacitors and load tap changers are supposed to maintain the feeder voltage but they are not fast enough to handle transient nature of solar generation i.e. cloud cover \cite{yeh_adaptive_2012,robbins_two-stage_2013,mcgranaghan_advanced_2008}. Therefore, PV inverter has emerged as an effective VVC solution to handle rapid variations in the modern distribution system by providing faster and continuous control capability in contrast to slower and discrete response of traditional devices \cite{turitsyn_options_2011,bollen_voltage_2005,carvalho_distributed_2008}.

The PV inverter VVC methods primarily fall into two broad categories: 1) optimal power flow (OPF) based centralized and distributed control approaches and 2) local control approaches. Most of the literature deals with the OPF based methods which are solved either in a centralized manner \cite{xu_multi-timescale_2017,dallanese_optimal_2014,yeh_adaptive_2012,farivar_optimal_2012} or using distributed algorithms \cite{zheng_fully_2016,zhang_optimal_2015,chen_robust_2017,robbins_two-stage_2013,turitsyn_distributed_2010}. There are several other distributed control methods proposed for PV inverter VVC which can be referred from the latest comprehensive survey papers \cite{antoniadou-plytaria_distributed_2017} and \cite{molzahn_survey_2017}.  However, the extensive communication requirements among the PV devices challenge the real-time implementation of these methods. Additionally, communication delays and the large time requirement to solve most OPFs limit their ability to respond to faster disturbances at seconds time scale such as cloud intermittency \cite{turitsyn_options_2011,carvalho_distributed_2008,zhu_fast_2016}. Though distributed algorithms are relatively faster, most of these methods assume constant substation voltage and rely on full feeder topology information for control parameter selection which is usually not fully known to the utilities or not always reliable. These issues make OPF based VVC methods difficult to implement and also vulnerable to fast external disturbances such as cloud transients, changes in substation voltage and topology changes. Therefore, we focus on the local VVC approaches in this work which are usually faster, simple to implement, and can respond to the sudden external disturbances in the distribution systems.

Among local approaches, droop VVC is the most popular local control framework among utilities and in the existing literature. It was first proposed by \cite{seal_standard_2010} which now has been adopted by the IEEE1547 integration standard \cite{noauthor_ieee_2018} and also being widely used by Rule 21 in California \cite{noauthor_rule_nodate}. Local control is simple to implement based on the local bus information, however, ensuring the system-wide control stability and performance is a challenge in the local control design. It has been identified an improper selection of control parameters can lead to control instability and voltage oscillation issues \cite{andren_stability_2015,farivar_equilibrium_2013,jahangiri_distributed_2013}. Most literature on droop control \cite{zhang_three-stage_2017,malekpour_dynamic_2017,karthikeyan_coordinated_2017} or other similar local control methods \cite{shah_online_2016,safavizadeh_voltage_2017} lack in analytical characterization and do not discuss the parameter selection and the control stability/convergence issues. Some work such as delayed droop control in \cite{jahangiri_distributed_2013}  discuss stability issue and scaled var control in \cite{zhu_fast_2016} provide rigorous performance analysis. But none of them adapt themselves in changing operating conditions and external disturbances to ensure control convergence. Another major challenge with droop control is its inherent inability to achieve a low steady state error (SSE) while ensuring convergence in all conditions. In other words, a certain slope selection which ensures the control stability, may also lead to high SSE as indicated by \cite{farivar_equilibrium_2013}; and as shown later in this paper, both the control stability and low SSE are crucial for the distribution systems operations. 
%\cite{zhu_fast_2016} improves the SSE but it requires full topology information and is not compatible with the standard local droop VVC framework.  

In this work, our focus is to analyze and design a droop-based local VVC which addresses following two challenges associated with the conventional droop VVC: a) To make the control parameters selection self-adaptive to changing operating conditions and external disturbances; and b) To achieve both low SSE and control stability simultaneously without compromising either. To achieve these objectives, we propose a fully local and real-time adaptive VVC within the IEEE1547 standard framework and provides its convergence and performance (SSE) analysis. We also compare our proposed VVC with the conventional droop VVC and it's improved version 'delayed' droop VVC \cite{jahangiri_distributed_2013} which improves the stability performance under normal condition but is vulnerable to control instability and high SSE under external disturbances due to lack of proper parameter selection, as detailed soon.
%However, these methods require full topology information for parameter selection and do not adapt themselves to changing operating conditions and disturbances. Moreover, they are not compatible with the IEEE1547 standard local droop VVC framework which may jeopardize their real-time implementation. \hl{Therefore, we focus on the standard droop based local control in this work.
 
%The droop based VVC has two major challenges, and addressing both of them is the main objective of this paper. The first challenge is to solve the issue of of appropriate control parameter selection under changing operating conditions and external disturbances. 
%The droop VVC is highly sensitive to its droop (slope) parameter, and an improper slope selection can lead to control instability or voltage oscillations} \cite{farivar_equilibrium_2013,jahangiri_distributed_2013}.\hl{ In fact, the existing standards also do not provide any guidelines on the parameter selection. 


%local droop VVC with high SSE becomes prone to voltage violations in external disturbances.
%However, the droop control is highly sensitive to its droop (slope) parameter and the existing standards do not provide guidelines on the parameter selection. Further, it has been shown by \cite{farivar_equilibrium_2013,jahangiri_distributed_2013} that the droop control is vulnerable to instability and voltage flicker due to improper selection of the parameter. The desired slope to ensure stability depends on the feeder topology and operating conditions. On the other hand, this slope selection might adversely affect the steady state performance of the control and leads to high steady state error (SSE) as indicated by \cite{farivar_equilibrium_2013}. As shown later in this paper, the control with high SSE is prone to voltage violations in external disturbances. 

%The delayed droop control, a variation of the conventional droop, is proposed by \cite{jahangiri_distributed_2013} which works well and improves the stability performance compared to the conventional droop under normal operating conditions; however, under external disturbances, it is vulnerable to instability and violations due to lack of real-time parameter adaption and high SSE respectively, as detailed soon in Section II. Thus, to ensure both control stability and set-point tracking accuracy (low SSE), we propose a fully local and real-time adaptive VVC where control parameters are made self-adaptive to  commonly occurring external disturbances such as cloud intermittency, cloud cover, changing load profile, and substation voltage changes without requiring centralized topology information.

% Overall, in this new environment of increasing renewables, demand response and other pro-active functionalities, the unexpected external disturbances in distribution system will become more common; and a standard and adaptive local VVC framework is needed to facilitate an easy “plug-and-play” implementation without reliance on much communication network, which is the main motivation behind this work.
%To meet these needs, we propose a fully local and real-time adaptive VVC to improve voltage regulation and to ensure control stability under a wide range of operating conditions/external disturbances. 
Our work extends the previous works and provides unique contributions in following way: 1) The proposed control achieves both low SSE and control stability simultaneously by decoupling the two objectives; 2) The control parameters are made self-adaptive to  commonly occurring external disturbances such as cloud intermittency, cloud cover, changing load profile, and substation voltage changes; 3) A theoretical analysis of convergence of the proposed adaptive VVC is discussed and a sufficient condition for convergence is derived; 4) It is compatible with the existing onboard droop controls specified in the recent standard (IEEE1547); 5) The real-time adaptive nature and tight voltage control feature of the proposed control opens interesting opportunities for operators to utilize PV inverters not only to mitigate over-voltage but for other volt/var related applications such as CVR, loss minimization, providing var support to transmission side etc.; and 6) A detailed modeling of secondary side of an unbalanced distribution system is used to verify the control approach with house-level loads and heterogeneous inverter population.

The layout of the remainder of the paper is as follows. In Section II, stability conditions and SSE expressions of the conventional droop VVC are derived and discussed to establish the base for the adaptive VVC development. Based on the analysis, the adaptive control strategy is developed in Section III. Section IV provides the convergence analysis of the proposed VVC with an illustration. Simulation results on the test system are discussed in Section V. Finally, concluding remarks are presented in Section VI.

\begin{figure}[t]
	\centering
	\includegraphics[trim=0in 0in 0in 0in,width=2.2in]{conventional_droop_deadband}
    \squeezeup
    \caption {Conventional droop VVC framework recommended by IEEE 1547 }
    \vspace{-2mm}
    \label{fig:std_droop}
   % \vspace{-1mm}
\end{figure}

\section{Background And Problem Setup }
Consider a general $N+1$ bus distribution system with one substation bus and $N$ load buses with PV inverters. The power flow equations for the system can be written as
\begin{equation}
\label{eq:power_flow}
\begin{split}
P^{inv}-P_d= g_p(V,\delta); \quad Q^{inv}-Q_d= g_q(V,\delta)\\
%Q^{inv}-Q_d= g_q(V,\delta)
\end{split}
\end{equation}
Where %$Q=[Q_2\: Q_3\: \dots\: Q_{N+1}]^T$ and $P^{inv}=[P_2\: P_3\: \dots\: P_{N+1}]^T$ 
$Q\!=\![Q_2 Q_3 \dots Q_{N+1}]^T$ and $P^{inv}\!=\![P_2 P_3 \dots P_{N+1}]^T$ are inverter reactive and real power injection vectors respectively at each bus. $P_d$ and $Q_d$ are similar vectors of real and reactive power loads at each bus. $g_p$ and $g_q$ are well-known power flow equations with voltage magnitude ($V$) and angles ($\delta$) as variables at all load buses \cite{grainger_power_1994}. 
The standard droop function $f_i (.)$ at $i^{th}$ bus is shown in \figurename\ref{fig:std_droop}. It is a piecewise linear function with a deadband $d$ and slope $m$. Assuming the operating point is in non-saturation region, the inverter var dispatch at time $t$ can be written as a function of previous voltage and other control parameters as
\begin{equation}
\label{eq:droop}
{{Q}_{i,{t+1}}}=f_i(V_{i,t})=-m_i({{V}_{i,t}}-\mu_i \pm d/2)
\end{equation}
where, $Q_{i,t}$ and $V_{i,t}$ are the inverter var injection and the voltage magnitude respectively. The sign $\pm d/2$ represents that $-d/2$ is used for $v_i>\mu_i$ and $+d/2$ is used for $v_i<\mu_i$. Subscripts  $i$ and $t$ denote $i^{th}$ bus and time instant $t$. $\mu_i$ is the reference voltage and $m_i$ is the slope of the curve.  We consider the same slope for both the regions in the droop control for a given inverter as shown in \figurename \ref{fig:std_droop}. $m_i$ can be maintained at desired value by changing control parameters as
\begin{equation*}
\label{eq:slope}
m_i=q_{i,max}/(\mu_i-d/2-v_{i,min})= q_{i,min}/(\mu_i+d/2-v_{i,max})
\end{equation*}
where $v_{i,min}$, $v_{i,max}$, $q_{i,min}$, $q_{i,max}$ are the four control set-points. It should be noted that, in the existing droop methods, these parameters are either constant or un-controlled. Whereas, in this work, these parameters are dispatched based on the proposed adaptive control strategy. As detailed soon, dynamic control over these parameters leads to more reliable control performance compared to the previous works.
\subsection{Stability Analysis}
As described in \cite{farivar_equilibrium_2013,jahangiri_distributed_2013},  the local droop VVC can be modeled as feedback dynamical system $\phi$ with $N$ states $[Q_{2,t}\;Q_{3,t}\; \dots\;Q_{N+1,t}]^T$ at discrete time $t$.
\begin{equation}
\label{eq:phi}
Q_{t+1}=\phi(Q_t)=f(h(Q_t))
\end{equation}
Where the vector $f(.)\!=\![f_2\;f_3\;\dots\;f_{N+1}]$ contains local VVC functions which map the current voltage vector $V_t$ to new inverter var injections vector $Q_{t+1}$ i.e. $Q_{i,t+1}\!=\!f_i (V_{i,t})$. The new var vector $Q_{t+1}$, in turn, leads to the new voltage vector $V_{t+1}$ according to power flow equations (1). The function $h$ is an implicit function vector derived from (1) i.e. $h_i (Q_{i,t} )=V_{i,t}$. It is shown in \cite{jahangiri_distributed_2013} that the system $\phi$ is locally stable in the vicinity of an equilibrium point ($\bar{Q}$) if all eigenvalues of the matrix $\partial \phi/\partial Q$ have magnitude less than 1.
\begin{equation}
\small
\label{eq:phi_slope}
\left[\frac{\partial\phi}{\partial Q}\right]_{Q=\bar{Q}}=
\left[\frac{\partial f}{\partial V}\right]
\left[\frac{\partial V}{\partial Q}\right]
\end{equation}
In the case of droop control, $\partial f/\partial V$ is a diagonal matrix with slope at each inverter as diagonal entries. 
\begin{equation}
\small
\label{eq:M}
\left[\frac{\partial f}{\partial V}\right]=-M=-diag(m_i)=-
\begin{bmatrix}
m_2 & \cdots & 0\\ 
\vdots & \ddots & \vdots\\
0 & \cdots & m_{N+1}\\ 
\end{bmatrix}
\end{equation}
Let’s define {\small $A\!=\!\partial V/\partial Q$} and  {\small $a_{ij}\!=\!\partial V_i/\partial Q_j$} which is a voltage sensitivity matrix with respect to var injection and can be extracted from the power flow Jacobian matrix from (1) as shown in \cite{jahangiri_distributed_2013}.   
\begin{equation}
\small
\label{eq:A}
\left[\Delta V\right]=\left[A\right]\left[\Delta Q\right]=
\begin{bmatrix}
a_{22} & \cdots & a_{2,N+1}\\ 
\vdots & \ddots & \vdots\\
a_{N+1,2} & \cdots & a_{N+1,N+1}\\ 
\end{bmatrix} \left[\Delta Q\right]
\end{equation} 
In other words, the sufficient condition for the control stability can be written as
\vspace{-2mm}
\begin{equation}
\small
\label{eq:rho}
\rho(MA)<1
\vspace{-3mm}
\end{equation}
Where $\rho$ is the spectral radius of a matrix which is defined as the largest absolute value of its eigenvalues. Condition (\ref{eq:rho}) provides useful information for evaluating the stability of specific inverter slope settings. However, in order to obtain information for selecting the inverter slopes, we will derive another conservative sufficient condition for stability using spectral radius upper bound theorem \cite{horn_matrix_2012}.

\textit{Theorem 1:} Let $\|.\|$ be any matrix norm on $\mathbb{R}^{n \times n}$ and let $\rho$ be the spectral radius of a matrix, then for all $X \in \mathbb{R}^{n \times n}$:
\begin{equation}
\label{eq:rho_X}
\rho(X)\leq ||X||
\end{equation}
\textit{Proposition:} If sum of each row of $MA$ is less than 1, i.e.
\begin{equation}
\label{eq:slope_cond}
m_i. \sum_{j=1}^{N}|a_{ij}|< 1 \quad \forall i ,
\end{equation}
Then the droop control will be stable i.e. $\rho(MA)<1$

\textit{Proof:} Using Theorem 1, if we apply {\small $\|.\|_\infty$} on {\small $MA$}, then,
{\small $\rho(MA)\leq \|MA\|_\infty=\max_{1\leq i\leq N}\sum_{j=1}^{N}|m_i.a_{ij}|$}. If condition (\ref{eq:slope_cond}) holds true i.e. {\small $m_i. \sum_{j=1}^{N}|a_{ij}|< 1 \quad \forall i$}, then the maximum of sum of rows will also be less than one. Thus, the upper bound on spectral radius will always be less than one i.e. {\small $\rho(MA)<1$}.  

\textit{Remark 1:} The condition (\ref{eq:slope_cond}) provides useful information for slope selection for each inverter to ensure control stability, i.e. {\small $m_i\!<\!m_i^c$}, where $m_i^c$ is critical slope given by 
% \begin{equation}
% \small 
% \label{eq:crit_slope}
$m_i^c\!=\!(\sum_j|a_{ij}|)^{-1}$
% \end{equation}
It should be noted that, usually, the entries of the sensitivity matrix $A$ do not remain constant. Changes in operating conditions (cloud cover, load changes) as well as changes in feeder topology lead to change in values of $a_{ij}$ and $m_i^c$; thus, they can potentially cause instability, if $m_i$ are not updated dynamically. Intuitively, entries of $A$ can also be seen as proportional to the reactance of the feeder lines \cite{farivar_equilibrium_2013} i.e. longer lines are more likely to have higher magnitude of $a_{ij}$ and lower value of critical slope. Therefore, PV inverters on rural network with longer lines, especially towards the feeder end, will be more sensitive to instability and their slope selection should be more conservative. Therefore, non-adaptive and homogeneous slope selection for all inverters make system prone to control instability. 
%Also, the un-controlled change in $q_{min}$ and $q_{max}$ with change in solar generation leads to undesired slope. For instance, in case of cloud cover, the generation drops and the $q_{max}$ limit is increased automatically leading to very high slope exactly when var support is not needed which creates stability/flicker issues. 
It is worth mentioning that an attempt to lower the effective slope by adding a delay block after droop in the delayed droop \cite{jahangiri_distributed_2013} improves the stability compared to the conventional droop i.e. $Q_{t+1}=f_i(V_{i,t})+\tau. Q_t$, where $\tau$ is a delay coefficient. However, because of its non-adaptive nature and un-controlled parameters, it may lead to issues under external disturbances and topology changes which will be illustrated through a comparison later in this section.

\subsection{Steady State Error (SSE) Concerns}
One of the major drawbacks of the droop control is the significant deviation from the set-point in steady state. To derive the analytical expression for SSE, let’s assume the system is at equilibrium point ($\overline{Q},\overline{V}$) at $t=0$. For simplicity, let's also assume $d=0$ in this analysis (it can be extended for non-zero $d$ values also, but this will make the analysis unnecessarily complicated, and will distract the reader from the main purpose of the section which is to illustrate the SSE concerns of the conventional droop controls). Control equation (2) can be written in vector form at $t=0$, as
\begin{equation}
\small
\label{eq:eqb}
[\overline{Q}]=-[M][\overline{V}-\mu]
\end{equation}
Now, consider an external disturbance perturb the equilibrium by causing sudden change in the voltage, $\Delta V^d$, at $t=0$ which changes the voltage at $t=0$ i.e. {\small $[V]_{t=0}=\overline V+ \Delta V^d$}. This drives control to dispatch the new var at $t=1$ i.e. $Q_{t=1}=-[M][V_{t=0}-\mu]$. Now, the following can be written,
\begin{equation}
\small
\label{eq:eqb}
[Q_{t=1}-\overline{Q}]=[\Delta Q]_{t=1}=-[M]\Delta V^d
\end{equation}
Using the similar procedure, following can be written for $t>0$
\begin{equation}
\small
\label{eq:diff1}
[\Delta Q]_{t+1}=-[M][\Delta V]_t
\end{equation}
Where {\small $[\Delta Q]_{t+1}=[Q_{t+1}-Q_t]$} and {\small $[\Delta V]_t=[V_t-V_{t-1}]$}.Using (6) and (12), we can write
\begin{equation}
\small
\label{eq:diff2}
[\Delta V]_{t+1}=-[A][M][\Delta V]_t
\end{equation}
\begin{equation}
\small
\label{eq:diff3}
[\Delta V]_{t+1}=[-A.M]^t[\Delta V]_{t=1}
\end{equation}
Using (6) and (11), replace {\small $[\Delta V]_{t=1}=A[\Delta Q]_{t=1}=-[AM]\Delta V^d$} in (14),
\begin{equation}
\small
\label{eq:diff3}
[V]_{t+1}=[V]_t+[-A.M]^{t+1}\Delta V^d
\end{equation}
By writing the (\ref{eq:diff3}) recursively and replacing {\small$[V]_{t=1}=(\overline V+\Delta V^d)-[AM]\Delta V^d$}, 
\begin{equation}
\small 
\label{eq:diff4}
[V]_{t+1}=\overline V+\sum_{i=0}^{t+1}[-A.M]^{i}\Delta V^d
\vspace{-1mm}
\end{equation}
In this case, the geometric progression series of matrices only converges if the condition (\ref{eq:rho}) holds true (the stable case). The new equilibrium voltage can be written as
\begin{equation}
\small
\label{eq:lim}
\lim_{t\to \infty}[V]_{t+1}=\overline V+[I+A.M]^{-1}\Delta V^d
\end{equation}
Finally, SSE vector can be written as
\begin{equation}
\small
\label{eq:sse}
SSE = \lim_{t\to \infty}[V]_{t+1} - \mu
\end{equation}
Equation (\ref{eq:lim}) and (\ref{eq:sse}) show that for a given disturbance, the only way to decrease SSE is to set higher values of slopes $m_i$ which in turn may violate stability condition (\ref{eq:slope_cond}). Usually, SSE is compromised to ensure control stability.
%In fact, in most cases, SSE is compromised to ensure control stability. It can also be shown that the delayed droop \cite{jahangiri_distributed_2013} does not improve SSE compared to conventional control. 

\textit{Remark 2:} Note that, in some cases, it might be possible to maintain voltages within the American National Standard Institute (ANSI) allowable limits with high SSE, close to the boundaries, for a given system condition. But, any external disturbance can instantly push the voltages out of the limits as illustrated later. Moreover, other than complying with ANSI standard, the tight voltage regulation capability (low SSE) makes the system more flexible and provides extra room to the operator to perform other voltage-dependent applications such as CVR, loss minimization etc.; thus fully utilizing the PV inverter’s capability. It can also be shown easily that the delayed droop has the same SSE as the conventional droop.

\begin{figure} [t]
	\centering
	\includegraphics[trim=0in 0in 0in 0in,width=2.6in]{toyExampleCircuit_switch}
    \vspace{-1mm}
    \caption {A small 4 bus system to illustrate the impact of external disturbances }
    \label{fig:toy_ckt}
\vspace{2mm}

	\centering
	\includegraphics[trim=0.0in 0in 0in 0in,clip,width=3.3in]{2busforPaper_latest}
    \vspace{-2.5mm}
    \caption {Non-adaptive droop VVC performance under impact of a) substation voltage change at conservative slope setting; b) sudden cloud cover at non-conservative setting and; c) topology change at non-conservative setting.}
    \label{fig:toy_results}   
    \vspace{-0.5mm}
\end{figure}

\subsection{Illustration}
To corroborate the above analysis, we will illustrate the impact of external disturbances using a small modified IEEE 4 bus test system shown in \figurename\ref{fig:toy_ckt}. 600 kW load and 900 kW solar generation is added at node 3. A similar node 4 is added via a normally open switch to simulate the change in feeder topology. We will consider two types of initial slope settings to convey the main outcome of the analysis i.e. conservative ($m=1$) and non-conservative ($m=6$). Solar generation is applied at $t=20$ to observe the impact of VVC with $\mu=1$ at node 3 voltage profile. 
\figurename\ref{fig:toy_results}(a) demonstrates how conservative setting causes high SSE (though, within the ANSI limit initially) for the droop controls (both conventional and delayed droop) which leads to over-voltage violation due to a small change in substation voltage from 1.03-1.05 pu at $t=80$. On the other hand, using non-conservative setting to reduce SSE makes the system prone to control instability or voltage flicker as shown in \figurename\ref{fig:toy_results}(b) and (c). Conventional droop is not shown as it is always unstable in these cases. \figurename\ref{fig:toy_results}(b) shows that a sudden drop in solar generation due to cloud cover at $t=80$ increases $q_{max}$ and makes the slope very high which causes voltage oscillations. Further, to simulate the impact of topology change or error in topology information, switch1 is closed at $t=80$. Delayed droop, as discussed before, is stable under normal conditions, however, change in feeder topology leads to voltage oscillations as shown in \figurename\ref{fig:toy_results}(c) at non-conservative settings. This example demonstrates that it is difficult to achieve both low SSE and control stability under external disturbances with the existing droop controllers. Moreover, this problem becomes more crucial in a large realistic system due to thousands of independent inverter devices, higher possibility of inaccuracy in topology information and in parameter selection, and increasing disturbances in this new environment of pro-active distribution system.
% Overall, in this new environment of increasing renewables, demand response and other pro-active functionalities, the unexpected external disturbances in distribution system will become more common; and a standard and adaptive local VVC framework is needed to facilitate an easy “plug-and-play” implementation without reliance on much communication network, which is the main motivation behind this work.

Therefore, our intention is to develop a new droop based adaptive VVC strategy 1) to achieve both low SSE and low voltage oscillations (stability) simultaneously; 2) to make control parameters dynamically self-adaptive to external disturbances in real-time.


%\vspace{-2.5mm}
% \section{Adaptive Control Strategy}
% This section will introduce the proposed adaptive local VVC function $f_i^p (V_{i,t})$ which can be written as follows:
% \setlength\abovedisplayskip{6pt}
% \setlength\belowdisplayskip{6pt}
% \begin{equation}
% \small
% \label{eq:adaptive_droop}
% {{Q}_{i,{t+1}}}=f^p_i(V_{i,t})=\mathbb{P}[q^p_{i}-m^p_i({{V}_{i,t}}-\mu_i)]
% \end{equation}
% Where $\mathbb{P}$ is the saturation operator with $(q_{min,i}^p,q_{max,i}^p)$ as saturation var limit parameters applied at cut-off parameters $(v_{max,i}^p,v_{min,i}^p)$. $q_i^p$ is error adaptive parameter and its main function is to provide SSE correction. Desired adaptive slope $m_i^p$ can be set as,
% \begin{equation}
% \small
% \label{eq:adaptive_slope}
% m_i^p=\frac{q_{min,i}^p-q_i^p}{\mu_i+d/2-v_{max,i}^p}=\frac{q_{max,i}^p-q_i^p}{\mu_i-d/2-v_{min,i}^p}
% \end{equation}
% There are two unique features of this control. First, the functions of maintaining control stability and low SEE are decoupled. Two different parameters $m^p$ and $q^p$ are used to achieve control stability and low SSE respectively with different approaches so that none of the objectives are compromised. Second, all these parameters are dynamically adapted in real-time. Superscript $p$ denotes the adaptive nature of the control parameters. To achieve this, a two-layer control framework is proposed as shown in \figurename \ref{fig:framework}. The inner layer is a fast VVC function $f_i^p (V_{i,t})$ to track the desired set-point $\mu$ according to (\ref{eq:adaptive_droop}).  The outer layer dispatches the control parameters $(m_i^p,q_i^p)$ based on the proposed adaptive algorithm described later in the section. The outer layer works on a relatively slower time scale $(t_o)$ to allow inner fast control to reach steady state before dispatching new control parameters, thus avoiding hunting and over-corrections. Control time-line is shown in \figurename \ref{fig:timeline}. Control parameters are updated at every period $T$, control horizon of the outer loop control. Each iteration of the inner and outer loop control is denoted by $t_{in}$ and $t_o$ respectively. 
% 	The outer loop adaptive algorithm consists of two strategies where $q_i^p$ and $m_i^p$ are dynamically adapted to take care of SSE and voltage instability/flicker respectively as described below.

% \subsection{Error Adaptive Control: Strategy I}
% The aim of the strategy I is to minimize SSE by utilizing var resources efficiently. Therefore, {\small$SSE_{avg}$} is used as the control criteria and defined for each outer loop as
% \setlength\abovedisplayskip{4pt}
% \setlength\belowdisplayskip{4pt}
% \begin{equation}
% \small
% \label{eq:sse_avg}
% % SSE_{avg,i}(t_0)=\sum_{t_i=1}^{T} \frac{V_{t_0t_{in},i}-\mu_i}{T}
%  SSE_{avg,i}(t_0)=\sum_{t_{in}=1}^{T}({V_{t_0t_{in},i}-\mu_i})/{T}
% \end{equation}
% $SSE_{avg,i}$ denotes the average set-point deviation of voltage at $i^{th}$ inverter bus. A tolerance band for $SSE_{avg,i}$ can be defined as $\mu_i\pm\epsilon_{sse}$, where $\epsilon_{sse}$ is tolerance for the deviation.

% In this strategy, the adaptive term $q_i^p (t_o)$ in (20) is updated at each outer loop interval $t_o$, based on $SSE_{avg,i}$ during the last time horizon $T$ as
% \begin{equation}
% \small
% \label{eq:qp_update}
% q^p_i (t_o )=q^p_i (t_o-1)-k_i^d.SSE_{avg,i}(t_o) 
% \end{equation}
% It is important to note that {\small $SSE_{avg}$} is used as an algebraic value with sign. The sign of the error decides whether $q_i^p$ needs to be moved positive or negative. If the voltage settles on a higher value than the set point, a negative term is added in 
% $q_i^p$ to facilitate more var absorption to lower the voltage. Similarly, a positive term is added in $q_i^p$ to provide more var when voltage settles lower than the set point. A constant {\small $k_i^d\!>\!0$} is a correction factor. A higher $k_i^d$  brings $SSE_{avg}$ within the desired range faster and vice versa.

% \figurename \ref{fig:strategy2} depicts the adaptive control $f_i^p (V_{i,t})$ with different $q_i^p$ values. It should be noted that the solid curve with $q_i^p=0$ is same as the conventional droop control $f_i (V_{i,t})$ in (2). \figurename \ref{fig:strategy2} brings out an important feature of the proposed control $f_i^p (V_{i,t})$ that it can be seen as “shifted and adaptive” droop VVC which makes it compatible with integration standards.

% While it is possible to defer real power solar generation, in this work the consumer value is maximized by limiting var output to leftover capacity and not deferring real power generation. To utilize the inverter capacity entirely, $q_{max}^p$ and $q_{min}^p$ are also updated in every outer loop as 
% \begin{equation}
% \small
% % \setlength\abovedisplayskip{6pt}
% % \setlength\belowdisplayskip{6pt}
% \label{eq:inv_capacity}
% q_{max}^p(t_o)\!=\!\sqrt{s^2\!-\!p_{pv}^2(t_o)}; q_{min}^p(t_o)\!=\!-\!\sqrt{s^2\!-\!p_{pv}^2(t_o)}
% \end{equation}
% where, $s$ is inverter rating and $p_{pv}(t_o)$ is the average solar PV real power generation in the last outer loop time interval. 

% \begin{figure}
% 	\centering
% 	\includegraphics[trim=0in 0in 0in 0in,width=2.5in]{strategy2}
%     \caption {Adaptive VVC with different error adaptive parameter $q_p$}
%     \label{fig:strategy2}
%     \squeezeup
% \end{figure}

% \subsection{SSE correction in Adaptive Control}
% In this section, we will verify how the proposed control (\ref{eq:adaptive_droop}) helps to mitigate the SSE. Consider the system is at equilibrium point {\small$(\overline{Q},\overline{V})$} at $t=0$ with {\small $SSE=\overline{V}-\mu$}. Now if the parameter $q^p$ is changed at $t=0$ by $\Delta q^p$, the new voltage deviation ($SSE^{adp}$) can readily be obtained by following the procedure provided in the Section II.B  by replacing the conventional droop (\ref{eq:droop}) with the adaptive control (\ref{eq:adaptive_droop}):
% \begin{equation}
% \small
% \label{eq:sse_adp}
% SSE^{adp}=\overline{V}-\mu+[I-AM]^{-1}A\Delta q^p
% \end{equation}
% To achieve $SSE^{adp}=0$, $\Delta q^p$ required will be,
% \begin{equation}
% \small
% \label{eq:sse_adp2}
% \Delta q_{req}^p=-(A^{-1}-M)SSE
% \end{equation}
	
% Equation (\ref{eq:sse_adp2}) provides the analytical expression of the required change in $q^p$ parameter to achieve zero SSE in just one iteration. However, the solution requires the information of $A$ matrix, SSE and slope $(M)$ at all inverter buses which is not available to local bus controllers. Moreover, estimation of $A$ is contingent to error in centralized feeder topology information and might not be reliable. Interestingly, our proposed local update strategy {\small$\Delta q_i^p (t_o )\!=\!-k_i^d.SSE_{avg,i} (t_o )$} is local version of the analytical solution (\ref{eq:sse_adp2}) and is able to correct SSE, though, it may take more than one iterations to achieve near zero SSE. The important part is that it requires only local bus information, making it purely local and more feasible. Value of $k_i^d$ can be decided once from the offline studies. Nevertheless, the update strategy can always be made faster and more accurate using (\ref{eq:sse_adp2}), if information at other nodes is also available in future.

\vspace{-2mm}
\section{Adaptive Control Strategy}
This section will introduce the proposed adaptive local VVC function $f_i^p (V_{i,t},cp_i)$ which can be written as follows:
\setlength\abovedisplayskip{6pt}
\setlength\belowdisplayskip{6pt}
\begin{equation}
\small
\label{eq:adaptive_droop}
{{Q}_{i,{t+1}}}=f^p_i(V_{i,t},cp_i)=\mathbb{P}[q^p_{i}-m^p_i({{V}_{i,t}}-\mu_i)]
\end{equation}
Where {\small $cp_i=[m_i^p,q_i^p,q_{min,i}^p,q_{max,i}^p,v_{max,i}^p,v_{min,i}^p,]$} are control parameters. The Function {\small $\mathbb{P}$} is a saturation operator with {\small$(q_{min,i}^p,q_{max,i}^p)$} as saturation var limit parameters applied at cut-off parameters {\small $(v_{max,i}^p,v_{min,i}^p)$}. Variable $q_i^p$ is an error adaptive parameter and its main function is to provide SSE correction. Desired adaptive slope $m_i^p$ can be set as,
\begin{equation}
\small
\label{eq:adaptive_slope}
m_i^p=\frac{q_{min,i}^p-q_i^p}{\mu_i-v_{max,i}^p}=\frac{q_{max,i}^p-q_i^p}{\mu_i-v_{min,i}^p}
\end{equation}
There are two unique features of this control. First, the functions of maintaining control stability and low SEE are decoupled. Two different parameters $m^p$ and $q^p$ are used to achieve control stability and low SSE respectively with different approaches so that none of the objectives are compromised. Secondly, all these parameters are dynamically adapted in real-time. Superscript $p$ denotes the adaptive nature of the control parameters. To achieve this, a two-layer control framework is proposed as shown in \figurename \ref{fig:framework}. The inner layer is a fast VVC function $f_i^p (V_{i,t})$ to track the desired set-point $\mu_i$ according to (\ref{eq:adaptive_droop}).  The outer layer dispatches the control parameters $(cp_i)$ based on the proposed adaptive algorithm described later in the section. The outer layer works on a relatively slower time scale $(t_o)$ to allow inner fast control to reach steady state before dispatching new control parameters, thus avoiding hunting and over-corrections. Control time-line is shown in \figurename \ref{fig:timeline}. Control parameters are updated at every period $T$, control horizon of the outer loop control. Each iteration of the inner and outer loop control is denoted by $t_{in}$ and $t_o$ respectively. 
	The adaptive algorithm consists of two strategies where $q_i^p$ and $m_i^p$ are dynamically adapted to take care of SSE and voltage instability/flicker respectively as described below.

\begin{figure}[h]
	\centering
	\includegraphics[trim=0in 0in 0in 0in,width=3.42in]{TwoLayerFrameworkBlackTRevised}
    \vspace{-1mm}
    \caption {Two-layer framework of the proposed adaptive control approach }
    \vspace{2mm}
    \label{fig:framework}
	\centering
	\includegraphics[trim=0in 0in 0in 0in,width=2.6in]{timeline_revised}
    \squeezeup
    \caption {Time-line of adaptive inner and outer loop control }
    \label{fig:timeline}
    \vspace{-3mm}
\end{figure}

\subsection{Error Adaptive Control: Strategy I}
The aim of the strategy I is to minimize SSE by adapting the error adaptive parameter $q_i^p$. We will analyze how the proposed control (\ref{eq:adaptive_droop}) helps to mitigate the SSE and accordingly develop a mechanism to adapt $q_i^p$ locally. Consider the system is at equilibrium point {\small$(\overline{Q},\overline{V})$} at $t=0$ with {\small $SSE=\overline{V}-\mu$}. Now if the parameter $q^p$ is changed at $t=0$ by $\Delta q^p$, the new voltage deviation ($SSE^{adp}$) can readily be obtained by following the procedure provided in the Section II.B  by replacing the conventional droop (\ref{eq:droop}) with the adaptive control (\ref{eq:adaptive_droop}):
\begin{equation}
\small
\label{eq:sse_adp}
SSE^{adp}=\overline{V}-\mu+[I+AM^p]^{-1}A\Delta q^p
\end{equation}
To achieve $SSE^{adp}=0$, $\Delta q^p$ required will be,
\begin{equation}
\small
\label{eq:sse_adp2}
\Delta q_{req}^p=-(A^{-1}+M^p)SSE
\end{equation}	
Equation (\ref{eq:sse_adp2}) provides the analytical expression of the required change in $q^p$ parameter to achieve zero SSE in just one iteration. However, this solution requires the information of $A$ matrix, SSE and slope $(M)$ at all inverter buses which is not available to local bus controllers. Moreover, estimation of $A$ is contingent to error in centralized feeder topology information and may not be reliable. Therefore, we propose a local version of the analytical solution (\ref{eq:sse_adp2}) i.e. {\small $\Delta q_i^p\!=\!-k_i^d.SSE_{avg,i}(t_o)$}, where {\small$SSE_{avg}$} defined for each outer loop as
\setlength\abovedisplayskip{2pt}
\setlength\belowdisplayskip{4pt}
\begin{equation}
\small
\label{eq:sse_avg}
% SSE_{avg,i}(t_0)=\sum_{t_i=1}^{T} \frac{V_{t_0t_{in},i}-\mu_i}{T}
 SSE_{avg,i}(t_0)=\sum_{t_{in}=1}^{T}({V_{t_0t_{in},i}-\mu_i})/{T}
\end{equation}
$SSE_{avg,i}$ denotes the average set-point deviation of voltage at $i^{th}$ inverter bus. A tolerance band for $SSE_{avg,i}$ can be defined as $\mu_i\pm\epsilon_{sse}$, where $\epsilon_{sse}$ is tolerance for the deviation. In this strategy, the adaptive term $q_i^p (t_o)$ in (\ref{eq:adaptive_droop}) is updated at each outer loop interval $t_o$, based on the real-time estimation of $SSE_{avg,i}$ during the last time horizon $T$ as
\setlength\abovedisplayskip{4pt}
\setlength\belowdisplayskip{4pt}
\begin{equation}
\small
\label{eq:qp_update}
q^p_i (t_o )=q^p_i (t_o-1)-k_i^d.SSE_{avg,i}(t_o) 
\end{equation}
Since the $SSE_{avg,i}$ requires only local voltages and only needs to be calculated in the outer loop, there is enough room to calculate this variable without causing any extra delay in the control. It is important to note that {\small $SSE_{avg}$} is used as an algebraic value with sign. The sign of the error decides whether $q_i^p$ needs to be moved positive or negative. If the voltage settles on a higher value than the set point, a negative term is added in $q_i^p$ to facilitate more var absorption to lower the voltage. Similarly, a positive term is added in $q_i^p$ to provide more var when voltage settles lower than the set point. A constant {\small $k_i^d\!>\!0$} is a correction factor which can be decided once from the offline studies.
%A higher $k_i^d$  brings $SSE_{avg}$ within the desired range faster and vice versa.
It's selection  affects the convergence speed of the control which is discussed in detail in Section IV.
\figurename \ref{fig:strategy2} depicts the adaptive control $f_i^p (V_{i,t},cp_i)$ with different $q_i^p$ values. Note that the solid curve with {$q_i^p\!=\!0$} is same as the conventional droop control $f_i (V_{i,t})$ in (2). \figurename \ref{fig:strategy2} brings out an important feature of the proposed control that it can be seen as ``shifted and adaptive" droop VVC which makes it compatible with the integration standards. 

{Nonetheless, it should be noted that the proposed approach may take more than one iterations to achieve near zero SSE, unlike the analytical solution. However, it is compensated by the advantage that it requires only local bus information, thus, making it more feasible. Nevertheless, the update strategy can always be made faster and more accurate using} (\ref{eq:sse_adp2}),{ if information at other nodes is also available in the future. A more detailed theoretical convergence analysis of this adaption strategy is discussed in Section IV.} 

\begin{figure}
	\centering
	\includegraphics[trim=0in 0in 0in 0in,width=2.2in]{strategy2}
    \caption {Adaptive strategy I: VVC droop curve with adaptive error adaptive parameter $q_p$ }
    \label{fig:strategy2}
    \squeezeup
\end{figure}

\subsection{Adaptive Slope Control: Strategy II}
The objectives of the strategy II are to ensure stability as well as to keep voltage fluctuations within the IEEE 141 standard\cite{ieee_141_1994} limit by adapting parameter $m^p_i$. Therefore, based on \cite{ieee_141_1994}, we use short-term voltage flicker (VF) as control criteria which is calculated for each inverter bus at the beginning of each outer loop as
\setlength\abovedisplayskip{2pt}
\setlength\belowdisplayskip{2pt}
\begin{equation}
\small
\label{eq:vf}
VF(t_o)=\sum_{t_i=1}^{T}\frac{(V_{t_ot_i}- V_{t_ot_i-1})/ V_{t_0t_i}}{T}\times 100
\end{equation}

As seen in (15), voltage fluctuations are proportional to slope and can be reduced by decreasing $m_i$. For this purpose, the voltage flicker range is divided into four control regions as shown in \figurename \ref{fig:flicekr_region}. The IEEE 141 flicker curve provides the maximum fluctuation limit {\small $(\overline{VF_{lim}})$} beyond which we define as \textit{critical flicker zone}. The same standard also gives a borderline flicker limit {\small $(VF_{lim})$}. The region between {\small$(\overline{VF_{lim}})$} and $VF_{lim}$ is termed as the \textit{subcritical flicker zone}. Further we define a tolerance {\small $(VF_{lim}\!-\!\epsilon_{vf})$} and the tolerance band is termed as the \textit{safe flicker zone}. The region below safe flicker zone is defined as the \textit{relaxed flicker zone}. In critical zone, we update the parameters by a larger amount {\small$(\overline{\Delta_{vf}})$} to avoid control instability and to return to subcritical zone faster. In subcritical zone, the slope is decreased in a smaller step {\small$(\Delta_{vf})$} to avoid over-correction which might impact  {\small$SSE_{avg}$}  negatively. As soon as we reach the safe zone, no control action is taken. This is the desired range of control parameters. Though rarely required, in the relaxed zone, slope is increased to improve SSE only if SSE is out of range. Correction factors {\small$(\Delta_{vf})$} are estimated offline in this work based on the sensitivity analysis and engineering judgment. The amount of slope change required to reduce $VF$ from {\small$(\overline{VF_{lim}})$} to {\small$({VF_{lim}})$} can be taken as an approximate value of {\small$\Delta_{vf}$}. Almost twice of  {\small$\Delta_{vf}$} can be taken as {\small$\overline{\Delta_{vf}}$}. They can also be made responsive to the online control performance, if required.   
\begin{figure}[]
%\vspace{-0.5em}
	\centering
	\includegraphics[trim=0in 0in 0in 0in,width=2.2in]{flicker_region}
    \caption {Control action region for adaptive outer loop control strategy II for flicker mitigation }
    \label{fig:flicekr_region}
    \vspace{-2mm}
\end{figure}
\begin{figure}[]
	\centering
	\includegraphics[trim=0in 0in 0in 0in,width=2in]{strategy1}
    \caption {Adaptive strategy II: changing slope of droop curve by changing $v_{min}$ and $v_{max}$ parameters to keep flicker in the limit }
    \label{fig:strategy1}
    \squeezeup
\end{figure}

It is worth noting here that the the main feature of this strategy lies in the decoupling of the two functionalities i.e. SSE and slope. Since SSE is catered by $q_i^p$, slope can always be in the conservative range (safe or relaxed zones) to ensure control stability. {This might cause a momentarily high SSE until the strategy I adapts the SSE again but it prevents the possibility of control instability over large range of operating points}. In this work, we use the earlier derived condition (\ref{eq:slope_cond}) to choose initial slopes. It is estimated using offline studies for the base case, however, to keep safe margin it can be further reduced by a certain factor. \figurename\ref{fig:strategy1}. depicts the control strategy II with adaptive $m^p_i$.

While it is possible to defer real power solar generation, in this work the consumer value is maximized by limiting var output to leftover capacity and not curtailing real power generation. To utilize the inverter capacity entirely, $q_{max}^p$ and $q_{min}^p$ are also updated in every outer loop as 
\begin{equation}
\small
% \setlength\abovedisplayskip{6pt}
% \setlength\belowdisplayskip{6pt}
\label{eq:inv_capacity}
q_{max}^p(t_o)\!=\!\sqrt{s^2\!-\!p_{pv}^2(t_o)}; q_{min}^p(t_o)\!=\!-\!\sqrt{s^2\!-\!p_{pv}^2(t_o)}
\end{equation}
where, $s$ is inverter rating and $p_{pv}(t_o)$ is the average solar PV real power generation in the last outer loop time interval. {Note that ideally we would want to use $p_{pv}$ forecast for the next outer loop. Various established methods for short-term solar PV forecasting can be used for this purpose such as statistical or neural network based approaches} \cite{pvforecast_2015}.{ However, since this is not the focus of this paper, for simplicity we use the simplest prediction available which is the previous $p_{pv}$ value.}   

Thus, we get the new parameters {\small $q_i^p (t_o)$} from strategy I, {\small $m_i^p(t_o)$} from strategy II and {\small $ q_{min,i}^p(t_o), q_{max,i}^p(t_o)$} from (\ref{eq:inv_capacity}). 
%Where $m(t_o)$ is now redefined to include impact of $q_p$ as shown in   
% \begin{equation}
% \label{eq:22}
% m(t_o)=\frac{q_{max}-q_p(t_o)}{\mu-v_{min}(t_o)}=\frac{q_{min}-q_p(t_o)}{\mu-v_{max}(t_o)}
% \end{equation}
Finally, {\small $v_{min,i}^p(t_o)$  and $v_{max,i}^p(t_o)$} parameters are calculated using (\ref{eq:adaptive_slope}) %and (\ref{eq:24})
and dispatched to be used in the inner loop. Overall detailed algorithm of the adaptive control is shown in \figurename \ref{fig:algorithm}. 
%\vspace{-0.5em}
% \begin{equation}
% \setlength\abovedisplayskip{6pt}
% \setlength\belowdisplayskip{6pt}
% \label{eq:23}
% \begin{split}
% v_{min}(t_o)=\mu-({q_{max}(t_o)-q_p(t_o)})/{m(t_o)}\\
% v_{min}(t_o)=\mu-({q_{max}(t_o)-q_p(t_o)})/{m(t_o)}
% \end{split}
% \end{equation}

\begin{figure}[]
\vspace{-2mm}
\small
\setlist[enumerate,1]{wide=\parindent}
\setlist[enumerate,2]{leftmargin=4.5em}
\setlist[enumerate,3]{leftmargin=4em}
\renewcommand\labelenumi{\theenumi.}
\renewcommand\labelenumii{\theenumi.\arabic{enumii}.}
\renewcommand\labelenumiii{}
\begin{tabular}{p{0.45\textwidth}}
\hline\\[-1.5ex]
\textbf{Algorithm 1}: Adaptive control scheme\\
\hline\\\vspace{-1.5em}
\begin{enumerate}
\item Real-time measurement and control criterion calculation
	\begin{enumerate}
	\item Collect $V_{t=t_o.t_{in}} \forall 			t_{in}=1,2,\dots,n$
	\item Calculate $SSE_{avg}(t_o)$ and $VF(t_o)$
	\end{enumerate}
\item Go to adaptive strategy I: error adaptive   
	\begin{enumerate}
	\item 	If $|SSE_{avg}(t_o)|>\mu+\epsilon_{sse}$
    	\begin{enumerate}
    	\item $q_p(t_o)=q_p(t_o-1)-k_d.SSE_{avg}(t_o)$
	  	\end{enumerate}
	\item Else, $q_p (t_o )=q_p (t_o-1)$
    \end{enumerate}    
\item Go to adaptive strategy II: slope adaptive

\begin{center}
$m^p(t_0)= m^p{(t_0-1)}+\Delta_{m}$
\end{center}
	\begin{enumerate}
	\item If $VF(t_o) > \overline{VF_{lim}}$ \hspace{56pt}	
		$\Delta_m=-\overline{\Delta_{vf}}$
	\item Else if $VF (t_0 )>VF_{lim}$   \hspace{37pt}	
		$\Delta_m=-{\Delta_{vf}}$
    \item Else if $VF (t_0 )>(VF_{lim}-\epsilon_{vf})$  \hspace{6pt}	
		$\Delta_m=0$
	\item Else, check if $|SSE_{avg}|>\mu+\epsilon_{sse}$ \hspace{-2pt}	
		$\Delta_m={\Delta_{vf}}$
	\end{enumerate}
\item Update $q_{max}(t_0)$ and $q_{min}(t_0)$: equation (\ref{eq:inv_capacity})
\item Update final parameters $v_{min}$ and $v_{max}$: equation (\ref{eq:adaptive_slope})
\item  $t_o=t_o+1$, go to step 1
\vspace{-1em}
\end{enumerate}\\
\hline 
\end{tabular}
%\vspace{0.1em}
\caption{Overall algorithm of the proposed adaptive control strategy}
\label{fig:algorithm}
\end{figure}
\section{{Convergence of the Proposed Local Adaptive Control Algorithm}}
The convergence properties and conditions of the proposed adaptive control (\ref{eq:adaptive_droop}) will be investigated in this section. Since it's a two-layer control, we need to study the convergence of both the control loops. If we assume the time horizon $T$ is sufficient for faster control to reach its steady state, the inner loop control equation within the time $T$ can be written as
\vspace{1em}
\begin{equation}
\label{eq:inner_loop}
{{Q}_{i,{t_{in}+1}}}=q^p_{i}-m^p_i({{V}_{i,t_{in}}}-\mu_i)
\end{equation}
Where adaptive parameters $q_i^p$ and $m_i^p$ are constant for the time horizon $T$. Using the stability analysis performed in Section II, a sufficient convergence criteria for (\ref{eq:inner_loop}) can be derived which is same as given by condition (\ref{eq:slope_cond}) and remark 1, i.e. $m_i^p<m_i^c$. Therefore, the inner local control will always converge if the chosen slopes are below critical slope values. Adaptive control strategy II helps to maintain this condition by keeping slope in conservative range as discussed earlier.

\subsection{Outer Loop Control Convergence}
%It is safe to assume here the faster inner control loop reaches its steady state before the next iteration of outer loop. 
%The proposed adaptive update of parameter $q^p$in outer loop is given by (\ref{eq:qp_update}) to minimize SSE.
To derive the analytical expression for the outer loop convergence criteria, lets assume the system is currently at outer loop iteration $t_o$ for which inner control has already reached its steady state ($\overline{V_{t_o}}$,$\overline{Q_{t_o}}$). This can be represented as
\vspace{2mm}
\begin{equation}
\small
\label{eq:outer_loop1}
{[\overline{Q}]_{t_o}}=[q^p]_{t_o}-[M^p] [S]_{t_o}
\end{equation}
Where, let's represent the $SSE_{t_0}$ vector with a shorter notation $ S_{t_o}=[\overline{V}_{t_o}-\mu]$. Now the $q^p$ parameter is locally updated for outer loop iteration $(t_o+1)$ based on the update formula given in (\ref{eq:qp_update}) i.e. $q^p_{i,{t_o+1}}=q^p_{i, t_o}-k_i^d.S_{i,(t_o)}$. Now a new updated inner loop steady state ($\overline{V}_{t_o+1}$,$\overline{Q}_{t_o+1}$) is reached for the outer loop iteration $(t_o+1)$ which can be written in vector form as

\begin{equation}
\small
\label{eq:outer_loop2}
{[\overline{Q}]_{t_o+1}}=[q^p]_{t_o}-[K^d][S]_{t_o}-[M^p] [S]_{t_o+1}
\end{equation}
Where $K^d$ is a diagonal matrix with the correction factor $k_i^d$ at each bus as its diagonal entries. Since the objective is to minimize SSE, we can replace all other state variable with $S_{t_o}$ by subtracting (\ref{eq:outer_loop2}) from (\ref{eq:outer_loop1}) and using (\ref{eq:A}) i.e. $\Delta V = A \Delta Q$ as following:
\begin{equation}
%\small
\label{eq:sse_conv}
[S]_{t_o}-[S]_{t_o+1}= A\Big[K^d.S_{t_o}-M^p(S_{t_o}-S_{t_o+1})\Big]
\end{equation}
After manipulating (\ref{eq:sse_conv}), the outer loop control can be written as a discrete feedback dynamical system with SSE as the only state variable as following.
\begin{equation}
%\small
\label{eq:sse_conv2}
[S]_{t_o+1}= B[S]_{t_o}
\end{equation}
Where $B=I-[I+AM^p]^{-1}AK^d$ is a constant matrix which defines the convergence behavior of the outer loop control. We know that any linear discrete system $x[k+1]=Bx[k]$ is asymptotically stable if and only if all eigenvalues of $B$ have magnitude less than 1 \cite{chen_linear_2013}. Consequently, the convergence condition for the outer loop control can be written as 
\begin{equation}
%\small
\label{eq:conv_cond}
\rho(B)<1
\end{equation}
Where $\rho$ is spectral radius of a matrix. Therefore, the convergence of the outer loop is ensured as long as the selection of $K^d$ matrix does not violate the condition (\ref{eq:conv_cond}). In such cases, the outer loop control system (\ref{eq:sse_conv2}) will always converge to zero SSE for any amount of initial SSE i.e. $\lim_{t_o\to\infty} S_{t_o}=0$ for any $S_{t_o=0}$.

\textit{Remark 3:}  {It is interesting to observe here if we choose a non-diagonal $K^d=A^{-1}+M^p$, $B$ turns out to be zero and the SSE can be converged to zero in just one iteration. However, in that case, the $q^p$ update ($q^p_{t_0+1}=q^p_{t_0}-K^d.S_{t_o}$) does not remain local i.e. to update $q^p_i$ at $i^{th}$ node, non-diagonal entries $k^d_{ij}$ need to be multiplied with SSE at all other $j^{th}$ nodes. Therefore, we compromise with the convergence speed to take advantage of the local feature of the control.}

For a better understanding of this analysis, let's consider a simple example system described in \figurename \ref{fig:toy_ckt}. Note that this example system can be seen as an equivalent two-bus system as there is only one load bus (bus 3) when switch is open. In that case, $B=b$ and $K=k$ will be a scalars, $b=1-k/(a^{-1}+m)$. The sensitivity matrix $A=a_{33}=0.2857$ var pu/volt pu is calculated offline for the base case. Using %(\ref{eq:crit_slope}), 
$m_i^c=(\sum_j{|a_{ij}|})^{-1}$, derived from condition (\ref{eq:slope_cond}), the critical slope for this system turns out to be $m^c=1/a_{33}=3.5$. In order to be in conservative range, $m=1$ is chosen as initial slope. This system converges to zero only if $|b<1|$ i.e. $0<k<2(a^{-1}+m)$. Further, within this stable region, three special cases can be analyzed i.e. $k<(a^{-1}+m), k=(a^{-1}+m)$ and $(a^{-1}+m)<k<2(a^{-1}+m)$. \figurename \ref{fig:2bus_kd_sse} demonstrates the SSE response of the example system under these three stable cases and one unstable case. In the first case of $k^d<4.5$, the system converges to zero without oscillations (overdamped response). In the second case of $k^d=4.5$, system reaches zero SSE in just one iteration and in the third case of $k^d>4.5$, it converges with decaying oscillations (underdamped response). For $k^d>9$, the SSE starts diverging with non-decaying oscillations.
\begin{figure}[h]
\vspace{-0.5em}
	\centering
	\includegraphics[trim=0.0in 0in 0in 0in,width=3.5in]{2bus_kd_sse}
    \caption {SSE convergence profile of the proposed adaptive outer loop control under different values of $k^d$}
    \label{fig:2bus_kd_sse}   
\end{figure}

\vspace{-0.5em}
 \section{Case Studies and Discussion}
\subsection{Small System Illustration}
{In this section, the proposed adaptive control performance is discussed and compared with the non-adaptive delayed control for the small example system described in the Section II.C in} \figurename \ref{fig:toy_ckt}. {Since it is a small system, outer loop time horizon of 10 seconds is adequate to demonstrate adaptive nature of the control. Other system setup and parameters selection are same as described earlier. Based on the convergence discussion in the last section, $k^d=4$ and $m=1$ are chosen. }\figurename \ref{fig:toy_adaptive} {compares the performance under three types of disturbances. In all cases, solar generation with VVC is applied at $t=20$. However, it can be seen that the adaptive VVC starts reducing SSE only after 10 seconds as outer adaptive loop works after the time horizon $T$. At $t=80$, when voltage profile gets a surge due to change in substation voltage from 1.03 to 1.05, the adaptive control adapts itself to re-track the set-point within just 1 iterations as visible in }\figurename \ref{fig:toy_adaptive}(a). {Whereas, the delayed VVC leads to voltage violation due to high SSE and non-adaptive nature. Under non-conservative settings, in} \figurename \ref{fig:toy_adaptive} {(b), adaptive control is able to maintain the smooth voltage profile under the impact of sudden cloud cover after $t=80$, unlike the delayed control. Note that for first 10 seconds (from $t=20$ to $t=30$), the adaptive VVC has higher SSE as it uses conservative slope setting and SSE correction starts only after 10 seconds. In case of }\figurename \ref{fig:toy_adaptive}{(c), switch is closed at $t=80$ which reduces the critical slope $m^c$ of the overall system and leads to voltage oscillations in the delayed control. However, the adaptive VVC re-tracks the set-point at both the nodes within few iterations. The new B matrix and convergence condition for $k^d=4$ can be verified as following.}
\begin{equation*}
B=\begin{bmatrix}
0.224 & -0.623\\
-0.646& -0.055\\
\end{bmatrix}, eigenvalues(B)=0.73, 0.56
\end{equation*}

\begin{figure}
	\centering
	\includegraphics[trim=0.0in 0in 0in 0in,width=3.5in]{2busAdaptiveforPaper_revised}
    \caption { Adaptive VVC performance comparison with delayed VVC under impact of : a) substation voltage change at conservative slope setting; b) sudden cloud cover at non-conservative slope setting and; c) topology change at non-conservative slope setting}
    \label{fig:toy_adaptive}   
    %\vspace{-1.5em}
\end{figure} 

% In order to clearly demonstrate the adaptive steps of the proposed adaptive VVC, the performance is first shown for the small example system described in the Section II.C in \figurename \ref{fig:toy_ckt}. Due to space limitation, only sudden cloud cover scenario is considered here. {\small $T\!=\!10$} seconds and {\small $k^d\!=\!2$} are taken. In \figurename \ref{fig:toy_adaptive}, at $t=60$, sudden cloud cover causes generation drop which leads to voltage oscillations and violations in delayed VVC whereas the adaptive control ensure smooth voltage profile. Adaptive control is momentarily knocked-out from the set-point by the sudden disturbance but starts re-tracking it after 2 iterations as visible in \figurename \ref{fig:toy_adaptive}. The performance is verified on a large system in the next section.  
% \begin{figure}
%  \vspace{-0mm}
% 	\centering
% 	\includegraphics[trim=0.0in 0in 0in 0in,width=3.5in]{4busAdaptiveforPaper_latest_short}
%     \vspace{-3.5mm}
%     \caption { Adaptive VVC performance on small system under cloud cover}
%     \label{fig:toy_adaptive}   
%     \squeezeup
% \end{figure}

\subsection{Large Test Case Modeling}
The proposed control is tested on a large unbalanced three-phase 1500 node system based on the IEEE123 bus feeder \cite{noauthor_ieee_nodate-1}. To create a realistic simulation, the 123 bus system is further expanded with detailed secondary side house-load modeling at 120 volts resulting in 1500 nodes as shown in \figurename \ref{fig:test_ckt} using GridLAB-D platform, an open-source agent-based simulation framework for smart grids developed by Pacific Northwest National Lab \cite{chassin_gridlab-d:_2014}. Each residential load is modeled in detail with ZIP loads and temperature dependent HVAC load \cite{schneider_multi-state_2011}. Diversity and distribution of parameters within the residential loads is discussed in \cite{fuller_evaluation_2012}. The feeder is populated with 1280 residential houses with approximately 6 MW peak load.  Inverter ratings are considered 1.1 times the panel ratings. Uniformly distributed solar PVs throughout the feeder create lesser problems than the PV units distributed in one area of the feeder. Therefore, to demonstrate the effectiveness of the control in more severe case, PV units are distributed randomly at 500 houses only in right half of the feeder. Temperature and solar irradiance data for January 2, 2011 is obtained from publicly available NREL data for Hawaii \cite{sengupta_oahu_2010}. Load and solar profiles for the day have been shown in \figurename \ref{fig:load_solar_profile}.  {Voltage regulator at the substation is not de-activated and has a time delay of 5 minutes. It is not expected to interfere in proposed VVC due to difference in time-scale.}

\begin{figure}
	\centering
	\includegraphics[trim=0in 0in 0in 0in,width=3.2in]{test_ckt}
    \caption {IEEE 123 bus test system with detailed secondary side modeling}
    \label{fig:test_ckt}
    \squeezeup
\end{figure}

\begin{figure}
	\centering
	\includegraphics[trim=0in 0in 0in 0in,width=2.8in]{load_solar_profile}
    \squeezeup
    \caption {Total feeder load and solar PV profile for 24 hours}
    \label{fig:load_solar_profile}
    \squeezeup
\end{figure}
\vspace{-0.2em}

\subsection{Performance Metrics}
We will be using three performance metrics to evaluate the proposed control approach. First metric is mean steady state error ($MSSE$) which evaluates control set-point tracking performance and
%\subsubsection{Mean Steady State Error (MSSE)}
% This is the absolute average (in percent) of voltage set point deviation at all of the houses with solar PV over the concerned time period. It is 
calculated as
\begin{equation}
\small
\label{eq:msse}
MSSE=\sum_{i=1}^n \sum_{t=1}^h \frac{|V_{t,i}-\mu|}{h}.\frac{1}{n}\times100
\end{equation}
where $n$ is the total number of solar PV units and $h$ is total time duration. A lower MSSE denotes better set point tracking performance of the control.
%\subsubsection{Flicker Count (FC)}
Second metric is flicker count ($FC$) where one flicker violation at one house is considered when $VF$ %the short-term flicker
value, as defined in (\ref{eq:vf}), exceeds $VF_{lim}$. %Short-term duration is considered as 5 minutes in this case. 
The total number of such flicker violations at all of the houses is termed as $FC$. Higher value of this metric is an indication of lesser power quality and an oscillatory voltage profile that in turn indicates the possibility of unstable control.
%\subsubsection{Voltage Violation Index (VVI)}
The third metric is voltage violation index ($VVI$) is the total number of voltage violations at all of the buses. Based on ANSI standards \cite{noauthor_ansi_2016}, %both instantaneous (type A) and continuous (type B) violation limits are considered to calculate voltage violations. 
a voltage violation is counted if the voltage at a bus violates either 1) 1.06-0.9 pu band instantaneously (range A) or 2) 1.05-0.95 pu band continuously for 5 minutes (range B).

\subsection{Results}
In this section, we will demonstrate the effectiveness of the proposed control scheme in a wide range of external disturbances and operating conditions. The adaptive control performance (blue) will be compared with existing droop controllers i.e. conventional (orange) and delayed droop (black) as shown in \figurename\ref{fig:setpoint_change}. Dashed and solid red lines denote the voltage violation limits and voltage set-point respectively. Voltage profiles and parameters dispatched are shown at a randomly chosen solar PV unit at bus 92 whereas the performance metrics are calculated for the whole system as shown in Table \ref{tab:24_hours}. Outer loop horizon $T\!=\!1$ minute and $k^d\!=\!4$ are considered.
\begin{figure}
	\centering
	\includegraphics[trim=0in 0in 0in 0in,width=3.4in]{setpoint_change}
    \vspace{-0.5em}
    \caption {Voltage profiles to compare set-point tracking performance of adaptive control with other methods}%; (b) inverter var and $q_p$ parameter dispatch in adaptive control scheme at bus 92}
    \label{fig:setpoint_change}
\end{figure}
%  \begin{figure}
% 	\centering
% 	\includegraphics[trim=0in 0in 0in 0in,width=3.2in]{Kd_sse}
%        \vspace{-0.5em}
%     \caption {Adaptive control performance under various correction factor, $k_d$ values: (a) inverter voltage; (b) parameter $q_p$ dispatch}
%     \label{fig:Kd_sse}
% \end{figure}
\subsubsection{Control Performance on Static Load Conditions}
% The outer-loop time horizon, $T=1$ minute, is used here. The speed of the adaptive set point tracking depends on the correction factor, $k_d$. \figurename\ref{fig:speed_factor} shows the control performance comparison at various $k_d$ values. It can be seen the voltage takes more time in converging to a set point at lower values of $k_d$. As we increase the $k_d$ magnitude, convergence is faster because the error adaptive parameter $q_p$ adapts itself faster as shown in \figurename\ref{fig:speed_factor}(b). A moderate value of $k_d$ can be chosen based on the offline studies.

In order to evaluate how well the proposed control tracks a set-point and minimizes SSE, a sudden set-point change is applied at the static load conditions in \figurename \ref{fig:setpoint_change}. Load and solar conditions at 11 AM are used for this purpose. %\figurename \ref{fig:set-point_change} compares the set point tracking performance of the adaptive control with conventional control and no control cases. 
Set point $\mu$ is changed from 1 to 0.96 pu at the middle of the simulation for all the inverters. As expected, the voltage profile in the no control case is steady. The conventional and delayed droop control have similar steady state performance which fails to track the set-point accurately and settles down with high SSE value. The adaptive control scheme, however, is able to track the set-point accurately by adapting $q_p$ parameter. It verifies the adaptive control's SSE minimization capability.

% \begin{figure}
% 	\centering
% 	\includegraphics[trim=0in 0in 0in 0in,width=3in]{24hr_voltage}
%     \vspace{-0.5em}
%     \caption {Comparison of adaptive control performance throughout the day: a) voltage profile; b) slope setting dispatch; c) inverter var dispatch}
%     \label{fig:24hr_voltage}
% \end{figure}
\subsubsection{Dynamic Tests with Daily Load and Solar Variation}
A day-long load and solar profile can be seen as continuous external disturbances in the system. \figurename\ref{fig:24hr_voltage}(a) shows that during the daytime, non-adaptive droop controls are not able to track the set point voltage which might lead to voltage violations e.g. around 12 noon when the solar generation is at peak. $\mu=0.97$ and homogeneous conservative settings $(m=3)$ are used for conventional and delayed control. Whereas the adaptive control  adapts it’s parameters at each bus differently to keep a flat voltage profile throughout the day; note, this may not be entirely desirable for the utility, or the owners, due to increased var flows, but rather indicates the flexibility of the system for applications such as CVR, loss minimization etc. {The OLTC tap counts are significantly reduced from 31 in no var control case to 5 in adaptive VVC case which improves the tap changer's life span. However, the tap changers can also be pre- dispatched based on the day-ahead load/solar profiles using a supervisory control, if required.} \figurename\ref{fig:24hr_voltage}(b) shows the dynamic dispatch of adaptive error parameter $q^p$ at bus 92. The performance metrics for the whole system are compared in Table \ref{tab:24_hours}. In this case, high MSSE in delayed control is because of selecting a conservative slope setting which can be improved by choosing higher slope, however, it will make the control highly vulnerable to sudden external disturbances as demonstrated in the next results. Whereas due to its decoupled functionality, the proposed control is capable of achieving near zero MSSE even at conservative settings, thus not making system prone to instability or voltage flicker.

\begin{figure}
	\centering
	\includegraphics[trim=0in 0in 0in 0in,width=3.5in]{24hrprofiles_revised}
    \vspace{-1.5em}
    \caption {Comparison of adaptive control performance throughout the day: a) voltage profile; b) dispatch of error adaptive parameter ($q^p$)}
    \label{fig:24hr_voltage}
    \vspace{-1.5mm}
\end{figure}

\begin{table}
\caption{Performance Metrics Comparison for 24-hour profile}
\vspace{-1em}
\label{tab:24_hours}
\begin{center}
\renewcommand{\arraystretch}{1}
\begin{tabular}{|c|c|c|c|c|}
\hline
% Metrics & \parbox[c]{1cm} {No\\control} & \parbox[c]{2cm} {Conventional \\droop} & \parbox[c]{1cm}{Delayed \\droop} & Adaptive\\
Metrics & \shortstack{No\\control} & \shortstack {Conventional \\droop} & \shortstack{Delayed  \\droop} & \shortstack{Adaptive \\control}\\
\hline
MSSE & 5.2\% & 4.3\% & 4.3\% & 0.3\%\\
VVI & $5\times 10^5$ & 21853 & 16137 & 0\\
%FC & 0 & 0 & 0 & 0\\
\hline
\end{tabular}
\end{center}
\squeezeup
\end{table}

% \begin{table}
% \centering
% \caption{Performance Metrics Comparison For Various Conventional Control Settings For One-Hour Window Of The Day}
% \vspace{-1em}
% \label{tab:poor_m}
% \begin{center}
% \renewcommand{\arraystretch}{1.4}
% \begin{tabular}{|>{\centering\arraybackslash}m{0.7cm}|>{\centering\arraybackslash}m{0.7cm}|>
% {\centering\arraybackslash}m{0.72cm}|>
% {\centering\arraybackslash}m{0.72cm}|>
% {\centering\arraybackslash}m{0.72cm}|>{\centering\arraybackslash}m{1.2cm}|>{\centering\arraybackslash}m{0.82cm}|}
% \hline
% \multirow{2}{*}{\centering Metrics} & \multirow{2}{0.7cm}{\centering No Control} & \multicolumn{4}{c|}{Conventional droop control}& \multirow{2}{*}{Adaptive}\\
% \cline{3-6}
% & & $m\!=\!1$ & $m\!=\!2$ & $m\!=\!3$ & $m\!=\!4$ \footnotesize (unstable) & \\
% \hline
% MSSE & 7.02\% & 6.01\% & 4.94\% & 4.49\% & 8.54\% & 0.59\% \\
% VVI & 689 & 139 & 75 & 0 & $10^6$ & 0 \\
% FC & 0 & 0 & 0 & 11 & $21705$ & 0 \\
% \hline
% \end{tabular}
% \end{center}
% \end{table}

% \paragraph{Adaptive to poor parameters selection}
% As discussed before, conventional droop control is highly parameter sensitive and a poor choice of parameter (such as slope, m) can result in an unstable control. Performance metrics of the conventional control at various $m$ values have been compared in Table \ref{tab:poor_m}. Results show that conventional control with lower $m$ gives high MSSE whereas increasing $m$ makes the control prone to instability. Until $m\!=\!3$, the voltage profile is quite stable without much voltage flicker but a slightly higher value of $m\!=\!4$ causes control instability. Whereas, the adaptive control scheme is able to keep the voltage profile stable as well as close to set point even if we accidentally choose $m=4$ as the initial condition to the adaptive control. \figurename\ref{fig:poor_parameter} presents the voltage profile of the conventional and adaptive controls with $m= 4$. Thus, the proposed control is adaptable to selection of even unstable parameters.
% \begin{figure}
% 	\centering
% 	\includegraphics[trim=0in 0in 0in 0in,width=3in]{poor_parameter}
%     \vspace{-0.5em}
%     \caption {The voltage profile with a poor slope parameter choice a) unstable conventional control and b) adaptive control}
%     \label{fig:poor_parameter}
% \end{figure}

\subsubsection{Dynamic Tests with Sudden External Disturbances}
Reliable performance under external disturbances is a unique feature of the proposed control. To demonstrate it, the control is tested with sudden external disturbances. A smaller window of 1-2 hours is considered when solar is at its peak to observe the most severe impact of disturbances.

\textit{a) Sudden cloud cover and cloud intermittency}: Usually cloud covers cause two types of disturbances in PV generation i.e. intermittency and sudden drop in the generation as shown in \figurename \ref{fig:cloud_profile}(a) and (b) respectively. Cloud intermittency data of 30 seconds scale is considered. Set-point $\mu=1$ and $m=5$ are used for the non-adaptive controls. \figurename \ref{fig:cloud_intermittent_voltage} shows how cloud intermittency causes high voltage fluctuations in conventional control which leads to violations. Delayed control reduces the flicker significantly compared to conventional (from 6919 to 107), however, still results in a good number of violations due to high SSE as shown in Table \ref{tab:cloud_intermittent}. Though, the effect of intermittency is visible in adaptive control profile (\figurename \ref{fig:cloud_intermittent_voltage}), it manages to achieve zero indices of flicker and violations. It shows the effectiveness of control in faster disturbances.

On the other hand, using non-conservative settings $(m=10)$ to decrease violations can cause stability issues with sudden cloud cover as shown in \figurename \ref{fig:cloud_cover_voltage}. At 11.30 AM, a cloud cover results in a sudden drop in real power generation (\figurename \ref{fig:cloud_profile}(b)) which frees the inverter capacity. Since conventional and delayed controls utilize all the free capacity immediately without monitoring, it increases the slope by a significant amount and results in voltage oscillations as shown in  \figurename \ref{fig:cloud_cover_voltage}. Whereas, the adaptive control dynamically regulates the  settings in real-time to ensure stable voltage profile as well as quick restore of the set-point tracking. fhl{Moreover, the proposed control can also be integrated with other smart grid applications where sudden change in real power generation is experienced such as when PV inverters are providing virtual inertia to the system.} 

\begin{figure}
	\centering
	\includegraphics[trim=0in 0in 0in 0in,width=3.4in]{cloud_profile_revised}
    \vspace{-0.5em}
    \caption {Solar profile with a) cloud intermittency b) cloud cover}
    \label{fig:cloud_profile}
    \vspace{-0.1mm}
\end{figure}

\begin{figure}
	\centering
    	\includegraphics[trim=0.0in 0.0in 0in 0in,clip,width=3.6in]{cloud_intermittency_voltage_paper_singlecolumn}
    \vspace{-1.9em}
    \caption {Control performance comparison under cloud intermittency}
    \label{fig:cloud_intermittent_voltage}
     \vspace{0.5em}
     
    \includegraphics[trim=0in 0.3in 0in 0in,clip,width=3.55in]{cloud_cover_voltage}
    \vspace{-1.5em}
    \caption {Control performance comparison under sudden cloud cover}
    \label{fig:cloud_cover_voltage}
    \vspace{-0.8em}
\end{figure}

\begin{table}
\vspace{-0.5em}
\caption{Performance Metrics Comparison For Intermittent Solar-Profile For A Two-Hour Window}
\vspace{-1em}
\label{tab:cloud_intermittent}
\begin{center}
\renewcommand{\arraystretch}{1.2}
\begin{tabular}{|c|c|c|c|c|}
\hline
Metrics & No control & Conventional & Delayed  & Adaptive\\
\hline
MSSE & 3.5\% & 2.00\% & 2.00\% & 0.40\%\\
VVI & $5\times 10^5$ & 21853 & 16137 & 0\\
FC & 122 & 6919 & 107 & 0\\
\hline
\end{tabular}
\end{center}
\end{table}

\textit{b) Change in substation voltage}: The primary side of substation voltage keeps changing due to changes in the transmission systems. Conservative setting $(m=5)$ is used here for non-adaptive controls. In  \figurename \ref{fig:substation_change}, at 12 noon, the feeder experiences a surge in primary substation voltage from 1 to 1.07 pu. Conventional control experiences high voltage oscillations. Delayed control does not experience voltage flicker but since it cannot reduce the SSE on its own, it waits for substation tap changer to operate to bring voltage within the limit again. Whereas adaptive control suffers from few instantaneous violations but immediately starts re-tracking the set-point, thus avoiding violations for long time period.
\begin{figure}
	\centering
	\includegraphics[trim=0in 0.3in 0in 0in,clip,width=3.55in]{substation_voltage_change_v2}
    \vspace{-1.5em}
    \caption {Impact of change in substation voltage on control performances}
    \label{fig:substation_change}
    \vspace{-.5em}
\end{figure}

%  \begin{figure}
% 	\centering
% 	\includegraphics[trim=0in 0in 0in 0in,width=3in]{adaptive_load_reduction_2hr_paper}
%     \vspace{-0.5em}
%     \caption {Impact of sudden load change on controls: a) sudden load reduction applied; b) voltage with conventional control; c) voltage with adaptive control}
%     \label{fig:load_change}
% \end{figure}
% \textit{d) Sudden load decrease}: A sudden 40\% load reduction is applied at 12:00 to test the robustness of the proposed control as shown in \figurename\ref{fig:load_change}. The conservative parameter ($m=5$), which provided stable voltage performance under normal conditions is causing instability on a sudden load disturbance as shown in \figurename\ref{fig:load_change}(b). Delayed control improves the stability but the voltage is vulnerable to overvoltage violations. Whereas, the adaptive control adapts itself to the disturbance and maintains a stable and flat voltage profile without set point deviation.

% \subsubsection{Adaptive to the error in feeder topology information}
% Usually in a large real-world system, fully reliable feeder topology info is not available or there are a lot of changes in the feeder which might not be communicated. This leads to change in feeder topology and sensitivity matrix $A$, thus old control settings might create issues. The proposed control is also adaptive to such errors or changes in feeder information. To simulate this, 25 new solar PV houses were added at the end of the original test system and the old non-conservative settings $(m=10)$ were used for delayed control. \figurename \ref{fig:topology_change} compares the voltage profile before and after feeder change for delayed and adaptive controls. It can be seen that the voltage profile changes from smooth (FC=0,VVI=0) to highly fluctuating (FC=4047,VVI=1615) in the slightly expanded feeder with the delayed control. Whereas, the adaptive control provides a better performance with zero flicker and violations.

% \begin{table}
% \vspace{-0.5em}
% \caption{Performance Metrics Comparison For Topology Change}
% \vspace{-1em}
% \label{tab:cloud_intermittent}
% \begin{center}
% \renewcommand{\arraystretch}{1}
% \begin{tabular}{|c|c|c|c|c|}
% \hline
% Metrics & No control & Conventional & Delayed  & Adaptive\\
% \hline
% MSSE & 3.5\% & 2.00\% & 2.00\% & 0.40\%\\
% VVI & $5\times 10^5$ & 21853 & 16137 & 0\\
% FC & 122 & 6919 & 107 & 0\\
% \hline
% \end{tabular}
% \end{center}
% \end{table}

% \begin{figure}
% 	\centering
% 	\includegraphics[trim=0in 0.3in 0in 0in,clip,width=3.4in]{topology_change}
%     \vspace{-0.5em}
%     \caption {Control performance comparison before and after feeder growth}
%     \label{fig:topology_change}
%     \vspace{-3mm}
% \end{figure}

% \begin{figure}
% 	\centering
% 	\includegraphics[trim=0in 0in 0in 0in,width=3in]{load_change}
%     \vspace{-0.5em}
%     \caption {Impact of sudden load change on controls: a) sudden load reduction applied; b) voltage with conventional control; c) voltage with adaptive control}
%     \label{fig:load_change}
% \end{figure}
% \textit{d) Sudden load decrease}: A sudden 40\% load reduction is applied at 12:00 to test the robustness of the proposed control as shown in \figurename\ref{fig:load_change}. The best case parameter, which provided stable voltage performance under normal conditions is causing instability on a sudden load disturbance as shown in \figurename\ref{fig:load_change}(b). Delayed control improves the stability but the voltage is vulnerable to overvoltage violations. Whereas, the adaptive control adapts itself to the disturbance and maintains a stable and flat voltage profile without set point deviation. 

%Similarly, the proposed approach is also verified in several other scenarios such as sudden load drop, feeder growth (topology change), increasing PV penetration etc. However, due to space limitation, only selected results could be presented here. 

Similarly, the proposed approach can also be verified in several other scenarios such as sudden load drop, increasing PV penetration etc. {However, in order to deploy the PV inverter VVC effectively, the customers need to be given proper incentives by the utilities to allow thier inverters to participate in var support. Moreover, in emergency situations, the real power curtailment might be necessary for which the owners needs to be compensated. Therefore, some utilities are moving towards the utility owned solar and encouraging community solar projects which are more viable in terms of providing VVC benefits.}

\vspace{-1em}
\section{Conclusion}
\vspace{-1mm}
In this study, a real-time adaptive and local VVC scheme with high PV penetration is proposed to addresses two major issues of conventional droop VVC methods. First,  the proposed adaptive droop framework enables VVC to achieve high set-point tracking accuracy and control stability (low voltage flicker) simultaneously without compromising either. Second, it enables dynamic self-adaption of control parameters in real-time 
%which eliminates another major challenge of selecting appropriate control settings 
under wide range of operating conditions/external disturbances. All this is achieved while keeping the control 
%with no need of centralized topology information and ensuring that the developed control framework is 
compatible with the integration standards (IEEE1547) and utility practices (Rule 21). 
%These features make the proposed VVC feasible and implementation friendly. 
The satisfactory performance is demonstrated by comparing with existing droop methods in several cases on a large unbalanced distribution system.

%It is worth mentioning that the proposed local VVC framework is easily extendable to centralized approaches. 
Being local in nature, the proposed control might be less effective for system-wide optimization, however, the proposed VVC framework can easily be combined with centralized approaches. In fact, due to its tight voltage regulation feature and adaptive nature under external disturbances, it facilitates the use of PV inverters for other system-wide volt/var applications such as CVR, loss minimization, var support to the grid, etc. The integration with supervisory control and coordination with conventional regulators will be explored in the future work.

% \appendix
% Small 4-bus example system information: length of lines 1-2, 2-3 and 3-4 are 2000, 4500 and 4500 feet respectively. Transformer is step-down (12.47kV/4.16 kV). In all cases $d=0$ and $\tau = 0.1$ are considered. Note that in conventional and delayed droop control, all settings remain constant throughout the day except $q_{min}$ and $q_{max}$ which change with change in PV generation and cloud cover. For adaptive control, all of these settings are decided by the proposed algorithm.

% \begin{table}[h]
% \caption{Conventional and delayed droop VVC settings: 4 bus system}
% \vspace{-1em}
% \label{tab:appendix1}
% \begin{center}
% %\renewcommand{\arraystretch}{1.3}
% \begin{tabular}{|c|c|c|c|c|}
% \hline
% droop & $q_{min}$ & $q_{max}$ & $v_{min}$ & $v_{max}$\\
% \hline
% conservative ($m=2$) & -0.2 & 0.2 & 0.9 & 1.1\\
% non-conservative ($m=6$) & -0.2 & 0.2 & 0.967 & 1.033\\
% \hline
% \end{tabular}
% \end{center}
% \end{table}
% \squeezeup \squeezeup \squeezeup 
% \begin{table}[h]
% \caption{Conventional and delayed droop VVC settings: 123 bus system}
% \vspace{-1em}
% \label{tab:appendix1}
% \begin{center}
% %\renewcommand{\arraystretch}{1.3}
% \begin{tabular}{|c|c|c|c|c|}
% \hline
% droop & $q_{min}$ & $q_{max}$ & $v_{min}$ & $v_{max}$\\
% \hline
% conservative ($m=3$) & -0.4 & 0.4 & 0.867 & 1.133\\
% conservative ($m=5$) & -0.4 & 0.4 & 0.92 & 1.08\\
% non-conservative ($m=10$) & -0.4 & 0.4 & 0.96 & 1.04\\
% \hline
% \end{tabular}
% \end{center}
% \end{table}

%\hfill mds
 
%\hfill August 26, 2015

% needed in second column of first page if using \IEEEpubid
%\IEEEpubidadjcol


% An example of a floating figure using the graphicx package.
% Note that \label must occur AFTER (or within) \caption.
% For figures, \caption should occur after the \includegraphics.
% Note that IEEEtran v1.7 and later has special internal code that
% is designed to preserve the operation of \label within \caption
% even when the captionsoff option is in effect. However, because
% of issues like this, it may be the safest practice to put all your
% \label just after \caption rather than within \caption{}.
%
% Reminder: the "draftcls" or "draftclsnofoot", not "draft", class
% option should be used if it is desired that the figures are to be
% displayed while in draft mode.
%
%\begin{figure}[!t]
%\centering
%\includegraphics[width=2.5in]{myfigure}
% where an .eps filename suffix will be assumed under latex, 
% and a .pdf suffix will be assumed for pdflatex; or what has been declared
% via \DeclareGraphicsExtensions.
%\caption{Simulation results for the network.}
%\label{fig_sim}
%\end{figure}

% Note that the IEEE typically puts floats only at the top, even when this
% results in a large percentage of a column being occupied by floats.


% An example of a double column floating figure using two subfigures.
% (The subfig.sty package must be loaded for this to work.)
% The subfigure \label commands are set within each subfloat command,
% and the \label for the overall figure must come after \caption.
% \hfil is used as a separator to get equal spacing.
% Watch out that the combined width of all the subfigures on a 
% line do not exceed the text width or a line break will occur.
%
%\begin{figure*}[!t]
%\centering
%\subfloat[Case I]{\includegraphics[width=2.5in]{box}%
%\label{fig_first_case}}
%\hfil
%\subfloat[Case II]{\includegraphics[width=2.5in]{box}%
%\label{fig_second_case}}
%\caption{Simulation results for the network.}
%\label{fig_sim}
%\end{figure*}
%
% Note that often IEEE papers with subfigures do not employ subfigure
% captions (using the optional argument to \subfloat[]), but instead will
% reference/describe all of them (a), (b), etc., within the main caption.
% Be aware that for subfig.sty to generate the (a), (b), etc., subfigure
% labels, the optional argument to \subfloat must be present. If a
% subcaption is not desired, just leave its contents blank,
% e.g., \subfloat[].


% An example of a floating table. Note that, for IEEE style tables, the
% \caption command should come BEFORE the table and, given that table
% captions serve much like titles, are usually capitalized except for words
% such as a, an, and, as, at, but, by, for, in, nor, of, on, or, the, to
% and up, which are usually not capitalized unless they are the first or
% last word of the caption. Table text will default to \footnotesize as
% the IEEE normally uses this smaller font for tables.
% The \label must come after \caption as always.
%
%\begin{table}[!t]
%% increase table row spacing, adjust to taste
%\renewcommand{\arraystretch}{1.3}
% if using array.sty, it might be a good idea to tweak the value of
% \extrarowheight as needed to properly center the text within the cells
%\caption{An Example of a Table}
%\label{table_example}
%\centering
%% Some packages, such as MDW tools, offer better commands for making tables
%% than the plain LaTeX2e tabular which is used here.
%\begin{tabular}{|c||c|}
%\hline
%One & Two\\
%\hline
%Three & Four\\
%\hline
%\end{tabular}
%\end{table}


% Note that the IEEE does not put floats in the very first column
% - or typically anywhere on the first page for that matter. Also,
% in-text middle ("here") positioning is typically not used, but it
% is allowed and encouraged for Computer Society conferences (but
% not Computer Society journals). Most IEEE journals/conferences use
% top floats exclusively. 
% Note that, LaTeX2e, unlike IEEE journals/conferences, places
% footnotes above bottom floats. This can be corrected via the
% \fnbelowfloat command of the stfloats package.












% if have a single appendix:
%\appendix[Proof of the Zonklar Equations]
% or
%\appendix  % for no appendix heading

% do not use \section anymore after \appendix, only \section*
% is possibly needed

% use appendices with more than one appendix
% then use \section to start each appendix
% you must declare a \section before using any
% \subsection or using \label (\appendices by itself
% starts a section numbered zero.)
%


% \appendices
% \section{Proof of the First Zonklar Equation}
% Appendix one text goes here.

% you can choose not to have a title for an appendix
% if you want by leaving the argument blank
% \section{}
% Appendix two text goes here.


% use section* for acknowledgment
%\section*{Acknowledgment}


%The authors would like to thank...


% Can use something like this to put references on a page
% by themselves when using endfloat and the captionsoff option.
\ifCLASSOPTIONcaptionsoff
  \newpage
\fi



% trigger a \newpage just before the given reference
% number - used to balance the columns on the last page
% adjust value as needed - may need to be readjusted if
% the document is modified later
%\IEEEtriggeratref{8}
% The "triggered" command can be changed if desired:
%\IEEEtriggercmd{\enlargethispage{-5in}}
% references section

% can use a bibliography generated by BibTeX as a .bbl file
% BibTeX documentation can be easily obtained at:
% http://mirror.ctan.org/biblio/bibtex/contrib/doc/
% The IEEEtran BibTeX style support page is at:
% http://www.michaelshell.org/tex/ieeetran/bibtex/
%\bibliographystyle{IEEEtran}
% argument is your BibTeX string definitions and bibliography database(s)
%\bibliography{IEEEabrv,../bib/paper}
%
% <OR> manually copy in the resultant .bbl file
% set second argument of \begin to the number of references
% (used to reserve space for the reference number labels box)

    
    
% \end{thebibliography)

\bibliographystyle{IEEEtran}
\bibliography{IEEEabrv,References_short}


% biography section
% 
% If you have an EPS/PDF photo (graphicx package needed) extra braces are
% needed around the contents of the optional argument to biography to prevent
% the LaTeX parser from getting confused when it sees the complicated
% \includegraphics command within an optional argument. (You could create
% your own custom macro containing the \includegraphics command to make things
% simpler here.)
% \begin{IEEEbiography}[{\includegraphics[width=1in,height=1.25in,clip,keepaspectratio]{mshell}}]{Michael Shell}
% or if you just want to reserve a space for a photo:
 \begin{IEEEbiography}
 [{\includegraphics[width=1.1in,height=1.2in,clip,keepaspectratio]{Ankit}}]{Ankit Singhal}
  (S'13) received the B.Tech. degree in electrical engineering
from the Indian Institute of
Technology-Delhi, India. He is currently
a Ph.D. student in the Department of Electrical and Computer Engineering at Iowa State
University, Ames, IA, USA.

His research interests include renewable integration, impact of high PV penetration on distribution and transmission network, smart inverter volt/var control and transmission-distribution co-simulation.
 \end{IEEEbiography}
 \begin{IEEEbiography}
 [{\includegraphics[width=1in,height=1.2in,clip,keepaspectratio]{Ajjarapu}}]{Venkataramana Ajjarapu}
  (S'86-M'86-SM'91-
F'07) received the Ph.D. degree in electrical engineering from the University of Waterloo, Waterloo, ON, Canada, in 1986.

Currently, he is a Professor in the Department of Electrical and Computer Engineering at Iowa State University, Ames, IA, USA. His present research is in the area of integration of distributed energy resources, reactive power planning, voltage stability analysis, and nonlinear voltage phenomena.
 \end{IEEEbiography}
 
  \begin{IEEEbiography}
 [{\includegraphics[width=1in,height=1.2in,clip,keepaspectratio]{jason}}]{Jason Fuller}
  (S'08-M'10) received the B.S. degree
in physics from the University of Washington,
Seattle, and the M.S. degree in electric engineering from Washington State University, Pullman.

He is currently the manager of Electricity Infrastructure group at the Pacific Northwest National Laboratory. His main areas of interest are distribution system analysis and renewable integration. He is currently the Secretary of the Distribution System Analysis Subcommittee’s Test Feeder Working Group.
\end{IEEEbiography}

  \begin{IEEEbiography}
 [{\includegraphics[width=1in,height=1.2in,clip,keepaspectratio]{Jacob}}]{Jacob Hansen}
 (S'12-M'14) received
the B.Sc. and M.Sc. degrees in electronics
engineering and information technology from
Aalborg University, Aalborg, Denmark.

Currently, he is employed at Pacific Northwest National Laboratory as a Power System Control Engineer. He was a Visiting Student with the Active Adaptive Control Laboratory, Massachusetts Institute of Technology, Cambridge, MA, USA,
from 2013 to 2014. His current research interests include smart grid, power systems, decentralized
control, and control systems in general.
\end{IEEEbiography}

% % if you will not have a photo at all:
% \begin{IEEEbiographynophoto}{John Doe}
% Biography text here.
% \end{IEEEbiographynophoto}

% % insert where needed to balance the two columns on the last page with
% % biographies
% %\newpage

% \begin{IEEEbiographynophoto}{Jane Doe}
% Biography text here.
% \end{IEEEbiographynophoto}

% You can push biographies down or up by placing
% a \vfill before or after them. The appropriate
% use of \vfill depends on what kind of text is
% on the last page and whether or not the columns
% are being equalized.

%\vfill

% Can be used to pull up biographies so that the bottom of the last one
% is flush with the other column.
%\enlargethispage{-5in}



% that's all folks
\end{document}




