\setcounter{equation}{0}
\setcounter{figure}{0}
\section{Roper Resonance in Experiment}
\label{Experiment}
%
\subsection{Sparse Data}
%The Roper resonance was discovered in 1964 \cite{Roper:1964zza, BAREYRE1964137, AUVIL196476, PhysRevLett.13.555, PhysRev.138.B190}; and, as we have outlined and shall discuss further, its characteristics have been the source of great puzzlement since that time.
%
One material source of the difficulty in understanding the Roper resonance is the quality of the data that was available in the previous millennium.  Illustrated by Fig.\,\ref{OldElectrocouplings}, it was poor owing to limitations in sensitivity to the channels $\gamma p \to \pi^0 p$ and $ep \to e\pi^0 p$ that were typically employed in analyses of the photo- and electrocoupling helicity amplitudes and transition form factors.  Such data could not reasonably be used to distinguish between competing theoretical models of the Roper resonance.  It was therefore evident, given that physics is an empirical science, that a key to resolving the conundrum was more and better data, \emph{i.e}.\ to replace the very limited amount of data available in the previous millennium with a much larger set of high-precision data. This was a strong motivation for a new experimental program at what is now known as the Thomas Jefferson National Accelerator Facility [JLab], which began operations in 1994 and was then called the Continuous Electron Beam Accelerator Facility [CEBAF].

\begin{figure}[t]
%\centerline{\includegraphics[width=0.66\textwidth]{zBurkert/TEOldHelicity.pdf}}
\centerline{\includegraphics[width=0.49\textwidth]{F3_III1.pdf}}
\caption{\label{OldElectrocouplings}
Data on the transverse [left panel] and longitudinal [right] photo- and electrocoupling helicity amplitudes for the Roper resonance, Eqs.\,\eqref{ThHelAmp}, as they were available in the last millennium.
%
%These quantities are related to the resonant portions of the multipole amplitudes determined at the resonance positions \cite{Walker:1968xu, Capstick:1994ne}, and are defined theoretically in terms of Poincar\'e-invariant form factors in Eqs.\,\eqref{ThHelAmp}.
%
\emph{Legend}.
Data:
open [red] circle -- 1998 estimate of $A_{1/2}$ at the photoproduction point \cite{Caso:1998} and % -0.065 \pm 0.004 ... Volker has the error much larger than I found
error bar [gray] -- our assessment of the true uncertainty in this value at that time;
%I purposely inflated the uncertainty after looking at publications in the PDG that were used. The PDG error did not take into account obvious uncertainties such as the branching ratio to N(1440)->Npi, the large spread in the width as well as differences in methodology of the analysis.
and solid squares and short-dashed [cyan] curves -- results from a fixed-$t$ dispersion relation fit \cite{Gerhardt:1980yg}, where the error bars on the squares are our estimate of the systematic uncertainty in these values.
%
Illustrative model results:
long-dashed [red] curves -- non-relativistic quark model \cite{Koniuk:1979vy, Close:1989aj} [incompatible with then-existing data];
%  ... these curves are actually from Eqs. (8) and (12) in Li, Burkert and Li [Li:1991yba] ... but, if one looks hard enough, the formulae can also be reconstructed from Close:1989aj
dotted curve [purple, left panel] -- relativized quark model \cite{Warns:1989ie};
and solid curve [green] -- model constructed assuming the Roper is a hybrid system, constituted from three constituent-quarks plus a type of gluon excitation \cite{Li:1991yba}, wherewith the longitudinal amplitude vanishes.
%
[The ordinate is expressed in units of $10^{-3}GeV^{-1/2}$.]}
\end{figure}

\subsection{Electroproduction Kinematics}
%
The data in Fig.\,\ref{OldElectrocouplings} were obtained in $e N \to e \pi N$ reactions, \emph{i.e}.\ single-pion photo- and electroproduction processes.
%
The production of a nucleon resonance in the intermediate part of such reactions is described by the following electromagnetic current, which connects $J=1/2$-baryon initial and final states and is completely expressed by two form factors:
\begin{equation}
%\textstyle
%\mbox{\rm\emph{i.e.}}\,
\bar u_{f}(P_f)\big[ \gamma_\mu^T F_{1}^{\ast}(Q^2)+\frac{1}{m_{{fi}}} \sigma_{\mu\nu} Q_\nu F_{2}^{\ast}(Q^2)\big] u_{i}(P_i)\,,
\label{NRcurrents}
\end{equation}
where: $u_{i}$, $\bar u_{f}$ are, respectively, Dirac spinors describing the incoming/outgoing baryons, with four-momenta $P_{i,f}$ and masses $m_{i,f}$ so that $P_{i,f}^2=-m_{i,f}^2$; $Q=P_f-P_i$; $m_{{fi}} = (m_f+m_{i})$; and $\gamma^T \cdot Q= 0$.
%
In terms of these quantities, the helicity amplitudes in Fig.\,\ref{OldElectrocouplings} are:
\begin{subequations}
\label{ThHelAmp}
\begin{align}
A_{\frac{1}{2}}(Q^2) & = c(Q^2) \left[ F_{1}^{\ast}(Q^2)+ F_{2}^{\ast}(Q^2) \right],\\
%
 S_{\frac{1}{2}}(Q^2) & =  \frac{q_{\rm CMS}}{\surd 2} c(Q^2)
\left[ F_{1}^{\ast}(Q^2) \frac{m_{fi}}{Q^2}  - \frac{F_{2}^{\ast}(Q^2)}{m_{fi}}  \right],
%-(qCMS[mr, mn, Q2]/Sqrt[2]) c[Q2, mr,mn] (-F1TAMM[Q2] ((mr + mn)/Q2) + F2TAMM[Q2]/(mr + mn))
\end{align}
\end{subequations}
with
\begin{equation}
c(Q^2) = \left[ \frac{ \alpha_{\rm em} \pi  Q^2_-}{m_f m_i K} \right]^{\frac{1}{2}},\;
%
q_{\rm CMS} = \frac{\sqrt{Q_-^2 Q_+^2}}{2 m_f},
\end{equation}
where $Q_{\pm}^2 = Q^2 + (m_f \pm m_i)^2$, $K= (m_f^2-m_i^2)/(2 m_f)$.
%Owing to the properties of on-shell baryon spinors, $\bar u_f(P_f) \gamma\cdot Q\, u_i(P_i) = 0$ when $i=f$.
%
%In computing all form factors, we follow Refs.\,\cite{Cloet:2008re, Wilson:2011aa, Segovia:2014aza} in every respect, including formulation of the current \cite{Oettel:1999gc,Chang:2011tx}.  The transition form factors are obtained from the nucleon elastic form factor expressions by replacing all inputs connected with the final state by those for the radial excitation associated with the wave function in Fig.\,\ref{figFA}.  The critical issue is whether the form factors thus obtained have any relationship to those measured in the proton-Roper transition.

The dominant Roper decay is $N(1440) \to N\pi$, where the neutron$+\pi^+$ $(n\,\pi^+)$ channel is most prominent.  It also couples to the two-pion channel, being there most conspicuous in $N(1440) \to p\,\pi^+\pi^-$, where $p$ labels the proton.  By design, the CEBAF Large Acceptance Spectrometer [CLAS] at JLab was ideally suited to measuring both these reactions in the same experiment, simultaneously employing the polarized high-precision continuous-wave electron beam at energies up to 6\,GeV.  This capability provided the CLAS Collaboration with a considerable advantage over earlier experiments because measurements and extractions of Roper resonance observables could be based on the analysis of complete centre-of-mass angular distributions and large energy range, and cross-checked against each other in different channels.
%% I have omitted the image CLAS because it won't reproduce well in B&W.

\begin{figure}[t]
\centerline{\includegraphics[width=0.42\textwidth]{F4_III2.pdf}}
\caption{\label{kinematics}
Kinematics of $\pi^+$ electroproduction from a proton.}
\end{figure}

A typical choice of kinematics for the reaction $ep\to en\pi^+$ is depicted in Fig.\,\ref{kinematics}: the incoming and outgoing electrons define the scattering plane; the $\pi^+$ and neutron momentum vectors define the hadronic production plane, characterised by polar angles $\theta_{\pi}$ and $\theta_n$; and the azimuthal angle $\phi_\pi$ defines the angle between the production plane and the electron scattering plane.  In these terms, the differential cross-section can be written:
\begin{equation}
{d^3\sigma \over dE_f d\Omega_e d\Omega} =: {\Gamma {d\sigma \over d\Omega}},
\end{equation}
where $\Gamma$ is the virtual photon flux:
\begin{equation}
\Gamma = {\alpha_{em} \over 2\pi^2Q^2} {(W^2 - m_N^2)E_f \over 2m_NE_i} {1\over 1-\epsilon}\,.
\end{equation}
Here: $\alpha_{\rm em}$ is the fine structure constant and $m_N$ is the nucleon mass; $W$ is the invariant mass of the hadronic final state; $Q^2=-(e_i - e_f)^2$ is the photon virtuality, where $e_i$ and $e_f$ are the four-momentum vectors of the initial and final state electrons, respectively, and $E_i$ and $E_f$ are their respective energies in the laboratory frame; $\epsilon$ is the polarization factor of the virtual photon; and $\Omega_e$ and $\Omega$ are the electron and the pion solid angles.  The unpolarized differential hadronic cross-section has the following $\phi_\pi$ dependence:
\begin{equation}
\label{dsigmadOmega}
 {d\sigma \over d\Omega} = \sigma_{L+T} + \epsilon \sigma_{TT}\cos{2\phi_\pi} + \sqrt{2\epsilon(1+\epsilon)}\sigma_{LT}\cos{\phi_\pi}\,,
\end{equation}
with the $\phi_\pi$-independent term defined as
%%\begin{equation}
 $\sigma_{L+T} = \sigma_T + \epsilon \sigma_L$.
%%\end{equation}
As distinct from photoproduction with real photons, the virtual photon in electroproduction has both transverse and longitudinal polarizations; and resolving the associated kinematic dependences reveals additional information about the  production process, especially interference effects.  By measuring the azimuthal dependence of the cross-section in Eq.\,\eqref{dsigmadOmega}, one can isolate the terms that describe transverse-transverse and transverse-longitudinal interference.  %In order to separate $\sigma_T$ and $\sigma_L$, it is necessary to perform measurements at fixed $Q^2$ and $W$ whilst varying the beam energy.

\begin{figure}[t]
\centerline{\includegraphics[width=0.45\textwidth]{F5_III3.pdf}}
\caption{\label{Roper_lowQ}
Cross-section data at $Q^2=0.45\,$GeV$^2$: $\gamma^\ast p \to \pi^0 p$ (upper panels) and $\gamma^\ast p \to \pi^+ n$ (lower panels).  The curves are results of global fits to this data using the UIM [solid] and DR [dashed] approaches.  [Details provided elsewhere \cite{Aznauryan:2004jd}.  The ordinate unit is $\mu$b.]}
\end{figure}

\subsection{Electroproduction Data at Low $\mathbf Q^2$}
%
Experiments with CLAS began in 1998. Following commissioning, the Collaboration took precise data covering a large mass range from pion threshold up to $W=1.55\,$GeV, \emph{i.e}.\ throughout the first and second resonance regions,\footnote{
The total cross-sections for photo-, electro- and hadro-production of pions from the nucleon exhibit a series of clear ``peak domains'' and each is described as a ``resonance region''.  The first is identified with $W\simeq 1.23\,$GeV (the $\Delta$-baryon); the second, $W\in (1.4,1.6)\,$GeV, contains the Roper resonance, etc.}
%
with $n\,\pi^+$  and $p\,\pi^0$ final states at two values of $Q^2$, pursuing a primary goal of studying the low-$Q^2$ behavior of the proton-Roper transition.  Analysis of the data was a complex and time-consuming task.

%In order to extract resonance amplitudes from electroproduction data, procedures that are largely model-insensitive must be employed.
Resonance electroexcitation amplitudes are extracted from exclusive electroproduction data by employing phenomenological reaction models capable of reproducing the full set of observables measured in the $N \pi$ and $p \pi^+ \pi^-$ channels, subject to general reaction theory constraints, such as analyticity and unitarity.  When analysing $n \pi^+$, $p \pi^0$, $p \eta$ final states, the most frequently used approaches are the Unitary Isobar Model (UIM) \cite{Drechsel:1998hk, Aznauryan:2002gd, Drechsel:2007if} and fixed-$t$ dispersion relations \cite{Aznauryan:2004jd}.  In both cases, resonances are described by a relativistic Breit-Wigner distribution involving an energy-dependent width.  Naturally, it is important to implement a good description of the background contributions.  With the UIM approach, these are described explicitly through inclusion of $s$- and $t$-channel meson exchange processes; whereas in the DR method they are calculated directly from the $s$-channel resonance terms using dispersion relations.  The DR approach is tightly constrained, but the UIM method, involving more fitting parameters, has greater flexibility.

\begin{figure}[t]
\centerline{\includegraphics[width=0.42\textwidth]{F6_III4.pdf}}
\caption{\label{A12_lowQ}
First results from CLAS on the Roper helicity amplitudes \cite{Aznauryan:2004jd} -- solid squares.
%
All curves are results from various types of CQM:
%
solid-bold and solid-thin -- results obtained using, respectively, relativistic and non-relativistic versions \cite{Capstick:1994ne};
%
dotted -- \cite{Warns:1989ie};
dashed \cite{Cano:1998wz};
dot-dashed, thin -- quark-gluon hybrid model \cite{Li:1991yba};
and dot-dashed --  \cite{Tiator:2003uu}.}
\end{figure}

Employing these schemes, the CLAS collaboration released an analysis of their low-$Q^2$ data shortly after the beginning of the new millennium \cite{Aznauryan:2004jd}.   As illustrated by Fig.\,\ref{Roper_lowQ}, both the UIM and DR methods give very similar results; and the Collaboration used the difference between them as an estimate of systematic uncertainties in the model analysis.  In this way they obtained the helicity amplitudes displayed in Fig.\,\ref{A12_lowQ}.  The results contrast starkly with the pre-2000 data in Fig.\,\ref{OldElectrocouplings}: now the transverse amplitude shows a clear zero-crossing near $Q^2=0.5\,$GeV$^2$, the first time this had been seen in any hadron form factor or transition amplitude; and the longitudinal amplitude is large and positive.  The power of precise, accurate data on the transition form factors is also evident in Fig.\,\ref{A12_lowQ}: the hybrid (constituent-quark plus gluon) Roper \cite{Li:1991yba} and two other constituent-quark models \cite{Warns:1989ie, Tiator:2003uu} are eliminated.

The model most favored by the new data is arguably that which describes the Roper as a radial excitation of the nucleon's quark-core dressed by a soft meson cloud \cite{Cano:1998wz}, where a detailed explanation of this ``cloud'' is presented in Sec.\,\ref{sec:DCC}, although the relativistic-CQM \cite{Capstick:1994ne} remains viable.
%
Notably, both these calculations predict the zero in the $A_{1/2}$ amplitude, although it is achieved through different mechanisms: the meson cloud is responsible in \cite{Cano:1998wz} and relativity plays a crucial role in \cite{Capstick:1994ne}.
%
It is apparent, too, that the predictions made by these two models are in marked disagreement at larger $Q^2$, \emph{i.e}.\ on the domain within which any soft meson-cloud component of a resonance should become invisible to the probe.  This is correlated with the differing dynamical origins of the $A_{1/2}$ zero in the two CQMs.
%
It was now clear that higher-$Q^2$ electroproduction data was necessary in order to determine the nature of the Roper resonance.

\begin{figure}[t]
\centerline{\includegraphics[width=0.46\textwidth]{F7_III5.pdf}}
\caption{\label{D0_moment}
Lowest moment of the polar-angle dependence in the Legendre expansion of the total cross-section $\sigma_{T+L}$ for the $n\,\pi^+$ and $p\,\pi^0$ electroproduction final states, where the solid [red] and dashed [blue] curves represent, respectively, DR and UIM fits \cite{Aznauryan:2009mx}.  Evidently, whilst the $\Delta(1232)$ is the most conspicuous feature at low-$Q^2$ [left panels], the Roper resonance becomes prominent in the $n \pi^+$ final state at large $Q^2$, generating the broad shoulder centered near $W=1.35\,$GeV [lower right panel].  \emph{N.B}.\ The strong peak at $1.5\,$GeV owes to two other resonances: $N(1520)\,3/2^-$, $N(1535)\,1/2^-$.
}
\end{figure}

\subsection{Pushing electroproduction experiments to higher $\mathbf Q^2$}
%
Using CLAS and the 6\,GeV continuous-wave electron beam at JLab, high-statistics data were subsequently collected and analyzed, extending the kinematic range to $W=2\,$GeV and $Q^2=4.5\,$GeV$^2$ \cite{Aznauryan:2008pe, Aznauryan:2009mx, Aznauryan:2011qj, Mokeev:2012vsa, Mokeev:2015lda}.
%
The new experiments revealed some surprising aspects of the Roper electroproduction amplitudes, overturning conclusions that might have been drawn from the low-$Q^2$ data alone.  For example, as highlighted by Fig.\,\ref{D0_moment}, whereas $A_{1/2}$ is small in the low-$Q^2$ range accessed by the earlier CLAS data, because it is undergoing a sign change at $Q^2\approx 0.5\,$GeV$^2$, and hence the Roper is not directly visible in the total cross-section, at high-$Q^2$ this resonance becomes very strong, even dominating over the $\Delta(1232)$ on $Q^2> 2\,$GeV$^2$ in the $n\,\pi^+$ final state.

The final data set used in the global fit contained over 120\,000 points in $e p \to e^\prime n \pi^+$ and $e p \to e^\prime p \pi^0$, measuring differential cross-section, and polarized beam and polarized target asymmetries, covering the complete range of azimuthal and polar angles, and the domains $W<1.8\,$GeV and $Q^2< 4.5\,$GeV$^2$.  The transverse and longitudinal helicity amplitudes for Roper-resonance electroproduction obtained from the complete analysis are displayed in Fig.\,\ref{A12S12}.  These results confirm those obtained in earlier analyses of much reduced data sets and significantly extend them.  Importantly, the evident agreement between independent analyses of single- and double-pion final states boosts confidence in both.  [\emph{N.B}.\ New CLAS data on $\pi^+ \pi^- p$ electroproduction \cite{Isupov:2017lnd}, with nine one-fold differential cross-sections covering a final hadron invariant mass range $W\in[1.4,2.0]\,$GeV and $Q^2\in [2,5]\,$GeV$^2$, will enable this agreement to be tested further.]

%%Finally, I appreciate it very much if you can mention something like this:
%%  New prospects in exploration of Roper resonance structure at photon virtualities 2.0 GeV^2<Q^2<5.0 GeV^2 from exclusive pi+pi^p  electroproduction have beed open up by the new CLAS data on pi^+pi^-p electroproduction [1]. Nine one-fold differential cross sections cover the final hadron invariant mass range W<2.0 GeV where most well established nucleon resonances are located. Good data description was achieved allowing us to determine elecgtrocouplings of N(1440)1/2+ from pi^+pi^-p electroproduction data at Q^2 up to 5.0 GeV^2 in the near term future.

%%1.  	Measurements of ep→e′π+π−p′
%%Cross Sections with CLAS at 1.40<W<2.0 GeV and 2.0<Q2<5.0 GeV2
%%CLAS Collaboration (E.L. Isupov (SINP, Moscow) et al.). May 4, 2017. 21 pp. In press by Phys. Rev. c
%%\cite{Isupov:2017lnd}


\begin{figure}[!t]
\centerline{%
\includegraphics[width=0.8\linewidth]{F8A_III6AR.pdf}}
%\vspace*{1ex}

\centerline{%
\includegraphics[width=0.8\linewidth]{F8B_III6BR.pdf}}
%%
%%\begin{center}
%%\hspace{-0.75cm}
%%    \includegraphics[width=0.4\linewidth]{zBurkert/EF6A.pdf}\hspace{0.5cm}
%%	\includegraphics[width=0.32\linewidth]{zBurkert/EF6B.pdf}
%%\end{center}
%%\vspace{-.25cm}	
%
\caption{\label{A12S12}
%
Transverse [upper panel] and longitudinal [lower] Roper resonance electrocoupling helicity amplitudes.
Legend:
circles [blue] -- analysis of single-pion final states \cite{Aznauryan:2008pe, Aznauryan:2009mx};
triangles [green] -- analysis of $ep\to e^\prime \pi^+ \pi^- p^\prime$ \cite{Mokeev:2012vsa, Mokeev:2015lda};
square [black] -- CLAS Collaboration result at the photoproduction point \cite{Dugger:2009pn} and
triangle [black] -- global average of this value \cite{Olive:2016xmw}.
}
\end{figure}
\subsection{Roper Resonance: Current Experimental Status}
%
It is appropriate here to summarize the modern empirical status.\\[-4ex]
%
\begin{itemize}
\setlength\itemsep{0em}
\item The Roper [N(1440)\,$1/2^+$] is a four-star resonance with pole mass $\approx 1.37\,$GeV and width $\approx 0.18\,$GeV \cite{Olive:2016xmw}.
\item Transverse helicity amplitude, $A_{1/2}(Q^2)$:\\[-1.7em]
\begin{itemize}
\setlength\itemsep{0em}
\item increases rapidly as $Q^2$ increases from the real photon point to $Q^2 \approx 2\,$GeV$^2$;
\item changes sign at $Q^2 \approx 0.5\,$GeV$^2$;
\item exhibits a maximum value at $Q^2 \approx 2\,$GeV$^2$, attaining a magnitude which matches or exceeds that at the real photon point;
\item decreases steadily toward zero with increasing $Q^2$ after reaching its maximum value.
\end{itemize}
\item Longitudinal helicity amplitude, $S_{1/2}(Q^2)$:\\[-1.7em]
\begin{itemize}
\setlength\itemsep{0em}
\item maximal near the real photon point;
\item decreases slowly as $Q^2$ increases toward 1\,GeV$^2$;
\item decreases more quickly on $Q^2 \gtrsim 1\,$GeV$^2$.
\end{itemize}
\item $N \pi$ and $p \pi^+ \pi^-$ final states in electroproduction:\\
%
The non-resonant contributions to these two final states are markedly dissimilar and hence very different analysis procedures are required to isolate the resonant contributions.  Notwithstanding this, the results for the resonant contributions agree on the domain of overlap, \emph{i.e}.\ $Q^2 \in [0.25,1.5]\,$GeV$^2$.

\end{itemize} 