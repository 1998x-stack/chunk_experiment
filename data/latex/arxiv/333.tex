\section{Introduction}
\label{sec:intro} 

The ability of users to organize events using mobile devices is a defining 
characteristic of today's social network systems. With the advancement of 
mobile technology, more and more people are digitally connected, which 
makes \added{the} analysis of \replaced{group event planning process and decision making }
{group behavior and suggestions}  for event organizers critically important. 
There is a rich history of UbiComp research concentrating on individual 
user behavior analysis, which treats individual user's data as a singleton.
However, social interactions among group members are often ignored, and
relatively scant research to date has explored the subject of 
\replaced{group event scheduling }{mobile group dynamics}. 
Some early work~\cite{beckmann2011agremo, park2008restaurant} has 
been confined to in-lab surveys and has not studied \added{the} real-world
\replaced{event scheduling process by}{dynamics of mobile}  
 groups of users, nor \replaced{the factors that would }{its} impact
\deleted{on} group \added{event} decision-making.  More recently, a study of university groups
using mobile phones was presented~\cite{jayarajah2015need}.  What has been missing to date is a detailed
understanding of the \replaced{process }{dynamics} of how groups 
make a decision to visit a particular place at a particular time \added{using their mobile devices}. 
What factors influence a group's final decision? This paper provides detailed novel insights in the \replaced{event scheduling }{decision-making} process of \replaced{social }{mobile} groups.  
We believe that this is an exciting area ripe for exploration by the ubiquitous computing research 
community. With ever-increasing popularity of smartphones, we 
expect that mobile computing will be used extensively to assist
groups of people in \replaced{event planning }{making decisions}
\replaced{in terms of}{about} when and where to rendezvous. Consider a group of
friends out on a weekend evening trying to decide what movie to see or where to eat, or consider a group of professional
colleagues trying to decide where to go for lunch. \added{Given the frequency with which people
schedule colocated events,} 
we believe mobile applications \added{for group event scheduling} can provide significant help.

%\added{However, today's technology does not always help us reduce the friction inherent in
%coordinating online and offline social events efficiently. There hasn't been much innovation in
%the ways we schedule events, despite the considerable potential.}
\added{However, despite this considerable potential, today's technology offers limited help when it 
comes to coordinating group events in online and offline scenarios.} 
Currently there are few group event organization applications on the market. 
The most commonly used services are Meetup~\cite{meetup-about}, 
Facebook Events~\cite{facebook} and Evite~\cite{evite}. 
In these services, hosts organize offline events and post them on the website.
Users or group members who are interested in these events RSVP and later
attend the events in real life. However, in all of these existing services, meeting 
time and location are settled by the host at the creation of the event. Potential
users are not able to sufficiently express their opinions on when and where to
meet. This will more or less have a negative impact on event attendance. 
Doodle~\cite{doodle} is an online event scheduling service which supports 
groups in finding a mutually agreeable meeting time. Participants are able to vote 
for their \added{time} preferences. But all the meeting time options are \replaced{pre-selected by the host }
{added by host in the beginning}. Group members have no permission to suggest new options. In addition,
group members cannot suggest or vote for meeting locations. 

To address the limitations of existing services, we developed OutWithFriendz, a mobile
application that enables groups of people to decide together through a voting
process the date/time the group would like to meet as well as the location where 
they would like to meet. OutWithFriendz is implemented as a client-server 
architecture that is comprised of both iOS and Android based clients that 
communicate with a server implemented as a Java Web application.

\begin{figure}[t] 
\centering
\includegraphics[width=0.9\linewidth]{intro}
\caption{An illustration of the key elements in our OutWithFriendz system. \added{The colored arrows (dots) indicate
which user suggested (voted) for a meeting time or location, and the red boxes indicate the final decisions.}}
\label{fig:intro}
\end{figure}

The main elements of our OutWithFriendz mobile application are shown in 
Figure~\ref{fig:intro}. To start using it, a user may create a new invitation acting
as a host. During this process, she can specify the details of this invitation including
a title, a list of suggested dates, a list of suggested locations and invited participants.
After this host submits a new invitation to the server, all invited participants
receive it and can view the detailed invitation information on their own clients. They 
can then suggest more dates, locations or vote for their preferred options and comment 
on the invitation. \added{After the voting process has ended, } the host then
decides on the final meeting location and time, whuch are then sent to all participants. 
In this example (Figure~\ref{fig:intro}), the host suggested four locations and three date/time options. After the suggesting and voting process, she selected Cafe Mexicali and Friday, 03-17-2017, 18:00 as final decisions, which received 
the most votes. Please note that, in our design, the host can make decisions based on the 
voting results, but she does not have to always obey them. We do find few hosts in our field study
whose final decisions was different from the options that received most votes.
This scenario will be discussed later.

Introducing our newly designed OutWithFriendz mobile application which embeds group
decision making into the voting process raises new questions: How do the mobile \added {app} users collaborate
to organize their group events? What are the major factors that will impact group decisions?
How is the voting behavior processed? And how to improve the event attendance rate?

Our mobile application also collects user mobility-related data. The app posts GPS user 
location traces to the server. Users may opt out from providing their location traces, although
most users did not disable location tracking for the entire duration of their participation in the
study. This user mobility data provides great opportunity to derive such input factors as spread,
movement, mobility, and to investigate their impact on group \replaced{event scheduling }{decisions}. 


\begin{comment}
(1) mobile group decision making
(2) use case: figure~\ref{fig:intro}
(3) important questions
(4) what we have done and contributions
\end{comment}

The contributions of this paper can be summarized as follows:
\begin{itemize}
\item The paper describes the design and implementation of a new mobile \deleted{group}  
application \added{for group event scheduling} and its supporting system: OutWithFriendz provides smooth functions for group hosts to 
easily create an invitation and invite other users to join. All group members can 
suggest their \replaced{preferred meeting locations and date/time, }{ideal locations, meeting dates} and vote for them. The host finalizes meeting 
location and time based on the voting results.

\item The OutWithFriendz system represents the first field-based study of
\replaced{the group event scheduling and decision-making process }{event organization behavior}
in the context of a deployed mobile application with widespread geographic
usage. This has \replaced{allowed us }{provides the possibility} to collect precise
user \replaced{trace }{traces} data.

\item Using the data collected from a field study of this novel system,
we discovered a series of
factors, such as
mobility, host preference, user preference, and social voting influence that
are significant in group \added{event planning and} decision
making processes.
A correlation analysis of these factors is also performed. Our study offers
new insights for group hosts and members to improve their real-life
event organization.
\end{itemize}
The rest of this paper is organized as follows: After discussing the related
works, we introduce our system design and data collection. Next we present our
data analysis and correlation analysis in more detail.
Finally, we \added{summarize the important results, highlight the key findings,}
discuss \replaced{their potential usage }{the limitations}, and conclude this work.

