\section{Model-based Evaluation of Resilience}
\label{sec:Evaluation}

The design of complete resilience solutions often requires the composition of multiple resilience design patterns. In a complex HPC environment with numerous hardware and software pattern instantiations in the various components, the resilience to different fault events is managed by this well-defined system of patterns. To developed a combined evaluation of the reliability and performance characteristics of a real system that consists of several pattern solutions implemented across the system stack requires composition of the pattern models. 

For the simplified case of a system configuration that consists of N independent components or tasks such that, if any one of the system components or tasks fails, the entire system fails, the overall reliability of the system may be modeled as: 

\begin{equation}
R_{system} = R_{1} \times R_{2} \times R_{3} \times . . .  R_{N}  
\label{eq:sys-reliability}
\end{equation}
where the reliability R$_{i}$ of a component is a function of the resilience pattern that it instantiates. For such a configuration, the performance overhead of applying patterns to the N components in the system is additive. 

For more intricate analytic evaluation of the performance and reliability, more complex models must be developed. There are several paradigms that are useful for this purpose, including fault trees, block diagrams, reliability \& task graphs, Markov \& semi-Markov chains, stochastic Petrinets, etc. 
Analytical models that use Markov models are useful to model the intricate dependencies between the pattern solutions in a complex multicomponent HPC environment. Markov chains are state-space-based methods that consist of states representing various conditions associated with the system, and the transition between states, which represent the changes in system state or configuration due to the occurrence of a simple or compound event such as the malfunction or failure of one or more components in the system. The assessment of system resilience using Markov models for a multicomponent HPC environment that experiences different modes of faults, as well as a model for the combined evaluation of performance and reliability is the subject of ongoing research.   
