In this section, via computer simulation we assess and compare performance
 of the $\theta$-MLE detector (\ref{proposed_lrt})
with that of the ANMF \cite{Conte95}, referred to as in the following as MLE-NMF since
the unknown $\boldsymbol{M}$ is replaced with its MLE.
\begin{equation}\label{anmf_recall}
\frac{\left |\boldsymbol{z}^H    \boldsymbol{M}_{MLE}^{-1}\boldsymbol{p} \right |^2}
     { \left (\boldsymbol{z}^H \boldsymbol{M}_{MLE}^{-1} \boldsymbol{z} \right )
   \left (\boldsymbol{p}^H \boldsymbol{M}_{MLE}^{-1} \boldsymbol{p} \right )}
   \thinspace\mathop{\gtrless}_{H_0}^{H_1}\thinspace G.
\end{equation}
%where $\boldsymbol{M}$ is replaced by its MLE and we will call the ANMF
%MLE-NMF detector.
For the simulation,
 we use an uniform linear array consisting of $N = 8$ antennas,
 assuming $\theta = \pi/3$ and $\beta = \pi/6$
 (i.e. $\phi \in [\pi/6, \pi/2]$) and $K = 32$.
 As to the clutter, we assume that $s_0,s_1,\ldots,s_K$
 follow the chi distribution, so $s_0^2,s_1^2,\ldots,s_K^2$ follow the chi-square distribution
 with degree of freedom $\nu = 3$ (\cite{Ward81}), i.e. $E[s_t^2] = 3$.
 The generation of $\boldsymbol{g}_t$
follows the guide in \cite{Rangaswamy95}. Briefly,
we firstly generate complex Gaussian random vectors $\boldsymbol{u}_t$
 of zero-mean and identity covariance matrix; next
 $\boldsymbol{g}_t = \boldsymbol{R} \boldsymbol{u}_t $,
where
 $\boldsymbol{R}$ is the Cholesky decomposition of $\boldsymbol{C}$, i.e,
 $\boldsymbol{R}\boldsymbol{R}^H = \boldsymbol{C}$
 where $\boldsymbol{C}_{nm} = \rho^{|n-m|}$
 and $\rho$ the correlation efficient.
 Clutter return at each range cell is $\boldsymbol{c}_t = s_t \boldsymbol{g}_t$.
 Since it is difficult to derive closed-forms of detection ($P_d$)
and false alarm ($P_{fa}$) probabilities, such quantities will be numerically analyzed
through independent $10^2 / P_{d}$ and $10^2 / P_{fa}$ Monte Carlo trials, respectively.
To lower the computational burden, we choose $P_{fa} = 10^{-3}$.
We use the software CVX (http://cvxr.com/) to solve the semi--definite problem
 (\ref{proposed_lrt}) on a computer equipped with a 3.4 GHz Intel processor.
Finally, the signal-to-noise ratio (SNR) is defined as
 \begin{equation}\label{SNR}
  \text{SNR} \thinspace = \thinspace
  \frac{| \alpha |^2 \| \boldsymbol{p} \|^2 }
        {N \times E[s^2]},
\end{equation}
which is $| \alpha |^2 \| \boldsymbol{p} \|^2 /(N \times \nu )$ in our simulation.
\subsection{Performance Assessment}
We first investigate if the $\theta$-MLE detector has a CFAR property
w.r.t structure of the clutter covariance matrix (i.e. $\boldsymbol{M}$).
Fig. \ref{pfa_thre_mle} shows false alarm probabilities versus threshold of
the $\theta$-MLE detector at varied degrees of correlation
$\rho = 0.1, 0.4, 0.8, 0.9, 0.99, 0.999$.
Here we used $5 \times 10^4$ Monte Carlo runs.
It is observed that $\theta$-MLE possesses CFAR w.r.t all simulated degrees of correlation.
%, while the $\theta$-NSCM experiences slightly different thresholds w.r.t different
%correlation coefficients, for that we do not consider the $\theta$-NSCM.
Hence, $\theta$-MLE detector possesses CFAR w.r.t all the statistics of the clutter,
a property that is also possessed by MLE-NMF \cite{Conte_Aug02}.
From now on, $\rho = 0.4$ in all simulations.

 \begin{figure}
  \includegraphics[width=21pc]{pfa_thre_mle}
  \caption{Probability of false alarm versus the detection threshold,
  $\theta$-MLE, $N = 8$, $K = 32$.}
  \label{pfa_thre_mle}
 \end{figure}


 \begin{figure}
  \includegraphics[width=21pc]{chi_pd_matched_thetaVSnmf}
  \caption{Detection probabilities versus SNR of $\theta$-MLE in comparison with MLE-NMF
   in perfectly matched case ($\phi = \theta$),
  $N = 8$, $K = 32$.}
  \label{pd_matched_thetaVSnmf}
 \end{figure}

  \begin{figure}
  \includegraphics[width=21pc]{chi_pd_mismatched_thetaVSnmf}
  \caption{Detection probabilities versus SNR of $\theta$-MLE in comparison with MLE-NMF
  in mismatched case ($\phi = \theta - \pi/15$), $N = 8$, $K = 32$.}
  \label{chi_pd_mismatched_thetaVSnmf}
 \end{figure}

  \begin{figure}
  \includegraphics[width=21pc]{chi_pd_mismatched_mle_theta_variedBeta}
  \caption{Detection probabilities versus SNR
  of $\theta$-MLE in varied values of $\beta$,
  $N = 8$, $K = 32$.}
  \label{chi_pd_mismatched_mle_theta_variedBeta}
   \end{figure}


  \begin{figure}
  \includegraphics[width=21pc]{chi_pd_mismatched_mle_theta_variedK}
  \caption{Detection probabilities versus SNR of $\theta$-MLE
  in varied values of K,
  $\beta = \pi/15$, $N = 8$.}
  \label{chi_pd_mismatched_scmAndmle_theta_variedK}
 \end{figure}

\indent In Fig. \ref{pd_matched_thetaVSnmf} we compare
detection probabilities of the $\theta$-MLE
 %and $\theta$-SCM
 with that of the
MLE-NMF in case that the actual steering vector $\boldsymbol{p}$
perfectly matches with the nominal one $\boldsymbol{s}$,
%and NMF-SCM
i.e. $\phi = \theta$.
With incomplete knowledge of the actual steering vector,
$\theta$-MLE suffers a detection loss w.r.t
that of the MLE-NMF,
defined as the horizontal displacement of the corresponding curves,
of nearly 2dB. However, it is obvious and
shown in Fig. \ref{chi_pd_mismatched_thetaVSnmf} that even with a slight mismatch,
 i.e., $\theta - \phi= \pi/15$,
  %$\theta$-SCM and
  $\theta$-MLE
  outperforms the
  %SCM-NMF and
  MLE-NMF, especially in the high SNR region.
  Robustness of the $\theta$-MLE detector to mismatched signal is further demonstrated, in
  Fig. \ref{chi_pd_mismatched_mle_theta_variedBeta}
  in cases of more serious mismatches, i.e., $\theta - \phi = -\pi/10, 0, \pi/15, \pi/6, \pi/24$.
  Loss in detection probabilities, in a comparison with perfectly matched case,
   of MLE-NMF
   % and $\theta$-SCM
   is comparatively small
   when a mismatch lies in the designed interval of the $\theta$-MLE
    and becomes significant with a mismatch lying outside
 the designed interval, i.e., $|\theta - \phi| > \beta$, i.e in case of
 $\theta - \phi = 5\pi/24$ (the designed $\beta = \pi/6$).
 % Fig. \ref{chi_pd_mismatched_scm_theta_variedBeta} and Fig. \ref{chi_pd_mismatched_mle_theta_variedBeta}.
 % However, even in this case, detection probabilities of proposed detect
    Finally, influence on $\theta$-MLE's detection performance of the size of secondary data
 %and $\theta$-SCM
 is investigated in Fig. \ref{chi_pd_mismatched_scmAndmle_theta_variedK}.
Interestingly, $P_d$ of the proposed detector exhibits a little improvement with an increasing value of $K$,
meaning that we do not need to collect more secondary data from the surrounding range cells to
achieve better detection capacity.
This property is also
reported in the previous research \cite{Conte95}--\cite{Conte_Aug02} and is opposite to the results in
case of homogeneous/partially homogeneous Gaussian noise \cite{Conte01}.

 