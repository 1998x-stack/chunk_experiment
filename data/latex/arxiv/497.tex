%%%%%%%%%%%%%%%%%%%%%%%%%%%%%%%%%%%%%%%%%%%%%%%%%
% topology subsection 3
%%%%%%%%%%%%%%%%%%%%%%%%%%%%%%%%%%%%%%%%%%%%%%%%%
%\subsection{Boundary and Interior Faces of $\tcomplex$, and the Dual Flip Complex $\fcomplex$}
\subsection{The Dual Flip Complex $\fcomplex$}
\label{sec:dual-complex}


To define the flip complex $\fcomplex$, we need the notion of \emph{dual cells} and the dual cell decomposition of a piecewise-linear ball; for the precise definition, we refer to \cite[Sec.~I.6]{Hudson:Piecewise-linear-topology-1969} or \cite[\S64 and \S70]{Munkres:Elements-of-algebraic-topology-1984}.\footnote{In \cite{Munkres:Elements-of-algebraic-topology-1984}, the terminology \emph{dual blocks} is used instead of dual cells, since the construction is described in a more general setting (for arbitrary triangulated manifolds or homology manifolds) in which the dual 
blocks need not be cells (homeomorphic to balls). In the setting of piecewise-linear manifolds, in particular piecewise-linear balls, however, this technical issue does not arise.}
Here, we simply collect the properties that we will need:


\begin{proposition} 
\label{prop-dual}
Let $\mathbbm{B}$ be a $d$-dimensional piecewise-linear ball.
\begin{enumerate}
\item For each interior $k$-dimensional face $F$ of $\mathbbm{B}$, one can define a \emph{dual cell} $F^*$ (a certain subcomplex of the \emph{barycentric subdivision} of 
$\mathbbm{B}$ that is a piecewise-linear ball of dimension $d-k$ \cite[Lemma~I.19]{Hudson:Piecewise-linear-topology-1969}).
\item The construction reverses inclusion, i.e., %if $F\subseteq G$ are interior faces of $\mathbbm{B}$ %then $F^* \supseteq G^*$.
for interior faces $F$, $G$ of $\mathbbm{B}$,
$F\subseteq G$ iff $F^* \supseteq G^*$.
\item The dual cells of the %\emph{interior} 
interior faces of $\mathbbm{B}$ form a regular cell complex, denoted $\mathbbm{B}^*$ and called the \emph{dual cell complex}.
$\mathbbm{B}^*$ need not be a manifold or pure $d$-dimensional, but it is homotopy equivalent to $\mathbbm{B}$ \cite[Lem.~70.1]{Munkres:Elements-of-algebraic-topology-1984}.\footnote{More specifically, the dual complex of a piecewise-linear manifold with boundary is a deformation retraction of the manifold. For manifolds without boundary, the dual complex is piecewise-linearly homeomorphic to the original manifold.}
\end{enumerate}
\end{proposition}

We define the \emph{flip complex} $\fcomplex:=\tcomplex^*$ 
as the dual complex of the simplicial complex  $\tcomplex$. 


\begin{proof}[Proof of Theorem~\ref{thm:flip-complex}] 
%\note{
By Proposition~\ref{prop-dual}, $\fcomplex=\tcomplex^*$ is a regular cell complex that is homotopy equivalent to the ball 
$\tcomplex$; consequently, the fundamental group $\pi_1(\fcomplex)$ vanishes. 

It remains to show the characterization of the vertices, edges, and $2$-cells of $\fcomplex$.%}

The vertices of $\fcomplex$ correspond (are dual) 
to the faces of $\tcomplex$ of the highest dimension $(m-1)=\dim \tcomplex$, i.e., to the triangulations of $P$ (these are automatically interior faces of $\tcomplex$). 

The edges of $\fcomplex$ correspond to interior $(m-2)$-dimensional faces $F$ of $\tcomplex$,
%that are \emph{interior}, i.e., such that $F$ is contained in two triangulations $T$ and $T'$ that differ by a flip.
i.e., faces $F$ that are contained in two triangulations $T$ and $T'$ that differ by a flip.
Thus, the $1$-skeleton of $\fcomplex$ is exactly the flip graph of $P$.

Every $2$-cell of $\fcomplex$ is the dual cell $F^*$ of an interior face $F$ of 
$\tcomplex$ of dimension $m-3 = \dim F$.
Consider an arbitrary triangulation $T$ containing $F$, i.e., $F$ is obtained
from $T$ by deleting two edges $e,f$. 
%\comment{
By Proposition~\ref{prop:interior-faces}, $e$ and $f$ are both flippable in $T$ since they lie in a convex polygon in $T$.
%} 
%If one of these edges, say $e$, were not flippable in $T$, then $T\setminus \{e\}$ and hence $F$ would lie in the boundary $\partial \tcomplex$ and not be interior. Thus, both edges $e$ and $f$ must be flippable. 

If $e$ and $f$ are not incident to a common triangle in $T$, (or, equivalently, removing both $e$ and $f$ from $T$ creates two internally disjoint convex quadrilaterals) then there exist four triangulations containing $F$
and these form an elementary $4$-cycle in the flip graph. The $4$-cycle is by definition the boundary of the 
dual cell $F^*$. 

%\comment{
Otherwise, $e$ and $f$ are incident to a common triangle in $T$.  By Proposition~\ref{prop:interior-faces} the union of the three triangles of $T$ containing either $e$ or $f$ forms a convex polygon, necessarily a pentagon.
There are five 
triangulations containing $F$ and these form an elementary $5$-cycle in the flip graph. 
The 5-cycle is by definition the boundary of the dual cell $F^*$.
%} 

%
%If $e$ and $f$ are incident to a common triangle in $T$, and the union of the three triangles of $T$ containing either $e$ or $f$ forms a convex pentagon, then there are five 
%triangulations containing $F$ and these form an elementary $5$-cycle in the flip graph. 
%The 5-cycle is by definition the boundary of the dual cell $F^*$. 
%
%It remains to consider the case that the union of the three triangles of $T$ containing
%$e$ or $f$ is a non-convex pentagon with a single reflex vertex, see Figure~\ref{fig:NC-pentagon}. 
%In that case, there are three triangulations of $P$ containing $F$, 
%corresponding to the three triangulations of such a non-convex pentagon.
%These triangulations form a path of length $2$ in the flip graph, say $T',T'',T'''$ in that order.
%Then $T'$ contains an edge ($f$ in Figure~\ref{fig:NC-pentagon}) that is not flippable in $T'$,
%hence $T'\setminus \{f\}$
%is a boundary face containing $F$, i.e., $F$ is not interior 
%and does not give rise to a dual $2$-cell of $\fcomplex$.

Hence, every $2$-cell of $\fcomplex$ corresponds to an elementary $4$- or $5$-cycle of the flip graph.

Conversely, every elementary $4$- or $5$-cycle of the flip graph gives rise to a 2-cell $F^*$ of $\fcomplex$: more precisely, $F^*$ corresponds to the intersection of the triangulations in the elementary cycle.  
\end{proof}


%\begin{figure}
%\centering
%\includegraphics[width=0.6\linewidth]{NC-pentagon.pdf}
%\caption{The case of the proof of Theorem~\ref{thm:flip-complex} where $e$ and $f$ lie in a non-convex pentagon.}
%\label{fig:NC-pentagon}
%\end{figure}








