\section{Proofs for Biconnected Components}\label{sec:appendix-bcc}

\begin{proof}[Proof of Lemma~\ref{lem:locgraph-cons}]
Each cluster in the \implicit{} has at most $k$ vertices, so finding the vertices $V_i$ takes $O(ck)$ cost where $c$ is the cost to compute the mapping $\rho(\cdot)$ of a vertex ($O(k)$ in expectation and $O(k\log n)$ \whp{}).  Since each vertex has a constant degree, there will be at most $O(k)$ neighbor clusters, so $|V_o|=O(k)$.   Enumerating and checking the other endpoint of the edges adjacent to $V_i$ takes $O(ck)$ cost.  Finding the new endpoint of an edge in category 3 requires constant cost after an $O(n/k)$ preprocessing of the Euler tour of the cluster spanning tree.  The number of neighbor clusters is $O(k)$ so checking the cluster labels and adding edges costs no more than $O(k)$.
The overall cost to construct one \localgraph{} is thus $O(k^2)$ in expectation and $O(k^2\log n)$ \whp{}.
%After plugging in $c$ is $O(k)$ in expectation and $O(k\log n)$ \whp{}, the overall cost matches the bounds in the lemma.
\end{proof}

\begin{proof}[Proof of Lemma~\ref{lem:bridge-clust}]
If an edge is a bridge in the original graph it means that there are no edges from the subtree of the lower vertex to the outside except for this edge itself.  By applying the modifications of the edges, this property still holds, which means the edge is still a bridge in the \localgraph{} and vice versa.
\end{proof}
