\section{Expected performances}
\label{sec:simulation}
This  section   is  dedicated   to  presenting  the   performances  of
\mbox{GIGAS61} and  GIGADuck detectors in terms of  EAS detection.  In
the section~\ref{sec:mbrsim}  we detail  the method we  implemented to
estimate the  flux from an EAS.   In the section~\ref{sec:eventnumber}
we apply this method to estimate  the expected number of events in one
year of operation of \mbox{GIGAS61} and GIGADuck detectors.
\subsection{Simulation of the MBR signal from EAS}
\label{sec:mbrsim} 
The longitudinal profile of EAS is parameterised with a Gaisser-Hillas
function~\cite{gh} characterising the number of charged particles at a
certain depth.  The mean values and  RMS of the parameters used in the
Gaisser-Hillas  function  are  first  tabulated for  energies  between
\unit[10$^{17.5}$  and   10$^{21}$]{eV}.   A  randomisation   is  then
performed when generating  an event using a Gaussian  function for all
the parameters except for the depth of first interaction which follows
an  exponential  distribution.  Starting  from  the first  interaction
point,  high in  the atmosphere,  the number  of primary  electrons is
calculated in grammage steps  of \unit[$\Delta X$ = 2.5]{g cm$^{-2}$}.
At  each step,  the mean  energy  deposit per  particle is  calculated
following a parameterisation at \unit[1]{MeV} given in~\cite{nerling}.
The  lateral  distribution function  of  the  electrons  in the  plane
orthogonal to the shower axis  is taken as an NKG function~\cite{nkg1,
  nkg2}.\\The estimation  of the  flux of MBR  photons emitted  by the
ionisation electrons and received at ground is based on the derivation
presented  in~\cite{imen2016}.  It accounts  for the  MBR differential
cross section obtained in~\cite{MBRXsec} and the time evolution of the
shower plasma  as the ionisation  electrons get attached or  see their
energy shifted as they undergo ionisation or excitation reaction.  The
flux folded  with the antenna  effective area and integrated  over the
frequency  bandwidth yields  the power  envelope  of the  signal as  a
function of the time at the receiver (an example of the power envelope
is shown  in Figure~\ref{fig:yieldnormal}-left).  Following  the model
in~\cite{imen2016}  we  find a  spectral  intensity of  \unit[$\mathrm
  2\times10^{-26}]{W   \  m^{-2}   \   Hz^{-1}}$  for   a  shower   of
\unit[$10^{17.5}$]{eV}  observed  at  \unit[10]{km}.  This  estimation
allows for  the comparison with  other MBR studies. For  instance, the
same reference shower in the  same conditions would produce a spectral
intensity of \unit[$\mathrm 2.77\times10^{-24}]{W \ m^{-2} \ Hz^{-1}}$
according to the results of SLAC T471~\cite{Gorham}, the original beam
test. We introduce  a scale factor $R$ based on  the comparison of the
reference shower,  with $R=1$ for the  model that we  used and $R=140$
for the  SLAC T471 assumption. The  parameter $R$ is used  in the next
section to  assess the performance  of the detectors.\\To  account for
the detector response, the voltage  deduced from the power envelope is
multiplied  with a noise  waveform produced  according to  the spectra
measured  and presented  in  section~\ref{sec:calibrationsensor}.  The
resulting waveform is the simulation  of the RF voltage induced at the
output of the  antenna by the EAS.  A noise  waveform is produced with
the same spectrum, but the average power is normalized with the system
noise temperature.  We add the  two waveforms to emulates the total RF
voltage.  The adaptation electronics is then simulated as described in
section~\ref{sec:elecimpulsive} to obtain a waveform in ADCu.
\subsection {Expected event rate}
\label{sec:eventnumber}
For  a scale  factor $R$,  the number  of expected  events for  a time
period $\Delta T$ and for an  area labeled $S$ inside which the shower
core position is $x,y$ reads as:
%%\begin{linenomath*}
\begin{equation}
	\mu(R) = J_0\int_{>E_0}dE~f(E) \int_{\Delta \Omega}d\Omega~\cos{\theta}\int_{S}dxdy~\int_{\Delta T}dt~\epsilon(E,\theta,\phi,x,y;R),
\end{equation}
%%\end{linenomath*}
where  $\epsilon$, the  detection  efficiency, is  estimated with  the
simulations  described  below.   The  energy  $E$  of  the  shower  is
generated randomly  following the energy  spectrum in the  range above
the so-called ankle energy which  can be parameterised between E$_0$ =
4$\times$10$^{18}$~eV      and     3$\times$10$^{20}$~eV     according
to~\cite{schulz}:
%%\begin{linenomath*}
\begin{equation}
J(E;   E  >   E_0)   = J_0 f(E) = J_0   E^{-\gamma_2}   \left(1+  \exp   \left({
  \frac{\log_{10}E          -          \log_{10}E_{1/2}}{\log_{10}W_c}
}\right)\right)^{-1},
\end{equation}
%%\end{linenomath*}
where  $J_0$ is  a flux  normalisation factor  and the  spectral index
above the ankle $\gamma_2$ is 2.63. The term $\log_{10}E_{1/2}$ is the
energy at which the flux has  dropped to half of its peak value before
suppression, and $\log_{10}W_c$ is  its associated steepness. They are
fixed to  19.63 and  0.15 respectively. \\  Shower cores  are randomly
generated over a surface covering  an Auger hexagon, while the arrival
directions  $\theta$ and  $\phi$ are  randomly generated  to guarantee
uniformity  in terms  of  $\phi$ and  $\sin^2{\theta}$ (with  $\theta$
limited  to  60$^\circ$).  For  the  three detectors,  \mbox{GIGAS61},
\mbox{GIGADuck-C} and \mbox{GIGADuck-L}, we  simulate 5000 proton showers.  For each
shower we compute the MBR power  at the seven antennas of the hexagon.
Scale factors  $R$ from 1 to  1000 are applied and  the electronics is
then  simulated  ten  times  for  each $R$.   The  radio  waveform  is
transformed  in SNR  unit according  to: $P  [\text{SNR}] =  \frac{P -
  <P>}{\text{RMS}(P)}$   (see   Fig.~\ref{fig:yieldnormal}-left).   We
apply simple  selection criteria on  these processed data.   We select
events with a  waveform that passes a  threshold of SNR = 5  in a time
window of  \unit[1]{$\mu$s} around the expected time  of maximum.  The
expected number  of events for one  equipped hexagon within  a year of
data taking is  shown in Figure~\ref{fig:yieldnormal}-right, where the
abscissa  axis  is  the  scale factor.   The  initial  implementation,
\mbox{GIGAS61},  is already  sensitive to  the level  of  intensity as
measured by  SLAC T471 ($R = 140$)  but would observe only  one or two
events.  For the same scale  factor, this number increases by a factor
of  3  with  \mbox{GIGADuck-C}.   The  best performances  are  obtained  with
\mbox{GIGADuck-L} with possible detection down to scale factor of around $R =
10$.\\Several improvements in the  analysis would help to identify MBR
and are worth noting here.   The basic event selection can be improved
using digital filtering.  Such  analysis will particularly enhance the
long  duration signal  (a  few  $\mu$s) as  expected  for MBR  signal.
Furthermore,  in  contrast  to  the  geosynchrotron  emission  or  the
Askaryan effect, also present at  GHz frequencies, the MBR emission is
isotropic,  this gives the  possibility to  identify it  by requesting
that the  GHz radiation is  detected at large distances.   A plausible
selection criteria would  thus be performed on the  number of stations
that detected a  radio signal in coincidence with  the same shower. By
requesting  at least  two  stations  spaced on  the  regular SD  array
(1500~m   spacing),   one  would   discard   emissions  arising   from
geosynchrotron  or Askaryan  effects  (the expected  signals of  which
expand  over  a  few   hundred  meters  only).   Indeed,  the  antenna
orientation of GIGADuck antennas  was chosen optimized the coincidence
probability.  \\However, even with these methods, if the MBR intensity
is at the level  of the reference model~\cite{imen2016}, its detection
with the presented instruments is  hardly possible with only 0.1 event
expected per hexagon per year  at best.  To improve further the signal
to  noise  ratio,  other  experimental  techniques,  like  cryo-cooled
detectors, should be  considered. Note that the estimation  of the MBR
flux  is  delicate and  if  numerous  processes  are already  included
in~\cite{imen2016}  the  conclusions  on theoretical  predictions  are
uncertain  justifying the experimental  prospection being  carried out
with \mbox{GIGAS61} and the GIGADuck detectors.
\begin{figure}[t!]
\centering \includegraphics[width=0.49\textwidth]{simexampletrace2.png}
\includegraphics[width=0.49\textwidth]{sensitivity.pdf}
\caption{Left: Example  of a simulated  radio signal at the  output of
  the antenna  for a scale  factor of 10,  and when combined  with the
  full electronics  simulation.  Right:  expectation of the  number of
  events  per  hexagon  per  year  as  a function  of  the  MBR  scale
  factor.   The  factor  for   SLAC  T471   is  $R$=140   (with  $R$=1
  corresponding to~\cite{imen2016} ).}
\label{fig:yieldnormal}
\end{figure}  
