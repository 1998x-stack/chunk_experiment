\begin{figure}
\centering
    \begin{tikzpicture}
        \node[master] (M)at(0,0){Master};
        %
        \node[worker,below left = 1 and 2 of M.center] (W1) {Worker 1};
        \node[worker,below = 1 of M.center] (W2) {Worker 2};
        \node[worker,below right = 1 and 2 of M.center] (W3) {Worker 3};
        %
        \node[local_variable,below = 0.5 of W1] (V1) {$f_1, g_1, \A_1$}; % \x_1
        \node[local_variable,below = 0.5 of W2] (V2) {$f_2, g_2, \A_2$};
        \node[local_variable,below = 0.5 of W3] (V3) {$f_3, g_3, \A_3$};
        \node[local_variable,above = 0.5 of M] (Z) {$F, G, r, \M$}; % \z
        %
        % Links
        \draw[<->,>=latex,dashed] (M.west) to[in=90,out=-180] (W1.north);
        \draw[<->,>=latex,dashed] (M.south) -- (W2.north);
        \draw[<->,>=latex,dashed] (M.east) to[in=90,out=0] (W3.north);
        %
        % Links for variables
        \draw[very thick] (W1.south) --  (V1.north);
        \draw[very thick] (W2.south) --  (V2.north);
        \draw[very thick] (W3.south) --  (V3.north);
        \draw[very thick] (M.north) --  (Z.south);
    \end{tikzpicture}
    \caption{Illustration of the master-slave architecture considered for the unmixing problem \eqref{eq:problem} with $\Omega = 3$ workers (the function and variables available at each node are given in light gray rectangles).}
    \label{fig:architecture}
\end{figure}
