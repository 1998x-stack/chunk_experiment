\section{Data Collection and General Characteristics} 
We first describe the dataset we collected using our OutWithFriendz system, then conduct a data distribution 
analysis to understand the key characteristics of our dataset.

\subsection{Data Collection}
We deployed our OutWithFriendz mobile application on the Google Play and Apple Store marketplaces.
To collect enough data for group dynamics analysis, we posted advertisements on 
Microworkers~\cite{microworkers} and Craigslist~\cite{craigslist}
for participants. For teaching these users how to use our app correctly, we also made an introductory 
video, which is included in our supplemental file: ``OutWithFriendzIntroductionVideo.mp4''. For each 
legitimate completed invitation, we paid the host of a group 20 dollars, with the provisions that: (1) The host 
and participants must live in the US; (2) The host should invite at least two other 
friends to the invitation using our app; 
(3) The group must demonstrate a full voting process; (4) The host must finalize the meeting time  and location
for the invitation; (5) Each participant would open their location services on their smartphone during the study
and allow us to track their mobility traces; (6) At least half of the group members attended the finalized 
event\footnote{\added{We added this requirement to prevent workers from creating fake invitations and making dishonest money. It would be interesting to analyze the low-attendance events, which we plan to investigate when we have more users and events.}}. From these 
two job post websites, we collected 246 legitimate invitations over a 5-month period from 432 users. In addition, 71 
students on our campus used the app without getting any payment, which contributed another
76 legitimate invitations. The whole data collection period spanned from January 2016 to May 2017. In total, 503 distinct users
of our OutWithFriendz application were identified, generating 322 legitimate invitations. \added{141 users have been the host of at least one event, and 72 hosts have created exactly one event each.} Please note that 
each user is allowed to create and join multiple invitations in this study. \added{Moreover, 11 groups (7 from paid users and 4 from students in our university) have used our mobile app frequently, with more than six legitimate invitations in our dataset.}
Figure~\ref{fig:map} shows the distribution of all suggested locations recorded in our server across the US. 
It indicates that our users are widespread in 34 
different states and 81 cities throughout the country.  

\begin{figure}
\centering
\includegraphics[width=0.6\linewidth]{map}
\caption{The geographic distribution of all finalized locations across the US.}
\label{fig:map}
\end{figure}

\added{To better understand the demographics of our users, we have conducted an anonymized survey using
Google Form.  We contacted each user through his/her Facebook Page to complete the survey. 
In total, 294 of our users have completed the survey.
The results are shown in Figure~\ref{fig:demographic}. 
We can see that 60.8\% of the users who responded to the survey are
female and 48.5\% are self-employed. 
Students, including some from our campus and some from the Microworker website,
account for 27.5\% of our participants. In addition, most (83.0\%) of our users are young, aged 18-35. } 
\begin{comment}
Further analysis of our user demographics would be interesting. For instance,  
an earlier study reported that many housewives are using
the crowdsourcing market to earn extra money~\cite{gray2016crowd}.
This may also be reflected in our survey results, as housewives may be more
suited for this study because they have more free time to spend with friends.
This may due to the fact that mobile device usage is widespread among young people.
\end{comment} 

\begin{figure}
\centering
\includegraphics[width=1\linewidth]{demographics}

\caption{User demographic information of OutWithFriendz Users (294 participants). (L) Gender distribution. (M) Age distribution. (R) Profession distribution}
\label{fig:demographic}
\end{figure}

\subsection{Location Trace Data}
\label{sec:trace}
In addition to collecting data about the event organization process, we also collected user mobility-related data.  The OutWithFriendz app posts GPS user location traces to the server
either every 5 minutes if the app is running in the background or every 30 seconds if the app is running
in the foreground. Before participation all of our users were required to provide informed IRB consent. They would  
turn on the location services on their smartphone during the test so we were able to collect the data.  
All location traces are anonymized and permission to use this anonymized data is provided when installing the 
app. The total amount of user traces data collected was about 1.1 GB. 

\subsection{Group Size Distribution}
Figure~\ref{fig:groupsize} left summaries the distribution of group size in our OutWithFriendz dataset. \added{Here we define group size as the number of participants who finally stayed in the invitation. We do not count users who were removed from the invitation, either by themselves or by the host, because they did not  participate in the whole scheduling process and their votes were not shown after the removal.}  We observe that most of the groups in our study are small, with the large majority being groups of three.  We were pleased to see a significant fraction of groups with five (11.5\%) and six members (4.2\%) who were able to use the app concurrently.
\iffalse
The size of most groups participating in our study is not large, which commonly represent the common
case of user hanging out with their close friends or family members for meals. In addition, we also 
have some medium groups. For example, some users on campus will create an event to decide when
to play basketball in Recreation Center. Please note that we don't count any events with two users in our study
as the group size is too small to use our voting mechanism. 
\fi
Our work focused primarily on obtaining data for groups of three or more, which we feel represent many typical social group interactions of interest to us.  As a result, we did not focus on examining pairwise groups in this study.  The figure's trend lines suggest that if we had opened up our study to pairwise groups, then our data would have been overwhelmingly skewed toward pairwise groups.  However, now that we have obtained substantial initial data for larger groups, we plan to also explore the behavior of pairwise groups in our future works.

\begin{figure}
\centering
\includegraphics[width=0.32\linewidth]{distribution_group_size}
\includegraphics[width=0.32\linewidth]{distribution_days_decision}
\caption{The distribution of group size (left) and number of days to make final decision (right).}
\label{fig:groupsize}
\end{figure}

\subsection{Distribution of Days to Make Final Decision}
We are interested in the duration that it took for event organizers to make their final decisions. 
\replaced{As shown by the right figure in Figure~\ref{fig:groupsize}, the number of days to make the final decision is somewhat evenly distributed, and there is no dominant duration in this distribution. This is a bit  surprising or counter-intuitive, since we expected that there may be a more pronounced duration of decision-making within the first couple of days of creating an invitation. }{It is a little
surprising that we find the distribution of days to make the final decision to be somewhat evenly distributed,
which is illustrated in Figure~\ref{fig:groupsize} right.  No duration dominates this distribution, while we expected that there might be a more pronounced bunching of decision-making in the first couple of days.} However, there are also a substantial fraction of events that took four or more days to decide (about 30\%), indicating that a large fraction of hosts are taking a long time to decide. 
\added{This may be affected by the type of events and the amount of lead time. For daily meals, users can make a decision within thirty minutes while for some weekend activities, they will start planning it at the beginning of the week.}

\subsection{Voting Distribution of Individual Users}
Voting distribution is based on the number of votes made per individual user. The distributions for time 
and place voting are shown in Figure~\ref{fig:votedis}. The majority of users will vote for one 
option as far as the event time.  Similarly, the majority of users will vote for one option in terms of the place voting. In both processes, around 10\% didn't vote and 10\% voted for more than
2 options.  This voting behavior is analyzed further in later sections.
\iffalse
To better understand the group voting process, we analyzed all these votes and observed 
some interesting patterns regarding group users' voting behaviors. This will be further
discussed in later sections.
\fi
\begin{figure}
\centering
\includegraphics[width=0.32\linewidth]{vote_date}
\includegraphics[width=0.32\linewidth]{vote_loc}
\caption{The distribution of number of votes by a single user for event time (left) and place (right).}
\label{fig:votedis}
\end{figure}

\subsection{Distribution of the Proportion of the Votes}

\replaced{We also analyze the proportion of users who voted for event time or location in the final decision.} 
{We conduct an analysis that refers to the proportion of votes the poll winner achieved versus the other contenders.}
The distribution is shown in Figure~\ref{fig:winner}. 
More than 70\% of the \replaced{final decisions }{winners} for both time and location
received majority votes to become the final choice. \added{This is understandable since groups tend to 
agree on the majority votes.} For the remaining 30\%, we observe some
very interesting behavior. In these polls, the final \replaced{decisions}{winners} did not receive the majority of the votes. In fact, in a small fraction of cases, there is a non-zero proportion of polls in which the \replaced{final decision}{winner} received no votes.  In these cases, the group host, who is the only one with the power to finalize the \replaced{event time and location }{place/date}, decided to override the \added{majority} voting results, either by personal fiat or possibly through a discussion with other group members that caused them to change their minds.

\iffalse
 not the ones which receive the most votes, or even receive 
zero votes. Due to the fact that final options are only decided by group host, maybe host doesn't like
the poll results. Another reasonable explanation can be that the group members discussed their 
availabilities and changed their minds in the end. We will further discuss this abnormal behavior in our
group decision section.
\fi
\begin{figure}
\centering
\includegraphics[width=0.7\linewidth]{winner}
\caption{The proportion of users who voted for final event time (red) and location (blue).}
\label{fig:winner}
\end{figure}

\subsection{Suggestion Distribution}
OutWithFriendz app allows group participants not only to vote for their preferences, but also to suggest 
new options. Figure~\ref{fig:suggest} shows the suggestion distributions for host and participants.
Most hosts will suggest 2 or more options for the event. We also observe a small portion of hosts who provide no options of their own, and rely on other group members to provide suggestions.  For
participants, more than 60\% didn't make new suggestions. They just vote for the existing options. 
Some made one new suggestion while very few of them would make too many new suggestions. 
We will further compare the influence of group host and participants in our group decision section.
\begin{figure}
\centering
\includegraphics[width=0.32\linewidth]{sugg_host}
\includegraphics[width=0.32\linewidth]{sugg_member}
\caption{The distribution of the number of suggestions made by group host (left) and other participants (right).}
\label{fig:suggest}
\end{figure}

\deleted{We calculated the gender distribution of our OutWithFriendz users, which were 
about 42\% male and 58\% female. Please note that our app requires a user's 
Facebook account to log in, but it is not permitted by Facebook to get the accurate 
gender information using Facebook user ID. So here we estimate users' gender based
on manually reading their Facebook name and profiles. It is possible that some users' genders
are miscounted here but we expect the proportion to be rare. }

\subsection{\added{Metro vs Non-metro Areas}}

\added{Using the location trace data we collected from our users and the U.S. census data, we are able to identify locations frequently visited by our users. The technique we used will be introduced in detail in Section 5. Then we can project each user's home county using the frequently visited locations. According to the latest Rural-Urban Continuum Codes released in May 2013~\cite{rural}, every county is classified as a non-metro or metro area. In our dataset, 48\% the users live in metro areas and 52\% live in non-metro areas.} \footnote{\added{Please note our dataset contains 71 students who lived in Boulder doing this study. If we remove this student population, the proportion of metro and non-metro users would be 39\% and 61\%, all from crowdsourcing market users.}}

\iffalse
The general gender distribution is reasonable because many users 
we hired from Microworkers or Craigslist are housewives who have time to make small
money. They created invitations regularly to have coffee or go shopping with their friends using
our App, which makes female ratio higher than male.
\fi

\subsection{Weather Factor}
An important external factor that can influence event organization relates to weather. Here we 
examine the impact of rain and temperature on our dataset. For this analysis, weather and
temperature information for each event was scraped from weathersource 
API~\cite{weathersource} at its location and starting time in our dataset. Note that we can only 
get hourly weather data from weathersource. If the starting time of one event is 19:35, in the 
analysis we use 19:00 weather data at the same day crawled from weathersource. Here we 
decide it is raining if the precipitation of an event's starting time is above 0. On snowy days, usually 
the precipitation will also be above 0, which we classify as rainy in our analysis.

Figure~\ref{fig:weather} shows the distribution of events that happened in 
rainy weather or not. 82.3\% of events organized in our app
occur in non-rainy weather. There are two reasonable explanations:
(1) In many places around the country non-rainy days happen more often than rainy
days; (2) Bad weather would have negative influence on real event attendance.
Looking deeper, bad weather appears to affect the types of events that are organized.
In OutWithFriendz, any place can be added to the Google Map as an option for voting, and need 
not be confined to a restaurant only.  For example, people have used the app to organize events such
as outdoor hiking and going to the movies.  We divide all events into two category types: 
meal events and other events. Meal events refer to people hanging out for lunch or
dinner, which is the majority event type in our dataset.  Other events include activities
that are not primarily dining, e.g. sporting and entertainment events.  In our study, we found 
that bad weather would have less impact on meal events compare with other types of events.
Figure~\ref{fig:weather} show that 66.7\% of the events belong to meal events on non-rainy 
weather while this number goes up to 81.1\% on rainy days.

\begin{figure}
\centering
\includegraphics[width=0.9\linewidth]{weather_combined}

\caption{(L) The distribution of events on rainy days vs. non-rainy days; (R) The 
distribution of meal events and other events on non-rainy days and rainy days.}
\label{fig:weather}
\end{figure}

\begin{comment}
We also find temperature plays a significant role and the effect is two-fold. The amount of events
in our dataset increases when temperature is below $70^oF$. This phenomenon can be caused 
by both long term and short term reasons. In the long term, people may go out more frequently as
the warm spring progresses. In the short term, users may choose not to meet at colder times. For
temperatures above $70^oF$, the amount of events decrease due to the very hot weather. 

\begin{figure}
\centering
\includegraphics[width=0.48\linewidth]{temp}
\caption{The hourly distribution of events by weather on (L) weekday and (R) weekend.}
\label{fig:temperature}
\end{figure}
\end{comment}




