%% bare_jrnl.tex
%% V1.4a
%% 2014/09/17
%% by Michael Shell
%% see http://www.michaelshell.org/
%% for current contact information.
%%
%% This is a skeleton file demonstrating the use of IEEEtran.cls
%% (requires IEEEtran.cls version 1.8a or later) with an IEEE
%% journal paper.
%%
%% Support sites:
%% http://www.michaelshell.org/tex/ieeetran/
%% http://www.ctan.org/tex-archive/macros/latex/contrib/IEEEtran/
%% and
%% http://www.ieee.org/

%%*************************************************************************

%% This work is distributed under the LaTeX Project Public License (LPPL)
%% ( http://www.latex-project.org/ ) version 1.3, and may be freely used,
%% distributed and modified. A copy of the LPPL, version 1.3, is included
%% in the base LaTeX documentation of all distributions of LaTeX released\appendix

%\documentclass[journal,12pt,onecolumn,draftclsnofoot]{IEEEtran}


\documentclass[journal]{IEEEtran}
\usepackage{ifpdf}
\usepackage[noadjust]{cite}
\usepackage[cmex10]{amsmath}
\usepackage{array}
\usepackage{dblfloatfix}
\usepackage{amssymb}
\usepackage{graphicx}
\usepackage{breqn}
\usepackage{amsthm}
\usepackage{amsmath}
\usepackage{color}
\usepackage{makecell}

\newtheorem{definition}{Definition}[section]
\newtheorem{theorem}{Theorem}

\DeclareMathOperator{\tr}{tr}
\DeclareMathOperator{\eig}{eig}
\DeclareMathOperator{\diag}{diag}
\DeclareMathOperator{\rank}{rank}
\bibliographystyle{IEEEtran}

%


% *** GRAPHICS RELATED PACKAGES ***
%
\ifCLASSINFOpdf

  % \usepackage[pdftex]{graphicx}
  % declare the path(s) where your graphic files are
  % \graphicspath{{../pdf/}{../jpeg/}}
  % and their extensions so you won't have to specify these with
  % every instance of \includegraphics
  % \DeclareGraphicsExtensions{.pdf,.jpeg,.png}
\else
  % or other class option (dvipsone, dvipdf, if not using dvips). graphicx
  % will default to the driver specified in the system graphics.cfg if no
  % driver is specified.
  % \usepackage[dvips]{graphicx}
  % declare the path(s) where your graphic files are
  % \graphicspath{{../eps/}}
  % and their extensions so you won't have to specify these with
  % every instance of \includegraphics
  % \DeclareGraphicsExtensions{.eps}
\fi
% graphicx was written by David Carlisle and Sebastian Rahtz. It is
% required if you want graphics, photos, etc. graphicx.sty is already
% installed on most LaTeX systems. The latest version and documentation
% can be obtained at:
% http://www.ctan.org/tex-archive/macros/latex/required/graphics/
% Another good source of documentation is "Using Imported Graphics in
% LaTeX2e" by Keith Reckdahl which can be found at:
% http://www.ctan.org/tex-archive/info/epslatex/
%
% latex, and pdflatex in dvi mode, support graphics in encapsulated
% postscript (.eps) format. pdflatex in pdf mode supports graphics
% in .pdf, .jpeg, .png and .mps (metapost) formats. Users should ensure
% that all non-photo figures use a vector format (.eps, .pdf, .mps) and
% not a bitmapped formats (.jpeg, .png). IEEE frowns on bitmapped formats
% which can result in "jaggedy"/blurry rendering of lines and letters as
% well as large increases in file sizes.
%
% You can find documentation about the pdfTeX application at:
% http://www.tug.org/applications/pdftex

% *** ALIGNMENT PACKAGES ***
% *** SUBFIGURE PACKAGES ***
% *** FLOAT PACKAGES ***
% *** PDF, URL AND HYPERLINK PACKAGES ***
%
% correct bad hyphenation here
\hyphenation{op-tical net-works semi-conduc-tor}

\begin{document}
\large
%
% paper title
% Titles are generally capitalized except for words such as a, an, and, as,
% at, but, by, for, in, nor, of, on, or, the, to and up, which are usually
% not capitalized unless they are the first or last word of the title.
% Linebreaks \\ can be used within to get better formatting as desired.
% Do not put math or special symbols in the title.
\title{
Robust Radar Detection of a Mismatched Steering Vector Embedded in Compound Gaussian Clutter
}
\author{Mai P. T. Nguyen and
        Iickho Song,~\IEEEmembership{Fellow,~IEEE}
        \thanks{The authors are with the School
of Electrical Engineering, Korean Advanced Institute of Science and Technology, Daejeon 34141, Korea.}
}% <-this % stops a space
% <-this % stops a space
%\thanks{Manuscript received April 19, 2005; revised September 17, 2014.}}
% make the title area
\maketitle

% As a general rule, do not put math, special symbols or citations
% in the abstract or keywords.
\begin{abstract}
The problem of radar detection in compound Gaussian clutter when
a radar signature is not completely known has not been considered yet and
is addressed in this paper.
We proposed a robust technique
to detect, based on the generalized likelihood ratio test, a point-like target 
embedded in compound Gaussian clutter.
Employing an array of antennas, we assume that
the actual steering vector departs from 
the nominal one, but lies in a known interval.
The detection is then secured by employing a semi-definite
programming.
It is confirmed via simulation that the proposed detector
experiences a negligible detection
loss compared to an adaptive normalized matched filter in a perfectly matched case,
 but outperforms in cases of mismatched signal.
 Remarkably, the proposed detector possesses constant false alarm rate with
respect to the clutter covariance matrix.
\end{abstract}

% Note that keywords are not normally used for peerreview papers.
\begin{IEEEkeywords}
Generalized likelihood ratio test, compound Gaussian clutter, semi-definite programming
\end{IEEEkeywords}

\IEEEpeerreviewmaketitle
\section{Introduction}
\input{sirv_intro.tex}

\section{Problem Formulation and Proposed Detector}
\input{prob_formulation}
\input{subsec_a}
\input{subsec_b}

\section{Numerical Results}
\input{numerical_results}

\section{Conclusion}
This paper has addressed the problem of detecting a mismatched signal embedded in compound Gaussian noise.
Specifically, phase shifting of the actual steering vector departs from that of the nominal one but
belongs to a known interval.
The proposed detector is shown to be more robust to mismatched signals than the adaptive NMF,
and even achieves reasonable detection probabilities when the signal to detect lying out of the designed interval.
 Remarkably, the $\theta$-MLE detector has CFAR w.r.t all statistic of noise.
%  but $\theta$-SCM do not have CFAR w.r.t \textit{texture} distribution.
%However, $\theta$-SCM requires less computation than $\theta$-MLE.
 A drawback of the proposed detector is that the likelihood ratio has no explicit form, for which
 it is difficult to gain a deeper insight into the performance of the detector.
Another drawback is the complexity associated with the SDP.
Though proposed scheme can detect a seriously mismatched signal, it does not include
effects of possible interference, which might be a topic for a further research.
%\input{app-is2.tex}

% use section* for acknowledgment
\section*{Acknowledgment}
The authors would like to thank the anonymous reviewers for their valuable comments and suggestions to improve the paper.
% Can use something like this to put references on a page
% by themselves when using endfloat and the captionsoff option.
\ifCLASSOPTIONcaptionsoff
  \newpage
\fi
% biography section
\input{reference}
%
% If you have an EPS/PDF photo (graphicx package needed) extra braces are
% needed around the contents of the optional argument to biography to prevent
% the LaTeX parser from getting confused when it sees the complicated
% \includegraphics command within an optional argument. (You could create
% your own custom macro containing the \includegraphics command to make things
% simpler here.)
%\begin{IEEEbiography}[{\includegraphics[width=1in,height=1.25in,clip,keepaspectratio]{mshell}}]{Michael Shell}
% or if you just want to reserve a space for a photo:

% if you will not have a photo at all:

% insert where needed to balance the two columns on the last page with
% biographies
%\newpage

% You can push biographies down or up by placing
% a \vfill before or after them. The appropriate
% use of \vfill depends on what kind of text is
% on the last page and whether or not the columns
% are being equalized.

%\vfill

% Can be used to pull up biographies so that the bottom of the last one
% is flush with the other column.
%\enlargethispage{-5in}
% that's all folks
\end{document}
