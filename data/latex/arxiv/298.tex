%!TEX root = main.tex
\section{Experiments}
\label{sec:exp}



\subsection{Datasets}

In our experiments, we use two datasets Flickr30K~\cite{elliott:2016} 
and MSCOCO~\cite{MSCOCO} which are provided by the WMT organization.
For both datasets, there are triples that contains English as source 
sentence, its German and French human translations and corresponding 
image. The system is only trained on Flickr30K datasets but are also tested on 
MSCOCO besides Flickr30K. 
MSCOCO datasets are considered out-of-domain (OOD) testing while 
Flickr30K dataset are considered in-domain testing.
The datasets' statics is shown in Table~\ref{tb:data}

\begin{table}[htbp]
\centering
\scalebox{0.9}{
\begin{tabular}{ |l|c|c|c|c| }
\hline
            Datasets     & Train     &  Dev &  Test  & OOD ?\\ 
\hline
            Flickr30K    &  $29,000$  &    $1,014$   & $1,000$   & No\\
\hline
            MSCOCO    &  -  &    -   & $461$   & Yes\\
\hline
\end{tabular}
}
\caption{Summary of datasets statistics.}
\label{tb:data}
\end{table}


\subsection{Training details}

For preprocessing, we convert all of the sentences to lower case, normalize the 
punctuation, and do the tokenization. For simplicity, our vocabulary keeps all the 
words that show in training set. 
For image representation, we use ResNet~\cite{kaiming:2016} generated image features which are
provided by the WMT organization. In our experiments, we only use average pooled features.

Our implementation is adapted from on Pytorch-based OpenNMT~\cite{2017opennmt}. We use two layered
bi-LSTM~\cite{Sutskever:2014} on the source side as encoder. Our batch size is 64, with SGD optimization and a learning rate at 1.
For English to German, the dropout rate 
is 0.6, and for English to French, the dropout rate is 0.4. These two parameters are selected by
observing the performance on development set. Our word embeddings are randomly initialized with 
500 dimensions. 
The source side vocabulary is 10,214 and the target side vocabulary is 18,726 for German 
and 11,222 for French.

\subsection{Beam search with length reward}

During test time, beam search is widely used to improve the output text quality by giving the decoder 
more options to generate the next possible word.
However, different from traditional beam search in phrase-based MT where all hypotheses 
know the number of steps to finish the generation, while in neural-based generation, 
there is no information about what is the most ideal number of steps to finish the decoding.
The above issue also leads to another problem that the beam search in neural-based MT prefers shorter sequences due to probability-based scores for evaluating different candidates.
In this paper, we use Optimal Beam Search~\cite{huang+:2017} (OBS) during decoding time.
OBS uses bounded length reward
mechanism which allows a modified version of
our beam search algorithm to remain optimal.

Figure~\ref{fig:b} and Figure~\ref{fig:l} show the BLEU score and length ratio with different rewards for different beam size. We choose beam size equals to 5 and reward equals to 0.1 during decoding. 

\begin{figure*}
\centering
\begin{minipage}{0.5\textwidth}
  \centering
  \includegraphics[width=1\linewidth]{figs/bs_bleus.pdf}
  \captionof{figure}{BLEU vs. beam size}
  \label{fig:b}
\end{minipage}%
\begin{minipage}{.5\textwidth}
  \centering
  \includegraphics[width=1\linewidth]{figs/bs_length.pdf}
  \captionof{figure}{length ratio vs. beam size}
  \label{fig:l}
\end{minipage}
\end{figure*}

\subsection{Results}

WMT organization provides three different evaluating metrics: BLEU~\cite{Papineni:2002}, METEOR~\cite{Lavie2009} and TER~\cite{Snover06astudy}. 

Table~\ref{tb:fen2de} to Table~\ref{tb:cen2fr} summarize the performance with their corresponding rank among all other systems. 
We only 
show a few top performing systems in the tables to make a comparison. OSU1 is our proposed model and OSU2 
is our baseline system without any image information. 
For MSCOCO dataset, the translation from English to 
German (Table~\ref{tb:cen2de}), which is the hardest tasks compared with others since it is from 
English to German on OOD dataset, we 
achieve best TER score across all other systems. 



\begin{table}[htbp]
\centering
\scalebox{0.9}{
\begin{tabular}{ |c|c|c|c|c| }
\hline 
            System      & Rank & TER     &  METEOR &  BLEU  \\ 
\hline
            UvA-TiCC    & 1   &  \textbf{47.5}  &    53.5   & \textbf{33.3} \\
\hline
            NICT        & 2 &  48.1 &    \textbf{53.9}   & 31.9  \\
\hline
            LIUMCVC   & 3 \& 4 &  48.2 &    53.8   & 33.2  \\
\hline
            CUNI   & 5 &  50.7 &    51   & 31.1  \\
\hline
            $\text{OSU2}^\dagger$  & 6 &  50.7 &   50.6   & 31  \\
\hline
            $\text{OSU1}^\dagger$  & 8 &  51.6 &   48.9   & 29.7  \\
\hline
\end{tabular}
}
\caption{Experiments on Flickr30K dataset for translation from English to German. 16 systems in total. $\dagger$ represents our system.}
\label{tb:fen2de}
\end{table}


\begin{table}[htbp]
\centering
\scalebox{0.9}{
\begin{tabular}{ |c|c|c|c|c| }
\hline
            System    & Rank & TER     &  METEOR &  BLEU  \\ 
\hline
            \color{blue}\text{OSU1}   & \color{blue}1 &  \color{blue}\textbf{52.3}  &    \color{blue}46.5   & \color{blue}27.4 \\
\hline
            UvA-TiCC   & 2 &  52.4 &    48.1   & 28  \\
\hline
            LIUMCVC   &  3 &  52.5 &    \textbf{48.9}   & \textbf{28.7}  \\
\hline
            \color{blue}$\text{OSU2}$  & \color{blue}8 &  \color{blue}55.9 &   \color{blue}45.7  & \color{blue}26.1  \\
\hline
\end{tabular}
}
\caption{Experiments on MSCOCO dataset for translation from English to German. 15 systems in total. $\dagger$ represents our system.}
\label{tb:cen2de}
\end{table}



\begin{table}[htbp]
\centering
\scalebox{0.9}{
\begin{tabular}{ |c|c|c|c|c| }
\hline
            System   & Rank  & TER     &  METEOR &  BLEU  \\ 
\hline
            LIUMCVC  & 1  &  \textbf{28.4}  &   \textbf{72.1}   & \textbf{55.9} \\
\hline
            NICT  & 2  &  28.4 &    72   & 55.3  \\
\hline
            DCU   & 3  &  30 &    70.1   & 54.1  \\
\hline
            $\text{OSU2}^\dagger$ & 5   &  32.7 &   68.3   & 51.9 \\
\hline
            $\text{OSU1}^\dagger$ & 6    &  33.6 &   67.2   & 51 \\
\hline
\end{tabular}
}
\caption{Experiments on Flickr30K dataset for translation from English to French. 11 systems in total. $\dagger$ represents our system.}
\label{tb:fen2fr}
\end{table}


\begin{table}[htbp]
\centering
\scalebox{0.9}{
\begin{tabular}{ |c|c|c|c|c| }
\hline
            System    & Rank & TER     &  METEOR &  BLEU  \\ 
\hline
            LIUMCVC  & 1  &  \textbf{34.2}  &   \textbf{65.9}   & \textbf{45.9} \\
\hline
            NICT   & 2   &  34.7 &    65.6   & 45.1  \\
\hline
            DCU  & 3   &  35.2 &    64.1   & 44.5 \\
\hline
            \color{blue}$\text{OSU2}$  & \color{blue}4  &  \color{blue}36.7 &   \color{blue}63.8   & \color{blue}44.1 \\
\hline
            \color{blue}$\text{OSU1}$  & \color{blue}6  &  \color{blue}37.8 &   \color{blue}61.6   & \color{blue}41.2 \\
\hline
\end{tabular}
}
\caption{Experiments on MSCOCO dataset for translation from English to French. 11 systems in total.}
\label{tb:cen2fr}
\end{table}

\begin{figure*}[htbp]
\centering
   \begin{subfigure}[b]{1\textwidth}
   \includegraphics[width=2.5cm]{figs/m2_1.jpg}
   \;
    \scalebox{0.85}{
    \begin{tabular}[b]{cl}\hline
      input & a finger pointing at a hotdog with cheese , sauerkraut and ketchup . \\
      OSU1 &  ein finger zeigt auf einen hot dog mit einem messer , wischmobs und napa . \\
      OSU2 &  ein finger zeigt auf einen hotdog mit hammer und italien .  \\
      Reference & ein finger zeigt auf einen hotdog mit käse , sauerkraut und ketchup .  \\ \hline
    \end{tabular}}
\end{subfigure}

\vspace{3mm}

\begin{subfigure}[b]{1\textwidth}
   \includegraphics[width=2.5cm]{figs/m2_2.jpg}
   \;
    \scalebox{0.85}{
    \begin{tabular}[b]{cl}\hline
      input & a man reaching down for something in a box  \\
      OSU1 &   ein mann greift nach unten , um etwas zu irgendeinem .  \\
      OSU2 &  ein mann greift nach etwas in einer kiste .   \\
      Reference & ein mann bückt sich nach etwas in einer schachtel .   \\ \hline
    \end{tabular}}
\end{subfigure}
\caption{Two testing examples that image information confuses the NMT model.}
\label{tb:bad}
\end{figure*}


\begin{figure*}[htbp]
\centering
   \begin{subfigure}[b]{1\textwidth}
   \includegraphics[width=2.5cm]{figs/m1_1.jpg}
   \;
    \scalebox{0.85}{
    \begin{tabular}[b]{cl}\hline
      input & there are two foods and one drink set on the clear table .  \\
      OSU1 & da sind zwei speisen und ein getränk am klaren tisch .   \\
      OSU2 &  zwei erwachsene und ein erwachsener befinden sich auf dem rechteckigen tisch .    \\
      Reference & auf dem transparenten tisch stehen zwei speisen und ein getränk .    \\ \hline
    \end{tabular}}
\end{subfigure}

\vspace{3mm}

\begin{subfigure}[b]{1\textwidth}
   \includegraphics[width=2.5cm]{figs/m1_2.jpg}
   \;
    \scalebox{0.85}{
    \begin{tabular}[b]{cl}\hline
      input & a camera set up in front of a sleeping cat .   \\
      OSU1 & eine kameracrew vor einer schlafenden katze .     \\
      OSU2 &  eine kamera vor einer blonden katze .     \\
      Reference & eine kamera , die vor einer schlafenden katze aufgebaut ist     \\ \hline
    \end{tabular}}
\end{subfigure}
\caption{Two testing examples that image information helps the NMT model.}
\label{tb:good}
\end{figure*}

As describe in section~\ref{sec:model}, OSU1 is the model with image information for both encoder 
and decoder, and OSU2 is only the neural machine translation baseline without any image information. 
From the above results table we found that image information would hurt the performance in some cases.
In order to have more detailed analysis, we show some test examples for the translation from English to 
German on MSCOCO dataset. 

Fig~\ref{tb:bad} shows two examples that NMT baseline model performances better than OSU1 model. 
In the first example, OSU1 generates several unseen objects from given image, such like knife. 
The image feature might not represent the image accurately.
For the second example, OSU1 model ignores the object ``box'' in the image.

Fig~\ref{tb:good} shows two examples that image feature helps the OSU1 to generate better results.
In the first example, image feature successfully detects the object ``drink'' while the baseline
completely neglects this.
In the second example, the image feature even help the model figure out the action of the cat is ``sleeping''.







% \begin{figure*}[htbp]
%     \centering
%     \includegraphics[width=8cm]{figs/m1_3.jpg}
%     \;
%     \scalebox{0.85}{
%     \begin{tabular}[b]{cl}\hline
%       input & four pictures of a room showing a bed , pictures and a laptop on a desk .  \\
%       OSU1 & vier bilder eines raums , bilder und einem laptop auf einem schreibtisch .    \\
%       OSU2 &  vier bilder aus einem raum , das ein bett zeigt , bilder und laptop an einem schreibtisch .  \\
%       Reference & vier bilder eines zimmers mit einem bett , bildern und einem laptop auf einem schreibtisch \\ \hline
%     \end{tabular}}
%     \caption{One example that image feature do not recognize bed but help the NMT to generate more accurate sentence.}
% \end{figure*}