\documentclass[a4paper]{jpconf}%\documentclass[12pt,a4paper,reqno]{amsart}
\usepackage{amsfonts,amsthm,amsmath,amssymb,%amscd,
	amsbsy,euscript}%,textcomp,url} %,stmaryrd} cannot find "stmaryrd"? %,float} %,textcomp}
%\usepackage{pgf,tikz,pifont}
%\usetikzlibrary{arrows}
%\usepackage{listings}

%\usepackage[english]{babel}

% 30 September, 6 October 2017; minor amendments 11 January 2018 (v2)
% Edited last by:  AVK

%\usepackage[usenames,dvipsnames]{xcolor}
%\definecolor{MapleRed}{rgb}{1,0,0}
%\definecolor{MapleBlue}{rgb}{0,0,1}
%\definecolor{MaplePink}{rgb}{1,0,1}

%\def\MapleInput#1{\noindent{{\small $>$ {\tt \color{MapleRed}{#1} }}}}
%\def\MapleOutput#1{{\begin{center} \color{MapleBlue}{#1} \end{center}}}
%\def\MapleWarning#1{\noindent{{\small {\tt \color{MaplePink}{#1} }}}}

%\setcounter{tocdepth}{1}

\newtheorem{theor}{Theorem}%[section]
\newtheorem{claim}[theor]{Claim}
\theoremstyle{definition}
\newtheorem*{theorNo}{Theorem}
\newtheorem*{convention}{Convention}
\newtheorem{state}[theor]{Proposition}%[section]
\newtheorem{proposition}[theor]{Proposition}%[section]
\newtheorem{lemma}[theor]{Lemma}%[section]
\newtheorem{cor}[theor]{Corollary}%[section]
\newtheorem{conjecture}[theor]{Conjecture}%[section]
\newtheorem{define}{Definition}%[section]
\newtheorem*{defNo}{Definition}
\newtheorem*{notation}{Notation}
\newtheorem{problem}{Problem}%[section]
\newtheorem{open}[problem]{Open problem}
\newtheorem{property}{Property}%[section]
\newtheorem{example}{Example}%[section]
\newtheorem{counterexample}[example]{Counterexample}%[section]
\newtheorem{exercise}{Exercise}%[section]
\theoremstyle{remark}
\newtheorem{rem}{Remark}%[section]
\newtheorem*{remNo}{Remark}
\newtheorem*{Finalrem}{Final remark}
\newtheorem*{motivation}{Motivation}%[section]

\newcommand{\cEv}{\partial}
\newcommand{\pinner}{\mathbin{\mathchoice
{\hbox{\vrule width0.6em depth0pt height0.4pt
	\vrule width0.4pt depth0pt height0.8ex}}
{\hbox{\vrule width0.6em depth0pt height0.4pt
	\vrule width0.4pt depth0pt height0.8ex}}
{\hbox{\kern0.14em
	\vrule width0.48em depth0pt height0.4pt
	\vrule width0.4pt depth0pt height0.6ex\kern0.14em}}
{\hbox{\kern0.1em
	\vrule width0.39em depth0pt height0.4pt
	\vrule width0.4pt depth0pt height0.5ex\kern0.1em}}}}
\newcommand{\inner}{\pinner\,}

\let \wt=\widetilde
\newcommand{\BBR}{\mathbb{R}}\newcommand{\BBC}{\mathbb{C}}
\newcommand{\BBF}{\mathbb{F}}\newcommand{\BBN}{\mathbb{N}}
\newcommand{\BBS}{\mathbb{S}}\newcommand{\BBT}{\mathbb{T}}
\newcommand{\BBZ}{\mathbb{Z}}\newcommand{\BBE}{\mathbb{E}}
\newcommand{\EuA}{{{\EuScript A}}}
\newcommand{\EuX}{{{\EuScript X}}}
\newcommand{\cA}{{{\EuScript A}}}%{\mathcal{A}}
\newcommand{\bcA}{\boldsymbol{\mathcal{A}}}
\newcommand{\mcA}{\mathcal{A}}
\newcommand{\bcP}{{\boldsymbol{\mathcal{P}}}}
\newcommand{\bcQ}{{\boldsymbol{\mathcal{Q}}}}
\newcommand{\cB}{\mathcal{B}}
\newcommand{\cC}{\mathcal{C}}\newcommand{\tcC}{\smash{\widetilde{\mathcal{C}}}}
\newcommand{\cD}{\mathcal{D}}
\newcommand{\tcX}{\smash{\widetilde{\mathcal{X}}}}
\newcommand{\cE}{\mathcal{E}}\newcommand{\tcE}{\smash{\widetilde{\mathcal{E}}}}
\newcommand{\cEL}{\mathcal{E}_{\IL}}
\newcommand{\cEEL}{{\cE}_{\text{\textup{EL}}}}
\newcommand{\cEKdV}{{\cE}_{\text{\textup{KdV}}}}
\newcommand{\cELiou}{{\cE}_{\text{\textup{Liou}}}}
\newcommand{\cEToda}{{\cE}_{\text{\textup{Toda}}}}
\newcommand{\cF}{\mathcal{F}}
\newcommand{\cH}{\mathcal{H}}
\newcommand{\cN}{\mathcal{N}}
\newcommand{\cI}{\mathcal{I}}
\newcommand{\cJ}{\mathcal{J}}
\newcommand{\cL}{\mathcal{L}}
\newcommand{\cO}{\mathcal{O}}
\newcommand{\cP}{\mathcal{P}}\newcommand{\cR}{\mathcal{R}}
\newcommand{\cQ}{\mathcal{Q}}
\newcommand{\cU}{\mathcal{U}}
\newcommand{\cV}{\mathcal{V}}
\newcommand{\cW}{\mathcal{W}}
\newcommand{\cX}{{\EuScript X}}    %{\mathcal{X}}
\newcommand{\cY}{{\EuScript Y}}    %{\mathcal{Y}}
\newcommand{\cZ}{{\EuScript Z}}    %{\mathcal{Y}}
\newcommand{\boldb}{{\boldsymbol{b}}}
\newcommand{\Bone}{{\boldsymbol{1}}}
\newcommand{\bc}{{\mathbf{c}}}
\newcommand{\ba}{{\boldsymbol{a}}}
\newcommand{\bb}{{\boldsymbol{b}}}
\newcommand{\bbD}{{\boldsymbol{\mathrm{D}}}}
\newcommand{\bbf}{{\boldsymbol{f}}}
%\newcommand{\bi}{{\boldsymbol{i}}}
\newcommand{\bn}{{\boldsymbol{n}}}
\newcommand{\bp}{{\boldsymbol{p}}}
\newcommand{\bq}{{\boldsymbol{q}}}
%\newcommand{\br}{{\boldsymbol{r}}}
%\newcommand{\bs}{{\boldsymbol{s}}}
\newcommand{\bu}{{\boldsymbol{u}}}
\newcommand{\bv}{{\boldsymbol{v}}}
\newcommand{\bw}{{\boldsymbol{w}}}
\newcommand{\bx}{{\boldsymbol{x}}}
\newcommand{\bby}{{\boldsymbol{y}}}
\newcommand{\bz}{{\boldsymbol{z}}}
\newcommand{\bA}{{\boldsymbol{A}}}
\newcommand{\bD}{{\boldsymbol{D}}}
\newcommand{\bF}{{\boldsymbol{F}}}
\newcommand{\bH}{{\boldsymbol{H}}}
\newcommand{\bL}{{\boldsymbol{L}}}
\newcommand{\bN}{{\boldsymbol{N}}}
\newcommand{\bP}{{\boldsymbol{P}}}
\newcommand{\bQ}{{\boldsymbol{Q}}}
\newcommand{\bbU}{{\boldsymbol{U}}}
\newcommand{\bV}{{\boldsymbol{V}}}
\newcommand{\bE}{\mathbf{E}}
\newcommand{\bR}{\mathbf{r}}
\newcommand{\bS}{{\boldsymbol{S}}}
\newcommand{\bal}{{\boldsymbol{\alpha}}}
\newcommand{\bpi}{{\boldsymbol{\pi}}}
\newcommand{\bpsi}{{\boldsymbol{\psi}}}
\newcommand{\bxi}{{\boldsymbol{\xi}}}
\newcommand{\bet}{{\boldsymbol{\eta}}}
\newcommand{\bom}{{\boldsymbol{\omega}}}
\newcommand{\bOm}{{\boldsymbol{\Omega}}}
\newcommand{\bPhi}{{\boldsymbol{\Phi}}}
\newcommand{\bU}{\mathbf{U}}
\newcommand{\binfty}{\pmb{\infty}}
\newcommand{\BOne}{{\boldsymbol{1}}}
\newcommand{\BTwo}{{\boldsymbol{2}}}
\newcommand{\bsquare}{\pmb{\square}}
\newcommand{\bun}{\mathbf{1}}
\newcommand{\ga}{\mathfrak{a}}
\newcommand{\gothe}{\mathfrak{e}}
\newcommand{\gf}{\mathfrak{f}}
\newcommand{\hgf}{\smash{\widehat{\mathfrak{f}}}}
\newcommand{\gh}{\mathfrak{h}}
\newcommand{\hgh}{\smash{\widehat{\mathfrak{h}}}}
\newcommand{\gm}{\mathfrak{m}}
\newcommand{\gothg}{\mathfrak{g}}
\newcommand{\gotht}{\mathfrak{t}}
\newcommand{\gu}{\mathfrak{u}}
\newcommand{\gA}{\mathfrak{A}}
\newcommand{\gB}{\mathfrak{B}}
\newcommand{\gN}{\mathfrak{N}}
\newcommand{\gM}{\mathfrak{M}}
\newcommand{\veps}{\varepsilon}
\newcommand{\vph}{\varphi}
\newcommand{\dd}{\partial}
\newcommand{\Id}{{\mathrm d}}
\newcommand{\ID}{{\mathrm D}}
\newcommand{\IL}{{\mathrm L}}
%\newcommand{\rmi}{{\mathrm i}}
\newcommand{\rP}{{\mathrm P}}
\newcommand{\fnh}{{\text{\textup{FN}}}}
%\newcommand{\rme}{{\mathrm{e}}}
\newcommand{\rmN}{{\mathrm{N}}}
\newcommand{\uu}{{\underline{u}}}
\newcommand{\uv}{{\underline{v}}}
\newcommand{\uw}{{\underline{w}}}
\newcommand{\sft}{{\mathsf{t}}}
\newcommand{\vx}{{\vec{\mathrm{x}}}}
\newcommand{\vy}{{\vec{\mathrm{y}}}}
\newcommand{\vz}{{\vec{\mathrm{z}}}}
\newcommand{\bvx}{{\vec{\mathbf{x}}}}
\newcommand{\vt}{{\vec{\mathrm{t}}}}
\newcommand{\vgdd}{{\vec{\mathfrak{d}}}}

\newcommand{\bbx}{{\boldsymbol{x}}}
\newcommand{\bbu}{{\boldsymbol{u}}}
\newcommand{\bbv}{{\boldsymbol{v}}}
\newcommand{\bbP}{{\boldsymbol{P}}}

\newcommand{\diftat}[2]{ \left. \frac{\Id}{\Id #1} \right|_{#1=#2} }    
\newcommand{\BV}{{\text{\textup{BV}}}}

\DeclareMathOperator{\Span}{span}
\DeclareMathOperator{\Sol}{Sol}
\DeclareMathOperator{\img}{im}
\DeclareMathOperator{\dom}{dom}
\DeclareMathOperator{\id}{id}
\DeclareMathOperator{\rank}{rank}
\DeclareMathOperator{\sym}{sym}
\DeclareMathOperator{\cosym}{cosym}
%\DeclareMathOperator{\pt}{pt}
\DeclareMathOperator{\arcsinh}{arcsinh}
\DeclareMathOperator{\coker}{coker}
\DeclareMathOperator{\Hom}{Hom}
\DeclareMathOperator{\End}{End}
\DeclareMathOperator{\Der}{Der}
\DeclareMathOperator{\Mat}{Mat}
\DeclareMathOperator{\poly}{poly}
\DeclareMathOperator{\CDiff}{\mathcal{C}Diff}
\DeclareMathOperator{\Diff}{Diff}
\DeclareMathOperator{\ord}{ord}
\DeclareMathOperator{\volume}{vol}
\DeclareMathOperator{\dvol}{d%\,
	vol}
\DeclareMathOperator{\diag}{diag}
%\DeclareMathOperator{\ad}{ad}
\DeclareMathOperator{\Ber}{Ber}
%\DeclareMathOperator{\tr}{tr}
\newcommand{\Free}{\text{\textsf{Free}}\,}
\newcommand{\Sl}{\mathfrak{sl}}
\newcommand{\Gl}{\mathfrak{gl}}
\DeclareMathOperator{\const}{const}

\DeclareMathOperator{\Alt}{Alt}

\DeclareMathOperator{\Jac}{Jac}
%\DeclareMathOperator{\Jac}{\rm \textsf{Jac}}

\DeclareMathOperator{\GH}{gh}
\DeclareMathOperator*{\bigotimesk}{{\bigotimes\nolimits_{\Bbbk}}}
\DeclareMathOperator{\supp}{supp}

\newcommand{\td}{\widetilde{d}}
\newcommand{\tu}{\widetilde{u}}
\newcommand{\tv}{\widetilde{v}}
\newcommand{\tV}{\widetilde{V}}
\newcommand{\hxi}{\widehat{\xi}}
%\newcommand{\lshad}{[\![}
%\newcommand{\rshad}{]\!]}
\newcommand{\ov}{\overline}
\newcommand{\nC}{{\text{\textup{nC}}}}
\newcommand{\KdV}{{\text{KdV}}}
\newcommand{\ncKdV}{{\text{ncKdV}}}
\newcommand{\mKdV}{{\text{mKdV}}}
\newcommand{\pmKdV}{{\text{pmKdV}}}
\newcommand{\EL}{{\text{EL}}}
\newcommand{\Liou}{{\text{Liou}}}
\newcommand{\scal}{{\text{scal}}} 

\newcommand{\ib}[3]{ \{\!\{ {#1},{#2} \}\!\}_{{#3}} }
\newcommand{\schouten}[1]{\lshad {#1} \rshad}

\newcommand{\by}[1]{\textit{{#1}}}
\newcommand{\jour}[1]{\textit{{#1}}}
\newcommand{\vol}[1]{\textbf{{#1}}}
\newcommand{\book}[1]{\textrm{{#1}}}

\newcommand{\ground}[1]{\text{\textit{\small #1}}}

\newcommand*{\vcenteredhbox}[1]{\begingroup
\setbox0=\hbox{#1}\parbox{\wd0}{\box0}\endgroup}

\DeclareSymbolFont{extraup}{U}{zavm}{m}{n}
\DeclareMathSymbol{\varheartsuit}{\mathalpha}{extraup}{86}
\DeclareMathSymbol{\vardiamondsuit}{\mathalpha}{extraup}{87}

\def\oldvec{\mathaccent "017E\relax } %due to bug in amsmath, need old definition of \vec
\DeclareMathOperator{\Ori}{\mathsf{O\oldvec{r}}}

\hyphenation{Kon-tse-vich}

\begin{document}\pagestyle{plain}

\title{Infinitesimal deformations of Poisson bi\/-\/vectors\\ %[3pt]
using the Kontsevich graph calculus}

\author{Ricardo Buring${}^1$, Arthemy V Kiselev${}^2$ and Nina Rutten${}^2$}

\address{${}^1$ %Algebraic Geometry, Topology \& Number Theory,
Institut f\"ur Mathematik, %FB 08 -- Physik, Mathematik und Informatik,
Johannes Gutenberg\/--\/Uni\-ver\-si\-t\"at,
Staudingerweg~9, %4.OG,
\mbox{D-\/55128} Mainz, Germany}
\address{${}^2$ Johann Ber\-nou\-lli Institute for Mathematics and Computer Science, University of Groningen,
P.O.~Box 407, 9700~AK Groningen, The Netherlands}

\ead{rburing@uni-mainz.de, A.V.Kiselev@rug.nl}

%\thanks{${}^\dag$\textit{Address}: . 
%\quad${}^{\ddag}$\textit{E-mail}: \texttt{A.V.Kiselev\symbol{"40}rug.nl}}%,\quad
%%\textit{Tel.} +31(0)\,50\,363-2451,\quad \textit{fax} +31(0)\,50\,363-3800}
%\\
%\mbox{ }\quad${}^{*}$ Institut des Hautes $\smash{\text{\'Etudes}}$ Scientifiques, %($\smash{\text{IH\'ES}}$), 
%35~route de Chart\-res, Bures\/-\/sur\/-\/Yvette, \mbox{F-91440} France.%
%\mbox{ }\quad${}^{\S}$ \textit{Present address}: Max Planck Institute %for Mathematics,
%Vi\-vats\-gas\-se~7, \mbox{D-53111} Bonn, Germany}

%\dedicatory{Talk given by AVK on 8 June 2017 at the ISQS'25 in CVUT Prague.}

%\date{29 September 2017}

%\subjclass[2010]{
%13D10, %	Deformations and infinitesimal methods %[See also 14B10, 14B12, 14D15, 32Gxx]
   %14D15  	Formal methods; deformations %[See also 13D10, 14B07, 32Gxx]
%32G81, %	Deformations of analytic structures, Applications to physics
%53D17, %Symplectic geometry, contact geometry, %[See also 37Jxx, 70Gxx, 70Hxx]
   %Poisson manifolds; Poisson groupoids and algebroids
%81S10, %Geometry and quantization, symplectic methods
   %37J15  	Symmetries, invariants, invariant manifolds, momentum maps, reduction [See also 53D20]
%also
   %47Fxx		Partial differential operators %[See also 35Pxx, 58Jxx]
   %or 47F99  	None of the above, but in this section
%53D55, %	Deformation quantization, star products
  %Not sure:
  %47B36 %Jacobi (tridiagonal) operators (matrices) and generalizations
   %58Jxx		Partial differential equations on manifolds; differential operators    %[See also 32Wxx, 35-XX, 53Cxx]
%58J10, %	Differential complexes [See also 35Nxx]; elliptic complexes
%90C35. %	Programming involving graphs or networks %[See also 90C27]
   %90C27  	Combinatorial optimization
%%%%%%
   %37K05 %Hamiltonian structures, symmetries, variational principles, conservation laws
   %Copied from below:
   %58Exx Variational problems in infinite-dimensional spaces
   %58E30 Variational principles
   %53D17, Poisson manifolds; Poisson groupoids and algebroids
%%%%
   %16E40 (Co)homology of rings and algebras (e.g. Hochschild, cyclic, dihedral, etc.)
%}

%\subjclass[2010]{
%53D55, %Deformation quantization, star products
%%
%%58-XX Global analysis, analysis on manifolds
%%58Exx Variational problems in infinite-dimensional spaces
%58E30, %Variational principles
%%81Sxx General quantum mechanics and problems of quantization
%81S10; %Geometry and quantization, symplectic methods
%%16-XX Associative rings and algebras
%%16E40 (Co)homology of rings and algebras (e.g. Hochschild, cyclic, %dihedral, etc.)
%secondary
%53D17, %Poisson manifolds; Poisson groupoids and algebroids
%%58-XX Global analysis, analysis on manifolds
%58Z05, %Applications to physics
%%70S05, %Lagrangian formalism and Hamiltonian formalism
%70S20.% More general nonquantum field theories
%%81R60, %Noncommutative geometry
%}

%\keywords{Kontsevich graph, Poisson geometry, symmetry, %deformation, 
%multi\/-\/vector, affine ma\-ni\-fold, bracket.}
   %{Poisson manifold, Kontsevich graph complex, Poisson bi\/-\/vector, exotic flow.}
   %Deformation quantisation, star\/-\/product, gauge fields, variational Poisson bracket, associativity
%%%%%%%%%%%%%%%%%%%%%%%%%%%%%%%%%%%%%%%%%%%%%%%%%%%%%%%%%%%%%%%%%%%%%%%

%\begin{document}
\begin{abstract}
Let $\cP$~be a Poisson structure on a %given 
finite\/-\/dimensional affine real manifold. Can $\cP$~be deformed in such a way that it stays Poisson\,? The language of Kontsevich graphs provides a universal approach --\,with respect to all affine Poisson manifolds\,-- to finding a class of solutions to this deformation problem. For that reasoning, several types of %Kontsevich 
graphs are needed. In this paper we outline the algorithms to generate %all 
those graphs.
The graphs that encode deformations are classified by the number of internal vertices~$k$; 
for~$k \leqslant 4$ we present all solutions of the deformation problem.
For $k\geqslant5$, first reproducing the pentagon\/-\/wheel picture %cocycle 
suggested at~$k=6$ by Kontsevich and Willwacher, we %now 
construct the heptagon\/-\/wheel cocycle that yields a new unique solution without $2$-\/loops and tadpoles at~$k=8$.
\end{abstract}
%We also construct %consider / discuss / outline
%a unique loopless solution for~$k=6$.
%predict the 7-wheel solution.

%\maketitle
\pagestyle{plain}

%\subsection*
\paragraph{\textbf{\textup{Introduction.}}}
This paper contains a set of algorithms to generate the Kontsevich graphs 
that encode polydifferential operators --\,in particular multi\/-\/vectors\,-- on Poisson manifolds.
We report a result of implementing such algorithms in the problem of 
fin\-ding symmetries %deformation 
of Poisson structures.
Namely, continuing the line of reasoning from %~
\cite{tetra16,f16}, we find all the solutions of this deformation problem that are expressed by the Kontsevich graphs with at most four internal vertices. Next, we present %/ ... /...
one six\/-\/vertex solution (based on the previous work by Kontsevich~\cite{Kontsevich2017private} 
and %by 
Willwacher~\cite{Willwacher2017private}).
Finally, we find a heptagon\/-\/wheel eight\/-\/vertex graph %cocycle %on 8~%internal 
%vertices 
which, after the orientation of its edges, gives %yields 
a new universal Kontsevich flow.
% Built of 'wedges'
% Do it yourself: algorithms can be implemented "as is" for higher k.
% Were used by [...] in [...] to do [...].
We refer to~\cite{Ascona96, Kontsevich2017Bourbaki} for motivations, to~\cite{f16,cpp} for an exposition
of basic theory, and to~\cite{WeFactorize5Wheel} and~\cite{JNMP2017} for more details about the pentagon\/-\/wheel ($5+1$)-\/vertex and heptagon\/-\/wheel ($7+1$)-\/vertex
solutions respectively.
Let us remark that all the algorithms outlined here can be used without %any
modification in the course of constructing all~$k$-\/vertex Kontsevich graph solutions 
with higher~$k \geqslant 5$ in the deformation problem under study.

%\enlargethispage{0.8\baselineskip}
%\subsection*
\paragraph{\textbf{\textup{Basic concept.}}}%Preliminaries: 
We work with real vector spaces %freely 
generated by finite graphs of the following two types: 
(1)~$k$-\/vertex non-oriented graphs, without multiple edges nor tadpoles, endowed with a wedge ordering of edges, e.g., $E=e_1\wedge\dots\wedge e_{2k-2}$; 
   % wedge ordering of edges???
(2)~oriented graphs on $k$ internal vertices and $n$ sinks such that every internal vertex is a tail of two edges with a given ordering Left $\prec$ Right. Every connected component of a non-oriented graph $\gamma$ is fully encoded by an ordering~$E$ on the set of adjacency relations for its vertices.\footnote{\label{FootZeroGraph}%
The edges are antipermutable so that a graph which equals minus itself 
--\,under a symmetry that induces a parity\/-\/odd permutation of edges\,--
is proclaimed to be equal to zero.
In particular (view $\bullet\!\!\!-\!\!\!\bullet\!\!\!-\!\!\!\bullet$), every graph possessing a symmetry which swaps an odd number of edge pairs is a zero graph. For example, the $4$-\/wheel $\mathsf{12}\wedge \mathsf{13}\wedge \mathsf{14}\wedge \mathsf{15}\wedge \mathsf{23}\wedge \mathsf{25}\wedge \mathsf{34}\wedge \mathsf{45} = I\wedge\cdots\wedge VIII$ or the $2\ell$-\/wheel at any $\ell>1$ is such; here, the %mirror 
reflection symmetry is $I\rightleftarrows III$,\ $V\rightleftarrows VII$, 
and~$VI \rightleftarrows VIII$.}
%%%
Every such oriented graph is given by the list of ordered pairs of directed edges. An edge swap $e_i\wedge e_j = -e_j\wedge e_i$ and the reversal 
Left $\leftrightarrows$ Right of those edges' order in the tail vertex implies the change of sign in front of the graph at hand.\footnote{An oriented graph equals minus itself, hence it is a zero graph if there is a permutation of labels for its internal vertices such that the adjacency tables for the two vertex labellings coincide but the two realisations of the same graph differ by the ordering of outgoing edges at an odd number of internal vertices (see Example~\ref{ExZeroGraph} below).} %the sinks not permuted

\begin{example}\label{Ex5Wheel}
The sum $\boldsymbol{\gamma}_5$ of two 6-vertex 10-edge graphs,
\begin{align*}
\boldsymbol{\gamma}_5 = \mathsf{12}^{(I)}\wedge \mathsf{23}^{(II)}\wedge \mathsf{34}^{(III)}\wedge \mathsf{45}^{(IV)} \wedge \mathsf{51}^{(V)} \wedge \mathsf{16}^{(VI)} \wedge \mathsf{26}^{(VII)}\wedge \mathsf{36}^{(VIII)}\wedge \mathsf{46}^{(IX)}\wedge \mathsf{56}^{(X)}&\\
 + \tfrac{5}{2}\cdot \mathsf{12}^{(I)}\wedge \mathsf{23}^{(II)}\wedge \mathsf{34}^{(III)} \wedge \mathsf{41}^{(IV)}\wedge \mathsf{45}^{(V)}\wedge \mathsf{15}^{(VI)}\wedge \mathsf{56}^{(VII)}\wedge \mathsf{36}^{(VIII)}\wedge \mathsf{26}^{(IX)}\wedge \mathsf{13}^{(X)}&,
\end{align*}
%\begin{align*}
%\gamma_5 =\quad& \mathsf{12}^{I}\wedge \mathsf{23}^{II}\wedge \mathsf{34}^{III}\wedge \mathsf{45}^{IV} \wedge \mathsf{51}^{V} \wedge \mathsf{16}^{VI} \wedge \mathsf{26}^{VII}\wedge \mathsf{36}^{VIII}\wedge \mathsf{46}^{IX}\wedge \mathsf{56}^{X}\\
 %&+ \frac{5}{2}\cdot \mathsf{12}^{I}\wedge \mathsf{23}^{II}\wedge \mathsf{34}^{III} \wedge \mathsf{41}^{IV}\wedge \mathsf{45}^{V}\wedge \mathsf{15}^{VI}\wedge \mathsf{56}^{VII}\wedge \mathsf{36}^{VIII}\wedge \mathsf{26}^{IX}\wedge \mathsf{13}^{X},
%\end{align*}
%RB: <Graph complex: differential(SAGE) p.1-2;15.07.17>
is drawn in Theorem~\ref{ThMainMany} on p.~\pageref{Where5WheelIs} %{ThMainMany} 
below.
\end{example}

\begin{example}\label{Ex3Wheel}
The sum $\cQ_{1:\frac{6}{2}}$ of three oriented 8-edge graphs on $k=4$ internal vertices and $n=2$ sinks (enumerated using 0 and 1, see the notation on p.~\pageref{DefEncodingKgraph}),
$$\cQ_{1:\frac{6}{2}} = \mathsf{2\ 4\ 1\ \ \ 0\ 1\ \ 2\ 4\ \ 2\ 5\ \ 2\ 3} - 3(\mathsf{2\ 4\ 1\ \ \ 0\ 3\ \ 1\ 4\ \ 2\ 5\ \ 2\ 3} + \mathsf{2\ 4\ 1\ \ \ 0\ 3\ \ 4\ 5\ \ 1\ 2\ \ 2\ 4})$$
is obtained from the non-oriented tetrahedron graph $\boldsymbol{\gamma}_3 = %\pm 
\mathsf{12}^{(I)}\wedge \mathsf{13}^{(II)}\wedge \mathsf{14}^{(III)}\wedge \mathsf{23}^{(IV)}\wedge \mathsf{24}^{(V)}\wedge \mathsf{34}^{(VI)}$ on four vertices and six edges by taking all the admissible edge orientations 
(see Theorem~\ref{ThOrient} and Remark~\ref{RemSigns}%below
). 
%unclear & imprecise; ?L\precR inherited from \wedge in \gamma_3?
\end{example}

%\subsection*
\paragraph{\textbf{\textup{I.1.}}}
Let $\gamma_1$ and $\gamma_2$ be connected non-oriented graphs. The definition of insertion $\gamma_1\circ_i\gamma_2$ %\mathbin{\circ_i}
of the entire graph $\gamma_1$ into vertices of $\gamma_2$ and the construction of Lie bracket $[\cdot,\cdot]$ of graphs and differential~$\Id$ in the non\/-\/oriented graph complex, referring to a sign convention, are as follows
(cf.~\cite{Ascona96} and~\cite{DolgushevRogersWillwacher,KhoroshkinWillwacherZivkovic,WillwacherGRT}); these definitions apply to sums of graphs by linearity.

\begin{define}\label{DefInsert}
The insertion $\gamma_1\circ_i\gamma_2$ of a $k_1$-vertex graph $\gamma_1$ with ordered set of edges $E(\gamma_1)$ into a graph $\gamma_2$ with $\# E(\gamma_2)$ edges on $k_2$ vertices is a %the 
sum of %$k_2$ 
graphs on $k_1+k_2-1$ vertices and $\# E(\gamma_1)+\# E(\gamma_2)$ edges. 
Topologically, the sum $\gamma_1\circ_i\gamma_2 = \sum(\gamma_1\rightarrow v \text{ in }\gamma_2)$ %\in V(\gamma_2)
 consists of all the graphs in which a vertex $v$ from $\gamma_2$ is replaced by the entire graph $\gamma_1$ and the edges tou\-ch\-ing $v$ in $\gamma_2$ are re-attached to the vertices of $\gamma_1$ in all possible ways.\footnote{Let the enumeration of vertices in every such term in the sum start running over the enumerated vertices in $\gamma_2$ until $v$ is reached.
Now the enumeration counts the vertices in the graph $\gamma_1$ and then it resumes with the remaining vertices (if any) that go after~$v$ %proceed %antecede
in~$\gamma_2$.}
By convention, in every new term the edge ordering is~$E(\gamma_1)\wedge E(\gamma_2)$. \end{define}
%Ricardo, we can have it vice versa as in SAGE: extra edge last, if you wish.
%See\footnote below% = page -4up- %

To simplify sums of graphs, first eliminate the zero graphs.
Now suppose that in a sum, two non\/-\/oriented graphs, say $\alpha$ and $\beta$, are isomorphic (topologically, i.e.\ regardless of the respective vertex labellings and edge orderings $E(\alpha)$ and~$E(\beta)$).
By using that isomorphism, which establishes a 1--1~correspondence between the edges, extract the sign from the equation $E(\alpha) = \pm E(\beta)$.
If~``+'', then $\alpha = \beta$; else $\alpha = -\beta$.
Collecting similar terms is now elementary.

% "Our" definition is ugly for \cdot\circ_i\(\cdot\cdot) without edges.
% Enumeration of vertices does play its role to encode a graph but it is irrelevant for determining the sign \pm of any term!
% Cancellations are possible only between the graphs which were produced from different pairs of arguments (\gamma_1,\gamma_2); in one sum \gamma_1\circ_i\gamma_2 all topologically equal terms are produced with a common sign (yet these newly produced terms can each intrinsically cancel out whenever this is a zero graph).
\begin{lemma}
The bi\/-\/linear graded skew\/-\/symmetric operation,
\[%\begin{equation}
[\gamma_1,\gamma_2] = \gamma_1\circ_i\gamma_2 - (-)^{\# E(\gamma_1)\cdot\# E(\gamma_2)} \gamma_2\circ_i\gamma_1,
\]%\end{equation}
is a Lie bracket on the vector space $\mathfrak{G}$ of non-oriented graphs.\footnote{The postulated precedence or antecedence of the wedge product of edges from $\gamma_1$ with respect to the edges from $\gamma_2$ in every graph within $\gamma_1\circ_i\gamma_2$ produce the operations $\circ_i$ which coincide with or, respectively, differ from Definition~\ref{DefInsert} by the sign factor $(-)^{\# E(\gamma_1)\cdot\# E(\gamma_2)}$. The same applies to the Lie bracket of graphs $[\gamma_1,\gamma_2]$ if the operation $\gamma_1\circ_i\gamma_2$ is %denotes
the insertion of $\gamma_2$ into $\gamma_1$ (as in~\cite{KhoroshkinWillwacherZivkovic}). Anyway, the notion of $\Id$-cocycles which we presently recall is well defined and insensitive to such sign~ambiguity.}
\end{lemma}

\begin{lemma}
The operator $\Id(\text{graph}) = [\bullet\!\!\!-\!\!\!\bullet, \text{graph}]$ is a differential: $\Id^2 = 0$.
\end{lemma}

In effect, the mapping $\Id$ blows up every vertex $v$ in its argument in such a way that whenever the number of adjacent vertices $N(v) \geqslant 2$ is sufficient, each end of the inserted edge $\bullet\!\!\!-\!\!\!\bullet$ is connected with the rest of the graph by at least one edge.

%\begin{theor}[\cite{Ascona96}] % = 
Summarising,
the real vector space~$\mathfrak{G}$ of non\/-\/oriented graphs is a differential graded Lie algebra \textup{(}dgLa\textup{)} with % respect to the
Lie bracket $[\cdot,\cdot]$ and differential~$\Id = [\bullet\!\!\!-\!\!\!\bullet, \cdot]$.
%\end{theor}
%%%
The graphs $\boldsymbol{\gamma}_5$ and $\boldsymbol{\gamma}_3$ from Examples~\ref{Ex5Wheel} and~\ref{Ex3Wheel} are $\Id$-\/cocycles.
% RB: either Examples in section ... or number examples by section.
Neither is exact, hence marking a nontrivial cohomology class in the non-oriented graph complex.

\begin{theor}[{{\cite[Th.\:5.5]{DolgushevRogersWillwacher}}}]\label{ThWheelCocycles}
At every $\ell \in \mathbb{N}$ in the connected graph complex there is a $\Id$-cocycle on $2\ell + 1$ vertices and $4\ell+2$ edges.
Such cocycle contains the $(2\ell+1)$-wheel in which, by definition, the axis vertex is connected with every other vertex by a spoke so that each of those $2\ell$ vertices is adjacent to the axis and two neighbours\textup{;} the cocycle marked by the $(2\ell+1)$-wheel graph can contain other $(2\ell+1, 4\ell+2)$-\/graphs \textup{(}see Example~\textup{\ref{Ex5Wheel}} and~\textup{\cite{JNMP2017})}.
% RB: be more precise: the wheel has a nontrivial coefficient in the cocycle
\end{theor}

%\subsection*
\paragraph{\textbf{\textup{I.2.}}} %B.2
The \emph{oriented} graphs under study are built over $n$ sinks from $k$ wedges $\xleftarrow{i_\alpha} \bullet \xrightarrow{j_\alpha}$ (here $\xleftarrow{i_\alpha} \prec \xrightarrow{j_\alpha}$) so that every edge is decorated with its own summation index which runs from $1$ to the dimension of a given affine Poisson manifold $(\cN, \cP)$.
Each edge $\xrightarrow{i}$ encodes the derivation $\partial/\partial x^i$ of the arrowhead object with respect to a local coordinate $x^i$ on $\cN$.
By placing an $\alpha$th copy $P^{i_\alpha j_\alpha}(\bx)$ of the Poisson bi-vector $\cP$ in the wedge top $(1 \leqslant \alpha \leqslant k$), by taking the product of contents of the $n+k$ vertices (and evaluating all objects at a point $\bx \in \cN$), and summing over all indices, we realise a polydifferential operator in $n$ arguments; the operator coefficients are differential-polynomial in~$\cP$.
Totally skew-symmetric operators of differential order one in each argument are well-defined $n$-vectors on the affine manifolds $\cN$.

The space of multi\/-\/vectors $G$ encoded by oriented graphs is equipped with a graded Lie algebra structure, namely the Schouten bracket $\schouten{\cdot,\cdot}$.
Its realisation in terms of oriented graphs is shown in \cite[Remark 4]{f16}.
Recall that by definition the bi-vectors $\cP$ at hand are Poisson by satisfying the Jacobi identity $\schouten{\cP, \cP} = 0$.
The Poisson differential $\partial_\cP = \schouten{\cP, \cdot}$ now endows the space of multi-vectors on $\cN$ with the differential graded Lie algebra (dgLa) structure.
The cohomology groups produced by the two dgLa structures introduced %mentioned 
so far are correlated %[in the following sense] 
by the edge orientation mapping~$\Ori$.

\begin{theor}[{{\cite{Ascona96} and \cite[App.\:K]{WillwacherGRT}}}]%
\label{ThOrient}
Let $\gamma \in \ker \Id$ be a cocycle on $k$~vertices and $2k-2$~edges in the non-oriented graph complex. Denote by~$\{\Gamma\} \subset G$ the subspace spanned by all those bi\/-\/vector graphs~$\Gamma$ which are obtained from \textup{(}each connected component in\textup{)} $\gamma$ by adding to it two edges to the new sink vertices and then by taking the sum of graphs with all the admissible orientations of the old $2k-2$~edges
%\footnote{We denote by $\Ori$ this orientation morphism.} 
% RB: footnote is too late.
\textup{(}so that a set of Kontsevich graphs built of %from 
$k$~wedges is produced\textup{)}.
Then in that subspace $\{\Gamma\}$ there is a sum of graphs that encodes a nonzero Poisson cocycle $Q(\cP) \in \ker \partial_\cP$.
\end{theor}

% ? Example: d-trivial => \partial_\cP-trivial ?

Consequently, to find some cocycle $Q(\cP) %\in \ker \partial_\cP
$ in the Poisson complex on any affine Poisson manifold it suffices to find a cocycle in the non-oriented graph complex and then consider %\footnote{\label{FootSigns}   }
the sum of graphs which are produced by the orientation mapping~$\Ori$. 
   %(see Remark~\ref{RemSigns} below).
On the other hand, to list all the $\partial_\cP$-cocycles $Q(\cP)$ encoded by the bi-vector graphs made of %from 
$k$ wedges ${\leftarrow}{ \bullet }{\rightarrow}$, one must generate all the relevant oriented graphs and solve the equation $\partial_\cP(Q) \doteq 0$ via $\schouten{\cP,\cP} = 0$, that is, solve graphically the factorisation problem $\schouten{\cP, Q(\cP)} = \Diamond(\cP, \schouten{\cP,\cP})$ in which the cocycle condition in the left-hand side holds by virtue of the Jacobi identity in the right.
Such construction of some and classification (at a fixed $k > 0$) of all universal infinitesimal symmetries of Poisson %[structures/
brackets are the problems which we explore in this paper.

\begin{rem}\label{RemSigns}
To the best of our knowledge \cite{Kontsevich2017private}, in a bi-vector graph $Q(\cP) = \Ori(\gamma)$, at every internal vertex which is the tail of two oriented edges towards other internal vertices, the edge ordering Left $\prec$ Right %of those edges 
is inherited from a chosen wedge product~$E(\gamma)$ of edges in the non-oriented 
graph~$\gamma$.
How are the new edges towards the sinks ordered, either between themselves at a vertex or with respect to two other oriented edges, coming from~$\gamma$ and issued from different vertices in~$Q(\cP)$\,?
Our findings in~\cite{WeFactorize5Wheel} will help us to verify the order preservation claim and assess answers to this question.
\end{rem}

% Or: /\ => \Prec ? [WeFactorize(5-wheel)] %

% How much are K. graphs NOT a Lie algebra w.r.t. plugging a graph into the other's sinks?
% Generator of _all_ K. graphs on k+n vertices

%-----------

\section{The Kontsevich graph calculus}\label{SecGraphs}
%\subsection{Definitions, conventions, and notation}

\begin{define}%[Kontsevich graph]
\label{DefKgraph}
Let us consider a class of oriented graphs on $n+k$ vertices labelled $0$,\ $\ldots$,\ $n+k-1$ such that the consecutively ordered vertices $0$,\ $\ldots$,\ $n-1$ are sinks, and each of the internal vertices $n$,\ $\ldots$,\ $n+k-1$ is a source for two edges. % RB: maybe aerial instead of internal? is clearer.
For every internal vertex, the two outgoing edges are ordered using $L \prec R$: the preceding edge is labelled $L$ (Left) and the other is $R$ (Right).
% The ordering L < R of outgoing edges reflects the skew-symmetry of Poisson structure contained in that (arrowtail) vertex
An oriented graph on $n$ sinks and $k$ internal vertices % RB: prefer either sinks and sources, or external and internal
is a \emph{Kontsevich graph} of type~$(n,k)$.
\end{define}

For the purpose of defining a graph normal form, we now consider a Kontsevich graph $\Gamma$ together with a {\em sign} $s \in \{0, \pm 1\}$, denoted by concatenation of the symbols: $s\Gamma$.

\begin{notation}[Encoding of the Kontsevich graphs]\label{DefEncodingKgraph}
The format to store a signed graph $s\Gamma$ for a Kontsevich graph $\Gamma$ is the integer number $n>0$, the integer $k \geqslant 0$, the sign $s$, followed by the (possibly empty, when $k=0$) list of $k$~ordered pairs of targets for edges issued from the internal vertices $n$,\ $\ldots$,\ $n+k-1$, respectively.
The full format is then ($n$,\ $k$,\ $s$; list of ordered pairs).
\end{notation}

\begin{define}[Normal form of a Kontsevich graph]
The list of targets in the encoding of a graph $\Gamma$ can be considered as a $2k$-digit integer written in base-$(n+k)$ notation.
By running over the entire group $S_k \times (\mathbb{Z}_2)^k$, and by this over all the different re\/-\/labellings of~$\Gamma$, we obtain many different integers written in base-$(n+k)$.
The {\em absolute value} $|\Gamma|$ of $\Gamma$ is the re\/-\/labelling of $\Gamma$ such that its list of targets is {\em minimal} as a nonnegative base-$(n+k)$ integer.
For a signed graph $s\Gamma$, the {\em normal form} is the signed graph $t|\Gamma|$ which represents the same polydifferential operator as $s\Gamma$.
Here we let $t=0$ if the graph is zero (see Example \ref{ExZeroGraph} below).
\end{define}

\begin{example}[Zero Kontsevich graph]\label{ExZeroGraph}
%%%%%%%%%%%%%%%%%
%\[
%\vcenteredhbox
{\unitlength=1mm
\special{em:linewidth 0.4pt}
\linethickness{0.4pt}
\begin{picture}(0,0)(-60,19)    %(31.00,25.67)(0,3)
\put(15.00,25.00){\vector(1,0){10.00}}
\put(25.00,25.00){\vector(-1,-3){5.00}}
\put(25.00,25.00){\vector(1,-3){5.00}}
\put(25.00,15.00){\vector(1,-1){4.67}}
\put(25.00,15.00){\vector(-1,-1){4.67}}
\put(15.00,25.00){\line(1,-1){6.67}}
\put(23,17){\vector(1,-1){2}}
\bezier{16}(23.00,17.00)(24.00,16.00)(25.00,15.00)
\put(15.00,25.00){\circle*{1}}
\put(25.00,25.00){\circle*{1}}
\put(25.00,15.00){\circle*{1}}
\put(30.33,9.67){\circle*{1}}
\put(19.67,9.67){\circle*{1}}
\put(13.33,25.00){\makebox(0,0)[rc]{4}}
\put(26.67,25.00){\makebox(0,0)[lc]{3}}
\put(20.00,25.67){\makebox(0,0)[cb]{\tiny$R$}}
\put(17.67,21.00){\makebox(0,0)[ct]{\tiny$L$}}
\put(25.00,13.33){\makebox(0,0)[ct]{2}}
\put(20.00,8){\makebox(0,0)[ct]{0}}
\put(30.00,7.67){\makebox(0,0)[ct]{1}}
\end{picture}
}%.
%\]
%%%%%%%%%%%%%%%%%
Consider the graph with the encoding\\[1pt]
\parbox{131.5mm}{{\tt 2 3 1\ \ 0 1 0 1 2 3}.
The swap of vertices $2 \rightleftarrows 3$ is a symmetry of this graph, yet it also swaps the ordered edges $(4\to 2) \prec (4 \to 3)$, producing a minus sign.
Equal to minus itself, this Kontsevich graph is zero.}
\end{example}

\begin{notation}%[Skew\/-\/symmetrisation]
Every Kontsevich graph~$\Gamma$ on $n$~sinks (or every sum~$\Gamma$ of such graphs) yields the sum~$\Alt\Gamma$ of Kontsevich graphs which is totally skew\/-\/symmetric with respect to the $n$~sinks content~$s_1$, $\ldots$, $s_n$. Indeed, let
%For each graph~$\Gamma(s_1,\ldots,s_m)$ on $m\geqslant 2$~sinks that contain the arguments (here $m=2$), take the alternating sum of graphs
\begin{equation}\label{EqMakeSkew}
(\Alt \Gamma)(s_1,\ldots,s_n) = \sum\nolimits_{\sigma\in\mathcal{S}_n} 
(-)^\sigma\, \Gamma(s_{\sigma(1)},\ldots,s_{\sigma(n)}).
\end{equation}
%Note that 
Due to skew\/-\/symmetrisation, the sum %list 
of graphs~$\Alt\Gamma$ can contain zero graphs or~\mbox{repetitions.}
\end{notation}

\begin{example}[The Jacobiator]
The left-hand side of the Jacobi identity is a skew sum of Kontsevich graphs (e.g. it is obtained by skew-symmetrizing the first term)
\begin{equation}\label{EqJacFig}
%\!\!\!\!\!\!\!\!
%\ \ 
\vcenteredhbox{
\raisebox{3.3mm}%{13mm}
[6.5mm][3.5mm]{
\unitlength=1mm
\special{em:linewidth 0.4pt}
\linethickness{0.4pt}
\begin{picture}(12,15)
\put(0,-10){
\begin{picture}(12.00,15.00)
\put(0.00,10.00){\framebox(12.00,5.00)[cc]{$\bullet\ \bullet$}}
\put(2.00,10.00){\vector(-1,-3){1.33}}
\put(6.00,10.00){\vector(0,-1){4.00}}
\put(10.00,10.00){\vector(1,-3){1.33}}
\put(0.00,4.00){\makebox(0,0)[cb]{\tiny\it1}}
\put(6.00,4.00){\makebox(0,0)[cb]{\tiny\it2}}
\put(11.67,4.00){\makebox(0,0)[cb]{\tiny\it3}}
\end{picture}
}\end{picture}}}\ \ \ 
%%%
\mathrel{{:}{=}}
\text{\raisebox{-12pt}[25pt]{
%\input{FG....PIC}
\unitlength=0.70mm
\linethickness{0.4pt}
\begin{picture}(26.00,16.33)
\put(0.00,5.00){\line(1,0){26.00}}
\put(2.00,5.00){\circle*{1.33}}
\put(13.00,5.00){\circle*{1.33}}
\put(24.00,5.00){\circle*{1.33}}
\put(2.00,1.33){\makebox(0,0)[cc]{\tiny\it1}}
\put(13.00,1.33){\makebox(0,0)[cc]{\tiny\it2}}
\put(24.00,1.33){\makebox(0,0)[cc]{\tiny\it3}}
\put(7.33,11.33){\circle*{1.33}}
\put(7.33,11.33){\vector(1,-1){5.5}}
\put(7.33,11.33){\vector(-1,-1){5.5}}
\put(13,17){\circle*{1.33}}
\put(13,17){\vector(1,-1){11.2}}
\put(13,17){\vector(-1,-1){5.1}}
\put(3.00,10.00){\makebox(0,0)[cc]{\tiny$i$}}
\put(12.00,10.00){\makebox(0,0)[cc]{\tiny$j$}}
\put(24.00,10.00){\makebox(0,0)[cc]{\tiny$k$}}
\end{picture}
}}
{-}
\text{\raisebox{-12pt}[25pt]{
%\input{FG....PIC}
\unitlength=0.70mm
\linethickness{0.4pt}
\begin{picture}(26.00,16.33)
\put(0.00,5.00){\line(1,0){26.00}}
\put(2.00,5.00){\circle*{1.33}}
\put(13.00,5.00){\circle*{1.33}}
\put(24.00,5.00){\circle*{1.33}}
\put(2.00,1.33){\makebox(0,0)[cc]{\tiny\it1}}
\put(13.00,1.33){\makebox(0,0)[cc]{\tiny\it2}}
\put(24.00,1.33){\makebox(0,0)[cc]{\tiny\it3}}
\put(13,11.33){\circle*{1.33}}
\put(13,11.33){\vector(2,-1){10.8}}
\put(13,11.33){\vector(-2,-1){10.8}}
\put(18.5,17){\circle*{1.33}}
\put(18.5,17){\vector(-1,-1){5.2}}
\put(18.5,17){\vector(-1,-2){5.6}}
\put(13,15){\tiny $L$}
\put(17,12){\tiny $R$}
\put(4.00,10.00){\makebox(0,0)[cc]{\tiny$i$}}
\put(11.00,8.00){\makebox(0,0)[cc]{\tiny$j$}}
\put(22.00,10.00){\makebox(0,0)[cc]{\tiny$k$}}
\end{picture}
}}
{-}
\text{\raisebox{-12pt}[25pt]{
%\input{FG....PIC}
\unitlength=0.70mm
\linethickness{0.4pt}
\begin{picture}(26.00,16.33)
\put(0.00,5.00){\line(1,0){26.00}}
\put(2.00,5.00){\circle*{1.33}}
\put(13.00,5.00){\circle*{1.33}}
\put(24.00,5.00){\circle*{1.33}}
\put(2.00,1.33){\makebox(0,0)[cc]{\tiny\it1}}
\put(13.00,1.33){\makebox(0,0)[cc]{\tiny\it2}}
\put(24.00,1.33){\makebox(0,0)[cc]{\tiny\it3}}
\put(18.33,11.33){\circle*{1.33}}
\put(18.33,11.33){\vector(1,-1){5.5}}
\put(18.33,11.33){\vector(-1,-1){5.5}}
\put(13,17){\circle*{1.33}}
\put(13,17){\vector(-1,-1){11.2}}
\put(13,17){\vector(1,-1){5.1}}
\put(3.00,10.00){\makebox(0,0)[cc]{\tiny$i$}}
\put(13.00,10.00){\makebox(0,0)[cc]{\tiny$j$}}
\put(24.00,10.00){\makebox(0,0)[cc]{\tiny$k$}}
\end{picture}
}}
%
.%\nonumber
\end{equation}
The default ordering of edges is the one which we see. % Left goes left and Right goes right.
\end{example}

\begin{define}[Leibniz graph]
A \emph{Leibniz graph} is a graph whose vertices are either sinks, or the sources for two arrows, or the Jacobiator (which is a source for three arrows).
There must be at least one Jacobiator vertex.
The three arrows originating from a Jacobiator vertex must land on three distinct vertices.
Each edge falling on a Jacobiator works by the Leibniz rule on the two internal vertices in it.
\end{define}

\begin{example}
The Jacobiator itself is a Leibniz graph (on one tri-valent internal vertex).
\end{example}

\begin{define}[Normal form of a Leibniz graph with one Jacobiator]
Let $\Gamma$~be a Leibniz graph with one Jacobiator vertex $\Jac$.
From \eqref{EqJacFig} we see that expansion of $\Jac$ into a sum of three Kontsevich graphs means adding one new edge $w \to v$ (namely joining the internal vertices $w$ and $v$ within the Jacobiator).
Now, from $\Gamma$ construct three Kontsevich graphs by expanding $\Jac$ using \eqref{EqJacFig} and letting the edges which fall on $\Jac$ in $\Gamma$ be directed only to $v$ in every new graph.
Next, for each Kontsevich graph find the relabelling $\tau$ which brings it to its normal form and re-express the edge $w \to v$ using $\tau$.
Finally, out of the three normal forms of three graphs pick the minimal one.
By definition, the \emph{normal form} of the Leibniz graph $\Gamma$ is the pair: normal form of Kontsevich graph, that edge $\tau(w) \to \tau(v)$.
\end{define}

We say that a sum of Leibniz graphs is a \emph{skew Leibniz graph}~$\Alt\Gamma$ 
if it is produced from a given Leibniz graph~$\Gamma$
by alternation using formula~\eqref{EqMakeSkew}.

\begin{define}[Normal form of a skew Leibniz graph with one Jacobiator]\label{DefSkewNormalForm}
Likewise, the normal form of a skew Leibniz graph~$\Alt\Gamma$ is the minimum of the normal forms of Leibniz graphs (specifically, of the graph but not edge encodings) which are obtained from~$\Gamma$ by running over the  group of permutations of the sinks content.
\end{define}

\begin{lemma}[\cite{sqs15}]
In order to show that a sum~$S$ of weighted skew\/-\/sym\-met\-ric 
   % Do we want to (re)formulate / define everything such that we can leave out all "skew"'s? 
   % They are left out at some places from here on.
Kontsevich graphs vanishes for all Poisson structures $\cP$, 
it suffices %~\cite{sqs15} %\footnote{Necessary: proof?} 
to express~$S$ as a sum of skew Leibniz graphs: 
$%\begin{equation}\label{EqDiamond}
S = \Diamond\bigl(\cP,\Jac(\cP)\bigr)$.
%\end{equation}
\end{lemma}
  % S, letter is used? % Define diamond(Jac)?
%%%

\subsection{Formulation of the problem}%(s)}
Let $\cP \mapsto \cP + \varepsilon \cQ (\cP) + \bar{o}(\varepsilon)$ be a deformation of bi-vectors that preserves their property to be Poisson at least infinitesimally on all affine manifolds:
$%\[%\begin{equation}
\lshad\cP + \varepsilon\cQ + \bar{o}(\varepsilon), \cP + \varepsilon\cQ + \bar{o}(\varepsilon)\rshad = \bar{o}(\veps)$.
%\]%\end{equation}
Expanding and equating the first order terms, we obtain the equation
$%\[%\begin{equation}
\lshad\cP,\cQ (\cP)\rshad \doteq 0$
% \qquad \text
 {via} %\quad 
$ \lshad\cP ,\cP\rshad = 0$.
%\]%\end{equation}
The language of Kontsevich graphs %~[\textbf{2}] 
allows one to convert this infinite analytic problem within a given set\/-\/up $\bigl(\mathcal{N}^n,\mathcal{P}\bigr)$ in dimension~$n$ into a set of finite combinatorial problems whose solutions are universal for all Poisson geometries in all dimensions~$n<\infty$.

Our first task in this paper is %(i) 
to find the space of flows $\dot{\cP} = \cQ (\cP)$ which are encoded by the Kontsevich graphs on a fixed number of internal vertices~$k$, for $1\leqslant k\leqslant 4$. Specifically, we solve the graph equation
\begin{equation}\label{EqDiamond}
\lshad\cP,\cQ (\cP)\rshad = \Diamond\bigl(\cP,\Jac(\cP)\bigr)
\end{equation}
for the Kontsevich bi-vector graphs~$\cQ(\cP)$ and %the 
Leibniz graphs~$\Diamond$. %$\lozenge$.
We then %(ii) to 
factor out the Poisson\/-\/trivial and improper solutions, that is, we quotient out all bi\/-\/vector graphs that can be written in the form 
$%\[%\begin{equation}
\cQ(\cP) = \lshad\cP ,X\rshad + \nabla\bigl(\cP,\Jac(\cP)\bigr)$,
%\]%\end{equation}
where $X$~is a Kontsevich one\/-\/vector graph and $\nabla$~is a Leibniz bi\/-\/vector graph. (The bi\/-\/vectors $\lshad\cP, X\rshad$ make $\lshad\cP , \cQ(\cP)\rshad$ vanish since $\lshad\cP,\cdot\rshad$~is a differential. % on the space of oriented graphs.
The improper graphs $\nabla (\cP,\Jac (\cP))$ %would 
vanish identically at all Poisson bi-vectors~$\cP$ on every affine manifold.

Before solving %the %To solve this 
factorisation problem~\eqref{EqDiamond} % $S = \Diamond\bigl(\cP,\Jac(\cP)\bigr)$ 
with respect to the operator~$\Diamond$,
we must generate --\,e.g., iteratively as described below\,--
an ansatz for expansion of the right\/-\/hand side %~$S$ 
using skew Leibniz graphs with un\-de\-ter\-mi\-ned coefficients. 
   %, one can proceed as follows.

%\section{Method} %algorithms 

\subsection{How to generate Leibniz graphs iteratively}\label{SecGenLeibnizIteratively}
The first step is to construct a %the first 
layer of skew Leibniz graphs, 
that is, all skew Leibniz graphs which produce at least one graph in the input
(in the course of expansion of skew Leibniz graphs using formula~\eqref{EqMakeSkew} and then
in the course of expansion of every Leibniz graph at hand to a sum of Kontsevich graphs).
For a given Kontsevich graph in the input~$S$, one such Leibniz graph can be constructed 
by contracting an edge between two internal vertices %in a graph in the input 
so that the new vertex with three outgoing edges becomes %can be seen as 
the Jacobiator vertex. 
Note that these Leibniz graphs, which are designed to reproduce~$S$,
may also produce extra Kontsevich graphs that are not given in the input.
Clearly, if the set of Kontsevich graphs in~%it is possible to express 
$S$ coincides with the set of such graphs obtained by expansion of all the Leibniz graphs
in the ansatz at hand, then we are done: the extra graphs, not present in~$S$, are known to all cancel.
   %in terms of these Leibniz graphs we are done 
   %(in such solution  cancel).
Yet it could very well be that it is not possible to express $S$ using only the Leibniz graphs 
from the set accumulated so far. %though.
%%%
Then we %can 
proceed by constructing the next %a second 
layer of skew Leibniz graphs that reproduce at least one of the extra Kontsevich graphs 
(which were not present in~$S$ but which are produced by the graphs in the previously constructed layer(s) 
of Leibniz graphs).
In this way we %can 
proceed iteratively until no new Leibniz graphs are found;
of course, the overall number of skew Leibniz graphs on a fixed number of internal vertices and sinks 
is bounded from above so that the algorithm always terminates.
%%%
Note that the Leibniz graphs %so 
obtained in this way are the only ones that can in principle be involved in the vanishing mechanism for~$S$.

%For each Kontsevich graph in the input, construct all Leibniz graphs that produce it, 
%by contracting each edge $w \to v$ such that the new vertex (with three outgoing edges) 
%can be seen as a Jacobiator.

%$v$ and $w$  (such that $v$ and $w$ have three distinct targets) and declaring the new vertex 
%(with three outgoing edges) to be a Jacobiator, with all incoming edge 

\begin{notation}\label{pNotationNHT}%Define T(v), H(v), still not F(v).
Let $v$~be a graph %n internal 
vertex. Denote by~$N(v)$ the set of neighbours of~$v$, 
by~$H(v)$ the (possibly %initially 
empty) set of arrowheads 
%targets 
of oriented edges issued from the vertex~$v$, and
by~$T(v)$ the (possibly %initially 
empty) set of tails %sources 
for oriented edges pointing at~$v$.
%By definition, put $F(v):=N(v)\setminus (H(v)\cup T(v))$; in other words, $F(v)$ is the subset of neighbours connected with $v$ by an non-oriented edge.
%\footnote{
For example, $\#N(\bullet)=2$, $\#H(\bullet)=2$, and $T(\bullet)=\varnothing$ 
for the top~$\bullet$ of the wedge graph~${\gets}\bullet{\to}$.
%}
\end{notation}

%\paragraph{
\smallskip\noindent%
{\textbf{Algorithm}}
%%%
Consider a skew\/-\/symmetric sum~$S_0$ of oriented Kontsevich graphs
   %\footnote{Reduced graph series [ref?]} 
with real coefficients.
Let $S_{\text{total}} \mathrel{{:}{=}} S_0$ and create an empty table~$L$. % maybe explain
We now describe the $i$th iteration of the algorithm~($i\geqslant 1$).\\[3pt]
%%%
%\subparagraph
\mbox{ }\quad{\textbf{\textsf{Loop~$\circlearrowright$}}}
Run over all Kontsevich graphs~$\Gamma$ in $S_{i-1}$: for each internal vertex~$v$ 
in a graph~$\Gamma$, run over all vertices~$w\in T(v)$ 
in the set of tails of oriented edges pointing at~$v$
such that $v\notin T(w)$ and $H(v)\cap H(w)=\varnothing$
for the sets of targets of oriented edges issued from~$v$ and~$w$.
%%%
Replace the edge $w\to v$ connecting $w$ to~$v$ by %the 
Jacobiator~\eqref{EqJacFig},
that is, by a single vertex~$\Jac$
   %can be seen as a single vertex, say $\Jac$, 
with three outgoing edges
  %, making in a skew Leibniz graph
 %\footnote{A skew Leibniz graph is defined to be the skewsymmetrisation of a Leibniz graph.} 
 %[ref? definition in F16?], 
and such that $T(\Jac)= \bigl(T(v)\setminus w\bigr)\cup T(w)$ 
and $H(\Jac)=H(v)\cup (H(w)\setminus v) \mathrel{{=}{:}} \{a,b,c\}$.

Because we shall always expand the skew Leibniz graphs in what follows, %the next steps [ref?]
we do not actually contract the edge $w\to v$ (to obtain a %the 
Leibniz graph explicitly) in this algorithm
but instead we continue working with the original Kontsevich graphs containing the distinct vertices~$v$ and~$w$.
% announce normal form, maybe

For every edge that points at~$w$, redirect it to~$v$.
Sum over the three cyclic permutations that provide %the 
three possible ways to attach the three outgoing edges for~$v$ and~$w$ (excluding~$w\to v$) 
--\,now %that can be 
seen as the outgoing edges of the Jacobiator\,-- to the target vertices~$a$, $b$, and~$c$
depending on~$w$ and~$v$.
Skew\/-\/symmetrise\footnote{% 
   %\begin{rem}
This algorithm can be modified so that it works for an input which is not skew, namely,
by replacing skew Leibniz graphs by ordinary Leibniz graphs (that is, by skipping the skew\/-\/symmetrisation).
For example, this strategy has been used in~\cite{sqs15,cpp} to show that the Kontsevich star 
product~$\star\mod\bar{o}(\hbar^4)$ is associative: although
the associator $(f\star g)\star h - f\star(g\star h) = \Diamond\bigl(\cP,\Jac(\cP)\bigr)$ 
is not skew, it does vanish for every Poisson structure~$\cP$.%
   %\end{rem}
}
each of these three graphs with respect to the sinks
by applying for\-mu\-la~\eqref{EqMakeSkew}.%to every such graph. %each of them.

% Make pairs: attach to each of these (nomalized) graphs the ..., (im(v),im(w)), where im(a) 
% is the image under the respective isomorphism.

%Suppose still that the Leibniz graphs in the sum at hand are assembled into totally skew\/-\/symmetric terms~\eqref{EqMakeSkew}.
%Then 
For every marked edge $w\to v$ %the mark 
indicating the internal %oriented
edge in the Jacobiator vertex in a graph, replace each %the respective 
sum of the Kontsevich graphs which is skew with respect to the sink content by using the normal form of the respective skew Leibniz graph, see Definition~\ref{DefSkewNormalForm}.
%
%in the resulting sum of %3\dot arg!
%graphs with , 
%replace each graph by its normal form (\emph{transporting the mark via the respective isomorphism}).
%Namely, out of the set of these marked graphs pick %select %the 
%the one with the minimal encoding. By definition, %We define 
%this is the \emph{normal form} of this skew Leibniz graph.
%Note that it is unique. %explain?!
%
If this skew Leibniz graph is not contained in~$L$, %then 
apply the Leibniz rule(s) for all the derivations acting on the Jacobiator vertex~$\Jac$. %to it. 
Otherwise speaking, %That is, 
sum over all possible ways to attach the incoming edges of the target~$v$ 
in the marked edge~$w\to v$ to its source~$w$ and target.
  % maybe use pairs description still?
To each Kontsevich graph resulting from a skew Leibniz graph at hand assign 
  %Give each of them 
the same undetermined coefficient, and add all these weighted Kontsevich graphs to the sum~$S_{i}$. 
Further, add a row to the table~$L$, that new row containing the normal form of
this skew Leibniz graph (with its coefficient
that has been made common to the Kontsevich graphs). %coefficient explicit? c_i_counter; unclear?

By now, the new sum of Kontsevich graphs~$S_i$ is fully composed.
Having thus finished %At the end of 
the current iteration over all graphs~$\Gamma$ in the set~$S_{i-1}$,
redefine the algebraic sum of weighted graphs~$S_{\text{total}}$ by subtracting from it the newly formed sum~$S_i$.
Collect similar terms in~$S_{\text{total}}$ and reduce this sum of Kontsevich graphs
modulo their skew\/-\/symmetry under swaps $L\rightleftarrows R$ of the edge ordering in every internal vertex,
so that all zero graphs (see Example~\ref{ExZeroGraph}) also get eliminated.
\hfill $\circlearrowright$~\textbf{\textsf{end loop}}

Increment~$i$ by~$1$ and repeat the iteration until the set of weighted (and skew) Leibniz graphs~$L$ stabilizes.
%%%
Finally, solve --\,with respect to the coefficients of skew Leibniz graphs\,--
the linear algebraic system obtained from the graph equation %problem 
$S_{\text{total}} = 0$ for the sum of Kontsevich graphs which has been produced from its initial value~$S_0$ by running the iterations of the above algorithm.
%by equating to zero the coefficients of graphs in the reduced sum~$S_{\text{total}}$.
  %, and print / output the solution.

\begin{example}
For the skew sum of Kontsevich graphs in the right-hand side of \eqref{EqJacFig}, the algorithm would produce just one skew Leibniz graph: namely, the Jacobiator itself.
\end{example}

\begin{example}[The $3$-\/wheel]
For the Kontsevich tetrahedral flow $\dot\cP=\cQ_{1:6/2}(\cP)$ on the spaces of Poisson bi\/-\/vectors~$\cP$, see~\cite{Ascona96,Kontsevich2017Bourbaki} and~\cite{tetra16,f16}, building a sufficient set of skew Leibniz graphs in the r.\/-\/h.\,s.\ of factorisation problem \eqref{EqDiamond} % $\lshad\cP,\cQ_{1:6/2}\rshad=\Diamond\bigl(\cP,\Jac(\cP)\bigr)$
requires two iterations of the above algorithm: 11
Leibniz graphs are produced at the first step and 50~%difference 2nd - 1st!
more are added by the second step, making~61 in total.
One of the two known solutions of this factorisation problem~\cite{f16} then consists of 8~%totally 
skew %\/-\/symmetric sums of 
Leibniz graphs (expanding to 27~Leibniz graphs). In turn, as soon as all the Leibniz rules acting on the Jacobiators are processed and every Jacobiator vertex is expanded via~\eqref{EqJacFig}, the right\/-\/hand side~$\Diamond\bigl(\cP,\Jac(\cP)\bigr)$ equals the sum of 39~Kontsevich graphs which are assembled into the 9~totally skew\/-\/symmetric terms in the left\/-\/hand side~$\lshad\cP,\cQ_{1:6/2}\rshad$. 
%of the cocycle condition
\end{example}

\begin{example}[The $5$-\/wheel]
Consider the factorisation problem $\lshad\cP, %\cQ_{\text{$5$-\/wheel}}
\Ori(\boldsymbol{\gamma}_5) \rshad = \Diamond\bigl(\cP,\Jac(\cP)\bigr)$ for the pentagon\/-\/wheel %universal
deformation $\dot\cP=\Ori(\boldsymbol{\gamma}_5) %\cQ_{\text{$5$-\/wheel}}
(\cP)$ of Poisson bi\/-\/vectors~$\cP$, see~\cite{Kontsevich2017private,Willwacher2017private} and~\cite{WeFactorize5Wheel}.
The ninety skew Kontsevich graphs encoding the bi\/-\/vector~$\Ori(\boldsymbol{\gamma}_5)
 %\cQ_{\text{$5$-\/wheel}}
$ are obtained by taking all the admissible orientations of two $(5+1)$-\/vertex graphs~$\boldsymbol{\gamma}_5$, one of which is the pentagon wheel with five spokes, the other graph complementing the former to a cocycle in the non\/-\/oriented graph complex.
   %Fishnet graph.
By running the iterations of the above algorithm for self\/-\/expanding construction of the %skew
Leibniz tri\/-\/vector graphs in this factorisation problem, we achieve %a 
stabilisation of the number of such graphs after the seventh iteration, see Table~\ref{Tab5WheelLeibniz} below.%
%%%
\begin{table}[htb]
\caption{The number of skew Leibniz graphs produced %RB: 25.SEP.2017
iteratively for~$\lshad\cP,
 \Ori(\boldsymbol{\gamma}_5 )   %\cQ_{\text{$5$-\/wheel}}
\rshad$.}\label{Tab5WheelLeibniz}
\centerline{\begin{tabular}{l c c c c c c c c }
\br%\hline
No.\ iteration~$i$\strut & 1 & 2 & 3 & 4 & 5 & 6 & 7 & 8 \\
\mr%\hline
% RB: the old numbers are for non-skew Leibniz graphs on three sinks
%$\#$ of graphs\strut & 7218 & 70573 & 192747 & 254060 & 264523 & 265465 & 265495 & 265495 \\
%$\#$ of graphs\strut & 1609 & 14937 & 41122 & 54279 & 56409 & 56594 & 56600 & 56600 \\
$\#$ of graphs\strut & 1518 & 14846 & 41031 & 54188 & 56318 & 56503 & 56509 & 56509 \\
\mr%\hline
%\mbox{ } of them new\strut & all & +63355 & +122174 & +61313 & +10463 & +942 & +30 & none \\
\mbox{ } of them new\strut & all & +13328 & +26185 & +13157 & +2130 & +185 & +6 & none \\
\br%\hline
\end{tabular}
}
\end{table}
\end{example}
%Seven layers of Leibniz graphs for the pentagon wheel: No. of layer, number of new / so far accumulated Leibniz graphs.

%\begin{rem}
%One could %also 
%execute a fixed number of iterations and try to solve a smaller linear system if a solution is expected to be reached after that many iterations.\footnote{Edit: One can also provisionally discard the Leibniz graphs in which a derivation falls back on the Jacobiator from one of its arguments, so that in the course of the Leibniz rule expansion, a 2-cycle is produced; yet the 2-cycles are a priori not expected even in the mutually cancelling terms in the right\/-\/hand side of factorisation.}
%\end{rem}


\section{Generating the Kontsevich multi\/-\/vector graphs}
%{Full classification of deformations $\dot\cP=\cQ(\cP)$ using $k$-\/vertex  graphs: $k\leqslant4$ and loops possible}
%%%
%\subsection{How to generate the Kontsevich one\/-, bi\/-, and tri-\/vectors}
\noindent%
Let us return to problem \eqref{EqDiamond}: it is the ansatz for bi\/-\/vector Kontsevich graphs $\cQ(\cP)$ with $k$ internal vertices, as well as the Kontsevich 1-vectors $\cX$ with $k-1$ internal vertices (to detect trivial terms $\cQ(\cP) = \schouten{\cP, \cX}$) which must be generated at a given $k$.
(At $1 \leqslant k \leqslant 4$, one can still expand with respect to \emph{all} the Leibniz graphs in the r.-h.s. of \eqref{EqDiamond}, not employing the iterative algorithm from \S\ref{SecGenLeibnizIteratively}.
So, a generator of the Kontsevich (and Leibniz) tri-vectors will also be described presently.)

The Kontsevich graphs corresponding to $n$-vectors are those graphs with $n$ sinks
(each containing the respective argument of $n$-\/vector) 
in which %with 
exactly one arrow comes into each sink, 
%corresponding to the arguments and a single incoming arrow at each sink, 
so that the order of the differential operator encoded by an $n$-\/vector graph 
equals one w.r.t.\ each argument, and which are totally skew-symmetric in their
$n$~arguments. 
Let us explain how one can economically obtain the set of one\/-\/vectors and
skew\/-\/symmetric bi\/- and tri\/-\/vectors with $k$~internal
vertices in three steps (including graphs with eyes $\bullet \rightleftarrows \bullet$
but excluding graphs with 
tadpoles). This approach can easily be extended to the construction of %any set of 
$n$-vectors with any~$n\geqslant1$. 
% but it is questionable whether it is usefull to extend this approach, since generating n-1 vectors will take more time when n increases.

\subsection{\textbf{One\/-\/vectors}}
Each one\/-\/vector under study is encoded by a Kontsevich graph with one sink.
Since the sink has one incoming arrow, there is an %a unique 
internal vertex 
as the tail of this incoming arrow. %edge with the sink as target. 
The target of another edge  issued from this internal vertex can be any internal vertex other then itself. 

%\paragraph*
\smallskip
\noindent{\textit{Step~1.}} 
%Using recursively ?? the algorithm which we presently describe,
%\marginpar{We need graphs with all possible numbers of arrows targeting at the sink right? 
%So we can not do this recursively. We need to do that the "inefficient way"}
Generate all Kontsevich graphs on $k-1$ internal vertices and one sink
(i.e.\ graphs including those with eyes yet excluding those with tadpoles, 
and not necessarily of differential order one with respect to the sink content). 
%The sink will become the~$k$th internal vertex adjacent to %pointing at 
%the new sink in the desired graphs.

%\paragraph*
\noindent%
{\textit{Step~2.}} For every such graph with $k-1$~internal vertices,
add the new sink and make it a target of the old sink, which itself becomes the
$k$th internal vertex.
Now run over the $k-1$ internal vertices excluding the old sink and --\,via the Leibniz rule\,-- make every such internal vertex the second target of the old sink.

%%%Note that the obtained graphs are skew-symmetric (Step 3 is not needed).

%\begin{example}
%An upper bound --\,still not the least upper bound in view of a possibility to have graph repetitions\,-- for the number of Kontsevich one\/-\/vector graphs on $k\leqslant8$ internal vertices is given in the table below.%
%\begin{table}[htb]
%\caption{How many one\/-\/vector graphs are there at most at a given number of internal vertices~$k$\,?}
%\begin{center}
%\begin{tabular}{|c|c|c|c|c|c|c|c|c|}
%\hline
%$k$ & 1 & 2 & 3 & 4 & 5 & 6 & 7 & 8 \\
%\hline
%$\#$~graphs &  &  &  &  &  &  &  &  \\
%\hline
%\end{tabular}
%\end{center}
%\end{table}
%\end{example}

%\subsubsection*{Step 3.}
%All graphs obtained after the first two steps must be skew-symmetrized. This means the following. For each graph consider its mirror image, obtained as follows: the two sinks are relabelled: sink 1 becomes sink 2 and visa versa. Now we redefine each graph as the difference of the graph and its mirror image (one minus the other).

%Note that this final list of graphs is complete, but that due to skew- symmetrisation it can contain zero\/-\/graphs or two copies of one and the same graph.

\subsection{\textbf{Bi\/-\/vectors}}
There are two cases in the construction of bi\/-\/vectors encoded by the
Kontsevich graphs. At all~$k\geqslant1$ the first variant is referred to
   %For $k\geq 2$, there exist two variants of bi\/-\/vectors. 
   %The first variant contains 
those graphs with an internal vertex that has both sinks as targets.

%\paragraph*
\smallskip\noindent%
{\textit{Variant 1}: \textit{Step 1.}} Generate all $k$-vertex graphs on
$k-1$~internal vertices and one sink.
%This sink will become the $k$th~internal
%vertex pointing at both the new sinks in the desired graphs.

%\paragraph*
\noindent%
{\textit{Variant 1}: \textit{Step 2.}} For every such graph, 
add two new sinks and proclaim them as targets of the old sink.

Note that the obtained graphs are skew\/-\/symmetric. %(Step~3 is not needed).

\smallskip
The second variant produces those graphs which contain two internal vertices such that 
one has the first sink as target and the other has the second sink as target.
The second target of either such internal vertex can be any internal vertex other then itself. Note that for $k = 1$ only %just 
the first variant applies.

%\paragraph*
\smallskip\noindent%
{\textit{Variant 2}: \textit{Step 1.}} Generate all $k$-vertex Kontsevich graphs on
$k-2$~internal vertices  and two sinks.
These sinks now become
the $(k-1)$th and $k$th~internal vertices.
%such that either is pointing 
%at a different new sink in the desired bi\/-\/vector graphs.

%\paragraph*
\noindent{\textit{Variant 2}: \textit{Step 2.}} For every such graph, add two new sinks, make the first new sink a target of the first old sink and make the second 
new sink a target of the second old sink. Now run over the $k-1$ internal 
vertices excluding the first old sink, each time proclaiming an internal vertex 
the second target of the first old sink. 
Simultaneously, run over the $k-1$ internal vertices excluding the second old
sink and likewise, declare an internal vertex to be the second target of the second 
old sink.

%\paragraph*
\noindent{\textit{Variant 2}: \textit{Step 3.}} %\textup{(}skew\/-\/symmetrisation\textup{)}}
   %All graphs obtained after the first two steps must be skew-symmetrized.
Skew\/-\/symmetrise each graph with respect to the content of two sinks using~\eqref{EqMakeSkew}.

%\begin{example}
%The table below contains upper bounds for the number of bi\/-\/vector graphs on $k$~internal vertices generated, in particular, by using Variant~1 and~2 of the above algorithm.%
%\begin{table}[htb]
%\caption{How many bi\/-\/vector graphs are there at most at a given number of internal vertices~$k$\,?}
%\begin{center}
%\begin{tabular}{|c|c|c|c|c|c|c|c|c|}
%\hline
%$k$ & 1 & 2 & 3 & 4 & 5 & 6 & 7 & 8 \\
%\hline
%Variant~1 &  &  &  &  &  &  &  &  \\
%\hline
%Variant~2 &  &  &  &  &  &  &  &  \\
%\hline
%Total &  &  &  &  &  &  &  &  \\
%\hline
%\end{tabular}
%\end{center}
%\end{table}
%\end{example}

%\smallskip
\subsection{\textbf{Tri\/-\/vectors}}
For $k\geqslant 3$, there exist two variants of tri\/-\/vectors. The first variant
at all~$k\geqslant2$ yields those Kontsevich graphs with two internal vertices such that one has 
two of the three sinks as its targets 
while another internal vertex has the third sink as one of its targets. 
The second target of this last vertex can be any internal vertex other then itself.
The second variant contains those graphs with three internal vertices such that
the first one has the first sink as a target, the second one has the second 
sink as a target, and the third one has the third sink as a target. 
For each of these three internal vertices with a sink as target, 
the second target can be any internal vertex other then itself.

\smallskip\noindent%
%\paragraph*
{\textit{Variant 1}: \textit{Step 1.}} Generate all 
$k$-vertex Kontsevich graphs on $k-2$~internal vertices and two sinks.
%The first sink will become an internal vertex with the two of the three new sinks as targets in the desired graphs. The second sink will become an internal vertex with the third new sink as target. The second target of this last vertex can be any internal vertex other than itself.

%\paragraph*
\noindent{\textit{Variant 1}: \textit{Step 2.}} For every such graph,
add three new sinks, make the first two new sinks the targets of the 
first old sink and make the third new sink a target of the second old sink.
Now run over the $k-1$ internal vertices excluding the second old sink and 
every time declare an internal vertex the second target of the second old sink.

%\paragraph*
\noindent{\textit{Variant 1}: \textit{Step 3.}}
%All graphs obtained after the first two steps must be skew-symmetrized.
Skew\/-\/symmetrise all %the 
graphs at hand by applying formula~\eqref{EqMakeSkew} to each of~them.

%\smallskip
Note that for $k=1$ there are no tri\/-\/vectors encoded by Kontsevich graphs 
and also note that for $k=2$ 
only %just
the first variant applies.

\smallskip\noindent%
%\paragraph*
{\textit{Variant 2}: \textit{Step 1.}} Generate all %$(k-3)$-vectors 
Kontsevich graphs on $k-3$~internal vertices and three sinks.
%These sinks will become the internal vertices such that each is pointing at 
%a different new sink in the desired graphs.

\noindent%\paragraph*
{\textit{Variant 2}: \textit{Step 2.}} For every such graph, add three new sinks, make the first new sink a target of the first old sink, 
make the second new sink a target of the second old sink and make the third 
new sink a target of the third old sink. Now run over the $k-1$~internal
 vertices excluding the first old sink and declare %each 
every such internal vertex 
the second target of the first old sink. Independently, run over the $k-1$~internal 
vertices excluding the second old sink and declare each internal vertex to be 
the second target of the second old sink. Likewise, run over the $k-1$ internal 
vertices excluding the third old sink and declare each internal vertex to 
be the second target of the third old sink.

%\paragraph*
\noindent{\textit{Variant 2}: \textit{Step 3}}
%All graphs obtained after the first two steps must be skew-symmetrized.
Skew\/-\/symmetrise all the graphs at hand using~\eqref{EqMakeSkew}.

%\begin{example}
%An upper bound for the number of the tri\/-\/vector Kontsevich graphs on $k$~internal vertices is given in the table below (specifically for Variants~1 and~2 of the above algorithm).%
%\begin{table}[htb]
%\caption{How many bi\/-\/vector graphs are there at most at a given number of internal vertices~$k$\,?}
%\begin{center}
%\begin{tabular}{|c|c|c|c|c|c|c|c|c|}
%\hline
%$k$ & 1 & 2 & 3 & 4 & 5 & 6 & 7 & 8 \\
%\hline
%Variant~1 &  &  &  &  &  &  &  &  \\
%\hline
%Variant~2 &  &  &  &  &  &  &  &  \\
%\hline
%Total &  &  &  &  &  &  &  &  \\
%\hline
%\end{tabular}
%\end{center}
%\end{table}
%\end{example}

\subsection{\textbf{Non\/-\/iterative generator of the Leibniz $n$\/-\/vector graphs}}
The following algorithm generates all Leibniz graphs with a prescribed number of internal vertices and sinks. %\footnote
{Note that not only multi\/-\/vectors, but also all graphs of arbitrary differential order with respect to the sinks can be generated this way.}

\smallskip\noindent%
%\paragraph*
{\textit{Step 1:}} Generate all Kontsevich graphs of prescribed type on $k-1$ internal vertices and $n$~sinks, e.g., all $n$\/-\/vectors.

\noindent%
%\paragraph*
{\textit{Step 2:}} Run through the set of these Kontsevich graphs and in each of them, run through the set of its internal vertices~$v$.
For every vertex~$v$ %at hand 
do the following: re\/-\/enumerate the internal vertices so that this vertex is enumerated by~$k-1$.
% Changed a little bit:
%Suppose that 
This %last 
vertex already targets two %at the 
vertices, $i$ and~$j$, where $i<j<k-1$.
Proclaim the last, ($k-1$)th vertex to be the placeholder of the Jacobiator 
(see~\eqref{EqJacFig}), so we must still add the third arrow.
%%%
Let a new index~$\ell$ run up to~$i-1$ starting at~$n$ if %since 
only the $n$\/-\/vectors are produced.\footnote{%
If we want to generate not only $n$\/-\/vectors but all graphs of arbitrary differential orders, then we let $\ell$~start at~$0$ (so that the sinks are included).}
%%%
For every admissible value of~$\ell$, generate a new graph where the $\ell$th~vertex is proclaimed the third target of the Jacobiator vertex~$k-1$.
%This is, expanding the Leibniz rule with the index~$\ell<i$ 
(%so that repetitions are avoided%
%\footnote{
Restricting~$\ell$ by~$<i$, we reduce the number of possible repetitions in the set of Leibniz graphs. Indeed, for every triple $\ell<i<j$, the same Leibniz graph in which the Jacobiator stands on those three vertices would be produced from the three Kontsevich graphs: namely, those in which the $(k-1)$th~vertex targets at the~$\ell$th and~$i$th, at the~$\ell$th and~$j$th, and at the~$i$th and~$j$th vertices. 
In these three cases it is the~$j$th, $i$th, and $\ell$th~vertex, respectively, which would be appointed by the algorithm as the Jacobiator's third target.)

%, this Leibniz graph would be produced.%}

We use this algorithm to generate the Leibniz tri\/-{} and bi\/-\/vector graphs:
to establish Theorem~\ref{ThMainFew}, %so that 
we list all possible terms in the right\/-\/hand side of %the 
factorisation problem~\eqref{EqDiamond} at~$k\leqslant 4$ and
  %, respectively, %why "respectively"? rather "secondly"
then we filter out the improper bi\/-\/vectors in %components of 
the found %bi\/-\/vector graph 
solutions~$\cQ(\cP)$.

\begin{rem}%end subsubsection::"generator"
There are %{at most} 875,000
at least 265,495
Leibniz % tri\/-\/vector
graphs on $3$~sinks and $6$~internal vertices of which one is the Jacobiator vertex. When compared with Table~\ref{Tab5WheelLeibniz} on p.~\pageref{Tab5WheelLeibniz}, this estimate %although not sharp
suggests why at large $k\gtrsim 5$, the %economical 
breadth\/-\/first\/-\/search iterative algorithm from~%section
\S\ref{SecGenLeibnizIteratively} generates a smaller number of the Leibniz  tri\/-\/vector
graphs, namely, only the ones which can in principle be involved in the factorisation under study.
\end{rem}

\section{Main result} %: $k\leqslant 4$} %Part~I}

%
\begin{theor}[$k\leqslant4$]\label{ThMainFew}
The few-vertex solutions of problem \eqref{EqDiamond} are these \textup{(}note that disconnected Kontsevich graphs in $\cQ(\cP)$ are allowed\,\textup{!):}
\mbox{ %
}\\[1pt]
$\bullet$\ $k=1$\textup{\textbf{:}} The dilation $\dot{\cP} = \cP$ is a % the
unique, nontrivial and proper solution.\\
$\bullet$\ $k=2$\textup{\textbf{:}} No solutions exist \textup{(}in particular, neither trivial nor improper\textup{)}. \\
$\bullet$\ $k=3$\textup{\textbf{:}} %[$\geq$ April 27 ?] [NR$\rightarrow$AVK May 4]
There are no solutions\textup{:} %at $k=3$ either. There are 
neither Poisson\/-\/trivial nor Leibniz bi\/-\/vectors.\\ % solutions.\\
$\bullet$\ $k=4$\textup{\textbf{:}} A %The
unique nontrivial and proper solution is the Kontsevich tetrahedral 
flow~$\cQ_{1:\frac{6}{2}}(\cP)$ from Example~\textup{\ref{Ex3Wheel}}
\textup{(}see~\textup{\cite{Ascona96,Kontsevich2017Bourbaki}} and~\textup{\cite{tetra16,f16})}.
There is a one\/-\/dimensional space of Poisson trivial \textup{(}still proper\textup{)} solutions $\lshad\cP,X\rshad$\textup{;} % spanned by...
the Kontsevich $1$-\/vector~$X$ on three internal vertices is drawn in~\textup{\cite[App.\:F]{f16}}. %Moreover,
Intersecting with the former by $\{0\}$, there is a three\/-\/dimensional space of improper \textup{(}still Poisson\/-\/nontrivial\textup{)} solutions of the form~$\nabla\bigl(\cP,\Jac(\cP )\bigr)$. %it is spanned by... <3 pictures>

None of the solutions~$\cQ(\cP)$ known so far contains any $2$\/-\/cycles \textup{(}or ``eyes"~$\bullet\rightleftarrows\bullet$\textup{)}.\footnote{Finding 
   %Note that no search for any 
solutions~$\cQ(\cP)$ with tadpoles or extra sinks --\,with fixed arguments\,-- 
   %has been performed.
is a separate %an independent 
problem% for future research
.} 
\end{theor}

%\section{Towards a classification of flows %deformations 
%$\dot\cP=\cQ(\cP)$ using $k$-\/vertex graphs: $5\leqslant k\leqslant 8$ yet %and %loops
%two\/-\/cycles excluded}

%\subsection{Main result: $5\leqslant k\leqslant 8$} %Part~II}
We now %continue 
report a %our 
classification of Poisson bi\/-\/vector symmetries $\dot\cP=\cQ(\cP)$ which are %now
given by those Kontsevich graphs~$\cQ=\Ori(\gamma)$ on $k$~internal vertices  that can be obtained at $5\leqslant k\leqslant 9$ by orienting $k$-\/vertex connected graphs~$\gamma$ without double edges. By construction, this extra assumption keeps %permits / yields / leaves us with
only those %oriented 
Kontsevich graphs which may not contain~eyes.
% any two\/-\/cycles. %\\ %(or ``eyes'').

We first find %the set of 
such graphs~$\gamma$ that satisfy~$\Id(\gamma)=0$, then we exclude the coboundaries $\gamma = \Id(\gamma')$ for some graphs~$\gamma'$ on~$k-1$ vertices and $2k-3$~edges.

%We use the cocycle preservation / correspondence by the orientation mapping.
%DOUBLE CONTENT\\

\begin{theor}[$5\leqslant k\leqslant 8$]\label{ThMainMany}
Consider the vector space of non\/-\/oriented connected graphs on $k$~vertices and $2k-2$~edges,
without tadpoles and without multiple edges. %\textup{;}
   %for every vertex~$v$ require its valency to satisfy~$\# N(v)\geq 2$.
%Under these assumptions, we now find 
All nontrivial $\Id$-\/cocycles for $5\leqslant k\leqslant 8$ are exhausted by the following ones\textup{:}%\mbox{ %
%}
\\[1pt]
$\bullet$\ $k=5,7$\textup{\textbf{:}} No solutions. \\ 
$\bullet$\ $k=6$\textup{\textbf{:}}\label{Where5WheelIs} %
%%%\marginpar{Add picture} 
{\unitlength=1mm
\begin{picture}(0,0)(-15,1.5)
\put(65,0){$\boldsymbol{\gamma}_5 = {}$}
%\raisebox{0pt}[1pt][1pt]{
\put(85,0){
{\unitlength=0.3mm
\begin{picture}(55,53)(5,-5)%(27.5,23)
\put(27.5,8.5){\circle*{3}}
\put(0,29.5){\circle*{3}}
\put(-27.5,8.5){\circle*{3}}
\put(-17.5,-23.75){\circle*{3}}
\put(17.5,-23.75){\circle*{3}}
\qbezier%[40]
(27.5,8.5)(0,29.5)(0,29.5)
\qbezier%[40]
(0,29.5)(-27.5,8.5)(-27.5,8.5)
\qbezier%[40]
(-27.5,8.5)(-17.5,-23.75)(-17.5,-23.75)
\qbezier%[40]
(-17.5,-23.75)(17.5,-23.75)(17.5,-23.75)
\qbezier%[40]
(17.5,-23.75)(27.5,8.5)(27.5,8.5)
%%%
\put(0,0){\circle*{3}}
\qbezier%[30]
(27.5,8.5)(0,0)(0,0)
\qbezier%[30]
(0,29.5)(0,0)(0,0)
\qbezier%[30]
(-27.5,8.5)(0,0)(0,0)
\qbezier%[30]
(-17.5,-23.75)(0,0)(0,0)
\qbezier%[30]
(17.5,-23.75)(0,0)(0,0)
\end{picture}
}
}
%}
\put(95,0){${}+\dfrac{5}{2}$}
\put(114,0){
{\unitlength=0.4mm
%\raisebox{0pt}[1pt][1pt]{
\begin{picture}(50,30)(0,-4)%(27.5,23)
\put(12,0){\circle*{2.5}}
\put(-12,0){\circle*{2.5}}
\put(25,15){\circle*{2.5}}
\put(-25,15){\circle*{2.5}}
\put(-25,-15){\circle*{2.5}}
\put(25,-15){\circle*{2.5}}
%%%
\put(-12,0){\line(1,0){24}}
\put(-25,15){\line(1,0){50}}
\put(-25,-15){\line(1,0){50}}
\put(-25,-15){\line(0,1){32}}
\put(25,15){\line(0,-1){32}}
%%%
\qbezier%[30]
(25,15)(12,0)(12,0)
\qbezier%[30]
(-25,15)(-12,0)(-12,0)
\qbezier%[30]
(-25,-15)(-12,0)(-12,0)
\qbezier%[30]
(25,-15)(12,0)(12,0)
%%%
\put(-12.5,17){\oval(25,10)[t]}
\put(12.5,-17){\oval(25,10)[b]}
\put(0,2){\line(0,1){11}}
\put(0,-2){\line(0,-1){11}}
\end{picture}
%}
}%
}
%\put(50,20){.}
\end{picture}%
}%
%%%
A unique solution\footnote{There are only 12
 %(There are 12 6\/-\/vertex graphs satisfying the assumptions above) } %RB.19.Jul.2017 
admissible graphs to build cocycles from; of these 12, as many as 6 %checked
are zero graphs. This count shows % how big the decrease of the number of graphs is
to what extent the number of graphs decreases if one restricts to only the flows~$\cQ=\Ori(\gamma)$ obtained from cocycles~$\gamma\in\ker(\Id)$ in the non\/-\/oriented graph complex.} 
%%%
is given by the Kontse-\\
\parbox[t]{86mm}{vich\/--\/Willwacher pentagon\/-\/wheel cocycle \textup{(}%\cite{Kontsevich2017private},\cite{Willwacher2017private},
see Example~\textup{\ref{Ex5Wheel}).} % and Theorem~\ref{ThWheelCocycles})
%%%
%The bi\/-\/vector $\Ori(\boldsymbol{\gamma}_5$ consists of $90$ skew Kontsevich graphs.
%%%
The established %R.B.:13.SEP.2017.(23:13)
factorisation $\lshad\cP,\Ori(\boldsymbol{\gamma}_5)\rshad=\Diamond\bigl(\cP,\Jac(\cP)\bigr)$ %via $8,691$ skew Leibniz tri\/-\/vector graphs
$\lefteqn{\text{will be addressed in a separate paper \textup{(}see~\textup{\cite{WeFactorize5Wheel})}.}}$}%
%and an algorithm for Or in section~\ref{SecOrient}.
\\[0.5pt]
%$\bullet$\ $k=7$\textup{\textbf{:}} No solutions.% 
%%%
%(There is only one 7-vertex cocycle, having four graphs in the balance $1:1:1:\frac{1}{2}$; it is a coboundary; namely, the differential of the graph $\mathsf{12,13,15,23,24,25,34,35,36,46,56}$.) %RB.19.Jul.2017
%\\
$\bullet$\ $k=8$\textup{\textbf{:}} The only solution~$\boldsymbol{\gamma}_7$ consists of the heptagon\/-\/wheel and $45$~other graphs \textup{(}see Table~\textup{\ref{Tab1+45}}, in which the coefficient of heptagon graph is~$\mathbf{1}$ in bold,
and~\textup{\cite{JNMP2017})}.%
%The space of 8-vertex cocycles is 33-dimensional; the space of coboundaries in it is 32-dimensional. %RB.19.Jul.2017
%\\
%$\bullet$\ $k=9$\textup{\textbf{:}} No solutions.
%%%%%%%%%%%%%%%%%%%%%%%%%%%%%%%%%%%%%%%%%%%%%%%%%%%%%%%%%%%%%%%%%%%%%%%%%
\begin{table}[htb]
\caption{The heptagon\/-\/wheel graph cocycle~$\boldsymbol{\gamma}_7$.}\label{Tab1+45}
% Even \tiny was not tiny enough % The problem is the space between the numbers
\centerline{{\upshape\footnotesize%\scriptsize%\tiny
 %\fontsize{6}{1}\selectfont 
\begin{tabular}{l r l r} 
\br%\hline
Graph encoding & Coeff.\ \  & Graph encoding & Coeff.\\
\mr%\hline
16	17	18	23	25	28	34	38	46	48	57	58	68	78	& 	$\mathbf{1}\phantom{/1}$		& 	12	13	18	25	26	37	38	45	46	47	56	57	68	78	&	$-7\phantom{/1}$	\\
12	14	18	23	27	35	37	46	48	57	58	67	68	78	& 	$-21/8$	& 	12	14	16	23	25	36	37	45	48	57	58	67	68	78	&	$77/8$	\\
13	14	18	23	25	28	37	46	48	56	57	67	68	78	& 	$-77/4$	& 	13	16	17	24	25	26	35	37	45	48	58	67	68	78	&	$-7\phantom{/1}$	\\
12	13	15	24	27	35	36	46	48	57	58	67	68	78	& 	$-35/8$	& 	14	15	17	23	26	28	37	38	46	48	56	57	68	78	&	$49/4$	\\
12	13	18	24	26	37	38	46	47	56	57	58	68	78	& 	$49/8$	& 	12	16	18	27	28	34	36	38	46	47	56	57	58	78	&	$-147/8$	\\
14	17	18	23	25	26	35	37	46	48	56	58	67	78	& 	$77/8$	& 	12	15	16	27	28	35	36	38	45	46	47	57	68	78	&	$-21/8$	\\
12	13	18	26	27	35	38	45	46	47	56	57	68	78	& 	$-105/8$		& 	12	14	18	23	27	35	36	45	46	57	58	67	68	78	&	$-35/8$	\\
12	14	18	23	27	36	38	46	48	56	57	58	67	78	& 	$7/8$	& 	14	15	16	23	26	28	37	38	46	48	57	58	67	78	&	$-49/4$	\\
12	14	15	23	27	35	36	46	48	57	58	67	68	78	& 	$35/8$	& 	12	15	18	23	28	34	37	46	48	56	57	67	68	78	&	$105/8$	\\
12	13	14	27	28	36	38	46	47	56	57	58	68	78	& 	$-49/8$	& 	12	14	17	23	26	37	38	46	48	56	57	58	68	78	&	$-49/8$	\\
12	13	18	25	27	34	36	47	48	56	58	67	68	78	& 	$35/4$	& 	12	16	18	25	27	35	36	37	45	46	48	57	68	78	&	$49/1\lefteqn{6}$	\\
12	13	14	25	26	36	38	45	47	57	58	67	68	78	& 	$-119/1\lefteqn{6}$	& 	12	13	18	25	27	35	36	46	47	48	56	57	68	78	&	$7\phantom{/1}$	\\
12	13	15	24	28	36	38	47	48	56	57	67	68	78	& 	$49/8$	& 	12	14	18	25	28	34	36	38	47	57	58	67	68	78	&	$-7\phantom{/1}$	\\
12	13	14	23	28	37	46	48	56	57	58	67	68	78	& 	$77/4$	& 	12	16	18	25	27	35	36	37	45	46	48	58	67	78	&	$-77/1\lefteqn{6}$	\\
12	15	17	25	26	35	36	38	45	47	48	67	68	78	& 	$-49/8$	& 	12	14	18	23	27	35	38	46	47	57	58	67	68	78	&	$77/4$	\\
13	15	18	24	26	28	37	38	46	47	56	57	68	78	& 	$-49/4$	& 	12	14	15	23	27	36	38	46	48	57	58	67	68	78	&	$35/2$	\\
13	14	18	25	26	28	36	38	47	48	56	57	67	78	& 	$-49/4$	& 	12	13	18	25	27	34	36	46	48	57	58	67	68	78	&	$-105/8$	\\
12	14	18	23	28	35	37	46	48	56	57	67	68	78	& 	$-7\phantom{/1}$ &	12	15	16	25	27	35	36	38	46	47	48	57	68	78	&	$-7\phantom{/1}$	\\
12	14	18	23	28	36	38	46	47	56	57	58	67	78	& 	$-7\phantom{/1}$		& 	12	13	16	25	28	34	37	47	48	57	58	67	68	78	&	$-147/1\lefteqn{6}$	\\
12	15	16	25	27	35	36	38	46	47	48	58	67	78	& 	$49/8$	& 	12	13	17	25	26	35	37	45	46	48	58	67	68	78	&	$-77/4$	\\
12	14	18	23	28	36	37	46	47	56	57	58	68	78	& 	$49/8$		& 	12	14	17	23	27	35	38	46	48	57	58	67	68	78	&	$-49/8$	\\
12	13	15	26	27	35	36	45	47	48	58	67	68	78	& 	$-7\phantom{/1}$	& 	12	13	15	26	28	35	37	45	46	47	58	67	68	78	&	$-7/4$	\\
12	13	18	24	28	35	38	46	47	57	58	67	68	78	& 	$7\phantom{/1}$	& 	12	14	18	23	26	36	38	47	48	56	57	58	67	78	&	$-7\phantom{/1}$ 	\\
\br%\hline
\end{tabular}}}
\end{table}
\end{theor}

\begin{rem} 
The wheel graphs are built of %from %out of 
triangles. The differential~$\Id$ cannot produce any triangle since %we did 
multiple edges are not allowed. Therefore, all wheel cocycles are nontrivial. Note also that every wheel graph with $2\ell$~spokes is invariant under a mirror reflection with respect to a diagonal consisting of two edges attached to the centre. Hence there exists an edge permutation that swaps $2\ell-1$~pairs of edges. By footnote~\ref{FootZeroGraph} such graph equals~zero.
\end{rem}

\appendix
\section{How the orientation mapping~$\Ori$ is calculated%
%sum of oriented Kontsevich graphs is obtained from a non\/-\/oriented graph
}\label{SecOrient}
%we need a specification of graphs to work on so that K. graphs can be obtained by orientation & () () sinks.
\noindent%
The algorithm lists all ways in which a given non-oriented graph can be oriented in such a way that it becomes a Kontsevich graph on two sinks. It consists of two steps: %parts 
\begin{enumerate}
	\item choosing %, within the internal vertices, 
	the source(s) of the two arrows pointing at the first and second sink, respectively;
%choosing an internal vertex or pair of different internal vertices which is or are the tool(s) of arrows going..
	\item orienting the edges between the internal vertices in all admissible ways, so that only Kontsevich graphs are obtained. %generated
\end{enumerate}

%\paragraph*
\smallskip\noindent%
{\textit{Step 1.}} Enumerate the $k$ vertices of a given non-oriented,
connected graph using~$2$, $\dots$, $k+1$. They become the internal vertices 
of the oriented graph. Now add the two sinks to the non\/-\/oriented graph, the sinks
enumerated using~$0$ and~$1$. Let~$a$ and~$b$ be a non\/-\/strictly ordered 
($a\leqslant b$) pair of internal vertices in the graph. Extend the graph by oriented edges $a\rightarrow 0$ and $b\rightarrow 1$ from vertices~$a$ and~$b$ to the sinks~$0$ and~$1$, respectively. 

\begin{rem} The choice of such a base pair, that is, the vertex or vertices from which two arrows are issued to the sinks, is an external input in the orientation procedure. %why do we remark this previous frase?
Let us agree that if, at any step of the algorithm, a contradiction is achieved so that a graph at hand cannot be of Kontsevich type, the oriented graph draft is discarded; one proceeds with the next options in that loop, or if the former loop is finished, with the next level-up loops, or -- having returned to the choice of base vertices -- with the next base. In other words, we do not exclude in principle a possibility to have no admissible orientations for a particular choice of the base for a given non-oriented graph.
\end{rem}

%\paragraph*{Step 2.}
\begin{notation}
Let $v$ be an internal vertex. 
Recalling from p.~\pageref{pNotationNHT} the notation for the set~$N(v)$ of neighbours of~$v$, the (initially empty) set~$H(v)$ of arrowheads of oriented edges issued from the vertex~$v$, and the (initially empty) set~$T(v)$ of tails for oriented edges pointing at $v$, we now put by definition
$F(v) \mathrel{{:}{=}} N(v)\setminus (H(v)\cup T(v))$. In other words, $F(v)$~is the subset of neighbours connected with~$v$ by a non\/-\/oriented edge.
\end{notation}

\noindent%\paragraph*
{\textit{Step~2.1. Inambiguous orientation of \textup{(}some\textup{)} edges.}}
Here we use that every internal vertex of a Kontsevich graph should be the tail of %have 
exactly two outgoing arrows. %added
We run %repetitively 
over the set of all internal vertices~$v$. For every vertex such that 
the number of elements $\# H(v)=2$%in the set of targets
, proclaim $T(v) \mathrel{{:}{=}} N(v)\setminus H(v)$, whence $F(v)=\varnothing$.
If for a vertex~$v$ we have that $\# H(v)=1$ and $\# F(v)=1$, then include $F(v)\hookrightarrow H(v)$, 
that is, convert %make from 
a unique non\/-\/oriented edge touching~$v$ into an outgoing edge 
issued from %with respect to 
this vertex. %~$v$.
If $\# H(v)=0$ and $\# F(v)=2$, also include $F(v)\hookrightarrow H(v)$, 
effectively making both non\/-\/oriented edges outgoing from~$v$.

Repeat the three parts of Step~2.1 while any of the sets~$F(v)$, $T(v)$, or% and
~$S(v)$
is modified for at least one internal vertex~$v$ unless a contradiction is revealed. 
Summarising, Step~2.1 amounts to finding the edge orientations which are implied by the choice of the base pair $a$,~$b$ and by all the orientations of edges fixed earlier.

\noindent%\paragraph*
{\textit{Step 2.2. Fixing the orientation of \textup{(}some\textup{)} remaining edges.}} 
%Now we make choices; we fix the orientation of some remaining edges.
Choose an internal ver\-tex~$v$ such that $H(v)<2$ and such that $H(v)\neq\varnothing$ or $T(v)\neq\varnothing$, that is, choose a vertex that is not yet equipped with two outgoing edges and that is attached to an oriented edge. %rewritten
   %\footnote{To do: By this we aim to minimize chances of abort}. 
If~$\# H(v)=1$, then run over the non\/-\/empty set~$F(v)$: for every vertex~$w$ in~$F(v)$, 
include $\{w\}\hookrightarrow H(v)$ and start over at Step~2.1.
Otherwise, i.e.\ if $H(v)=\varnothing$, run over all ordered pairs $(u,v)$ of vertices in the set~$F(v)$: for every such pair, make $H(v) \mathrel{{:}{=}} \{u,w\}$ and start over at Step~2.1.

By realising Steps~1 and~2 we accumulate the sum of fully oriented Kontsevich graphs.
   %WHAT TO DO with such graphs, produced by the above???

\ack
%\subsection*{Acknowledgements}
A.\,V.\,Kiselev thanks the Organising committee of the international 
conference ISQS'25 on integrable systems and quantum symmetries 
(6--10~June 2017 in $\smash{\text{\v{C}VUT}}$ Prague, Czech Republic) for a warm atmosphere 
during the meeting. 
The authors are grateful %to G.~Chadzitaskos for stimulating discussions and 
to %A.~Bouisagouane for fruitful cooperation and S.\,O.\,Kri\-vo\-nos 
M.~Kontsevich and T.~Willwacher for helpful discussion.
%%% 
We also thank %the 
Center for Information Technology of the University of Groningen 
%for their support and 
for providing access to %the 
\textsf{Peregrine} high performance computing cluster.\quad
%%%
This research %A.\,K. 
was supported in part by %by NWO grant VENI~639.031.623 (The Netherlands) and 
JBI~RUG project~106552 (Groningen, The Netherlands).
A part of this research was done while R.\,Buring and A.\,V.\,Kiselev were 
visiting at the~IH\'ES (Bures\/-\/sur\/-\/Yvette, France) and A.\,V.\,Kiselev 
was visiting at the~MPIM (Bonn, Germany).
   %This paper was written in part while NJR and AVK were visiting at CVUT Decin in June~2017.

\section*{References}
\begin{thebibliography}{10}
%%%%%%%%%%%%%%%%%%%%%%%%%%%%%  arXiv  %%%%%%%%%%%%%%%%%%%%%%%%%%%%%%%%%%

\bibitem{tetra16}
\by{Bouisaghouane A., Kiselev A.\,V.} (2017)
Do the Kon\-tse\-vich tetrahedral flows preserve or destroy the space of Poisson bi\/-\/vectors\,?
%\jour{Pre\-print} $\smash{\text{IH\'ES}}$/M/16/12 (Bures\/-\/sur\/-\/Yvette, 
%France), %12~p.
\jour{J.~Phys.\textup{:}\ Conf.\ Ser.} \vol{804} 
Proc.\ XXIV Int.\ conf.\
`Integrable Systems and Quantum Symmetries' (14--18 June 2016, $\smash{\text{\v{C}VUT}}$ Prague,
Czech Republic), Paper~012008, 10~p.\ %\texttt{arXiv:1609.06677} [q-alg]
(\jour{Pre\-print} 
%$\smash{\text{IH\'ES}}$/M/16/12 (Bures\/-\/sur\/-\/Yvette, %2015
%France), 
\texttt{arXiv:1609.06677} [q-alg])

\bibitem{f16}
\by{Bouisaghouane A., Buring R., Kiselev A.} (2017)
The Kontsevich tetrahedral flow revisited, \jour{J.~Geom.\ Phys.} 
\vol{119}%September 2017 issue.
, 272--285.\ %20${}+{}$ix~p.
(\jour{Preprint} \texttt{arXiv:1608.01710} %(v4)
[q-alg])

\bibitem{sqs15}
\by{Buring R., Kiselev A.\,V.} (2017) On the Kon\-tse\-vich $\star$-\/product asso\-ci\-a\-ti\-vi\-ty me\-cha\-n\-ism, \jour{PEPAN Letters} \vol{14}:2, 403--407.\ %
(\jour{Preprint} \texttt{arXiv:1602.09036} [q-alg])

\bibitem{cpp}
\by{Buring R., Kiselev A.\,V.} (2017) %Software modules 
The expansion $\star$ mod~$\bar{o}(\hbar^4)$
and computer\/-\/assisted proof schemes in the Kon\-tse\-vich deformation quantization,
\jour{Preprint} %$\smash{\text{IH\'ES}}$/M/17/05, %50${}+{}$xvi~p., 
\texttt{arXiv:1702.00681}~%(v2)
[math.CO] %68~p.
%%%%%%%%%%%%%%%%%%%%%%%%%%%%%%%%%%%%%%%%%%%%5
%\jour{Preprint} \texttt{arXiv:1702.00681} (v1) [math.CO], 44~+ xvi~p.
%see link:\\
%\texttt{https://github.com/rburing/kontsevich\symbol{"5F}graph\symbol{"5F}series-cpp}

\bibitem{JNMP2017}
\by{Buring R., Kiselev A.\,V., Rutten N.\,J.} (2017)
The heptagon\/-\/wheel cocycle in the Kontsevich graph complex,
   %as universal symmetry of Poisson structures,
\jour{J.~Nonlin.\ Math.\ Phys.} \textbf{24} %Open Access
Suppl.~1 `Local \& Nonlocal Symmetries in Mathematical Physics', 
%(N.~Euler and E.~%Enrique
%Reyes, eds), in preparation. %15pp.
157--173.
(\jour{Preprint} \texttt{arXiv:1710.00658} [math.CO])

\bibitem{WeFactorize5Wheel}
\by{Buring R., Kiselev A.\,V., Rutten N.\,J.} (2017)
Poisson brackets symmetry from the pentagon\/-\/wheel cocycle in the graph complex,
\jour{Preprint} \texttt{arXiv:1712.05259} [math-ph]
%The Kontsevich\/--\/Willwacher pentagon\/-\/wheel symmetry of %classical
%Poisson structures%: construction and approbation/verification%
%, SDSP~IV (12--16~June 2017, $\smash{\text{\v{C}VUT}}$ D\v{e}\v{c}\'\i{}n, Czech Republic)%,
%in preparation
%.

\bibitem{DolgushevRogersWillwacher}
\by{Dolgushev V. A., Rogers C. L., Willwacher T. H.} (2015) Kontsevich's graph complex, GRT, and the deformation complex of the sheaf of polyvector fields,
\jour{Ann.\ Math.} \vol{182}:3, 855--943.\ %
(\jour{Preprint} \texttt{arXiv:1211.4230} [math.KT])

\bibitem{Ascona96}
\by{Kontsevich M.} (1997)
Formality conjecture. 
\book{Deformation theory and symplectic geometry} (Ascona %June 17--21,
1996, D.\,%aniel 
Sternheimer, J.\,%ohn 
Rawnsley and S.\,%imone 
Gutt, eds), 
Math.\ Phys.\ Stud.~\vol{20}, Kluwer Acad.\ Publ., Dordrecht, 139--156.

\bibitem{Kontsevich2017Bourbaki}
\by{Kontsevich M.} (2017) Derived Grothendieck\/--\/Teichm\"uller group and graph complexes [after T.~Will\-wa\-cher], \jour{S\'eminaire Bourbaki} (69\`eme ann\'ee, Janvier 2017), no.~1126, 26~p.

\bibitem{Kontsevich2017private}
\by{Kontsevich M.} (2017) Private communication.%[11 APR 2017]

\bibitem{KhoroshkinWillwacherZivkovic}
\by{Khoroshkin A., Willwacher T., \v{Z}ivkovi\'c M.} (2017) Differentials on graph complexes, \jour{Adv.\ Math.} \vol{307}, 1184--1214.\ %
(\jour{Preprint} \texttt{arXiv:1411.2369} [q-alg])

\bibitem{WillwacherGRT}
\by{Willwacher T.} (2015) M.~Kontsevich's graph complex and the Grothendieck\/--\/Teichm\"uller Lie algebra,
\jour{Invent.\ Math.} \vol{200}:3, 671--760.\ %
(\jour{Preprint} \texttt{arXiv:1009.1654} [q-alg]) %v4 (2013)

\bibitem{Willwacher2017private}
\by{Willwacher T.} (2017) Private communication.%[26 APR 2017]

%%%%%%%%%%%%%%%%%%%%%%%%%%%%%  arXiv  %%%%%%%%%%%%%%%%%%%%%%%%%%%%%%%%%%

\end{thebibliography}
\end{document}


%%%%%%%%%%%%%%%%%%%%%%%%%%%%%%%%%   JPCS  %%%%%%%%%%%%%%%%%%%%%%%%%%%%%%%%%%
\bibitem{tetra16}
Bouisaghouane A and Kiselev A V 2017
Do the Kon\-tse\-vich tetrahedral flows preserve or destroy the space of Poisson bi\/-\/vectors\,?
%\jour{Pre\-print} $\smash{\text{IH\'ES}}$/M/16/12 (Bures\/-\/sur\/-\/Yvette, 
%France), %12~p.
\jour{J.~Phys.\textup{:}\ Conf.\ Ser.} \vol{804} 
012008
%Proc.\ XXIV Int.\ conf.\
%`Integrable Systems and Quantum Symmetries' (14--18 June 2016, $\smash{\text{\v{C}VUT}}$ Prague,
%Czech Republic), Paper~012008, 10~p.\ %\texttt{arXiv:1609.06677} [q-alg]
(\jour{Pre\-print} 
%$\smash{\text{IH\'ES}}$/M/16/12 (Bures\/-\/sur\/-\/Yvette, %2015
%France), 
\texttt{arXiv:1609.06677} [q-alg])

\bibitem{f16}
Bouisaghouane A, Buring R and Kiselev A 2017
The Kontsevich tetrahedral flow revisited \jour{J.~Geom.\ Phys.} 
\vol{119}%September 2017 issue.
\ 272--85\ %20${}+{}$ix~p.
(\jour{Preprint} \texttt{arXiv:1608.01710} %(v4)
[q-alg])

\bibitem{sqs15}
Buring R and Kiselev A V 2017 On the Kon\-tse\-vich $\star$-\/product asso\-ci\-a\-ti\-vi\-ty me\-cha\-n\-ism \jour{PEPAN Letters} \vol{14}:2 403--7\ %
(\jour{Preprint} \texttt{arXiv:1602.09036} [q-alg])

\bibitem{cpp}
Buring R and Kiselev A V 2017 %Software modules 
The expansion $\star$ mod~$\bar{o}(\hbar^4)$
and computer\/-\/assisted proof schemes in the Kon\-tse\-vich deformation quantization
\jour{Preprint} %$\smash{\text{IH\'ES}}$/M/17/05 %, 50${}+{}$xvi~p.
\texttt{arXiv:1702.00681}~%(v2)
[math.CO] %68~p.
%%%%%%%%%%%%%%%%%%%%%%%%%%%%%%%%%%%%%%%%%%%%5
%\jour{Preprint} \texttt{arXiv:1702.00681} [math.CO], 44~+ xvi~p.
%see link:\\
%\texttt{https://github.com/rburing/kontsevich\symbol{"5F}graph\symbol{"5F}series-cpp}

\bibitem{JNMP2017}
Buring R, Kiselev A V and Rutten N J 2017
The heptagon\/-\/wheel cocycle in the Kontsevich graph complex
   %as universal symmetry of Poisson structures,
\jour{J.~Nonlin.\ Math.\ Phys.} %Open Access
\textbf{24} 
Suppl.~1 `Local \& Nonlocal Symmetries in Mathematical Physics' 
%(N.~Euler and E.~%Enrique
%Reyes, eds)
157--73
(\textit{Preprint} \texttt{arXiv:1710.00658} [math.CO])

\bibitem{WeFactorize5Wheel}
Buring R, Kiselev A V and Rutten N J 2017
Poisson brackets symmetry from the pentagon\/-\/wheel cocycle in the graph complex
\textit{Preprint} \texttt{arXiv:1712.05259} [math-ph]
%The Kontsevich\/--\/Willwacher pentagon\/-\/wheel symmetry of %classical
%Poisson structures %: construction and approbation/verification%
%, SDSP~IV (12--16~June 2017, $\smash{\text{\v{C}VUT}}$ D\v{e}\v{c}\'\i{}n, Czech Republic)%,
%\textit{SDSP~IV} (12--16~June 2017, $\smash{\text{\v{C}VUT}}$ D\v{e}\v{c}\'\i{}n, Czech Republic)

\bibitem{DolgushevRogersWillwacher}
Dolgushev V A, Rogers C L and Willwacher T H 2015 Kontsevich's graph complex, GRT, and the deformation complex of the sheaf of polyvector fields
\jour{Ann.\ Math.} \vol{182}:3 855--943\ %
(\jour{Preprint} \texttt{arXiv:1211.4230} [math.KT])

\bibitem{Ascona96}
Kontsevich M 1997
Formality conjecture
\emph{Deformation theory and symplectic geometry} June 17--21 1996 Ascona
\jour{Math.\ Phys.\ Stud.}~vol 20 
ed D\,%aniel 
Sternheimer et al %, J\,%ohn 
%Rawnsley and S\,%imone 
%Gutt
(Dordrecht: Kluwer Acad.\ Publ.)
pp 139--56

\bibitem{Kontsevich2017Bourbaki}
Kontsevich M 2017 Derived Grothendieck\/--\/Teichm\"uller group and graph complexes [after T.~Will\-wa\-cher] \jour{S\'eminaire Bourbaki} %(69\`eme ann\'ee, Janvier 2017), no.~
1126 %, 26~p.

\bibitem{Kontsevich2017private}
Kontsevich M 2017 Private communication%[11 APR 2017]

\bibitem{KhoroshkinWillwacherZivkovic}
Khoroshkin A, Willwacher T and \v{Z}ivkovi\'c M 2017 Differentials on graph complexes \jour{Adv.\ Math.} \vol{307} 1184--214\ %
(\jour{Preprint} \texttt{arXiv:1411.2369} [q-alg])

\bibitem{WillwacherGRT}
Willwacher T 2015 M.~Kontsevich's graph complex and the Grothendieck\/--\/Teichm\"uller Lie algebra
\jour{Invent.\ Math.} \vol{200}:3 671--760\ %
(\jour{Preprint} \texttt{arXiv:1009.1654} [q-alg]) %v4 (2013)

\bibitem{Willwacher2017private}
Willwacher T 2017 Private communication%[26 APR 2017]
%%%%%%%%%%%%%%%%%%%%%%%%%%%%%%%%%   JPCS  %%%%%%%%%%%%%%%%%%%%%%%%%%%%%%%%%%




%\paragraph*{Open problem:} 
\begin{open}
The infinitesimal deformations $\frac{d}{d\epsilon_*}(\cP)=\cQ_*(\cP)$, e.g., the ones encoded by the Kontsevich graphs, generate the Lie algebra under taking the commutators $\bigl[%\frac
{d}/{d\epsilon_i},%\frac
{d}/{d\epsilon_j}\bigr]$ (that is, %respectively, 
under the insertion of the Kontsevich graph~$\cQ_i$ into every internal vertex of~$\cQ_j$ via the Leibniz rule, minus visa versa).
How big is this Lie algebra\,? Does it amount, modulo zero graphs and improper terms of the form~$\nabla\bigl(\cP,\Jac (\cP)\bigr)$, to the vector space of its generators\,?

For example, the commutator $\bigl[\Ori(\boldsymbol{\gamma}_3),\Ori(\boldsymbol{\gamma}_5)\bigr]$ of the tetrahedral and pentagon\/-\/wheel flows 
is a graph on $k=9$~internal vertices and without eyes. 
%hence it is not obtained by any $(2l+1)$-wheel cocycle form \cite{DolgushevRogersWillwacher}. 
%Is this a solution as well?
By Theorems~\ref{ThOrient} and~\ref{ThMainMany}, 
%and results from~\cite{DolgushevRogersWillwacher,WillwacherGRT},
this bi\/-\/vector %encoded by the Kontsevich graphs 
graph is trivial in the respective Poisson cohomology. %group.
\end{open}
