%%%%%%%%%%%%%%%%%%%%%%% file template.tex %%%%%%%%%%%%%%%%%%%%%%%%%
%
% This is a general template file for the LaTeX package SVJour3
% for Springer journals.          Springer Heidelberg 2010/09/16
%
% Copy it to a new file with a new name and use it as the basis
% for your article. Delete % signs as needed.
%
% This template includes a few options for different layouts and
% content for various journals. Please consult a previous issue of
% your journal as needed.
%
%%%%%%%%%%%%%%%%%%%%%%%%%%%%%%%%%%%%%%%%%%%%%%%%%%%%%%%%%%%%%%%%%%%
%
% First comes an example EPS file -- just ignore it and
% proceed on the \documentclass line
% your LaTeX will extract the file if required
\begin{filecontents*}{example.eps}
%!PS-Adobe-3.0 EPSF-3.0
%%BoundingBox: 19 19 221 221
%%CreationDate: Mon Sep 29 1997
%%Creator: programmed by hand (JK)
%%EndComments
gsave
newpath
  20 20 moveto
  20 220 lineto
  220 220 lineto
  220 20 lineto
closepath
2 setlinewidth
gsave
  .4 setgray fill
grestore
stroke
grestore
\end{filecontents*}
%
\RequirePackage{fix-cm}
%
%\documentclass{svjour3}                     % onecolumn (standard format)
%\documentclass[smallcondensed]{svjour3}     % onecolumn (ditto)
\documentclass[smallextended]{svjour3}       % onecolumn (second format)
%\documentclass[twocolumn]{svjour3}          % twocolumn
%
\smartqed  % flush right qed marks, e.g. at end of proof
%
\usepackage{graphicx}
%
% \usepackage{mathptmx}      % use Times fonts if available on your TeX system
%
% insert here the call for the packages your document requires
%\usepackage{latexsym}
% etc.
\usepackage[utf8]{inputenc}
\usepackage{dsfont,xspace,amsmath,amssymb}
%\usepackage[allowspaces]{mathtools}
%\usepackage{mathrsfs}
\usepackage[marginclue,langtrack]{fixme} % Fixme und todo-notes
\usepackage[autostyle=true]{csquotes}
%\usepackage[algoruled]{algorithm2e}
%\usepackage{enumerate}
%\usepackage{tikz}
%\usetikzlibrary{arrows,shapes,decorations.pathreplacing,positioning,matrix,calc}
%\tikzstyle{depot}=[rectangle,draw=black,thick,minimum size=7mm,inner sep=0pt]
%\tikzstyle{cust}=[circle,draw=black,thick,minimum size=8mm,inner sep=0pt]
%\tikzstyle{nodeSimple}=[circle,draw=black,thick,width=2mm]
%\tikzstyle{edgeThick}=[->,shorten >=1pt,thick,>=stealth',shorten <=1pt]
%\tikzstyle{edgeThickLeft}=[->,bend left,shorten >=1pt,thick,>=stealth',shorten <=1pt]
%\tikzstyle{edgeThickDoubleLeft}=[->,double,bend left,shorten >=1pt,thick,>=stealth',shorten <=1pt]
%\tikzstyle{edgeThickRight}=[->,bend right,shorten >=1pt,thick,>=stealth',shorten <=1pt]
%\tikzstyle{edgeThickDoubleRight}=[->,double,bend right,shorten >=1pt,thick,>=stealth',shorten <=1pt]
%\tikzstyle{edgeThin}=[->,shorten >=1pt,thin,>=stealth',shorten <=1pt]
%\tikzstyle{edgeThinShort}=[->,shorten >=3pt,thin,>=stealth',shorten <=3pt]
%\tikzstyle{edgeThinDashed}=[->,shorten >=1pt,thin,>=stealth',shorten <=1pt,dashed]
%\tikzstyle{edgeThickStraight}=[->,shorten >=1pt,thick,>=stealth',shorten <=1pt]
%\tikzstyle{edgeThickShort}=[->,shorten >=3pt,thick,>=stealth',shorten <=3pt]
%\usepackage{subcaption}
%\captionsetup{compatibility=false}
%
% please place your own definitions here and don't use \def but
% \newcommand{}{}
\newcommand{\Rbb}{\ensuremath{\mathds{R}}}
\newcommand{\set}[1]{\ensuremath{\left\{#1\right\}}}
\newcommand{\setval}[2]{\set{#1, \ldots, #2}}
%\newcommand{\card}[1]{\left|#1\right|}
%\newcommand{\Wlog}{w.\,l.\,o.\,g.\xspace}%
\newcommand{\cf}{cf.\xspace}%
\newcommand{\eg}{e.\,g.\xspace}%
%\newcommand{\GenStarCardPath}{Star-Shaped $r$-Cardinality-Path Inequalities\xspace}
%\newcommand{\dep}{\delta^+}
%\newcommand{\dem}{\delta^-}
\newcommand{\Nbb}{\ensuremath{\mathds{N}}}
%\newcommand{\Mcal}{\ensuremath{\mathcal{M}}}
%\newcommand{\Onebb}{\ensuremath{\mathds{1}}}
%\newcommand{\PstC}{\ensuremath{\mathscr{P}_{s,t}^{\leq k}}\xspace}
%\newcommand{\PstCcust}[1]{\ensuremath{\mathscr{P}_{s,t}^{\leq #1}}\xspace}
%\newcommand{\pPathPolytope}{$(s,t)$-$p$-Path Polytope\xspace}
%\DeclareMathOperator{\conv}{conv}
\newcommand{\Ocal}{\ensuremath{\mathcal{O}}}
%\newcommand{\Pcal}{\ensuremath{\mathcal{P}}}
\newcommand{\ie}{i.\,e.\xspace}%
%\newcommand{\Acal}{\ensuremath{\mathcal{A}}}
%\newcommand{\Bcal}{\ensuremath{\mathcal{B}}}
%\newcommand{\Ical}{\ensuremath{\mathcal{I}}}
%\DeclareMathOperator{\lin}{lin}
%\newcommand{\stern}{\raisebox{0.8ex}{\(\ast\)}}
%\DeclareMathOperator{\bid}{bid}
%\newcommand{\floor}[1]{\ensuremath{\left\lfloor #1 \right\rfloor}}
%\newcommand{\ceil}[1]{\ensuremath{\left\lceil #1 \right\rceil}}
%\DeclareMathOperator{\even}{even}
%\DeclareMathOperator{\odd}{odd}
%\newcommand{\abs}[1]{\ensuremath\left|#1\right|}
%\newcommand{\Zbb}{\ensuremath{\mathds{Z}}}
%\newcommand{\texorpdfstring}[2]{#2}
%\newcommand{\Scal}{\ensuremath{\mathcal{S}}}
%\newcommand{\argmin}{\operatorname{argmin}}
%\newcommand{\argmax}{\operatorname{argmax}}
%\newcommand{\R}{\ensuremath{\mathds{R}}}
\DeclareMathOperator{\Proj}{Proj}
\DeclareMathOperator{\argmin}{argmin}

%\allowdisplaybreaks
%
% Insert the name of "your journal" with
\journalname{Mathematical Programming}
%
\begin{document}

\title{The quadratic assignment problem: the linearization of Xia and Yuan is weaker than the linearization of Adams and Johnson
%\thanks{Grants or other notes
%about the article that should go on the front page should be
%placed here. General acknowledgments should be placed at the end of the article.}
}
\subtitle{and a family of cuts to narrow the gap}

\titlerunning{The QAP: a comparison of the linearizations of Xia\&Yuan and Adams\&Johnson}        % if too long for running head

\author{Christine Huber \and
		Wolfgang~F.~Riedl
}

%\authorrunning{Short form of author list} % if too long for running head

\institute{Wolfgang~F.~Riedl \at	
           Universität der Bundeswehr München\\
					 Professur für Statistik, insb. Risikomanagement\\
					 Werner-Heisenberg-Weg 39\\
					 D - 85577 Neubiberg\\
					 Germany\\
              %Tel.: +123-45-678910\\
              %Fax: +123-45-678910\\
              \email{wf.riedl@unibw.de}        
}

\date{Received: date / Accepted: date}
% The correct dates will be entered by the editor


\maketitle

\begin{abstract}
The quadratic assignment problem is a well-known optimization problem with numerous applications. A common strategy to solve it is to use one of its linearizations and then apply the toolbox of mixed integer linear programming methods. One measure of quality of a mixed integer formulation is the quality of its linear relaxation.

In this paper, we compare two linearizations of the quadratic assignment problem and prove that the linear relaxation of the linearization of Adams and Johnson is contained in the linear relaxation of the linearization of Xia and Yuan. We furthermore develop a Branch and Cut approach using the insights obtained in the proof that enhances the linearization of Xia and Yuan via a new family of cuts called $ab$-cuts.

\keywords{quadratic assignment problem\and linearization\and lift and project\and linear relaxation\and cutting planes}
% \PACS{PACS code1 \and PACS code2 \and more}
\subclass{MSC 52-B05 \and MSC 05-C38}
\end{abstract}

\input{qap}

%\section{Introduction}
%\label{intro}
%Your text comes here. Separate text sections with
%\section{Section title}
%\label{sec:1}
%Text with citations \cite{RefB} and \cite{RefJ}.
%\subsection{Subsection title}
%\label{sec:2}
%as required. Don't forget to give each section
%and subsection a unique label (see Sect.~Sec.~\ref{sec:1}).
%\paragraph{Paragraph headings} Use paragraph headings as needed.
%\begin{equation}
%a^2+b^2=c^2
%\end{equation}
%
%% For one-column wide figures use
%\begin{figure}
%% Use the relevant command to insert your figure file.
%% For example, with the graphicx package use
  %\includegraphics{example.eps}
%% figure caption is below the figure
%\caption{Please write your figure caption here}
%\label{fig:1}       % Give a unique label
%\end{figure}
%%
%% For two-column wide figures use
%\begin{figure*}
%% Use the relevant command to insert your figure file.
%% For example, with the graphicx package use
  %\includegraphics[width=0.75\textwidth]{example.eps}
%% figure caption is below the figure
%\caption{Please write your figure caption here}
%\label{fig:2}       % Give a unique label
%\end{figure*}
%%
%% For tables use
%\begin{table}
%% table caption is above the table
%\caption{Please write your table caption here}
%\label{tab:1}       % Give a unique label
%% For LaTeX tables use
%\begin{tabular}{lll}
%\hline\noalign{\smallskip}
%first & second & third  \\
%\noalign{\smallskip}\hline\noalign{\smallskip}
%number & number & number \\
%number & number & number \\
%\noalign{\smallskip}\hline
%\end{tabular}
%\end{table}


%\begin{acknowledgements}
%If you'd like to thank anyone, place your comments here
%and remove the percent signs.
%\end{acknowledgements}

% BibTeX users please use one of
%\bibliographystyle{spbasic}      % basic style, author-year citations
\bibliographystyle{spmpsci}      % mathematics and physical sciences
%\bibliographystyle{spphys}       % APS-like style for physics
\bibliography{literatur}   % name your BibTeX data base

% Non-BibTeX users please use
%\begin{thebibliography}{}
%%
%% and use \bibitem to create references. Consult the Instructions
%% for authors for reference list style.
%%
%\bibitem{RefJ}
%% Format for Journal Reference
%Author, Article title, Journal, Volume, page numbers (year)
%% Format for books
%\bibitem{RefB}
%Author, Book title, page numbers. Publisher, place (year)
%% etc
%\end{thebibliography}

\end{document}
% end of file template.tex

