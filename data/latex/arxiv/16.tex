\section{Conclusion} \label{sec:conclusion}

We proposed an effective working process to create a low cost classroom response system, based on recognizing fiducial markers as response devices. In order to release the solution, we have solved several technical issues due to reliability of fiducial markers detection in our application context, and usability challenges. The solution is available in both binary and source-code, as we hope it will be useful for other research teams interested in the subject.

Making available a low-cost educational CRSs is only part of what is needed to bring active learning to disfavored communities. It is critical to address teachers' and instructors' concerns related to adopting not only the tecnology, but mainly new teaching methods, especially when that represents leaving behind the safety, predictability and control of a lecture classroom setup~\cite{beatty2005transforming}. Our current work, thus, focus on creating pedagogical material for paperclickers, considering its usage by Brazilian teachers of Mathematics and Physics, in real Peer Instruction scenarios. We will also evaluate the user experience on the released paperclickers solution, including a diverse audience.

% We have verified our solution on users with similar classroom experience --- university STEM students; we understand testers with different backgrounds should be considered in order to better explore how our solution would perform on a broader audience.


