
\section{Towards Spectrum Reduction with Frequency Reuse} \label{sec:towards_spectrum_reduction}
In this paper, we use the sum spectrum to describe how many resources are needed to
serve all users' traffic demands in cellular networks.
This may not directly reflect the total required spectrum for cellular operators, because
the same spectrum can be spatially reused by multiple BSs who are sufficiently far away from each other.
The benefit of spectrum spatial reuse is characterized by the
frequency reuse factor $K$, which represents the proportion of the total
spectrum that one cell can utilize. For instance, $K=1$ means that any cell
can use all spectrum, and $K=1/7$ means that one cell can only utilize $1/7$ of the total spectrum,
to avoid excessive interference among adjacent cells.
A \emph{back-of-the-envelope} calculation suggests that, if the total number of required channels for all $N$  BSs is $C$,
then $\frac{C/N}{K}$ distinct radio channels are needed to serve the entire cellular
network.

In the case without D2D, the sum spectrum of all BSs is $F^{\textsf{ND}}$, which
corresponds to the total number of channels for all cells. Thus, with frequency reuse
factor $K$, $\frac{F^{\textsf{ND}}}{NK}$ distinct channels are needed without D2D.

In the case with D2D, D2D communication can degrade the original frequency reuse pattern
if they are sharing the same spectrum with cellular users (which is called underlay D2D \cite{Doppler09}).
Given the new frequency reuse factor $K^{\textsf{D2D}} (\le K)$. A back-of-the-envelope analysis suggests that
$\frac{F^{\textsf{D2D}}}{NK^{\textsf{D2D}}}$ distinct radio channels are needed with D2D load balancing. Consequently,
the spectrum reduction can be estimated as
\be
\frac{\frac{F^{\textsf{ND}}}{NK} - \frac{F^{\textsf{D2D}}}{NK^{\textsf{D2D}}}}{\frac{F^{\textsf{ND}}}{NK}}
= 1- \frac{K}{K^{\textsf{D2D}}} \times \frac{F^{\textsf{D2D}}}{F^{\textsf{ND}}}
= 1- \frac{K}{K^{\textsf{D2D}}} (1-\rho).
\label{equ:relation_spectrum_reduction}
\ee
Eq. \eqref{equ:relation_spectrum_reduction} suggests that our calculation of $\rho$ without frequency reuse
gives us a first-order understanding of how much spectrum reduction can be achieved by D2D load balancing with frequency reuse.
