\setcounter{equation}{0}
\setcounter{figure}{0}
\section{Dynamical Coupled Channels Calculations}
\label{sec:DCC}
As highlighted in Sec.\,\ref{Experiment}, the last twenty years have seen an explosion in the amount of available data on resonance photo- and electroproduction, \emph{e.g}.\ the reactions $\gamma^{(\ast)} N \to \pi N$ and $\gamma^{(\ast)} N \to \pi \pi N$, which are particularly relevant to discussions of the Roper resonance.  As the data accumulated, so grew an appreciation of the need for a sound theoretical analysis which unified all its reliable elements.  At the beginning of 2006, this culminated with establishment of the Excited Baryon Analysis Center [EBAC] at JLab \cite{Lee:2007zy, Kamano:2011ut, Lee:2012mea}, whose primary goals were: to perform a dynamical coupled-channels [DCC] analysis of the world's data on meson production reactions from the nucleon in order to determine the meson-baryon partial-wave amplitudes; and identify and characterise all nucleon resonances that contribute to these reactions.

In contrast to the familiar and commonly used partial wave analyses, which are model-independent to some extent, but also, therefore, limited in the amount of information they can provide about resonance structure, modern DCC analyses are formulated via a Hamiltonian approach to multichannel reactions \cite{JuliaDiaz:2007kz, Suzuki:2009nj, Kamano:2010ud, Ronchen:2012eg, Kamano:2013iva}.  The Hamiltonian expresses model assumptions, \emph{e.g}.\ statements about the masses of bare/undressed baryons [in the sense of particle versus quasi-particle] and the dominant meson-baryon reaction channels that transform the bare baryon into the observed quasi-particle.  Naturally, such assumptions can be wrong.  Equally: the models are flexible; they can be falsified and thereby improved, given the vast amount of existing data; and, used judiciously, they can be provide a critical bridge between data and QCD-connected approaches to the computation of baryon properties.

%% provided by VM
%{\bf The familiar and commonly used partial wave analyses, are model-independent, but also, therefore, limited in the amount of information they can provide about resonance parameters. All PWA approaches employed in the experimental data analyses up to date do require the modeling of non-resonant contributions in order to get N$^*$ electromagnetic and hadronic couplings. Furthermore, most of them incorporate the hadronic interaction with the N$\pi\pi$ final states very approximately in a model dependent way. However, interactions with these final states provide strong impact on reaction amplitudes of most exclusive meson photo-/electroproduction channels in the N$^*$ region}
%{\bf Modern DCC analysis recently developed by Argonne-Osaka group is formulated via a Hamiltonian approach to multichannel reactions.  The Hamiltonian expresses model assumptions, \emph{e.g}.\ statements about the masses of bare/undressed baryons (in the sense of particle versus quasiparticle) and the dominant meson-baryon reaction channels that transform the bare baryon into the observed quasiparticle.
%{\bf This approach for the first time account consistently for hadronic interaction with N$\pi\pi$ final states and employs the restrictions imposed by not only two- but also by the three -body unitarity condition} {\bf Naturally, such assumptions should be tested against the experimental data on a large body of photo- electro- and hadroproduction reactions for all final hadron states included to the coupled channel models. New data on single- and multi-meson photoproduction from MAMI, ELSA, CLAS, electroproduction data from CLAS and upcoming hadroproduction data from JPARC offer new opportunities to check the coupled channel model assumptions. Furthermore, Argonne-Osaka approach disentangling the bare quark core and meson-baryon cloud contributions to the electromagnetic excitation and hadronic decay resonance vertices offer valuable information on the N$^*$ structure directly from the analyses of the measured observables}.

The EBAC approach,\footnote{The EBAC projected terminated in 2012, but the effort is continuing as part of the  Argonne-Osaka collaboration, from which it initially grew \cite{Sato:1996gk, Matsuyama:2006rp}.} for instance, describes meson-baryon ($MB$) reactions involving the following channels: $\pi N$, $\eta N$ and $\pi\pi N$, the last of which has $\pi \Delta$, $\rho N$ and $\sigma N$ resonant components.  The excitation of the internal structure of a given initial-state baryon ($B$) by a meson ($M$) to produce a bare nucleon resonance, $\bar N^\ast$, is implemented by an interaction vertex, $\Gamma_{MB\to \bar N^\ast}$.  Importantly, the Hamiltonian also contains energy-independent meson-exchange terms, $v_{MB,M^\prime B^\prime}$, deduced from widely-used meson-exchange models of $\pi N$ and $NN$ scattering.
%% See Appendix A of Matsuyama:2006rp

In such an approach, the features of a given partial wave amplitude may be connected with dressing of the bare resonances included in the Hamiltonian ($\bar N^\ast$), in which case the resulting $N^\ast$ states are considered to be true resonance excitations of the initial state baryon.  On the other hand, they can also be generated by attraction produced by the $v_{MB,M^\prime B^\prime}$ interaction and channel-coupling effects, in which case they are commonly described as ``molecular states'' so as to differentiate them from true resonance excitations.  The need to reliably distinguish between these two different types of systems in the solution of the coupled channels problem defined by the model Hamiltonian requires that the form and features of $v_{MB,M^\prime B^\prime}$ must be very carefully constrained by,  \emph{e.g}.\ elastic scattering data, throughout the region of relevance to the resonance production reactions.

\begin{figure}[t]
\centerline{\includegraphics[width=0.4\textwidth]{F9_IV1.pdf}}
\caption{\label{EBACRoper}
Open circle [black]: mass of the bare Roper state determined in the EBAC DCC analysis of $\pi N$ scattering \cite{JuliaDiaz:2007kz, Suzuki:2009nj, Kamano:2010ud}.
%
This bare Roper state, with full spectral weight at mass $1.763\,$GeV, splits and evolves following the inclusion of meson-baryon final-state interactions, with the trajectories in this complex-energy plane depicting the motion of the three, distinct daughter poles as the magnitude of those interactions is increased from zero to their full strength.
%
The horizontal dashed lines [black] mark the branch cuts associated with all thresholds relevant to the solution of the DCC scattering problem in this channel.
%
Filled star [green]: mass of the dressed-quark core of the proton's first radial excitation predicted by a three valence-quark Faddeev equation \cite{Segovia:2015hra}.
}
\end{figure}

Being aware of the challenges associated with understanding the Roper resonance, the EBAC collaboration made a determined effort to produce a sound description of the spectrum of baryon resonances with masses below 2\,GeV using their DCC model.  Refining this tool by developing an excellent description of 22\,348 independent data points, representing the complete array of partial waves, they arrived at some very striking conclusions \cite{JuliaDiaz:2007kz, Suzuki:2009nj, Kamano:2010ud}, illustrated in Fig.\,\ref{EBACRoper}:\\[-4ex]
%
\begin{itemize}
\setlength\itemsep{0em}
\item From a bare state with mass $1.763\,$GeV, three distinct features appear in the $P_{11}$ partial wave, as described by Fig.\,\ref{EBACRoper}.  [We will subsequently return to the interpretation of the bare state.]
%
\item Of the three spectral features that emerge in this channel, two are associated with the Roper resonance.  [This two-pole character of the Roper is common to many analyses of the scattering data, including one involving Roper himself \cite{Arndt:1985vj} and more recent analyses of $\pi N$ scattering data \cite{Cutkosky:1990zh, Arndt:2006bf, Doring:2009yv}.]
%
\item The third pole is located farther from the origin [position C in Fig.\,\ref{EBACRoper}] and might plausibly be associated with the $N(1710)\,1/2^+$ state listed by the Particle Data Group \cite{Olive:2016xmw}.
%
%\item This same analysis finds a bare state with mass $1.800\,$GeV associated with the $N(1535)\,1/2^-$ and a bare state with mass $1.391\,$GeV \cite{JuliaDiaz:2007kz} associated with the $\Delta(1232)\,3/2^+$.
%It is worth noting here that the same EBAC DCC analysis finds a bare state with mass $1.800\,$GeV associated with the $N(1535)\,1/2^-$ and a bare state with mass $1.391\,$GeV \cite{JuliaDiaz:2007kz} associated with the $\Delta(1232)\,3/2^+$.

\end{itemize}
[\emph{N.B}.\ The same EBAC DCC analysis identifies a bare state with mass $1.800\,$GeV as the origin of the $N(1535)\,1/2^-$ and a bare state with mass $1.391\,$GeV associated with the $\Delta(1232)\,3/2^+$ \cite{JuliaDiaz:2007kz}.]

Evidently, as emphasized by the trajectories in Fig.\,\ref{EBACRoper}, the coupling between channels required to simultaneously describe all partial waves has an extraordinary effect with, \emph{e.g}.\ numerous spectral features in the $P_{11}$ channel evolving from a single bare state, expressed as a pole on the real axis, through its coupling to the $\pi N$, $\eta N$ and $\pi\pi N$ reaction channels.  It follows that no analysis of one partial wave in isolation can reasonably be claimed to provide an understanding of such a complex array of emergent features.

