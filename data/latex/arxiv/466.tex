\subsection{Detection with estimated structure of covariance matrix $\boldsymbol{C}$}
As mentioned in the previous subsection,
the likelihood ratio (and its statistic) in (\ref{glrt_equi})
remains unchanged if we substitute
$\boldsymbol{C}$ by $\boldsymbol{M}$.
Hence, instead of estimating $\boldsymbol{C}$, we employ
%could be substitute by
an estimate of $\boldsymbol{M}$.
The MLE of $\boldsymbol{M}$ %in case of deterministic \textit{texture} component 
has been proved to uniquely exist and derived in
\cite{Pascal08}, in which
MLE of $\boldsymbol{M}$
satisfies the equation
\begin{equation}\label{MLE_compoundG}
     \boldsymbol{M}_{MLE} \thinspace = \thinspace f(\boldsymbol{M}_{MLE}),
\end{equation}
where % $f(\boldsymbol{M}_{MLE})$ is given by
\begin{equation}\label{MLE_func}
  f(\boldsymbol{M}_{MLE}) \thinspace = \thinspace
  \frac{N}{K} \sum_{t=1}^{K} \frac{ \boldsymbol{c}_t \boldsymbol{c}_t^H}
                                  {\boldsymbol{c}_t^H \boldsymbol{M}_{MLE}^{-1}\boldsymbol{c}_t}.
  \end{equation}
  Solution for the above equation uniquely exists but
a closed-form for such solution does not exist \cite{Pascal08}.
Instead, MLE of $\boldsymbol{M}$ is computed by recursion computing \cite{Pascal08},
which
is employed in this paper.
%From \cite{Pascal08}, the MLE of $\boldsymbol{M}$ satisfies

Now, we replace $\boldsymbol{C}$ in (\ref{glrt_equi}) by the MLE of $\boldsymbol{M}$
and assess the CFAR property of the resulting detector,
 called $\theta-$MLE detector, employing the likelihood ratio
\begin{equation}\label{proposed_lrt}
  \frac{1}
     {  \boldsymbol{z}^H \widehat{\boldsymbol{M}}^{-1} \boldsymbol{z} } \hat{t}^{\ast}
   \thinspace\mathop{\gtrless}_{H_0}^{H_1}\thinspace G,
\end{equation}
where
$\hat{t}^\ast$ is the optimal value attained from (\ref{SDP_detector}), in which
$\boldsymbol{y}$ and $\boldsymbol{x}$ are computed with
$\boldsymbol{M}_{MLE}$.
It is easily to see that the
%$\theta-$NSCM and
$\theta-$MLE
detector have CFAR w.r.t \textit{texture} components
$\left\{s_0,s_1,s_2,\ldots,s_K\right\}$. This claim is easily proved based on the
following arguments.
Firstly, notice that
the MLE of $\boldsymbol{M}$ can be derived based on the relation % rewritten in the form
\begin{equation}\label{equi_mle}
  \boldsymbol{M}_{MLE} \thinspace = \thinspace
  \frac{N}{K} \sum_{t=1}^{K} \frac{ \boldsymbol{g}_t \boldsymbol{g}_t^H}
                                  {\boldsymbol{g}_t^H \boldsymbol{M}_{MLE}^{-1}\boldsymbol{g}_t},
\end{equation}
 which
 is independent of $\left \{s_0,s_1,s_2,\ldots,s_K \right \}$.
 In addition, the \textit{texture} component $s_0$ embedded in $\boldsymbol{z}$
 has been cancelled out in the numerator and denominator of
   $
 \dfrac{\left | \boldsymbol{z}^H  \boldsymbol{M}_{MLE} ^{-1}\boldsymbol{p} \right |^2}
     { \left (\boldsymbol{z}^H \boldsymbol{M}_{MLE}^{-1} \boldsymbol{z} \right )
   \left (\boldsymbol{p}^H \boldsymbol{M}_{MLE}^{-1} \boldsymbol{p} \right )}
 $.
 Hence,
  the likelihood ratio (\ref{proposed_lrt}) is independent of $\left \{s_0,s_1,s_2,\ldots,s_K\right \}$.
 Regarding the CFAR property w.r.t $\boldsymbol{M}$,
 it is very hard to analyze
 the dependence of  the false alarm rate of $\theta$-MLE detector on $\boldsymbol{M}$,
 %since the likelihood ratio here employed $\boldsymbol{M}_{MLE}$ which depends not only
 %on $\boldsymbol{g}_t$ but also on $\boldsymbol{M}_{MLE}^{-1}$.
 such dependence will be numerically analyzed. 