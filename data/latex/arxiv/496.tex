%%%%%%%%%%%%%%%%%%%%%%%%%%%%%%%%%%%%%%%%%%%%%%%%%
% topology subsection 2
%%%%%%%%%%%%%%%%%%%%%%%%%%%%%%%%%%%%%%%%%%%%%%%%%
\subsection{The Simplicial Complex of %Non-Crossing Edge Sets
Plane Graphs} 
\label{sec:tcomplex}

%\comment{
In this section and the following one, we give a proof of Theorem~\ref{thm:flip-complex}.
This section is about the simplicial complex 
$\tcomplex=\tcomplex(P)$ whose faces are the sets of pairwise non-crossing edges (line segments) spanned by $P$.
%}

Let $P$ be a set of $n$ points in general position in the plane.
Let $E$ be the set of edges (closed line segments) spanned by $P$.
Two edges $e,f\in E$ are said to be \emph{non-crossing} if they
are disjoint or if they intersect in a single point of $P$ that is an endpoint of both
edges. We say that a subset $F\subseteq E$ is \emph{non-crossing}
if every pair of distinct edges $e,f\in F$ is non-crossing. If $G$ is non-crossing and $F\subseteq G$ then $F$ is non-crossing as well. Thus, the non-crossing sets
of edges form an abstract simplicial complex

$$\tcomplex=\tcomplex(P):=\{F \colon F\subseteq E, F \textrm{ non-crossing}\},$$
which we call the \emph{complex of %non-crossing edge sets
plane graphs on $P$}. 
We collect some basic properties of $\tcomplex$:

\begin{enumerate}
\item The \emph{facets} (inclusion-maximal faces) of $\tcomplex$ are exactly the triangulations of $P$
(every non-crossing set of edges $F\subseteq E$ can be extended to a 
triangulation). Thus, the simplicial complex $\tcomplex$ is of dimension $m-1$,
where $m$ is the number of edges in any triangulation of $P$, %\note{
and it is \emph{pure}, i.e., every face of $\tcomplex$ is
contained in a face of dimension $m-1$.%}
\item Every face $F$ of $\tcomplex$ of dimension $m-2$ is contained in either one or two triangulations. In the latter case, $F$ corresponds to a flip between these two triangulations.
\end{enumerate}

%\note{
We will show that the topology of $\tcomplex$ is particularly simple, namely that $\tcomplex$ 
is a homeomorphic to an $(m-1)$-dimensional ball. Furthermore, there is a combinatorial certificate 
(\emph{shellability}) for this homeomorphism. This implies that the homeomorphism is particularly nice 
and that $\tcomplex$ is a \emph{piecewise-linear ball}. We refer to  \cite{Hudson:Piecewise-linear-topology-1969} and \cite[Appendix~4.7]{Bjorner:Oriented-matroids-1999} for more details and further references on shellability and piecewise-linear balls, spheres, and manifolds.
In this extended abstract, we will leave the notion of piecewise-linearity undefined---the only property that we will need is that it ensures that the construction of the \emph{dual cell complex} $\tcomplex^*$ (see Proposition~\ref{prop-dual} below) is well-behaved.
%}

We recall that a pure $d$-dimensional simplicial complex is \emph{shellable} if 
there exists a total ordering of its facets $F_{1},F_{2},\cdots,F_{N}$ (called a \emph{shelling order})
such that, for every $2\leq j\leq N$, the intersection of $F_j$ with the simplicial complex 
generated by the preceding facets\footnote{More formally, for any set $F$, let $2^F$ denote the simplicial complex of all subsets of $F$. Then the requirement for a shelling is that, for $2\leq j\leq N$, the intersection of the complexes $2^{F_j}$ and $\bigcup_{i<j}2^{F_i}$ be pure of dimension $d-1$.} is pure of dimension $d-1$.




We will need the following result (which appears implicitly in \cite{Bing:Some-aspects-of-the-topology-of-3-manifolds-1964}, and explicitly in \cite{danaraj1974shellings}; see \cite[Prop.~4.7.22]{Bjorner:Oriented-matroids-1999} for a short proof):
\begin{proposition}
\label{prop:shellable}
%\note{
Suppose $\mathbbm{K}$ is a finite $d$-dimensional simplicial complex that is a pseudomanifold, i.e., $\mathbbm{K}$ is pure and every $(d-1)$-dimensional face of $\mathbbm{K}$ is contained in at most two $d$-faces. 
If $\mathbbm{K}$ is shellable then $\mathbbm{K}$ is either a piecewise-linear ball or a piecewise-linear sphere. The former case occurs iff 
there is at least one $(d-1)$-dimensional face that is contained in only one $d$-face of $\mathbbm{K}$.\footnote{We remark that the property of being a shellable pseudomanifold (which is a combinatorial and algorithmically verifiable condition) is strictly stronger than being a piecewise-linear ball or sphere, which in turn is strictly stronger than being a simplicial complex homeomorphic to a ball or sphere.}
%}
\end{proposition}

\begin{theorem}
\label{Ball theorem} 
\label{ball thm}
%\note{
$\tcomplex$ is \emph{shellable}, and hence a piecewise-linear $(m-1)$-dimensional ball.%}
\end{theorem}

\begin{proof}
%\note{
We observed earlier that $\tcomplex$ is a pure $(m-1)$-dimensional simplicial complex,
and that every $(m-2)$-dimensional face of $\tcomplex$ is contained in at most two $(m-1)$-dimensional faces, hence $\tcomplex$ is a pseudomanifold. Moreover, if $T$ is a triangulation of $P$ and if $e\in T$ is a non-flippable edge
(e.g., if $e$ is a convex hull edge) then $F:=T\setminus\{e\}$ is an $(m-2)$-dimensional face of $\tcomplex$ that is contained in a unique $(m-1)$-face, namely $T$.%}

Thus, by Proposition~\ref{prop:shellable}, it suffices to show that $\tcomplex$ is shellable, i.e., to exhibit a shelling order for the facets of $\tcomplex$. 

With every triangulation $T$ of $P$, we associate the sorted vector of angles $\alpha(T)=(\alpha_{1}(T),\alpha_{2}(T),\cdots,\alpha_{3t}(T))$, where $\alpha_{1}(T) \leq \alpha_{2}(T) \leq \cdots \leq\alpha_{3t}(T)$ are the angles occurring in the triangulation $T$. We order the triangulations of $P$ 
by sorting the corresponding angle vectors $\alpha(T)$ lexicographically from largest to smallest; if the point set is in general position, this defines a total ordering
%\comment{
\begin{equation}
\label{eq:angle-ordering}
T_1,T_2,\dots,T_N,\qquad \alpha(T_1)>_{\textrm{LEX}}\alpha(T_2)>_{\textrm{LEX}}\dots>_{\textrm{LEX}}\alpha(T_N),
\end{equation}
%}
where $N$ is the number of triangulations of $P$.

It is well known (see, for example, \cite[Chap.~3.4]{devadoss2011discrete})
that in this ordering, $T_{1}$ is the Delaunay triangulation of
$P$. Moreover, if we consider only triangulations containing a particular
plane subgraph corresponding to a face $F$ of $\tcomplex$ 
and the corresponding subsequence of the angle vectors,
the first of these vectors corresponds to the Delaunay triangulation
constrained to $F$. 

We claim that the triangulation ordering \eqref{eq:angle-ordering} defines a
shelling.  For this, we need to prove that the following holds for $2\leq j\leq N$:
If $F$ is a face of $\tcomplex$ that is contained in $T_j\cap T_i$ for some $i<j$,
then there exists an $(m-2)$-dimensional face $G$ of $T_j$ and some $i'<j$
such that $F\subseteq G=T_{i'}\cap T_j$.

To see this, consider the subsequence 
$T_{k_1},T_{k_2},\dots $ of the sequence \eqref{eq:angle-ordering} consisting only of those triangulations that contain the edge set $F$. Then $T_{k_1}$ is the constrained Delaunay 
triangulation with respect to the edge set $F$, and $T_i$ and $T_j$ both appear in that
subsequence; in particular, $T_j\neq T_{k_1}$ since $T_i$ precedes it. Since every triangulation
containing $F$ can be transformed to the constrained Delaunay triangulation $T_{k_1}$,
(see, e.g., the description of the Lawson flip algorithm in \cite{devadoss2011discrete})
there must exist an edge $e\in T_{j}\setminus T_{k_1}$ such that flipping $e$ (a Lawson flip)
increases the angle vector; thus, the triangulation resulting from flipping $e$ is some
$T_k$ with $k<j$ and satisfies $F\subseteq T_k\cap T_j$ as desired.
\end{proof}

%\comment{
Finally, we need a characterization of interior versus boundary faces of  $\tcomplex$.
%}
Let $\mathbbm{B}$ be a piecewise-linear ball of dimension $d$. By definition, the \emph{boundary} $\partial\mathbbm{B}$ of $\mathbbm{B}$ is the subcomplex of $\mathbbm{B}$ consisting of all faces $F$ for which there exists a $(d-1)$-dimensional face $G$ of 
$\mathbbm{B}$, with $F\subseteq G$, such that $G$ is contained in a unique $d$-dimensional face of $\mathbbm{B}$. (In the case $\mathbbm{B}=\tcomplex$, the latter condition means that $G=T\setminus \{e\}$ for some triangulation $T$ and some edge $e\in T$ that is not flippable.)
A face $F$ of $\mathbbm{B}$ that does not lie in $\partial \mathbbm{B}$ is called an \emph{interior face}.

%\comment{
For the proof of Theorem~\ref{thm:flip-complex} we need properties of interior faces of $\tcomplex$ of dimensions $m-1$, $m-2$ and $m-3$.  The following proposition characterizes interior faces more generally.
%} 


\begin{proposition} 
\label{prop:interior-faces}
Let $\tcomplex$ be the simplicial complex of plane graphs on the
point set $P$. A non-crossing set of edges $F$ on $P$ is an interior face of
$\tcomplex$ if and only if the following conditions hold:

(i) $F$ contains all convex hull edges of $P$,

(ii) Every bounded region in the complement of the plane graph $\left(P,F\right)$ is convex.
\end{proposition}

\begin{proof}
Note that a polygon is non-convex iff it has a reflex vertex.  More generally,
a bounded region in the complement of the plane graph $\left(P,F\right)$ is non-convex iff there is an interior point $p$ of $P$ and a half-plane $H$ through $p$ with no edge of $F$ from $p$ to a point interior to $H$---in this case we say that $p$ ``has no edge in a half-plane''.  
The statement of the proposition is then equivalent to the following:
$F$ is a boundary face if and only if $F$ misses a convex hull edge or there is an interior point $p$ of $P$
with no edge in a half-plane.  We prove this statement.  

For the forward direction, 
suppose that $F$ is a boundary face.  Then there is a triangulation $T$, $F \subseteq T$, and an edge $e \in T - F$ such that $e$ is not flippable in $T$.  If $e$ is a convex hull edge, then $F$ does not contain all convex hull edges.  Otherwise $e$ is a diagonal of a non-convex quadrilateral in $T$.  Set $p$ to be the reflex vertex of the non-convex quadrilateral and $H$ to contain the other end of $e$ but not the two other vertices of the quadrilateral.  Then $p$ has no edge in half-plane $H$.

For the other direction, first note that if 
$F$ misses a convex hull edge then $F$ is a boundary face.  For the other case, suppose there is a non-convex hull point $p$ of $P$ that has no edge in half-plane $H$.  
Augment $F$ to a maximal set $F'$ of non-crossing edges without using any edge from $p$ into $H$.  This will not yet be a triangulation (because in a triangulation $p$ is surrounded by triangles and they have angles bounded by $\pi$).  Now augment further to a triangulation $T$.  Then $T - F'$ contains some edge $e$ incident to $p$, and $e$ is not flippable otherwise we could have further augmented $F'$. Thus $F$ is a boundary face.
\end{proof}



