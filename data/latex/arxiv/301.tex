\section{Characterizing $n$ with exponential orthomorphisms}
\label{sec:expreduce}
In this section our aim is to show that
if $\sigma$ is an exponential orthomorphism modulo $n$,
then $n$ has the form described in Theorem~\ref{thm:whichexp}.

Fix $n \ge 3$ an integer
and $\sigma$ an exponential orthomorphism on $\{1, \dots, n-1\}$.

\begin{proposition}
	If $n$ is not squarefree, then $n = 4$.
\end{proposition}
\begin{proof}
	As before we note that
	\[ R_n(x^e) \ge R_n(x) \]
	for each $x \in \ZZ/n$ and $e \in \ZZ_{>0}$.
	In particular, $R_n(x^{\sigma(x)}) \ge R_n(x)$.
	Again since $x^{\sigma(x)}$ and $x$ are supposed to
	be permutations of each other
	we must have $R_n(x^{\sigma(x)}) = R_n(x)$ for each $x$.

	Now suppose $p$ is a prime with $p^2$ dividing $n$.
	Let $x$ be any element of $\ZZ/n$
	for which $\gcd(x, n) = p$.
	Then $G(x^e) > G(x)$ whenever $e > 1$, forcing $\sigma(x) = 1$.

	In particular $\sigma(p) = \sigma(n-p) = 1$.
	This is only possible if $p = n-p$, or $n = 2p$.
	Since we assumed $p^2 \mid n$, this means $p=2$ and $n=4$.
\end{proof}

Thus, we henceforth assume $n$ is a product of distinct primes.
\begin{proposition}
	If $n$ is squarefree,
	then it is either prime, or twice a prime.
 \end{proposition}
\begin{proof}
	First, suppose $n = p_1 p_2 \dots p_r$ is odd,
	where $p_1 < p_2 < \dots < p_r$ are different primes.
	We observe that if $r > 1$ we have
	\[ \prod_i \left( \frac{p_i+1}{2} \right) - 1 < \frac{n-1}{2}. \]
	(Indeed, we note that $\frac{p_1+1}{2} \cdot \frac{p_1+1}{2} < \frac12 p_1p_2$
	rearranges to $(p_1-1)(p_2-1) > 2$,
	and then simply use $\frac{p_i+1}{2} \le p_i$ for $i \ge 3$.)

	But the left-hand side is the number of nonzero quadratic residues
	in $\ZZ/n$ while the right-hand is the number of even elements
	in $\{1, \dots, n-1\}$.
	This is a contradiction since whenever $\sigma(x)$ is even
	the number $x^{\sigma(x)}$ should be a quadratic residue.

	In exactly the same way, if $n = 2 p_1 \cdots p_r$ is even
	and $r > 1$ then we obtain
	\[ 2\prod_i \left( \frac{p_i+1}{2} \right) - 1 < \frac n2 \]
	which is a contradiction in the same way.
\end{proof}

We now handle the prime case.
\begin{proposition}
	The number $n$ cannot be prime unless $n = 3$.
	\label{prop:primroot}
\end{proposition}
\begin{proof}
	Fix an isomorphism $\theta : (\ZZ/n)^\times \to \ZZ/(n-1)$
	given by taking a primitive root of $\ZZ/n$.
	This gives us a diagram
	\begin{center}
	\begin{tikzcd}
		(\ZZ/n)^\times \ar[d, "\theta"] \ar[r, "\sigma"] & \{1, \dots, n-1\} \\
		\ZZ/(n-1) \ar[ru, dashed, swap, "\tilde{\sigma}"] & 
	\end{tikzcd}
	\end{center}
	where we have a natural map
	$\tilde{\sigma} : \ZZ/(n-1) \to \{1, \dots, n-1\}$
	which makes the diagram commute.

	Obviously $\sigma(1) = n-1$,
	since otherwise $1 = 1^{\sigma(1)} = (\sigma\inv(n-1))^{n-1}$.
	Consequently, $\tilde{\sigma}(0) = 0$.
	Looking at the remaining elements,
	$\tilde{\sigma}$ induces a multiplicative orthomorphism on $\ZZ/(n-1)$,
	which we know is only possible if $n-1=2$.
	Hence we conclude $n = 3$.
\end{proof}

Thus we may henceforth assume that $n = 2p$, where $p$ is prime.
We may as well assume $p$ is odd.
Then in $\ZZ/(2p)$ there are three types of nonzero elements:
\begin{itemize}
	\ii The odd numbers
	$O = \{1, 3, \dots, p-1, p+1, \dots, 2p-1\}$
	(of rank $1$).
	These remain odd under exponentiation,
	and as a multiplicative group is
	isomorphic $(\ZZ/2p)^\times \cong (\ZZ/p)^\times \cong \ZZ/(p-1)$.
	\ii The even numbers
	$E = \{2, \dots, 2p-2 \}$
	(of rank $2$).
	These remain even under exponentiation,
	and as a multiplicative group is isomorphic
	$(\ZZ/p)^\times$ as well.
	\ii The special element $p$ (of rank $p$),
	for which $p^c \equiv p \pmod{2p}$ for any $c \in \ZZ$.
\end{itemize}
As all the elements above have order dividing $p-1$,
we may consider the image of $\sigma$ modulo $p-1$
to obtain the multiset
\[ S = \left\{ 1,1,1,2,2,3,3,\dots,p-1,p-1 \right\} \]
of size $n-1 = 2p-1$.
In other words, we may instead consider
$\sigma : \{1, \dots, n-1\} \to S$.
Thus, for $k = 1, \dots, p-1$ viewed as elements of $(\ZZ/p)^\times$,
we define
\begin{align*}
	a_k &= \begin{cases}
		\sigma(2k-1) & k \le \frac{p-1}{2} \\
		\sigma(2k+1) & k \ge \frac{p+1}{2}
	\end{cases} \\
	b_k &= \sigma(2k) \\
	c &= \sigma(p).
\end{align*}
Diagramatically, we have drawn the diagram
\begin{center}
\begin{tikzcd}
	O \sqcup E \ar[r, "\sigma"] \ar[d, swap, "\simeq"] & S \\
	(\ZZ/p)^\times \sqcup (\ZZ/p)^\times
		\ar[ru, "(a_\bullet{,} b_\bullet)", swap] &
\end{tikzcd}
\end{center}
Thus, we have reformulated the problem as follows:
\begin{proposition}
	Assume $n = 2p$ with $p$ an odd prime.
	Then $n$ satisfies the problem conditions
	if and only if there exists a permutation
	\[ (a_1, \dots, a_{p-1}, b_1, \dots, b_{p-1}, c)
		\quad\text{of}\quad S \]
	such that
	\[ (a_1, 2a_2, \dots, (p-1)a_{p-1})
		\quad\text{and}\quad (b_1, 2b_2, \dots, (p-1)b_{p-1}) \]
	are permutations of $\ZZ/(p-1)$.
\end{proposition}

With this formulation we may now show the following.
\begin{proposition}
	If $n = 2p$ with $p$ prime, then $p-1$ is squarefree.
\end{proposition}
\begin{proof}
	This mirrors the proof of \ref{prop:mult_squarefree},
	with small modifications.
	As before we have
	\begin{align*}
		R_{p-1}\left( ka_k \right)
			&\ge \max \left\{ R_{p-1}(k), R_{p-1}(a_k) \right\}
			\ge R_{p-1}(k) \\
		R_{p-1}\left( kb_k \right)
			&\ge \max \left\{ R_{p-1}(k), R_{p-1}(b_k) \right\}
			\ge R_{p-1}(k).
	\end{align*}
	The change to the argument is that
	$a_k$ and $b_k$ are not collectively a permutation of $S$
	(since there is an extra unused element $c$).
	However, we may still conclude
	(since $ka_k$, $kb_k$ and $k$ are permutations of each other)
	that
	\[ R_{p-1}(ka_k) = R_{p-1}(kb_k) = R_{p-1}(k). \]
	Now suppose $q$ is a prime for which $q^2 \mid p-1$.
	Then as before, whenever the exponent of $q$ in $k$ is at most one,
	we would require $a_k$ and $b_k$ to not be divisible by $q$.
	So among $a_k$ and $b_k$ we need at least
	\[ 2 \cdot \frac{q^2-1}{q^2} (p-1) \]
	values to be not divisible by $q$,
	but in the multiset $S$ the number of such elements is
	\[ 1 + \frac{q-1}{q} \cdot 2(p-1)
		< 2 \cdot \frac{q^2-1}{q^2} (p-1) \]
	which is a contradiction.
\end{proof}

Together these propositions establish that $n$
must have the form described in Theorem~\ref{thm:whichexp}.
