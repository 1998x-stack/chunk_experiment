
\chapter{Proofs for Chapter~\ref{chap:fibadjmodels}}
\label{chap:appendixC4}


\section{Proof of Proposition~\ref{prop:indexedelimcolimits}}
\label{sect:proofofprop:indexedelimcolimits}

{
\renewcommand{\thetheorem}{\ref{prop:indexedelimcolimits}}
\begin{proposition}
Let us assume a full split comprehension category with unit \linebreak $p : \mathcal{V} \longrightarrow \mathcal{B}$ that has split fibred strong colimits of shape $\mathcal{D}$, a diagram of the form \linebreak $J : \mathcal{D} \longrightarrow \mathcal{V}_X$, and an object $A$ in $\mathcal{V}_{\ia {\mathsf{\ul{colim}}(J)}}$. Then, given a family of vertical morphisms $f_{D} : 1_{\ia {J(D)}} \longrightarrow \ia {\mathsf{\ul{in}}^{J}_D} ^*(A)$, for all objects $D$ in $\mathcal{D}$, such that for all morphisms \linebreak$g : D_i \longrightarrow D_{\!j}$ in $\mathcal{D}$ we have $\ia {J(g)} ^*(f_{D_{\!j}}) = f_{D_i}$, there exists a unique vertical morphism $[f_D]_{D \in \mathcal{D}} : 1_{\ia {\mathsf{\ul{colim}}(J)}} \longrightarrow A$ in $\mathcal{V}_{\ia {\mathsf{\ul{colim}}(J)}}$ satisfying the following ``$\beta$-equations":
\[
\ia {\mathsf{\ul{in}}^{J}_{D_i}}^*([f_D]_{D \in \mathcal{D}}) = f_{D_i}  : 1_{\ia {J(D_i)}} \longrightarrow \ia {\mathsf{\ul{in}}^{J}_{D_i}} ^*(A)
\]
for all objects $D_i$ in $\mathcal{D}$.
\end{proposition}
\addtocounter{theorem}{-1}
}

\begin{proof}
In order to give the definition of $[f_D]_{D \in \mathcal{D}}$, we first define an auxiliary ``pairing \linebreak functor"  $\langle \sigma_f \rangle : ({\underset{\circlearrowleft}{0} \overset{s}{\longrightarrow} \underset{\circlearrowleft}{1}}) \longrightarrow \mathsf{lim}(\widehat{J})$, using the universal property of the limit \linebreak $\mathsf{pr}^{\widehat{J}} : \Delta(\mathsf{lim}(\widehat{J})) \longrightarrow \widehat{J}$ for 
a cone $\sigma_f : \Delta(\underset{\circlearrowleft}{0} \overset{s}{\longrightarrow} \underset{\circlearrowleft}{1}) \longrightarrow \widehat{J}$ that is defined as follows: 
\[
(\sigma_f)_D(0) \defeq 1_{\ia {J(D)}}
\qquad
(\sigma_f)_D(1) \defeq \ia {\mathsf{\ul{in}}^{J}_{D}} ^*(A)
\qquad
(\sigma_f)_D(s) \defeq f_D
\]

In detail, $\langle \sigma_f \rangle$ arises as a unique mediating morphism because for all $g : D_i \longrightarrow D_j$ in $\mathcal{D}$, the outer triangle commutes in the following diagram:
\[
\xymatrix@C=1em@R=2em@M=0.5em{
& & \underset{\circlearrowleft}{0} \overset{s}{\longrightarrow} \underset{\circlearrowleft}{1} \ar@/_2pc/[dddll]_-{(\sigma_f)_{D_j}} \ar@/^2pc/[dddrr]^-{(\sigma_f)_{D_i}} \ar@{-->}[dd]^{\langle \sigma_f \rangle}
\\
\\
& & \mathsf{lim}(\widehat{J}) \ar[dl]_{\mathsf{pr}^{\widehat{J}}_{D_j}} \ar[dr]^{\mathsf{pr}^{\widehat{J}}_{D_i}}
\\
\mathcal{V}_{\ia {J(D_j)}} \ar[r]_-{=} & \widehat{J}(D_j) \ar[rr]_{\ia {J(g)} ^*} & & \widehat{J}(D_i) \ar[r]_-{=} & \mathcal{V}_{\ia {J(D_i)}}
}
\]
because our assumptions about the fibration $p$ and the morphisms $f_D$ give us
\[
\begin{array}{c}
\ia {J(g)} ^*(1_{\ia {J(D_j)}}) = 1_{\ia {J(D_i)}}
\\[2mm]
\ia {J(g)} ^*(\ia {\mathsf{\ul{in}}^{J}_{D_j}} ^*(A)) = \ia {\mathsf{\ul{in}}^{J}_{D_i}} ^*(A)
\\[2mm]
\ia {J(g)} ^*(f_{D_j}) = f_{D_i}
\end{array}
\]

Next, our aim is to define $[f_D]_{D \in \mathcal{D}}$ using the fully-faithfulness of $\langle \ia {\mathsf{\ul{in}}^J_D}^* \rangle_{D \in \mathcal{D}}$ on the morphism $\langle \sigma_f \rangle(s)$. However, before we can do so, we have to show that  
\[
\langle \sigma_f \rangle(0) = \langle \ia {\mathsf{\ul{in}}^J_D}^* \rangle_{D \in \mathcal{D}}(1_{\ia {\mathsf{\ul{colim}}(J)}})
\qquad
\langle \sigma_f \rangle(1) = \langle \ia {\mathsf{\ul{in}}^J_D}^* \rangle_{D \in \mathcal{D}}(A)
\]
in order to ensure that $\langle \sigma_f \rangle(s)$ is in 
\[
\mathsf{lim}(\widehat{J})(\langle \ia {\mathsf{\ul{in}}^J_D}^* \rangle_{D \in \mathcal{D}}(1_{\ia {\mathsf{\ul{colim}}(J)}}) , \langle \ia {\mathsf{\ul{in}}^J_D}^* \rangle_{D \in \mathcal{D}}(A))
\]
To this end, we use the universal property of of the limit $\mathsf{pr}^{\widehat{J}} : \Delta(\mathsf{lim}(\widehat{J})) \longrightarrow \widehat{J}$. 

The left-hand equation follows from observing that for all $g : D_i \longrightarrow D_j$ in $\mathcal{D}$, the following diagram commutes:
\[
\xymatrix@C=1em@R=2em@M=0.5em{
& & \mathbf{1} \ar@/_2pc/[dddll]_-{\star \,\mapsto\, 1_{\ia {J(D_j)}}} \ar@/^2pc/[dddrr]^-{\star \,\mapsto\, 1_{\ia {J(D_i)}}} \ar[dd]
\\
\\
& & \mathsf{lim}(\widehat{J}) \ar[dl]_{\mathsf{pr}^{\widehat{J}}_{D_j}} \ar[dr]^{\mathsf{pr}^{\widehat{J}}_{D_i}}
\\
\mathcal{V}_{\ia {J(D_j)}} \ar[r]_-{=} & \widehat{J}(D_j) \ar[rr]_{\ia {J(g)} ^*} & & \widehat{J}(D_i) \ar[r]_-{=} & \mathcal{V}_{\ia {J(D_i)}}
}
\]
when the unlabelled mediating functor $\mathbf{1} \longrightarrow \mathsf{lim}(\widehat{J})$ is given by either $\star \mapsto \langle \sigma_f \rangle(0)$ or $\star \mapsto \langle \ia {\mathsf{\ul{in}}^J_D}^* \rangle_{D \in \mathcal{D}}(1_{\ia {\mathsf{\ul{colim}}(J)}})$. As a result, these functors must be equal to the unique such functor induced by the universal property of $\mathsf{pr}^{\widehat{J}}$ and, therefore, be equal to each other. 

The right-hand equation is proved analogously. In particular, we observe that for all $g : D_i \longrightarrow D_j$ in $\mathcal{D}$, the following diagram commutes:
\[
\xymatrix@C=1em@R=2em@M=0.5em{
& & \mathbf{1} \ar@/_2pc/[dddll]_-{\star \,\mapsto\, \ia {\mathsf{\ul{in}}^{J}_{D_j}} ^*(A)} \ar@/^2pc/[dddrr]^-{\star \,\mapsto\, \ia {\mathsf{\ul{in}}^{J}_{D_i}} ^*(A)} \ar[dd]
\\
\\
& & \mathsf{lim}(\widehat{J}) \ar[dl]_{\mathsf{pr}^{\widehat{J}}_{D_j}} \ar[dr]^{\mathsf{pr}^{\widehat{J}}_{D_i}}
\\
\mathcal{V}_{\ia {J(D_j)}} \ar[r]_-{=} & \widehat{J}(D_j) \ar[rr]_{\ia {J(g)} ^*} & & \widehat{J}(D_i) \ar[r]_-{=} & \mathcal{V}_{\ia {J(D_i)}}
}
\]
when the unlabelled mediating functor $\mathbf{1} \longrightarrow \mathsf{lim}(\widehat{J})$ is given by either $\star \mapsto \langle \sigma_f \rangle(1)$ or $\star \mapsto \langle \ia {\mathsf{\ul{in}}^J_D}^* \rangle_{D \in \mathcal{D}}(A)$. As a result, these functors must be equal to the unique such functor induced by the universal property of $\mathsf{pr}^{\widehat{J}}$ and, therefore, be equal to each other. 

Now, based on the above observations, we can use the fully-faithfulness of \linebreak $\langle \ia {\mathsf{\ul{in}}^J_D}^* \rangle_{D \in \mathcal{D}}$ and define the required vertical morphism $[f_D]_{D \in \mathcal{D}}$ as
\[
[f_D]_{D \in \mathcal{D}} \defeq \langle \ia {\mathsf{\ul{in}}^J_D}^* \rangle_{D \in \mathcal{D}}^{-1}(\langle \sigma_f \rangle(s)) : 1_{\ia {\mathsf{\ul{colim}}(J)}} \longrightarrow A
\]

Finally, we prove that this morphism $[f_D]_{D \in \mathcal{D}}$ satisfies the required ``$\beta$-equations" $\ia {\mathsf{\ul{in}}^J_{D_i}}^*([f_D]_{D \in \mathcal{D}}) = f_{D_i}$, for all $D_i$ in $\mathcal{D}$, and also that it is a unique such morphism. 

First, the ``$\beta$-equations" are proved as follows:
\begin{fleqn}[0.3cm]
\begin{align*}
& \ia {\mathsf{\ul{in}}^J_{D_i}}^*([f_D]_{D \in \mathcal{D}}) 
\\
=\,\, & \ia {\mathsf{\ul{in}}^J_{D_i}}^*(\langle \ia {\mathsf{\ul{in}}^J_D}^* \rangle_{D \in \mathcal{D}}^{-1}(\langle \sigma_f \rangle(s)))
\\
=\,\, & \mathsf{pr}^{\widehat{J}}_{D_i}(\langle \ia {\mathsf{\ul{in}}^J_{D}}^* \rangle_{D \in \mathcal{D}}(\langle \ia {\mathsf{\ul{in}}^J_D}^* \rangle_{D \in \mathcal{D}}^{-1}(\langle \sigma_f \rangle(s))))
\\
=\,\, & \mathsf{pr}^{\widehat{J}}_{D_i}(\langle \sigma_f \rangle(s))
\\
=\,\, & (\sigma_f)_{D_i}(s)
\\
=\,\, & f_{D_i}
\end{align*}
\end{fleqn}
for all objects $D_i$ in $\mathcal{D}$, using the definitions of $\langle \ia {\mathsf{\ul{in}}^J_D}^* \rangle_{D \in \mathcal{D}}$ and $\langle \sigma_f \rangle$, in combination with the fully-faithfulness of the former. 

In order to show that $[f_D]_{D \in \mathcal{D}}$ is the unique such morphism, we assume that there exists another vertical morphism $h : 1_{\ia {\mathsf{\ul{colim}}(J)}} \longrightarrow A$ in $\mathcal{V}_{\ia {\mathsf{\ul{colim}}(J)}}$, satisfying the ``$\beta$-equations" $\ia {\mathsf{\ul{in}}^{J}_{D_i}}^*(h) = f_{D_i}$ for all objects $D_i$ in $\mathcal{D}$. 

Next, we observe that a functor $\widehat{h} : (\underset{\circlearrowleft}{0} \overset{s}{\longrightarrow} \underset{\circlearrowleft}{1}) \longrightarrow \mathsf{lim}(\widehat(J))$, given by
\[
\widehat{h}(0) \defeq \langle \ia {\mathsf{\ul{in}}^J_D}^* \rangle_{D \in \mathcal{D}}(1_{\ia {\mathsf{\ul{colim}}(J)}})
\quad
\widehat{h}(1) \defeq \langle \ia {\mathsf{\ul{in}}^J_D}^* \rangle_{D \in \mathcal{D}}(A)
\quad
\widehat{h}(s) \defeq \langle \ia {\mathsf{\ul{in}}^J_D}^* \rangle_{D \in \mathcal{D}}(h)
\]
makes the following diagram commute:
\[
\xymatrix@C=1em@R=2em@M=0.5em{
& & \underset{\circlearrowleft}{0} \overset{s}{\longrightarrow} \underset{\circlearrowleft}{1} \ar@/_2pc/[dddll]_-{(\sigma_f)_{D_j}} \ar@/^2pc/[dddrr]^-{(\sigma_f)_{D_i}} \ar[dd]^{\widehat{h}}
\\
\\
& & \mathsf{lim}(\widehat{J}) \ar[dl]_{\mathsf{pr}^{\widehat{J}}_{D_j}} \ar[dr]^{\mathsf{pr}^{\widehat{J}}_{D_i}}
\\
\mathcal{V}_{\ia {J(D_j)}} \ar[r]_-{=} & \widehat{J}(D_j) \ar[rr]_{\ia {J(g)} ^*} & & \widehat{J}(D_i) \ar[r]_-{=} & \mathcal{V}_{\ia {J(D_i)}}
}
\]
for all morphisms $g : D_i \longrightarrow D_j$ in $\mathcal{D}$. 

Therefore, $\widehat{h}$ must be equal to the unique such functor, namely, $\langle \sigma_f \rangle$. 
As a result,
\[
\langle \ia {\mathsf{\ul{in}}^J_D}^* \rangle_{D \in \mathcal{D}}(h) = \langle \sigma_f \rangle(s)
\]
from which it follows that 
\begin{fleqn}[0.3cm]
\begin{align*}
[f_D]_{D \in \mathcal{D}} = \langle \ia {\mathsf{\ul{in}}^J_D}^* \rangle_{D \in \mathcal{D}}^{-1}(\langle \sigma_f \rangle(s)) = \langle \ia {\mathsf{\ul{in}}^J_D}^* \rangle_{D \in \mathcal{D}}^{-1}(\langle \ia {\mathsf{\ul{in}}^J_D}^* \rangle_{D \in \mathcal{D}}(h)) = h
\end{align*}
\end{fleqn}
using the definition of $[f_D]_{D \in \mathcal{D}}$ and the fully-faithfulness of $\langle \ia {\mathsf{\ul{in}}^J_D}^* \rangle_{D \in \mathcal{D}}$. 
\end{proof}


\section{Proof of Proposition~\ref{prop:fibredcolimits}}
\label{sect:proofofprop:fibredcolimits}

{
\renewcommand{\thetheorem}{\ref{prop:fibredcolimits}}
\begin{proposition}
Let us assume a full split comprehension category with unit \linebreak $p : \mathcal{V} \longrightarrow \mathcal{B}$ that has split fibred strong colimits of shape $\mathcal{D}$. Then, given a diagram of the form $J : \mathcal{D} \longrightarrow \mathcal{V}_X$, the cocone $\mathsf{\ul{in}}^{J} : J \longrightarrow \Delta(\mathsf{\ul{colim}}(J))$, induced by the existence of split fibred strong colimits of shape $\mathcal{D}$, is a colimit of $J$ in $\mathcal{V}_X$ in the standard sense, i.e., the cocone $\mathsf{\ul{in}}^{J} : J \longrightarrow \Delta(\mathsf{\ul{colim}}(J))$ is initial amongst the cocones over $J$ in $\mathcal{V}_X$.
\end{proposition}
\addtocounter{theorem}{-1}
}

\begin{proof}
We prove this proposition by appropriately instantiating Proposition~\ref{prop:indexedelimcolimits}. 
Namely, given another cocone $\alpha : J \longrightarrow \Delta(A)$ in $\mathcal{V}_X$, we choose the object in $\mathcal{V}_{\ia {\mathsf{\ul{colim}}(J)}}$ to be $\pi^*_{\mathsf{\ul{colim}}(J)}(A)$ 
and define the morphisms $f_D$ using $\alpha_D$ as the following composites:
\[
\xymatrix@C=3.75em@R=6em@M=0.5em{
1_{\ia{J(D)}} \ar[r]^-{1({\delta_{J(D)}})} & 1_{\ia{\pi^*_{J(D)}(J(D))}} \ar[rr]^-{1(\ia {\pi^*_{J(D)}(\alpha_D)})} && 1_{\ia{\pi^*_{J(D)}(A)}} \ar[d]^-{\varepsilon^{1 \,\dashv\, \ia -}_{\pi^*_{J(D)}(A)}}
\\
\ia{\mathsf{\ul{in}}^J_D}^*(\pi^*_{\mathsf{\ul{colim}}(J)}(A)) &&&  \pi^*_{J(D)}(A) \ar[lll]^-{=}
}
\]
where $\delta_{J(D)}$ is a diagonal morphism, as defined in Definition~\ref{def:diagonalmorphisminbase}; and where the last equality follows from applying $\mathcal{P}$ to $\mathsf{\ul{in}}^J_D$, i.e., from $\pi_{\mathsf{\ul{colim}}(J)} \comp \ia{\mathsf{\ul{in}}^J_D} = \id_X \comp \pi_{J(D)}$.
%
Furthermore, that $f_D$ is vertical over $\id_{\ia {J(D)}}$ follows from the commutativity of
\[
\xymatrix@C=3.75em@R=6em@M=0.5em{
\ia{J(D)} \ar@/^3pc/[rrr]^-{p(f_D)}_*+<1em>{\dcomment{\text{def. of }f_D}}
\ar[r]_-{\delta_{J(D)}}
\ar@/_1pc/[dr]_-{\id_{\ia{J(D)}}}
& 
\ia {\pi^*_{J(D)}(J(D))}
\ar[r]_-{\ia {\pi^*_{J(D)}(\alpha_D)}} 
\ar[d]_-{\pi_{\pi^*_{J(D)}(J(D))}}^-{\qquad\,\,\,\dcomment{\mathcal{P}({\pi^*_{J(D)}(\alpha_D)})}}_<<<<<{\dcomment{\text{def. of } \delta_{J(D)}}\quad}
& 
\ia {\pi^*_{J(D)}(A)} 
\ar[r]_-{p(\varepsilon^{1 \,\dashv\, \ia -}_{\pi^*_{J(D)}(A)})}
\ar[d]^>>>>>>>>>{\pi_{\pi^*_{J(D)}(A)}}^<<<<<<<<{\!\!\!\!\!\!\!\!\!\quad\dcomment{\text{def. of } \pi_{\pi^*_{J(D)}(A)}}}
& 
\ia {J(D)}
\\
&
\ia{J(D)}
\ar[r]_-{\id_{\ia{J(D)}}}
&
\ia{J(D)}
\ar@/_1pc/[ur]_-{\id_{\ia{J(D)}}}
}
\]

Next, to be able to use Proposition~\ref{prop:indexedelimcolimits}, we also have to check that for all morphisms $g : D_i \longrightarrow D_j$, we have $\ia {J(g)}^*(f_{D_j}) = f_{D_i}$. We do so by recalling that the reindexing $\ia {J(g)}^*(f_{D_j})$ is defined as the unique mediating morphism induced by the Cartesian morphism $\overline{\ia {J(g)}}(\pi^*_{J(D_j)}(A))$. As a consequence, proving that the equation $\ia {J(g)}^*(f_{D_j}) = f_{D_i}$ holds amounts to showing that $f_{D_i}$ satisfies the same universal property as $\ia {J(g)}^*(f_{D_j})$, i.e., we have to show that the following diagram commutes:
\[
\scriptsize
\xymatrix@C=4em@R=4em@M=0.5em{
1_{\ia{J(D_i)}}
\ar[r]_-{1(\delta_{J(D_i)})}
\ar@/^2pc/[rrr]^-{f_{D_i}}_*+<0.75em>{\dscomment{\text{def. of } f_{D_i}}}
\ar@{<-}[d]_-{=}
\ar@/^4pc/[dddd]^-{1(\ia {J(g)})}_-{\dscomment{1 \text{ is s. fib.}}\quad\!\!}
&
1_{\ia {\pi^*_{J(D_i)}(J(D_i))}}
\ar[r]_-{1(\ia {\pi^*_{J(D_i)}(\alpha_{D_i})})}
\ar[d]_-{=}
&
1_{\ia {\pi^*_{J(D_i)}(A)}}
\ar[r]_-{\varepsilon^{1 \,\dashv\, \ia -}_{\pi^*_{J(D_i)}(A)}}
\ar[d]^-{=}_<<<<<{\dscomment{p \text{ is a split fibration}}\qquad\quad}
&
\pi^*_{J(D_i)}(A)
\ar[d]^-{=}
\\
\ia {J(g)}^*(1_{\ia {J(D_j)}})
\ar[ddd]_-{\overline{\ia {J(g)}}(1_{\ia {J(D_j)}})}
&
1_{\ia {\ia {J(g)}^*(\pi^*_{J(D_j)}(J(D_i)))}}
\ar[r]^-{1(\ia {\ia {J(g)}^*(\pi^*_{J(D_j)}(\alpha_{D_i}))})}
\ar[dd]^-{1(\ia {\overline{\ia {J(g)}}(\pi^*_{J(D_j)}(J(D_i)))})}_<<<<<{\dscomment{(a)}\qquad}
&
1_{\ia {\ia {J(g)}^*(\pi^*_{J(D_j)}(A))}}
\ar[ddd]^-{1(\ia {\overline{\ia {J(g)}}(\pi^*_{J(D_j)}(A))})}^<<<<<<{\!\!\!\!\qquad\qquad\dscomment{\text{nat. of } \varepsilon^{1 \,\dashv\, \ia -}}}_<<<<<<{\dscomment{\text{def. of } \ia {J(g)}^*(-)}\qquad\quad}
&
\ia {J(g)}^*(\pi^*_{J(D_j)}(A))
\ar[ddd]^-{\overline{\ia {J(g)}}(\pi^*_{J(D_j)}(A))}
\\
\\
&
1_{\ia {\pi^*_{J(D_j)}(J(D_i))}}
\ar@/^1pc/[dr]^<<<<<<{1(\ia{\pi^*_{J(D_j)}(\alpha_{D_i})})}
\ar[d]_-{1(\ia {\pi^*_{J(D_j)}(J(g))})}^<<<<<{\qquad\dscomment{\text{nat. of } \alpha}}
\\
1_{\ia {J(D_j)}}
\ar[r]^-{1(\delta_{J(D_j)})}
\ar@/_2pc/[rrr]_-{f_{D_j}}^*+<0.75em>{\dscomment{\text{def. of } f_{D_j}}}
&
1_{\ia {\pi^*_{J(D_j)}(J(D_j))}}
\ar[r]^-{1(\ia {\pi^*_{J(D_j)}(\alpha_{D_j})})}
&
1_{\ia {\pi^*_{J(D_j)}(A)}}
\ar[r]^-{\varepsilon^{1 \,\dashv\, \ia -}_{\pi^*_{J(D_j)}(A)}}
&
\pi^*_{J(D_j)}(A)
}
\]
where we prove the commutativity of the subdiagram marked with $(a)$ by showing that 
\[
\hspace{-0.1cm}
\xymatrix@C=3em@R=5em@M=0.5em{
\ia {J(D_i)}
\ar[r]^-{\delta_{J(D_i)}}
&
\ia {\pi^*_{J(D_i)}(J(D_i))}
\ar[r]^-{=}
&
\ia {\ia {J(g)}^(\pi^*_{J(D_j)}(J(D_i)))}
\ar[d]^-{\ia {\overline{\ia {J(g)}}(\pi^*_{J(D_j)}(J(D_i)))}}
\\
\ia {\pi^*_{J(D_j)}(J(D_j))}
&
&
\ia {\pi^*_{J(D_j)}(J(D_i))}
\ar[ll]^-{\ia {\pi^*_{J(D_j)}(J(g))}}
}
\]
and
\[
\xymatrix@C=5em@R=4em@M=0.5em{
\ia {J(D_i)} 
\ar[r]^-{\ia {J(g)}}
& 
\ia {J(D_j)} 
\ar[r]^-{\delta_{J(D_j)}}
& 
\ia {\pi^*_{J(D_j)}(J(D_j))}
}
\]
satisfy the same universal property as the unique 
mediating morphism in the following pullback situation (i.e., they make the two triangles involving $\ia {J(g)}$ commute):
\[
\xymatrix@C=5em@R=5em@M=0.5em{
\ia {J(D_i)} \ar@/_2pc/[dr]_{\ia {J(g)}} \ar@/^5pc/[rr]^{\ia {J(g)}} \ar@{-->}[r] & \ia {\pi^*_{J(D_j)}(J(D_j))} \ar[d]_{\pi_{\pi^*_{J(D_j)}(J(D_j))}}^<{\,\big\lrcorner} \ar[r]^-{\ia {\overline{\pi_{J(D_j)}}(J(D_j))}} & \ia {J(D_j)} \ar[d]^{\pi_{J(D_j)}}_-{\dcomment{\mathcal{P}(\overline{\pi_{J(D_j)}}(J(D_j)))}\quad\,\,\,\,\,\,\,\,\,\,\,\,}
\\
& \ia {J(D_j)} \ar[r]_-{\pi_{J(D_j)}} & X
}
\]
We omit the details of these proofs because they consist of straightforward diagram chasing, based on using the definitions of the diagonal morphisms $\delta_{J(D_i)}$ and $\delta_{J(D_j)}$ as unique mediating morphisms into certain pullback squares (see Definition~\ref{def:diagonalmorphisminbase}).

Now, using Proposition~\ref{prop:indexedelimcolimits}, we get that there exists a unique vertical morphism 
\[
[f_D]_{D \in \mathcal{D}} : 1_{\ia {\mathsf{\ul{colim}}(J)}} \longrightarrow \pi^*_{\mathsf{\ul{colim}}(J)}(A)
\]
such that for all $D_i$ in $\mathcal{D}$ the following ``$\beta$-equation" holds:
\[
\ia {\mathsf{\ul{in}}^J_{D_i}}^*([f_D]_{D \in \mathcal{D}}) = f_{D_i}
\]

Based on this, we define the candidate mediating morphism $[\alpha]$ from $\mathsf{\ul{in}}^J$ to $\alpha$ using the fully-faithfulness of $\mathcal{P}$ on the following morphism in $\mathcal{B}/X$ from $\pi_{\mathsf{\ul{colim}}(J)}$ to $\pi_A$:
\[
\hspace{-0.1cm}
\xymatrix@C=4.5em@R=5em@M=0.5em{
\ia {\mathsf{\ul{colim}}(J)} 
\ar[r]^-{\eta^{1 \,\dashv\, \ia -}_{\ia {\mathsf{\ul{colim}}(J)} }}
\ar@/_2pc/[drr]_-{\id_{\ia {\mathsf{\ul{colim}}(J)}}}
& 
\ia {1_{\ia {\mathsf{\ul{colim}}(J)}}}
\ar[r]^-{\ia {[f_D]_{D \in \mathcal{D}}}}
\ar[dr]_-{\pi_{1_{\ia {\mathsf{\ul{colim}}(J)}}}}_<<<<<{\dcomment{\eta^{1 \,\dashv\, \ia -} \text{ is iso.}}\qquad\quad}
&
\ia {\pi^*_{\mathsf{\ul{colim}}(J)}(A)}
\ar[r]^-{\ia {\overline{\pi_{\mathsf{\ul{colim}}(J)}}(A)}}
\ar[d]^>>>>>>>>{\pi_{\pi^*_{\mathsf{\ul{colim}}(J)}(A)}}_<<<<<{\dcomment{\mathcal{P}([f_D]_{D \in \mathcal{D}})}\quad}
&
\ia A
\ar[d]^-{\pi_A}_<<<<<<{\dcomment{\mathcal{P}(\overline{\pi_{\mathsf{\ul{colim}}(J)}}(A))}\qquad\!\!\!\!}
\\
&
&
\ia {\mathsf{\ul{colim}}(J)} 
\ar[r]_-{\pi_{\mathsf{\ul{colim}}(J)}}
&
X
}
\]
As $p = \mathsf{cod}_{\mathcal{B}} \comp \mathcal{P}$, the fully-faithfulness of $\mathcal{P}$ also gives us that $[\alpha]$ is vertical over $\id_X$.

Next, we check that $[\alpha]$ is indeed a morphism of cocones from $\mathsf{\ul{in}}^J$ to $\alpha$, i.e., we prove that $[\alpha] \comp \mathsf{\ul{in}}^J_D = \alpha_D$ holds, for all $D$ in $\mathcal{D}$. We do so by making use of the fully-faithfulness of $\mathcal{P}$ and instead prove that $\mathcal{P}([\alpha] \comp \mathsf{\ul{in}}^J_D) = \mathcal{P}(\alpha_D)$ holds in $\mathcal{B}/X$, for all $D$ in $\mathcal{D}$, which amounts to showing that the following diagram commutes:
\[
\scriptsize
\xymatrix@C=3em@R=6em@M=0.5em{
\ia {J(D)}
\ar[r]^-{\ia {\mathsf{\ul{in}}^J_D}}
\ar[drr]_-{\mathsf{s}(\ia {\mathsf{\ul{in}}^J_D}^*([f_D]_{D \in \mathcal{D}}))\quad}^>>>>>>{\quad\qquad\dscomment{\text{Proposition~\ref{prop:reindexinginthebasecategory}}}}^-{\quad\dscomment{\mathsf{s}(-) \text{ is iso.}}}_>>>>>>>>>>{\dscomment{\text{Proposition~\ref{prop:indexedelimcolimits}}}\qquad\quad}
\ar@/_3pc/[drr]_-{\mathsf{s}(f_D)}
\ar@/_2pc/[ddr]^>>>>>>>>{\!\!\!\!\!\eta^{1 \,\dashv\, \ia -}_{\ia {J(D)}}}
\ar[ddd]^-{\delta_{J(D)}}
\ar@/_3pc/[dddd]^-{\id_{\ia {J(D)}}}^>>>>>{\qquad\quad\dscomment{\text{def. of } \delta_{J(D)}}}
&
\ia {\mathsf{\ul{colim}}(J)}
\ar[r]^-{\eta^{1 \,\dashv\, \ia -}_{\ia {\mathsf{\ul{colim}}(J)}}}
\ar@/_2pc/[rr]_{\mathsf{s}([f_D]_{D \in \mathcal{D}})}^*+<0.4em>{\dscomment{\text{def. of } \mathsf{s}([f_D]_{D \in \mathcal{D}})}}
&
\ia {1_{\ia {\mathsf{\ul{colim}}(J)}}}
\ar[r]^-{\ia {[f_D]_{D \in \mathcal{D}}}}
&
\ia {\pi^*_{\mathsf{\ul{colim}}(J)}(A)}
\ar[r]^-{\ia {\overline{\pi_{\mathsf{\ul{colim}}(J)}}(A)}}
&
\ia A
\\
&
&
\ia {\ia {\mathsf{\ul{in}}^J_D}^*(\pi^*_{\mathsf{\ul{colim}}(J)}(A))}
\ar[ur]_-{\,\,\,\ia {\overline{\ia {\mathsf{\ul{in}}^J_D}} (\pi^*_{\mathsf{\ul{colim}}(J)}(A))}}
&
\ia {\pi^*_{J(D)}(A)}
\ar[l]_-{=}
\ar[ur]_-{\!\!\!\!\!\!\!\ia {\overline{\pi_{J(D)}}(A)}}^>>>>>>>>>>>>>{\dscomment{p \text{ is a split fib.}}\quad}
\\
&
\ia {1_{\ia {J(D)}}}
\ar[ur]_-{\ia {f_D}}^-{\dscomment{\text{def. of } \mathsf{s}(f_D)}\qquad}_<<<<{\qquad\dscomment{\text{def. of } f_D}}
\ar[d]^-{\!\!\!\ia {1(\delta_{J(D)})}}_-{\dscomment{\text{nat. of } \varepsilon^{1 \,\dashv\, \ia -}}\qquad}
&
\ia {1_{\ia {\pi^*_{J(D)}(A)}}}
\ar[ur]^-{\ia {\varepsilon^{}_{\pi^*_{J(D)}(A)}}\!\!\!\!\!\!\!\!\!}_<<<<<<<<<<{\quad\dscomment{\text{nat. of } \varepsilon^{1 \,\dashv\, \ia -}}}
\\
\ia {\pi^*_{J(D)}(J(D))}
\ar@/_2pc/[rr]_-{\id_{\ia {\pi^*_{J(D)}(J(D))}}}^*+<-0.05em>{\dscomment{1 \,\dashv\, \ia -}}
\ar[r]^-{\eta^{1 \,\dashv\, \ia -}_{\ia {\pi^*_{J(D)}(J(D))}}}
& 
\ia {1_{\ia {\pi^*_{J(D)}(J(D))}}}
\ar[ur]_>>>>>>>{\,\,\,\,\ia {1(\ia {\pi^*_{J(D)}(\alpha_D)})}}
\ar[r]^-{\ia {\varepsilon^{1 \,\dashv\, \ia -}_{\pi^*_{J(D)}(J(D))}}}
&
\ia {\pi^*_{J(D)}(J(D))}
\ar@/_2pc/[uur]^-{\ia {\pi^*_{J(D)}(\alpha_D)}\!\!\!\!\!}_-{\dscomment{\text{def. of } \pi^*_{J(D)}(\alpha_D)}}
\ar@/^1.5pc/[dll]^<<<<<<<<<<{\,\,\,\,\,\,\,\,\ia {\overline{\pi_{J(D)}}(J(D))}}
\\
\ia {J(D)}
\ar@/_10pc/[rrrruuuu]_-{\!\!\!\!\ia {\alpha_D}}
&
&
&
&
}
\]

Finally, we prove that $[\alpha]$ is the unique morphism from $\mathsf{\ul{in}}^J$ to $\alpha$. Namely, assuming another morphism of cocones $h$ from $\mathsf{\ul{in}}^J$ to $\alpha$, we show that $[\alpha] = h$. To this end, we first define a morphism $\widehat{h} : 1_{\ia {\mathsf{\ul{colim}}(J)}} \longrightarrow \pi^*_{\mathsf{\ul{colim}}(J)}(A)$ as the following composite:
\[
\xymatrix@C=3em@R=4em@M=0.5em{
1_{\ia {\mathsf{\ul{colim}}(J)}}
\ar[r]^-{1(\delta_{\mathsf{\ul{colim}}(J)})}
&
1_{\ia {\pi^*_{\mathsf{\ul{colim}}(J)}({\mathsf{\ul{colim}}(J)})}}
\ar[r]^-{1(\ia {\pi^*_{\mathsf{\ul{colim}}(J)}(h)})}
&
1_{\ia {\pi^*_{\mathsf{\ul{colim}}(J)}(A)}}
\ar[r]^-{\varepsilon^{1 \,\dashv\, \ia -}_{\pi^*_{\mathsf{\ul{colim}}(J)}(A)}}
&
\pi^*_{\mathsf{\ul{colim}}(J)}(A)
}
\]
which is vertical over $\id_{\ia {\mathsf{\ul{colim}}(J)}}$ because the following diagram commutes:
\[
\xymatrix@C=5em@R=5em@M=0.5em{
\ia {\mathsf{\ul{colim}}(J)}
\ar[r]^-{\delta_{\mathsf{\ul{colim}}(J)}}
\ar@/_2pc/[dr]_-{\id_{\ia {\mathsf{\ul{colim}}(J)}}}
&
\ia {\pi^*_{\mathsf{\ul{colim}}(J)}(\mathsf{\ul{colim}}(J))}
\ar[r]^-{\ia {\pi^*_{\mathsf{\ul{colim}}(J)}(h)}}
\ar[d]_-{\pi_{\pi^*_{\mathsf{\ul{colim}}(J)}(\mathsf{\ul{colim}}(J))}}_<<<<{\dcomment{\text{def. of } \delta_{\mathsf{\ul{colim}}(J)}}\qquad\,\,\,}
&
\ia {\pi^*_{\mathsf{\ul{colim}}(J)}(A)}
\ar[d]^-{\pi_{\pi^*_{\mathsf{\ul{colim}}(J)}(A)}}_-{\dcomment{\mathcal{P}(\pi^*_{\mathsf{\ul{colim}}(J)}(h))}\qquad\qquad\!\!\!\!}
\\
& 
\ia {\mathsf{\ul{colim}}(J)}
\ar[r]_-{\id_{\ia {\mathsf{\ul{colim}}(J)}}}
&
\ia {\mathsf{\ul{colim}}(J)}
}
\]

Next, as we can show that for all $D$ in $\mathcal{D}$ we have $\ia {\mathsf{\ul{in}}^J_D}^*(\widehat{h}) = f_D$ (we omit the proofs of these equations as they are analogous to the proofs of $\ia {J(g)}^*(f_{D_j}) = f_{D_i}$ given earlier), then the uniqueness of $[f_D]_{D \in \mathcal{D}}$ means that we have $\widehat{h} = [f_D]_{D \in \mathcal{D}}$. Finally, we prove that $[\alpha] = h$ holds by making use of the fully-faithfulness of $\mathcal{P}$ and instead show that $\mathcal{P}([\alpha]) = \mathcal{P}(h)$ holds in $\mathcal{B}/X$, which amounts to proving that the following holds:
\[
\scriptsize
\xymatrix@C=5em@R=5em@M=0.5em{
\ia {\mathsf{\ul{colim}}(J)}
\ar[r]^-{\eta^{1 \,\dashv\, \ia -}_{\ia {\mathsf{\ul{colim}}(J)}}}
\ar[d]^-{\delta_{\mathsf{\ul{colim}}(J)}}^<<<<{\!\!\!\!\qquad\qquad\dscomment{\text{nat. of } \eta^{1 \,\dashv\, \ia -}}}
\ar@/_4pc/[ddd]^-{\id_{\ia {\mathsf{\ul{colim}}(J)}}}
&
\ia {1_{\ia {\mathsf{\ul{colim}}(J)}}}
\ar@/^1pc/[r]^-{\ia {[f_D]_{D \in \mathcal{D}}}}_*+<0.75em>{\dscomment{\text{uniq. of } [f_D]_{D \in \mathcal{D}}}}
\ar@/_1pc/[r]_-{\ia {\widehat{h}}}
\ar[d]_-{\ia {1(\delta_{\mathsf{\ul{colim}}(J)})}}^>>>>>>{\quad\qquad\dscomment{\text{def. of } \widehat{h}}}
&
\ia {\pi^*_{\mathsf{\ul{colim}}(J)}(A)}
\ar[r]^-{\ia {\pi^*_{\mathsf{\ul{colim}}(J)}(A)}}
&
\ia A
\\
\ia {\pi^*_{\mathsf{\ul{colim}}(J)}(\mathsf{\ul{colim}}(J))}
\ar[r]^-{\eta^{1 \,\dashv\, \ia -}_{\ia {\pi^*_{\mathsf{\ul{colim}}(J)}(\mathsf{\ul{colim}}(J))}}}
\ar@/_2pc/[drr]^<<<<<<<<<{\,\,\,\,\id_{\ia {\pi^*_{\mathsf{\ul{colim}}(J)}(\mathsf{\ul{colim}}(J))}}}
\ar@/^1pc/[dd]^-{\ia {\ia {\overline{\pi_{\mathsf{\ul{colim}}(J)}}(\mathsf{\ul{colim}}(J))}}}^>>>>>>>>>>{\qquad\qquad\qquad\qquad\dscomment{\text{id. law}}}_>>>>>>>>>>>>>>>>>>>{\dscomment{\text{def. of } \delta_{\mathsf{\ul{colim}}(J)}}\,\,\,\,}
&
\ia {1_{\ia {\pi^*_{\mathsf{\ul{colim}}(J)}({\mathsf{\ul{colim}}(J)})}}}
\ar[r]_-{\ia {1(\ia {\pi^*_{\mathsf{\ul{colim}}(J)}(h)})}}
\ar[dr]_-{\ia {\varepsilon^{1 \,\dashv\, \ia -}_{\pi^*_{\mathsf{\ul{colim}}(J)}({\mathsf{\ul{colim}}(J)})}}\,\,\,\,}^-{\qquad\qquad\dscomment{\text{nat. of } \varepsilon^{1 \,\dashv\, \ia -}}}_<<<<{\dscomment{1 \,\dashv\, \ia -}\qquad\quad}
&
\ia {1_{\ia {\pi^*_{\mathsf{\ul{colim}}(J)}(A)}}}
\ar[u]^-{\ia {\varepsilon^{1 \,\dashv\, \ia -}_{\pi^*_{\mathsf{\ul{colim}}(J)}(A)}}}
\\
&
&
\ia {\pi^*_{\mathsf{\ul{colim}}(J)}({\mathsf{\ul{colim}}(J)})}
\ar@/_3pc/[uu]_>>>>>>>{\!\!\!\!\ia {\pi^*_{\mathsf{\ul{colim}}(J)}(h)}}
\ar@/^2pc/[dll]_-{\ia {\overline{\pi_{\mathsf{\ul{colim}}(J)}}(\mathsf{\ul{colim}}(J))}\quad}
\\
\ia {\mathsf{\ul{colim}}(J)}
\ar@/_10pc/[uuurrr]_-{\ia {h}}^-{\dscomment{\text{def. of } \pi^*_{\mathsf{\ul{colim}}(J)}(h)}\qquad\qquad\qquad}
}
\]
\end{proof}


\section{Proof of Proposition~\ref{prop:fibredNNO}}
\label{sect:proofofprop:fibredNNO}

{
\renewcommand{\thetheorem}{\ref{prop:fibredNNO}}
\begin{proposition}
Let us assume a full split comprehension category with unit \linebreak $p : \mathcal{V} \longrightarrow \mathcal{B}$ such that $\mathcal{B}$ has a terminal object and $p$ has weak split fibred strong 
natural numbers. Then, each fibre of $p$ has a weak NNO and this structure is preserved on-the-nose by reindexing.
\end{proposition}
\addtocounter{theorem}{-1}
}

\begin{proof}
Given any object $X$ in $\mathcal{B}$, we claim that the diagram
\[
\xymatrix@C=7em@R=6em@M=0.5em{
1_X \ar[r]^-{!_X^*(\mathsf{zero})} & !_X^*(\mathbb{N}) & !_X^*(\mathbb{N}) \ar[l]_-{!_X^*(\mathsf{succ})}
}
\]
defines a weak NNO in $\mathcal{V}_X$. 

To show that this is the case, we assume another diagram in $\mathcal{V}_X$, given by
\[
\xymatrix@C=7em@R=6em@M=0.5em{
1_X \ar[r]^-{f_z} & A & A \ar[l]_-{f_s}
}
\]

\pagebreak

Next, we observe that the morphisms $f_z$ and $f_s$ induce two composite morphisms
\[
\hspace{-0.2cm}
\xymatrix@C=0.7em@R=1em@M=0.5em{
1_{\ia {1_X}} \ar[r]^-{=} & \pi^*_{1_X}(1_X) \ar[rr]^-{\pi^*_{1_X}(f_z)} && \pi^*_{1_X}(A) \ar[r]^-{=} & \ia {!_X^*(\mathsf{zero})}^*(\pi^*_{!_X^*(\mathbb{N})}(A)) \ar[rrrr]^-{\overline{\ia {!_X^*(\mathsf{zero})}}(\pi^*_{!_X^*(\mathbb{N})}(A))} &&&& \pi^*_{!_X^*(\mathbb{N})}(A)
}
\]
\[
\hspace{-0.25cm}
\xymatrix@C=1em@R=1em@M=0.5em{
\pi^*_{!_X^*(\mathbb{N})}(A) \ar[rr]^-{\pi^*_{!_X^*(\mathbb{N})}(f_s)} && \pi^*_{!_X^*(\mathbb{N})}(A) \ar[r]^-{=} & \ia {!_X^*(\mathsf{succ})}^*(\pi^*_{!_X^*(\mathbb{N})}(A)) \ar[rrrr]^-{\overline{\ia {!_X^*(\mathsf{succ})}}(\pi^*_{!_X^*(\mathbb{N})}(A))} &&&& \pi^*_{!_X^*(\mathbb{N})}(A)
}
\]
which we denote in the rest of this proof by $\widehat{f_z}$ and $\widehat{f_s}$, respectively. 

It is easy to see that $\widehat{f_z}$ and $\widehat{f_s}$ are over $\ia {!_X^*(\mathsf{zero})}$ and $\ia {!_X^*(\mathsf{succ})}$, respectively. As a result, we can use the existence of weak split fibred strong natural numbers in $p$ to get a section of $\pi_{\pi^*_{!_X^*(\mathbb{N})}(A)}$, which we denote by $\mathsf{rec}(\widehat{f_z},\widehat{f_s}) : \ia {!_X^*(\mathbb{N})} \longrightarrow \ia {\pi^*_{!_X^*(\mathbb{N})}(A)}$.

Using $\mathsf{rec}(\widehat{f_z},\widehat{f_s})$, we can derive a vertical morphism in $\mathcal{V}_X$, denoted by 
\[
\mathsf{rec}_X(f_z,f_h) :\, !_X^*(\mathbb{N}) \longrightarrow A
\]
and defined using the fully-faithfulness of $\mathcal{P}$ on the next commuting diagram in $\mathcal{B}/X$.
\[
\xymatrix@C=9em@R=4em@M=0.5em{
\ia {!_X^*(\mathbb{N})} \ar[r]^-{\mathsf{rec}(\widehat{f_z},\widehat{f_s})} \ar@/_4pc/[ddr]_-{\pi_{!_X^*(\mathbb{N})}}  \ar@/_2pc/[dr]_-{\id_{\ia {!_X^*(\mathbb{N})}}} & \ia {\pi^*_{!_X^*(\mathbb{N})}(A)} \ar[r]^-{\ia {\overline{\pi_{!_X^*(\mathbb{N})}}(A)}} \ar[d]^-{\pi_{\pi^*_{!_X^*(\mathbb{N})}(A)}}_<<<<<<<{\dcomment{\text{def. of } \mathsf{rec}(\widehat{f_z},\widehat{f_s})}\qquad} \ar@{}[dd]^-{\qquad\qquad\dcomment{\mathcal{P}(\overline{\pi_{!_X^*(\mathbb{N})}}(A))}} & \ia {A} \ar@/^4pc/[ddl]^-{\pi_A}
\\
& \ia {!_X^*(\mathbb{N})} \ar[d]^-{\pi_{!_X^*(\mathbb{N})}}_-{\dcomment{\text{id. law}}\qquad\quad\!\!\!}
\\
& X
}
\]

Next, we proceed by showing that the two standard diagrams describing the interaction of $\mathsf{rec}_X(f_z,f_h)$ with $!_X^*(\mathsf{zero})$ and $!_X^*(\mathsf{succ})$ commute. 

\pagebreak

First, we show the commutativity of
\[
\xymatrix@C=7em@R=5em@M=0.5em{
1_X \ar[r]^-{!_X^*(\mathsf{zero})} \ar[d]_-{\id_{1_X}} & !_X^*(\mathbb{N}) \ar[d]^-{\mathsf{rec}_X(f_z,f_s)}
\\
1_X \ar[r]_-{f_z} & A
}
\]
by using the fully-faithfulness of $\mathcal{P}$ on a diagram in $\mathcal{B}/X$ between $\pi_{1_X} : \ia {1_X} \longrightarrow X$ and $\pi_A : \ia {A} \longrightarrow X$ that is given between the domains of $\pi_{1_X}$ and $\pi_A$ by
\[
\scriptsize
\xymatrix@C=4em@R=3em@M=0.5em{
\ia {1_X} \ar@/_4.5pc/[dddd]_-{\id_{\ia {1_X}}} \ar[rrr]^-{\ia {!_X^*(\mathsf{zero})}} \ar[d]^-{\eta^{1 \,\dashv\, \ia -}_{\ia {1_X}}} &&& \ia {!_X^*(\mathbb{N})} \ar[d]^-{\mathsf{rec}(\widehat{f_z},\widehat{f_s})}_-{\dscomment{\text{def. of } \mathsf{rec}(\widehat{f_z},\widehat{f_s}) }\qquad\qquad\qquad\qquad\qquad\qquad\qquad}
\\
\ia {1_{\ia {1_X}}} \ar[ddr]_-{=}^-{\qquad\qquad\dscomment{\text{def. of } \widehat{f_z} }} \ar[ddd]_-{\pi_{1_{\ia {1_X}}}} \ar@{}[dd]_-{\dscomment{\eta^{1 \,\dashv\, \ia -}_{\ia {1_X}} \text{ is an iso.}}\,\,\,} \ar[rrr]_-{\ia {\widehat{f_z}}} &&& \ia {\pi^*_{!_X^*(\mathbb{N})}(A)} \ar[ddd]^-{\ia {\overline{\pi_{!_X^*(\mathbb{N})}}(A)}}
\\
\ar@{}[dd]^-{\,\,\,\,\,\dscomment{1 \text{ is split fibred}}} && \ia {\ia {!_X^*(\mathsf{zero})}^*(\pi^*_{!_X^*(\mathbb{N})}(A))} \ar[ur]^-{\ia {\overline{\ia {!_X^*(\mathsf{zero})}}(\pi^*_{!_X^*(\mathbb{N})}(A))}} &
\\
& \ia {\pi^*_{1_X}(1_X)} \ar[r]^-{\ia {\pi^*_{1_X}(f_z)}} \ar[dl]^-{\ia {\overline{\pi_{1_X}}(1_X)}}^<<<{\qquad\qquad\qquad\qquad\qquad\dscomment{\text{def. of } \pi^*_{1_X}(f_z)}} & \ia {\pi^*_{1_X}(A)} \ar[u]^-{=}_-{\qquad\quad\dscomment{p \text{ is a split fibration}}} \ar[dr]_-{\ia {\overline{\pi_{1_X}}(A)}} &
\\
\ia {1_X} \ar[rrr]_-{\ia {f_z}} &&& \ia {A}
}
\]

Second, we show the commutativity of 
\[
\xymatrix@C=7em@R=5em@M=0.5em{
!_X^*(\mathbb{N}) \ar[r]^-{!_X^*(\mathsf{succ})} \ar[d]_-{\mathsf{rec}_X(f_z,f_s)} & !_X^*(\mathbb{N}) \ar[d]^-{\mathsf{rec}_X(f_z,f_s)}
\\
A \ar[r]_-{f_s} & A
}
\]
by using the fully-faithfulness of $\mathcal{P}$ on a commuting diagram in $\mathcal{B}/X$ between \linebreak ${\pi_{!_X^*(\mathbb{N})} : \ia {!_X^*(\mathbb{N})} \longrightarrow X}$ and ${\pi_A : \ia {A} \longrightarrow X}$ that is given between the domains of $\pi_{!_X^*(\mathbb{N})}$ and $\pi_A$ by

\pagebreak

\[
\scriptsize
\xymatrix@C=4em@R=3em@M=0.5em{
\ia {!_X^*(\mathbb{N})} \ar[rrr]^-{\ia {!_X^*(\mathsf{succ})}} \ar[d]_-{\mathsf{rec}(\widehat{f_z},\widehat{f_s})}^-{\qquad\qquad\qquad\qquad\qquad\qquad\qquad\dscomment{\text{def. of } \mathsf{rec}(\widehat{f_z},\widehat{f_s})}} &&& \ia {!_X^*(\mathbb{N})} \ar[d]^-{\mathsf{rec}(\widehat{f_z},\widehat{f_s})}
\\
\ia {\pi^*_{!_X^*(\mathbb{N})}(A)} \ar[ddd]_-{\ia {\overline{\pi_{!_X^*(\mathbb{N})}}(A)}} \ar[rrr]_-{\ia {\widehat{f_s}}} \ar[ddr]_-{\id_{\ia {\pi^*_{!_X^*(\mathbb{N})}(A)}}} & \ar@{}[dd]^-{\dscomment{\text{def. of } \widehat{f_s}}} && \ia {\pi^*_{!_X^*(\mathbb{N})}(A)} \ar[ddd]^-{\ia {\overline{\pi_{!_X^*(\mathbb{N})}}(A)}}
\\
\ar@{}[dd]^-{\qquad\dscomment{\text{id. law}}} && \ia {!_X^*(\mathsf{succ})}^*(\pi^*_{!_X^*(\mathbb{N})}(A)) \ar[ur]^-{\ia {\overline{\ia {!_X^*(\mathsf{succ})}}(\pi^*_{!_X^*(\mathbb{N})}(A))}} &
\\
& \ia {\pi^*_{!_X^*(\mathbb{N})}(A)} \ar[r]^-{\ia {\pi^*_{!_X^*(\mathbb{N})}(f_s)}} \ar[dl]^-{\ia {\overline{\pi_{!_X^*(\mathbb{N})}}(A)}}^<<{\qquad\qquad\qquad\qquad\qquad\dscomment{\text{def. of } \pi^*_{!_X^*(\mathbb{N})}(f_s)}} & \ia {\pi^*_{!_X^*(\mathbb{N})}(A)} \ar[u]_-{=}_-{\qquad\quad\dscomment{1 \text{ is a split fibration}}} \ar[dr]_-{\ia {\overline{\pi_{!_X^*(\mathbb{N})}}(A)}} &
\\
\ia {A} \ar[rrr]_-{\ia {f_s}} &&& \ia {A} 
}
\]

Finally, we show that this weak NNO structure is preserved on-the-nose by reindexing. In particular, for all morphisms $f : Y \longrightarrow X$ in $\mathcal{B}$, we have
\[
f^*(!^*_X(\mathbb{N})) = (!_X \comp f)^*(\mathbb{N}) = \,!^*_Y(\mathbb{N})
\]
The proofs of preservation are analogous for ${!_X^*(\mathsf{zero})}$, ${!_X^*(\mathsf{succ})}$, and $\mathsf{rec}_X(f_z,f_h)$.
\end{proof}

\section{Proof of Proposition~\ref{prop:equivalenceofnaturalnumbersinthesisandpaper}}
\label{sect:proofofprop:equivalenceofnaturalnumbersinthesisandpaper}

{
\renewcommand{\thetheorem}{\ref{prop:equivalenceofnaturalnumbersinthesisandpaper}}
\begin{proposition}
Let us assume a full split comprehension category with unit \linebreak $p : \mathcal{V} \longrightarrow \mathcal{B}$ such that $\mathcal{B}$ has a terminal object. Then, $p$ having weak split fibred strong natural numbers is equivalent to $p$ supporting weak natural numbers as in~\cite{Ahman:FibredEffects}, i.e., for every object $X$ in $\mathcal{B}$, every object $A$ in $\mathcal{V}_{\ia {!_X^*(\mathbb{N})}}$, every morphism 
\[
f_z : 1_X \longrightarrow (\funsection(!_X^*(\mathsf{zero})))^*(A)
\]
in $\mathcal{V}_X$, and every morphism 
\[
f_s : 1_{\ia A} \longrightarrow \pi_A^*(\ia {!_X^*(\mathsf{succ})}^* (A))
\]
in $\mathcal{V}_{\ia A}$, there exists a morphism 
\[
\mathsf{i}_A(f_z,f_s) : 1_{\ia {!^*_X(\mathbb{N})}} \longrightarrow A
\]
in $\mathcal{V}_{\ia {!^*_X(\mathbb{N})}}$ such that 
\[
\begin{array}{c}
(\funsection(!_X^*(\mathsf{zero})))^*(\mathsf{i}_A(f_z,f_s)) = f_z
\\[3mm]
\ia {!_X^*(\mathsf{succ})}^* (\mathsf{i}_A(f_z,f_s)) 
=
(\mathsf{s}(\mathsf{i}_A(f_z,f_s)))^*(f_s) 
\end{array}
\vspace{0.2cm}
\]
\end{proposition}
\addtocounter{theorem}{-1}
}

\pagebreak

\begin{proof}
In both directions, we assume an object $X$ in $\mathcal{B}$ and an object $A$ in $\mathcal{V}_{\ia {!_X^*(\mathbb{N})}}$.

\vspace{0.2cm}
\noindent \textbf{Weak natural numbers in~\cite{Ahman:FibredEffects} imply Definition~\ref{def:strongsplitfibredweaknaturals}:}

Given a pair of morphisms
\[
\hspace{-0.5cm}
\xymatrix@C=7em@R=1em@M=0.5em{
1_{\ia {1_X}} \ar[r]^-{f_z} & A & A \ar[l]_-{f_s}
}
\]
in $\mathcal{V}$, with 
\[
p(A) = \ia {!_X^*(\mathbb{N})}
\qquad
p(f_z) = \ia {!_X^*(\mathsf{zero})}
\qquad
p(f_s) = \ia {!_X^*(\mathsf{succ})}
\]
we aim to define a morphism
\[
\mathsf{rec}(f_z,f_s) : \ia {!_X^*(\mathbb{N})} \longrightarrow \ia {A}
\]
that must be a section of $\pi_A : \ia {A} \longrightarrow \ia {!_X^*(\mathbb{N})}$. 

First, we define vertical morphisms
\[
g_z : 1_X \longrightarrow (\mathsf{s}(!^*_X(\mathsf{zero})))^*(A)
\qquad
g_s : 1_{\ia A} \longrightarrow \pi^*_A(\ia {!^*_X(\mathsf{succ})}^*(A))
\]
in $\mathcal{V}_X$ and $\mathcal{V}_{\ia A}$, respectively, by
\[
g_z \defeq (\eta^{1 \,\dashv\, \ia -}_{\ia {1_X}})^*(f_z^\dagger)
\qquad
g_s \defeq \pi^*_A(f_s^\dagger) \comp \pi^*_A(\mathsf{fst}) \comp \eta^{\Sigma_A \,\dashv\, \pi^*_A}_{1_{\ia A}}
\]
where the two vertical morphisms (in $\mathcal{V}_{\ia {1_X}}$ and $\mathcal{V}_{\ia {!_X^*(\mathbb{N})}}$, respectively)
\[
f_z^\dagger : 1_{\ia {1_X}} \longrightarrow \ia {!_X^*(\mathsf{zero})} ^*(A)
\qquad
f_s^\dagger : A \longrightarrow \ia {!_X^*(\mathsf{succ})}^*(A)
\]
arise from using the universal properties of the following two Cartesian morphisms: 
\[
\overline{\ia {!_X^*(\mathsf{zero})}}(A) : \ia {!_X^*(\mathsf{zero})} ^*(A) \longrightarrow A
\qquad
\overline{\ia {!_X^*(\mathsf{succ})}}(A) : \ia {!_X^*(\mathsf{succ})}^*(A) \longrightarrow A
\]
on the given morphisms $f_z$ and $f_s$, respectively, as discussed in Definition~\ref{def:uniquemediatingmorphismforCartesianmorphism}. 

\pagebreak

Next, we recall that the existence of weak natural numbers in 
the sense of~\cite{Ahman:FibredEffects} means that $g_z$ and $g_s$ induce a vertical morphism
\[
\mathsf{i}_A(g_z,g_s) : 1_{\ia {!^*_X(\mathbb{N})}} \longrightarrow A
\]
in $\mathcal{V}_{\ia A}$, which we can in turn use to define $\mathsf{rec}(f_z,f_s)$ by letting
\[
\mathsf{rec}(f_z,f_s) \defeq \mathsf{s}(\mathsf{i}_A(g_z,g_s))
\]

Finally, we prove that $\mathsf{rec}(f_z,f_s)$ makes the next diagram commute in $\mathcal{B}$.
\[
\xymatrix@C=5em@R=5em@M=0.5em{
\ia {1_X} \ar[r]^-{\ia {!_X^*(\mathsf{zero})}} \ar[d]_-{\eta^{1 \,\dashv\, \ia -}_{\ia {1_X}}}^-{\quad\qquad\dcomment{(a)}} & \ia {!_X^*(\mathbb{N})} \ar[d]_-{\mathsf{rec}(f_z,f_s)}^-{\qquad\qquad\dcomment{(b)}} & \ia {!_X^*(\mathbb{N})} \ar[l]_-{\ia {!_X^*(\mathsf{succ})}} \ar[d]^-{\mathsf{rec}(f_z,f_s)}
\\
\ia {1_{\ia {1_X}}} \ar[r]_-{\ia {f_z}} & \ia {A} & \ia {A} \ar[l]^-{\ia {f_s}}
}
\]

In order to show that the square marked with $(a)$ commutes, we recast the equation
\[
(\funsection(!_X^*(\mathsf{zero})))^*(\mathsf{i}_A(g_z,g_s)) = g_z
\]
in $\mathcal{B}$ using Proposition~\ref{prop:reindexinginthebasecategory}. As a result, the outer perimeter of the next diagram (i.e., one of the triangles of the pullback situation in Proposition~\ref{prop:reindexinginthebasecategory}) commutes in $\mathcal{B}$.
\[
\xymatrix@C=4em@R=2em@M=0.5em{
&& \ia {!^*_X(\mathbb{N})} \ar[r]^-{\eta^{1 \,\dashv\, \ia -}_{\ia {!^*_X(\mathbb{N})}}} & \ia {1_{\ia {!^*_X(\mathbb{N})}}} \ar[dd]^-{\ia {\mathsf{i}_A(g_z,g_s)}}
\\
&& \ia {1_X} \ar[d]^-{\ia {g_z}} \ar[u]_-{\,\,\,\ia {!_X^*(\mathsf{zero})}}
\\
X \ar[rr]_-{\mathsf{s}(g_z)} \ar[urr]^-{\eta^{1 \,\dashv\, \ia -}_X\!\!\!\!\!} \ar@/^3pc/[uurr]^-{\funsection(!_X^*(\mathsf{zero}))} \ar@{}[uu]_<<<<{\quad\qquad\qquad\qquad\qquad\dcomment{\text{def. of } \mathsf{s}(g_z)}}_>>>>>>>{\quad\qquad\qquad\qquad\dcomment{\text{def. of } \funsection(!_X^*(\mathsf{zero}))}}_-{\qquad\qquad\qquad\qquad\qquad\qquad\qquad\qquad\qquad\qquad\dcomment{(c)}} && \ia {(\funsection(!_X^*(\mathsf{zero})))^*(A)} \ar[r]_-{\ia {\overline{\funsection(!_X^*(\mathsf{zero}))}(A)}} & \ia A
}
\]

Further, as $\eta^{1 \,\dashv\, \ia -}_X$ is an epimorphism (because it is an isomorphism according to Proposition~\ref{prop:compcatunitiso}), the commutativity of the outer perimeter of this diagram also implies that the square marked with $(c)$ commutes on its own. 

\pagebreak

Based on this last observation, the required commutativity of $(a)$ now follows from 
the commutativity of the following diagram:
\vspace{-1cm}
\[
\scriptsize
\xymatrix@C=6em@R=6em@M=0.5em{
\ar@{}[dddd]^<<<<<<<<<<<<<<{\quad\qquad\qquad\qquad\qquad\qquad\qquad\qquad\qquad\qquad\qquad\dscomment{\text{def. of } \mathsf{rec}(f_z,f_s)}}^<<<<<<<<<<<<<<<<<<<<<<<<<<<<<<{\qquad\qquad\qquad\qquad\qquad\qquad\qquad\dscomment{(c)}}^<<<<<<<<<<<<<<<<<<<<<<<<<<<<<<<<<<<<<<<<<<<<<<<<<<<<{\,\,\,\,\,\,\qquad\qquad\qquad\dscomment{\text{def. of } g_z}}^<<<<<<<<<<<<<<<<<<<<<<<<<<<<<<<<<<<<<<<<<<<<<<<<<{\qquad\qquad\qquad\qquad\qquad\qquad\qquad\qquad\qquad\qquad\dscomment{\text{def. of } \funsection(!_X^*(\mathsf{zero}))}}^<<<<<<<<<<<<<<<<<<<<<<<<<<<<<<<<<<<<<<<<<<<<<<<<<<<<<<<<<<<<<<<<<<<<<<<<<<<{\quad\qquad\qquad\dscomment{\text{def. of } (\eta^{1 \,\dashv\, \ia -}_X)^*(f_z^\dagger)}}^<<<<<<<<<<<<<<<<<<<<<<<<<<<<<<<<<<<<<<<<<<<<<<<<<<<<<<<<<<<<<<<<{\qquad\qquad\qquad\qquad\qquad\qquad\qquad\qquad\qquad\qquad\qquad\qquad\qquad\qquad\dscomment{\text{p is split a fibration}}} &
\\
\ia {!^*_X(\mathbb{N})} \ar[rr]^-{\eta^{1 \,\dashv\, \ia -}_{\ia {!^*_X(\mathbb{N})}}} \ar@/^3pc/[rrr]^-{\mathsf{rec}(f_z,f_s)} && \ia {1_{\ia {!^*_X(\mathbb{N})}}} \ar[r]^-{\ia {\mathsf{i}_A(g_z,g_s)}} & \ia {A}
\\
\ia {1_X} \ar[u]^-{\ia {!_X^*(\mathsf{zero})}} \ar[r]^-{\ia {g_z}} \ar[d]^-{=} & \ia {(\funsection(!_X^*(\mathsf{zero})))^*(A)} \ar[urr]^-{\ia {\overline{\funsection(!_X^*(\mathsf{zero}))}(A)}} \ar[d]_-{=}
\\
\ia {(\eta^{1 \,\dashv\, \ia -}_{X})^*(1_{\ia {1_X}})} \ar[r]^-{\ia {(\eta^{1 \,\dashv\, \ia -}_X)^*(f_z^\dagger)}} \ar[d]_-{\ia {\overline{\eta^{1 \,\dashv\, \ia -}_X}(1_{\ia {1_X}})}} & \ia {(\eta^{1 \,\dashv\, \ia -}_X)^*((\ia {!_X^*(\mathsf{zero})})^*(A))} \ar[d]^-{\ia {\overline{\eta^{1 \,\dashv\, \ia -}_X}((\ia {!_X^*(\mathsf{zero})})^*(A))}} \ar@/_2pc/[uurr]_-{\ia {\overline{\ia {!_X^*(\mathsf{zero})} \,\comp\, \eta^{1 \,\dashv\, \ia -}_X}(A)}}
\\
\ia {1_{\ia {1_X}}} \ar[r]_-{\ia {f_z^\dagger}} & \ia {(\ia {!_X^*(\mathsf{zero})})^*(A)} \ar@/_6pc/[uuurr]_-{\!\!\!\!\!\!\ia {\overline{\ia {!_X^*(\mathsf{zero})}}(A)}}
}
\]
and by observing that we have the following equations:
\[
f_z = \overline{\ia {!_X^*(\mathsf{zero})}}(A) \comp f_z^\dagger
\qquad
\ia {\overline{\eta^{1 \,\dashv\, \ia -}_X}(1_{\ia {1_X}})} = \ia {1(\eta^{1 \,\dashv\, \ia -}_X)} = \eta^{1 \,\dashv\, \ia -}_{\ia {1_X}}
\]
where the two equations on the right follow from $1$ being a split fibred functor, $\eta^{1 \,\dashv\, \ia -}_X$ being an isomorphism, and $\eta^{1 \,\dashv\, \ia -}$ being a natural transformation.

In order to show that the square marked with $(b)$ commutes, we again use Proposition~\ref{prop:reindexinginthebasecategory} to recast the equation
\[
\ia {!_X^*(\mathsf{succ})}^* (\mathsf{i}_A(g_z,g_s)) 
=
(\mathsf{s}(\mathsf{i}_A(f_z,f_s)))^*(g_s) 
\]
in $\mathcal{B}$. As a result, we get that the next diagram, marked with $(d)$, commutes in $\mathcal{B}$.
\[
\xymatrix@C=4em@R=2em@M=0.5em{
&& \ia {!^*_X(\mathbb{N})} \ar[r]^-{\eta^{1 \,\dashv\, \ia -}_{\ia {!^*_X(\mathbb{N})}}} & \ia {1_{\ia {!^*_X(\mathbb{N})}}} \ar[dd]^-{\ia {\mathsf{i}_A(g_z,g_s)}}_-{\dcomment{(d)}\qquad\qquad\qquad\qquad\qquad\,\,\,\,}
\\
&&
\\
\ia {!^*_X(\mathbb{N})} \ar[rr]_-{\mathsf{s}((\mathsf{s}(\mathsf{i}_A(g_z,g_s)))^*(g_s) )} \ar@/^2.5pc/[uurr]^-{\ia {!_X^*(\mathsf{succ})}} && \ia {\ia {!_X^*(\mathsf{succ})}^*(A)} \ar[r]_-{\ia {\overline{\ia {!_X^*(\mathsf{succ})}}(A)}} & \ia A
}
\]
Based on this observation, the commutativity of $(b)$ now follows from that of
\[
\scriptsize
\xymatrix@C=2.75em@R=6em@M=0.5em{
\ia {!^*_X(\mathbb{N})} \ar[rrr]^-{\mathsf{rec}(f_z,f_s)} &&& \ia {A}
\\
\ia {!^*_X(\mathbb{N})} \ar[u]^-{\ia {!_X^*(\mathsf{succ})}}_-{\qquad\qquad\qquad\qquad\qquad\qquad\qquad\qquad\qquad\dscomment{(d)}} \ar[rr]^-{\mathsf{s}((\mathsf{s}(\mathsf{i}_A(g_z,g_s)))^*(g_s) )} \ar[d]^-{\eta^{1 \,\dashv\, \ia -}_{\ia {!^*_X(\mathbb{N})}}}^-{\qquad\qquad\qquad\qquad\qquad\dscomment{\text{def. of } \mathsf{s}((\mathsf{s}(\mathsf{i}_A(g_z,g_s)))^*(g_s) )}} \ar@/_4pc/[ddddddd]^-{\id_{\ia {!^*_X(\mathbb{N})}}}^>>>>>>>>>>>>>>>>>>>>>>>>>>>>>>>>>>>>>>>>>>>>>>>>>>>>>>>>>>>>>>>{\,\,\dscomment{\eta^{1 \,\dashv\, \ia -}_{\ia {!^*_X(\mathbb{N})}} \text{ is iso.}}} &&  \ia {\ia {!_X^*(\mathsf{succ})}^*(A)} \ar[ur]^-{\ia {\overline{\ia {!_X^*(\mathsf{succ})}}(A)}} \ar[d]_-{=} 
\\
\ia {1_{\ia {!^*_X(\mathbb{N})}}} \ar[d]_-{=}^-{\quad\qquad\qquad\dscomment{\text{def. of } (\mathsf{s}(\mathsf{i}_A(g_z,g_s)))^*(g_s)}} \ar[rr]^-{\ia {(\mathsf{s}(\mathsf{i}_A(g_z,g_s)))^*(g_s)}} && \ia {(\mathsf{s}(\mathsf{i}_A(g_z,g_s)))^*(\pi^*_A(\ia {!_X^*(\mathsf{succ})}^*(A)))} \ar[d]_-{\ia {\overline{\mathsf{s}(\mathsf{i}_A(g_z,g_s))}(\pi^*_A(\ia {!_X^*(\mathsf{succ})}^*(A)))}}^<<<<<<<{\quad\dscomment{p \text{ is a s. fib.}}}
\\
\ia {(\mathsf{s}(\mathsf{i}_A(g_z,g_s)))^*(1_{\ia A})} \ar[d]_-{=}^-{\quad\qquad\dscomment{\mathcal{P}(\overline{\mathsf{s}(\mathsf{i}_A(g_z,g_s))}(1_{\ia A}))}} \ar[r]^-{\ia {\overline{\mathsf{s}(\mathsf{i}_A(g_z,g_s))}(1_{\ia A})}} & \ia {1_{\ia A}} \ar[r]^-{\ia {g_s}} \ar[d]_-{=}^-{\qquad\qquad\qquad\dscomment{\text{def. of } g_s}} & \ia {\pi^*_A(\ia {!_X^*(\mathsf{succ})}^*(A))} \ar@/_5.75pc/[uu]_<<<<<<<{\,\,\,\,\ia {\overline{\pi_A}(\ia {!_X^*(\mathsf{succ})}^*(A))}} 
\\
\ia {1_{\ia {!^*_X(\mathbb{N})}}} \ar[dddd]^-{\pi_{1_{\ia {!^*_X(\mathbb{N})}}}} & \ia {\pi^*_A(1_{\ia {!^*_X(\mathbb{N})}})} \ar@/^3pc/[d]^-{\ia {\eta^{\Sigma_A \,\dashv\, \pi^*_A}_{\pi^*_A(1_{\ia {!^*_X(\mathbb{N})}})}}}_>>>>>{\dscomment{\text{def. of } \kappa_{A,\pi^*_A(1_{\ia {!^*_X(\mathbb{N})}})}}\,\,\,\,\,} \ar@/_4pc/[dd]_>>>>>>>{\kappa_{A,\pi^*_A(1_{\ia {!^*_X(\mathbb{N})}})}\!\!\!\!\!} \ar@/_6.5pc/[ddd]_<<<<<<<<<<<<<<<{\id_{\ia {\pi^*_A(1_{\ia {!^*_X(\mathbb{N})}})}}\!\!\!\!\!\!\!\!}
\\
& \ia {\pi^*_A(\Sigma_A(\pi^*_A(1_{\ia {!^*_X(\mathbb{N})}})))} \ar[r]^-{\ia {\pi^*_A(\mathsf{fst})}} \ar[d]^-{\ia {\overline{\pi_A}(\Sigma_A(\pi^*_A(1_{\ia {!^*_X(\mathbb{N})}})))}}^<<<{\qquad\qquad\qquad\dscomment{\text{def. of } \pi^*_A(\mathsf{fst})}} & \ia {\pi^*_A(A)} \ar[uu]^-{\ia {\pi^*_A(f_s^\dagger)}}_-{\qquad\qquad\dscomment{\text{def. of } \pi^*_A(f_s^\dagger)}} \ar[r]^-{\ia {\overline{\pi_A}(A)}} & \ia {A} \ar@/_5.5pc/[uuuul]_>>>>>>>>>>>>>>>>>>>>>>{\ia {f_s^\dagger}} \ar@/_2.75pc/[uuuuu]^>>>>>>>>>>>>>>>>>>>>>>>>>{\ia {f_s}}^>>>>>>>>>>>>>>>>>>>{\dscomment{\text{def. of } f_s^\dagger}\qquad\qquad}
\\
& \ia {\Sigma_A(\pi^*_A(1_{\ia {!^*_X(\mathbb{N})}}))} \ar[urr]^-{\ia {\mathsf{fst}}} \ar@/^1.5pc/[d]^-{\kappa_{A,\pi^*_A(1_{\ia {!^*_X(\mathbb{N})}})}^{-1}}_<<<<{\dscomment{\kappa_{A,\pi^*_A(1_{\ia {!^*_X(\mathbb{N})}})} \text{ is iso.}}\,\,\,}^<<<<{\quad\qquad\dscomment{\text{def. of } \mathsf{fst}}} &
\\
& \ia {\pi^*_A(1_{\ia {!^*_X(\mathbb{N})}})} \ar@/_2pc/[uurr]^-{\pi_{\pi^*_A(1_{\ia {!^*_X(\mathbb{N})}})}}
\\
\ia {!^*_X(\mathbb{N})} \ar@/_4.5pc/[uuurrr]_-{\mathsf{s}(\mathsf{i}_A(g_z,g_s))}
}
\]

\vspace{0.25cm}
\noindent \textbf{Definition~\ref{def:strongsplitfibredweaknaturals} implies weak natural numbers in~\cite{Ahman:FibredEffects}:} 

Given vertical morphisms
\[
f_z : 1_X \longrightarrow (\mathsf{s}(\mathsf{zero}))^*(A)
\qquad
f_s : 1_{\ia A} \longrightarrow \pi^*_A(\ia {!^*_X(\mathsf{succ})}^*(A))
\]
in $\mathcal{V}_X$ and $\mathcal{V}_{\ia A}$, respectively, we aim to define a vertical morphism
\[
\mathsf{i}_A(f_z,f_s) : 1_{\ia {!^*_X(\mathbb{N})}} \longrightarrow A
\]
in $\mathcal{V}_{\ia {!^*_X(\mathbb{N})}}$. 

First, we define morphisms
\[
g_z : 1_{\ia {1_X}} \longrightarrow A
\qquad
g_s : A \longrightarrow A
\]
in $\mathcal{V}$, over $\ia {!^*_X(\mathsf{zero})}$ and $\ia {!^*_X(\mathsf{succ})}$, respectively, by
\[
g_z \defeq \overline{\ia {!^*_X(\mathsf{zero})}}(A) \comp \pi^*_{1_X}(f_z)
\qquad
g_s \defeq \overline{\ia {!^*_X(\mathsf{succ})}}(A) \comp \varepsilon^{\Sigma_A \,\dashv\, \pi^*_A}_{\ia {!^*_X(\mathsf{succ})}^*(A)} \comp \Sigma_A(f_s) \comp \langle \id_A , ! \rangle
\]

Next, according to Definition~\ref{def:strongsplitfibredweaknaturals}, $g_z$ and $g_s$ induce a morphism
\[
\mathsf{rec}(g_z,g_s) : \ia {!^*_X(\mathbb{N})} \longrightarrow \ia {A}
\]
in $\mathcal{B}$, that is the section of $\pi_A$ and makes the following two squares commute:
\[
\xymatrix@C=5em@R=5em@M=0.5em{
\ia {1_X} \ar[r]^-{\ia {!_X^*(\mathsf{zero})}} \ar[d]_-{\eta^{1 \,\dashv\, \ia -}_{\ia {1_X}}} & \ia {!_X^*(\mathbb{N})} \ar[d]_-{\mathsf{rec}(g_z,g_s)} & \ia {!_X^*(\mathbb{N})} \ar[l]_-{\ia {!_X^*(\mathsf{succ})}} \ar[d]^-{\mathsf{rec}(g_z,g_s)}
\\
\ia {1_{\ia {1_X}}} \ar[r]_-{\ia {g_z}} & \ia {A} & \ia {A} \ar[l]^-{\ia {g_s}}
}
\]

As a result, we can now use $\mathsf{rec}(g_z,g_s)$ to define $\mathsf{i}_A(f_z,f_s)$ as
\[
\mathsf{i}_A(f_z,f_s) \defeq \mathsf{s}^{-1}(\mathsf{rec}(g_z,g_s))
\]

Finally, we need to prove that this definition of $\mathsf{i}_A(f_z,f_s)$ satisfies the two equations
\[
(\funsection(!_X^*(\mathsf{zero})))^*(\mathsf{i}_A(f_z,f_s)) = f_z
\qquad
\ia {!_X^*(\mathsf{succ})}^* (\mathsf{i}_A(f_z,f_s)) 
=
(\mathsf{s}(\mathsf{i}_A(f_z,f_s)))^*(f_s) 
\]
in $\mathcal{V}_X$ and $\mathcal{V}_{\ia {!^*_X(\mathbb{N})}}$, respectively, which we use later in Section~\ref{sect:interpretation} to show that the interpretation of eMLTT validates the $\beta$-equations for primitive recursion.

In order to prove that the first of these equations holds in $\mathcal{V}_X$, namely,  
\[
(\funsection(!_X^*(\mathsf{zero})))^*(\mathsf{i}_A(f_z,f_s)) = f_z
\]
we first recast the left-hand side of this equation in $\mathcal{B}$, using Proposition~\ref{prop:reindexinginthebasecategory} and the definition of $\mathsf{i}_A(f_z,f_s)$ from above. In particular, we get that $\mathsf{s}((\funsection(!_X^*(\mathsf{zero})))^*(\mathsf{i}_A(f_z,f_s)))$ is equal to the unique (unnamed) mediating morphism in the next pullback situation.
\[
\xymatrix@C=6em@R=4em@M=0.5em{
& \ia {!_X^*(\mathbb{N})} \ar[r]^-{\eta^{1 \,\dashv\, \ia -}_{\ia {!_X^*(\mathbb{N})}}} & \ia {1_{\ia {!_X^*(\mathbb{N})}}} \ar[d]^{\ia {\mathsf{s}^{-1}(\mathsf{rec}(g_z,g_s))}}_{\dcomment{(e)}\qquad\qquad\qquad\qquad}
\\
X \ar@/_2pc/[dr]_{\id_X} \ar[ur]^{\funsection(!_X^*(\mathsf{zero}))} \ar@{-->}[r] & \ia {(\funsection(!_X^*(\mathsf{zero})))^*(A)} \ar[d]_{\pi_{(\funsection(!_X^*(\mathsf{zero})))^*(A)}}^<{\,\big\lrcorner}_<<<{\dcomment{(f)}\qquad\qquad\qquad} \ar[r]^-{\ia {\overline{\funsection(!_X^*(\mathsf{zero}))}(A)}} & \ia A \ar[d]^{\pi_A}_{\dcomment{\mathcal{P}(\overline{\funsection(!_X^*(\mathsf{zero}))}(A))}\qquad\quad}
\\
& X \ar[r]_-{\funsection(!_X^*(\mathsf{zero}))} & \ia {!_X^*(\mathbb{N})}
}
\]

Now, recalling that $\mathsf{s}$ is an isomorphism, it suffices to show that the equation
\[
\mathsf{s}((\funsection(!_X^*(\mathsf{zero})))^*(\mathsf{i}_A(f_z,f_s))) = \mathsf{s}(f_z)
\]
holds in $\mathcal{B}$, in order to prove that the required equation holds in $\mathcal{V}_X$. We note that this equation  holds because $\mathsf{s}(f_z)$ satisfies the same universal property as the unique unnamed mediating morphism in the above pullback situation, i.e., setting the unnamed morphism to be $\mathsf{s}(f_z)$ makes $(e)$ and $(f)$ commute. 

First, we show the commutativity of $(e)$ by
\[
\scriptsize
\xymatrix@C=5.5em@R=7em@M=0.5em{
\ia {!^*_X(\mathbb{N})} \ar[rrr]^{\eta^{1 \,\dashv\, \ia -}_{\ia {!^*_X(\mathbb{N})}}} \ar@/^8pc/[ddddrrr]^-{\mathsf{rec}(g_z,g_s)} &&& \ia {1_{\ia {!_X^*(\mathbb{N})}}} \ar[dddd]^{\ia {\mathsf{s}^{-1}(\mathsf{rec}(g_z,g_s))}}_<<<<<<<{\dscomment{\text{def. of } \mathsf{s}(\mathsf{s}^{-1}(\mathsf{rec}(g_z,g_s)))}\qquad\qquad}
\\
& & \ia {1_{\ia {1_X}}} \ar[d]^-{=}_>>>{\dscomment{(g)}\qquad} \ar[dddr]^-{\ia {g_z}}_>>>>>>>>>>>>>>>>>>>>>>>>>>>>>>>>>>{\dscomment{\text{def. of } g_z}\qquad}
\\
& & \ia {\pi^*_{1_X}(1_X)} \ar[dl]^<<<<<<<<<<{\ia {\overline{\pi_{1_X}}(1_X)}} \ar[d]^-{\ia {\pi^*_{1_X}(f_z)}}_>>>>{\dscomment{\text{def. of } \pi^*_{1_X}(f_z)}\qquad} &
\\
& \ia {1_X} \ar[dr]_-{\ia {f_z}} \ar[uuul]^-{\ia {!_X^*(\mathsf{zero})}}^<<<<<{\dscomment{\text{def. of } \funsection(!_X^*(\mathsf{zero}))}\quad}_>>>>>>>>>>>>>>>>>>>>>{\qquad\qquad\dscomment{\text{property of } \mathsf{rec}(g_z,g_s)}} \ar[uur]^-{\eta^{1 \,\dashv\, \ia -}_{\ia {1_X}}} & \ia {\pi^*_{1_X}((\funsection(!_X^*(\mathsf{zero})))^*(A))} \ar[d]_<<{\ia {\overline{\pi_{1_X}}((\funsection(!_X^*(\mathsf{zero})))^*(A))}\!\!\!}^-{\!\!\!\!\!\quad\dscomment{p \text{ is a s. fib.}}} \ar[dr]_>>>>>>>>>>{\ia {\overline{\funsection(!_X^*(\mathsf{zero})) \,\comp\, \pi_{1_X} }(A)}\,\,\,\,\,\,\,\,\,}
\\
X \ar[rr]_-{\mathsf{s}(f_z)} \ar[uuuu]^-{\funsection(!_X^*(\mathsf{zero}))} \ar[ur]^-{\eta^{1 \,\dashv\, \ia -}_X\!\!\!\!\!\!}_-{\,\,\,\,\qquad\dscomment{\text{def. of } \mathsf{s}(f_z)}} && \ia {(\funsection(!_X^*(\mathsf{zero})))^*(A)} \ar[r]_-{\ia {\overline{\funsection(!_X^*(\mathsf{zero}))}(A)}} & \ia {A}
}
\]
where we show the commutativity of $(g)$ by
\[
\xymatrix@C=6em@R=6em@M=0.5em{
\ia {1_{\ia {1_X}}} \ar[rr]^-{=} \ar@/^4.5pc/[ddrr]^-{\ia {1(\pi_{1_X})}} &&  \ia {\pi^*_{1_X}(1_X)} \ar[dd]^-{\ia {\overline{\pi_{1_X}}(1_X)}}_<<<<{\dcomment{1 \text{ is split fibred}}\qquad}
\\
& X \ar[dr]^-{\eta^{1 \,\dashv\, \ia -}_X}_-{\dcomment{\eta^{1 \,\dashv\, \ia -}_X \text{ is an iso.}}\qquad} &
\\
\ia {1_X} \ar[uu]^-{\eta^{1 \,\dashv\, \ia -}_{\ia {1_X}}}_>>>>>>>>>>>>>>>>{\qquad\qquad\dcomment{\text{nat. of } \eta^{1 \,\dashv\, \ia -}_{\ia {1_X}}}} \ar[ur]^-{\pi_{1_X}} \ar[rr]_-{\id_{\ia {1_X}}} && \ia {1_X}
}
\]

Second, the commutativity of $(f)$ follows directly from the definition of $\mathsf{s}$. Namely, by definition, the morphism $\mathsf{s}(f_z) : X \longrightarrow \ia {(\funsection(!_X^*(\mathsf{zero})))^*(A)}$ is a section of the projection morphism $\pi_{(\funsection(!_X^*(\mathsf{zero})))^*(A)} : \ia {(\funsection(!_X^*(\mathsf{zero})))^*(A)} \longrightarrow X$, giving us the equation
\[
\pi_{(\funsection(!_X^*(\mathsf{zero})))^*(A)} \comp \mathsf{s}(f_z) =  \id_X
\]

In order to prove that the second required equation holds in $\mathcal{V}_{\ia {!^*_X(\mathbb{N})}}$, namely, 
\[
\ia {!_X^*(\mathsf{succ})}^* (\mathsf{i}_A(f_z,f_s)) 
=
(\mathsf{s}(\mathsf{i}_A(f_z,f_s)))^*(f_s) 
\]
we first recast the left-hand side of this equation in $\mathcal{B}$, using Proposition~\ref{prop:reindexinginthebasecategory} and the definition of $\mathsf{i}_A(f_z,f_s)$ from above. In particular, we get that $\mathsf{s}(\ia {!_X^*(\mathsf{succ})}^* (\mathsf{i}_A(f_z,f_s)))$ is equal to the unique (unnamed) mediating morphism in the next pullback situation.
\[
\xymatrix@C=6em@R=4em@M=0.5em{
& \ia {!_X^*(\mathbb{N})} \ar[r]^-{\eta^{1 \,\dashv\, \ia -}_{\ia {!_X^*(\mathbb{N})}}} & \ia {1_{\ia {!_X^*(\mathbb{N})}}} \ar[d]^{\ia {\mathsf{s}^{-1}(\mathsf{rec}(g_z,g_s))}}_{\dcomment{(h)}\qquad\qquad\qquad\qquad}
\\
\ia {!_X^*(\mathbb{N})} \ar@/_2pc/[dr]_{\id_{\ia {!_X^*(\mathbb{N})}}} \ar[ur]^{\ia {!_X^*(\mathsf{succ})}} \ar@{-->}[r] & \ia {\ia {!_X^*(\mathsf{succ})}^*(A)} \ar[d]_{\pi_{\ia {!_X^*(\mathsf{succ})}^*(A)}}^<{\,\big\lrcorner}_<<<{\dcomment{(i)}\qquad\qquad\qquad} \ar[r]^-{\ia {\overline{\ia {!_X^*(\mathsf{succ})}}(A)}} & \ia A \ar[d]^{\pi_A}_{\dcomment{\mathcal{P}(\overline{\ia {!_X^*(\mathsf{succ})}}(A))}\qquad\quad}
\\
& \ia {!_X^*(\mathbb{N})} \ar[r]_-{\ia {!_X^*(\mathsf{succ})}} & \ia {!_X^*(\mathbb{N})}
}
\]

Next, recalling that $\mathsf{s}$ is an isomorphism and combining this fact with the definition of $\mathsf{i}_A(f_z,f_s)$ from above, it suffices to show that the equation 
\[
\mathsf{s}(\ia {!_X^*(\mathsf{succ})}^* (\mathsf{i}_A(f_z,f_s)))
=
\mathsf{s}((\mathsf{rec}(g_z,g_s))^*(f_s))
\]
holds in $\mathcal{B}$, in order to prove that the required equation holds in $\mathcal{V}_{\ia {!^*_X(\mathbb{N})}}$. We note that this equation holds because $\mathsf{s}((\mathsf{rec}(g_z,g_s))^*(f_s) )$ satisfies the same universal property as the unique unnamed mediating morphism in the above pullback situation, i.e., setting the unnamed morphism to be $\mathsf{s}((\mathsf{rec}(g_z,g_s))^*(f_s))$ makes $(h)$ and $(i)$ commute. 

First, we show the commutativity of  $(h)$ by
\[
\scriptsize
\xymatrix@C=3.5em@R=8em@M=0.5em{
\ia {!^*_X(\mathbb{N})} \ar[rrrr]^-{\eta^{1 \,\dashv\, \ia -}_{\ia {!^*_X(\mathbb{N})}}} \ar@/^8.5pc/[dddddrrrr]^-{\mathsf{rec}(g_z,g_s)} &&&& \ia {1_{\ia {!^*_X(\mathbb{N})}}} \ar[ddddd]_<<<<<<<<<<<<<<<<<<<<<<<{\ia {\mathsf{s}^{-1}(\mathsf{rec}(g_z,g_s))}}_<<<<<<<<<<<{\dscomment{\text{def. of } \mathsf{s}(\mathsf{s}^{-1}(\mathsf{rec}(g_z,g_s)))}\qquad\qquad\qquad\qquad}
\\
& \ia {A} \ar@/^4.5pc/[ddddrrr]^-{\ia {g_s}}
\\
& \ia {1_{\ia {A}}} \ar@/^1pc/[dr]^{\ia {f_s}}
\\
& \ia {(\mathsf{rec}(g_z,g_s))^*(1_{\ia {A}})} \ar[u]_-{\ia {\overline{\mathsf{rec}(g_z,g_s)}(1_{\ia {A}})}} & \ia {\pi^*_A(\ia {!^*_X(\mathsf{succ})} ^*(A))} \ar@/^1pc/[ddrr]_<<<<<<<<<<<<{\ia {\overline{\ia {!^*_X(\mathsf{succ})} \,\comp\, \pi_A}(A)}}
\\
& \ia {1_{\ia {!^*_X(\mathbb{N})}}} \ar[dr]^-{\quad\ia {(\mathsf{rec}(g_z,g_s))^*(f_s)}} \ar[u]^-{=} & \ia {(\mathsf{rec}(g_z,g_s))^*(\pi^*_A(\ia {!^*_X(\mathsf{succ})} ^*(A)))} \ar[u]^-{\ia {\overline{\mathsf{rec}(g_z,g_s)}(\pi^*_A(\ia {!^*_X(\mathsf{succ})} ^*(A)))}}^>>>>>{\dscomment{\text{def. of } (\mathsf{rec}(g_z,g_s))^*(f_s)}\qquad\qquad\,\,} 
\\
\ia {!^*_X(\mathbb{N})} \ar[ur]^-{\!\!\!\!\!\eta^{1 \,\dashv\, \ia -}_{\ia {!^*_X(\mathbb{N})}}}_-{\,\,\,\qquad\quad\dscomment{\text{def. of } \mathsf{s}((\mathsf{rec}(g_z,g_s))^*(f_s))}} \ar[rr]_-{\mathsf{s}((\mathsf{rec}(g_z,g_s))^*(f_s))} \ar[uuuuu]_>>>>>>>>>>>>>>>{\ia {!^*_X(\mathsf{succ})}}_>>>>>>>>>>>>>>>>{\quad\qquad\qquad\dscomment{\text{property of } \mathsf{rec}(g_z,g_s)}} \ar@/^1.8pc/[uuuur]^>>>>>>>>>>>>>>>>>>{\mathsf{rec}(g_z,g_s)}_>>>>>>>>>>>>{\qquad\qquad\dscomment{(j)}} && \ia {\ia {!^*_X(\mathsf{succ})} ^*(A)} \ar[u]_-{=}_-{\qquad\qquad\dscomment{p \text{ is a split fibration}}} \ar[rr]_-{\ia {\overline{\ia {!^*_X(\mathsf{succ})}}(A)}} && \ia {A}
}
\vspace{1cm}
\]
where we show the commutativity of $(j)$ by
\[
\scriptsize
\xymatrix@C=6em@R=4em@M=0.5em{
\ia {!^*_X(\mathbb{N})} \ar[ddd]_-{\eta^{1 \,\dashv\, \ia -}_{\ia {!^*_X(\mathbb{N})}}}^-{\quad\qquad\dscomment{\text{nat. of } \eta^{1 \,\dashv\, \ia -}}} \ar[r]^-{\mathsf{rec}(g_z,g_s)} & \ia {A} \ar[rr]^-{\ia {g_s}} \ar[dddd]_-{\eta^{1 \,\dashv\, \ia -}_{\ia {A}}} \ar[dr]^-{\ia {\langle \id_A , ! \rangle}} && \ia {A}
\\
&& \ia {\Sigma_A(\pi^*_A(1_{\ia {!^*_X(\mathbb{N})}}))} \ar[d]^-{=}^-{\qquad\qquad\qquad\dscomment{\text{def. of } g_s}}_>{\dscomment{(k)}\qquad\qquad\qquad}
\\
&& \ia {\Sigma_A(1_{\ia {A}})} \ar[d]^-{\ia {\Sigma_A(f_s)}}
\\
\ia {1_{\ia {!^*_X(\mathbb{N})}}} \ar[d]_-{=}^>>>>>{\quad\dscomment{1 \text{ is s. fib.}}} \ar[dr]^-{\ia {1(\mathsf{rec}(g_z,g_s))}} && \ia {\Sigma_A(\pi^*_A(\ia {!^*_X(\mathsf{succ})}^*(A)))} \ar[dr]_<<<<<<<<<{\ia {\varepsilon^{\Sigma_A \,\dashv\, \pi^*_A}_{\ia {!^*_X(\mathsf{succ})}^*(A)}}}
\\
\ia {(\mathsf{rec}(g_z,g_s))^*(1_{\ia {A}})} \ar[r]_-{\ia {\overline{\mathsf{rec}(g_z,g_s)}(1_{\ia A})}} & \ia {1_{\ia {A}}} \ar[r]_-{\ia {f_s}} & \ia {\pi^*_A(\ia {!^*_X(\mathsf{succ})} ^*(A))} \ar[r]_-{\ia {\overline{\pi_A}(\ia {!^*_X(\mathsf{succ})} ^*(A))}} & \ia {\ia {!^*_X(\mathsf{succ})} ^*(A)} \ar[uuuu]^-{\overline{\ia {!^*_X(\mathsf{succ})}}(A)}
}
\]
and the commutativity of $(k)$ by
\[
\scriptsize
\xymatrix@C=5.75em@R=6em@M=0.5em{
\ia {A} \ar[rr]^-{\ia {\langle \id_A , ! \rangle}} \ar[dd]_-{\eta^{1 \,\dashv\, \ia -}_{\ia A}}^-{\quad\dscomment{\eta^{1 \,\dashv\, \ia -}_{\ia A} \text{ satisfies same univ. prop. as } h}}^-{\qquad\qquad\qquad\qquad\qquad\qquad\qquad\qquad\qquad\qquad\qquad\qquad\dscomment{\text{def. of } \kappa_{A,\pi^*_A(1_{\ia {!^*_X(\mathbb{N})}})}}} \ar[dr]^-{h} \ar@{}[d]^<<<<<<<<{\!\!\!\!\qquad\qquad\qquad\qquad\qquad\qquad\qquad\qquad\dscomment{\text{def. of } \langle \id_A , ! \rangle}} && \ia {\Sigma_A(\pi^*_A(1_{\ia {!^*_X(\mathbb{N})}}))} \ar[d]_-{\id_{\ia {\Sigma_A(\pi^*_A(1_{\ia {!^*_X(\mathbb{N})}}))}}}
\\
& \ia {\pi^*_A(1_{\ia {!^*_X(\mathbb{N})}})} \ar[ur]^-{\kappa_{A,\pi^*_A(1_{\ia {!^*_X(\mathbb{N})}})}} \ar[dl]^-{=} \ar[d]_>>>>>>>{\ia {\eta^{\Sigma_A \,\dashv\, \pi^*_A}_{\ia {\pi^*_A(1_{\ia {!^*_X(\mathbb{N})}})}}}\!\!\!} \ar@{}[dd]^-{\quad\qquad\qquad\qquad\dscomment{1 \text{ is split fibred}}}_-{\dscomment{1 \text{ is split fibred}}\quad\qquad\qquad\qquad} & \ia {\Sigma_A(\pi^*_A(1_{\ia {!^*_X(\mathbb{N})}}))} \ar[d]^-{=}
\\
\ia {1_{\ia A}} \ar[dd]_-{\ia {f_s}} \ar[dr]_-{\ia {\eta^{\Sigma_A \,\dashv\, \pi^*_A}_{1_{\ia A}}}} & \ia {\pi^*_A(\Sigma_A(\pi^*_A(1_{\ia {!^*_X(\mathbb{N})}})))} \ar[ur]_-{\qquad\ia {\overline{\pi_A}(\Sigma_A(\pi^*_A(1_{\ia {!^*_X(\mathbb{N})}})))}} \ar[d]_-{=} & \ia {\Sigma_A (1_{\ia A})} \ar[ddd]^-{\ia {\Sigma_A (f_s)}}
\\
& \ia {\pi^*_A(\Sigma_A(1_{\ia A}))} \ar[ur]_-{\quad\ia {\overline{\pi_A}(\Sigma_A(1_{\ia A}))}} \ar[d]^-{\ia {\pi^*_A(\Sigma_A(f_s))}}^-{\qquad\qquad\qquad\qquad\dscomment{\text{def. of } \pi^*_A(\Sigma_A(f_s))}}_<<<<<{\dscomment{\text{nat. of } \eta^{\Sigma_A \,\dashv\, \pi^*_A}}\qquad\qquad\qquad\quad} &
\\
\ia {\pi^*_A(\ia {!^*_X(\mathsf{succ})} ^*(A))} \ar[d]_-{\id_{\ia {\pi^*_A(\ia {!^*_X(\mathsf{succ})} ^*(A))}}}^<<<<<{\qquad\quad\dscomment{\Sigma \dashv \pi^*_A}} \ar[r]^-{\ia {\eta^{\Sigma_A \,\dashv\, \pi^*_A}_{\pi^*_A(\ia {!^*_X(\mathsf{succ})} ^*(A))}}} & \ia {\ia {\pi^*_A(\Sigma_A(\pi^*_A(\ia {!^*_X(\mathsf{succ})}^*(A))))}} \ar[dr]_>>>>>>>>>>>>>>>>{\ia {\overline{\pi_A}(\Sigma_A(\pi^*_A(\ia {!^*_X(\mathsf{succ})}^*(A))))}\qquad\quad} \ar[dl]^>>>>>>>>>>>>>>>>>>{\ia {\pi^*_A(\varepsilon^{\Sigma_A \,\dashv\, \pi^*_A}_{\ia {!^*_X(\mathsf{succ})}^*(A)})}}^<<<<<<<{\qquad\qquad\dscomment{\text{def. of } \pi^*_A(\varepsilon^{\Sigma_A \,\dashv\, \pi^*_A}_{\ia {!^*_X(\mathsf{succ})}^*(A)})}} &
\\
\ia {\pi^*_A(\ia {!^*_X(\mathsf{succ})} ^*(A))} \ar[r]_-{\ia {\overline{\pi_A}(\ia {!^*_X(\mathsf{succ})} ^*(A))}} & \ia {\ia {!^*_X(\mathsf{succ})} ^*(A)} & \ia {\Sigma_A(\pi^*_A(\ia {!^*_X(\mathsf{succ})}^*(A)))} \ar[l]^-{\ia {\varepsilon^{\Sigma_A \,\dashv\, \pi^*_A}_{\ia {!^*_X(\mathsf{succ})}^*(A)}}}
}
\]

\pagebreak\noindent
and where $h$ is defined as the unique mediating morphism into the pullback square given by $\mathcal{P}(\overline{\pi_A}(1_{\ia {!^*_X(\mathbb{N})}}))$, for $\ia {!} : \ia {A} \longrightarrow \ia {1_{\ia {!_X^*(\mathbb{N})}}}$ and $\id_{\ia A} : \ia {A} \longrightarrow \ia {A}$. The proof that $\eta^{1 \,\dashv\, \ia -}_{\ia A}$ is equal to $h$ can be found in the proof of Proposition~\ref{prop:semsubstintoweakenedterm}.

Finally, we note that the commutativity of $(i)$ follows directly from the definition of $s$. In particular, we know by definition that the morphism 
\[
\mathsf{s}((\mathsf{rec}(g_z,g_s))^*(f_s)) : \ia {!^*_X(\mathbb{N})} \longrightarrow \ia {\ia {!^*_X(\mathsf{succ})}^*(A)}
\]
is a section of $\pi_{\ia {!^*_X(\mathsf{succ})}^*(A)}$, giving us the required equation
\[
\pi_{\ia {!^*_X(\mathsf{succ})}^*(A)} \comp \mathsf{s}((\mathsf{rec}(g_z,g_s))^*(f_s)) = \id_{\ia {!^*_X(\mathbb{N})}}
\]
\end{proof}

\section{Proof of Proposition~\ref{prop:strengthofsplitfibredmonads}}
\label{sect:proofofprop:strengthofsplitfibredmonads}


{
\renewcommand{\thetheorem}{\ref{prop:strengthofsplitfibredmonads}}
\begin{proposition}
Given a split comprehension category with unit $p : \mathcal{V} \longrightarrow \mathcal{B}$ with strong split dependent sums and a split fibred monad $\mathbf{T} = (T,\eta,\mu)$ on it, then there exists a family of natural transformations 
\[
\sigma_A : \Sigma_A \comp T \longrightarrow T \comp \Sigma_A \qquad\qquad\qquad (A \in \mathcal{V})
\]
collectively called the \emph{dependent strength} of $\mathbf{T}$, satisfying the diagrams $(1)$--$(4)$.
\end{proposition}
\addtocounter{theorem}{-1}
}

\noindent
\textit{Proof.} 
Before proving that the four diagrams $(1)$--$(4)$ given in the proposition commute, we first show that the natural transformation $\alpha_{A,B}$ given in the proposition \linebreak is indeed a natural isomorphism. To this end, we first define its candidate inverse \linebreak 
$\alpha^{-1}_{A,B} : \Sigma_A \comp \Sigma_B \comp \kappa^*_{A,B} \longrightarrow \Sigma_{\Sigma_A(B)}$ as the following composite natural transformation:
\[
\xymatrix@C=13em@R=6em@M=0.5em{
\Sigma_A \comp \Sigma_B \comp \kappa^*_{A,B} \ar[r]^-{\Sigma_A \,\comp\, \Sigma_B \,\comp\, \kappa^*_{A,B} \,\comp\, \eta^{\Sigma_{\Sigma_A(B)} \,\dashv\, \pi^*_{\Sigma_A(B)}}} & \Sigma_A \comp \Sigma_B \comp \kappa^*_{A,B} \comp \pi^*_{\Sigma_A(B)} \comp \Sigma_{\Sigma_A(B)} \ar[d]^-{=}
\\
\Sigma_{\Sigma_A(B)}& \Sigma_A \comp \Sigma_B \comp \pi^*_B \comp \pi^*_A \comp \Sigma_{\Sigma_A(B)} \ar[l]^-{\varepsilon^{\Sigma_A \,\comp \Sigma_B \,\dashv\, \pi^*_B \,\comp\, \pi^*_A} \,\comp\, \Sigma_{\Sigma_A(B)}}
}
\]

\pagebreak

For better readability, we often omit the subscripts from $\kappa^*_{A,B}$ and $(\kappa^{-1}_{A,B})^*$ in the diagrams given below. For the same reason, we also often abbreviate the four functors $\pi^*_B \comp \pi^*_A$, $\Sigma_A \comp \Sigma_B$, $\pi^*_{\Sigma_A(B)}$, and $\Sigma_{\Sigma_A(B)}$ as $\pi^*$, $\Sigma$, $\pi'^*$, and $\Sigma'$, respectively. 

We also note that most of the equality morphisms used in the diagrams given below are induced by reindexing along the following commuting diagram:
\[
\xymatrix@C=6em@R=7em@M=0.5em{
\ia {B} \ar[r]_-{\ia {\eta^{\Sigma_A \,\dashv\, \pi^*_A}_B}} \ar[dr]_-{\pi_B} \ar@/^2.25pc/[rr]^-{\kappa_{A,B}}_{\dcomment{\text{def. of } \kappa_{A,B}}} & \ia {\pi^*_A(\Sigma_A(B))} \ar[r]_-{\ia {\overline{\pi_A}(\Sigma_A(B))}} \ar[d]^-{\pi_{\pi^*_A(\Sigma_A(B))}}_<<<<<<<<<<{\dcomment{\mathcal{P}(\eta^{\Sigma_A \,\dashv\, \pi^*_A}_B)}\,\,\,\,\,\,} & \ia {\Sigma_A(B)} \ar[d]^-{\pi_{\Sigma_A(B)}}_-{\dcomment{\mathcal{P}(\overline{\pi_A}(\Sigma_A(B)))}\quad\,\,\,\,\,} \ar@/_4.8pc/[ll]_-{\kappa^{-1}_{A,B}}^-{\dcomment{\kappa_{A,B} \text{ is an iso.}}}
\\
& \ia A \ar[r]_-{\pi_A} & p(A)
}
\]

We now return to proving that $\alpha_{A,B}$ is a natural isomorphism, by showing that the following two equations hold:
\[
\begin{array}{c}
\alpha^{-1}_{A,B} \comp \alpha_{A,B} = \id_{\Sigma_{\Sigma_A(B)}}
\\[5mm]
\alpha_{A,B} \comp \alpha^{-1}_{A,B} = \id_{\Sigma_A \,\comp\, \Sigma_B \,\comp\, \kappa^*_{A,B}}
\end{array}
\]
We prove these equations by showing that the following two diagrams commute:


\mbox{}

\[
\xymatrix@C=5em@R=6em@M=0.5em{
\Sigma' \ar[r]^-{=} \ar[dddddd]_-{\id_{\Sigma'}}^>>>>>>>>>>>>>>>>>>>>>>>>>>>>>>>>{\qquad\quad\dcomment{\Sigma' \,\dashv\, \pi'^*}} \ar@/_8pc/[dddddr]^<<<<<<<<<<<<<<<<<<<<<<<<{\eta^{\Sigma' \,\dashv\, \pi'^*} \,\comp\, \Sigma'} & \Sigma' \comp (\kappa^{-1})^* \comp \kappa^* \ar[r]^-{\Sigma' \,\comp\, (\kappa^{-1})^* \,\comp\, \eta^{\Sigma \,\dashv\, \pi^*} \,\comp\, \kappa^*} \ar[d]^<<<<<{\Sigma_A \,\comp\, (\kappa^{-1})^* \,\comp\, \kappa^* \comp \eta^{\Sigma' \,\dashv\, \pi'^*}}_-{\dcomment{p \text{ is a split fibration}}\qquad}^-{\qquad\qquad\dcomment{\text{nat. of } \eta^{\Sigma \,\dashv\, \pi^*}}}^>>>>>{\!\!\!\qquad\qquad\Sigma' \,\comp\, (\kappa^{-1})^* \,\comp\, \pi^* \,\comp\, \Sigma \,\comp\, \kappa^* \,\comp\, \eta^{\Sigma' \,\dashv\, \pi'^*}} & \Sigma' \comp (\kappa^{-1})^* \comp \pi^* \comp \Sigma \comp \kappa^* \ar[dd]^-{=} \ar@/^4pc/[ddl]
\\
& \Sigma' \comp (\kappa^{-1})^* \comp \kappa^* \comp \pi'^* \comp \Sigma' \ar@/_0.5pc/[d]^<<<<<<{\Sigma' \,\comp\, (\kappa^{-1})^* \,\comp\, \eta^{\Sigma \,\dashv\, \pi^*} \,\comp\, \kappa^* \,\comp\, \pi'^* \,\comp\, \Sigma'} \ar@/_6pc/[dddd]^-{=} & 
\\
& \txt<7pc>{$\Sigma' \comp (\kappa^{-1})^* \comp \pi^* \comp$\\$\Sigma \comp \kappa^* \comp\pi'^* \comp \Sigma'$}  \ar[d]_{=}_>>>>>>>{\dcomment{p \text{ is a s. fib.}}\quad\!\!\!}^<<<{\qquad\qquad\dcomment{p \text{ is a split fibration}}} & \Sigma' \comp \pi'^* \comp \Sigma \comp \kappa^* \ar[d]^-{\varepsilon^{\Sigma' \,\dashv\, \pi'^*} \,\comp\, \Sigma \,\comp\, \kappa^*}_>>>>>{\dcomment{\text{nat. of } \varepsilon^{\Sigma' \,\dashv\, \pi'^*}}\quad} \ar@/^2pc/[dl]_-{\Sigma' \,\comp\, \pi'^* \,\comp\, \Sigma \,\comp\, \kappa^* \,\comp\, \eta^{\Sigma' \,\dashv\, \pi'^*}}
\\
& \txt<6pc>{$\Sigma' \comp \pi'^* \comp$\\$\Sigma \comp \kappa^* \comp\pi'^* \comp \Sigma'$} \ar@/^1.75pc/[ddr]^<<<<<<<{\,\,\,\,\,\,\,\,\varepsilon^{\Sigma' \,\dashv\, \pi'^*} \,\comp\, \Sigma \,\comp\, \kappa^* \,\comp\, \pi'^* \,\comp\, \Sigma'} \ar[d]_-{=}_<<<<<{\dcomment{\Sigma \,\dashv\, \pi^*}\quad\,\,\,\,\,\,}^>>>{\qquad\dcomment{p \text{ is a s. fib.}}} & \Sigma \comp \kappa^* \ar[dd]^-{\Sigma \,\comp\, \kappa^* \,\comp\, \eta^{\Sigma' \,\dashv\, \pi'^*}}
\\
& \txt<4.5pc>{$\Sigma' \comp \pi'^* \comp$\\$\Sigma \comp \pi^* \comp \Sigma'$} \ar@/^2pc/[ddr]_>>>>>>>>>{\varepsilon^{\Sigma' \,\dashv\, \pi'^*} \,\comp\, \Sigma \,\comp\, \pi^* \,\comp\, \Sigma'} \ar[d]^>>>>>>{\Sigma' \,\comp\, \pi'^* \,\comp\, \varepsilon^{\Sigma \,\dashv\, \pi^*} \,\comp\, \Sigma'} & 
\\
& \Sigma' \comp \pi'^* \comp \Sigma' \ar[dl]^-{\varepsilon^{\Sigma' \,\dashv\, \pi'^*} \,\comp\, \Sigma'}^<<<<<<{\qquad\qquad\quad\dcomment{\text{nat. of } \varepsilon^{\Sigma' \,\dashv\, \pi'^*}}} & \Sigma \comp \kappa^* \comp \pi'^* \comp \Sigma' \ar[d]^-{=}
\\
\Sigma' & & \Sigma \comp \pi^* \comp \Sigma' \ar[ll]^-{\varepsilon^{\Sigma \,\dashv\, \pi^*} \,\comp\, \Sigma'}
}
\]

\pagebreak

\mbox{}

\vspace{-1cm}

\[
\xymatrix@C=5em@R=5em@M=0.5em{
\Sigma \comp \kappa^* \ar[r]^-{\Sigma \,\comp\, \kappa^* \,\comp\, \eta^{\Sigma' \,\dashv\, \pi'^*}} \ar[ddddddd]_-{\id_{\Sigma \,\comp\, \kappa^*}} \ar@/_3pc/[ddddddr]_>>>>>>>>>>>>>>>>>>>>>>>>>>>>>>{\Sigma \,\comp\, \eta^{\Sigma \,\dashv\, \pi^*} \,\comp\, \kappa^*}_>{\dcomment{\Sigma \,\dashv\, \pi^*}\qquad\qquad\quad} & \Sigma \comp \kappa^* \comp \pi'^* \comp \Sigma' \ar[r]^-{=} \ar[d]^-{=}^<<<<{\quad\dcomment{p \text{ is a split fibration}}}_-{\dcomment{(a)}\qquad\qquad\quad\!\!\!\!} & \Sigma \comp \pi^* \comp \Sigma' \ar[d]^{\varepsilon^{\Sigma \,\dashv\, \pi^*} \,\comp\, \Sigma'} \ar[dl]^-{=}
\\
& \Sigma \comp \pi^* \comp \Sigma' \comp (\kappa^{-1})^* \comp \kappa^* \ar@/^2pc/[ddr]_<<<<<<<<<<<<<<<<<<<<<{\varepsilon^{\Sigma \,\dashv\, \pi^*} \,\comp\, \Sigma' \,\comp\, (\kappa^{-1})^* \,\comp\, \kappa^*} \ar[dd]^>>>>>>>>{\Sigma \,\comp\, \pi^* \,\comp\, \Sigma' \,\comp\, (\kappa^{-1})^* \,\comp\, \eta^{\Sigma \,\dashv\, \pi^*} \,\comp\, \kappa^*}^>>>>>>>>>>>>>>>>>{\qquad\quad\dcomment{\text{nat. of } \varepsilon^{\Sigma \,\dashv\, \pi^*}}} & \Sigma' \ar[dd]^{=}_<<<<{\dcomment{p \text{ is a split fibration}}\qquad}
\\
& &
\\
& \txt<7pc>{$\Sigma \comp \pi^* \comp \Sigma' \comp $\\$ (\kappa^{-1})^* \comp \pi^* \comp \Sigma \comp \kappa^*$} \ar@/_2pc/[ddr]^<<<<<<<<<{\varepsilon^{\Sigma \,\dashv\, \pi^*} \,\comp\, \Sigma' \,\comp\, (\kappa^{-1})^* \,\comp\, \pi'^* \,\comp\, \Sigma' \,\comp\, \kappa^*} \ar@/_1pc/[d]^-{=} & \Sigma' \comp (\kappa^{-1})^* \comp \kappa^* \ar[dd]_-{\Sigma' \,\comp\, (\kappa^{-1})^* \,\comp\, \eta^{\Sigma \,\dashv\, \pi^*} \,\comp\, \kappa^*}
\\
& \txt<5pc>{$\Sigma \comp \pi^* \comp \Sigma' \comp $\\$ \pi'^* \comp \Sigma \comp \kappa^*$} \ar@/^1.75pc/[dddr]_>>>>>>>>>>>>{\varepsilon^{\Sigma \,\dashv\, \pi^*} \,\comp\, \Sigma' \,\comp\, \pi'^* \,\comp\, \Sigma \,\comp\, \pi^*} \ar[dd]^>>>>>>>>>>>{\Sigma \,\comp\, \pi^* \,\comp\, \varepsilon^{\Sigma' \,\dashv\, \pi'^*} \,\comp\, \Sigma \,\comp\, \kappa^*} &
\\
& & \Sigma' \comp (\kappa^{-1})^* \comp \pi^* \comp \Sigma \comp \kappa^* \ar[dd]^-{=}_<<<<<{\dcomment{p \text{ is a s. fib.}}\,\,\,\,\,\,\,}
\\
& \Sigma \comp \pi^* \comp \Sigma \comp \kappa^* \ar[dl]^-{\varepsilon^{\Sigma \,\dashv\, \pi^*} \,\comp\, \Sigma \,\comp\, \kappa^*}^<<<<<<<<<<<{\qquad\qquad\quad\dcomment{\text{nat. of } \varepsilon^{\Sigma \,\dashv\, \pi^*}}} & 
\\
\Sigma \comp \kappa^* & & \Sigma' \comp \pi'^* \comp \Sigma \comp \kappa^* \ar[ll]^-{\varepsilon^{\Sigma' \,\dashv\, \pi'^*} \,\comp\, \Sigma \,\comp\, \kappa^*}
}
\vspace{0.5cm}
\]

\noindent
We conclude the proof of these two isomorphism equations by showing that the subdiagram marked with $(a)$ commutes. Its commutativity is proved as follows:

\[
\scriptsize
\xymatrix@C=4em@R=7em@M=0.5em{
\Sigma \comp \kappa^* \comp \pi'^* \comp \Sigma' \ar[rrr]^-{=} \ar[drr]^-{=}^-{\qquad\qquad\qquad\qquad\qquad\qquad\dscomment{p \text{ is a split fibration}}} & & & \txt<4.75pc>{$\Sigma \comp \pi^* \comp \Sigma'  \comp $\\$ (\kappa^{-1})^* \comp \kappa^*$} \ar@/^2pc/[ddddd]_>>>>>>>>>>>>>>>>>>>>>>>>>>>>>>>>>>>>{\Sigma \,\comp\, \pi^* \,\comp\, \Sigma' \,\comp\, (\kappa^{-1})^* \,\comp\, \eta^{\Sigma \,\dashv\, \pi^*} \!\comp\, \kappa^*}
\\
\Sigma \comp \kappa^* \ar[dddddd]_-{\Sigma \,\comp\, \eta^{\Sigma \,\dashv\, \pi^*} \,\comp\, \kappa^*} \ar[u]^-{\Sigma \,\comp\, \kappa^* \,\comp\, \eta^{\Sigma' \,\dashv\, \pi'^*}}_<{\quad\qquad\qquad\qquad\dscomment{p \text{ is a split fibration}}} \ar[dr]_-{=} & & \txt<6pc>{$\Sigma \comp \kappa^* \comp \pi'^* \comp $\\$ \Sigma' \comp (\kappa^{-1})^* \comp \kappa^*$} \ar[ur]^-{=} \ar@/^5pc/[ddd]_<<<<<<<<<<<<<<<<<<<<<<<<<<{\Sigma \,\comp\, \kappa^* \,\comp\, \pi'^* \,\comp\, \Sigma' \,\comp\, (\kappa^{-1})^* \,\comp\, \eta^{\Sigma \,\dashv\, \pi^*} \,\comp\, \kappa^*}^-{\,\,\,\quad\dscomment{p \text{ is a split fibration}}} & 
\\
& \txt<4pc>{$\Sigma \comp \kappa^* \comp $\\$ (\kappa^{-1})^* \comp \kappa^*$} \ar[ur]^>>>>>>>>>>{\Sigma \,\comp\, \kappa^* \,\comp\, \eta^{\Sigma' \,\dashv\, \pi'^*} \,\comp\, (\kappa^{-1})^* \,\comp\, \kappa^*} \ar[d]^>>>>>>>{\Sigma \,\comp\, \kappa^* \,\comp\, (\kappa^{-1})^* \,\comp\, \eta^{\Sigma \,\dashv\, \pi^*} \,\comp\, \kappa^*}_-{\dscomment{p \text{ is a split fibration}}\qquad\,\,\,\,\,\,}^>>>>>>>>>>>>>{\qquad\qquad\qquad\quad\dscomment{\text{nat. of } \eta^{\Sigma' \,\dashv\, \pi'^*}}} & & 
\\
& \txt<4pc>{$\Sigma \comp \kappa^* \comp $\\$ (\kappa^{-1})^* \comp \pi^* \comp \Sigma \comp \kappa^*$} \ar@/_1pc/[dr]^<<<<{\Sigma \,\comp\, \kappa^* \,\comp\, \eta^{\Sigma' \,\dashv\, \pi'^*} \,\comp\, (\kappa^{-1})^*  \,\comp\, \pi^* \,\comp\, \Sigma \,\comp\, \kappa^*} \ar[d]^-{=}^>{\,\,\,\,\,\qquad\dscomment{p \text{ is a s. fib.}}} & &
\\
& \txt<4pc>{$\Sigma \comp \kappa^* \comp $\\$ \pi'^* \comp \Sigma \comp \kappa^*$} \ar@/^1pc/[dr]_>>>>>>{\Sigma \,\comp\, \kappa^* \,\comp\, \eta^{\Sigma' \,\dashv\, \pi'^*} \,\comp\, \pi'^* \,\comp\, \Sigma \,\comp\, \kappa^*\!\!\!\!} \ar@/_4pc/[ddr]_-{\id_{\Sigma \,\comp\, \kappa^* \,\comp\, \pi'^* \,\comp\, \Sigma \,\comp\, \kappa^*}} & \txt<6pc>{$\Sigma \comp \kappa^* \comp \pi'^* \comp $\\$ \Sigma' \comp (\kappa^{-1})^* \comp \pi^* \comp \Sigma \comp \kappa^*$} \ar[dr]^-{=} \ar[d]^-{=} &
\\
& & \txt<5pc>{$\Sigma \comp \kappa^* \comp $\\$ \pi'^* \comp \Sigma' \comp \pi'^* \comp \Sigma \comp \kappa^*$} \ar[d]^-{\Sigma \,\comp\, \kappa^* \,\comp\, \pi'^* \,\comp\, \varepsilon^{\Sigma' \,\dashv\, \pi'^*} \,\comp\, \Sigma \,\comp\, \kappa^*}_-{\dscomment{\Sigma' \,\dashv\, \pi'^*}\qquad} & \txt<6pc>{$\Sigma \comp \pi^* \comp \Sigma' \comp $\\$ (\kappa^{-1})^* \comp \pi^* \comp \Sigma \comp \kappa^*$} \ar[dd]^-{=}
\\
& & \txt<4.5pc>{$\Sigma \comp \kappa^* \comp $\\$ \pi'^* \comp \Sigma \comp \kappa^*$} \ar[dll]^-{=}^-{\qquad\qquad\qquad\qquad\qquad\qquad\qquad\qquad\dscomment{p \text{ is a split fibration}}} & 
\\
\Sigma \comp \pi^* \comp \Sigma \comp \kappa^* & & & \txt<5pc>{$\Sigma \comp \pi^* \comp \Sigma' \comp $\\$ \pi'^* \comp \Sigma \comp \kappa^*$} \ar[lll]^-{\Sigma \,\comp\, \pi^* \,\comp\, \varepsilon^{\Sigma' \,\dashv\, \pi'^*} \,\comp\, \Sigma \,\comp\, \kappa^*}
}
\]

Next, we show that the four diagrams $(1)$--$(4)$ commute. We again omit the subscripts from $\kappa^*_{A,B}$ and $(\kappa^{-1}_{A,B})^*$, and abbreviate the functors $\pi^*_B \comp \pi^*_A$, $\Sigma_A \comp \Sigma_B$, $\pi^*_{\Sigma_A(B)}$, and $\Sigma_{\Sigma_A(B)}$ as $\pi^*$, $\Sigma$, $\pi'^*$, and $\Sigma'$, respectively. In order to further optimise the size of the proof of diagram $(2)$, we instead prove the commutativity of an equivalent diagram, in which we have replaced $\alpha_{A,B,T(C)}$ and $T(\alpha_{A,B,C})$ with their respective inverses.

First, diagram $(1)$ commutes because we have
\[
\scriptsize
\xymatrix@C=8em@R=7em@M=0.5em{
\Sigma_{1_{p(A)}}(\pi^*_{1_{p(A)}}(T(A))) \ar@/_9pc/[ddddrr]_-{\varepsilon^{\Sigma_{1_{p(A)}} \!\!\dashv\, \pi^*_{1_{p(A)}}}_{T(A)}\!\!\!\!\!\!\!\!\!\!} \ar@/_3.5pc/[dddr]^>>>>>>>>>>>>>{\!\!\!\!\!\!\!\!\!\!\!\id_{\Sigma_{1_{p(A)}}(\pi^*_{1_{p(A)}}(T(A)))}} \ar@/_3pc/[ddr]_-{=}_>>>>{\dscomment{T \text{ is s. fib.}}\qquad\quad} \ar[r]^-{=} & \Sigma_{1_{p(A)}}(T(\pi^*_{1_{p(A)}}(A))) \ar[d]^>>>>>>{\Sigma_{1_{p(A)}}(T(\eta^{\Sigma_{1_{p(A)}} \!\!\dashv\, \pi^*_{1_{p(A)}}}_{\pi^*_{1_{p(A)}}(A)}))}^<<<<<{\quad\dscomment{\text{def. of } \sigma_{1_{p(A)},\pi^*_{1_{p(A)}}(A)}}} \ar[r]^-{\sigma_{1_{p(A)},\pi^*_{1_{p(A)}}(A)}} \ar@/_4.5pc/[dd]_<<<<<<<<<<{\id_{\Sigma_{1_{p(A)}}(T(\pi^*_{1_{p(A)}}(A)))}\!\!\!\!}_-{\dscomment{\text{id. law}}\quad}^<<<<<<<<<<<<<<{\,\,\dscomment{\Sigma_{1_{p(A)}} \,\dashv\, \pi^*_{1_{p(A)}}}} & T(\Sigma_{1_{p(A)}}(\pi^*_{1_{p(A)}}(A))) \ar@/^4pc/[dddd]_-{T(\varepsilon^{\Sigma_{1_{p(A)}} \!\!\dashv\, \pi^*_{1_{p(A)}}}_A)}_>>>>>>>>>>>>>>>>>>>>{\dscomment{\text{nat. of } \varepsilon^{\Sigma_{1_{p(A)}} \!\!\dashv\, \pi^*_{1_{p(A)}}}}\qquad\qquad}
\\
& \txt<5pc>{$\Sigma_{1_{p(A)}}(T(\pi^*_{1_{p(A)}}($\\$\Sigma_{1_{p(A)}}(\pi^*_{1_{p(A)}}(A)))))$} \ar[d]^-{\Sigma_{1_{p(A)}}(T(\pi^*_{1_{p(A)}}(\varepsilon^{\Sigma_{1_{p(A)}} \!\!\dashv\, \pi^*_{1_{p(A)}}}_{A})))} \ar[r]_-{=} & \txt<5pc>{$\Sigma_{1_{p(A)}}(\pi^*_{1_{p(A)}}(T($\\$\Sigma_{1_{p(A)}}(\pi^*_{1_{p(A)}}(A)))))$} \ar@/^3pc/[ddl]^>>>>>>>>>>>>>>{\!\!\!\!\Sigma_{1_{p(A)}}(\pi^*_{1_{p(A)}}(T(\varepsilon^{\Sigma_{1_{p(A)}} \!\!\dashv\, \pi^*_{1_{p(A)}}}_{A})))}_>>>>>>>>>>>>>>>>>{\dscomment{T \text{ is split fibred}}\qquad} \ar[u]^>>>>>>{\varepsilon^{\Sigma_{1_{p(A)}} \!\!\dashv\, \pi^*_{1_{p(A)}}}_{T(\Sigma_{1_{p(A)}}(\pi^*_{1_{p(A)}}(A)))}}
\\
& \Sigma_{1_{p(A)}}(T(\pi^*_{1_{p(A)}}(A))) \ar@/^1pc/[d]^-{=} & 
\\
& \Sigma_{1_{p(A)}}(\pi^*_{1_{p(A)}}(T(A))) \ar[dr]^-{\varepsilon^{\Sigma_{1_{p(A)}} \!\!\dashv\, \pi^*_{1_{p(A)}}}_{T(A)}}_<<<<{\dscomment{\text{id. law}}\qquad\quad}
\\
& & T(A)
}
\]

Next, diagram $(3)$, which we prove before diagram $(2)$ for better layout, commutes because we have
\[
\scriptsize
\xymatrix@C=6em@R=7em@M=0.5em{
& \Sigma_A(B) \ar[r]^-{\Sigma_A(\eta_B)} \ar[d]^<<<<<{\Sigma_A(\eta^{\Sigma_A \,\dashv\, \pi^*_A}_B)}^-{\,\,\,\,\,\qquad\qquad\dscomment{\text{nat. of } \eta}} \ar@/_2.5pc/[dddl]_-{\id_{\Sigma_A(B)}} & \Sigma_A(T(B)) \ar@/^2.5pc/[dddr]^-{\sigma_{A,B}} \ar[d]_>>>>>{\Sigma_A(T(\eta^{\Sigma_A \,\dashv\, \pi^*_A}_B))}
\\
& \Sigma_A(\pi^*_A(\Sigma_A(B))) \ar[ddl]^-{\!\!\!\!\varepsilon^{\Sigma_A \,\dashv\, \pi^*_A}_{\Sigma_A(B)}}_<<<<<<<<<<<<<{\dscomment{\Sigma_A \,\dashv\, \pi^*_A}\quad\,\,\,\,} \ar[dr]_-{\Sigma_A(\pi^*_A(\eta_{\Sigma_A(B)}))\,\,\,\,\,} \ar[r]_-{\Sigma_A(\eta_{\pi^*_A(\Sigma_A(B))})} & \Sigma_A(T(\pi^*_A(\Sigma_A(B)))) \ar[d]^-{=}_<<<<<<{\dscomment{\eta \text{ is split fibred}}\quad}^-{\qquad\quad\dscomment{\text{def. of } \sigma_{A,B}}}
\\
& & \Sigma_A(\pi^*_A(T(\Sigma_A(B)))) \ar[dr]_-{\varepsilon^{\Sigma_A \,\dashv\, \pi^*_A}_{T(\Sigma_A(B))}}_<<<<{\dscomment{\text{nat. of } \varepsilon^{\Sigma_A \,\dashv\, \pi^*_A}}\qquad\qquad\qquad\qquad\qquad\qquad} &
\\
\Sigma_A(B) \ar[rrr]_-{\eta_{\Sigma_A(B)}} & & & T(\Sigma_A(B))
}
\]

Next, diagram $(2)$ commutes because we have
\[
\scriptsize
\xymatrix@C=3.9em@R=8.5em@M=0.5em{
\Sigma'(T(C)) \ar[r]^-{\Sigma'(T(\eta^{\Sigma' \,\dashv\, \pi'^*}_C))} & \Sigma'(T(\pi'^*(\Sigma'(C)))) \ar[r]^-{=} & \Sigma'(\pi'^*(T(\Sigma'(C)))) \ar[r]^-{\varepsilon^{\Sigma' \,\dashv\, \pi'^*}_{T(\Sigma'(C))}} &T(\Sigma'(C))
\\
& & & \txt<4pc>{$\Sigma(\pi^*($\\$T(\Sigma'(C))))$} \ar[u]^-{\varepsilon^{\Sigma \,\dashv\, \pi^*}_{T(\Sigma'(C))}\!\!\!}
\\
\txt<3pc>{$\Sigma(\pi^*($\\$\Sigma'(T(C))))$} \ar[r]^-{\Sigma(\pi^*(\Sigma'(T(\eta^{\Sigma' \,\dashv\, \pi'^*}_C))))} \ar@/_1pc/[uu]_-{\varepsilon^{\Sigma \,\dashv\, \pi^*}_{\Sigma'(T(C))}} \ar@{}[uu]^<<<<<<<<<<<{\alpha^{-1}_{A,B,T(C)}} & \txt<4pc>{$\Sigma(\pi^*(\Sigma'(T($\\$\pi'^*(\Sigma'(C))))))$} \ar[uu]_-{\varepsilon^{\Sigma \,\dashv\, \pi^*}_{\Sigma'(T(\pi'^*(\Sigma'(C))))}}^>>>>>>>>>>>>>>{\dscomment{\text{nat. of } \varepsilon^{\Sigma \,\dashv\, \pi^*}}\qquad\quad}_>>>>>>>>>>>>>>{\qquad\qquad\dscomment{T \text{ is split fibred}}} \ar[r]^-{=} & \txt<4pc>{$\Sigma(\pi^*(\Sigma'(\pi'^*($\\$T(\Sigma'(C))))))$} \ar[uu]_-{\varepsilon^{\Sigma \,\dashv\, \pi^*}_{\Sigma'(\pi'^*(T(\Sigma'(C))))}\!\!\!}_>>>>>>>>>>>>>>{\qquad\quad\dscomment{\text{nat. of } \varepsilon^{\Sigma \,\dashv\, \pi^*}}} \ar[ur]^-{\Sigma(\pi^*(\varepsilon^{\Sigma' \,\dashv\, \pi'^*}_{T(\Sigma'(C))}))\!\!\!\!\!\!} & \txt<4pc>{$\Sigma(\kappa^*(\pi'^*($\\$T(\Sigma'(C)))))$} \ar[u]^-{=}^<<<{\dscomment{p \text{ is a split fibration}}\qquad}
\\
\txt<4.5pc>{$\Sigma(\kappa^*(\pi'^*($\\$\Sigma'(T(C)))))$} \ar[r]^>>>>>{\Sigma(\kappa^*(\pi'^*(\Sigma'(T(\eta^{\Sigma' \,\dashv\, \pi'^*}_C)))))} \ar@/_1.75pc/[u]_-{=}^-{\dscomment{\text{def. of } \alpha^{-1}_{A,B,T(C)}}\,} & \txt<5.5pc>{$\Sigma(\kappa^*(\pi'^*($\\$\Sigma'(T(\pi'^*(\Sigma'(C)))))))$} \ar[u]_-{=}^-{\dscomment{p \text{ is a split fibration}}\quad}_-{\qquad\qquad\dscomment{T \text{ is split fibred}}} \ar[r]^-{=} & \txt<4pc>{$\Sigma(\kappa^*(\pi'^*(\Sigma'(\pi'^*($\\$T(\Sigma'(C)))))))$} \ar[u]^-{=} \ar[ur]^>>>>>>>>>>{\Sigma(\kappa^*(\pi'^*(\varepsilon^{\Sigma' \,\dashv\, \pi'^*}_{T(\Sigma'(C))})))\!\!\!\!\!}_<<<<{\qquad\quad\dscomment{\Sigma' \,\dashv\, \pi'^*}}
\\
\Sigma(\kappa^*(T(C))) \ar@{}[d]^-{\qquad\qquad\qquad\qquad\qquad\qquad\qquad\qquad\qquad\dscomment{(b)}} \ar@/_2.15pc/[ddd]^-{=} \ar@/^2.5pc/[uuuu] \ar[r]_-{\Sigma(\kappa^*(T(\eta^{\Sigma' \,\dashv\, \pi'^*}_C)))} \ar[u]_<<<<<<{\Sigma(\kappa^*(\eta^{\Sigma' \,\dashv\, \pi'^*}_{T(C)}))}_-{\qquad\quad\dscomment{\text{nat. of } \eta^{\Sigma' \,\dashv\, \pi'^*}}} & \txt<4pc>{$\Sigma(\kappa^*(T($\\$\pi'^*(\Sigma'(C)))))$} \ar[r]_-{=} \ar[u]^>>>>>>{\Sigma(\kappa^*(\eta^{\Sigma' \,\dashv\, \pi'^*}_{T(\pi'^*(\Sigma'(C)))}))} & \txt<4pc>{$\Sigma(\kappa^*(\pi'^*($\\$T(\Sigma'(C)))))$} \ar[u]^<<<<<<{\Sigma(\kappa^*(\eta^{\Sigma' \,\dashv\, \pi'^*}_{\pi'^*(T(\Sigma'(C)))}))}^-{\dscomment{T \text{ is split fibred}}\qquad\qquad} \ar@/_2pc/[uur]^<<<<<<<<<<<<<<<<<<{\id_{\Sigma(\kappa^*(\pi'^*(T(\Sigma'(C)))))}\!\!\!\!}_-{\,\,\,\,\,\,\,\,\,\,\,\qquad T(\varepsilon^{\Sigma \,\dashv\, \pi^*}_{\Sigma'(C)})} & T(\Sigma(\pi^*( \Sigma'(C)))) \ar@/_2.5pc/[uuuu]
\\
\txt<4pc>{$\Sigma(T(\pi^*_B($\\$\Sigma_B(\kappa^*(C)))))$} \ar[r]^-{=} & \txt<4pc>{$\Sigma(\pi^*_B(T($\\$\Sigma_B(\kappa^*(C)))))$} \ar@/_3pc/[dd]_<<<<<<<{\Sigma_A(\varepsilon^{\Sigma_B \,\dashv\, \pi^*_B}_{T(\Sigma_B(\kappa^*(C)))})} & \Sigma_A(\pi^*_A(T(\Sigma(\kappa^*(C))))) \ar@/_2pc/[ddr]_-{\varepsilon^{\Sigma_A \,\dashv\, \pi^*_A}_{T(\Sigma(\kappa^*(C)))}\!\!\!\!\!} & T(\Sigma(\kappa^*(\pi'^*( \Sigma'(C))))) \ar[u]^-{=}_-{\,\,\,\,\,\,\,\,\,\,\,\,\quad T(\alpha^{-1}_{A,B,C})}_>>>>>{\,\,\,\,\dscomment{\text{def. of } \alpha^{-1}_{A,B,C}}}
\\
& \txt<4pc>{$\Sigma_A(T(\pi^*_A($\\$\Sigma(\kappa^*(C)))))$} \ar[ur]^-{=} &
\\
\Sigma(T(\kappa^*(C))) \ar[uu]_-{\Sigma(T(\eta^{\Sigma_B \,\dashv\, \pi^*_B}_{\kappa^*(C)}))}_<<<<<<<<<{\qquad\dscomment{\text{def. of } \sigma_{B,\kappa^*(C)}}} \ar[r]_-{\Sigma_A(\sigma_{B,\kappa^*(C)})} & \Sigma_A(T(\Sigma_B(\kappa^*(C)))) \ar[u]_-{\Sigma_A(T(\eta^{\Sigma_A \,\dashv\, \pi^*_A}_{\Sigma_B(\kappa^*(C))}))}_-{\qquad\qquad\qquad\qquad\qquad\dscomment{\text{def. of } \sigma_{A,\Sigma_B(\kappa^*(C))}}} \ar[rr]_-{\sigma_{A,\Sigma_B(\kappa^*(C))}} & & T(\Sigma(\kappa^*(C))) \ar[uu]^-{T(\Sigma(\kappa^*(\eta^{\Sigma' \,\dashv\, \pi'^*}_C)))} \ar@/_4.25pc/[uuuuuuu]
}
\]

\pagebreak

Finally, diagram $(4)$ commutes because we have
\[
\scriptsize
\xymatrix@C=5.75em@R=7em@M=0.5em{
T(\Sigma_A(T(B))) \ar@/^1.5pc/[rrr]^-{T(\sigma_{A,B})}_-{\dscomment{\text{def. of } \sigma_{A,B}}} \ar[r]_-{T(\Sigma_A(T(\eta^{\Sigma_A \,\dashv\, \pi^*_A}_B)))} & T(\Sigma_A(T(\pi^*_A(\Sigma_A(B))))) \ar[r]_-{=} & T(\Sigma_A(\pi^*_A(T(\Sigma_A(B))))) \ar[r]_-{T(\varepsilon^{\Sigma_A \,\dashv\, \pi^*_A}_{T(\Sigma_A(B))})} & T(T(\Sigma_A(B))) \ar[dddd]^-{\mu_{\Sigma_A(B)}}
\\
\txt<4pc>{$\Sigma_A(\pi^*_A($\\$T(\Sigma_A(T(B)))))$} \ar[r]^-{\Sigma_A(\pi^*_A(T(\Sigma_A(T(\eta^{\Sigma_A \,\dashv\, \pi^*_A}_B)))))} \ar[u]_-{\varepsilon^{\Sigma_A \,\dashv\, \pi^*_A}_{T(\Sigma_A(T(B)))}}_-{\qquad\qquad\qquad\dscomment{\text{nat. of } \varepsilon^{\Sigma_A \,\dashv\, \pi^*_A}}} & \txt<5pc>{$\Sigma_A(\pi^*_A(T(\Sigma_A($\\$T(\pi^*_A(\Sigma_A(B)))))))$} \ar[r]^-{=} \ar[u]_-{\varepsilon^{\Sigma_A \,\dashv\, \pi^*_A}_{T(\Sigma_A(T(\pi^*_A(\Sigma_A(B)))))}} \ar[d]_-{=}^<<<<<{\quad\dscomment{T \text{ is split fibred}}}_-{\dscomment{T \text{ is split fibred}}\qquad\qquad} & \txt<5pc>{$\Sigma_A(\pi^*_A(T(\Sigma_A($\\$\pi^*_A(T(\Sigma_A(B)))))))$} \ar[u]_-{\varepsilon^{\Sigma_A \,\dashv\, \pi^*_A}_{T(\Sigma_A(\pi^*_A(T(\Sigma_A(B)))))}}^-{\dscomment{T \text{ is split fibred}}\quad\,\,\,\,\,\,} \ar[dd]_-{\Sigma_A(\pi^*_A(T(\varepsilon^{\Sigma_A \,\dashv\, \pi^*_A}_{T(\Sigma_A(B))})))}^<<<<<<<<<<{\qquad\dscomment{\text{nat. of } \varepsilon^{\Sigma_A \,\dashv\, \pi^*_A}}}_>>>>>>>>>>>>>>>{\dscomment{\Sigma_A \,\dashv\, \pi^*_A}\qquad}_>>>>>>>{\dscomment{T \text{ is split fibred}}\quad}
\\
\txt<4pc>{$\Sigma_A(T(\pi^*_A($\\$\Sigma_A(T(B)))))$} \ar[u]_-{=}^-{\dscomment{\text{def.}}\,\,\,\,\,\,\,\,} \ar[r]^-{\Sigma_A(\pi^*_A(T(\Sigma_A(T(\eta^{\Sigma_A \,\dashv\, \pi^*_A}_B)))))} & \txt<5pc>{$\Sigma_A(T(\pi^*_A(\Sigma_A($\\$T(\pi^*_A(\Sigma_A(B)))))))$} \ar[ur]_-{=}
\\
\Sigma_A(T(T(B))) \ar[r]_-{\Sigma_A(T(T(\eta^{\Sigma_A \,\dashv\, \pi^*_A}_B)))} \ar[d]_-{\Sigma_A(\mu_B)} \ar[u]_-{\Sigma_A(T(\eta^{\Sigma_A \,\dashv\, \pi^*_A}_{T(B)}))} \ar@/^2.5pc/[uuu]^<<<<<<<<<<{\sigma_{A,T(B)}\!\!} & \Sigma_A(T(T(\pi^*_A(\Sigma_A(B))))) \ar[u]_-{\Sigma_A(T(\eta^{\Sigma_A \,\dashv\, \pi^*_A}_{T(\pi^*_A(\Sigma_A(B)))}))}^-{\dscomment{\text{nat. of } \eta^{\Sigma_A \,\dashv\, \pi^*_A}}\quad\,\,\,\,\,} \ar[r]_-{=} \ar[d]^-{\Sigma_A(\mu_{\pi^*_A(\Sigma_A(B))})}_-{\dscomment{\text{nat. of } \mu}\qquad\qquad\quad\,\,\,\,\,\,} & \Sigma_A(\pi^*_A(T(T(\Sigma_A(B))))) \ar@/_2pc/[uuur]_<<<<<<<<<<<<<<<<<<<<<<{\!\!\!\!\!\varepsilon^{\Sigma_A \,\dashv\, \pi^*_A}_{T(T(\Sigma_A(B)))}} \ar[d]^-{\Sigma_A(\pi^*_A(\mu_{\Sigma_A(B)}))}^<<<<{\,\,\,\,\qquad\qquad\dscomment{\text{nat. of } \varepsilon^{\Sigma_A \,\dashv\, \pi^*_A}}}_-{\dscomment{\mu \text{ is split fibred}}\qquad\quad}
\\
\Sigma_A(T(B)) \ar[r]^-{\Sigma_A(T(\eta^{\Sigma_A \,\dashv\, \pi^*_A}_B))} \ar@/_1.5pc/[rrr]_-{\sigma_{A,B}}^-{\dscomment{\text{def. of } \sigma_{A,B}}} & \Sigma_A(T(\pi^*_A(\Sigma_A(B)))) \ar[r]^-{=} & \Sigma_A(\pi^*_A(T(\Sigma_A(B)))) \ar[r]^-{\varepsilon^{\Sigma_A \,\dashv\, \pi^*_A}_{T(\Sigma_A(B))}} & T(\Sigma_A(B))
}
\]

We conclude by noting that the subdiagram marked with $(b)$ in the proof of diagram $(2)$ commutes because we have

\pagebreak

\[
\scriptsize
\xymatrix@C=3.5em@R=9em@M=0.5em{
\Sigma(\pi^*(T(\Sigma'(C)))) \ar[rrr]^-{\varepsilon^{\Sigma \,\dashv\, \pi^*}_{T(\Sigma'(C))}} & & & T(\Sigma'(C))
\\
& & & T(\Sigma(\pi^*(\Sigma'(C)))) \ar[u]_-{T(\varepsilon^{\Sigma \,\dashv\, \pi'^*}_{\Sigma'(C)})}^-{\dscomment{\text{nat. of } \varepsilon^{\Sigma \,\dashv\, \pi^*}}\qquad\qquad\qquad\qquad\qquad}
\\
& & \txt<5pc>{$\Sigma(\pi^*(T(\Sigma($\\$\pi^*(\Sigma'(C))))))$} \ar[ur]^-{\varepsilon^{\Sigma \,\dashv\, \pi^*}_{T(\Sigma(\pi^*(\Sigma'(C))))}} \ar[uull]_-{\Sigma(\pi^*(T(\varepsilon^{\Sigma \,\dashv\, \pi^*}_{\Sigma'(C)})))} & T(\Sigma(\kappa^*(\pi'^*(\Sigma'(C))))) \ar[u]_-{=}^<{\dscomment{p \text{ is a split fibration}}\qquad\qquad\quad}
\\
& & \txt<5pc>{$\Sigma(\pi^*(T(\Sigma(\kappa^*($\\$\pi'^*(\Sigma'(C)))))))$} \ar[ur]_-{\varepsilon^{\Sigma \,\dashv\, \pi^*}_{T(\Sigma(\kappa^*(\pi'^*(\Sigma'(C)))))}} \ar[u]^{=}^>{\dscomment{p \text{ is a split fibration}}\qquad\qquad\qquad\qquad}^>>>>>>>>>{\dscomment{T \text{ is split fibred}}\qquad\qquad\qquad\qquad\,\,\,\,\,\,\,}^>>>>>>>>>>>>>>>>>{\dscomment{\Sigma \,\dashv\, \pi^*}\qquad\qquad\qquad\qquad\qquad\!\!} & T(\Sigma(\kappa^*(C))) \ar[u]_-{T(\kappa^*(\eta^{\Sigma' \,\dashv\, \pi'^*}_C))}
\\
& \txt<4pc>{$\Sigma(T(\pi^*(\Sigma(\kappa^*($\\$\pi'^*(\Sigma'(C)))))))$} \ar@/^2pc/[ur]^-{=} & \Sigma(\pi^*(T(\Sigma(\kappa^*(C))))) \ar[r]_-{\Sigma_A(\varepsilon^{\Sigma_B \,\dashv\, \pi^*_B}_{\pi^*_A(T(\Sigma(\kappa^*(C))))})} \ar[u]_-{\Sigma(\pi^*(T(\Sigma(\kappa^*(\eta^{\Sigma' \,\dashv\, \pi'^*}_C)))))}^-{\dscomment{T \text{ is split fibred}}\qquad} & \Sigma_A(\pi^*_A(T(\Sigma_A(\Sigma_B(\kappa^*(C)))))) \ar[u]_-{\varepsilon^{\Sigma_A \,\dashv\, \pi^*_A}_T(\Sigma(\kappa^*(C)))}^>>>>>>>>{\dscomment{\text{def. of } \varepsilon^{\Sigma \,\dashv\, \pi^*}}\qquad\qquad\quad}^<<<<<<{\dscomment{\text{nat. of } \varepsilon^{\Sigma \,\dashv\, \pi^*}}\qquad\qquad\quad}
\\
\Sigma(\kappa^*(\pi'^*(T(\Sigma'(C))))) \ar[uuuuu]^-{=}  & & \txt<3.5pc>{$\Sigma(\pi^*_B(T(\pi^*_A($\\$\Sigma(\kappa^*(C))))))$} \ar[u]_-{=} \ar[r]^-{\Sigma_A(\varepsilon^{\Sigma_B \,\dashv\, \pi^*_B}_{T(\pi^*_A(\Sigma(\kappa^*(C))))})} & \Sigma_A(T(\pi^*_A(\Sigma_A(\Sigma_B(\kappa^*(C)))))) \ar[u]_-{=}^-{\dscomment{T \text{ is split fibred}}\qquad\qquad}
\\
\txt<3.5pc>{$\Sigma(\kappa^*(T($\\$\pi'^*(\Sigma'(C)))))$} \ar[r]_-{=} \ar[u]^-{=}_>>>>>>>>>>>>>>{\qquad\dscomment{T \text{ is split fibred}}}_<<<<<<{\qquad\dscomment{p \text{ is a split fibration}}} & \txt<3.5pc>{$\Sigma(T(\kappa^*($\\$\pi'^*(\Sigma'(C)))))$} \ar@/^5pc/[uuuuuul]^-{=} \ar[uu]_>>>>>>>>>{\Sigma(T(\eta^{\Sigma \,\dashv\, \pi^*}_{\kappa^*(\pi'^*(\Sigma'(C)))}))} & \Sigma(T(\pi^*(\Sigma(\kappa^*(C))))) \ar[u]_-{=} \ar@/^4pc/[uu]^-{=} & \Sigma_A(T(\Sigma_B(\kappa^*(C)))) \ar[u]_-{\Sigma_A(T(\eta^{\Sigma_A \,\dashv\, \pi^*_A}_{\Sigma_B(\kappa^*(C))}))}^-{\dscomment{\text{nat. of } \varepsilon^{\Sigma_B \,\dashv\, \pi^*_B}}\qquad\,\,\,\,\,}
\\
\Sigma(\kappa^*(T(C))) \ar[u]_>>>>>>{\Sigma(\kappa^*(T(\eta^{\Sigma' \,\dashv\, \pi'^*}_C)))}_-{\qquad\quad\dscomment{T \text{ is split fibred}}} \ar[r]_-{=} & \Sigma(T(\kappa^*(C))) \ar[r]_-{\Sigma(T(\eta^{\Sigma_B \,\dashv\, \pi^*_B}_{\kappa^*(C)}))} \ar[u]^<<<<<<{\Sigma(T(\kappa^*(\eta^{\Sigma' \,\dashv\, \pi'^*}_C)))}_>>>>{\,\,\,\,\,\,\,\quad\qquad\dscomment{\text{def. of } \eta^{\Sigma \,\dashv\, \pi^*}}}_<<<<<{\,\,\,\,\,\,\,\qquad\quad\dscomment{\text{nat. of } \eta^{\Sigma \,\dashv\, \pi^*}}} & \Sigma(T(\pi^*_B(\Sigma_B(\kappa^*(C))))) \ar[r]_-{=} \ar[u]^-{\Sigma(T(\pi^*_B(\eta^{\Sigma_A \,\dashv\, \pi^*_A}_{\Sigma_B(\kappa^*(C))})))}_>>>>>>>>>>>{\qquad\dscomment{T \text{ is split fibred}}} & \Sigma(\pi^*_B(T(\Sigma_B(\kappa^*(C))))) \ar[u]_-{\Sigma_A(\varepsilon^{\Sigma_B \,\dashv\, \pi^*_B}_{T(\Sigma_B(\kappa^*(C)))})} \ar[uul]^<<<<<<<<<<<{\Sigma(\pi^*_B(T(\eta^{\Sigma_A \,\dashv\, \pi^*_A}_{\Sigma_B(\kappa^*(C))})))\!\!\!\!}
}
\]

\section{Proof of Theorem~\ref{thm:dependentproductsinEMfibration}}
\label{sect:proofofthm:dependentproductsinEMfibration}

{
\renewcommand{\thetheorem}{\ref{thm:dependentproductsinEMfibration}}
\begin{theorem}
Given a split comprehension category with unit $p : \mathcal{V} \longrightarrow \mathcal{B}$ with split dependent products and a split fibred monad  $\mathbf{T} = (T,\eta,\mu)$ on it, then the corresponding EM-fibration $p^{\mathbf{T}} : \mathcal{V}^{\mathbf{T}} \longrightarrow \mathcal{B}$ has split dependent $p$-products.
\end{theorem}
\addtocounter{theorem}{-1}
}

\begin{proof}
Given an object $A$ in $\mathcal{V}$, the functor $\Pi^{\mathbf{T}}_A : \mathcal{V}^{\mathbf{T}}_{\ia A} \longrightarrow \mathcal{V}^{\mathbf{T}}_{p(A)}$ is given on objects by
\[
\Pi^{\mathbf{T}}_A(B,\beta) \defeq (\Pi_A(B), \beta_{\Pi^{\mathbf{T}}_A})
\]
where the candidate EM-algebra structure map $\beta_{\Pi^{\mathbf{T}}_A} : T(\Pi_A(B)) \longrightarrow \Pi_A(B)$ is defined as the following composite morphism:
\[
\xymatrix@C=5em@R=0.05em@M=0.5em{
T(\Pi_A(B)) \ar[r]^-{\eta^{\pi^*_A \,\dashv\, \Pi_A}_{T(\Pi_A(B))}} & \Pi_A(\pi^*_A(T(\Pi_A(B)))) \ar[dr]^-{=}
\\
& & \Pi_A(T(\pi^*_A(\Pi_A(B)))) \ar[dl]^-{\Pi_A(T(\varepsilon^{\pi^*_A \,\dashv\, \Pi_A}_B))}
\\
\Pi_A(B) & \Pi_A(T(B)) \ar[l]^-{\Pi_A(\beta)}
}
\]
using the split dependent products in $p$, i.e., the adjunction $\pi^*_A \dashv \Pi_A : \mathcal{V}_{\ia A} \longrightarrow \mathcal{V}_{p(A)}$; and making use of Proposition~\ref{prop:verticalEMalgebras} to ensure that $\beta$ is a vertical morphism.

Next, we prove that the morphism $\beta_{\Pi^{\mathbf{T}}_A} : T(\Pi_A(B)) \longrightarrow \Pi_A(B)$ is indeed a structure map of an EM-algebra, by showing that the next two diagrams commute in $\mathcal{V}_{p(A)}$.

\[
\hspace{-0.3cm}
\xymatrix@C=5em@R=4em@M=0.5em{
\Pi_A(B) \ar[rr]^{\eta_{\Pi_A(B)}} \ar[dr]^-{\eta^{\pi^*_A \,\dashv\, \Pi_A}_{\Pi_A(B)}}_>>>>>>>>>>>{\dcomment{\pi^*_A \,\dashv\, \Pi_A}\,\,\,\,\,\,\,\,\,\,\,\,} \ar@/_2pc/[ddr]^-{\!\!\id_{\Pi_A(B)}} \ar@/_7pc/[dddrr]_>>>>>>>>>>>>>>>{\Pi_A(\eta_B)\,\,\,\,}  \ar@/_7pc/[ddddrr]_-{\id_{\Pi_A(B)}\!\!}  & & T(\Pi_A(B)) \ar[d]_-{\eta^{\pi^*_A \,\dashv\, \Pi_A}_{T(\Pi_A(B))}}_-{\dcomment{\text{nat. of } \eta^{\pi^*_A \,\dashv\, \Pi_A}}\qquad\qquad\qquad\quad} \ar@/^6pc/[dddd]^-{\beta_{\Pi^{\mathbf{T}}_A}}
\\
& \Pi_A(\pi^*_A(\Pi_A(B))) \ar[r]^-{\Pi_A(\pi^*_A(\eta_{\Pi_A(B)}))} \ar@/_2pc/[dr]^-{\Pi_A(\eta_{\pi^*_A(\Pi_A(B))})}_>>>{\dcomment{\text{nat. of } \eta}\qquad\qquad} \ar[d]_<<<<<<{\!\!\Pi_A(\varepsilon^{\pi^*_A \,\dashv\, \Pi_A}_B)} & \Pi_A(\pi^*_A(T(\Pi_A(B)))) \ar[d]_-{=}_<<<<{\dcomment{\eta \text{ is split fibred}}\qquad}^-{\,\,\,\,\,\,\dcomment{\text{def. of } \beta_{\Pi^{\mathbf{T}}_A}}}
\\
& \Pi_A(B) \ar@/_2pc/[dr]^<<<<<<<<<{\Pi_A(\eta_B)}_<{\dcomment{\text{id. law}}\,\,\,} & \Pi_A(T(\pi^*_A(\Pi_A(B)))) \ar[d]_-{\Pi_A(T(\varepsilon^{\pi^*_A \,\dashv\, \Pi_A}_B))}
\\
& & \Pi_A(T(B)) \ar[d]_-{\Pi_A(\beta)}_<<<<<{\dcomment{(B,\beta) \text{ is an EM-algebra}}\qquad\,\,\,\,\,\,\,}
\\
& & \Pi_A(B)
}
\]

\pagebreak

\mbox{}

\vspace{0.5cm}

\[
\scriptsize
\xymatrix@C=5em@R=12em@M=0.5em{
T(T(\Pi_A(B))) \ar[rrrr]^-{\mu_{\Pi_A(B)}} \ar[d]^-{T(\eta^{\pi^*_A \,\dashv\, \Pi_A}_{\Pi_A(B)})}^-{\qquad\qquad\qquad\qquad\dscomment{\text{nat. of } \eta^{\pi^*_A \,\dashv\, \Pi_A}}} \ar@/^2.5pc/[drrr]_-{\eta^{\pi^*_A \,\dashv\, \Pi_A}_{T(T(\Pi_A(B)))}} 
& & & & T(\Pi_A(B)) \ar[d]_-{\eta^{\pi^*_A \,\dashv\, \Pi_A}_{T(\Pi_A(B))}}_-{\dscomment{\text{nat. of } \eta^{\pi^*_A \,\dashv\, \Pi_A}}\qquad\qquad\qquad\qquad}
\\
\txt<3.5pc>{
$T(\Pi_A(\pi^*_A($
\\
$T(\Pi_A(B)))))$
}
\ar[rr]^-{\eta^{\pi^*_A \,\dashv\, \Pi_A}_{T(\Pi_A(\pi^*_A(T(\Pi_A(B)))))}}
\ar[d]^-{=}^-{\qquad\qquad\qquad\dscomment{T \text{ is split fibred}}}
 & & 
\txt<3.5pc>{
$\Pi_A(\pi^*_A(T($
\\
$\Pi_A(\pi^*_A(T(\Pi_A(B)))))))$
}
\ar[dl]_-{=}^>>>>>>>>>>>>>{\quad\dscomment{T \text{ is split fibred}}}
\ar[d]_-{=}^<<<<<<<<<<{\!\!\qquad\qquad\dscomment{\pi^*_A \,\dashv\, \Pi_A}}^>>>>>>>>>>{\quad\qquad\dscomment{T \text{ is split fibred}}}
&
\txt<3.5pc>{
$\Pi_A(\pi^*_A(T($
\\
$T(\Pi_A(B)))))$
}
\ar[r]^-{\Pi_A(\pi^*_A(\mu_{\Pi_A(B)}))}
\ar[l]_-{\Pi_A(\pi^*_A(T(\eta^{\pi^*_A \,\dashv\, \Pi_A}_{T(\Pi_A(B))})))}
\ar[d]_-{=}^-{\,\,\,\,\,\,\qquad\dscomment{\mu \text{ is split fibred}}}
& 
\txt<3.5pc>{
$\Pi_A(\pi^*_A($
\\
$T(\Pi_A(B))))$
}
\ar[d]_-{=}
\\
\txt<3pc>{
$T(\Pi_A(T($
\\
$\pi^*_A(\Pi_A(B)))))$
}
\ar[r]^-{\eta^{\pi^*_A \,\dashv\, \Pi_A}_{T(\Pi_A(T(\pi^*_A(\Pi_A(B)))))}}
\ar[d]^>>>>>>>>{T(\Pi_A(T(\varepsilon^{\pi^*_A \,\dashv\, \Pi_A}_B)))}^-{\quad\qquad\dscomment{\text{nat. of } \eta^{\pi^*_A \,\dashv\, \Pi_A}}}
 & 
\txt<4pc>{
$\Pi_A(\pi^*_A(T(\Pi_A($
\\
$T(\pi^*_A(\Pi_A(B)))))))$
}
\ar[r]^-{=}
\ar[d]_<<<<<<<{\Pi_A(\pi^*_A(T(\Pi_A(T(\varepsilon^{\pi^*_A \,\dashv\, \Pi_A}_B)))))}
& 
\txt<4pc>{
$\Pi_A(T(\pi^*_A(\Pi_A($
\\
$T(\pi^*_A(\Pi_A(B)))))))$
}
\ar[r]^-{\Pi_A(T(\varepsilon^{\pi^*_A \,\dashv\, \Pi_A}_{T(\pi^*_A(\Pi_A(B)))}))}
\ar[d]_-{\Pi_A(T(\pi^*_A(\Pi_A(T(\varepsilon^{\pi^*_A \,\dashv\, \Pi_A}_B)))))}_<<<<<<<<<{\dscomment{T \text{ is split fibred}}\qquad\quad}
& 
\txt<3.5pc>{
$\Pi_A(T(T($
\\
$\pi^*_A(\Pi_A(B)))))$
}
\ar[r]^-{\Pi_A(\mu_{\pi^*_A(\Pi_A(B))})}
\ar[d]_-{\Pi_A(T(T(\varepsilon^{\pi^*_A \,\dashv\, \Pi_A}_B)))}_<<<<<<<<<{\dscomment{\text{nat. of } \varepsilon^{\pi^*_A \,\dashv\, \Pi_A}}\qquad\quad}
& 
\txt<3.5pc>{
$\Pi_A(T($
\\
$\pi^*_A(\Pi_A(B))))$
}
\ar[d]_-{\Pi_A(T(\varepsilon^{\pi^*_A \,\dashv\, \Pi_A}_B))}_<<<<<<<<<{\dscomment{\text{nat. of } \mu}\qquad\qquad}
\\
T(\Pi_A(T(B))) \ar[r]_-{\eta^{\pi^*_A \,\dashv\, \Pi_A}_{T(\Pi_A(T(B)))}} \ar[d]^>>>>>>>>>>{T(\Pi_A(\beta))}^-{\qquad\quad\dscomment{\text{nat. of } \eta^{\pi^*_A \,\dashv\, \Pi_A}}}
&
\txt<3.5pc>{
$\Pi_A(\pi^*_A(T($
\\
$\Pi_A(T(B)))))$
}
\ar[r]_-{=}
\ar[d]_<<<<<<<<<{\Pi_A(\pi^*_A(T(\Pi_A(\beta))))}
& 
\txt<3.5pc>{
$\Pi_A(T(\pi^*_A($
\\
$\Pi_A(T(B)))))$
}
\ar[r]_-{\Pi_A(T(\varepsilon^{\pi^*_A \,\dashv\, \Pi_A}_{T(B)}))}
\ar[d]_-{\Pi_A(T(\pi^*_A(\Pi_A(\beta))))}_<<<<<<<<<{\dscomment{T \text{ is split fibred}}\qquad\quad}
&
\Pi_A(T(T(B))) \ar[r]_-{\Pi_A(\mu_B)} \ar[d]_-{\Pi_A(T(\beta))}_<<<<<<<<<<<{\dscomment{\text{nat. of } \varepsilon^{\pi^*_A \,\dashv\, \Pi_A}}\qquad\quad\,\,\,} & \Pi_A(T(B)) \ar[d]_-{\Pi_A(\beta)}_<<<<<<<<<<<{\dscomment{(B,\beta) \text{ is an EM-algebra}}\qquad\!\!\!}
\\
T(\Pi_A(B)) \ar[r]_-{\eta^{\pi^*_A \,\dashv\, \Pi_A}_{T(\Pi_A(B))}} \ar@/_4pc/[rrrr]_-{\beta_{\Pi^{\mathbf{T}}_A}} \ar@{}[d]^<<<<<<<<{\,\,\quad\qquad\qquad\qquad\qquad\qquad\qquad\qquad\qquad\qquad\qquad\dscomment{\text{def. of } \beta_{\Pi^{\mathbf{T}}_A}}} & \txt<3.5pc>{$\Pi_A(\pi^*_A($\\$T(\Pi_A(B))))$} \ar[r]_-{=} & \txt<3.5pc>{$\Pi_A(T($\\$\pi^*_A(\Pi_A(B))))$} \ar[r]_-{\Pi_A(T(\varepsilon^{\pi^*_A \,\dashv\, \Pi_A}_B))} & \Pi_A(T(B)) \ar[r]_-{\Pi_A(\beta)} & \Pi_A(B)
\\
&
}
\]

The functor $\Pi^{\mathbf{T}}_A$ is defined on morphisms $h : (B,\beta) \longrightarrow (B',\beta')$ simply by letting $\Pi^{\mathbf{T}}_A(h) \defeq \Pi_A(h)$. It is easy to see that this gives us an EM-algebra homomorphism:
\[
\xymatrix@C=7em@R=5em@M=0.5em{
T(\Pi_A(B)) \ar[r]^-{T(\Pi_A(h))} \ar[d]^-{\eta^{\pi^*_A \,\dashv\, \Pi_A}_{T(\Pi_A(B))}}^-{\,\,\,\,\quad\qquad\qquad\dcomment{\text{nat. of } \eta^{\pi^*_A \,\dashv\, \Pi_A}}} \ar@/_6pc/[dddd]_{\beta_{\Pi^{\mathbf{T}}_A}} & T(\Pi_A(B')) \ar[d]_-{\eta^{\pi^*_A \,\dashv\, \Pi_A}_{T(\Pi_A(B'))}} \ar@/^6pc/[dddd]^{\beta'_{\Pi^{\mathbf{T}}_A}}
\\
\Pi_A(\pi^*_A(T(\Pi_A(B)))) \ar[r]_-{\Pi_A(\pi^*_A(T(\Pi_A(h))))} \ar[d]^-{=}^-{\,\,\quad\qquad\qquad\dcomment{T \text{ is split fibred}}}_-{\dcomment{\text{def. of } \beta_{\Pi^{\mathbf{T}}_A}}\quad\!\!\!\!} & \Pi_A(\pi^*_A(T(\Pi_A(B')))) \ar[d]_-{=}
\\
\Pi_A(T(\pi^*_A(\Pi_A(B)))) \ar[r]_-{\Pi_A(T(\pi^*_A(\Pi_A(h))))} \ar[d]^>>>>>{\Pi_A(T(\varepsilon^{\pi^*_A \,\dashv\, \Pi_A}_B))}^-{\,\,\,\,\quad\qquad\qquad\dcomment{\text{nat. of } \varepsilon^{\pi^*_A \,\dashv\, \Pi_A}}} & \Pi_A(T(\pi^*_A(\Pi_A(B')))) \ar[d]_<<<<<{\Pi_A(T(\varepsilon^{\pi^*_A \,\dashv\, \Pi_A}_{B'}))}^-{\!\!\!\!\quad\dcomment{\text{def. of } \beta'_{\Pi^{\mathbf{T}}_A}}}
\\
\Pi_A(T(B)) \ar[r]_-{\Pi_A(T(h))} \ar[d]^>>>>>{\Pi_A(\beta)} & \Pi_A(T(B')) \ar[d]_<<<<<{\Pi_A(\beta')}_-{\dcomment{h \text{ is an EM-algebra homomorphism}}\quad\qquad\!\!\!\!\!\!\!\!\!}
\\
\Pi_A(B) \ar[r]_-{\Pi_A(h)} & \Pi_A(B')
}
\]
Further, it is also easy to see that $\Pi^{\mathbf{T}}_A$ preserves identities and composition---these properties follow directly from the functoriality of $\Pi_A$. We therefore omit these proofs.

We proceed by proving that we have an adjunction $\pi^*_A \dashv \Pi^{\mathbf{T}}_A : \mathcal{V}^{\mathbf{T}}_{\ia A} \longrightarrow \mathcal{V}^{\mathbf{T}}_{p(A)}$.
%

First, we note that the components of the unit and counit natural transformations 
\[
\eta^{\pi^*_A \,\dashv\, \Pi^{\mathbf{T}}_A} : \id_{\mathcal{V}^{\mathbf{T}}_{p(A)}} \longrightarrow \Pi^{\mathbf{T}}_A \comp \pi^*_A
\qquad
\varepsilon^{\pi^*_A \,\dashv\, \Pi^{\mathbf{T}}_A} : \pi^*_A \comp \Pi^{\mathbf{T}}_A \longrightarrow \id_{\mathcal{V}^{\mathbf{T}}_{\ia A}}
\]
are given simply by
\[
\eta^{\pi^*_A \,\dashv\, \Pi^{\mathbf{T}}_A}_{(B,\beta)} \defeq \eta^{\pi^*_A \,\dashv\, \Pi_A}_B
\qquad
\varepsilon^{\pi^*_A \,\dashv\, \Pi^{\mathbf{T}}_A}_{(B,\beta)} \defeq \varepsilon^{\pi^*_A \,\dashv\, \Pi_A}_B
\]

Next, we prove that these components are indeed EM-algebra homomorphisms, by showing that the next two diagrams commute, in $\mathcal{V}^{\mathbf{T}}_{p(A)}$ and $\mathcal{V}^{\mathbf{T}}_{\ia A}$, respectively.
\[
\xymatrix@C=11em@R=5em@M=0.5em{
T(B) \ar[r]^{T(\eta^{\pi^*_A \,\dashv\, \Pi_A}_B)} \ar[ddddd]_-{\beta} \ar@/_5pc/[ddddr]_>>>>>>>>>>>>>>>>>>>>>>>>>>>>>{\eta^{\pi^*_A \,\dashv\, \Pi_A}_{T(B)}\!\!\!} & T(\Pi_A(\pi^*_A(B))) \ar[d]_-{\eta^{\pi^*_A \,\dashv\, \Pi_A}_{T(\Pi_A(\pi^*_A(B)))}}_-{\dcomment{\text{nat. of } \eta^{\pi^*_A \,\dashv\, \Pi_A}}\qquad\qquad\qquad\qquad} \ar@/^9pc/[ddddd]^-{(\pi^*_A(\beta))_{\Pi^{\mathbf{T}}_A}}
\\
& \Pi_A(\pi^*_A(T(\Pi_A(\pi^*_A(B))))) \ar[d]^-{=}^>>>>>{\!\!\quad\dcomment{\text{def. of } (\pi^*_A(\beta))_{\Pi^{\mathbf{T}}_A}}}
\\
& \Pi_A(T(\pi^*_A(\Pi_A(\pi^*_A(B))))) \ar@/^2pc/[d]^-{\Pi_A(T(\varepsilon^{\pi^*_A \,\dashv\, \Pi_A}_{\pi^*_A(B)}))}_-{\dcomment{\pi^*_A \dashv \Pi_A}\,\,}
\\
& \Pi_A(T(\pi^*_A(B))) \ar[d]^-{=} \ar@/^2pc/[u]^-{\Pi_A(T(\pi^*_A(\eta^{\pi^*_A \,\dashv\, \Pi_A}_B)))}^>>>>>{\dcomment{\eta \text{ is split fibred}}\quad}
\\
& \Pi_A(\pi^*_A(T(B))) \ar[d]_-{\Pi_A(\pi^*_A(\beta))}_-{\dcomment{\text{nat. of } \eta^{\pi^*_A \,\dashv\, \Pi_A}}\qquad\qquad\qquad\qquad} \ar@/^9.5pc/[uuu]^>>>>>>>>>>>>{\Pi_A(\pi^*_A(T(\eta^{\pi^*_A \,\dashv\, \Pi_A}_B)))}
\\
B \ar[r]_-{\eta^{\pi^*_A \,\dashv\, \Pi_A}_B} & \Pi_A(\pi^*_A(B))
}
\]

\pagebreak

\mbox{}

\vspace{-1cm}

\[
\xymatrix@C=11em@R=5em@M=0.5em{
T(\pi^*_A(\Pi_A(B))) \ar[r]^-{T(\varepsilon^{\pi^*_A \,\dashv\, \Pi_A}_B)} \ar[d]^-{=} \ar@/_9pc/[ddddd]_-{\pi^*_A(\beta_{\Pi^{\mathbf{T}}_A})}^-{\,\,\,\quad{\dcomment{\text{def. of } \pi^*_A(\beta_{\Pi^{\mathbf{T}}_A})}}} & T(B) \ar[ddddd]^-{\beta}
\\
\pi^*_A(T(\Pi_A(B))) \ar@/_2pc/[d]_-{\pi^*_A(\eta^{\pi^*_A \,\dashv\, \Pi_A}_{T(\Pi_A(B))})}^-{\,\,\,\dcomment{\pi^*_A \dashv \Pi_A}}
\\
\pi^*_A(\Pi_A(\pi^*_A(T(\Pi_A(B))))) \ar[d]_-{=}^<<<<{\,\,\dcomment{T \text{ is split fibred}}} \ar@/_2pc/[u]_-{\varepsilon^{\pi^*_A \,\dashv\, \Pi_A}_{\pi^*_A(T(\Pi_A(B)))}}
\\
\pi^*_A(\Pi_A(T(\pi^*_A(\Pi_A(B))))) \ar[d]_-{\pi^*_A(\Pi_A( T(\varepsilon^{\pi^*_A \,\dashv\, \Pi_A}_B)))} \ar@/_9pc/[uuu]_-{\varepsilon^{\pi^*_A \,\dashv\, \Pi_A}_{T(\pi^*_A(\Pi_A(B)))}}_>>>>>>>>>>>>>{\qquad\qquad\dcomment{\text{nat. of } \varepsilon^{\pi^*_A \,\dashv\, \Pi_A}}}
\\
\pi^*_A(\Pi_A(T(B))) \ar[d]^-{\pi^*_A(\Pi_A(\beta))}^-{\qquad\qquad\qquad\qquad\dcomment{\text{nat. of } \varepsilon^{\pi^*_A \,\dashv\, \Pi_A}}} \ar@/_5pc/[uuuur]_<<<<<<<<<<<<<<<<<<<<<<<<<<<<<{\!\!\!\!\!\!\varepsilon^{\pi^*_A \,\dashv\, \Pi_A}_{T(B)}}
\\
\pi^*_A(\Pi_A(B)) \ar[r]_-{\varepsilon^{\pi^*_A \,\dashv\, \Pi_A}_B} & B
}
\]

The naturality of $\eta^{\pi^*_A \,\dashv\, \Pi^{\mathbf{T}}_A}$ and $\varepsilon^{\pi^*_A \,\dashv\, \Pi^{\mathbf{T}}_A}$, and the two unit-counit laws follow directly from the corresponding properties of the adjunction $\pi^*_A \dashv \Pi_A : \mathcal{V}_{\ia A} \longrightarrow \mathcal{V}_{p(A)}$.

We conclude by noting that the adjunction $\pi^*_A \dashv \Pi^{\mathbf{T}}_A : \mathcal{V}^{\mathbf{T}}_{\ia A} \longrightarrow \mathcal{V}^{\mathbf{T}}_{p(A)}$ also satisfies the split Beck-Chevalley condition from Definition~\ref{def:splitdependentcompproducts}---similarly to the other properties, it also follows directly from the corresponding property of $\pi^*_A \dashv \Pi_A$.
\end{proof}


\newpage

\section{Proof of Theorem~\ref{thm:dependentsumsinEMfibrationwhenmonadpreservesthem}}
\label{sect:proofofthm:dependentsumsinEMfibrationwhenmonadpreservesthem}

{
\renewcommand{\thetheorem}{\ref{thm:dependentsumsinEMfibrationwhenmonadpreservesthem}}
\begin{theorem}
Given a split comprehension category with unit $p : \mathcal{V} \longrightarrow \mathcal{B}$ with strong split dependent sums and a split fibred monad $\mathbf{T} = (T,\eta,\mu)$ on it, then the corresponding EM-fibration $p^{\mathbf{T}} : \mathcal{V}^{\mathbf{T}} \!\longrightarrow\! \mathcal{B}$ has split dependent $p$-sums if the dependent strength of $\mathbf{T}$ is given by a family of natural isomorphisms, i.e., if for every $A$ in $\mathcal{V}$, $\sigma_A : \Sigma_A \comp T \longrightarrow T \comp \Sigma_A$ is a natural isomorphism.
Furthermore, these split dependent $p$-sums are preserved on-the-nose by $U^{\mathbf{T}}$, i.e., we have $U^{\mathbf{T}}(\Sigma^{\mathbf{T}}_A(B,\beta)) = \Sigma_A(U^{\mathbf{T}}(B,\beta))$.
\end{theorem}
\addtocounter{theorem}{-1}
}

\begin{proof}
Given an object $A$ in $\mathcal{V}$, the functor $\Sigma^{\mathbf{T}}_A : \mathcal{V}^{\mathbf{T}}_{\ia A} \longrightarrow \mathcal{V}^{\mathbf{T}}_{p(A)}$ is given on objects by
\[
\Sigma^{\mathbf{T}}_A(B,\beta) \defeq (\Sigma_A(B), \beta_{\Sigma^{\mathbf{T}}_A})
\]
where the candidate EM-algebra structure map $\beta_{\Sigma^{\mathbf{T}}_A} : T(\Sigma_A(B)) \longrightarrow \Sigma_A(B)$ is defined as the following composite morphism:
\[
\xymatrix@C=5em@R=5em@M=0.5em{
T(\Sigma_A(B)) \ar[r]^-{\sigma^{-1}_{A,B}} & \Sigma_A(T(B)) \ar[r]^-{\Sigma_A(\beta)} & \Sigma_A(B)
}
\]
using the split dependent sums in $p$, i.e., the adjunction $\Sigma_A \dashv \pi^*_A : \mathcal{V}_{p(A)} \longrightarrow \mathcal{V}_{\ia A}$; and making use of Proposition~\ref{prop:verticalEMalgebras} to ensure that $\beta$ is a vertical morphism.

Next, we prove that the morphism $\beta_{\Sigma^{\mathbf{T}}_A} : T(\Sigma_A(B)) \longrightarrow \Sigma_A(B)$ is indeed a structure map of an EM-algebra, by showing that the next two diagrams commute in $\mathcal{V}_{p(A)}$.
\[
\xymatrix@C=6em@R=5em@M=0.5em{
\Sigma_A(B) \ar[rrr]^-{\eta_{\Sigma_A(B)}} \ar@/_2pc/[ddrr]^-{\id_{\Sigma_A(B)}} \ar@/_6pc/[dddrrr]_{\id_{\Sigma_A(B)}}^-{\quad\dcomment{\text{id. law}}} &&& T(\Sigma_A(B)) \ar[ddd]^-{\beta_{\Sigma^{\mathbf{T}}_A}} \ar@{}[dd]_-{\dcomment{\text{def. of } \beta_{\Sigma^{\mathbf{T}}_A}}\quad\,\,\,\,\,\,} \ar[dl]_-{\sigma^{-1}_{A,B}}_-{\dcomment{(a)}\qquad\qquad\qquad\qquad\qquad\qquad}
\\
&& \Sigma_A(T(B)) \ar@/^2.25pc/[ddr]^-{\Sigma_A(\beta)} &
\\
&& \Sigma_A(B) \ar[u]^-{\Sigma_A(\eta_B)}_<<<{\!\!\!\!\!\!\!\quad\dcomment{(B,\beta) \text{ is EM-alg.}}} \ar[dr]_-{\id_{\Sigma_A(B)}} &
\\
&&& \Sigma_A(B)
}
\]

\pagebreak

\mbox{}

\vspace{-1cm}

\[
\xymatrix@C=3.5em@R=5em@M=0.5em{
T(T(\Sigma_A(B))) \ar[rrr]^-{T(\beta_{\Sigma^{\mathbf{T}}_A})} \ar[dddd]_-{\mu_{\Sigma_A(B)}}^-{\,\,\,\,\qquad\dcomment{(b)}} \ar[dr]_-{T(\sigma^{-1}_{A,B})} &&& T(\Sigma_A(B)) \ar[dddd]^-{\beta_{\Sigma^{\mathbf{T}}_A}}_-{\dcomment{\text{def. of } \beta_{\Sigma^{\mathbf{T}}_A}}\quad} \ar[ddl]_-{\sigma^{-1}_{A,B}\!\!}
\\
& T(\Sigma_A(T(B))) \ar[d]_-{\sigma^{-1}_{A,{T(B)}}}^-{\qquad\quad\dcomment{\text{nat. of } \sigma^{-1}_A}} \ar[urr]_-{\,\,\,\,\,\,T(\Sigma_A(\beta))}^-{\dcomment{\text{def. of } \beta_{\Sigma^{\mathbf{T}}_A}}\qquad\qquad\qquad\qquad\qquad}
\\
& \Sigma_A(T(T(B))) \ar[d]_-{\Sigma_A(\mu_B)}^-{\quad\dcomment{(B,\beta) \text{ is an EM-algebra}}} \ar[r]^-{\Sigma_A(T(\beta))} & \Sigma_A(T(B)) \ar[ddr]^-{\Sigma_A(\beta)}
\\
& \Sigma_A(T(B)) \ar[drr]_-{\Sigma_A(\beta)}
\\
T(\Sigma_A(B)) \ar[ur]_-{\!\!\!\!\sigma^{-1}_{A,B}}_-{\qquad\qquad\qquad\dcomment{\text{def. of } \beta_{\Sigma^{\mathbf{T}}_A}}} \ar[rrr]_-{\beta_{\Sigma^{\mathbf{T}}_A}} &&& \Sigma_A(B)
}
\]

We conclude the proof that $\beta_{\Sigma^{\mathbf{T}}_A} : T(\Sigma_A(B)) \longrightarrow \Sigma_A(B)$ is an EM-algebra structure map by noting that 
as $\id_{\Sigma_A(B)}$, $\sigma^{-1}_{A,B}$, and $\sigma^{-1}_{A,{T(B)}} \comp T(\sigma^{-1}_{A,B})$ are all isomorphisms, it suffices to show that the subdiagrams marked with $(a)$ and $(b)$ commute when these morphisms are replaced with their inverses. However, after replacing these morphisms with their inverses, we see that the corresponding diagrams are exactly diagrams $(3)$ and $(4)$ from Proposition~\ref{prop:strengthofsplitfibredmonads}, governing the interaction of $\sigma_A$ with $\eta$ and $\mu$.

Next, the functor $\Sigma^{\mathbf{T}}_A$ is defined on morphisms $h : (B,\beta) \longrightarrow (B',\beta')$ simply by letting $\Sigma^{\mathbf{T}}_A(h) \defeq \Sigma_A(h)$. It is easy to see that this gives us an EM-algebra homomorphism:

\[
\xymatrix@C=9em@R=5em@M=0.5em{
T(\Sigma_A(B)) \ar[d]_-{\sigma^{-1}_{A,B}}^-{\,\,\,\,\,\quad\qquad\qquad\dcomment{(c)}} \ar[r]^-{T(\Sigma_A(h))} & T(\Sigma_A(B')) \ar[d]^-{\sigma^{-1}_{A,{B'}}}
\\
\Sigma_A(T(B)) \ar[r]^-{\Sigma_A(T(h))} \ar[d]_-{\Sigma_A(\beta)}^-{\!\!\!\!\!\!\!\!\!\qquad\dcomment{h \text{ is an EM-algebra homomorphism}}} & \Sigma_A(T(B')) \ar[d]^-{\Sigma_A(\beta')}
\\
\Sigma_A(B) \ar[r]_-{\Sigma_A(h)} & \Sigma_A(B')
}
\]

We show that the square marked with $(c)$ commutes by observing that $\sigma^{-1}_{A,B}$ and $\sigma^{-1}_{A,{B'}}$ are both isomorphisms. As a result, it suffices to show that the next diagram commutes, where we have replaced these two morphisms with their respective inverses.
\[
\xymatrix@C=7em@R=5em@M=0.5em{
T(\Sigma_A(B)) \ar[r]^-{T(\Sigma_A(h))} & T(\Sigma_A(B'))
\\
\Sigma_A(\pi^*_A(T(\Sigma_A(B)))) \ar[r]^-{\Sigma_A(\pi^*_A(T(\Sigma_A(h))))} \ar[u]_-{\varepsilon^{\Sigma_A \,\dashv\, \pi^*_A}_{T(\Sigma_A(B))}}_-{\quad\qquad\qquad\dcomment{\text{nat. of } \varepsilon^{\Sigma_A \,\dashv\, \pi^*_A}}} & \Sigma_A(\pi^*_A(T(\Sigma_A(B')))) \ar[u]^-{\varepsilon^{\Sigma_A \,\dashv\, \pi^*_A}_{T(\Sigma_A(B'))}}
\\
\Sigma_A(T(\pi^*_A(\Sigma_A(B)))) \ar[r]^-{\Sigma_A(T(\pi^*_A(\Sigma_A(h))))} \ar[u]_-{=}_-{\!\!\!\!\quad\qquad\qquad\dcomment{T \text{ is split fibred}}}^-{\dcomment{\text{def. of } \sigma_{A,B}}\,\,\,\,\,\,\,\,\,} & \Sigma_A(T(\pi^*_A(\Sigma_A(B')))) \ar[u]^-{=}_-{\,\,\,\,\,\,\,\,\,\dcomment{\text{def. of } \sigma_{A,{B'}}}}
\\
\Sigma_A(T(B)) \ar[r]_-{\Sigma_A(T(h))} \ar@/^6pc/[uuu]^-{\sigma_{A,B}} \ar[u]_>>>>>{\Sigma_A(T(\eta^{\Sigma_A \,\dashv\, \pi^*_A}_B))}_-{\qquad\qquad\quad\dcomment{\text{nat. of } \eta^{\Sigma_A \,\dashv\, \pi^*_A}}} & \Sigma_A(T(B')) \ar@/_6pc/[uuu]_-{\sigma_{A,{B'}}} \ar[u]^<<<<<{\Sigma_A(T(\eta^{\Sigma_A \,\dashv\, \pi^*_A}_B))}
}
\]

The naturality of $\eta^{\Sigma^{\mathbf{T}}_A \,\dashv\, \pi^*_A}$ and $\varepsilon^{\Sigma^{\mathbf{T}}_A \,\dashv\, \pi^*_A}$, and the two unit-counit laws follow directly from the corresponding properties for the adjunction $\Sigma_A \dashv \pi^*_A : \mathcal{V}_{p(A)} \longrightarrow \mathcal{V}_{\ia A}$.

We conclude by noting that the adjunction $\Sigma^{\mathbf{T}}_A \dashv \pi^*_A : \mathcal{V}^{\mathbf{T}}_{p(A)} \longrightarrow \mathcal{V}^{\mathbf{T}}_{\ia A}$ also satisfies the split Beck-Chevalley condition from Definition~\ref{def:splitdependentcompproducts}---similarly to the other properties, it also follows directly from the corresponding property of $\Sigma_A \dashv \pi^*_A$.
\end{proof}


\section{Proof of Theorem~\ref{thm:dependentsumsinEMfibration}}
\label{sect:proofofthm:dependentsumsinEMfibration}

{
\renewcommand{\thetheorem}{\ref{thm:dependentsumsinEMfibration}}
\begin{theorem}
Given a split comprehension category with unit $p : \mathcal{V} \longrightarrow \mathcal{B}$ with strong split dependent sums and a split fibred monad $\mathbf{T} = (T,\eta,\mu)$ on it, then the corresponding EM-fibration $p^{\mathbf{T}} : \mathcal{V}^{\mathbf{T}} \longrightarrow \mathcal{B}$ has split dependent $p$-sums if $p^{\mathbf{T}}$ has split fibred reflexive coequalizers.
\end{theorem}
\addtocounter{theorem}{-1}
}

\begin{proof}
Given an object $A$ in $\mathcal{V}$, the functor $\Sigma^{\mathbf{T}}_A : \mathcal{V}^{\mathbf{T}}_{\ia A} \longrightarrow \mathcal{V}^{\mathbf{T}}_{p(A)}$ is given on an object $(B,\beta)$ of $\mathcal{V}^{\mathbf{T}}_{\ia A}$ as the reflexive coequalizer 
\[
\xymatrix@C=5em@R=5em@M=0.5em{
(T(\Sigma_A(B)), \mu_{\Sigma_A(B)}) \ar[r]^-{e_{A,(B,\beta)}} & \Sigma^{\mathbf{T}}_A(B,\beta)
}
\]
of the following pair of morphisms in $\mathcal{V}^{\mathbf{T}}_{p(A)}$, given by
\[
\xymatrix@C=5em@R=5em@M=0.5em{
(T(\Sigma_A(T(B))), \mu_{\Sigma_A(T(B))}) \ar[r]^-{T(\Sigma_A(\beta))} & (T(\Sigma_A(B)), \mu_{\Sigma_A(B)})
}
\]
and
\[
\xymatrix@C=6em@R=3em@M=0.5em{
(T(\Sigma_A(T(B))), \mu_{\Sigma_A(T(B))}) \ar[r]^-{T(\Sigma_A(T(\eta^{\Sigma_A \,\dashv\, \pi^*_A}_B)))} & (T(\Sigma_A(T(\pi^*_A(\Sigma_A(B))))), \mu_{\Sigma_A(T(\pi^*_A(\Sigma_A(B))))}) \ar[d]^-{=}
\\
& (T(\Sigma_A(\pi^*_A(T(\Sigma_A(B))))), \mu_{\Sigma_A(\pi^*_A(T(\Sigma_A(B))))}) \ar[d]^-{T(\varepsilon^{\Sigma_A \,\dashv\, \pi^*_A}_{T(\Sigma_A(B))})}
\\
(T(\Sigma_A(B)), \mu_{\Sigma_A(B)}) & \ar[l]^-{\mu_{\Sigma_A(B)}} (T(T(\Sigma_A(B))), \mu_{T(\Sigma_A(B))})
}
\]
using the split dependent sums in $p$, i.e., the adjunction $\Sigma_A \dashv \pi^*_A : \mathcal{V}_{p(A)} \longrightarrow \mathcal{V}_{\ia A}$; and making use of Proposition~\ref{prop:verticalEMalgebras} to ensure that $\beta$ is a vertical morphism.
In the rest of this proof, we systematically refer to these two morphisms  as $(1)$ and $(2)$, respectively. Further, as a notational convenience, we often write $(U^{\mathbf{T}}(\Sigma^{\mathbf{T}}_A(B,\beta)), \beta_{\Sigma^{\mathbf{T}}_A})$ for $\Sigma^{\mathbf{T}}_A(B,\beta)$.

It is easy to see that both $(1)$ and $(2)$ are in $\mathcal{V}^{\mathbf{T}}_{p(A)}$. On the one hand, all morphisms used in the definitions of $(1)$ and $(2)$ are vertical over $\id_{p(A)}$. On the other hand, all morphisms used in the definitions of $(1)$ and $(2)$ are EM-algebra homomorphisms because i) we know from the definition of the Eilenberg-Moore resolution that ${F^{\mathbf{T}}(f) = T(f)}$, and ii) it follows from the definition of monads that the components of $\mu$ are EM-algebra homomorphisms, i.e., we have $\mu_{\Sigma_A(B)} \comp \mu_{T(\Sigma_A(B))} = \mu_{\Sigma_A(B)} \comp T(\mu_{\Sigma_A(B)})$. 

The final ingredient we need to construct the reflexive coequalizer $e_{A,(B,\beta)}$ is 
\[
\xymatrix@C=5em@R=5em@M=0.5em{
(T(\Sigma_A(B)), \mu_{\Sigma_A(B)}) \ar[r]^-{T(\Sigma_A(\eta_B))}  & (T(\Sigma_A(T(B))), \mu_{\Sigma_A(T(B))}) 
}
\]
that forms the common section of $(1)$ and $(2)$, as shown by the commutativity of the next diagram. 
In order to minimise the space taken by this diagram, we present it (and other diagrammatic proofs below) in $\mathcal{V}$ rather than in $\mathcal{V}^{\mathbf{T}}$, working directly with the underlying morphisms of the EM-algebra homomorphisms involved. 
\[
\hspace{-0.3cm}
\xymatrix@C=4.5em@R=5em@M=0.5em{
T(\Sigma_A(B)) \ar[rr]^-{T(\Sigma_A(\eta_B))} \ar[ddddd]_-{T(\Sigma_A(\eta_B))} \ar[dr]^-{\,\,\,\,\,T(\Sigma_A(\eta^{\Sigma_A \,\dashv\, \pi^*_A}_B))} \ar@/_8pc/[dddddrr]_>>>>>>>>>>>>>>>>>>>>>>>>>>>>>>>>>>>>>>>>>>>>>>>>>>>{\id_{T(\Sigma_A(B))}\!\!\!} \ar@/_2pc/[dddr]^-{\id_{T(\Sigma_A(B))}} & & T(\Sigma_A(T(B))) \ar[d]^-{T(\Sigma_A(T(\eta^{\Sigma_A \,\dashv\, \pi^*_A}_B)))}_-{\dcomment{\text{nat. of } \eta}\qquad\qquad\qquad}
\\
& T(\Sigma_A(\pi^*_A(\Sigma_A(B)))) \ar[r]^-{T(\Sigma_A(\eta_{\pi^*_A(\Sigma_A(B))}))} \ar@/_2pc/[dr]^-{\,\,\,T(\Sigma_A(\pi^*_A(\eta_{\Sigma_A(B)})))} \ar[dd]^-{T(\varepsilon^{\Sigma_A \,\dashv\, \pi^*_A}_{\Sigma_A(B)})}_<<<<<{\dcomment{\Sigma_A \dashv \pi^*_A}\qquad\,\,\,} & T(\Sigma_A(T(\pi^*_A(\Sigma_A(B))))) \ar[d]^-{=}_<<<<{\dcomment{\eta \text{ is split fibred}}\qquad\quad}
\\
& & T(\Sigma_A(\pi^*_A(T(\Sigma_A(B))))) \ar[d]^-{T(\varepsilon^{\Sigma_A \,\dashv\, \pi^*_A}_{T(\Sigma_A(B))})}_-{\dcomment{\text{nat. of } \eta^{\Sigma_A \,\dashv\, \pi^*_A}}\qquad\qquad}
\\
& T(\Sigma_A(B)) \ar[r]^-{T(\eta_{\Sigma_A(B)})} \ar@/_2pc/[ddr]^-{\!\!\!\!\id_{T(\Sigma_A(B))}}_<<<<<<{\dcomment{\text{id. law}}\quad\!\!\!\!} & T(T(\Sigma_A(B))) \ar[dd]^-{\mu_{\Sigma_A(B)}}_<<<<<<<<<<<{\dcomment{(T,\eta,\mu) \text{ is a monad}}\qquad}
\\
\ar@{}[d]^-{\qquad\quad\dcomment{(B,\beta) \text{ is an EM-algebra}}} &
\\
T(\Sigma_A(T(B))) \ar[rr]_-{T(\Sigma_A(\beta))} & & T(\Sigma_A(B))
}
\]

\pagebreak

Next, we define the action of $\Sigma^{\mathbf{T}}_A$ on morphisms of $\mathcal{V}^{\mathbf{T}}_{\ia A}$ using the universal property of reflexive coequalizers. In detail, given a morphism $h : (B,\beta) \longrightarrow (B',\beta')$, we define the corresponding morphism $\Sigma^{\mathbf{T}}_A(h)$ in $\mathcal{V}^{\mathbf{T}}_{p(A)}$ as the unique mediating morphism in
\[
\xymatrix@C=4.5em@R=4em@M=0.5em{
(T(\Sigma_A(T(B'))),\mu_{\Sigma_A(T(B'))}) \ar@<-0.75ex>[r]_-{(2)'} \ar@<0.75ex>[r]^-{(1)'} & (T(\Sigma_A(B')), \mu_{\Sigma_A(B')}) \ar[r]^-{e_{A,(B',\beta')}} & \Sigma^{\mathbf{T}}_A(B',\beta')
\\
(T(\Sigma_A(T(B))),\mu_{\Sigma_A(T(B))}) \ar[u]^-{T(\Sigma_A(T(h)))} \ar@<-0.75ex>[r]_-{(2)} \ar@<0.75ex>[r]^-{(1)} & (T(\Sigma_A(B)), \mu_{\Sigma_A(B)}) \ar[r]_-{e_{A,(B,\beta)}} \ar[u]_-{T(\Sigma_A(h))} & \Sigma^{\mathbf{T}}_A(B,\beta) \ar@{-->}[u]_-{\Sigma^{\mathbf{T}}_A(h)}
}
\]

In order for $\Sigma^{\mathbf{T}}_A(h)$ to exist and to be the unique such morphism, we need to prove 
\[
e_{A,(B',\beta')} \comp T(\Sigma_A(h)) \comp (1) = e_{A,(B',\beta')} \comp T(\Sigma_A(h)) \comp (2)
\]
We prove this equation by first observing that we have
\[
e_{A,(B',\beta')} \comp (1)' = e_{A,(B',\beta')} \comp (2)'
\]
because $e_{A,(B',\beta')}$ is the reflexive coequalizer of $(1)'$ and $(2)'$.
As a result, it suffices to show that the two left-hand squares given in the above diagram commute.

The left-hand square involving $(1)$ and $(1)'$ commutes because  $h$ is a EM-algebra homo\-morphism---this is best seen when we rotate this square by $90$ degrees:
\[
\xymatrix@C=4.5em@R=3em@M=0.5em{
T(\Sigma_A(T(B))) \ar[r]^-{T(\Sigma_A(T(h)))} \ar[d]_-{T(\Sigma_A(\beta))} & T(\Sigma_A(T(B'))) \ar[d]^-{T(\Sigma_A(\beta'))}
\\
T(\Sigma_A(B)) \ar[r]_-{T(\Sigma_A(h))} & T(\Sigma_A(B'))
}
\]
The other left-hand square, involving $(2)$ and $(2)'$, commutes due to naturality:
\[
\xymatrix@C=7em@R=7.5em@M=0.5em{
T(\Sigma_A(T(B))) \ar[r]^-{T(\Sigma_A(T(h)))} \ar[d]_-{T(\Sigma_A(T(\eta^{\Sigma_A \,\dashv\, \pi^*_A}_B)))} & T(\Sigma_A(T(B'))) \ar[d]^-{T(\Sigma_A(T(\eta^{\Sigma_A \,\dashv\, \pi^*_A}_{B'})))}_-{\dcomment{\text{nat. of } \eta^{\Sigma_A \,\dashv\, \pi^*_A}}\qquad\qquad\qquad\,\,}
\\
T(\Sigma_A(T(\pi^*_A(\Sigma_A(B))))) \ar[r]_-{T(\Sigma_A(T(\pi^*_A(\Sigma_A(h)))))} \ar[d]_-{=} & T(\Sigma_A(T(\pi^*_A(\Sigma_A(B'))))) \ar[d]^-{=}_-{\dcomment{T \text{ is split fibred}}\qquad\qquad\qquad}
\\
T(\Sigma_A(\pi^*_A(T(\Sigma_A(B))))) \ar[r]_-{T(\Sigma_A(\pi^*_A(T(\Sigma_A(h)))))} \ar[d]_-{T(\varepsilon^{\Sigma_A \,\dashv\, \pi^*_A}_{T(\Sigma_A(B))})} & T(\Sigma_A(\pi^*_A(T(\Sigma_A(B'))))) \ar[d]^-{T(\varepsilon^{\Sigma_A \,\dashv\, \pi^*_A}_{T(\Sigma_A(B'))})}_-{\dcomment{\text{nat. of } \varepsilon^{\Sigma_A \,\dashv\, \pi^*_A}}\qquad\qquad\qquad\,\,}
\\
T(T(\Sigma_A(B))) \ar[r]_-{T(T(\Sigma_A(h)))} \ar[d]_-{\mu_{\Sigma_A(B)}} & T(T(\Sigma_A(B'))) \ar[d]^-{\mu_{\Sigma_A(B')}}_-{\dcomment{\text{nat. of } \mu}\qquad\qquad\qquad\quad}
\\
T(\Sigma_A(B)) \ar[r]_-{T(\Sigma_A(h))} & T(\Sigma_A(B'))
}
\]

We omit the proofs showing that $\Sigma^{\mathbf{T}}_A$ preserves identities and composition---both  properties follow directly from using the universal property of reflexive coequalizers.

We proceed by proving that we have an adjunction $ \Sigma^{\mathbf{T}}_A  \dashv \pi^*_A : \mathcal{V}^{\mathbf{T}}_{p(A)} \longrightarrow \mathcal{V}^{\mathbf{T}}_{\ia A}$.

\pagebreak

First, the underlying morphism of a component $\eta^{\Sigma^{\mathbf{T}}_A \,\dashv\, \pi^*_A}_{(B,\beta)}$ of the unit natural transformation
\[
\eta^{\Sigma^{\mathbf{T}}_A \,\dashv\, \pi^*_A} : \id_{\mathcal{V}_{\ia A}} \longrightarrow \pi^*_A \comp \Sigma^{\mathbf{T}}_A
\]
is given by the following composite morphism:
\[
\xymatrix@C=4em@R=4em@M=0.5em{
B \ar[r]^-{\eta^{\Sigma_A \,\dashv\, \pi^*_A}_B} & \pi^*_A(\Sigma_A(B)) \ar[r]^-{\pi^*_A(\eta_{\Sigma_A(B)})} & \pi^*_A(T(\Sigma_A(B))) \ar[r]^-{\pi^*_A(e_{A,(B,\beta)})} & \pi^*_A(U^{\mathbf{T}}(\Sigma^{\mathbf{T}}_A(B,\beta)))
}
\]
which we prove to be a EM-algebra homomorphism from $(B,\beta)$ to  $\pi^*_A(\Sigma^{\mathbf{T}}_A(B,\beta))$ by \linebreak showing that the next diagram commutes in $\mathcal{V}_{\ia A}$.
\[
\scriptsize
\xymatrix@C=5.5em@R=6em@M=0.5em{
T(B) \ar@/_3pc/[dddd]_-{\beta} \ar[dd]^-{\eta^{\Sigma_A \,\dashv\, \pi^*_A}_{T(B)}} \ar[r]^-{T(\eta^{\Sigma_A \,\dashv\, \pi^*_A}_B)} & T(\pi^*_A(\Sigma_A(B))) \ar[r]^-{T(\pi^*_A(\eta_{\Sigma_A(B)}))} \ar[d]_-{=}_-{\dscomment{(b)}\qquad\qquad\quad} & T(\pi^*_A(T(\Sigma_A(B)))) \ar[r]^-{T(\pi^*_A(e_{A,(B,\beta)}))} \ar[d]_-{=}_-{\dscomment{\eta \text{ is split fibred}}\qquad\qquad} & T(\pi^*_A(U^{\mathbf{T}}(\Sigma^{\mathbf{T}}_A(B,\beta)))) \ar[d]^-{=}_-{\dscomment{T \text{ is split fibred}}\qquad\qquad}
\\
 & \pi^*_A(T(\Sigma_A(B))) \ar[r]^-{\pi^*_A(T(\eta_{\Sigma_A(B)}))} \ar@/^2pc/[dddr]_<<<<<<<{\id_{\pi^*_A(T(\Sigma_A(B)))}\!\!\!\!\!\!} & \pi^*_A(T(T(\Sigma_A(B)))) \ar[r]^-{\pi^*_A(T(e_{A,(B,\beta)}))} \ar[ddd]^-{\pi^*_A(\mu_{\Sigma_A(B)})}_<<<<<<{\dscomment{(T,\eta,\mu) \text{ a monad}}\quad} & \pi^*_A(T(U^{\mathbf{T}}(\Sigma^{\mathbf{T}}_A(B,\beta)))) \ar[ddd]^-{\pi^*_A(\beta_{\Sigma^{\mathbf{T}}_A})}_<<<<<<<<<<<<<<<{\dscomment{h \text{ is an EM-algebra homomorphism}}\quad\,\,\,\,\,}
\\
\pi^*_A(\Sigma_A(T(B))) \ar[r]^-{\!\pi^*_A(\eta_{\Sigma_A(T(B))})} \ar@/_2pc/[ddr]^-{\pi^*_A(\Sigma_A(\beta))}_-{\dscomment{\text{nat. of } \eta^{\Sigma_A \,\dashv\, \pi^*_A}}} & \pi^*_A(T(\Sigma_A(T(B)))) \ar@/_2pc/[ddr]_-{\pi^*_A(T(\Sigma_A(\beta)))\!\!\!\!}_<<<<<<{\dscomment{\text{nat. of } \eta}\qquad\qquad} \ar@/^2pc/[ddr]^<<<<<<{\!\!\!\pi^*_A((2))}_-{\dscomment{(a)}\qquad}
\\
&
\\
B \ar[r]_-{\eta^{\Sigma_A \,\dashv\, \pi^*_A}_B} & \pi^*_A(\Sigma_A(B)) \ar[r]_-{\pi^*_A(\eta_{\Sigma_A(B)})} & \pi^*_A(T(\Sigma_A(B))) \ar[r]_-{\pi^*_A(e_{A,(B,\beta)})} & \pi^*_A(U^{\mathbf{T}}(\Sigma^{\mathbf{T}}_A(B,\beta)))
}
\]
Here, the subdiagram marked with $(a)$ commutes because $e_{A,(B,\beta)}$ is the coequalizer of $T(\Sigma_A(\beta))$ and $(2)$; and the subdiagram marked with $(b)$ commutes because we have
\[
\scriptsize
\xymatrix@C=5.5em@R=6em@M=0.5em{
T(B) \ar[rr]^-{T(\eta^{\Sigma_A \,\dashv\, \pi^*_A}_B)} \ar[d]_-{\eta^{\Sigma_A \,\dashv\, \pi^*_A}_{T(B)}}^-{\qquad\qquad\qquad\qquad\qquad\dscomment{\text{nat. of } \eta^{\Sigma_A \,\dashv\, \pi^*_A}}} & & T(\pi^*_A(\Sigma_A(B))) \ar[dd]^-{=} \ar@/_2.5pc/[dddl]_>>>>>>>>>>>>>>>>>>>>>>>>>{\eta_{T(\pi^*_A(\Sigma_A(B)))}\!\!\!}^-{\,\,\,\,\,\,\dscomment{(T,\eta,\mu) \text{ is a monad}}} \ar[dl]_-{\eta^{\Sigma_A \,\dashv\, \pi^*_A}_{T(\pi^*_A(\Sigma_A(B)))}}
\\
\pi^*_A(\Sigma_A(T(B))) \ar[d]_-{\pi^*_A(\eta_{\Sigma_A(T(B))})}^>>>>>{\qquad\quad\dscomment{\text{nat. of } \eta}} \ar[r]^-{\pi^*_A(\Sigma_A(T(\eta^{\Sigma_A \,\dashv\, \pi^*_A}_B)))} & \pi^*_A(\Sigma_A(T(\pi^*_A(\Sigma_A(B))))) \ar[ddl]_<<<<<<<<<<<<{\pi^*_A(\eta_{\Sigma_A(T(\pi^*_A(\Sigma_A(B))))})\!\!\!\!}^>>>>>>>>>>>>>>>>>>>>>>>>>{\qquad\qquad\dscomment{\text{nat. of } \eta}}^>>>>>>>>>>>>>>{\qquad\qquad\dscomment{\eta \text{ is split fibred}}}
\\
\pi^*_A(T(\Sigma_A(T(B)))) \ar[d]_-{\pi^*_A(T(\Sigma_A(T(\eta^{\Sigma_A \,\dashv\, \pi^*_A}_B))))} & & \pi^*_A(T(\Sigma_A(B)))
\\
\pi^*_A(T(\Sigma_A(T(\pi^*_A(\Sigma_A(B)))))) \ar[dd]_-{=}^-{\qquad\quad\dscomment{T \text{ is split fibred}}} & T(T(\pi^*_A(\Sigma_A(B)))) \ar[d]_-{T(\eta^{\Sigma_A \,\dashv\, \pi^*_A}_{T(\pi^*_A(\Sigma_A(B)))})} \ar@/_2.5pc/[uuur]_<<<<<<<<<<<<<<{\!\!\!\mu_{\pi^*_A(\Sigma_A(B))}}
\\
& T(\pi^*_A(\Sigma_A(T(\pi^*_A(\Sigma_A(B)))))) \ar[ul]^-{=} \ar[dl]_-{=}^<<<<<<<{\qquad\qquad\qquad\dscomment{\Sigma_A \dashv \pi^*_A}}^>>>>>>>>>>>>>>>{\!\!\!\qquad\qquad\qquad\qquad\qquad\qquad\qquad\dscomment{T \text{ is split fibred}}}
\\
\pi^*_A(T(\Sigma_A(\pi^*_A(T(\Sigma_A(B)))))) \ar[rr]_-{\pi^*_A(T(\varepsilon^{\Sigma_A \,\dashv\, \pi^*_A}_{T(\Sigma_A(B))}))} & & \pi^*_A(T(T(\Sigma_A(B)))) \ar[uuu]_-{\pi^*_A(\mu_{\Sigma_A(B)})}^>>>>>>>>>>>>>>>>>>>>>>>>{\dscomment{\mu \text{ is split fibred}}\qquad\quad} \ar[uul]_-{=}
}
\]

We omit the proof showing that $\eta^{\Sigma^{\mathbf{T}}_A \,\dashv\, \pi^*_A}$ is natural---it follows directly from the naturality of $\eta$ and $\eta^{\Sigma_A \,\dashv\, \pi^*_A}$, combined with the definition of $\Sigma^{\mathbf{T}}_A$ on morphisms.

Next, the underlying morphism of a component $\varepsilon^{\Sigma^{\mathbf{T}}_A \,\dashv\, \pi^*_A}_{(B,\beta)}$ of the counit natural transformation
\[
\varepsilon^{\Sigma^{\mathbf{T}}_A \,\dashv\, \pi^*_A} : \Sigma^{\mathbf{T}}_A \comp \pi^*_A \longrightarrow \id_{\mathcal{V}_{p(A)}} 
\]
is given by the unique mediating morphism in 

\[
\hspace{-0.05cm}
\xymatrix@C=1.5em@R=3em@M=0.5em{
& & (B,\beta)
\\
(T(\Sigma_A(T(\pi^*_A(B)))),\mu_{\Sigma_A(T(\pi^*_A(B)))}) \ar@<-0.75ex>[r]_-{(2)} \ar@<0.75ex>[r]^-{(1)} & (T(\Sigma_A(\pi^*_A(B))), \mu_{\Sigma_A(\pi^*_A(B))}) \ar[dr]_-{e_{A,\pi^*_A(B,\beta)}\,\,\,\,} \ar[ur]^-{\beta \,\,\comp\,\, T(\varepsilon^{\Sigma_A \,\dashv\, \pi^*_A}_B)\,\,\,\,\,\,\,\,\,\,} & 
\\
& & \Sigma^{\mathbf{T}}_A(\pi^*_A(B,\beta)) \ar@{-->}[uu]_-{\varepsilon^{\Sigma^{\mathbf{T}}_A \,\dashv\, \pi^*_A}_{(B,\beta)}}
}
\]
using the universal property of the reflexive coequalizer $e_{A,(\pi^*_A(B),\pi^*_A(\beta))}$. In order to do so, we first prove that $\beta \comp T(\varepsilon^{\Sigma_A \,\dashv\, \pi^*_A}_B)$ is an EM-algebra homomomorphism, by showing
\[
\xymatrix@C=10em@R=6em@M=0.5em{
T(T(\Sigma_A(\pi^*_A(B)))) \ar[d]_-{\mu_{\Sigma_A(\pi^*_A(B))}} \ar[r]^-{T(T(\varepsilon^{\Sigma_A \,\dashv\, \pi^*_A}_B))} & T(T(B)) \ar[d]^-{\mu_B}_-{\dcomment{\text{nat. of } \mu}\qquad\qquad\qquad\,\,} \ar[r]^-{T(\beta)} & T(B) \ar[d]^-{\beta}_-{\dcomment{(B,\beta) \text{ is an EM-algebra}}\qquad\,\,\,}
\\
T(\Sigma_A(\pi^*_A(B))) \ar[r]_-{T(\varepsilon^{\Sigma_A \,\dashv\, \pi^*_A}_B)} & T(B) \ar[r]_-{\beta} & B
}
\]
Further, for $\varepsilon^{\Sigma^{\mathbf{T}}_A \,\dashv\, \pi^*_A}_{(B,\beta)}$ to exist and be unique such morphism, we also need to show that
\[
\beta \comp T(\varepsilon^{\Sigma_A \,\dashv\, \pi^*_A}_B) \comp (1) 
= 
\beta \comp T(\varepsilon^{\Sigma_A \,\dashv\, \pi^*_A}_B) \comp (2) 
\]
This last equation follows from the commutativity of the next diagram in $\mathcal{V}_{p(A)}$.

\[
\scriptsize
\xymatrix@C=9.5em@R=6em@M=0.5em{
T(\Sigma_A(T(\pi^*_A(B)))) \ar[r]^-{=} \ar[d]_-{T(\Sigma_A(T(\eta^{\Sigma_A \,\dashv\, \pi^*_A}_{\pi^*_A(B)})))}^>>>>>>{\qquad\qquad\dscomment{\Sigma_A \dashv \pi^*_A}} \ar[dr]^-{\id_{T(\Sigma_A(T(\pi^*_A(B))))}}^>>>>>>>>>{\qquad\qquad\quad\dscomment{\text{id. law}}} & T(\Sigma_A(\pi^*_A(T(B)))) \ar[r]^-{T(\Sigma_A(\pi^*_A(\beta)))} \ar@/^4.5pc/[ddd]^-{T(\varepsilon^{\Sigma_A \,\dashv\, \pi^*_A}_{T(B)})} & T(\Sigma_A(\pi^*_A(B))) \ar[ddd]^-{T(\varepsilon^{\Sigma_A \,\dashv\, \pi^*_A}_B)}_<<<<<<<<<<{\dscomment{\text{nat. of } \varepsilon^{\Sigma_A \,\dashv\, \pi^*_A}}\qquad\quad}
\\
T(\Sigma_A(T(\pi^*_A(\Sigma_A(\pi^*_A(B)))))) \ar[d]_-{=} \ar[r]_-{T(\Sigma_A(T(\pi^*_A(\varepsilon^{\Sigma_A \,\dashv\, \pi^*_A}_B))))} & T(\Sigma_A(T(\pi^*_A(B)))) \ar[d]_-{=}_-{\dscomment{T \text{ is split fibred}}\qquad\qquad\qquad\quad\!\!\!\!}
\\
T(\Sigma_A(\pi^*_A(T(\Sigma_A(\pi^*_A(B)))))) \ar[d]_-{T(\varepsilon^{\Sigma_A \,\dashv\, \pi^*_A}_{T(\Sigma_A(\pi^*_A(B)))})} \ar[r]_-{T(\Sigma_A(\pi^*_A(T(\varepsilon^{\Sigma_A \,\dashv\, \pi^*_A}_B))))} & T(\Sigma_A(\pi^*_A(T(B)))) \ar[d]_-{T(\varepsilon^{\Sigma_A \,\dashv\, \pi^*_A}_{T(B)})}_-{\dscomment{\text{nat. of } \varepsilon^{\Sigma_A \,\dashv\, \pi^*_A}}\qquad\qquad\qquad\quad}
\\
T(T(\Sigma_A(\pi^*_A(B)))) \ar[d]_-{\mu_{\Sigma_A(\pi^*_A(B))}} \ar[r]_-{T(T(\varepsilon^{\Sigma_A \,\dashv\, \pi^*_A}_B))} & T(T(B)) \ar[d]_-{\mu_B}_-{\dscomment{\text{nat. of } \mu}\qquad\qquad\qquad\qquad\!\!\!} \ar[r]_-{T(\beta)} & T(B) \ar[d]^-{\beta}_-{\dscomment{(B,\beta) \text{ is an EM-algebra}}\qquad\qquad\,\,\,\,}
\\
T(\Sigma_A(\pi^*_A(B))) \ar[r]_-{T(\varepsilon^{\Sigma_A \,\dashv\, \pi^*_A}_B)} & T(B) \ar[r]_-{\beta} & B
}
\]

\pagebreak

Next, we prove that $\varepsilon^{\Sigma^{\mathbf{T}}_A \,\dashv\, \pi^*_A}$ is a natural transformation: given a morphism \linebreak $h : (B,\beta) \longrightarrow (B',\beta')$ in $\mathcal{V}^{\mathbf{T}}_{p(A)}$, we show that the following diagram commutes in $\mathcal{V}_{p(A)}$:
\[
\xymatrix@C=6em@R=9em@M=0.5em{
U^{\mathbf{T}}(\Sigma^{\mathbf{T}}_A(\pi^*_A(B,\beta))) \ar[rr]^-{\varepsilon^{\Sigma^{\mathbf{T}}_A \,\dashv\, \pi^*_A}_{(B,\beta)}} \ar@/_7pc/[ddd]_-{\Sigma^{\mathbf{T}}_A(\pi^*_A(h))} && B \ar[ddd]^-{h}_<<<<<<<<<<<<<<<<<<<<<<<<<<<<<<<<<<<<<<<<<<<<<<<<<<<<{\dcomment{h \text{ is EM-alg. hom.}}\,\,\,\,\,\,\,\,\,\,\,\,}
\\
T(\Sigma_A(\pi^*_A(B))) \ar[u]_-{e_{A,\pi^*_A(B,\beta)}}_-{\quad\qquad\qquad\qquad\dcomment{\text{def. of } \varepsilon^{\Sigma^{\mathbf{T}}_A \,\dashv\, \pi^*_A}_{(B,\beta)}}} \ar[d]^-{T(\Sigma_A(\pi^*_A(h)))}_-{\dcomment{\text{def. of } \Sigma^{\mathbf{T}}_A(\pi^*_A(h))}\,\,\,} \ar[r]^-{T(\varepsilon^{\Sigma_A \,\dashv\, \pi^*_A}_B)} & T(B) \ar[d]^-{T(h)}_-{\dcomment{\text{nat. of } \varepsilon^{\Sigma_A \,\dashv\, \pi^*_A}}\qquad} \ar[ur]_-{\beta}
\\
T(\Sigma_A(\pi^*_A(B'))) \ar[d]^-{e_{A,\pi^*_A(B',\beta')}}^-{\quad\qquad\qquad\qquad\dcomment{\text{def. of } \varepsilon^{\Sigma^{\mathbf{T}}_A \,\dashv\, \pi^*_A}_{(B',\beta')}}} \ar[r]_-{T(\varepsilon^{\Sigma_A \,\dashv\, \pi^*_A}_{B'})} & T(B') \ar[dr]^-{\beta'}
\\
U^{\mathbf{T}}(\Sigma^{\mathbf{T}}_A(\pi^*_A(B',\beta'))) \ar[rr]_-{\varepsilon^{\Sigma^{\mathbf{T}}_A \,\dashv\, \pi^*_A}_{(B',\beta')}} && B'
}
\]
and then recall an important property of coequalizers---they are epimorphisms. As $e_{A,(\pi^*_A(B),\pi^*_A(\beta))}$ is a coequalizer and thus an epimorphism, the outer square starting \linebreak at $U^{\mathbf{T}}(\Sigma^{\mathbf{T}}_A(\pi^*_A(B),\pi^*_A(\beta)))$ commutes because epimorphisms are right-cancellative.

Next, we prove that the two unit-counit laws hold for $\eta^{\Sigma^{\mathbf{T}}_A \,\dashv\, \pi^*_A}$ and $\varepsilon^{\Sigma^{\mathbf{T}}_A \,\dashv\, \pi^*_A}$, by showing that the next two diagrams commute in $\mathcal{V}_{\ia A}$ and $\mathcal{V}_{p(A)}$, respectively.

\pagebreak

\[
\scriptsize
\xymatrix@C=7em@R=4em@M=0.5em{
\pi^*_A(B) \ar[r]^-{\eta^{\Sigma^{\mathbf{T}}_A \,\dashv\, \pi^*_A}_{\pi^*_A(B,\beta)}} \ar[d]_-{\eta^{\Sigma_A \,\dashv\, \pi^*_A}_{\pi^*_A(B)}}^-{\quad\qquad\dscomment{\text{def. of } \eta^{\Sigma^{\mathbf{T}}_A \,\dashv\, \pi^*_A}}} \ar@/_5pc/[ddd]_-{\id_{\pi^*_A(B)}} & \pi^*_A(U^{\mathbf{T}}(\Sigma^{\mathbf{T}}_A(\pi^*_A(B,\beta)))) \ar@/^5pc/[ddd]^-{\pi^*_A(\varepsilon^{\Sigma^{\mathbf{T}}_A \,\dashv\, \pi^*_A}_{(B,\beta)})}
\\
\pi^*_A(\Sigma_A(\pi^*_A(B))) \ar[r]_-{\pi^*_A(\eta_{\Sigma_A(\pi^*_A(B))})} \ar[dd]_-{\pi^*_A(\varepsilon^{\Sigma_A \,\dashv\, \pi^*_A}_B)} \ar@{}[d]_-{\dscomment{\Sigma_A \dashv \pi^*_A}\quad}^>>>>{\qquad\qquad\dscomment{\text{nat. of } \eta}} & \pi^*_A(T(\Sigma_A(\pi^*_A(B)))) \ar[d]_-{\pi^*_A(T(\varepsilon^{\Sigma_A \,\dashv\, \pi^*_A}_B))}^-{\,\,\,\,\dscomment{\text{def. of } \varepsilon^{\Sigma^{\mathbf{T}}_A \,\dashv\, \pi^*_A}}} \ar[u]^-{\pi^*_A(e_{A,\pi^*_A(B,\beta)})}
\\
& \pi^*_A(T(B)) \ar[d]^-{\pi^*_A(\beta)}_>>>{\dscomment{(B,\beta) \text{ is an EM-algebra}}\qquad\!\!\!\!\!\!\!\!\!\!\!\!\!\!\!\!\!\!\!\!}
\\
\pi^*_A(B) \ar[r]_-{\id_{\pi^*_A(B)}} \ar[ur]^-{\pi^*_A(\eta_B)} & \pi^*_A(B)
}
\]

\[
\scriptsize
\xymatrix@C=3.25em@R=5em@M=0.5em{
U^{\mathbf{T}}(\Sigma^{\mathbf{T}}_A(B,\beta)) \ar[rrr]^-{\Sigma^{\mathbf{T}}_A(\eta^{\Sigma^{\mathbf{T}}_A \,\dashv\, \pi^*_A}_{(B,\beta)})} \ar@/_3.5pc/[dddddd]^<<<<<<<<<<<<{\id_{U^{\mathbf{T}}(\Sigma^{\mathbf{T}}_A(B,\beta))}}^<<<<<<<<<<<<<<<<<<<<<<<<<{\qquad\dscomment{\text{id. law}}} &&& U^{\mathbf{T}}(\Sigma^{\mathbf{T}}_A(\pi^*_A(\Sigma^{\mathbf{T}}_A(B,\beta)))) \ar[dddddd]^-{\varepsilon^{\Sigma^{\mathbf{T}}_A \,\dashv\, \pi^*_A}_{\Sigma^{\mathbf{T}}_A(B,\beta)}}
\\
& T(\Sigma_A(B)) \ar@{}[dd]^-{\quad\qquad\qquad\dscomment{\text{def. of } \eta^{\Sigma^{\mathbf{T}}_A \,\dashv\, \pi^*_A}}} \ar[ul]_-{e_{A,(B,\beta)}}_-{\quad\qquad\qquad\qquad\qquad\qquad\qquad\dscomment{\text{def. of } \Sigma^{\mathbf{T}}_A(\eta^{\Sigma^{\mathbf{T}}_A \,\dashv\, \pi^*_A}_{(B,\beta)})}} \ar[r]^-{T(\Sigma_A(\eta^{\Sigma^{\mathbf{T}}_A \,\dashv\, \pi^*_A}_{(B,\beta)}))} \ar[d]^>>>>>>{T(\Sigma_A(\eta^{\Sigma_A \,\dashv\, \pi^*_A}_B))}_>>>>>{\dscomment{\Sigma_A \dashv \pi^*_A}\,\,\,\,\,\,\,\,\,\,} \ar@/_4pc/[ddddl]_-{\id_{T(\Sigma_A(B))}} & T(\Sigma_A(\pi^*_A(U^{\mathbf{T}}(\Sigma^{\mathbf{T}}_A(B,\beta))))) \ar[ur]^-{e_{A,\pi^*_A(\Sigma^{\mathbf{T}}_A(B,\beta))}\,\,\,\,} \ar[ddd]^-{T(\varepsilon^{\Sigma_A \,\dashv\, \pi^*_A}_{U^{\mathbf{T}}(\Sigma^{\mathbf{T}}_A(B,\beta))})} \ar@{}[dd]^<<<<<<<<<<<<<{\,\,\,\,\,\qquad\qquad\dscomment{\text{def. of } \varepsilon^{\Sigma^{\mathbf{T}}_A \,\dashv\, \pi^*_A}}}
\\
& T(\Sigma_A(\pi^*_A(\Sigma_A(B)))) \ar[d]^-{T(\Sigma_A(\pi^*_A(\eta_{\Sigma_A(B)})))} \ar@/_2pc/[dddl]_<<<<<<<<<<{T(\varepsilon^{\Sigma_A \,\dashv\, \pi^*_A}_{\Sigma_A(B)})\!\!\!\!}
\\
& T(\Sigma_A(\pi^*_A(T(\Sigma_A(B))))) \ar@/_3pc/[uur]^>>>>>{T(\Sigma_A(\pi^*_A(e_{A,(B,\beta)})))} \ar[d]^-{T(\varepsilon^{\Sigma_A \,\dashv\, \pi^*_A}_{T(\Sigma_A(B))})}^-{\qquad\qquad\qquad\dscomment{\text{nat. of } \varepsilon^{\Sigma_A \,\dashv\, \pi^*_A}}}_-{\dscomment{\text{nat. of } \varepsilon^{\Sigma_A \,\dashv\, \pi^*_A}}\qquad} &
\\
& T(T(\Sigma_A(B))) \ar[r]_-{T(e_{A,(B,\beta)})} \ar[dr]_-{\mu_{\Sigma_A(B)}}^>>>>>>>>>{\qquad\qquad\dscomment{e_{A,(B,\beta)} \text{ is an EM-alg. hom.}}} & T(U^{\mathbf{T}}(\Sigma^{\mathbf{T}}_A(B,\beta))) \ar@/^2pc/[ddr]^-{\!\!\!\!\beta_{\Sigma^{\mathbf{T}}_A}}
\\
T(\Sigma_A(B)) \ar[ur]^-{T(\eta_{\Sigma_A(B)})}_-{\qquad\qquad\dscomment{(T,\eta,\mu) \text{ is a monad}}} \ar[rr]_-{\id_{T(\Sigma_A(B))}} && T(\Sigma_A(B)) \ar[dr]_-{e_{A,(B,\beta)}}
\\
U^{\mathbf{T}}(\Sigma^{\mathbf{T}}_A(B,\beta)) \ar[rrr]_-{\id_{U^{\mathbf{T}}(\Sigma^{\mathbf{T}}_A(B,\beta))}} &&& U^{\mathbf{T}}(\Sigma^{\mathbf{T}}_A(B,\beta))
}
\]
Similarly to the naturality proof of $\varepsilon^{\Sigma^{\mathbf{T}}_A \,\dashv\, \pi^*_A}$, the outer square in the second diagram commutes because $e_{A,(B,\beta)}$ is a coequalizer, and thus an epimorphism and right-cancellative.

We conclude our proof of the existence of split dependent $p$-sums by proving that the functors $\Sigma^{\mathbf{T}}_A $ satisfy the split Beck-Chevalley condition. In particular, we show that for any Cartesian morphism $\overline{f}(A) : f^*(A) \longrightarrow A$, the next diagram commutes. 
\[
\scriptsize
\xymatrix@C=5em@R=5.5em@M=0.5em{
U^{\mathbf{T}}(\Sigma^{\mathbf{T}}_{f^*(A)}(\ia {\overline{f}(A)}^*(B,\beta))) \ar[rr]^-{\Sigma^{\mathbf{T}}_{f^*(A)}(\ia {\overline{f}(A)}^*(\eta^{\Sigma^{\mathbf{T}}_A \,\dashv\, \pi^*_A}_{(B,\beta)}))} \ar@/_5pc/[dddddddd]_<<<<<<<{\id_{U^{\mathbf{T}}(\Sigma^{\mathbf{T}}_{f^*(A)}(\ia {\overline{f}(A)}^*(B,\beta)))}} && U^{\mathbf{T}}(\Sigma^{\mathbf{T}}_{f^*(A)}(\ia {\overline{f}(A)}^*(\pi^*_A(\Sigma^{\mathbf{T}}_A(B,\beta))))) \ar@/^4pc/[ddd]^-{=}_-{\dscomment{p \text{ is a s. fib.}}\quad}_-{\dscomment{p \text{ is a split fibration}}\qquad\qquad\qquad\qquad}
\\
\txt<5pc>{$T(\Sigma_{f^*(A)}($\\$\ia {\overline{f}(A)}^*(B)))$}
\ar@/_4pc/[dddd]^-{=}
\ar[d]^<<<<<{T(\Sigma_{f^*(A)}(\ia {\overline{f}(A)}^*(\eta^{\Sigma_A \,\dashv\, \pi^*_A}_B)))}^<<{\qquad\qquad\qquad\qquad\qquad\qquad\dscomment{\text{def. of } \eta^{\Sigma^{\mathbf{T}}_A \,\dashv\, \pi^*_A}}} \ar[u]_-{e_{f^*(A),\ia {\overline{f}(A)}^*(B,\beta)}}_-{\qquad\qquad\qquad\qquad\qquad\dscomment{\text{def. of } \Sigma^{\mathbf{T}}_{f^*(A)}(\ia {\overline{f}(A)}^*(\eta^{\Sigma^{\mathbf{T}}_A \,\dashv\, \pi^*_A}_{(B,\beta)})) }} \ar[rr]^-{T(\Sigma_{f^*(A)}(\ia {\overline{f}(A)}^*(\eta^{\Sigma^{\mathbf{T}}_A \,\dashv\, \pi^*_A}_{(B,\beta)})))} && 
\txt<5pc>{$T(\Sigma_{f^*(A)}(\ia {\overline{f}(A)}^*($\\$\pi^*_A(U^{\mathbf{T}}(\Sigma^{\mathbf{T}}_A(B,\beta))))))$} 
\ar[u]^-{e_{f^*(A),\ia {\overline{f}(A)}^*(\pi^*_A(\Sigma^{\mathbf{T}}_A(B,\beta)))}} \ar[d]_-{=}
\\
\txt<5pc>{$T(\Sigma_{f^*(A)}(\ia {\overline{f}(A)}^*($\\$\pi^*_A(\Sigma_A(B)))))$} \ar[d]^-{=}^-{\qquad\dscomment{p \text{ is a split fibration}}}_-{\dscomment{\text{split BC}}\quad}
\ar@/^1.5pc/[dr]^<<<<<{\,\,\,\,\,\,\,\quad T(\Sigma_{f^*(A)}(\ia {\overline{f}(A)}^*(\pi^*_A(\eta_{\Sigma_A(B)}))))} && 
\txt<5pc>{$T(\Sigma_{f^*(A)}(\pi^*_{f^*(A)}(f^*($\\$U^{\mathbf{T}}(\Sigma^{\mathbf{T}}_A(B,\beta))))))$} \ar@/_3pc/[dd]_-{T(\varepsilon^{\Sigma_{f^*(A)} \,\dashv\, \pi^*_{f^*(A)}}_{f^*(U^{\mathbf{T}}(\Sigma^{\mathbf{T}}_A(B,\beta)))})} 
\ar[d]_>>>>>>{\dshide{e_{A,\pi^*_{f^*(A)}(f^*(\Sigma^{\mathbf{T}}_A(B,\beta)))}}}
\\
\txt<5pc>{$T(\Sigma_{f^*(A)}(\pi^*_{f^*(A)}($\\$f^*(\Sigma_A(B)))))$} \ar@/^1.5pc/[dr]_>>>>>>>{T(\Sigma_{f^*(A)}(\pi^*_{f^*(A)}(f^*(\eta_{\Sigma_A(B)}))))\!\!\!} \ar[dd]^-{T(\varepsilon^{\Sigma_{f^*(A)} \,\dashv\, \pi^*_{f^*(A)}}_{f^*(\Sigma_A(B))})}
& 
\txt<5pc>{$T(\Sigma_{f^*(A)}(\ia {\overline{f}(A)}^*($\\$\pi^*_A(T(\Sigma_A(B))))))$} 
\ar@/^3.5pc/[uur]^-{T(\Sigma_{f^*(A)}(\ia {\overline{f}(A)}^*(\pi^*_A(e_{A,(B,\beta)}))))\!\!\!\!} \ar[d]_-{=} 
\ar@{}[dd]^<<<<<<<<<<<<<<<{\qquad\qquad\qquad\dscomment{\text{nat. of } \varepsilon^{\Sigma_{f^*(A)} \,\dashv\, \pi^*_{f^*(A)}}}}
& 
\txt<5pc>{$U^{\mathbf{T}}(\Sigma^{\mathbf{T}}_{f^*(A)}(\pi^*_{f^*(A)}($\\$f^*(\Sigma^{\mathbf{T}}_A(B,\beta)))))$} \ar@/^3.5pc/[ddddd]^<<<<<<<<<<<<<<<<{\varepsilon^{\Sigma^{\mathbf{T}}_{f^*(A)} \,\dashv\, \pi^*_{f^*(A)}}_{f^*(B,\beta)}}
\\
& 
\txt<5pc>{$T(\Sigma_{f^*(A)}(\pi^*_{f^*(A)}($\\$f^*(T(\Sigma_A(B))))))$} 
\ar@/^1.75pc/[uur]^>>>{T(\Sigma_{f^*(A)}(\pi^*_{f^*(A)}(f^*(e_{A,(B,\beta)}))))\,\,\,\,\,\,} \ar[d]_-{T(\varepsilon^{\Sigma_{f^*(A)} \,\dashv\, \pi^*_{f^*(A)}}_{f^*(T(\Sigma_A(B)))})}_-{\dscomment{\text{nat. of } \varepsilon^{\Sigma_{f^*(A)} \,\dashv\, \pi^*_{f^*(A)}}}\qquad\qquad\qquad\quad} 
\ar@{}[dd]^-{\qquad\qquad\qquad\dscomment{T \text{ is split fibred}}}
&
\txt<5pc>{$T(f^*(U^{\mathbf{T}}($\\$\Sigma^{\mathbf{T}}_{A}(B,\beta))))$} \ar[d]_-{=}
\\
T(f^*(\Sigma_A(B))) \ar[r]_-{T(f^*(\eta_{\Sigma_A(B)}))} \ar[d]_-{=}
& 
T(f^*(T(\Sigma_A(B)))) \ar[d]_-{=}_-{\dscomment{T \text{ is split fibred}}\qquad\qquad\quad} \ar[ur]^-{T(f^*(e_{A,(B,\beta)}))\,\,\,\,}
&
\txt<5pc>{$f^*(T(U^{\mathbf{T}}($\\$\Sigma^{\mathbf{T}}_{A}(B,\beta))))$} \ar@/_2pc/[ddd]_-{f^*(\beta_{\Sigma^{\mathbf{T}}_A})}^-{\dscomment{\text{def. of } \varepsilon^{\Sigma^{\mathbf{T}}_{f^*(A)} \,\dashv\, \pi^*_{f^*(A)}}}}
\\
f^*(T(\Sigma_A(B))) \ar[r]_{f^*(T(\eta_{\Sigma_A(B)}))} \ar[dr]_-{\id_{f^*(T(\Sigma_A(B)))}} 
& 
f^*(T(T(\Sigma_A(B)))) \ar[d]^-{f^*(\mu_{(\Sigma_A(B))})}^<{\!\!\!\!\!\!\!\!\quad\qquad\qquad\dscomment{e_{A,(B,\beta)} \text{ is an EM-alg. hom.}}} \ar[ur]^-{f^*(T(e_{A,(B,\beta)}))\,\,\,\,}
\\
\ar@{}[d]^-{\qquad\quad\dscomment{e_{A,(B,\beta)} \text{ is a split fibred refl. coequalizer}}}
& f^*(T(\Sigma_A(B))) \ar[dr]_-{f^*(e_{A,(B,\beta)})}
\\
U^{\mathbf{T}}(\Sigma^{\mathbf{T}}_{f^*(A)}(\ia {\overline{f}(A)}^*(B,\beta))) \ar[rr]_-{=} && f^*(U^{\mathbf{T}}(\Sigma^{\mathbf{T}}_A(B,\beta)))
}
\]

Analogously to the naturality proof of $\varepsilon^{\Sigma^{\mathbf{T}}_A \,\dashv\, \pi^*_A}$ and the proof of the second unit-counit law, the outer square again commutes because $e_{f^*(A),\ia {\overline{f}(A)}^*(B,\beta)}$ is a coequalizer, and therefore an epimorphism and right-cancellative.
\end{proof}

\renewcommand\thesection{\thechapter.\arabic{section}}

