\section{System Model} \label{sec:system_model}
In this section, we present the system model for a general network topology
and a general traffic demand model beyond the simple example expounded
in the previous section. Such models will be used to analyze the benefit
of D2D load balancing in general settings, in terms of spectrum
reduction ratio, and the cost in terms of D2D traffic overhead ratio.



\subsection{Cellular Network Topology} \label{sec:topology_model}
Consider an uplink wireless cellular network with multiple cells and multiple mobile users.
We assume that each cell has one BS and each user is associated with one BS\footnote{We say that user $u$ is associated with BS $b$ if user $u$ is
in the cellular cell covered by BS $b$. When a user is covered by multiple BSs,
we assume that this user has been associated with one of them, e.g., the one with the strongest signal-to-noise ratio.
In the rest of this paper,
we will also use the terminology, cell $b$, to represent the cell covered by BS $b$.}.
Define $\mathcal{B}$ as the set of all BSs,
$\mathcal{U}_b$ as the set of users belonging to BS $b \in \mathcal{B}$,
and $\mathcal{U} = \cup_{b \in \mathcal{B}} \mathcal{U}_b$ as
the set of all users in the cellular network.
Let $b_u \in \mathcal{B}$ denote the cell (or BS) with which user $u \in \mathcal{U}$ is associated.
We model the uplink cellular network topology as a directed graph
$\mathcal{G}=(\mathcal{V},\mathcal{E})$ with vertex set
$\mathcal{V} = \mathcal{U} \cup \mathcal{B}$ and edge set
$\mathcal{E}$ where $(u,v) \in \mathcal{E}$ if there is
a wireless link from vertex (user) $u \in \mathcal{U}$ to vertex (BS or user) $v \in \mathcal{V}$.


\subsection{Traffic Model} \label{sec:traffic_model}
We consider a time-slotted system with $T$ slots in total, indexed from 1 to $T$.
Each user can generate a delay-constrained traffic demand at the beginning of any slot.
We denote $\mathcal{J}$ as the demand set. Each demand $j \in \mathcal{J}$ is characterized by the tuple $(u_j, s_j, e_j, r_j)$
where
\begin{itemize}
\item $u_j \in \mathcal{U}$ is the user that generates demand $j$;
\item $s_j \ge 1$ is the starting time/slot of demand $j$;
\item $e_j \in [s_j, T]$ is the ending time/slot (deadline) of demand $j$;
\item $r_j > 0$ is the volume of demand $j$ with unit of bits.
\end{itemize}
Namely, demand $j$ is generated by user $u_j$ at the beginning of slot $s_j$
with the volume of $r_j$ bits and it must be delivered to BSs before/on the end of slot $e_j$,
implying a \emph{delay} requirement $(e_j - s_j +1)$.
We also call interval $[s_j, e_j]$ the \emph{lifetime} of the demand $j$.
We further denote $\mathcal{J}_b$
as the set of demands that are generated by the users in BS $b \in \mathcal{B}$, i.e.,
$
\mathcal{J}_b \triangleq \{j \in \mathcal{J}: u_j \in \mathcal{U}_b\}.
$
Demand $j$ is delivered in time if every bit of demand $j$ reaches a BS
before/on the end of slot $e_j$. Note that different bits in demand $j$ could reach different BSs.
Thus, every user can transmit a bit either to its own BS directly in a single hop or to another user
via the D2D link between them such that the bit can reach another BS in multiple hops.



\subsection{Wireless Channel/Spectrum Model}
For each link $(u,v)\in\mathcal{E}$, we
denote its link rate as $R_{u,v}$ (units: bits per slot per Hz), which
is the number of bits that can be transmitted in one unit (slot) of time resource and with
one unit (Hz) of spectrum resource. Then if we allocate $x \in \mathbb{R}^+$ (unit: Hz) spectrum to link $(u,v)$ at slot $t$,
this link can transmit $x\cdot R_{u,v}$ bits of data from node $u$ to node $v$ in slot $t$.
Note that we simplify the channel model by assuming a linear relationship between the allocated spectrum
and the transmitted data. This assumption is reasonable for the high-SNR scenario when we use
Shannon capacity as the link rate \cite{shannon}.
In addition, we assume that the total spectrum is not divided into uplink spectrum and downlink
spectrum. Instead, our scheme allocates spectrum from a spectrum pool to mobile users for transmitting or
receiving data. Thus, in this paper, we do not consider the switching issue between uplink spectrum and downlink spectrum.





\subsection{Performance Metrics} \label{sec:performance_metrics}
In this paper, we aim at minimizing the total (amount of) spectrum to deliver all demands in $\mathcal{J}$ in time.
In particular, we need to obtain the minimum spectrum/frequency to serve all demands in time without D2D (resp. with D2D), denoted by $F^{\textsf{ND}}$
(resp. $F^{\textsf{D2D}}$).
To evaluate the impact of D2D load balancing, we characterize both the benefit and the cost for D2D load balancing.
The benefit is in terms of \emph{spectrum reduction ratio},
\be
\rho \triangleq \frac{F^{\textsf{ND}}-F^{\textsf{D2D}}}{F^{\textsf{ND}}}\in [0,1).
\label{equ:spectrum-reduction-ratio}
\ee
The cost is in terms of  \emph{(D2D traffic) overhead ratio},
\be
\eta \triangleq \frac{V^{\textsf{D2D}}}{V^{\textsf{D2D}}+V^{\textsf{BS}}}\in [0,1),
\label{equ:overhead-ratio}
\ee
where $V^{\textsf{D2D}}$ is the volume of all D2D traffic and $V^{\textsf{BS}}$
is the volume of all traffic directly sent by cellular users to BSs.

The spectrum reduction ratio $\rho$ evaluates how much spectrum we can save if we apply D2D load balancing.
The overhead ratio $\eta$ evaluates the percentage of D2D traffic among all traffic.
D2D traffic incurs cost in the sense that any traffic going through D2D links
will consume spectrum and energy of user devices but do not immediately reach any BS.
Overall, the spectrum reduction ratio $\rho$ captures
the benefit of D2D load balancing and hence larger $\rho$ means larger benefit;
the overhead ratio $\eta$ captures the cost of D2D load balancing
and hence smaller $\eta$ means smaller cost.
In the following, we will discuss how to obtain $F^{\textsf{ND}}$ in Sec. \ref{sec:optimal_no_d2d}
and $F^\textsf{D2D}$ in Sec. \ref{sec:optimal_d2d}.
Then we will show the theoretical upper bounds for $\rho$ and $\eta$ in Sec. \ref{sec:theoretical_results}.
%and empirical evaluations in Sec. \ref{sec:simulation}.
