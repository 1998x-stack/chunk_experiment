\section{Construction}
\label{sec:construct}
It remains to prove the converse of Theorem~\ref{thm:whichexp}
as well as Theorem~\ref{thm:countexp}.
This estimate requires several different components.

\subsection{Decomposition of functions as sums of two permutations}
We take the following lemma from \cite{SL2005C7}.
\begin{lemma}
	\label{lem:2005C7}
	Let $G$ be a finite abelian group.
	Given a function $f \colon G \to G$
	for which $\sum_{g \in G} f(g) = 0$,
	there exists two permutations $\pi_1, \pi_2 \colon G \to G$
	for which \[ f = \pi_1 + \pi_2. \]
\end{lemma}
The results of \cite[Theorem 1.3]{eberhard} suggest
that it may be possible to improve this bound significantly
given ``reasonable'' assumptions on $f$,
but we will not do so here.

\subsection{Splitting Lemma}
For a set $T$ let $\Sigma T$ denotes the sum of the elements of $T$.
We prove the following result.
\begin{lemma}
	\label{lem:split}
	Let $G$ be a finite abelian group of order $N$,
	and let $S = G \coprod G$ be considered a set of $2N$
	distinct elements.
	Then there exists at least
	\[ \frac{4^N}{2(N+1)^{\frac32}} \]
	subsets $T \subset S$ for which $|T| = N$,
	$\Sigma T = 0$.
\end{lemma}
\begin{proof}
	According to the structure theorem of abelian groups
	we may write $G = \ZZ/r_1 \times \dots \times \ZZ/r_m$,
	where $r_1 \mid r_2 \mid \dots \mid r_m$.
	In this way, we may think of each element $g \in G$
	as a vector $g = (g_1, \dots, g_m) \in G$.
	(In particular $(\Sigma T)_1$ refers to the first coordinate of $\Sigma T$,
	since $\Sigma T \in G$).

	For each $i$ let $\zeta_i$ be a primitive $r_i$th root of unity,
	and let $\eta$ be a primitive $N$th root of unity.
	We now define
	\begin{align*}
		F(e_1, \dots, e_m, d)
		&= \prod_{g \in G} \left( 1 +
			\zeta_1^{e_1 g_1} \cdots \zeta_m^{e_m g_m} \eta^{d} \right)^2 \\
		&= \sum_{T \subset S}
			\zeta_1^{e_1 (\Sigma T)_1}
			\cdots \zeta_m^{e_m (\Sigma T)_m}
			\eta^{|T|}.
	\end{align*}
	Now consider the sum
	\[ A = \sum_{e_1=0}^{r_1-1} \dots \sum_{e_m=0}^{r_m-1}
		\sum_{d=0}^{N-1} F(e_1, \dots, e_m, d). \]
	On the one hand, we find that
	\begin{align*}
		A &= \sum_{e_1=0}^{r_1-1} \dots \sum_{e_m=0}^{r_m-1}
			\sum_{\substack{T \subset S
				\\ |T| \equiv 0 \pmod n}}
			N \zeta_1^{e_1 (\Sigma T)_1} \cdots \zeta_m^{e_m (\Sigma T)_m} \\
		&= \sum_{\substack{T \subset S
				\\ |T| \equiv 0 \pmod n}}
			N \prod_{i=1}^m \left(  \sum_{e_i=0}^{r_i-1}
				\zeta_i^{e_i (\Sigma T)_i} \right) \\
		&= \sum_{\substack{T \subset S
				\\ |T| \equiv 0 \pmod n \\ \Sigma T = 0}}
			N r_1 \cdots r_m \\
		&= N^2 \# \left\{ T \subset S :
			|T| \equiv 0 \pmod n, \; \Sigma T = 0 \right\} \\
		&= N^2 \left( 2 + \# \left\{ T \subset S :
			|T| = n, \; \Sigma T = 0 \right\} \right).
	\end{align*}
	On the other hand, we have the bounds
	\[ |F(e_1, \dots, e_m, d)| < \left( 2^{\frac{N}{r_i}} \right)^2
		\text{ if } e_i \neq 0. \]
		Moreover,
	\[ \sum_d F(0,\dots,0,d) = \sum_d (1+\eta^d)^{2N}
		= N \left( 2 + \binom{2N}{N} \right). \]
	Thus, we have the estimate
	\[ A \ge N \left( 2 + \binom{2N}{N} \right) - N(N-1) \cdot 2^N \]
	and consequently
	\[
		\# \left\{ T \subset S :
			|T| = n, \; \Sigma T = 0 \right\}
		\ge -2 + \frac{2 + \binom{2N}{N} - (N-1) \cdot 2^N}{N}.
	\]
	Using the estimate $\binom{2N}{N} \ge \frac{4^N}{\sqrt{4N}}$
	one can verify the above is at least
	\[ \frac{A}{N^2}-2 \ge \frac{4^N}{2(N+1)^{3/2}} \]
	for $N \ge 8$.
	All that remains is to examine the cases $N \le 7$,
	which can be checked by hand by explicitly computing $A$.
\end{proof}
\begin{remark*}
	Lemma~\ref{lem:split} has appeared in various specializations;
	for example, the case where $G = \ZZ/p$ was 
	the closing problem of the 1996 International Mathematical Olympiad,
	in which the exact answer
	$\frac1p \left( \binom{2p}{p} -2 \right) + 2$ is known.
\end{remark*}

\subsection{Main construction}
We now prove Theorem~\ref{thm:countexp}.
\begin{proof}
We begin by constructing a partially ordered set
on the divisors of $p-1 = 2q_1 \cdots q_k$, ordered by divisibility;
hence we obtain the Boolean lattice with $2^{k+1}$ elements.
At the node $d$ in the poset we write down the elements $x \in \{1, \dots, n-1\}$
for which $\gcd(x,p-1) = d$;
this gives $2\varphi( (p-1)/d )$ elements written at each node except the first one,
for which we have $2\varphi(p-1)+1$ elements.

Then, we iteratively repeat the following process,
starting at the bottom node $d=1$:
\begin{itemize}
	\ii Note there are three labels which are $1 \pmod{\frac{p-1}{d}}$.
	Pick one of these three numbers $x$ arbitrarily, and erase it.
	\ii If $d=p-1$, stop.
	Otherwise, pick one node $d'$ immediately above $d$,
	and write $x$ at that node $d'$.
	\ii Move to the node $d'$,
	which now has three labels which are $1 \pmod{\frac{p-1}{d'}}$,
	and continue the process.
\end{itemize}
An example of this process with $n=14$ is shown in Figure~\ref{fig:ex1}.

\begin{figure}[ht]
	\begin{center}
	\begin{tikzcd}
		& 6: \{6, 12\} \ar[ld, dash] \ar[rd, dash] & \\
		2: \{2,4,8,10\} \ar[rd, dash] & & 3: \{3, 9\} \ar[ld, dash] \\
		& 1: \{1,5,7,11,13\} &
	\end{tikzcd}
	\bigskip
	\begin{tikzcd}
		& 6: \{6, \mathbf{10} \} \ar[ld, leftarrow, "10", swap] \ar[rd, dash]
		\ar[r, "12"]
		& (\text{delete } 12) \\
		2: \{2,4,\mathbf{7},8\} \ar[rd, leftarrow, "7"] & & 3: \{3, 9\} \ar[ld, dash] \\
		& 1: \{1,5,11,13\} &
	\end{tikzcd}
	\end{center}
	\caption{An example of the algorithm described.
	The initial poset before the algorithm is shown on top.
	Thereafter, we pick the chain $1 \to 2 \to 6$
	and move the elements $7$, $10$, $12$.
	This gives the poset at the bottom.}
	\label{fig:ex1}
\end{figure}

Evidently, there are $3^{k+2} (k+1)!$ ways to run the algorithm,
and each application gives a different set of labels at the end.
We will use each labeled poset to exhibit several exponential orthomorphisms.
For each $d \mid p-1$, let $L_d$ denote the labels at the node $d$.

As in the previous section,
we identify all the elements of $\{1, \dots, 2p-1\} \setminus \{p\}$
with the set
\[ Z = E \sqcup O = (\ZZ/p)^\times \sqcup (\ZZ/p)^\times. \]
Now consider any $d \mid p-1$, let $e= \frac{p-1}{d}$
and let $m = \varphi(e)$.
There are $2m$ elements $x \in Z$
for which $R_{p-1}(x) = d$;
they can be thought of as $G \sqcup G$ where
$G = (\ZZ/\frac{p-1}{d})^\times \cong \ZZ/m$.
The labels written at node $d$ can be thought of in the same way.

We will match these to the labels written at the node $d$ in our poset.
By Lemma~\ref{lem:split}, the number of ways to split the labels 
into two halves $L = L_E \sqcup L_O$,
such that each half has vanishing product,
is at least
\[ \max\left(\frac{4^{m}}{2(m+1)^{3/2}}, 2\right)
	\ge \frac{4^{\varphi(e)}}{2e^{3/2}}. \]
(Here we have used the fact that $\varphi(e)+1 \le e$ for $e \neq 1$).
Moreover, by Lemma~\ref{lem:2005C7},
there exists at least one way to choose a bijection
$\sigma \colon E \to L_E$ so that the map $x \mapsto x\sigma(x)$
is a bijection on $E$;
of course the analogous result holds for $\sigma \colon O \to L_O$.
Hence we've defined $\sigma$ as a bijection
on the elements $x \in Z$ with $R_{p-1}(x) = d$, as desired.

Finally, we label the special element $p$
with the single unused number left over from the algorithm.
Thus we get a bijection $\sigma$ on
the entirety of $\{1, \dots, 2p-1\}$.

The number of orthomorphisms we've constructed is at least
\begin{align*}
	(k+2)! \cdot 3^{k+1}
	\prod_{e \mid p-1} \frac{4^{\varphi(e)}}{2e^{3/2}}
	&= (k+2)! \cdot 3^{k+1}
	\frac{4^{p-1}}{2^{2^{k+1}} \left[ (p-1)^{2^k} \right]^{3/2}} \\
	&= (k+2)! \cdot 3^{k+1}
	\frac{2^{n-2}}{2^{2^{k+1}} \left( \frac{n-2}{2} \right)^{3 \cdot 2^{k-1}}} \\
	&= (k+2)! \cdot 3^{k+1}
	\frac{2^{n-2-2^{k+1}+3\cdot2^{k-1}}}{(n-2)^{3 \cdot 2^{k-1}}} \\
	&= \frac{(k+2)! \cdot 3^{k+1} \cdot
		2^{n-2^{k-1}}}{4(n-2)^{3 \cdot 2^{k-1}}}.
\end{align*}
This concludes the proof.
\end{proof}
