
\chapter{Conclusion and future work}
\label{chap:conclusions}

In this thesis we have developed and studied the foundations for combining dependent types and computational effects, two important areas of modern programming language research. In the Introduction, we set out to establish the following claim:
\begin{displayquote}
\vspace{0.15cm}
\textit{Dependent types and computational effects admit a natural combination.}
\end{displayquote}
In retrospect, we can confirm that this is indeed the case. Specifically, we have provided language-based, category-theoretic, and algebraic evidence to support this claim.

\paragraph*{Language-based evidence.}

We have demonstrated that dependent types and computational effects can be naturally combined in a single programming language.
We achieved this by developing a core \emph{effectful dependently typed language}, called eMLTT, that extends intensional MLTT with general computational effects, based on a clear separation between values and computations.
Using eMLTT, we demonstrated that---with minor changes to the typing rules of effectful computations---one can readily use familiar combinators from simply typed languages to program with computational effects in the dependently typed setting, e.g., using sequential composition.
To overcome the limitations caused by these changes to the typing rules of effectful computations,
we introduced eMLTT's distinguishing feature, the \emph{computational $\Sigma$-type}, 
which allows us to uniformly ``close-off" free variables in computation types. 

\paragraph*{Category-theoretic evidence.}

We have also demonstrated that dependent types and computational effects can be naturally combined category-theoretically. To this end, we defined and studied a class of category-theoretic models, called \emph{fibred adjunction models}, suitable for defining a sound and complete interpretation of eMLTT. Specifically, fibred adjunction models naturally combine standard category-theoretic models of dependent types (split closed comprehension categories) and the corresponding generalisation of adjunction-based models of computational effects (split fibred adjunctions).
The naturality of this combination was demonstrated by being able to reuse and generalise various established results about monads and adjunctions, such as the existence of the Eilenberg-Moore resolution, and by showing that the computational $\Sigma$- and $\Pi$-types can be modelled analogously to their value counterparts, namely, as adjoints to weakening functors.
We further presented various examples of fibred adjunction models, ranging from i) those built from models of EEC, to ii) those based on the families of sets fibration, to iii) those built around the fibred Eilenberg-Moore resolutions of split fibred monads, to iv) those based on the fibration of continuous families of $\omega$-complete partial orders. The latter enabled us to extend eMLTT with recursion. 


\paragraph*{Algebraic evidence.}

We also investigated the algebraic treatment of computational effects in the presence of dependent types. Specifically, we showed how to extend eMLTT with \emph{fibred algebraic effects} and their \emph{handlers}. To specify such effects, we introduced a dependently typed generalisation of Plotkin and Pretnar's effect theories, whose dependently typed operation symbols enable us to capture precise notions of computation such as state with location-dependent store types and dependently typed update monads. For handlers, we observed that their conventional term-level definition leads to unsound program equivalences becoming derivable in languages that include a notion of homomorphism, such as eMLTT. To solve this problem, we provided a novel type-based treatment of handlers via a new computation type, the \emph{user-defined algebra type}, which pairs a value type (the carrier) with a family of value terms (the operations). This type internalises Plotkin and Pretnar's insight that handlers denote algebras for a given equational theory of computational effects. We demonstrated the generality of this type-based treatment by showing that the conventional presentation of handlers can be routinely derived from it, and that this treatment provides a useful mechanism for reasoning about effectful computations. We also showed that eMLTT with fibred algebraic effects and their handlers can be soundly interpreted in a fibred adjunction model based on the families of sets fibration and models of countable Lawvere theories.


\vspace{0.3cm}

\noindent
In conclusion, the contributions of this thesis can be summed up as follows:
\begin{itemize}
\item one can readily take well-known and established methods for, and results about, programming with computational effects in simply typed languages and successfully adapt them to the dependently typed setting; and 
\item the presence of dependent types, in combination with basing our work on adjunctions rather than monads, provides an opportunity to discover new and interesting language features, and corresponding mathematical structures.
\end{itemize}



\section{Future work directions}
\label{sect:futurework}

There are many directions in which one can take this work forward. 
We discuss some of them in detail, including work on the foundations, improvements to the expressive power of  
eMLTT and its extensions, and developing a (prototype) implementation. 

\subsection{Fibred notions of Lawvere theory}
\label{sect:fiblawveretheories}

In future, we plan to study the denotational semantics of eMLTT$_{\mathcal{T}_{\text{eff}}}$ and eMLTT$_{\mathcal{T}_{\text{eff}}}^{\mathcal{H}}$ at the same level of generality as we did for eMLTT in Chapters~\ref{chap:fibadjmodels} and~\ref{chap:interpretation}. In particular, we plan to extend the denotational semantics of eMLTT$_{\mathcal{T}_{\text{eff}}}$ and eMLTT$_{\mathcal{T}_{\text{eff}}}^{\mathcal{H}}$ from the families of sets fibration to more general fibational models of dependent types. Towards this end, we plan to develop a fibred notion of (countable) Lawvere theory, together with a framework for defining corresponding equational presentations. In particular, we conjecture that our fibred effect theories can be used as a basis for such presentations, by extending them to proper equational theories, i.e., closing the set of equations under reflexivity, symmetry, transitivity, substitution and replacement, and developing the corresponding proof theory. We then plan to study (fibred) local presentability conditions on split closed comprehension categories under which a split fibred free model adjunction exists. This adjunction can then be used as a basis for constructing a fibred adjunction model suitable for defining the interpretations of eMLTT$_{\mathcal{T}_{\text{eff}}}$ and eMLTT$_{\mathcal{T}_{\text{eff}}}^{\mathcal{H}}$.

A related future work direction involves extending eMLTT with local effects, e.g., local names and local state. One possible way forward to account for such computational effects would be to first develop a fibred notion of indexed Lawvere theory~\cite{Power:IndexedLawvereTheories} by working with suitable fibrations of presheaves indexed by names, locations, etc., and then extend eMLTT accordingly. Another way forward could involve developing a fibred version of Staton's parameterised algebraic theories and  their model theory~\cite{Staton:Instances}.

Finally, we also plan to give a general treatment of inequationally presentable computational effects, such as divergence, so as to provide a more general treatment of recursion than our use of the fibration $\mathsf{cfam}_{\CPO} : \CFam(\CPO) \longrightarrow \CPO$ of continuous families of $\omega$-complete partial orders in Section~\ref{sect:continuousfamilies}. Towards this end, we plan to develop a fibred notion of discrete countable enriched Lawvere theory~\cite{Hyland:DiscreteLawTh}, together with a framework for defining inequational presentations corresponding to enrichment in $\omega$-complete partial orders. An important question here involves the exact notion of enrichment one would use for defining such fibred enriched Lawvere theories. As discussed in Section~\ref{sect:shallowenrichment}, there are multiple candidates one could consider using, including those developed in~\cite{Shulman:EnrichedIndexedCategories} and~\cite[Section~8.1]{Vasilakopoulou:Thesis}, and the notion of pre-enrichment we use to model eMLTT's homomorphic function type in Section~\ref{sect:shallowenrichment}.

\subsection{Extending eMLTT with more expressive computation types}
\label{sect:fibredparametrisedeffects}

As it stands, the computation types of eMLTT and its extensions cannot be used to encode detailed specifications about effectful computations except for very basic descriptions of their general shape, e.g., whether a computation is an effectful function.  
To overcome this limitation, we plan to extend eMLTT with dependently typed variants of type-and-effect systems based on, e.g., Katsumata et al.'s graded monads and graded adjunctions~\cite{Katsumata:EffectMonads,Fujii:GradedMonads}, and Atkey's parameterised monads and parameterised adjunctions~\cite{Atkey:ParametrizedNotions,Atkey:Algebras}. 
Specifically, we plan to generalise from grading and parameterising adjunctions by categories to grading and parameterising fibred adjunctions by suitable fibrations, e.g., by a fibred monoidal fibration in the case of graded adjunctions. From a programming language perspective, this means that the gradings and parameters would become first-class citizens, given by value terms of some specified pure value type of ``worlds" of computation, e.g., describing whether a file is open or closed. 
%

In the case of a type-and-effect system based on parameterised adjunctions, we plan to generalise from working with $\mathcal{W}$-parameterised adjunctions, given by functors
\[
F : \mathcal{W} \times \mathcal{V} \longrightarrow \mathcal{C}
\qquad
U : \mathcal{W}^{\text{op}} \times \mathcal{C} \longrightarrow \mathcal{V}
\]
to working with \emph{split $r$-parameterised fibred adjunctions}, for some given split fibration $r : \mathcal{W} \longrightarrow \mathcal{B}$. We define these to be given by a pair of fibred functors
\[
\xymatrix@C=1.5em@R=2.5em@M=0.5em{
\bigintsss (X \mapsto \mathcal{W}_X \times \mathcal{V}_X) \ar[rrr]^-{F} \ar[dr] &&& \mathcal{C} \ar[dll]^-{q}
\\
& \mathcal{B} &
}
\]
\[
\xymatrix@C=1.5em@R=2.5em@M=0.5em{
\bigintsss (X \mapsto \mathcal{W}^{\text{op}}_X \times \mathcal{C}_X) \ar[rrr]^-{U} \ar[dr] &&& \mathcal{V} \ar[dll]^-{p}
\\
& \mathcal{B} &
}
\vspace{0.2cm}
\]
where $p$ and $q$ are split fibrations used to model value and computation types, respectively, and the domains of these functors are derived from $p$, $q$, and $r$ using the Grothendieck construction. Specifically, based on the discussion in Section~\ref{sect:shallowenrichment}, the domains of these two functors are the product split fibrations $r \times p$ and $r^{\text{op}} \times q$, respectively, thus demonstrating that we indeed have defined a natural fibred generalisation of Atkey's parameterised adjunctions. Of course, we would also generalise the parameterised unit $\eta$ and counit $\varepsilon$ transformations to the fibrational setting.
In addition, there exists an analogous definition of a split $r$-parameterised fibred monad, naturally generalising the notion of a $\mathcal{W}$-parameterised monad $T : \mathcal{W}^{\text{op}} \times \mathcal{W} \times \mathcal{V} \longrightarrow \mathcal{V}$.

 
Regarding the corresponding extension of eMLTT, such split $r$-parameterised fibred adjoints would give rise to corresponding eMLTT types, namely, $F_W\, A$ and $U_W\, \ul{C}$, where $W$ is a value term of some specified closed pure value type $\mathsf{World}$. For example, to model file-based I/O, $\mathsf{World}$ could be an inductive type with two constructors, called $\mathtt{open}$ and $\mathtt{closed}$. Intuitively, $F_W\,  A$ would be the type of computations that return values of type $A$ and finish evaluating in the world denoted by $W$; and $U_W\, \ul{C}$ would be the type of thunks that can only be forced in a world denoted by $W$. In addition, we plan to develop a fibred generalisation of Atkey's $\mathcal{W}$-parameterised algebraic theories, and investigate the corresponding extensions of eMLTT$_{\mathcal{T}_{\text{eff}}}$ and eMLTT$_{\mathcal{T}_{\text{eff}}}^{\mathcal{H}}$.


As discussed in Section~\ref{sect:relatedwork}, Brady has previously used the corresponding split fibred  parameterised monads $T_{W_1,W_2}\, A$ to extend Idris with computational effects.
As also mentioned in op.~cit., Brady has more recently proposed extending split fibred parameterised monads with additional type-dependency, so as to enable the postcondition world $W_2$ to depend on the return values of the given computation. In more detail, this extension can be illustrated with the following type formation rule:
\vspace{0.25cm}
\[
\mkrule
{\lj \Gamma {T_{W_1,\, x.W_2}\, A}}
{
\vj \Gamma {W_1} {\mathsf{World}}
\quad
\lj \Gamma A
\quad
\vj {\Gamma, x \!:\! A} {W_2} {\mathsf{World}}
}
\]
%
This additional type-dependency enabled Brady to accommodate generic effects whose postcondition world crucially depends on the outcome of the effect. 
A prototypical example of such an effect is the possibly erroneous file opening operation, typed as
\[
{\cj \Gamma {\mathtt{open\text{-}file}} {T_{\mathtt{closed},x.\,\mathtt{case~} x \mathtt{~of} \mathtt{~} ({\inl {\!} {\!\!(x_1 : 1)} \,\mapsto\, \mathtt{open}}, {\inr {\!} {\!\!(x_2 : 1)} \,\mapsto\, \mathtt{closed}})}(1+1)}}
\]
%

However, as part of our preliminary work on extending eMLTT with fibred parameterised effects, we have noticed that there does not seem to be a category-theoretically natural notion of adjunction corresponding to $T_{W_1,\, x.W_2}\, A$. In particular, the beautiful symmetries involved in the definition of split $r$-parameterised fibred adjunctions are lost because the functor corresponding to $F_{x.\, W} A$ would be   ``dependently typed", while the functor corresponding to $U_W\, \ul{C}$ remains split fibred as before.
A similar loss of symmetry also affects the unit $\eta$ and the counit $\varepsilon$, where the components of the unit become ``dependently typed" morphisms\footnote{The components of the unit $\eta$ would correspond to terms $\cj {\Gamma,x\!:\!A} {\return x} {T_{W\![x/y],y.\, W}\, A}$, whose type (the codomain of the morphism) crucially depends on the variable $x$ (the domain of the morphism).}.
This has led us to conclude that $T_{W_1,\, x.W_2}\, A$ does not in fact denote some more dependently typed version of a split $r$-parameterised fibred monad. Instead, it can be shown that $T_{W_1,\, x.W_2}\, A$ corresponds to the composition of split $r$-parameterised fibred adjoints (as defined earlier in this section) with the dependent sum functor that models our computational $\Sigma$-type. In particular, in an extension of eMLTT based on a split $r$-parameterised adjunction, we can define  $T_{W_1,\, x.W_2}\, A$ as 
\[
T_{W_1,\, x.W_2}\, A \defeq U_{W_1} (\Sigma\, x \!:\! A .\, (F_{W_2} 1))
\]
and also derive the correspondingly typed combinators for returning values and sequential composition. This is further evidence that the clear distinction between values and computations, together with the computational $\Sigma$-type, have an important and fundamental role to play in combining dependent types and computational effects.

\subsection{Fibrational account of Dijkstra monads}
\label{sect:fibDijkstramonads}

In addition to type-and-effect systems based on graded and parameterised adjunctions, we plan to investigate extending eMLTT's type system with ideas based on how Dijkstra monads are used in F*. To this end, we first need to find an appropriate notion of adjunction corresponding to F*'s Dijkstra monads. As a starting point, we note that in the fibrational setting, Dijkstra monads can be understood as certain relative monads~\cite{Altenkirch:RelMon2}, with respect to the monad of weakest precondition predicate transformers.

Specifically, 
this relative monads based axiomatisation of a Dijkstra monad on a split comprehension category with unit $p : \mathcal{V} \longrightarrow \mathcal{B}$, indexed by a split fibred (weakest preconditions) Kleisli triple $(W\!P, \eta, (-)^\dagger)$ on $p$, involves giving the following data:
\begin{itemize}
\item a functor $T : \mathcal{V} \longrightarrow \mathcal{V}$ such that $T$ strictly preserves Cartesian morphisms and 
\[
\xymatrix@C=2em@R=2em@M=0.5em{
& \mathcal{V} \ar[dl]_-{W\!P} \ar[dr]^-{T}
\\
\mathcal{V} \ar[dr]_-{\ia -} & & \mathcal{V} \ar[dl]^-{p}
\\
& \mathcal{B}
}
\]
\item a unit $\eta^T_A : \ia A \longrightarrow \ia {T(A)}$ in $\mathcal{B}$, for every $A$ in $\mathcal{V}$, such that
\[
\xymatrix@C=7em@R=3em@M=0.5em{
\ia A \ar[d]_-{\id_{\ia A}} \ar[r]^-{\eta^T_A} & \ia {T(A)} \ar[d]^-{\pi_{T(A)}}
\\
\ia A \ar[r]_-{\ia {\eta_A}} & \ia {W\!P(A)}
}
\]
\item and for every commuting square of the form
\[
\xymatrix@C=7em@R=3em@M=0.5em{
\ia A \ar[d]_-{\id_{\ia A}} \ar[r]^-{f} & \ia {T(B)} \ar[d]^-{\pi_{T(B)}}
\\
\ia A \ar[r]_-{\ia {g}} & \ia {W\!P(B)}
}
\]
a Kleisli extension $f_T^\dagger : \ia {T(A)} \longrightarrow \ia {T(B)}$  in $\mathcal{B}$ such that
\[
\xymatrix@C=7em@R=3em@M=0.5em{
\ia {T(A)} \ar[d]_-{\pi_{T(A)}} \ar[r]^-{f_T^\dagger} & \ia {T(B)} \ar[d]^-{\pi_{T(B)}}
\\
\ia {W\!P(A)} \ar[r]_-{\ia {g^\dagger}} & \ia {W\!P(B)}
}
\]
\end{itemize}
such that the natural laws for the interaction of the unit and the Kleisli extension hold.

From a programming language perspective, the functors $W\!P$ and $T$, and their interaction in the  above diagram, can be described using two type formation rules:
\vspace{0.15cm}
\[
\mkrule
{\lj {\Gamma} {W\!P~A}}
{\lj \Gamma A}
\qquad
\mkrule
{\lj {\Gamma, x \!:\! W\!P~A} {T~A}}
{\lj \Gamma A \quad x \not\in V\!ars(\Gamma)}
\]
The unit $\eta^T_A$ and Kleisli extension $f_T^\dagger$ correspond to F*'s typing rules for returning values and sequential composition. For example, the unit $\eta^T_A : \ia {A} \longrightarrow \ia {T(A)}$ in $\mathcal{B}$ can be shown to correspond to a global element $1_{\ia {A}} \longrightarrow \ia {\eta_A}^*(T(A))$ in $\mathcal{V}_{\ia {A}}$, which in turn corresponds to (an idealised version of) F*'s typing rule for returning values:
\vspace{0.15cm}
\[
\mkrule
{\cj \Gamma {\return V} {T~A~[W\!P.\return V/x]}}
{\vj \Gamma V A}
\]

On closer inspection, the above data corresponds exactly to the definition of a relative monad from $\mathcal{V}$ to a certain subcategory of the arrow category $\mathcal{B}^\to$, relative to a functor that maps an object $A$ in $\mathcal{V}$ to the identity morphism $\id_{\ia A} : \ia A \longrightarrow \ia A$.

Based on these observations, we plan to investigate whether the corresponding relative adjunctions can be used to extend eMLTT with weakest precondition based reasoning about computational effects. In addition, we plan to explore an algebraic account of F*'s Dijkstra monads, so as to specify them using operations and equations.




\subsection{Allowing types to depend on effectful computations}
\label{sect:typedependencyonfeffects}

Recall the two key design choices we made when developing eMLTT. These were: \linebreak i) allowing types to depend only on values, and ii) fixing the typing rule for sequential composition by restricting the free variables in the type of the second computation. In future, we plan to extend eMLTT\footnote{In this section, we use eMLTT to jointly refer to eMLTT, eMLTT$_{\mathcal{T}_{\text{eff}}}$, and eMLTT$_{\mathcal{T}_{\text{eff}}}^{\mathcal{H}}$.} so as to lift both these restrictions. In particular, we plan to develop a version of eMLTT in which types could depend on effectful computations directly rather than via thunks. While type-dependency on computations is an intriguing question in itself, these future work plans are also motivated by the problems that arise in the recent work of V{\'{a}}k{\'{a}}r, as discussed in Section~\ref{sect:relatedwork}.

In particular, recall from op.~cit.~that V{\'{a}}k{\'{a}}r investigates a dependently typed version of CBPV built around a dependently typed version of sequential composition:
\[
\mkrule
{\cj {\Gamma_1, \Gamma_2[\thunk M/y]} {\doto {M} {x \!:\! A} {} {N}} {\ul{C}[\thunk M/y]}}
{
\begin{array}{c}
\cj {\Gamma_1} M {FA}
\quad
\lj {\Gamma_1, y \!:\! U\!FA, \Gamma_2} {\ul{C}}
\\[-0.5mm]
\cj {\Gamma_1, x \!:\! A, \Gamma_2[\thunk\! (\return x)/y]} {N} {\ul{C}[\thunk\! (\return x)/y]}
\end{array}
}
\]

While this typing rule solves the problem of simultaneously allowing the type of $N$ to depend on $x$ and restricting it from appearing free in the conclusion, and also enables V{\'{a}}k{\'{a}}r to define call-by-value and call-by-name translations from a dependently typed $\lambda$-calculus into his language, it introduces new problems in the presence of fibred algebraic effects, as discussed in Section~\ref{sect:relatedwork}.
%
We conjecture that the root cause of these problems is the thunks-based type-dependency on computations.

Consequently, we plan to investigate how to extend eMLTT with computation types that can depend  directly on effectful computations via computation variables. In order to avoid the problems arising from the thunks-based type-dependency in V{\'{a}}k{\'{a}}r's work, we anticipate that the computation variables must be treated in computation types similarly to the way they are currently used in homomorphism terms. Consequently, in addition to computation types $\ul{C}$, $\ul{D}$, $\ldots$ that are dependent on only values, we plan to include \emph{homomorphic} computation types $\ul{\ul{C}}$, $\ul{\ul{D}}$, $\ldots$ that further depend on computation variables, with well-formed such types defined using a judgement $\lj {\Gamma \vertbar z \!:\! \ul{C}} \ul{\ul{D}}$. We can then equip sequential composition with naturally dependent typing rules, given by
\vspace{0.15cm}
\[
\mkrule
{\cj \Gamma {\doto M {x \!:\! A} {} {N}} {\ul{\ul{C}}[M/z]}}
{\cj \Gamma M FA
\quad
\lj {\Gamma \vertbar z \!:\! FA} \ul{\ul{C}}
\quad
\cj {\Gamma, x \!:\! A} N \ul{\ul{C}}[\return x/z]
}
\]
\[
\mkrule
{\hj \Gamma {z_1 \!:\! \ul{C}} {\doto K {x \!:\! A} {} {N}} {\ul{\ul{D}}[K/z_2]}}
{\hj \Gamma {z_1 \!:\! \ul{C}} K FA
\quad
\lj {\Gamma \vertbar z_2 \!:\! FA} \ul{\ul{D}}
\quad
\cj {\Gamma, x \!:\! A} N \ul{\ul{D}}[\return x/z_2]
}
\]
We further speculate that other elimination rules for computation types, such as computational pattern matching, can be given analogous naturally dependent typing rules.

In order that computation variables can be used as they are used in homomorphism terms, we speculate 
that both kinds of computation types will need to include \emph{elimination forms} such as sequential composition, computational pattern matching, and the composition operations. In particular, we anticipate their grammar to be given by
\[
\begin{array}{r c l}
\ul{C} & ::= & \ldots \,\,\,\vertbar\,\,\, \doto M {x \!:\! A} {} {\ul{C}} \,\,\,\vertbar\,\,\, \doto M {(x \!:\! A, z \!:\! \ul{C})} {} \ul{\ul{D}} \,\,\,\vertbar\,\,\, \runas M {x \!:\! U\ul{C}} {} {\ul{D}}
\\[2mm]
\ul{\ul{C}} & ::= & \doto K {x \!:\! A} {} {\ul{C}\,} \,\,\,\vertbar\,\,\, \doto K {(x \!:\! A, z \!:\! \ul{C})} {} \ul{\ul{D}\,} \,\,\,\vertbar\,\,\, \runas K {x \!:\! U\ul{C}} {} {\ul{D}}
\end{array}
\vspace{0.1cm}
\]

While so far it might seem that this extension of eMLTT is going to be straightforward, we expect significant challenges regarding its equational theory and category-theoretic denotational semantics. 
Based on the author's joint work with Plotkin on refinement types for algebraic effects~\cite{Ahman:RefTypes}, we speculate that the natural choice for modelling such computation types is to use families of subsets (more generally, subobjects) of carriers of models of the given fibred effect theory. If we think of these carriers as sets of computation trees, a computation type $\doto {\mathtt{get}^{F\State}(y.\, \return y)} {x \!:\! \State} {} {\ul{C}}$ would denote a family of sets of computation trees each with the following shape:
\vspace{-0.1cm}
\[
\xymatrix@C=1.5em@R=2.5em@M=0.3em{
& & \mathsf{get} \ar@{-}[dll] \ar@{-}[d] \ar@{-}[drr]
\\
c_{s_1} & \ldots & c_{s_i} & \ldots & c_{s_n}
}
\]
where $c_{s_i}$ would be an element of the set of computation trees denoted by $\ul{C}[s_i/x]$. \linebreak
However, it is important to point out that one cannot naively lift all equations from the term level to the type level in this setting. In particular, for general $\lj \diamond \State$ and $\lj \Gamma \ul{C}$, this semantics of computation types would not validate the following definitional equation:
\[
\ljeq \Gamma {\ul{C}} {\doto {\mathtt{get}^{F\State}(y.\, \return y)} {x \!:\! \State} {} {\ul{C}}}
\]
Namely, compared to the corresponding definitional equation between computation terms, a general computation type $\ul{C}$ would denote a non-trivial family of sets of computation trees. As a result, the computation trees $c_{s_i}$ in the above diagram can all be different, meaning that the composite tree might not be in the family denoted by $\ul{C}$.

This is an instance of a general phenomenon that only linear equations can be lifted from a carrier of an algebra to the powerset of the carrier---see the work of Gautam~\cite{gautam:validity} for more details. 
However, it is worth noting that while the semantics in question would not validate the above equation, it would validate the following subtyping inequality:
\[
\Gamma \vdash {\ul{C}} \sqsubseteq {\doto {\mathtt{get}^{F\State}(y.\, \return y)} {x \!:\! \State} {} {\ul{C}}}
\]
This suggests that we probably have to extend eMLTT with a subtyping relation. For refinement types, a general schema for valid such subtyping rules can be found in~\cite{Ahman:RefTypes}.


\subsection{Normalisation and implementation}
\label{sect:normalisationandimplementation}

We also plan to develop a prototype implementation of eMLTT\footnote{In this section, we again use eMLTT to jointly refer to eMLTT, eMLTT$_{\mathcal{T}_{\text{eff}}}$, and eMLTT$_{\mathcal{T}_{\text{eff}}}^{\mathcal{H}}$.} and its extensions. 

As a first step towards implementing a prototype, we plan to develop a normalisation algorithm for the equational theory of eMLTT, using the well-known technique of normalisation-by-evaluation (NBE)~\cite{Dybjer:NBE}. More specifically, we plan to combine the existing work on NBE for dependent types~\cite{Altenkirch:NBEforTT} with the author's previous work on NBE for simply typed languages with algebraic effects~\cite{Ahman:NBE}. Regarding the normalisation algorithm, we could start by normalising the types and terms of eMLTT modulo the given fibred effect theory, and then specialise the normalisation algorithm to specific important computational effects, such as state, analogously to~\cite[Section~5.2]{Ahman:NBE}.

We expect that we would have to weaken the equational theory of eMLTT so as to ensure that its type- and term-equality are decidable. In particular, while part of the equational theory is already set up so as to avoid known sources of undecidability, e.g., we use intensional propositional equality and we omit the $\eta$-equation for primitive recursion\footnote{Already in the simply typed setting, the term-equality in G\"{o}del's System T becomes undecidable when one introduces  the $\eta$-equation (i.e., a uniqueness axiom) for primitive recursion~\cite{Okada:Rewriting}.}, other parts of the equational theory might pose further problems, e.g., the $\eta$-equation for the coproduct type~\cite{Balat:NBE}. Furthermore, the decidability of the equations corresponding to the composition operations is an altogether unknown territory. 

Regarding other sources of undecidability of typechecking, it is worthwhile to recall that checking the correctness of handlers of algebraic effects is an undecidable problem in general~\cite[\S6]{Plotkin:HandlingEffects}. Accordingly, the same would hold for verifying the well-formedness of the user-defined algebra type in eMLTT.
As discussed in Section~\ref{sect:derivingconventionalhandlers}, one way to tackle this problem would be  
to require  
programmers to manually prove equational proof obligations that cannot be established automatically. 
To enable this kind of interaction with programmers, we could replace definitional equations in proof obligations with propositional equalities, and annotate the user-defined algebra type and the composition operations with the corresponding proof terms. 

Note that by changing the proof obligations from definitional equations to propositional equalities, we raise interesting questions regarding the soundness of the denotational semantics of eMLTT. In particular, while definitional and propositional equality conveniently coincide in fibred adjunction models based on the families of sets fibration, the former is usually only included in the latter in more general models of (higher) dependent types, e.g., see~\cite{Bezem:Cubical}. 
Consequently, we have to be careful about which proof terms we allow to witness these proof obligations, so as to ensure that the interpretation of the user-defined algebra type and the composition operations remains sound.

Regarding the implementation, we also need to equip eMLTT 
with a suitable operational semantics. As a starting point, we plan to investigate an operational semantics  based on Lindley and Hillerstr\"{o}m's~\cite{Hillerstrom:Liberating} abstract machine based semantics for handlers of algebraic effects (in the simply typed setting). 
Regarding eMLTT, we expect that the most challenging problem will be accommodating the unfolding of algebraic operations at the user defined algebra type (see the equation in Definition~\ref{def:extensionofeMLTTwithhandlers}).





















