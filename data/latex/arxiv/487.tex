\section{Theoretical Results} \label{sec:theoretical_results}
From the two preceding sections, we can compute $F^{\textsf{ND}}$ with
the (adaptive) YDS algorithm (Theorem \ref{the:YDS}) and $F^{\textsf{D2D}}$ by
solving the large-scale LP problem \textsf{Min-Spectrum-D2D} (Sec. \ref{sec:problem_formulation_D2D}).
Hence, numerically we can get the  spectrum reduction and
the overhead ratio. In this section, however, we seek to
derive theoretical upper bounds on both spectrum reduction and overhead ratio.
Such theoretical upper bounds provide insights for the key factors to
achieve large spectrum reduction and thus provide guidance to determine whether
it is worthwhile to implement D2D load balancing scheme in real-world cellular systems.

%We summarize our results in Tab. \ref{tab:theo_results}.
%
%\begin{table}[h]
%   \caption{Theoretical Results \newline}
%      \label{tab:theo_results}
%   \centering
%\begin{tabular}{|c|c|c|c|}
%\hline
%\textbf{Topology} & \textbf{Traffic} & \textbf{$\rho$}& \textbf{$\eta$}\\
%\hline
%Arbitrary& Arbitrary & $\rho  \le \frac{\max\{r,1\} + \tilde{r} \Delta^--1}{\max\{r,1\} + \tilde{r} \Delta^-}$ &  N/A \\
%\hline
%%Torus & Specified  & $\rho = \frac{\Delta^-}{\Delta^- + \frac{D}{D-1}}$ & $\eta = $\\
%Ring& Specified & $\rho = \frac{2(D-1)}{3D-2}$ & $\eta = \frac{D(D-1)}{D^2+2D-2}$\\
%\hline
%Complete& Specified & $\rho = \frac{N-1}{N+1}$ & $\eta = \frac{N-1}{2N}$\\
%\hline
%\end{tabular}
%\end{table}

%
%\begin{figure}
%  \centering
%  % Requires \usepackage{graphicx}
%  \includegraphics[width=0.5\linewidth]{intra_cell_benefit}\\
%  \caption{A simple example to show the benefit of intra-cell D2D communications.}
%  \label{fig:intra_cell_benefit}
%\end{figure}


\subsection{A Simple Upper Bound for Spectrum Reduction}
We can get a simple upper bound for $F^{\textsf{D2D}}$ by assuming no cost for D2D communication in the sense
that any D2D communication will not consume bandwidth and will not incur delays. Then we can
construct a virtual grand BS and all users $\mathcal{U}$ are in this BS. Then the system becomes similar
to the case without D2D. We can apply the YDS algorithm
to compute the minimum peak traffic, which is a lower bound
for $F^{\textsf{D2D}}$, i.e.,
$
\underline{F}^{\textsf{D2D}}=  \max_{I \subset [T]} g(I),
$
where
\be
g(I) = \frac{\sum\limits_{j \in \mathcal{A}(I)} \frac{r_j}{R_{\max}}}{z'-z + 1}.
\label{equ:intensity_d2d}
\ee
Here in \eqref{equ:intensity_d2d},
$
\mathcal{A}(I) = \{j \in \mathcal{J}: [s_j, e_j] \subset [z,z']\}
$
is the set of all active traffic demands whose lifetime is within the interval $I=[z,z']$
and $R_{\max} = \max_{s \in \mathcal{U}} R_{s,b_s}$ is the best user-to-BS link. Then we have the following theorem.

%An trivial upper bound is $P_{D}^{ub} = \sum_{b \in \mathcal{B}} P^{ND}_b$, which means that we do not use D2D
%communications but just schedule all traffics to its own BS.

\begin{theorem}
%$P_{D}^{lb} \le P_{D}^* \le P_{D}^{ub}.$
$\rho \le \frac{F^{\textsf{ND}}-\underline{F}^{\textsf{D2D}}}{F^{\textsf{ND}}}$.
\label{the:trivial_upper_bound}
\end{theorem}
\begin{IEEEproof}
\ifx \ISTR \undefined
Please see our technical report \cite{TR}.
\else
Please see Appendix~\ref{app:proof_trivial_upper_bound}.
% in the supplementary materials.
\fi
\end{IEEEproof}

Note that both $\underline{F}^{\textsf{D2D}}$ and $F^{\textsf{ND}}$ can be computed by the YDS algorithm, much
easier than solving the large-scale LP \textsf{Min-Spectrum-D2D}. Therefore, numerically we can get
a quick understanding of the maximum benefit that can be achieved by D2D load balancing.

\subsection{A General Upper Bound for Spectrum Reduction} \label{sec:a_general_upper_bound}
We next describe another general upper
bound for any arbitrary topology and any arbitrary traffic demand set. We will begin with
some preliminary notations.

We first define some preliminary notations. Let $N=|\mathcal{B}|$ be the number of BSs and we define
a directed \emph{D2D communication graph} $\mathcal{G}^{\textsf{D2D}}= (\mathcal{B}, \mathcal{E}^{\textsf{D2D}})$
where the vertex set is the BS set $\mathcal{B}$ and $(b,b') \in \mathcal{E}^{\textsf{D2D}}$
if there  exists at least one inter-cell D2D link from user $u \in \mathcal{U}_b$ in BS $b \in \mathcal{B}$
to user $v \in \mathcal{U}_{b'}$ in BS $b' \in \mathcal{B}$.
%Also, we say that BS $j$ can do load balancing for BS $i$ if $(i,j) \in \mathcal{E}_{D2D}$.
Denote $\delta^-_b$ as the in-degree of BS $b$ in the graph $\mathcal{G}^{\textsf{D2D}}$
and define the maximum in-degree of the graph $\mathcal{G}^{\textsf{D2D}}$ as $\Delta^- = \max_{b \in \mathcal{B}} \delta^-_b$.
In addition, we define some notations in Tab. \ref{tab:Discrepancy Notations}
to capture the discrepancy of D2D links and non-D2D links for users and BSs.
%\footnote{
Note that these definitions will be used thoroughly
\ifx \ISTR \undefined
in our technical report \cite{TR} to prove {Theorem} \ref{the:ratio_bound}.
\else
in Appendix \ref{app:proof_upper_bound} to prove {Theorem} \ref{the:ratio_bound}.
%in the supplementary materials to prove {Theorem} \ref{the:ratio_bound}.
\fi


%
%\begin{table}[h]
%   \caption{Discrepancy Notations}
%      \label{tab:Discrepancy Notations}
%   \centering
%\begin{tabular}{|l|}
%\hline
%%Self-discrepancy of user $s$: &
%$r_s = \max_{v: (s,v) \in \mathcal{E}, v \in \mathcal{U}_{b_s}} \frac{R_{s,v}}{R_{s,b_s}},
%\quad \forall s \in \mathcal{U}$
%\\%
%%Cross-discrepancy ratio between user $s$ and BS $j$ &
%$\tilde{r}_s^b = \max_{v: (s,v) \in \mathcal{E}, v \in \mathcal{U}_{b}} \frac{R_{s,v}}{R_{s,b_s}},
%\quad \forall s \in \mathcal{U}, b \in \mathcal{B}$
%\\
%%Self-discrepancy ratio of BS $i$ &
%$r_b = \max_{s \in \mathcal{U}_b} r_s,
%\quad \forall b \in \mathcal{B}$
%\\
%%Cross-discrepancy ratio between BS $i$ and BS $j$ &
%$\tilde{r}_{b,b'} = \max_{s \in \mathcal{U}_b} \tilde{r}_s^{b'},
%\quad \forall b \in \mathcal{B}, b' \in \mathcal{B}$
%\\
%%Maximal self-discrepancy ratio of all BSs &
%$r = \max_{b \in \mathcal{B}}{r_b}$
%\\
%%Maximal cross-discrepancy ratio of all BS pairs &
%$\tilde{r} = \max_{(b,b') \in \mathcal{E}^{\textsf{D2D}}} \tilde{r}_{b,b'}$
%\\
%\hline
%\end{tabular}
%\end{table}


Now we have the following theorem.

\begin{theorem}
\label{the:ratio_bound}
For an arbitrary network topology $\mathcal{G}$ associated with a D2D communication
graph $\mathcal{G}^{\textsf{D2D}}= (\mathcal{B}, \mathcal{E}^{\textsf{D2D}})$ and an arbitrary
traffic demand set, the  spectrum reduction is upper bounded by
\be
\rho \le \frac{\max\{r,1\} + \tilde{r} \Delta^--1}{\max\{r,1\} + \tilde{r} \Delta^-}.
\ee
%where $\Delta^-$ is the maximal in-degree of the graph $\mathcal{G}_{D2D}$.
\end{theorem}
\begin{IEEEproof}
\ifx \ISTR \undefined
Please see Appendix \ref{app:proof_upper_bound}.
\else
Please see  our technical report \cite{TR}.
\fi
\end{IEEEproof}

Based on this upper bound, we observe that the benefit of D2D load
balancing comes from two parts: intra-cell D2D and inter-cell D2D.
More interestingly, we can obtain the individual benefit of intra-cell
D2D and inter-cell D2D separately, as shown in the following Corollaries \ref{cor:intra_cell_benefit}
and \ref{cor:inter_cell_benefit}. One can go through the proof for Theorem \ref{the:ratio_bound}
by disabling inter-cell or intra-cell D2D communication and get the proof of these
two corollaries.



\begin{figure*}
\begin{minipage}[c]{0.32\linewidth}
\captionof{table}{Discrepancy Notations.} \label{tab:Discrepancy Notations}
\begin{tabular}{|l|}
\hline
%Self-discrepancy of user $s$: &
\scriptsize{$r_s = \max_{v: (s,v) \in \mathcal{E}, v \in \mathcal{U}_{b_s}} \frac{R_{s,v}}{R_{s,b_s}},
 \forall s \in \mathcal{U}$}
\\%
%Cross-discrepancy ratio between user $s$ and BS $j$ &
\scriptsize{$\tilde{r}_s^b = \max_{v: (s,v) \in \mathcal{E}, v \in \mathcal{U}_{b}} \frac{R_{s,v}}{R_{s,b_s}},
 \forall s \in \mathcal{U}, b \in \mathcal{B}$}
\\
%Self-discrepancy ratio of BS $i$ &
\scriptsize{$r_b = \max_{s \in \mathcal{U}_b} r_s, \forall b \in \mathcal{B}$}
\\
%Cross-discrepancy ratio between BS $i$ and BS $j$ &
\scriptsize{$\tilde{r}_{b,b'} = \max_{s \in \mathcal{U}_b} \tilde{r}_s^{b'}, \forall b \in \mathcal{B}, b' \in \mathcal{B}$}
\\
%Maximal self-discrepancy ratio of all BSs &
\scriptsize{$r = \max_{b \in \mathcal{B}}{r_b}$,  $\tilde{r} = \max_{(b,b') \in \mathcal{E}^{\textsf{D2D}}} \tilde{r}_{b,b'}$}
\\
\hline
\end{tabular}
\end{minipage}
\hfill
\begin{minipage}[c]{0.3\linewidth}
%\centering
%\includegraphics[width=0.82\columnwidth]{BS_ring_topology}
%\caption{\label{fig:BS_ring_topology}  Ring topology with $N=5$.}
  \centering
  % Requires \usepackage{graphicx}
  \includegraphics[width=\linewidth]{intra_cell_benefit}\\
  \caption{The benefit of intra-cell D2D communications.}
  \label{fig:intra_cell_benefit}
\end{minipage}
\hfill
\begin{minipage}[c]{0.3\linewidth}
\centering
\includegraphics[width=\linewidth]{tradeoff_ring_complete}
\caption{\label{fig:tradeoff_ring_complete}  Tradeoff between $\rho$ and $\eta$.}
\end{minipage}
\end{figure*}


%\subsubsection{Benefit of Intra-cell D2D}
\begin{corollary} \label{cor:intra_cell_benefit}
If only intra-cell D2D communication is enabled, the  spectrum reduction is upper bounded by
\be
\rho \le \frac{\max\{r,1\} -1}{\max\{r,1\}}.
\ee
\end{corollary}

This upper bound is quite intuitive. When $r \le 1$, then for any user $s$, there does
not exist any intra-cell D2D link with better link quality than its direct link to BS $b_s$.
Therefore, using the user-to-BS link is always the optimal choice. Thus the spectrum
reduction is 0. When $r > 1$, larger $r$ means more advantages for intra-cell D2D links
over the user-to-BS links. Therefore, D2D can exploit more benefit.

Moreover, this upper bound can be achieved by the simple example in Fig. \ref{fig:intra_cell_benefit}.
Suppose that user $a$ generates one traffic demand with volume $V$ and delay $D \ge 2$ at slot 1.
Suppose link rates $R_1=1, R_2=r, R_3=(D-1)r$.
Then without intra-cell D2D, the (peak) spectrum requirement is $F_1=\frac{V}{D}$. With intra-cell
D2D, user $a$ transmits $\frac{V}{D-1}$ traffic to user b from slot
1 to slot $D-1$ and then user $b$ transmits all $V$ traffic to BS at slot $D$.
The (peak) spectrum requirement is $F_2 = \max\{\frac{V}{(D-1)R_2}, \frac{V}{R_3}\} = \frac{V}{(D-1)r}$.
Then the spectrum reduction is
\be
\frac{F_1-F_2}{F_1} = 1 - \frac{\frac{V}{(D-1)r}}{\frac{V}{D}} \to \frac{r-1}{r},
\text{as} \; D \to \infty.
\ee


The benefit of intra-cell D2D communication is widely studied (see \cite{Doppler09} \cite{Foder12}).
However, in this paper, we mainly focus on the benefit of inter-cell D2D load balancing. Indeed,
in our simulation settings in Sec. \ref{sec:simulation}, the intra-cell D2D brings negligible benefit.

%\subsubsection{Benefit of Inter-cell D2D}
\begin{corollary} \label{cor:inter_cell_benefit}
If only inter-cell D2D communication is enabled, the  spectrum reduction is upper bounded by
$
\rho \le \frac{\tilde{r} \Delta^-}{1 + \tilde{r} \Delta^-}.
$
\end{corollary}

The intuition behind the parameter $\tilde{r}$ is similar to the effect of
parameter $r$ in the intra-cell D2D case. In what follows, we will
only discuss the effect of parameter $\Delta^-$, which actually
reveals the insight of our advocated D2D load balancing scheme.
Now suppose that all the links have the
same quality and \emph{w.l.o.g.} let $R_{u,v}=1, \forall (u,v) \in \mathcal{E}$. Then
$r = \tilde{r} =1$, meaning that no intra-cell D2D benefit exists. And the benefit
of inter-cell D2D is reduced to the following upper bound
\be
\rho \le \frac{\Delta^-}{1+\Delta^-}.
\label{equ:bound_delta}
\ee

The rationale to understand this upper bound is as follows.
On a high level of understanding,
the main idea for load balancing is traffic aggregation.
If each BS can aggregate more traffic from other BSs, it can exploit more
statistical multiplexing gains to serve more traffic with the same amount of spectrum.
Since the in-degree for each BS indeed measures its capacity of traffic aggregation,
it is not surprising that the upper bound for $\rho$ is related to maximum in-degree $\Delta^-$.

To evaluate how good the upper bound in \eqref{equ:bound_delta} is, two natural questions can
be asked. The first is: \emph{Is this upper bound tight}?
Another observation is that if we want to achieve unbounded benefit,
i.e., $\rho \to 1$,
it is necessary to let $\frac{\Delta^-}{\Delta^-+1} \to 1$,
which means that $\Delta^- \to \infty$.
Then the second question is:
\emph{Can $\rho$ indeed approach 100\% as  $\Delta^- \to \infty$}?

In the rest of this subsection, we will answer these two questions
by  constructing a specified network and traffic demand set.
Specifically, we consider $N=|\mathcal{B}|$ BSs each serving one user only. To facilitate analysis, let
$b_i$ be the $i$-th BS and $u_i$ be the user in BS $i$, for
all $i\in[N]$. We consider a \emph{singleton-decoupled} traffic
demand set as follows. Each user has one and only one traffic
demand with the same volume $V$ and the same delay $D \ge 2$.
Let $T=ND$ and the traffic generation time of user $i$ be slot $D(i-1)+1$.
Therefore, the lifetime of user $u_i$'s traffic demand is $[D(i-1)+1,Di]$,
during which there are no other demands.

Under such settings, we will vary the user-connection pattern such that
the D2D communication graph is different.
Specifically, we will prove that this upper bound is asymptotically
tight in the ring topology for $\Delta^-=2$ in Fact \ref{fact:bs_ring},
and $\rho \to$ 100\% in the complete topology
as the number of BSs $N \to \infty$ in Fact \ref{fact:bs_complete}.
Moreover, we will also discuss the overhead ratio for these
two special topologies.

\begin{fact} \label{fact:bs_ring}
If $N=2D-1$ and the D2D communication graph
forms a bidirectional ring graph, then
there exists a traffic scheduling policy
such that the  spectrum reduction is
\be
\rho =\frac{2(D-1)}{3D-2} \to \frac{2}{3} = \frac{\Delta^-}{\Delta^-+1},
\text{as} \; {D \to \infty}.
\ee
Besides, the overhead ratio in this case is
\be
\eta = \frac{D(D-1)}{D^2+2D-2}.
\ee
\end{fact}
\begin{IEEEproof}
\ifx \ISTR \undefined
Please see  our technical report \cite{TR}.
\else
Please see Appendix \ref{app:proof_ring}.
% in the supplementary materials.
\fi
\end{IEEEproof}
\begin{fact} \label{fact:bs_complete}
If the D2D communication graph forms a bidirectional complete graph,
then there exists a traffic scheduling policy such that the spectrum reduction is
\be
\rho = \frac{N-1}{N+1} \to 100\%, \text{as} \; N \to \infty.
\ee
Besides, the overhead ratio in this case is
\be
\eta = \frac{N-1}{2N}.
\ee
\end{fact}
\begin{IEEEproof}
\ifx \ISTR \undefined
Please see  our technical report \cite{TR}.
\else
Please see Appendix \ref{app:proof_complete}.
% in the supplementary materials.
\fi
\end{IEEEproof}



\textbf{Remark:}
(i) Fact \ref{fact:bs_ring} shows the tightness of the upper bound in \eqref{equ:bound_delta}
for the ring-graph topology when $\Delta^-=2$.
(ii) { Fact \ref{fact:bs_complete} shows that $\rho$ can indeed approach $100\%$,
implying that in the best case, $\rho$ goes to $100\%$.
This  gives us strong motivation to investigate D2D load balancing scheme both theoretically and practically.}
(iii) For the complete-graph topology, the upper bound $\frac{\Delta^-}{\Delta^-+1}$ is not tight.
Indeed, since $\Delta^- =N-1$ in the complete-graph topology,
we have
\bee
\frac{\Delta^-}{\Delta^-+1} = \frac{N-1}{N} > \frac{N-1}{N+1}.
\eee
(iv) Let us revisit the toy example in Fig. \ref{fig:example}
which forms a complete-graph topology with $N=2$. It verifies the spectrum reduction
and overhead ratio in Fact \ref{fact:bs_complete},
i.e., $\rho = \frac{1}{3} = \frac{N-1}{N+1}$ and $\eta = \frac{1}{4} = \frac{N-1}{2N}$.
(v) We also highlight the tradeoff between the benefit $\rho$ and the cost $\eta$,
as illustrated in Fig. \ref{fig:tradeoff_ring_complete}.
Furthermore, Fig. \ref{fig:tradeoff_ring_complete} shows that
the complete-graph topology outperforms the ring-graph topology asymptotically
because $\rho \to \frac{2}{3}$ and $\eta \to 1 $ for the ring-graph topology but
$\rho \to 1 > \frac{2}{3}$ (larger benefit) and $\eta \to \frac{1}{2} < 1$ (smaller cost) for the complete-graph topology.


\subsection{An Upper Bound for Overhead Ratio}
Previously we study upper bounds for the spectrum reduction. Now
we instead propose an upper bound for overhead ratio. Recall that
$d_{\max}$ is the maximum demand delay. We then have the following result.

\begin{comment}
We further denote
\be
R^{\textsf{D2D}}_{\max} \triangleq \max_{(u,v) \in \mathcal{E}: (u,v) \text{ is a D2D link }} R_{u,v},
\ee
as the maximum D2D link rate.
\end{comment}

\begin{theorem} \label{the:overhead-upper-bound}
$\eta \le \frac{d_{\max}-1}{d_{\max}}$.
\end{theorem}
\begin{IEEEproof}
\ifx \ISTR \undefined
Please our technical report \cite{TR}.
\else
Please see Appendix~\ref{app:proof-of-overhead-upper-bound}.
% in the supplementary materials.
\fi
\end{IEEEproof}

The upper bound in Theorem~\ref{the:overhead-upper-bound} increases when the maximum
demand delay $d_{\max}$ increases. This is reasonable because a traffic demand can travel more
D2D links (and thus incurs more D2D traffic overhead) if its delay is large.
For our toy example in Fig.~\ref{fig:example}, we have $d_{\max}=2$ and thus
the upper bound for the overhead ratio is $\frac{d_{\max}-1}{d_{\max}}=50\%$, which
is in line with our actual overhead ratio 25\%.


\begin{comment}
\begin{theorem}
$\eta \le \frac{F^{\textsf{D2D}} R^{\textsf{D2D}}_{\max} (d_{\max}-1)}{F^{\textsf{D2D}} R^{\textsf{D2D}}_{\max} (d_{\max}-1) + V^{\textsf{BS}}}$.
\end{theorem}
\begin{IEEEproof}
According to \eqref{equ:D2D-traffic}, we have
\bee
V^{\textsf{D2D}} & = \sum_{t=1}^{T} \sum_{j \in \mathcal{J}: t \in [s_j, e_j-1]}
\sum_{u \in \mathcal{U}} \sum_{v:v \in \mathcal{U}, (u,v) \in \mathcal{E}} x^{j}_{u,v}(t) {R_{u,v}} \nnb \\
& = \sum_{j \in \mathcal{J}} \sum_{t=s_j}^{e_j-1} \sum_{u \in \mathcal{U}} \sum_{v:v \in \mathcal{U}, (u,v) \in \mathcal{E}} x^{j}_{u,v}(t) {R_{u,v}} \nnb \\
& \le \sum_{j \in \mathcal{J}} \sum_{t=s_j}^{e_j-1} R^{\textsf{D2D}}_{\max} \sum_{u \in \mathcal{U}} \sum_{v:v \in \mathcal{U}, (u,v) \in \mathcal{E}} x^{j}_{u,v}(t) \nnb \\
& = \sum_{j \in \mathcal{J}} \sum_{t=s_j}^{e_j-1} R^{\textsf{D2D}}_{\max} F^{\textsf{D2D}} \nnb \\
& = \sum_{j \in \mathcal{J}} R^{\textsf{D2D}}_{\max} F^{\textsf{D2D}}
\eee
\red{there are some problems about the proof...}
\end{IEEEproof}
\end{comment}
