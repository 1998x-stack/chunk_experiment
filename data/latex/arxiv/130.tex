\documentclass{article}
\pdfoutput=1

\usepackage{framed,multirow}
\usepackage{epsfig}

%% The amssymb package provides various useful mathematical symbols
\usepackage{amssymb,amsmath,amsthm}
\usepackage{latexsym}

% Following three lines are needed for this document.
% If you are not loading colors or url, then these are
% not required.
\usepackage{url}
\usepackage{xcolor}
\definecolor{newcolor}{rgb}{.8,.349,.1}

\usepackage{hyperref}

%\usepackage[switch,pagewise]{lineno} %Required by command \linenumbers below

\newcommand{\mathR}{\mathbb{R}}
\newcommand{\XI}{{\cal X}}
\newcommand{\bu}{{\bf u}}
\newcommand{\bp}{{\bf p}}
\newcommand{\bn}{{\bf n}}
\newcommand{\bX}{{\bf X}}
\newcommand{\bY}{{\bf Y}}
\newcommand{\Lag}{\mathrm Lag}
\newcommand{\entreguillemets}[1]{``{#1}''} 
%\newcommand{\eqref}[1]{\ref{#1}}

\newtheorem{observation}{Observation}
\newtheorem{theorem}{Theorem}
\newtheorem{definition}{Definition}

\def \Vor {\mbox{Vor}}
\def \Pow {\mbox{Pow}}
\def \Del {\mbox{Del}}
\def \Reg {\mbox{Reg}}

\begin{document}

%\verso{Preprint Submitted for review}

%\begin{frontmatter}

\title{Notions of optimal transport theory and how to implement them on a computer}
%\tnotetext[tnote1]{Only capitalize first word and proper nouns in the title.}

\author{Bruno L\'{e}vy and Erica Schwindt}
%\address{Inria centre Nancy Grand-Est and LORIA, France}

%\received{\today}

%%%% Do not use the below for submitted manuscripts
%\finalform{28 March 2017}
%\accepted{2 April 2017}
%\availableonline{15 May 2017}
%\communicated{S. Sarkar}

\maketitle

\begin{abstract}
  This article gives an introduction to optimal transport, a mathematical
theory that makes it possible to measure distances between
functions (or distances between more general objects), to
interpolate between objects or to enforce mass/volume conservation
in certain computational physics simulations. Optimal transport
is a rich scientific domain, with active research communities, both on its theoretical
aspects and on more applicative considerations, such as geometry
processing and machine learning. This article aims at explaining
the main principles behind the theory of optimal transport, introduce
the different involved notions, and more importantly, how they
relate, to let the reader grasp an intuition of the elegant
theory that structures them. Then we will consider a specific setting,
called semi-discrete, where a continuous function is transported to a discrete
sum of Dirac masses. Studying this specific setting naturally leads to an
efficient computational algorithm, that uses classical notions of computational
geometry, such as a generalization of Voronoi diagrams called Laguerre diagrams.
\end{abstract}

%\begin{keyword}
%% MSC codes here, in the form: \MSC code \sep code
%% or \MSC[2008] code \sep code (2000 is the default)
%\MSC 41A05\sep 41A10\sep 65D05\sep 65D17
%% Keywords
%\KWD Computers and Graphics\sep Formatting\sep Guidelines
%Optimal transport, shape interpolation, fluid dynamics.  
%\end{keyword}

%\end{frontmatter}

%\linenumbers



\input{body.inc}

\bibliographystyle{alpha}
\bibliography{ot}

\end{document}