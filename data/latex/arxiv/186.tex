

\documentclass[10pt,reqno]{amsart}
\usepackage{amsmath}
\usepackage{amsfonts}
\usepackage{mathrsfs}
\usepackage{amssymb}
\usepackage{amsthm}
\usepackage{enumitem}
\usepackage{bbm}
\usepackage{times}
\usepackage{tabularx}
\usepackage{amsbsy}
\usepackage{mathtools}
\usepackage{tikz}
\usetikzlibrary{arrows,decorations.markings}
\usetikzlibrary{matrix}
\usetikzlibrary{graphs}
\usetikzlibrary{backgrounds}
\usepackage{setspace}     \spacing{1}
\usepackage{color}
\usepackage[normalem]{ulem}
\usepackage{soul}
\usepackage{cancel}


%\usepackage{ulem}
\usepackage[colorlinks=true,linkcolor=blue,citecolor=blue]{hyperref}

%    \usepackage[abbrev,nobysame,alphabetic]{amsrefs}
%\renewcommand{\MR}[1]{}


\theoremstyle{Theorem}
\newtheorem{theorem}{Theorem} [section]

\newtheorem{proposition}[theorem]{Proposition}
\newtheorem{claim}[theorem]{Claim}
\newtheorem{lemma}[theorem]{Lemma}
\newtheorem{corollary}[theorem]{Corollary}
\newtheorem{custheorem}{Theorem}
\renewcommand{\thecustheorem}{\Alph{custheorem}}
%\newtheorem{custheorem}{Theorem}
%\renewcommand{\thecustheorem}{\Alph{custheorem}}
\theoremstyle{definition}
\newtheorem{definition}[theorem]{Definition}
\newtheorem{question}[theorem]{Question}
\newtheorem{example}[theorem]{Example}
\newtheorem{remark}[theorem]{Remark}
\newtheorem{conjecture}[theorem]{Conjecture}
\theoremstyle{remark}


\newlist{enumlemma}{enumerate}{3}
\setlist[enumlemma]{label*={(\alph*)}, ref= {(\alph*)} }


\usepackage{marginnote}
\newcommand{\note}[1]{\marginpar{{\color{red}\footnotesize \begin{spacing}{1}#1\end{spacing}}}}
\newcommand{\noted}[1]{\marginpar{{\color{blue}\footnotesize \begin{spacing}{1}#1\end{spacing}}}}





\newcommand{\restrict}[2]{{#1}{|_{{ #2}}}}
\newcommand{\diff}{\mathrm{Diff}}



\renewcommand{\epsilon}{\varepsilon}
\renewcommand{\emptyset}{\varnothing}

\def \ad{\mathrm{ad}}
\def \Lie{\mathrm{Lie}}
\def \diag{\mathrm{diag}}
\def \Ad{\mathrm{Ad}}
\def \Aut{\mathrm{Aut}}
\newcommand{\td}{\tilde}
\newcommand{\wtd}{\widetilde}
\def\Fib{\mathrm{Fiber}}
\DeclareMathOperator{\supp}{supp}
\DeclareMathOperator{\Diff}{Diff}
\DeclareMathOperator{\Isom}{Isom}
\DeclareMathOperator{\Stab}{Stab}
\DeclareMathOperator{\Id}{Id}

\newcommand{\sm}{\smallsetminus}
\newcommand{\R}{\mathbb {R}}
\newcommand{\Q}{\mathbb {Q}}
\newcommand{\Z}{\mathbb {Z}}
\newcommand{\N}{\mathbb {N}}
\newcommand{\T}{\mathbb {T}}
\newcommand{\C}{\mathbb {C}}


\newcommand{\e}{\epsilon}
\newcommand{\Xt}{X_{\mathrm{thick}}}



\def \sl{\mathfrak{sl}}
\newcommand{\Gl}{\mathrm{GL}}
\newcommand{\Sl}{\mathrm{SL}}
\newcommand{\PSl}{\mathrm{PSL}}
\newcommand{\Sp}{\mathrm{Sp}}
\newcommand{\So}{\mathrm{SO}}
\def\SO{\So}
\def\SL{\Sl}
\newcommand{\inv}{^{-1}}
\newcommand{\id}{\mathrm{Id}}
\def\frakI{\mathfrak I}
\def\calI{\mathcal I}
\def\calA{\mathcal A}
\def\calE{\mathcal E}
\def\calJ{\mathcal J}
\def\calC{\mathcal C}
\def\calL{\mathcal L}
\def\calF{\mathcal F}
\def\calG{\mathcal G}
\def\scrW{\mathcal W}
\def\loc{\mathrm{loc}}
\def \RP{\R P}
\def\partP{\mathcal P}
\def\partQ{\mathcal Q}
\def\orb{\mathcal O}
%\section{}
 \newcommand{\oldepsilon}{\mathchar"10F}
\newcommand{\eps}{\oldepsilon}


\def\ae{a.e.\ }
\def\as{a.s.\ }


 \newcommand{\lieg}{\mathfrak g}
\newcommand{\lieh}{\mathfrak h}
\newcommand{\liek}{\mathfrak k}
\newcommand{\liem}{\mathfrak m}
\newcommand{\lien}{\mathfrak n}
\newcommand{\liea}{\mathfrak a}
\newcommand{\lie}{\mathrm{Lie}}
\newcommand{\liep}{\mathfrak p}
\newcommand{\lieu}{\mathfrak u}
 \newcommand{\lieq}{\mathfrak q}
\def\Balg{\mathcal B}
\def\calB{\mathcal B}
\newcommand{\sign}{\mathrm {sgn}}

\newcommand{\Lip}{\, \mathrm{Lip}}
\newcommand{\lip}{\Lip}
\renewcommand\P{\mathbb{P}}

\def\calM{\mathcal M}
\def\calI{\mathcal I}

\def\wl{\mathrm{l}}
\def\Folner{F{\o}lner }
\def\vecm{\mathbf{m}}
\def\vect{\mathbf{t}}


\def\red{\color{red}}
\def\blue{}

\title{Zimmer's conjecture for actions of $\Sl(m,\Z)$}

\author[A.~Brown]{Aaron Brown}
\address{University of Chicago, Chicago, IL 60637, USA}
\email{awb@uchicago.edu}

\author[D.~Fisher]{David Fisher}
\address{Indiana University, Bloomington, Bloomington, IN 47401, USA}
\email{fisherdm@indiana.edu}

\author[S.~Hurtado]{Sebastian Hurtado}
\address{University of Chicago, Chicago, IL 60637, USA}
\email{shurtados@uchicago.edu}

\thanks{DF was partially supported by NSF Grants DMS-1308291 and DMS-1607041.  DF was also partially supported
by the University of Chicago, and by NSF grants DMS 1107452, 1107263, 1107367, ``RNMS: Geometric Structures and Representation Varieties" (the GEAR Network) during a visit to the Isaac Newton Institute in Cambridge.}


\long\def\symbolfootnote[#1]#2{\begingroup\def\thefootnote{\fnsymbol{footnote}}
\footnote[#1]{#2}\endgroup}


\begin{document}
%\symbolfootnote[0]{\it Preliminary version.  Last updated: \today}
\maketitle

\begin{abstract}
We prove Zimmer's conjecture for $C^2$ actions by finite-index subgroups of $\Sl(m,\Z)$ provided $m>3$. The method utilizes many ingredients from our earlier proof of the conjecture for actions by cocompact lattices in $\Sl(m,\R)$  \cite{BFH} but new ideas are needed to overcome the lack of compactness
of the  space $(G \times M)/\Gamma$ (admitting the induced $G$-action).  Non-compactness allows both measures and Lyapunov exponents to escape to infinity under averaging and
a  number of algebraic, geometric, and dynamical tools are used control this escape.
 New ideas are provided by the
work of Lubotzky, Mozes, and Raghunathan on the structure of nonuniform lattices and, in particular, of $\Sl(m,\Z)$ providing a   geometric decomposition of the cusp into rank one directions, whose  geometry is more easily controlled. The proof also makes use of a precise quantitative form of non-divergence of unipotent orbits by Kleinbock and Margulis, and an extension by de la Salle of strong property (T) to representations of nonuniform lattices.
\end{abstract}


%\setcounter{tocdepth}{1}


%\renewcommand\listfigurename{}
%\listoffigures



%\vspace{-1cm}

%\tableofcontents


%\section*{General things to fix}
%\begin{enumerate}
%%\item Clarify what the  metric in the fibers is
%\item norms are given by \verb! $\|$ !    not \verb! $||$ !
%%\item we never define $M^\alpha$.  Its a mess as to what we are working with
%%\item end sentences with a period even if they end with mathematics
%%\item the metric is right-invariant, not left-invariant.  Otherwise our result would have been known in the 1970s
%%\item   define a function by \verb!$f \colon \R \to \R$ !    not \verb!$f : \R \to \R $ !
%%\item too many phrases like ``subexponential fiber growth" that need to be precise and consistent.
%%\item lots of inappropriate apostrophes
%%\item  big-O notation seems nonsensical.  e.g.\ the zero function is $O(n^2)$ which seems useless for the lower bound estimates we claim.
%%\item \verb!\eqref! for equations, not \verb!\ref!
%\end{enumerate}
\section{Introduction}
\subsection{Statement of results}
The main result of this paper is the following:

\begin{custheorem}\label{main1} Let $\Gamma$ be a finite-index subgroup of $\Sl(m,\Z)$ and  let $M$ be a closed manifold of dimension $\dim(M) \leq m - 2$.  If  $\alpha\colon \Gamma \to \Diff^2(M)$ is a group homomorphism  then $\alpha(\Gamma)$ is finite\footnote{After this work was completed, Brown-Damjanovic-Zhang showed that some modifications of our arguments also give a proof for $C^1$ diffeomorphisms \cite{BDZ}.}.  In addition, if $\omega$ is a volume form on $M$, $m>2$ and if $\dim(M) \leq m-1$, then if and $\alpha\colon \Gamma \to \Diff^2(M, \omega)$ is a group homomorphism then $\alpha(\Gamma)$ is finite.
\end{custheorem}


For $m\ge 3$, we remark that the conclusion of Theorem \ref{main1} is known for actions on the circle by results of Witte Morris  \cite{MR1198459} (see also \cite{MR1703323,MR1911660} for actions by more general lattices on the circle) and for volume-preserving actions on surfaces by results of Franks and Handel and of Polterovich  \cite{MR2219247, MR1946555}.   The proof in this paper  requires that $m\ge 4$ though we expect it can be modified to cover actions by $\Sl(3,\Z)$; since these results are not new, we only present the case for $m\ge 4$.  While this is a very special case of Zimmer's conjecture, it is a key example.  For instance, the version of Zimmer's conjecture restated by Margulis in his problem list \cite{MR1754775} is a special case of Theorem \ref{main1}.
% is not new,
%we do not include these arguments here.

%\note{reduce?}
Note that if $\Gamma$ is a finite-index subgroup of  $\Sl(m,\Z)$ acting on compact manifold $M$, we may induce an action of $\Sl(m,\Z)$  on a (possibly non-connected) compact manifold $\td M= (\Sl(m,\R)\times M)/\sim$ where $(\gamma,x)\sim (\gamma', x')$ if there is $\hat \gamma\in \Gamma$ with $\gamma' = \gamma \hat \gamma$ and $x' = \alpha(\hat \gamma\inv)(x)$.
%consider $\Gamma$ acting on $\Sl(m,\Z)\times M$ by
%$$(\gamma', x) \cdot \gamma =  (\gamma' \gamma, \alpha(\gamma\inv)(x))$$
%and  $\Sl(m,\Z)$ actting on $\Sl(m,\Z)\times M$ by
%$$\gamma' \cdot (\gamma, x)  =  (\gamma'\gamma x).$$
%Then  $\Sl(m,\Z)$ acts on the compact, possibly non-connected, manifold $$N = (\Sl(m,\Z)\times M)/\Gamma.$$
Connectedness of $M$ is neither assumed nor is it  used in either the proof of Theorem \ref{main1} or in \cite{BFH}.  Thus, for the remainder we will simply  assume $\Gamma = \Sl(m,\Z)$.


This paper is a first step in extending the results in \cite{BFH} to the case where $\Gamma$ is a nonuniform lattice in a split simple Lie group $G$ and the strategy of the proof of Theorem \ref{main1} relies strongly on the strategy used in \cite{BFH}.  In the remainder of the introduction, we recall the proof in the cocompact case, indicate where the difficulties arise in the nonuniform case, and outline the proof of Theorem \ref{main1}.  At the end of the introduction we make some remarks on other approaches and difficulties we encountered.  % and also fix some notations and conventions for the rest of the paper.


 We recall a   key definition from \cite{BFH}.
Let $\Gamma$ be a finitely generated group.  Let $\ell\colon \Gamma \to \N$ denote the word-length function with respect to some choice of finite generating set for $\Gamma$.  Given a $C^1$ diffeomorphism $f\colon M\to M$ let $\|Df \| = \sup_{x \in M} \|D_xf \|$ (for some choice of norm on $TM$).
\begin{definition}
An action $\alpha\colon \Gamma\to \diff^1(M)$   has \emph{uniform subexponential growth of derivatives} if
\begin{equation}\label{eq:USEGOD}\text{   for every $\e>0$, there is $C_{\e}$ such  that  $\|D\alpha(\gamma)\| \leq C_{\e}e^{\e \ell(\gamma)}$ for all $\gamma\in \Gamma.$}\end{equation}
\end{definition}

The  main result of the paper is the following: % \note{where is m>3 actually used?}
 \begin{custheorem}\label{main2} For {$m\ge 4$}, let $\Gamma= \Sl(m,\Z)$ and let $M$ be a closed manifold.
 \begin{enumerate}
 \item If  $\dim(M) \leq m - 2$ then any action  $\alpha\colon  \Gamma \to \Diff^2(M)$ has {uniform subexponential growth of derivatives};
 \item  if $\omega$ is a volume form on $M$ and $\dim(M) \leq m-1$ then any action  $\alpha\colon \Gamma \to \Diff^2(M, \omega)$ has {uniform subexponential growth of derivatives}.
 \end{enumerate}
\end{custheorem}





%Given subexponential growth of derivatives of $\Gamma$, w
%\note{AWB moved this to here.}
To deduce Theorem \ref{main1} from  Theorem \ref{main2}, we apply \cite[Theorem 2.9]{BFH} and de la Salle's recent result establishing  strong property $(T)$ for nonuniform lattices  \cite[Theorem 1.2]{delaSallenonuniform} and  conclude that any action $\alpha$ as  in Theorem \ref{main1}  preserves a continuous Riemannian metric.  For clarity, we point out that we need de la Salle's Theorem $1.2$ and not his Theorem $1.1$ because we need the measures converging to the projection to be positive measures.  That Theorem \cite[Theorem 1.2]{delaSallenonuniform} provides positive measures where \cite[Theorem 1.1]{delaSallenonuniform} does not is further clarified in \cite[Section 2.3]{delaSallenonuniform}.   Once a continuous invariant metric is preserved, the   image of any homomorphism $\alpha$  in Theorem \ref{main1} is contained in  a compact Lie group $K$.  All such homomorphisms necessarily  have finite image due to the presence of unipotent elements in $\Sl(m,\Z)$.  We remark that  while the finiteness of the image of $\alpha$  was deduced using Margulis's superrigidity theorem in  \cite{BFH},  it is unnecessary in the setting  of Theorem \ref{main1} since, as  any unipotent element of $\Sl(m,\Z)$ lies in the center of some integral Heisenberg subgroup of $\Sl(m,\Z)$, all unipotent elements have finite image in $K$ and therefore so does $\Sl(m,\Z)$. %{\red I found the last paragraph about margulis unnecessary for the introduction. SHS.} {\red where would you want to put it?.. i don't see any obvious other place.}


\subsection{Review of the cocompact case} To explain the proof of Theorem \ref{main2}, we briefly  explain the   difficulties in extending the arguments from \cite{BFH} to the setting of actions by nonuniform lattices.  We begin by recalling the proof in the cocompact setting.

%The proof of \cite{BFH} starts by showing the action of $\Gamma$ on $M$ lacks of any hyperbolic behaviour in a uniform sense. More precisely, we show the action has subexponential growth of derivatives with respect to word metric on $\Gamma$, see \cite{BFH}. Once this is achieved, a strong fixed point property of the group $\Gamma$, Lafforgue's Strong property $(T)$, is used to construct an invariant Riemannian metric, after which Margulis superrigidity theorem is used to show the action factors through a finite quotient of $\Gamma$.

In both  \cite{BFH} and the proof of Theorem \ref{main2}, we consider a fiber bundle $$M \rightarrow M^{\alpha}:=(G \times M)/\Gamma \xrightarrow{\pi} G/\Gamma$$ which allows us to replace  the $\Gamma$-action on $M$ with a $G$-action on $M^\alpha$.  In the case that $\Gamma$ is cocompact,  showing subexponential growth of derivatives of the $\Gamma$-action is equivalent to showing subexponential growth of the fiberwise derivative cocycle for the $G$-action.



%When $\Gamma$ is a cocompact lattice of a Lie group $G$ acting on a manifold $M$, by
%%In the nonuniform setting, this equivalence fails due to the fact that far in the cusp $G/\Gamma$ moving
%
%%While we will consider the suspension action on $M^{\alpha}$ in the nonuniform case, this equivalence fails.  Subexponential growth of the fiberwise derivative for the $G$ action is now stronger and implies that the derivatives of the $\Gamma$ action are uniformly bounded.}

To prove such subexponential growth for the $G$-action on $M^{\alpha}$ we argued by contradiction to obtain a sequence of points $x_n \in M^{\alpha}$ and semisimple elements $a_n$ in a Cartan subgroup $A \subset G$  which satisfy $\|\restrict{D_{x_n}a_n}{F}\|  \geq e^{\lambda d(a_n, \Id)}$ for some $\lambda>0$. Here $D_{x} g$ denotes the derivative of translation by $g$ at $x\in M^\alpha$, $F$ is the fiberwise tangent bundle of $M^\alpha$, and $\restrict{D_{x_n}a_n}{F}$ is the restriction of $ D_{x_n}a_n $ to ${F(x_n)}$.
 % and $\| \cdot \|_\text{Fiber}$ denotes the supremum of the norm of $D_x g$ over the vectors in the fiber direction. %, $\Id$ denotes the identity element in $G$ and $d$ is any right-invariant metric on $G$.

%On the product $G\times M$ consider the right $\Gamma$-action
%$$ (g,x)\cdot \gamma= (g\gamma , \alpha(\gamma\inv)(x))$$ and the left $G$-action $$a\cdot (g,x) = (ag, x).$$
%Define  the quotient manifold $M^\alpha:= G\times M/\Gamma $.  As the  $G$-action on $G\times M$ commutes with the $\Gamma$-action, we have an induced left $G$-action  on $M^\alpha$.

The pairs $(x_n, a_n)$ determine empirical measures $\mu_n$ on $M^{\alpha}$ supported on the orbit $\{a_n^s(x_n): 0\le s\le t_n\}$
which accumulate on a  % by average over an $a$-orbit for some $a \in A$) which by a limiting argument gives a
measure $\mu$ that is $a$-invariant for some $a \in A$ and has a positive Lyapunov exponent for the fiberwise derivative cocycle of size at least  $\lambda$.
Using classical results in homogeneous dynamics in conjunction with the key proposition from \cite{AWBFRHZW-latticemeasure}, we averaged the measure $\mu$ to obtain a $G$-invariant measure $\mu'$ on $M^\alpha$ with a non-zero fiberwise Lyapunov exponent;  the existence of such a measure $\mu'$ contradicts Zimmer's cocycle superrigidity theorem.  %n low enough dimensions, Zimmer's theorem produces a measurable invariant metric, which immediately implies all Lyapunov exponents are zero.

\subsection{Difficulties in the nonuniform setting.} When $\Gamma$ is nonuniform
%The main difficulty to extend the previous arguments to the nonuniform setting is that
the space $M^{\alpha}$ is not compact and the sequence of empirical measures $\mu_n$ might  diverge to infinity in $M^{\alpha}$; that is, in the limit we might have a ``loss of mass".
Additionally, even if the measures $\{\mu_n\}$ satisfy some tightness criteria so as to prevent escape of mass, one might have  ``escape of Lyapunov exponents:" for a limiting measure $\mu$, the Lyapunov exponents may be infinite or the value could drop below the value expected by the growth of fiberwise cocycles along the orbits $\{a^s(x_n): 0\le s\le t_n\}$.
For instance,  the contribution to the exponential growth of derivatives along the sequence of empirical measures could arise primarily from excursions of orbits deep into the cusp.  If one makes na\"ive computations with the \emph{return cocycle} $\beta\colon G \times G/\Gamma \rightarrow \Gamma$ (measuring for $x$ in a fundamental domain $D$ the element of $\Gamma$ needed to bring $gx$ back to a $D$)  one in fact expects that the fiberwise derivative are very large for translations of points far out in the cusp since the orbits of such points cross a large number of fundamental domains. %  the values  of $\beta$ that arise there are very large for small values of $g$.
The weakest consequence of this observation is that subexponential growth of the fiberwise derivative of the induced $G$-action is much stronger than subexponential growth of derivatives of the $\Gamma$-action.  While we   still     work with the induced $G$-action and the fiberwise derivative in many places, the arguments become more complicated than in the cocompact case.

%There are many cases i
In the  homogeneous dynamics literature, there  are many tools to study  escape of mass.  Controlling the escape of Lyapunov exponents seems to be more novel.  To rule out escape of mass, it suffices to prove  tightness of family of  measures $\{\mu_n\}.$
% Given a non-compact topological space $X$ a sequence of measures $\mu_n$ on $X$ is said to be {\em tight} if for every $\varepsilon > 0$ there is an compact set $K$ such that $\mu_n(K) > 1- \varepsilon$.  See e.g. \cite{MR1652916, MR1437473, MR2680500} for some important arguments around escape of mass and tightness.
To control Lyapunov exponents, we introduce a quantitative tightness condition: we construct measures  $\{\mu_n\}$ with \emph{uniformly exponentially small mass in the cusps}. See Section \ref{sec:sampson}.
%To make this formal, assume $X$ is metric space, fix base point $x_0$ in $X$ and let $K_i = \{ x | d(x, x_0) <i\}$.  The measures $\mu_n$ have {\em uniformly exponentially small mass in the cusps} if there are $\eta >0$ such that $\mu_n(K_i) <  e^{-\eta i}$.
 It is a standard computation to show the Haar measure on $\Sl(m,\R)/\Sl(m,\Z)$  (or any  $G/\Gamma$ where $G$ is semisimple and $\Gamma$ is a lattice) has exponentially small mass in the cusps.  %for other measures on $G/\Gamma$  this definition
 % and also for Haar measure on $G/\Gamma$ when $G$ is a semisimple Lie group and $\Gamma$ is a lattice, but we have not found other places in the literature where this definition was required for non-homogeneous measures.

\subsection{Outline of proof} % for $\Sl(m,\Z)$.}
 With the above difficulties in mind, we outline   the strategy of the proof of Theorem \ref{main2}.  Lubotzky, Mozes and Raghunathan proved  that $\Sl(m, \Z)$ is quasi-isometrically embedded in $\Sl(m, \R)$.  And in this special case, they give a proof that every element  $\gamma\in \Sl(m, \Z)$ can be written   as a product of at most $m^2$ elements $\delta_i$ contained in  canonical copies of $\Sl(2, \Z)$ determined by pairs of standard basis vectors for $\R^m$; moreover the word-length of each $\delta_i$ is at most proportional to the word-length of $\gamma$ \cite[Corollary 3]{MR1244421}.  (We note however that such effective generation of $\Gamma$ only holds for $\Sl(m,\Z)$;  for  the general case, in  \cite{MR1828742} a weaker generation of $\Gamma$ in terms of $\Q$-rank 1 subgroups is shown.) Thus, to show uniform subexponential growth of derivatives for the action of $\Sl(m, \Z)$, it suffices to show uniform subexponential growth of derivatives for the restriction of our action to each canonical copy of $\Sl(2,\Z)$.


% each canonical $\Sl(2, \Z)$.  To achieve subexponential growth of derivatives  the restriction of our action to each copy of $\Sl(2,\Z)$. % for this subgroup, w

We first obtain uniform subexponential growth of derivatives for the unipotent elements in $\Sl(2,\Z)$  in Section \ref{unipotents}.  See Proposition \ref{unipotentisgood}.
The strategy is to consider a subgroup of the form $\Sl(2,\Z)\ltimes \Z^2 \subset \Sl(m,\Z)$. We first prove that a large proportion of    elements in $\Sl(2, \Z)$ satisfy \eqref{eq:USEGOD}. To prove this, we use that if $a^t:= \text{diag}(e^t, e^{-t}) $ %,1 , \dots, 1)$
 then a typical $a^t$-orbit in $\SL(2, \R)/\Sl(2, \Z)$ equidistributes to the Haar measure.  %with well-behaved cusps.
 In particular, for the empirical measures along such $a$-orbits we   apply the techniques from \cite{BFH} to show subexponential growth of fiberwise derivatives along such orbits and conclude that a large proportion of $\Sl(2, \Z)$ satisfies \eqref{eq:USEGOD}.   See Proposition \ref{mainunipo}. The proof of this fact repeats most of the ideas and techniques from \cite{BFH} as well a quantitative non-divergence of unipotent averages following  Kleinbock and Margulis. %on which are required to apply  averaging arguments like those in \cite{BFH} in the presence of cusps.
The precise averaging procedure is different here than in \cite{BFH}.
% because our initial measure is only invariant under the singular flow  $a^t$ and we cannot average over all of the Cartan subgroup without escape of mass until we have improved the measure by averaging over unipotents.   This part of the argument is in Subection \ref{subsection:slowgrowthgeneric} and uses preliminaries covered in Section \ref{section:preliminaries}.

%Once this is achieved,
Having shown Proposition \ref{mainunipo}, we consider the $\Sl(2,\Z)$-action on the normal  subgroup $\Z^2$ of $\Sl(2,\Z)\ltimes \Z^2$  to show that   for every $n \geq 0$, the ball $B_n$ of radius $n$ in $\Z^2$ contains a positive-density subset  of unipotent elements satisfying \eqref{eq:USEGOD}.  Taking iterated sumsets of such good unipotent elements of $B_n(\Z^2)$  with   a finite set  one obtains uniform subexponential growth of derivatives for every element in $B_n$. This relies heavily on the fact that $\Z^2$ is abelian. See Subsection \ref{sec:mutualmastication}.

%(*maybe the most fundamental idea of the paper).

It is worth noting that the subgroups of the form $\Sl(2,\Z)\ltimes \Z^2\subset \Gamma$ are also considered in the work of Lubotzky, Mozes, and Raghunathan in \cite{MR1244421} as well as in Margulis's early constructions of expander graphs and   subsequent work on property (T) and expanders \cite{MR0484767}. % and in the study of property (T) and relative property (T) going back to work of Margulis. (find citation, we can also %mention Burger and Shalom here if we want to.)

Having established Proposition \ref{unipotentisgood},  we assume for the sake of  contradiction that the restriction of $\alpha$ to $\Sl(2,\Z)$ fails to exhibit uniform subexponential growth of derivatives.  We  obtain in Subsection \ref{maximal} a sequence $\zeta_n$ of $a^t$-orbit segments  in $\Sl(2, \R)/\Sl(2, \Z)$ which drift  only a sub-linear distance into the cusp with respect to their length and accumulate exponential growth of the fiberwise derivative.
Here we use that orbits deep in the cusp of $\Sl(2,\R)/\Sl(2,\Z)$ correspond to unipotent deck transformations and  that Proposition \ref{unipotentisgood} implies that these do not contribute to the exponential growth of the fiberwise derivative.
%This is possible because the slow growth of the unipotents imply the growth of the derivative along excursions deep into the cusp in $\Sl(2,\R)/\Sl(2,\Z)$ is subexponential. This argument is in Section \ref{maximal} and is the pivotal point in the argument.
Here, we heavily use the structure of % It is at this point in the proof that it is important that we have reduced our question to one about $
$\Sl(2,\Z)$ subgroups. %To see that excursions into the cusp have slow growth of derivatives we need that the cocycle $\beta: \Sl(2,\R) \times \Sl(2,\R)/\Sl(2,\Z) \rightarrow \Sl(2,\Z)$ takes only unipotent values deep enough in the cusp.  This is only true for Lie groups of real rank one.

We   promote the family of  orbit segments $\zeta_n$  in $M^\alpha$ to a family of  measures $\{\mu_n\}$ all of whose  subsequential limits   are $A$-invariant measures $\mu$ on $M^{\alpha}$ with non-zero fiberwise exponents. To construct $\mu_n$,   we construct a  \Folner sequence $F_n \subset G$ inside a  solvable subgroup $AN'$ where $A$ is the full Cartan subgroup of $\Sl(m,\R)$ and $N'$ is a well-chosen abelian subgroup of unipotent elements. We average our orbit segments $\zeta_n$  over  $F_n$ to obtain  the sequence of measures $\mu_n$ in $M^{\alpha}$. In general, \Folner sets for $AN'$ are subsets which are linearly large in the $A$-direction and exponentially large in the $N'$ direction. In our case the $N'$-part will not affect the Lyapunov exponent because we work inside a subset where the return cocycle $\beta$ restricted to $N'$ takes unipotent values and we have already proven subexponential growth of the fiberwise derivatives for unipotent elements.


The fact that $\mu_n$ behaves well in the cusp is due to two facts: First, the segments obtained in Subsection \ref{maximal} do not drift  too deep into the cusp of $\Sl(2,\R)/\Sl(2,\Z)$.   Second, we choose our subgroup $N'$ such that the $N'$-orbits of each  point  along each $\zeta_n$ is a closed torus that is  well-behaved when translated by $A$.  The argument here is related to the fact  closed horocycles in the cusp of $\Sl(2,\R)/\Sl(2,\Z)$ equidistribute  to the Haar measure when flowed backwards by the geodesic flow.
%, the horocycle fills the surface and converges to Haar measure.  The main point in both cases is to show that the $\mu_n$ are tight, or in our setting, that they have uniformly exponentially small mass in the cusps.

% and
To finish the argument, we show that any $AN'$-invariant measure on $M^{\alpha}$ projects to Haar measure on $\Sl(m,\R)/\Sl(m,\Z)$ using  Ratner's measure classification and equidistribution theorems.  Then, as in \cite{BFH}, we can use \cite[Proposition 5.1]{AWBFRHZW-latticemeasure} and argue as in the cocompact case in \cite{BFH} show that  $\mu$ is in fact $G$-invariant and thereby obtain a contradiction with Zimmer's cocycle superrigidity theorem.

\subsection{A few remarks on other approaches.}
 \label{subsection:failure}

 We close the introduction by making some remarks on other approaches, particularly other approaches for controlling the  escape of mass.  We emphasize here that one key difficulty for all approaches is that we are not able to control the ``images" of the cocycle $\beta\colon G \times G/\Gamma \rightarrow \Gamma$ in either our special case or in general. To understand this remark better, consider first the case where $G=\Sl(2,\R)$ and $\Gamma =\Sl(2,\Z)$. If we take a one-parameter subgroup $c(t)<\Sl(2,\R)$ and take the trajectory $c(t)x$ for $t$ in some interval $[0,T]$ and assume and assume the entire trajectory on $G/\Gamma$ lies deep enough in the cusp, then $\beta(a(t),x)$ is necessarily unipotent for all $t$ in $[0,T]$.  No similar statement is true for $G=\Sl(m,\R)$ and $\Gamma=\Sl(m,\Z)$.  In fact analogous statements are true if and only if $\Gamma$ has $\Q$-rank one, this is closely related to the fact that higher $\Q$-rank locally symmetric spaces are $1$-connected at infinity.  This forces us to ``factor" the action into actions of rank-one subgroups in order to  control the growth of derivatives.

%One might hope to obtain subexponential growth of derivatives more directly for all elements of $\Sl(2,\Z)$, or even directly in $\Sl(m,\Z)$, by proving better estimates on the size of the ``generic" subsets of $\Sl(2,\R)$ and $Sl(2,\Z)$ (or $\Sl(m, \R)$ and $\Sl(m,\Z)$ that and then try to use an argument like the one we apply to $GU$ directly to the balls in $\Sl(2,\Z)$ or $\Sl(m,\Z)$.  While one can improve our estimates on the size of those sets using Margulis functions and large deviation estimates as in  \cite{MR2247652, MR2787598}, the resulting estimates are not sharp enough to allow us to prove subexponential growth of derivatives on balls in $\Sl(2,\Z)$.  Because those balls grow exponentially, one needs to prove that the ``bad set" grows at a slower exponential rate to obtain the desired result.  While this is true if one only cares about tightness, it is not true if one cares about exponentially small mass in the cusps.  The difficulty in applying an argument of this nature coming from the fact that even if for a fixed $a \in A$, most of the $a$-trajectories in $G/\Gamma$ converge to Haar measure, for a given $t>0$, the subset $S_t$ of points in $G/\Gamma$ whose trajectories defined by $a^t$  behave well in the cusp is not an exponentially small set in terms of $t$. One can compare with the conjectures in \cite{KKLM}.

One might hope to obtain subexponential growth of derivatives more directly for all elements of $\Sl(2,\Z)$, or even directly in $\Sl(m,\Z)$, by proving better estimates on the size of the ``generic" subsets of $\Sl(2,\R)$  (or $\Sl(m, \R)$) whose $A$-orbits define empirical measures satisfying some tightness condition. While one can get good estimates on the size of the sets in Proposition \ref{mainunipo} using Margulis functions and large deviation estimates as in  \cite{MR2247652, MR2787598}, the resulting estimates are not sharp enough to allow us to prove subexponential growth of derivatives. One can compare with the conjectures in \cite{KKLM} about loss of mass.

An elementary related question is the following:  Let $B_n$ be a ball of radius $n$ in a Lie group $G$ (or a lattice $\Gamma$) and suppose there exists subset $S_n$ of $B_n$ such that $S_n$ and $B_n$ have more or less equal mass, meaning that: $$\frac{vol(B_n \setminus S_n)}{vol(B_n)} < \e_n $$ for a certain sequence $\e_n$ of numbers converging to zero. Does there exists an integer $k$ (independent of $n$) such that for $n$ large:

\begin{equation}\label{mof}
B_{n} \subset S_n*S_n*\stackrel{k}{\cdots}*S_n
\end{equation}
Observe that the question depends on how fast $\e_n$ is decreasing and on the group $G$. For example if $G$ abelian, $\e_n$ can be a sufficiently small constant as a consequence of Proposition \ref{babycombinatorics}.  Also, it is not hard to see that for any group $G$ the existence of $k$ is guaranteed if $\e_n$ decreases exponentially quickly.  So the real question is how fast $\e_n$ has to decrease to zero in order for this statement to hold.  Does (\ref{mof}) holds for $ G = \Sl_3(\Z)$ and $\e_n = 2^{-n^c}$  for some $c < 1$?  If the answer to this question is yes, then it would be possible to approach our results via Margulis functions and large deviation estimates.

%{\bf Acknowledgements:}

\subsection*{Acknowledgements}  We thank Dave Witte Morris for his generous willingness to answer questions of all sorts throughout the production of this paper and \cite{BFH}.  We also thank to Shirali Kadyrov, Jayadev Athreya and Alex Eskin for helpful conversations, particularly on the material in Subsection \ref{subsection:failure} and Mikael de la Salle for many helpful conversations regarding strong property $(T)$.  {\blue We also thank the anonymous referee for a very careful reading and numerous comments which helped improve the exposition.}

\section{Standing notation} %\note{This should probably be its own little  section}
We review the notation introduced in \cite{BFH} and establish some standing notation and conventions as well as state some facts  used in the remainder of the paper.

\subsection{Lie theoretic and geometric notation}
We write $G= \Sl(m,\R)$ and $\Gamma= \Sl(m,\Z)$.  Let $\lieg$ denote the Lie algebra of $G$.  Let $\Id$ denote the identity element of  $G$.
We fix the standard Cartan involution $\theta \colon \lieg\to \lieg$ given by $\theta (X) = -X^{t}$ and write $\liek$ and $\liep$, respectively,  for the $+1$ and $-1$ eigenspaces of $\theta$.   Define $\liea$  to be a maximal abelian subalgebra of $\liep$.  Then $\liea$ is the vector space of diagonal matrices.


The roots of $\lieg$ are the  linear functionals $\beta_{i,j}\in \liea^*$ defined as $$\beta_{i,j}(\diag (t_1, \dots, t_m)) = t_i-t_j.$$  The simple positive roots are $\alpha_{j} = \beta_{j,j+1}$ and the positive roots are the positive integral combinations of $\{\alpha_j\}$ that are still roots.
%\item

For a root $\beta$, write $\lieg^\beta$ for the associated root space.   Each root space $\lieg^\beta$ exponentiates to a 1-parameter unipotent subgroup $U^\beta\subset G$.
The Lie subalgebra $\lien$ generated by all root spaces $\lieg^\beta$ for positive roots $\beta$, coincides with the Lie algebra of all strictly upper-triangular matrices.

Let $A,N,$ and $ K$ be the {analytic subgroups} of $G$ corresponding to $\liea, \lien$ and $\liek$.  Then
\begin{enumerate}
\item $A= \exp (\liea)$ is the group of all diagonal matrices with positive entries.  $A$ is an abelian group and we identity linear functionals on $\liea$ with linear functionals on $A$ via the exponential map $\exp\colon \liea\to A$;
\item $N= \exp (\lien)$ is the group of upper-triangular matrices with $1$s on the diagonal;
\item $K = \So(m)$.
\end{enumerate}

%\item
The Weyl group of $G$ is the group of permutation matrices.  This acts transitively on the set of all roots $\Sigma$.

For $1\le i,j\le m$, the subgroup of $G$ generated by $U^{\beta_{i,j}}$ and $U^{\beta_{j,i}}$ is isomorphic to $\Sl(2,\R)$.  We denote this subgroup by  $H_{i,j} = \Sl_{{i,j}}(2,\R).$    Then $\Lambda_{i,j}:=H_{i,j} \cap \Gamma$ is a lattice in $\Sl_{{i,j}}(2,\R)$   isomorphic to $\Sl(2,\Z)$.  %\note{Aaron: Why use $e_{i,j}$ and $E_{i,j}$? David: I removed the $e$ from the subscript on the $\SL$.  I like the way it looks better now anyway.}
 Note then that $X_{i,j}:= H_{i,j}/\Lambda_{i,j}$ is the unit tangent bundle to the modular surface.
We will use the standard notation $E_{i,j}$ for an elementary matrix with 1s on the diagonal and in the $(i,j)$-place and 0s everywhere else.  Note that $E_{i,j}$ and $E_{j,i}$ generate $\Lambda_{i,j}$.



We equip $G$  with a  left-$K$-invariant and  right-$G$-invariant metric.  Such a metric is unique up to scaling.  Let $d$ denote be the induced distance on $G$.
With respect to this metric and distance $d$, each $H_{i,j}$ is geodesically embedded.  By rescaling the metric, %\note{specfied metric on H}
  we may assume the restriction of $d$ to each $H_{i,j}$ {\blue coincides with the standard metric of constant curvature $-1$ on the upper half plane $\So(2)\backslash \Sl(2,\R)$.}
This metric has the following properties that we exploit throughout.
\begin{enumerate}
\item  For any matrix norm $\|\cdot \|$ on $H_{i,j}\simeq \Sl(2,\R)$  there is a $C_1$ such that
\begin{equation}\label{normdistance}  {2}\log \|A\| - C_1 \leq d(A, \Id) \leq    {2}\log \|A\| + C_1\end{equation}
 for all $A\in H_{i,j}$.
 \item   Let $B(\Id,r)$ denote the metric ball of radius $r$  in   $H_{i,j}$ centered at $\id$.  Then with respect to the induced Riemannian  volume on $H_{i,j}$ we have $$\mathrm{vol}(B(\id,r))= 4\pi (\cosh(r) -1)\le 4\pi e^r$$
and for all sufficiently large $r>0$
\begin{equation}\label{decapitationOrImpeachment?}
\mathrm{vol}(B(x,r))\ge e^r.
\end{equation}
\item For any matrix norm $\|\cdot \|$ on $\Sl(m,\R)$, there are constants $C_0>1$ and $\kappa>1$ such that for any matrix $A\in \Sl(m,\R)$ we have %\note{conorm removed here}
    \begin{equation}\label{eq:easy} \begin{gathered}
\kappa\inv\log \|A\| -  C_0
 \le d(A,\id) \le \kappa \log \|A\| +  C_0.
% \\
%{\blue -\kappa}\inv\log m(A) -  C_0
% \le d(A,\id) \le {\blue -\kappa} \log m(A) +  C_0
%
%\frac 1 {C_0} e^{-\kappa d(A,\id)}\le m(A) \le \|A\| \le C_0e^{\kappa d(A,\id)}
%
\end{gathered}
\end{equation}
%where $m(A) := \|A\inv\| \inv$ denotes the conorm of $A$ associated to $\|\cdot \|$.
\item In particular, there are $C_2$ and $C_3$ so that if $E_{i,j}\in \Sl(m,\Z)$ is an elementary unipotent matrix  then
\begin{equation}\label{unipotentgrowth}d(E_{i,j}^k,\id) \le C_2 \log  k + C_3.\end{equation}  %\red{check if used}
\end{enumerate}




%\begin{enumerate}
%\item
%\item We fix the standard cartan involution $\theta \colon \lieg\to \lieg$ given by $\theta (X) = -X^{t}$ and write $\liek$ and $\liep$, respectively,  for the $+1$ and $-1$ eigenspaces of $\theta$.   Define $\liea$  to be the maximal abelian subalgebra of $\liep$.  Then $\liea$ is the group of all diagonal matrices.


%\item The roots of $\lieg$ are the  linear functionals $\beta_{i,j}\in \liea^*$ defined as $$\beta_{i,j}(\diag (t_1, \dots, t_m)) = t_i-t_j.$$  The simple positive roots are $\alpha_{j} = \beta_{j,j+1}$ and the positive roots are the positive integral combinations of $\{\alpha_j\}$ that are still roots.
%\item For a root $\beta$ write $\lieg^\beta$ for the associated root space.  Then, $\lien $ is the Lie algebra spanned by all positive roots  is the group of all strictly upper-trianglar matrices.


%\item Let $A,N,$ and $ K$ be the {analytic subgroups} of $G$ corresponding to $\liea, \lien$ and $\liek$.  Then
%\begin{enumerate}
%\item $A= \exp (\liea)$ is the group of all diagonal matrices with positive entries.  $A$ is an abelian group and we identity linear functionals on $\liea$ with linear functionals on $A$ via the Lie-exponential $\exp$
%\item $N= \exp (\lien)$ is the group of upper-triangular matrices with $1$s on the diagonal
%\item $K = \So(m)$
%\end{enumerate}
%
%\item The Weyl group is the group of permutation matrices.  This acts transitively on the set of all roots $\Sigma$.
%\item We replace $\Gamma$ by $\Sl(m,\Z)$ by inducing the action over the (finite) quotient $\Sl(m,\Z)/\Gamma$.  This requires replacing $M$ by $(\Sl(m,\Z) \times M)/\Gamma$, a compact but not necessarily connected manifold.  The careful reader will note that connectedness of $M$ is not used anywhere in the proofs, either here or in \cite{BFH}.  For simplicity, we will use the notation $\Gamma$ and $M$ throughout, even
%    when we have made these changes.
%\item Take $H_{i,j} \cong \Sl_{e_{i,j}}(2,\R)$.  Then $H_{i,j} \cap \Gamma = \Lambda_{i,j}$ is a lattice in $\Sl_{e_{i,j}}(2,\R)$ isomorphic to $\Sl(2,\Z)$.
%\item Note $H_{i,j}/\Lambda_{i,j}$ is the modular surface.
%\item We will use the standard notation $E_{i,j}$ for an elementary matrix with one's one the diagonal and at the $(i,j)$ place and zeros everywhere else.  $E_{i,j}$ and $E_{j,i}$ generate $\Lambda_{i,j}$.
%\item Unless otherwise remarked $\Id$ will be the identity element in $G= \Sl(m,\R)$.
%\item Equip $G$ will be equipped with a  left-$K$-invariant and  right-$G$-invariant metric.  Such a metric is unique up to scaling.  Let $d$ denote be the induced distance on $G$.
%
%With respect to this metric and distance $d$, each $H_{i,j}$ is geodesically embedded.  By rescaling the metric,
%  we may assume the restriction of $d$ to $H_{i,j}$ coincides with the standard metric on the upper half plane $\So(2)\backslash \Sl(2,\R)$.
%This metric has the following properties
%\begin{enumerate}
%\item  for any norm $\|\cdot \|$ on $H_{i,j}\simeq \Sl(2,\R)$ that there is a $C_1$ with
%\begin{equation}\label{normdistance}  {2}\log \|A\| - C_1 \leq d(A, \Id) \leq    {2}\log \|A\| + C_1\end{equation}
% for all $A\in H_{i,j}$.
% \item   Let $B(\Id,r)$ denote the metric ball of radius $r$ centered at $\id$.  Then with respect to the standard volume on the unit tangent bundle to the upper half plane we have $$\mathrm{vol}(B(x,r))= 4\pi (\cosh(r) -1)\le 4\pi e^r$$
%and for all sufficiently large $r>0$
%\begin{equation}\label{decapitationOrImpeachment?}
%\mathrm{vol}(B(x,r))\ge e^r.
%\end{equation}
%\end{enumerate}
%We recall the following elementary estimate.

%\begin{claim}\label{normdistance} Let $\|\cdot \|$ be any matrix norm in $\Gl(2, \R)$.  Then there exists a constants $B,C > 0$ such that for any $A \in \Sl(2, \R)$ we have:

%\end{claim}

\subsection{Suspension space and induced $G$-action} \label{sec:lolo}

 Let $M^{\alpha} = (G \times M)/\Gamma$ be the fiber-bundle over $\Sl(m,\R)/\Sl(m, \Z)$ obtained as follows: on $G\times M$ let $\Gamma$ act as
$$(g,x)\cdot \gamma = (g\gamma, \alpha(\gamma\inv)(x))$$
and let $G$ act as $$g'\cdot(g,x) = (g'g,x).$$
The $G$-action on $G\times M$ descends to a $G$-action  on the quotient $M^{\alpha} = (G \times M)/\Gamma$.
 Let $\pi\colon M^\alpha \to \Sl(m,\R)/\Sl(m, \Z)$ be the canonical projection.
%\item As in \cite{BFH}, we   write  $F= \ker D\pi$ for the fiberwise tangent bundle to $M^{\alpha}$ and $UF$ for the unit tangent bundle of that bundle.  In addition, we write ${\red D_x g}\colon F(x) \rightarrow F(gx)$ for the fiberwise derivative as in \cite{BFH}.
%\item
As in \cite{BFH}, we   write  $F= \ker D\pi$ for the fiberwise tangent bundle to $M^{\alpha}$.  Write  $\P F$ for the projectivization of the fiberwise tangent bundle.
We write $\restrict { D_x g} F \colon F(x) \rightarrow F(gx)$ for the fiberwise derivative as in \cite{BFH}.  For $(x,[v])\in \P F$ and $g\in G$, write $$g\cdot (x,[v]) := (g\cdot x, [\restrict { D_x g} {F (x)} v])$$ for the action of $g$ on $\P F$ induced by $\restrict { D_x g} F$.

We follow  \cite[Section  2.1]{AWBFRHZW-latticemeasure} and equip $G\times M$ with a $C^1$ Riemannian metric $\langle \cdot, \cdot \rangle$ with the following properties:
\begin{enumerate}
\item $\langle \cdot, \cdot \rangle$ is $\Gamma$-invariant.
\item for $x\in M$ and $g\in G$, under the canonical identification of the $G$-orbit of $(g,x)$ with $G$, the restriction of $\langle \cdot, \cdot \rangle$ to the   $G$-orbit of $(g,x)$  coincides with the fixed right-invariant metric on $G$.
\item There is a Siegel fundamental set $D\subset G$ and $C>1$ such that for any $g_1,g_2\in D$, the map $(g_1,x)\mapsto (g_2,x)$ distorts the restrictions of $\langle \cdot, \cdot \rangle$ to $\{g_1\} \times M$ and $\{g_2\} \times M$  by at most $C$.
\end{enumerate}
The metric then descends to a $C^1$ Riemannian metric on $M^\alpha$.  {\blue Note that by averaging the metric over the left action of $K$, we may also assume that the metric on $M^\alpha$ is left-$K$-invariant.   This, in particular, implies the right-invariant metric on $G$ in $(2)$ above is chosen to be left-$K$-invariant.} %\note{Inserted this here and in section 3.8 below.}


To analyze the coarse dynamics of the suspension action, it is often useful to consider the \emph{return cocycle}
$\beta\colon G \times G/\Gamma \rightarrow \Gamma$.  This cocycle is defined relative to a fundamental
domain $\calF$ for the right $\Gamma$-action on $G$.  For any $x\in G/\Gamma$, take $\tilde x$ to be the unique
lift of $x$ in $\calF$ and define $\beta(g,x)$ to be the unique element of $\gamma\in \Gamma$ such that
$g\tilde{x}\gamma^{-1} \in \mathcal F$.  Any two choices of fundamental domain for $\Gamma$
define cohomologous cocycles but we require a choice of well-controlled fundamental domains $\calF$.  Namely, we choose $\calF$ to either be contained in a Siegel fundamental set or to be a Dirichlet domain for the identity.  With these choices, we have the following.


{\blue Let $\wtd {\mathcal D} \subset \Sl(m,\R)$ denote the Dirichlet domain of the identity for the $\Sl(m,\Z)$ action on $\Sl(m,\R)$; that is
%$\mathcal F$ is an almost-open, fundamental domain for the $\Sl(m,\Z)$ action on $\Sl(n,\R)$ and
$$\wtd {\mathcal D}:= \{ g\in \Sl(n,\R): d(g,\id) \le d(g\gamma, \id) \text{ for all $\gamma\in \Sl(m,\Z)$}\}.$$
Since each $H_{i,j}$ is geodesically embedded in $\Sl(m,\R)$ and since $\Lambda_{i,j} = H_{i,j} \cap \Sl(m,Z)$, it follows
\begin{equation} \label{eq:dirc}\mathcal D := H_{1,2} \cap  \wtd {\mathcal D}\end{equation}
is a Dirichlet domain of the identity for the $\Lambda_{1,2}$-action on $H_{1,2}$.  % for the identity.
Viewing $H_{1,2}\simeq \Sl(2,\R)$ acting  on the upper half-plane model of hyperbolic space $\mathbb H^2= \So(2)\backslash \Sl(2,\R)$ by M\"obius transformations ${\So(2)\backslash \mathcal D}$ is the standard Dirichlet domain for the modular surface, the hyperbolic triangle with endpoints at $1/2 + i\sqrt 3/2$, $-1/2 + i\sqrt 3/2$, and $\infty$.
}



\begin{lemma}
\label{lemma:fromlmr}
If $\mathcal F$ is either contained in either a Siegel fundamental set or a Dirichlet domain for the identity then
there is a constant $C$ such that  for all $g\in G$ and $x\in G/\Gamma$ $$\ell (\beta(g,x)) \leq C d(g,e)+ C d(x, \Gamma) + C.$$
\end{lemma}
In the above lemma, $\ell$ is the word-length of $\beta(g,x)$, $d(g,e)$ is the distance from $g$ to $e$ in $G$, and $d(x, \Gamma)$ is the distance from $x\in G/\Gamma$ to the identity coset $\Gamma$ in $G/\Gamma$.
For a Dirichlet domain for the identity, the Lemma is shown in \cite[\S 2]{MR1767270}; for fundamental domains contained in Siegel fundamental sets, the estimate follows from  \cite[Corollary 3.19]{MR2039990} and   the fact that the distance to the identity in a  Siegel domain is quasi-Lipschitz equivalent to the distance to the identity in the quotient   $G/\Gamma$.   Both estimates heavily use
the main theorem of Lubotzky, Mozes, and Raghunathan \cite{MR1244421,MR1828742} to compare the word-length of $\beta(g,x)\in \Sl(m,\Z)$ with   $\log(\|\beta(g,x)\|)$. %(which holds for all higher-rank lattices
%  \cite{MR2039990}.

{\blue  Fix once and for all a fundamental domain $\mathcal F\subset \wtd {\mathcal D} \subset \Sl(m,\R)$.}


The estimates in  Lemma \ref{lemma:fromlmr} is often used to obtain integrability properties of $\beta$ and   related cocycles with respect  to the Haar measure on $G/\Gamma$.
As the function $x\mapsto d(x, \Gamma)$ is in  $L^p(G/\Gamma,\mathrm{Haar})$ for any compact set $K\subset G$ we have that $$x\mapsto \sup _{g\in K} \ell(\beta (g,x))$$ is in $L^p(G/\Gamma,\text{Haar})$ for all $p\ge 1$.
In the sequel, we typically do not directly use the integrability properties  (since we work with measures other than Haar)  but rather  the   estimate in Lemma \ref{lemma:fromlmr}.


%it is  proven in \cite{MR2039990}.  }


%From Lemma \ref{lemma:fromlmr}, we obtain integrability properties of $\beta$ and   related cocycles.   We first recall the following fact that we use throughout.
%To state the integrability properties of $\beta$, first recall the following fact about Siegel domains.
%\begin{lemma}
%If $\calF$ is a Siegel fundamental set for $\Gamma$ then there is $\eta>0$ such that $$\int _\calF e^{\eta d(\td x, e)} \ d \td x<\infty. $$
%\end{lemma}
%As a consequence, for any compact set $K\subset G$ we have that $$x\mapsto \sup _{g\in K} \ell(\beta (g,x))$$ is in $L^p(G/\Gamma,\text{Haar})$ for all $p\ge 1$.

%\end{enumerate}



\section{Preliminaries on measures, averaging, and Lyapunov exponents}
\label{section:preliminaries}

%In this section we introduce the concept of measures with exponential cusps, this definition will allow us to deal with convergence of sequences %of measures on $G/\Gamma$, to have well defined Lyapunov exponents for such measures over the suspension $M^{\alpha}$ and to average such %measures over unipotent subgroups of $G$. Vaguely speaking, a measure on $G/\Gamma$ has exponentially small cusps if the mass outside a ball of %radius $d$ around a fixed point in $G/\Gamma$ is smaller than $e^{-\eta d}$ for some $\eta> 0$.
We present a number of technical facts regarding invariant measures, equidistribution, averaging, and Lyapunov exponents  that will be used in the remainder of the paper.

\subsection{Ratner's measure classification and equidistribution theorems}
We recall Ratner's theorems on equidistribution of unipotent flows.
Let $U = \{u(t) = \exp_\lieg (t X)\}$ be a 1-parameter unipotent subgroup in $G$.
Given any Borel probability measure $\mu$ on $G/\Gamma$ let
$$U^T\ast \mu := \frac 1 T \int_0^T  u(t)_* \mu \ d t.$$




\begin{theorem}[Ratner]\label{thm:ratner}
%Let $X= G/\Gamma$ and l
Let $U = \{u(t) = \exp_\lieg (t X)\}$ be a 1-parameter unipotent subgroup and consider the action on $G/\Gamma$.  The following hold:
\begin{enumlemma}
	\item \label{ratner1}Every ergodic, $U$-invariant probability measure on $G/\Gamma$ is homogeneous \cite[Theorem 1]{MR1262705}.
	\item \label{ratner2} The orbit closure $\orb_x:= \overline{\{u \cdot x :u\in U\} }$ is homogeneous for every $x\in G/\Gamma$  \cite[Theorem 3]{MR1262705}.
	\item \label{ratner4} The orbit  ${U \cdot x }$ equidistributes in $\orb_x$; that is $U^T\ast \delta_x$ converges to the Haar  measure on $\orb_x$ as $T\to \infty$.

%	\item \label{ratner4} The orbit  ${F_\vecm \cdot x }$ equidistributes in $\orb_x$; that is $\nu_x^{F_\vecm}$ converges to the Haar  measure on $\orb_x$ as $m_1\to \infty, \dots, m_k\to \infty$  \cite[Corollary 1.3]{MR1291701}.
\item \label{ratner3} \label{thisone} Let $\beta$ be a root of $\lieg$ and let $\sl_\beta(2)\subset \lieg$ be the Lie subalgebra generated by $\lieg^\beta$ and $\lieg^{-\beta}$.  Let $e,f,h\subset \sl_\beta(2)$ be an $\sl(2,\R)$ triple with $e\in \lieg^\beta$ and $f\in \lieg^{-\beta}$ and let $\lieh^\beta = \mathrm{span} (h)$.  Let $H^\beta= \exp \lieh^\beta$.

Let $\mu$ be a   $U^{\beta}$-invariant Borel probability measure on $G/\Gamma$.
	If $\mu$ is $H^\beta$-invariant, then $\mu$ is $U^{-\beta}$-invariant.
\end{enumlemma}
\end{theorem}
Conclusion \ref{thisone} follows from \cite[Proposition 2.1]{MR1135878} and the structure of $\mathfrak{sl}(2,\R)$-triples. See also the discussion in the paragraph preceding  \cite[Theorem 9]{MR1262705}.  In our earlier work on cocompact lattices \cite{BFH}, we averaged over higher-dimensional unipotent subgroups and required
a variant of (c) due to Nimish Shah  \cite{MR1291701}.  Here we only average over one-dimensional root subgroups and can use the earlier version due to  Ratner.

From Theorem  \ref{thm:ratner}, for any probability measure $\mu$ on $G/\Gamma$ it follows
that   the weak-$*$ limit $$U\ast \mu:= \lim_{T\to \infty} U^T\ast \mu $$ exists and that the $U$-ergodic components of $U\ast \mu$ are  homogeneous.

\subsection{Measures with exponentially small mass in the cusps}\label{sec:sampson}
We now define precisely  the notion of measures with exponentially small mass in the cusps from the introduction.
Let $(X,d)$ be a complete,  second countable, metric space.  Then $X$ is Polish.  Let  $\mu$ be a finite Borel (and hence Radon) measure on $X$.  We say that $\mu$ has \emph{exponentially small mass in the cusps with exponent $\eta_\mu$} if for all  $0<\eta<\eta_\mu$
\begin{equation}\label{eq:racistPrez}\int_X e^{\eta d(x_0, x)} \ d \mu(x) <\infty\end{equation}
for some (and hence any) choice of base point $x_0\in X$.
We say that a collection $\mathcal M=\{\mu_\zeta \}$ of probability measures on $X$ has \emph{uniformly exponentially small mass in the cusps with exponent $\eta_0$} if for all  $0<\eta<\eta_0$
$$\sup_{\mu_\zeta\in \mathcal M} \left\{\int e^{\eta d(x_0, x)} \ d \mu_\zeta(x)\right\} <\infty.$$

Below, we often work in in the setting $X= G/\Gamma$ where $G = \Sl(m,\R)$ and $\Gamma= \Sl(m,\Z)$ and where $d$ the distance induced from a right-invariant metric on $G$.
%Let $\mu$ be a finite measure on $G/\Gamma$.  We assume that $\mu$ has \emph{exponential cusps with coefficient $\eta_\mu$:} for all  $0<\eta<\eta_\mu$
%$$\int e^{\eta d(e\Gamma, g\Gamma)} \ d \mu(g\Gamma) <\infty.$$
%We say that a collection $\{\mu^\alpha\}$ or probability measures has \emph{uniformly exponential cusps with coefficient $\eta_0$:} if for all  $0<\eta<\eta_0$
%$$\sup_{\mu^\alpha} \left\{\int e^{\eta d(e\Gamma, g\Gamma)} \ d \mu^\alpha(g\Gamma)\right\} <\infty.$$
When $X=   \Sl(m,\R)/\Sl(m,\Z)$ we interpret  a point $x= g\Gamma \in G/\Gamma$ as a unimodular lattice $\Lambda_g = g\cdot \Z^m$.   Fix any norm on $\R^m$ and define the systole of a lattice  $\Lambda\subset \R^m$ to be $$\delta(\Lambda):= \inf\left\{ \| v\| : v\in \Lambda\sm \{0\}\right\}.$$
We have that  \begin{equation}\label{eq:ploy}c_1\le \frac{1- \log(\delta(\Lambda_g))}{1+ (d(g\Gamma, e\Gamma))} \le c_2\end{equation}
%\note{I found this estimate on Margulis-Kleinbock's paper without citation.  Probably due to Mahler or some such, ask %Dave or maybe Dima.} {\red cite?}
for some constants whence $$C_1 e^{c_1d(g\Gamma, e\Gamma)} \le \frac {1}{\delta(\Lambda_g)} \le C_2 e^{c_2 d(g\Gamma, e\Gamma)}.$$
Thus, if we only care about finding a positive exponent $\eta_\mu>0$  such that \eqref{eq:racistPrez} holds for all $\eta<\eta_\mu$, it suffices to find $\eta$ such that
\begin{equation}\label{eq:systole} \int \delta(\Lambda_g)^{-\eta}  \ d \mu(g\Gamma) <\infty.\end{equation}
We define the \emph{systolic exponent} $\eta^S_\mu$ to be the supremum of all $\eta$ satisfying \eqref{eq:systole}.



%Let $\nu_e^U: = U^\infty\ast \delta_{e\Gamma}$.  We have that $\nu_e^U$ is  $U$-ergodic and hence homogeneous, whence $\nu_e^U$ has exponential cusps with coefficient $\eta^U$.

In the sequel, we will frequently use the following proposition to avoid  escape of mass into the cusps of $G/\Gamma$ when  averaging a measure along a unipotent flow.
\begin{proposition}\label{prop:bananas}Let $U$ be a 1-parameter unipotent subgroup of $G$.
Let $\mu$ be a probability measure on $X = \Sl(m,\R)/\Sl(m,\Z)$ with  exponentially small mass  in the cusps.  Then the family of  measures $$\{U^T\ast \mu: T\in \R\} \cup \{U\ast \mu\}$$
has uniformly exponentially small mass in the cusps.   % with coefficient $\min\{\eta_\mu, \eta^U\} $.}  %The same holds for the limit $U^\infty\ast \mu$
\end{proposition}

\subsection{Proof of Proposition \ref{prop:bananas}}

We first show that the family of averaged measures  $$\{U^T\ast \mu: T\in \R\}$$ has uniformly exponentially small mass in the cusps.  The key idea is to use the quantitative non-divergence of unipotent orbits following Kleinbock and Margulis.
\begin{lemma}
\label{lemma:translates}
Let $\mu$ be a probability measure on $X = \Sl(m,\R)/\Sl(m,\Z)$ with    exponentially small mass in the cusps   and systolic exponent $\eta^S_\mu$.

Then the  family of measures  $\{U^T\ast \mu: T\in \R\}$ has uniformly exponentially small mass in the cusps with   systolic exponent   $\min\{\eta^S_\mu,  \frac 1{m^2}\}$.
\end{lemma}

\begin{proof}
Let $\Delta\subset \R^m$ be a discrete subgroup.  Let $\|\Delta\|$ denote the volume of $\Delta_\R /\Delta$ where  $\Delta_\R$ denotes the $\R$-span of $\Delta$.  It follows from Minkowski's lemma that there is a constant $c_m$ (depending only on $m$) such that if
$$\|\Delta\| \le (\rho')^{\mathrm{rk}(\Delta)}$$ then there is a non-zero vector $v\in \Delta$ with $\|v\| \le c_m \rho'$.
In particular, if $\delta(\Lambda)\ge \rho$ then for some constant $c'_m$ we have $$
\|\Delta\| \ge (c_m' \rho)^{\mathrm{rk}(\Delta)}$$ for all discrete subgroups $\Delta\subset \Lambda$.

From \cite[Theorem 5.3]{MR1652916} as extended in \cite[Theorem 0.1]{MR2434296}, there is a $C>1$ such that  for every $\Lambda_g\in G/\Gamma$ and $\epsilon > 0$, if $\delta(\Lambda_g)\ge \rho$  then, since $
\|\Delta\| \ge (c_n' \rho)^{\mathrm{rk}(\Delta)}$ for every discrete subgroup $\Delta\subset \Lambda_g$,
we have
\begin{equation}\label{KM}m\{ t\in [0,T] : \delta( \Lambda_{u_tg}) \le \epsilon \}  \le C\left(\frac {\epsilon}{(c_n')\inv \rho}\right)^{\frac 1 {m^2}}T =
\hat C\left(\frac {\epsilon}{ \rho}\right)^{\frac 1 {m^2}}T\end{equation}
where $m(A)$ is  the Lebesgue measure of the set $A\subset \R$.
Note that   \eqref{KM} still holds % makes sense for % stupid reasons
even in the case $\epsilon \ge \rho$.
Note that if $\beta<{\frac 1 {m^2}}$ then for $\epsilon <\rho$ we have
$$\left(\frac {\epsilon}{ \rho}\right)^{\frac 1 {m^2}}T<\left(\frac {\epsilon}{ \rho}\right)^{\beta}T.$$
In particular, when $\beta<\frac 1 {m^2}$ we   have (for all $\epsilon>0$ including   $\epsilon>\delta(\Lambda_g)$) that
$$m\{ t\in [0,T] : \delta( \Lambda_{u_tg}) \le \epsilon \}  \le \hat C\left(\frac {\epsilon}{\delta(\Lambda_g)}\right)^{\beta}T.$$

Then for $\eta>0$  and $\beta< \frac 1{m^2}$ we have
\begin{align*}
\int  [\delta(\Lambda_{g})]^{-\eta} \ d U^T\ast \mu (g) &=
\int _M \frac{1}{T} \int_{0}^{T} [\delta(\Lambda_{u_t g})]^{-\eta} \ d  t \ d  \mu(g) \\
&= \int _M \frac{1}{T}  \int_{0}^{\infty} m\{t\in [0,T] : [\delta(\Lambda_{u_t g})]^{-\eta} \ge \ell \}   \ d \ell  \ d  \mu(g) \\
&\le
\int _M \frac{1}{T}\left[T+  \int_{1}^{\infty} m\{t\in [0,T] : [\delta(\Lambda_{u_t g})]^{-\eta} \ge \ell \}   \ d \ell\right] \ d  \mu(g) \\
&= 1+
\int _M \frac{1}{T} \int_{1}^{\infty} m\{t\in [0,T] : [\delta(\Lambda_{u_t g})] \le {\ell^{-\frac 1\eta}} \} |  \ d \ell \ d  \mu(g) \\
& \le  1 + \int _M \frac{1}{T}  \int_1^\infty
\left[\hat C\left(\frac {1}{\ell ^{\frac 1\eta} \delta(\Lambda_g)}\right)^{\beta}T\right] \ d \ell
\ d  \mu(g) \\
& =1+  \hat C \left(\int _M
\left(\frac {1}{\delta(\Lambda_g)}\right)^{\beta} \ d  \mu(g)\right)
\left(\int _1 ^\infty \left(\frac {1}{\ell ^{\frac 1\eta}}\right)^{\beta}  \ d \ell
        \right)
\end{align*}
which is uniformly bounded in $T$ as long as $\eta<\beta< \min\{\eta_\mu^S,  \frac 1{m^2}\} $.
\end{proof}

For the limit measure $U \ast \mu= \lim_{T\to \infty} U^T\ast \mu$ we have the following which holds in full generality.
\begin{lemma}\label{lem:pastrydough}

Let $(X,d)$ be a complete,  second countable, metric space.
Let $\nu_j$ be a sequence of  Borel probability  measures on $X$ converging in the weak-$*$ topology to a measure $\nu$.   If the family
$\{ \nu_j\}$ has uniformly exponentially small mass in the cusps with exponent   $\eta_0$ then the limit $\nu$ has   exponentially small mass in the cusps with  exponent   $\eta_0$.   \end{lemma}
\begin{proof}
We have that $\nu_j\to \nu$ in the weak-$*$ topology.  In particular, for any closed set $C\subset X$ and open set  $U\subset X$ we have
$$\limsup_{j\to \infty} \nu_j (C) \le  \nu(C) \quad \text{and} \quad \liminf_{j\to \infty} \nu_j (U) \ge  \nu(U).$$

Fix $0<\eta'<\eta< \eta_0$  and take $\delta:= \frac{\eta}{\eta'}-1.  $
Fix $N$ with  $$\int e^{\eta d(x, x_0)} \ d \nu_j(x) <N$$
for all $j$.
Using Markov's inequality,  for all  $M>0$ and every $j$ we have
$$\nu_j\{ x : e^{\eta d(x,x_0)} > M\} \le  N/M $$
so $$\nu\{ x : e^{\eta d(x_0, x)} > M\} \le  N/M.$$



%We have $$\sup \nu_j\{ x : e^{\eta d(x_0, x)} \ge M\} \le  N $$
%so that
%$$\nu_j\{ x : e^{\eta d(x_0, x)} < M\} \le  N/M$$ for all $j\ge 0$ whence
%$$\nu\{ x : e^{\eta d(x_0, x)} < M\} \le  N/M$$

Then, for the limit measure $\nu$, we have
\begin{align*}
\int_{G/\Gamma} e^{\eta' d(x_0, x)} \ d \nu(x)
&=
\int_0^\infty
\nu\{ x : e^{\eta'  d(x_0, x)} \ge M\} \ d M\\
&=
\int_0^\infty
\nu\{ x : \left(e^{\eta  d(x_0, x)}\right)^{1/(1+\delta)} \ge {M}   \} \ d M\\
&= \int_0^\infty \nu\{ x : e^{\eta  d(x_0, x)} \ge {M}^{1+\delta}   \} \ d M\\
&\le 1 + \int_1^\infty  \frac{N}{{M}^{1+\delta} }  \ d M.\qedhere
\end{align*}
\end{proof}

\subsection{Averaging certain measures on $\Sl(m,\R)/\Sl(m,\Z)$}
Take  $\{\alpha_1, \dots, \alpha_m\}$ to be the standard set of simple positive roots of $\Sl(m,\R)$:
	$$\alpha_j (\diag(e^{t_1}, \dots, e^{t_m})) = t_j-t_{j+1}.$$

Let $H_1 $  be the analytic subgroup of $\Sl(m,\R)$ whose Lie algebra is generated by roots spaces associated to  $\{\pm \alpha_1\}$ and let  $H_2$ be the analytic subgroup of $\Sl(m,\R)$ whose Lie algebra is  generated by roots spaces associated to $\{\pm \alpha_3, \dots, \pm \alpha_n\}$.  We have  $H_1 \cong \Sl(2,\R)$ and $H_2 \cong \Sl(m-2, \R)$.  Then $H = H_1\times H_2\subset \Sl(m,\R)$ is the subgroup of all matrices of the form
$$\left(\begin{array}{cc}B& 0 \\0 & C\end{array}\right)$$
where $\det(B) = \det(C) =1$.

 We let $A'$ be the  the co-rank-1 subgroup $A'\subset A$  of the Cartan subgroup $A$  given by $A'= A\cap H$.
Let $\delta= \alpha_1 + \dots + \alpha_n$ be the highest positive  root.
\begin{proposition}
\label{proposition:averaging}
Let $\mu$ be any $H$-invariant probability on $\Sl(m,\R)/\Sl(m,\Z)$.  Let $\beta'= \alpha_2$ or $\beta' = \delta$ and let $\hat \beta = - \alpha_2$ or $\hat \beta =- \delta$.

Then $ U^{\beta'}\ast \mu$ is $H$-invariant and  $$U^{\hat \beta}\ast  U^{\beta'}\ast \mu$$ is the Haar measure on $G/\Gamma$.
\end{proposition}
\begin{proof}
We have that $\mu$ is $A'$-invariant.  Let $\mu'= U^{\beta'}\ast \mu$ and note that $\mu'$ remains $H$- and $A'$-invariant.

 {\bf Case 1(a) : $\beta' = \alpha_2$.} Consider first the case that $\beta' = \alpha_2$.  Then $\mu'$ remains invariant under $U^{-\alpha_1}$ and $U^{-\alpha_j}$ for all $3\le j\le n$ since these roots commute with $\beta'$.
By Theorem \ref{thm:ratner}\ref{ratner3}   we have that $\mu'$ is also invariant under
$U^{\alpha_1}$ and $U^{\alpha_j}$ for all $3\le j\le n$.  Taking brackets, $\mu'$ is invariant under $U^\beta$ for every positive root $\beta\in \Sigma_+$.


 {\bf Case 1(b) : $ \beta' = \delta$.}
Consider now the case that $\beta' = \delta$.  Then $\mu'$ remains invariant under $U^{\alpha_1}$ and $U^{\alpha_j}$ for all $3\le j\le n$ since these roots commute with $\delta$.
By   Theorem \ref{thm:ratner}\ref{ratner3}  we have that $\mu'$ is also invariant under
$U^{-\alpha_1}$ and $U^{-\alpha_j}$ for all $3\le j\le n$.  Taking brackets, $\mu'$ is invariant under $U^\beta$ for every positive root $\beta$ of the form $\delta - \alpha_n - \alpha_{n-1} - \dots - \alpha_j = \alpha_1 +\dots + \alpha_{j-1}$ for each $j\ge 3$.  In particular, $\mu'$ is invariant under $U^{ \alpha_1 +\alpha_2}$ and hence also  invariant under $U^{ \alpha_2}$.  In particular $\mu'$ is invariant under $U^\beta$ for every positive root $\beta\in \Sigma_+$.

Note that in either case, we have that $\mu'$ is invariant under $U^\beta$ for every positive root $\beta\in \Sigma_+$.

Let $\hat \mu = U^{\hat \beta}\ast \mu'$.

 {\bf Case 2(a) : $\hat \beta = -\alpha_2$.}
If $\hat \beta = -\alpha_2$, then $\hat \mu$ remains invariant under $U^{\alpha_1}$ and $U^{\alpha_j}$ for all $3\le j\le n$.  Note additionally $\hat \mu$ remains invariant under the highest-root group $U^{\delta}$.  Again, by  Theorem \ref{thm:ratner}\ref{ratner3}  we have that $\hat \mu$ is also invariant under
$U^{-\alpha_1}$ and $U^{-\alpha_j}$ for all $3\le j\le n$.  In particular $\hat \mu$ is also invariant under  $U^\beta$ for every negative  root $\beta\in \Sigma_-$.   It follows as in Case 1(b) that $\hat \mu$ is invariant under $U^{\alpha_2}$ and hence invariant under $U^\beta$ for every positive  root $\beta\in \Sigma_+$.  Thus $\mu$ is $G$-invariant.

 {\bf Case 2(b) : $\hat \beta = -\delta$.}
If $\hat \beta = -\delta$, then $\hat \mu$ remains invariant under $U^{-\alpha_1}$ and $U^{-\alpha_j}$ for all $3\le j\le n$.  Note additionally $\hat \mu$ remains invariant under   $U^{\alpha_2}$.  Again,  we have that $\hat \mu$ is also invariant under
$U^{\alpha_1}$ and $U^{\alpha_j}$ for all $3\le j\le n$.  In particular $\hat \mu$ is also invariant under  $U^\beta$ for every positive  root $\beta\in \Sigma_+$.   As in Case 1(b) that $\hat \mu$ is invariant under $U^{-\alpha_2}$ and hence invariant under $U^\beta$ for every negative  root $\beta\in \Sigma_-$.  Thus $\mu$ is $G$-invariant.
\end{proof}




%
%\section{OLD: Averaging certain measures on $\Sl(n,\R)/\Sl(n,\Z)$}
%Let $\{\alpha_1, \dots, \alpha_n\}$ be the standard set of simple roots of $\Sl(n,\R)$ where
%	$$\alpha_j (\diag(e^{t_1}, \dots e^{t_n})) = t_i-t_{i+1}.$$
%
%Sebastian asks us to consider the co-rank-1 subgroup $A'\subset A$  of the Cartan subgroup $A$ where $A'=\ker(\alpha_2)$ and the reductive group $H = H_1\times H_2$ where $H_1 $ is generated by $\{\pm \alpha_1\}$ and  $H_2$ is generated by $\{\pm \alpha_3, \dots, \pm \alpha_n\}$.
%
%Let $\delta= \alpha_1 + \dots + \alpha_n$ be the heights root.
%\begin{proposition}
%Let $\nu$ be any $H$-invariant probability on $\Sl(n,\R)/\Sl(n,\Z)$.  Let $\beta'= \alpha_2$ or $\beta' = \delta$.  The $U^{\beta'}\ast \mu$ is the Haar measure on $G/\Gamma$.
%\end{proposition}
%\begin{proof}
%Let $\mu'= U^{\beta'}\ast \mu.$
%
%Consider first the case that $\beta' = \alpha_2$.  Then $\mu'$ remains invariant under $U^{-\alpha_1}$ and $U^{-\alpha_j}$ for all $3\le j\le n$ since these roots commute with $\beta'$
%By Ratner (or the fact that upper triangular groups are epimorphic in $\Sl(2,\R)$) we have that $\mu'$ is also invariant under
%$U^{\alpha_1}$ and $U^{\alpha_j}$ for all $3\le j\le n$.  Taking brackets, $\mu'$ is invariant under $U^\beta$ for every positive root $\beta\in \Sigma_+$.
%Note that $A'\subset H$ and that there is $s\in A$ with $\alpha_i(s)>0$ for all $i\neq 2$.  \note{aint no such $s$ in $A'$.  You need to average again over $-\alpha_2$ or $-\delta$.}   It follows that $$h_{\mu'}(s) = \sum _{\beta\in \Sigma_+, \beta \neq \alpha_2} \beta(s).$$
%Then
%$$h_{\mu'}(-s) = \sum _{\beta\in \Sigma_-, \beta \neq -\alpha_2} \beta(-s)$$
%which implies $\mu'$ is invariant under $U^\beta$ for every negative  root $\beta\in \Sigma_-$ with $\beta\neq \alpha_2$.  This $\mu'$ is $G$-invariant.
%
%
%
%Consider now the case that $\beta' = \delta$.  Then $\mu'$ remains invariant under $U^{\alpha_1}$ and $U^{\alpha_j}$ for all $3\le j\le n$ since these roots commute with $\delta$.
%By Ratner's trick we have that $\mu'$ is also invariant under
%$U^{-\alpha_1}$ and $U^{-\alpha_j}$ for all $3\le j\le n$.  Taking brackets, $\mu'$ is invariant under $U^\beta$ for every positive root $\beta$ of the form $\delta - \alpha_n - \alpha_{n-1} - \dots - \alpha_j = \alpha_1 +\dots + \alpha_{j-1}$ for each $j\ge 3$.  In particular, $\mu'$ is invariant under $U^{ \alpha_1 +\alpha_2}$ and hence also  invariant under $U^{ \alpha_2}$.  In particular $\mu'$ is invariant under $U^\beta$ for every positive root $\beta\in \Sigma_+$.
%The same entropy argument  above again shows that $\mu'$ is Haar.
%\end{proof}

\def \top{\mathrm{top}}



%%%%%%%%%%%%%%%%% START  AARON %%%%%%%%%%%%%%%%%%%%


\subsection{Lyapunov exponents for unbounded cocycles}
\label{subsection:Lyap}
\def\calE{\mathcal E}
Let $(X,d)$ be a second countable, complete metric space.  We moreover   assume the metric $d$ is proper.  Let $G$ act continuously  on $X$.

Let $\calE\to X$ be a continuous,  finite-dimensional  vector bundle  equipped with a norm  $\|\cdot\|$.  A linear cocycle over the $G$-action on $X$ is an action $\calA\colon G\times \calE\to \calE$ by vector-bundle automorphisms that projects to the $G$-action on $X$.
We write $\calA(g,x)$ for the linear map between Banach spaces $\calE_x$ and $\calE_{g\cdot x}$.  By the norm of $\calA(g,x)$ we mean the operator norm and the conorm is $m(\calA(g,x))= \|\calA(g,x)^{-1}\|^{-1}$.
% Let $\Phi\colon E\to Y\times \R^d$ be a Borel trivialization of $E$.  We say a measurable linear cocycle $\calA\colonG\times Y\to \Gl(d,\R)$ be a continuous linear cocycle over the $G$-action on $Y$ is continuous with respect to $\Phi$ if the action by vector bundle automorphisms $
We say that $\calA$ is \emph{tempered} with respect to the metric $d$ if there is a $k\ge0$ such that  for any compact set $K\subset G$ and base point $x_0\in X$ there is $C>1$ so that $$\sup_{g\in K} \|\calA(g,x)\| \le Ce^{k d(x,x_0)}$$ and
 $$\inf_{g\in K} m(\calA(g,x)) \ge \frac{1}{C}e^{-k d(x,x_0)}$$
 where $\|\cdot\|$ denotes the operator norm and $m(\cdot)$ denotes the operator conorm applied to  linear maps between Banach spaces $\calE_x$ and $\calE_{g\cdot x}$.  %Let $m(\calA(g,x))$ denote the conorm
%applied to  linear maps between Banach spaces $\calE_x$ and $\calE_{g\cdot x}$.


%
%Let   $m(\cdot)$ denotes the conorm of an operator between Banach spaces.  Then,   as the cocycle property implies $\calA(s\inv, s\cdot x) = \calA( s, x)\inv $, we have for  all $x$ and $s\in K$
%%$$\|\calA(s  ,x)\| \ge
%$$ m(\calA(s  ,x))= \frac{ 1}{\|\calA(s\inv, s\cdot x)\| } \ge \frac{1} {C} e^{-k d(s\cdot x,x_0)}.$$
%As $e^{-k d(s\cdot x,x_0)} \ge e^{-k d(x,x_0)}
%
If $\mu$ is a probability measure on $(X,d)$ with exponentially small mass in the cusps, it follows that the function  $x\mapsto d(x,x_0)$ % and $x\mapsto d(sx,x_0)$
is $L^1(\mu)$ whence  we immediately  obtain the following.  %$x\mapsto  \log \|\sup_{s\in K} \calA(s, x)\|  $ is $L^1(\mu)$.
\begin{claim}\label{lem:foobar}
Let $\mu$ a probability  measure on $X$ with exponentially small mass in the cusps.   Suppose that $\calA$ is tempered.  Then for any compact $K\subset G$, the functions   $$x\mapsto  \sup_{s\in K} \log \left \| \calA(s, x)\right \| ,\quad \quad x\mapsto    \inf_{s\in K}\log m\left( \calA(s, x)\right)  $$ are $L^1(\mu)$.
\end{claim}
%\begin{proof}
%
%
%
%There exists $C$ such that for  all $x$ and $s\in K\cup K\inv$ have that $ \|\calA(s, x)\| \le Ce^{k d(x,x_0)}$. Also, as $\calA(s\inv, s\cdot x) = \calA( s, x)\inv $, we have for  all $x$ and $s\in K$
%%$$\|\calA(s  ,x)\| \ge
%$$ m(\calA(s  ,x))= \frac{ 1}{\|\calA(s\inv, s\cdot x)\| } \ge \frac{1} {C} e^{-k d(s\cdot x,x_0)}$$
%
%%By submultiplicativity,  we have $$-  \log C -   \frac{k}{n}\sum_{j=1}^n   d(s^j\cdot x,x_0)  \le \frac 1 n \log  \|\calA(s, x)\| \le  \log C + \frac{k}{n}\sum_{j=0}^{n-1}   d(s^j\cdot x,x_0). $$
%Since $\mu$ has with exponentially small mass in the cusps, the functions $x\mapsto d(x,x_0)$ and $x\mapsto d(sx,x_0)$ are $L^1(\mu)$ whence $x\mapsto  \log \|\sup_{s\in K} \calA(s, x)\|  $ is $L^1(\mu)$.
%\end{proof}


Given $s\in G$ and an $s$-invariant Borel probability measure $\mu$ on $X$ we define the \emph{average leading} (or \emph{top})  \emph{Lyapunov exponent of $\calA$} to be
\begin{equation} \label{eq:toplyap} \lambda_{\top,s,\mu, \calA} := \inf _{n\to \infty} \frac 1 n  \int \log \|\calA(s^n, x)\| \ d \mu (x).\end{equation}
From the   integrability of the function $x\mapsto  \log \|\calA(s, x)\|  $   we obtain   the finiteness of Lyapunov exponents.
\begin{corollary}
\label{corollary:finite}
For $s\in G$ and   $\mu$   an $s$-invariant probability measure on $X$ with exponentially small mass in the cusps, if  $\calA$ is tempered
then the average leading   Lyapunov exponent $\lambda_{\top,s,\mu, \calA}$ of $\calA$ is finite.
%then  $\lambda_{\top,s,\mu, \calA}$ is finite.
\end{corollary}
Note that for an $s$-invariant measure $\mu$,  the sequence  $  \int \log \|\calA(s^n, x)\| \ d \mu (x)$ is subadditive whence the infimum in \eqref{eq:toplyap} maybe replaced by a limit.


As in the case of bounded continuous linear cocycles, we  obtain  upper-semicontinuity of leading Lyapunov exponents   for continuous tempered cocycles when restricted to families of  measures with uniformly exponentially  small measure in the cusp.


\begin{lemma}\label{claim:jjjiiilllkkk}
Let $\calA$ be a tempered cocycle.  Given $s\in G$ suppose  the restriction of the cocycle  $\calA\colon G\times\calE\to \calE$ to the action of $s$ is continuous.

Then---when restricted to a set of $s$-invariant Borel probability measures with  uniformly exponentially small mass in the cusps---the function  $$\mu \mapsto \lambda_{\top,s,\mu, \calA} $$ is upper-semicontinuous with respect to the weak-$*$ topology.
\end{lemma}

\begin{proof}

Let $\calM = \{\mu_\zeta\}_{\zeta\in \calI}$ be a family of $s$-invariant Borel probability measures with  uniformly exponentially small mass in the cusps.  As the pointwise infimum of continuous functions is upper-semicontinuous, is enough to show that the function $$\calM\to \R,\quad \quad \mu\mapsto \int \log \|\calA(s^n, x)\| \ d \mu (x)$$
is  continuous with respect to the weak-$*$ topology for each $n$.  As the weak-$*$ topology is first countable, it is enough to show $\mu\mapsto \int \log \|\calA(s^n, x)\| \ d \mu (x)$ is sequentially continuous.


Let $\mu_j \to \mu_\infty$ in $\calM$.  Given $M>0$, fix a continuous
$\psi_M\colon X\to [0,1]$ with $$ \text{$ \psi_M(x) = 1$ if $d(x,x_0) \le M$ and $\psi_M(x) = 0$ if $d(x,x_0)\ge M+ 1$}.$$


%let $$\psi_M(x)  =\max\{\min \{M, \log \|\calA(s^n, x)\|\}, -M\}.$$
As we assume our metric   is proper,  $x\mapsto \psi_M (x)\log \|\calA(s^n, x)\|$ is a bounded continuous function whence  $$\int \log\psi_M(x)  \log \|\calA(s^n, x)\|\  d \mu_j (x) \to \int\psi_M(x) \log \|\calA(s^n, x)\|  \ d \mu_\infty  (x).$$
Moreover, there are $C>1, k\ge 1,$ and $\eta>0$ such that for all $x\in X$ and $\mu_\zeta\in \calM$
$$- \log C- {k d(x,x_0)} \le \log \|\calA(s^n, x)\| \le \log C+ {k d(x,x_0)}, $$ and $$ \int e^{\eta d(x, x_0)} \ d \mu_\zeta(x)\le C.$$
In particular, $$\mu_\zeta(\{x: d(x,x_0) \ge   M\})\le C e^{-\eta   M}.$$

Thus for any $\mu_\zeta\in \calM$,  we have
\begin{align*}
\int  \big | \log \|\calA(s^n, x)\| &- \psi_M(x)\log \|\calA(s^n, x)\|   \big|  \ d \mu_\zeta(x) \\
& \le  \int_{\{x: d(x,x_0) \ge   M \}}\big | \log \|\calA(s^n, x)\|  - \psi_M(x)\log \|\calA(s^n, x)\|   \big|   \ d \mu_\zeta(x) \\
& \le  \int_{\{x: d(x,x_0) \ge   M \}}\big | \log \|\calA(s^n, x)\|   \big|   \ d \mu_\zeta(x) \\
    &\le  \int_{\{x: d(x,x_0) \ge   M\}}  \log C+ {k d(x,x_0)}    \ d \mu_\zeta(x)  \\
    &\le (\log C  ) C e^{-\eta  M}   + k \int_{\{x: d(x,x_0) \ge   M\}}   { d(x,x_0)}   \ d \mu_\zeta(x)  \\
%        &\le (\log C  ) C e^{-\eta  M }   + k \int_{\ell = 0} ^\infty \mu_\zeta\{ x:    { d(x,x_0)} \ge \ell\} \ d \ell\\
        &\le (\log C   + k M) C e^{-\eta M}  + k\int_{\ell =  M} ^\infty \mu_\zeta\{ x:    { d(x,x_0)} \ge \ell\} \ d \ell\\
                &\le (\log C +k  M) C e^{-\eta M}  + k\int_{\ell =  M} ^\infty  Ce^{-\eta \ell}  \ d \ell\\
                                &\le (\log C + k M) C e^{-\eta M}  + k\frac{C e^{\eta  (- M)}}{\eta }.
\end{align*}
It follows that given $\epsilon>0$ there is $M$ so that
$$\int  \big | \log \|\calA(s^n, x)\| - \psi_M(x)\log \|\calA(s^n, x)\|   \big|  \ d \mu_\zeta(x) \\  \le \epsilon$$
for all $\mu_\zeta\in \calM$.

In particular, taking $M$ and $j$ sufficiently large we have
\begin{align*}
\Big|\int& \log \|\calA(s^n, \cdot )\|   \ d \mu_\infty  - \int \log \|\calA(s^n, \cdot )\| \ d \mu_j  \Big|
\\
&\le
\int \big|\log \|\calA(s^n, \cdot )\|  - \psi_M \log \|\calA(s^n, \cdot )\|\big|    \ d \mu_\infty  \\ &\quad\quad +
\Big|\int \psi_M  \log \|\calA(s^n, \cdot )\| \ d \mu_j  - \int \psi_M  \log \|\calA(s^n, \cdot )\| \ d \mu_\infty  \Big| \\ & \quad \quad +
\int \big|\log \|\calA(s^n, \cdot )\|  - \psi_M \log \|\calA(s^n, \cdot )\| \big|   \ d \mu_j    \\
&\le  3 \epsilon. %\qedhere
\end{align*}
Sequential continuity then follows.
\end{proof}


%{\red
%Moreover, there are $C>1, k\ge 1,$ and $\eta>0$ such that for all $x\in Y$ and $\mu_\gamma\in \calM$
%$$- \log C- {k d(x,x_0)} \le \log \|\calA(s^n, x)\| \le \log C+ {k d(x,x_0)}, $$ and $$ \int e^{\eta d(x, x_0)} \ d \mu_\gamma(x)\le C.$$
%In particular, $$\mu_\gamma(\{x: d(x,x_0) \ge \td M\})\le C e^{-\eta \td M}.$$
%Thus for any $\mu_\gamma\in \calM$, with $\td M = (M-\log C)/k$ we have
%\begin{align*}
%\int   | \log \|\calA(s^n, x)\| &- \psi_M(x)   |  \ d \mu_\zeta(x) \\
%& \le  \int_{\{x: d(x,x_0) \ge \td M \}} \left| \log \|\calA(s^n, x)\|- \psi_M(x)  \right |  \ d \mu_\zeta(x) \\
%  &\le  \int_{\{x: d(x,x_0) \ge \td M\}}  | \log \|\calA(s^n, x)\| |   \ d \mu_\zeta(x)  \\
%    &\le  \int_{\{x: d(x,x_0) \ge \td M\}}  \log C+ {k d(x,x_0)}    \ d \mu_\gamma(x)  \\
%    &\le (\log C  ) C e^{-\eta \td M}   + k \int_{\{x: d(x,x_0) \ge \td M\}}   { d(x,x_0)}   \ d \mu_\gamma(x)  \\
%%        &\le (\log C  ) C e^{-\eta \td M }   + k \int_{\ell = 0} ^\infty \mu_\gamma\{ x:    { d(x,x_0)} \ge \ell\} \ d \ell\\
%        &\le (\log C   + k\td M) C e^{-\eta\td M}  + k\int_{\ell = \td M} ^\infty \mu_\gamma\{ x:    { d(x,x_0)} \ge \ell\} \ d \ell\\
%                &\le (\log C +k \td M) C e^{-\eta\td M}  + k\int_{\ell = \td M} ^\infty  Ce^{-\eta \ell}  \ d \ell\\
%                                &\le (\log C + k\td M) C e^{-\eta\td M}  + k\frac{C e^{\eta  (-\td M)}}{\eta }.
%\end{align*}
%Since $\td M\to \infty$ as $M\to \infty$, given $\epsilon>0$ there is $M$ so that
%$$\int   | \log \|\calA(s^n, x)\| - \psi_M(x)   |  \ d \mu_\gamma(x) \le \epsilon$$
%for all $\mu_\gamma\in \calM$.

%In particular, taking $M$ and $n$ sufficiently large we have
%\begin{align*}
%\Big|\int \log \|\calA(s^n, x)\| & \ d \mu_\infty (x)- \int \log \|\calA(s^n, x)\| \ d \mu_n (x)\Big|
%\\
%&\le
%\left|\int \log \|\calA(s^n, x)\|  - \psi_M(x)  \ d \mu_\infty (x)\right|  \\ &\quad +
%\left|\int \psi_M(x) \ d \mu_n(x) - \int \psi_M(x)  \ d \mu_\infty (x)\right| \\ & \quad \quad +
%\left|\int \log \|\calA(s^n, x)\|  - \psi_M(x)  \ d \mu_n (x)\right| \\
%&\le  3 \epsilon. \qedhere
%\end{align*}}
%\end{proof}

\subsection{Lyapunov exponents under averaging and limits}
We now consider the behavior of the top Lyapunov exponent $\lambda_{\top, s, \mu, \calA}$ as we average an $s$-invariant probability measure $\mu$ over an amenable subgroup of $G$ contained in the centralizer of $s$.
\begin{lemma}
\label{lemma:averaging}
Let $s\in G$ and let $\mu$ be an $s$-invariant probability measure on $X$ with exponentially small mass in the cusps.  Let $\calA\colon G\times \calE\to \calE$ be a tempered continuous cocycle.
%\begin{enumerate}
%\item

For any amenable subgroup $H\subset C_G(s)$ and any   \Folner sequence of precompact sets $F_n$ in $H$,  if the family
$\{F_n \ast \mu\}$ has uniformly exponentially small mass in the cusps  then for any subsequential limit
$\mu'$ of $\{F_n \ast \mu\}$ we have
  $$\lambda_{\top, s, \mu, \calA} \le \lambda_{\top, s, \mu', \calA}.$$
%\item For any unipotent subgroup $U$ centralized by $s$,
%and any regular \Folner sequence $F_n$ in $U$ and any limit measure
%$\mu'$ of $\{F_n \ast \mu\}$,
%$$\lambda_{\top, \mu,s} \le \lambda_{\top,\mu',s}.$$
%\end{enumerate}
\end{lemma}

%{\blue
\begin{proof}
{\blue First note that  Lemma \ref{lem:pastrydough} implies the family $\{F_n \ast \mu\}\cup \{\mu'\}$  has uniformly exponentially small mass in the cusps.}    Note also that for every $m$, the measure $F_m \ast \mu$ is $s$-invariant.

%As we assume
We first claim that  $\lambda_{\top, s, F_m \ast \mu, \calA} =  \lambda_{\top, s, \mu, \calA}$ for every $m$.
 %To show that $\lambda_+(s,F_\vecm \ast \mu, \calA) =  \lambda_+(s,\mu, \calA)$, define
For   $t \in H$ define $c_t (x)=   \sup\{\|\calA(t , x)\| , m(\calA(t, x))\inv \} $ and let $c_m(x) = \sup _{t\in F_m } c_t(x)$.
As $F_m$ is precompact, from  Claim \ref{lem:foobar} we have  that $\log c_m \in L^1(\mu)$.  % whence for any $\epsilon>0$ and almost every $x\in Y$   there is $C(x)$ such that for any $n\in \Z$,  $$c_m(s^n \cdot x)  \le C(x) e^{|n| \epsilon}.$$

 For  $x \in M$ and   $t \in F_m$, the cocycle property and subadditivity of norms yields
\begin{align*}
\log \|\calA(s^n, tx)\| &\leq  \log \|\calA(t\inv, tx)\| + \log \|\calA(s^n, x)\| + \log \|\calA(t , s^nx)\|  \\
&=  \log \|\calA(t, x)\inv\| + \log \|\calA(s^n, x)\| + \log \|\calA(t , s^nx)\|  \\  %\ \ \text{ (By the Cocycle sub-additivity) } \\
&\leq \log c_m(x) + \log c_m (s^n(x)) + \log \|\calA(s^n, x)\|. %&\leq 2\log C(x) + {\epsilon |n|} + \log \|\calA(s^n, x)\|.
\end{align*}
Using that $\mu$ is $s$-invariant, we have for every $n$ that
\begin{align*}
\int \log &\|\calA(s^n, x) \| \ d (F_m\ast \mu )(x) = \dfrac 1 {|F_m|}\int_{F_m} \int \log \|\calA(s^n, x)\| \ dt*\mu (x) \ d t \\
&= \dfrac 1 {|F_{m}|}\int_{F_{m}} \int \log \|\calA(s^n, tx)\| \ d\mu (x) \ d t \\
&\leq \dfrac 1 {|F_{m}|}\int_{F_{m}}  \bigg(\int  \log c_m(x) + \log c_m (s^n(x)) +  \log \|\calA(s^n, x)\|  \ d \mu (x) \bigg) \ d t \\
&\leq 2\int  \log  c_m(x)   \ d \mu(x) + \int \log \|\calA(s^n, x)\| \ d \mu{(x)}
\end{align*}
%\noindent for some constant $C$ independent of $n$.
Dividing by $n$ yields $\lambda_{\top, s, F_m \ast \mu,\calA}\le \lambda_{\top, s, \mu, \calA}$.   The reverse  inequality is similar.


%\lambda_+(s,F_{m} \ast \mu, \calA) \leq  \lambda_+(s,\mu, \calA)$.   % be proven in the same fashion, therefore  $\lambda_+(s,F_{m} \ast \mu, \calA) =  \lambda_+(s,\mu, \calA)$ and Item
The inequality then follows from the  upper-semicontinuity  in Lemma \ref{claim:jjjiiilllkkk}.
\end{proof}



Consider now any $Y\in \lieg$ with $\|Y\|=1$, a point  $x\in X$,  and $t>0$.
%We denote by $\eta_n= \eta(Y_n,t_n,x_n)$   the empirical measure along the orbit $\exp (tY_n) x_n$ until time $t_n$: given a bounded continuous $\phi\colon Y\to \R$ define
%$$\int_X \phi (x) \ d \eta_n(x) :=
%\frac{1}{t_n} \int_0^{t_n} \phi \big(\exp (tY_n) \cdot x_n\big) \ d t.$$
The     empirical measure $\eta(Y,t,x)$ along the orbit $\exp (sY) x$ until time $t$ is the measure defined as follows: given a bounded continuous $\phi\colon X\to \R$, the integral of $\phi$ with respect to the empirical measure  $\eta(Y,t,x)$ is
$$
\int \phi \ d \eta (Y, t, x) := \frac{1}{t} \int_0^{t} \phi \big(\exp (sY) \cdot x\big) \ d s.$$
Similarly, given a probability measure $\mu$ on $X$,   the empirical distribution $\eta(Y,t,\mu)$ of $\mu$ along the orbit of $\exp (sY) $ until time $t$ is defined as % follows: given a bounded continuous $\phi\colon Y\to \R$, the integral of $\phi$ with respect to the empirical measures is
$$
\int \phi \ d \eta(Y,t, \mu):= \frac{1}{t} \int_X \int_0^{t} \phi \big(\exp (sY) \cdot x\big) \ d s \ d \mu(x).$$
%{\blue Even more generally, consider a (continuous) multiplicative homomorphism $\psi\colon \R \to \R_>0$.  Define the  $\psi$-weighted empirical distribution $\eta^\psi(Y,t,\mu)$ of $\mu$ along the orbit of $\exp (sY) $ until time $t$  as follows: given a bounded continuous $\phi\colon Y\to \R$, the integral of $\phi$ with respect to the empirical measures is
%$$
%\int \phi \ d \eta^(Y,t, \mu):=\frac {1}{\int _0^t \psi(t) \ dt}  \frac{1}{t} \int_X \int_0^{t} \psi(t) \phi \big(\exp (sY) \cdot x\big) \ d s \ d \mu(x).$$}


%\note{put in "weighted empirical averages"}
Consider now sequences  $Y_n\in \lieg$ with $\|Y_n\|=1$ and  $t_n>0$.   For part \ref{lazylemmac} of the following lemma, we add an additional  assumption that the action of $G$ on $(X,d)$ has \emph{uniform displacement:} for any compact $K\subset G$ there is $C'$ such that for all $x\in X$ and $g\in K$, $$d(g\cdot x, x )\le C'.$$


\begin{lemma} %\note{rewrote for empirical measures of measures rather than points}
\label{lemma:firstexponents}
Suppose the action of $G$ on $(X,d)$ has uniform displacement and let $\calA\colon G\times \calE\to \calE$ be a tempered continuous cocycle.


Let $Y_n\in \lieg$ and  $t_n\ge 0$ be  sequences with  $\|Y_n\|=1$ for all $n$ and $t_n\to \infty$. Let $\mu_n$ be a sequence of Borel probability measures on $X$ and define $\eta_n :=  \eta(Y_n,t_n,\mu_n)$ to be the empirical distribution of $\mu_n$ along the orbit of $\exp (sY_n)$ for $0\le s\le t_n$.
 Assume that
\begin{enumerate}%\note{when we apply this, we use averages of emperical measures}
\item the family of empirical distributions  $\{\eta_n\} $ defined above has uniformly exponentially small mass in the cusps; and
\item $\int \log  \|\calA ( \exp (t_nY_n),x) \|  \ d \mu_n(x) \geq \epsilon t_n$.
%\item $Y_n \to Y_\infty$ in $\lieg$.
\end{enumerate}
Then
\begin{enumlemma}
\item \label{lazylemmaa} the family $\{\eta_n\} $ is pre-compact;
\item \label{lazylemmab}for any subsequential limit $Y_\infty = \lim_{j\to \infty}  Y_{n_j},$  any subsequential limit $\eta_\infty $ of  $\{ \eta_{n_j}\}$ is invariant under the 1-parameter subgroup $\{ \exp (tY_\infty): t\in \R\}$;
\item \label{lazylemmac}$\lambda_{\top,  \exp (Y_\infty), \eta_\infty,\calA}\ge\epsilon>0$.
%converge to an $a^1$ invariant measure $\mu$ on $N^{\alpha}$ whose projection to $\Sl(2,\R)/\Sl(2,\Z)$ is Haar and for which we have $\lambda_{top,\mu, a^1}>0$. \note{Why would the projection be Haar? I don't think this is correct/}
\end{enumlemma}
\end{lemma}

\begin{proof}[Proof of Lemma \ref{lemma:firstexponents} \ref{lazylemmaa} and  \ref{lazylemmab}]
As in the proof of Lemma \ref{claim:jjjiiilllkkk}, from the assumption that $\{\eta_n\}$ has uniformly exponentially small mass in the cusps we obtain uniform bounds $$\eta_n (\{x: d(x,x_0) \ge \ell\})\le C e^{-\eta \ell}$$ for all $n$.  Combined with the properness of $d$, this establishes uniform tightness of the family of  measures $\{\eta_n\}$ and  \ref{lazylemmaa} follows.

For  \ref{lazylemmab},    let  $\phi \colon X \to \R$ be a compactly supported  continuous function.  Then for any $s>0$
\begin{align*}
\int_{X} \phi \circ \exp (sY_\infty) -  \phi \ d \eta_n
&= \int_{X} \phi \circ \exp (sY_\infty)  -  \phi \circ \exp(s Y_n) \ d\eta_n\\
&+ \int_{X}  \phi \circ \exp(sY_n) - \phi \ d\eta_n
\end{align*}

%For fixed $t$, t
\noindent The first integral converges to zero as the functions $\phi \circ \exp (wY_\infty)  -  \phi \circ \exp(w Y_n)$ converges uniformly to zero in $n$ for fixed $w$.  The second integral clearly converges to zero  since for $t_n\ge s$ we have
\begin{align*}
\int_{X}  \phi &\circ \exp(s Y_n) - \phi \ d\eta_n=
\frac{1}{t_n} \int_0^{t_n} \int_X \phi \left( \exp\left((s+t) Y_n\right)  x \right) - \phi \left(\exp (tY_n) x \right) \ d \mu_n(x) \ d t\\
&=
\frac{1}{t_n}\left[ - \int_0^{s} \int _X\phi \left( \exp\left(t Y_n\right)  x \right) \mu_n(x) \ d t + \int_{t_n}^{t_n+s}  \int_X \phi \left( \exp\left(t Y_n\right)  x \right) \ d \mu_n(x) \ d t\right]
\end{align*}
which converges to 0 as $t_n\to \infty$ as $\phi$ is bounded.
\end{proof}

%\note{Added three sentences before dropping the reader off a cliff. Someone should check them. -D}


The proof of Lemma \ref{lemma:firstexponents}\ref{lazylemmac} is quite involved. It is the analogue in the non-compact
setting of \cite[Lemma 3.6]{BFH};  we recommend the reader read the proof of of \cite[Lemma 3.6]{BFH} first.  Two technical
complications arise in the proof of Lemma \ref{lemma:firstexponents}\ref{lazylemmac}.
 First, we must control for ``escape of Lyapunov exponent'' as our cocycle is unbounded.    Second, in \cite{BFH} it was sufficient to  consider the average  of Dirac masses $\delta_{x_n}$ along a single orbit $\exp(sY_n)x_n$; here we average measures $\mu_n$  along an  orbit of $\exp (sY_n).$


% need one for more general measures.
To prove Lemma \ref{lemma:firstexponents}\ref{lazylemmac} we first introduce a number of {\blue standard} auxiliary objects.
 Let $\P \calE\to X$ denote the projectivization of the  tangent bundle    $\calE\to X$.
 We represent a point in $\P\calE$ as $(x,[v])$ where $[v]$ is an equivalence class of non-zero vectors in the fiber $\calE(x)$.
 For each $n$, let $\sigma_n\colon X\to \calE\sm\{0\}$ be a nowhere vanishing Borel section such that
$$\|\calA (\exp (t_nY_n),x ) (\sigma_n(x))\| \|(\sigma_n(x))\|\inv = \|\calA (\exp (t_nY_n),x)\|$$
for every $x\in X$.
The $G$-action on $\calE$ by vector-bundle automorphisms induces  a natural $G$-action on $\P \calE$ which restricts to projective transformations between each fiber and its image.    For each $n$, let $\td \eta_n$ be the probability measure  on $\P\calE$ given as follows: given a bounded continuous $\phi\colon \P \calE\to \R$ define
$$\int _{\P \calE} \phi  \ d \td \eta_n  :=
\frac{1}{t_n} \int_0^{t_n} \int _X \phi \big(\exp (tY_n) \cdot (x,  [\sigma_n(x)] )\big )\ d \mu_n(x) \ d t.$$%\note{removed incorrect equation.}
%{\red That is, $$ \td \eta_n= \int  \delta _{(x, [\sigma_n(x)])} \ d \eta_n (x).$$}
We have that $\td \eta_n$ projects to $ \eta_n$ under the natural projection $\P\calE\to X$; moreover, if $\eta_{j_k}$ is a subsequence converging to $\eta_\infty,$ then   any weak-$*$ subsequential limit $\td \eta_\infty$ of $\{ \td \eta_{n_{j_k}}\}$   projects to $ \eta_\infty$.


Define  $\Phi\colon \lieg \times \P\calE\to \R$ by  $$\Phi\big(Y,(x, [v])\big) := \log \left(\left\| \calA\big(\exp
(Y), x\big)v\right\|\|v\|\inv\right).$$
Note for each fixed $Y\in \lieg$ that $\Phi$ satisfies a cocycle property:
\begin{equation}\label{eq:cocycle}\Phi\big((s+t)Y,(x, [v])\big)= \Phi\big(tY,(x, [v])\big) +  \Phi\big(sY,\exp(tY)\cdot(x, [v])\big) \end{equation}

By hypothesis, there are $C>1$, $ k\ge 1$, and $\eta>0$ such that $$\int e^{\eta d(x,x_0)} \ d \eta_n \le C$$ for all $n$ and
$$ \frac{1}{C}  e^{-k d(x,x_0)} \le \left \|\calA(\exp (Y),x)v\right\| \|v\|\inv  \le  C  e^{k d(x,x_0)}$$ for all $(x,[v])\in \P\calE$ and $Y\in \lieg$ with $\|Y\|\le 1.$

%Let $$M_n = \sup _{0\le t\le t_n}\left\{d\left( \big(\exp (tY_n) x_n\big),x_0\right)\right\}.$$\note{ISSUES here with measures??}
%As we assume the  $G$-action on $(X,d)$ has uniform displacement, take $$C_1 = \sup_{\|Y\|\le 1, x\in X} \{d(\exp (Y)\cdot x, x)\}.$$
%As $$ \frac{1}{t_n} \int_0^{t_n} e^{\eta d\left( (\exp (tY_n) x_n,x_0\right)} \ d t= \int e^{\eta d(x,x_0)} \ d \eta_n \le C$$
%it follows if $t_n\ge 1$ that
%$$e^{\eta (M_n -  C_1)}  \le  Ct_n$$ \note{estimate here is a little unclear to AWB}
%whence $$M_n \le \eta\inv( \log  C  + \log t_n )  + C_1 =: \eta\inv \log t_n +  C_2.$$
%Thus since $\|Y_n \| = 1,$\begin{align*}
%\sup _{0\le t\le t_n, 0\le s\le 1} \Phi(sY_n, \exp(t Y_n)\cdot (x_n, [v_n]))
%&\le \log C + k M_n \\
%&\le \log C + k ( \eta\inv \log t_n +  C_2) \\
%&=:   k  \eta\inv \log t_n +  C_3.
%\end{align*}
%In particular, we have
% \begin{align*}\frac{1}{t_n} &\log \|\calA(t_nY_n, x_n) \| \\
% & = \frac{1}{t_n} \Phi(t_nY_n, (x_n,[v_n]))\\
% & = \frac{1}{t_n}\Phi(\lfloor t_n\rfloor Y_n, (x_n,[v_n]))+
% \frac{1}{t_n}\Phi((t-\lfloor t_n\rfloor) Y_n, \exp (\lfloor t_n\rfloor Y_n)\cdot (x_n,[v_n])).
% \end{align*}
% Since
% $$\left|\frac{1}{t_n}\Phi((t-\lfloor t_n\rfloor) Y_n, \exp (\lfloor t_n\rfloor Y_n)\cdot (x_n,[v_n]))\right|\le \frac{1}{t_n}( k  \eta\inv \log t_n +  C_3)$$ goes to 0 as $t_n\to \infty$
% it follows that
% \begin{equation}\label{eq:thiseq}\liminf_{n\to \infty}  \frac{1}{t_n}\Phi(\lfloor t_n\rfloor Y_n, (x_n,[v_n])) = \liminf_{n\to \infty} \frac{1}{t_n} \log \|\calA(t_nY_n, x_n) \| \ge \epsilon >0.\end{equation}



For each $n$, let $$M_n(x) = \sup _{0\le t\le t_n}\left\{d\left( \big(\exp (tY_n) x\big),x_0\right)\right\}.$$ %\note{ISSUES here with measures??}
As we assume the  $G$-action on $(X,d)$ has uniform displacement, take $$C_1 = \sup_{\|Y\|\le 1, x\in X} \{d(\exp (Y)\cdot x, x)\}.$$
%\note{please check this explaination}
{\blue
We have  $$ \frac{1}{t_n} \int_0^{t_n} \int_X e^{\eta d\left( (\exp (tY_n) x ,x_0\right)} \ d \mu_n(x) \ d t= \int e^{\eta d(x,x_0)} \ d \eta_n \le C.$$
If $t_n\ge 1$ then for every $x$ there is an interval $I_x\subset [0,t_n]$ of length $1$ on which $$d\left( (\exp (tY_n) x ,x_0\right)\ge (M_n (x)-  C_1)$$ for all $t\in I_x.$
It follows that
$$\int_X e^{\eta (M_n (x)-  C_1)}   \ d \mu_n(x) \le    \int_X \int_{I_x} e^{\eta d\left( (\exp (tY_n) x ,x_0\right)} \ d t  \ d \mu_n(x) \le  Ct_n.  $$ %\note{estimate here is a little unclear to AWB}
}

%As $$ \frac{1}{t_n} \int_0^{t_n} \int_X e^{\eta d\left( (\exp (tY_n) x ,x_0\right)} \ d \mu_n(x) \ d t= \int e^{\eta d(x,x_0)} \ d \eta_n \le C$$
%it follows if $t_n\ge 1$ that
%$$\int_X e^{\eta (M_n (x)-  C_1)}   \ d \mu_n(x) \le  C(t_n +1).  $$ %\note{estimate here is a little unclear to AWB}
By Jensen's inequality we have
$$\int_X \eta ( M_n(x) - C_1) \ d \mu_n(x) \le \log \int_X e^{\eta (M_n (x)-  C_1)}  d \mu_n (x)$$
whence
 $$\int M_n (x) \ d \mu_n(x) \le \eta\inv( \log  C  + \log t_n )  + C_1 =: \eta\inv \log t_n +  C_2.$$


Since $\|Y_n \| = 1,$ %for every $(x,[v])\in \P\calE$ we have
we have \begin{equation}\label{eq:autocracy} \begin{aligned}
 \sup _{0\le t\le t_n, 0\le s\le 1}  \int_X &\left| \Phi(sY_n, \exp(t Y_n)\cdot (x, [\sigma_n(x)]))\right |  \ d \mu_n(x)\\
&\le \int_X \sup _{0\le t\le t_n, 0\le s\le 1} \left  | \Phi(sY_n, \exp(t Y_n)\cdot (x, [\sigma_n(x)]))\right |  \ d \mu_n(x)\\
&\le \int |\log C| + k M_n (x) \ d \mu_n(x)\\
&\le |\log C |+ k ( \eta\inv \log t_n +  C_2) \\
&=:   k  \eta\inv \log t_n +  C_3.
\end{aligned}\end{equation}
%\note{integrate against $\mu_n$}
In particular, we have
 \begin{align*}\frac{1}{t_n} & \int_X \log \|\calA(\exp(t_nY_n), x) \| \ d \mu_n(x) \\
 & = \frac{1}{t_n}\int_X \Phi(t_nY_n, (x,[\sigma_n(x)])) \ d \mu_n(x) \\
 & = \frac{1}{t_n}\int_X\Phi(\lfloor t_n\rfloor Y_n, (x,[\sigma_n(x)]))\ d \mu_n(x)  \\ &\quad  +
 \frac{1}{t_n}\int_X \Phi((t-\lfloor t_n\rfloor) Y_n, \exp (\lfloor t_n\rfloor Y_n)\cdot (x,[\sigma_n(x)]))\ d \mu_n(x) .
 \end{align*}
 Since
 $$\left|\frac{1}{t_n} \int_X \Phi((t-\lfloor t_n\rfloor) Y_n, \exp (\lfloor t_n\rfloor Y_n)\cdot (x,[\sigma_n(x)]))\ d \mu_n(x) \right|\le \frac{1}{t_n}( k  \eta\inv \log t_n +  C_3)$$ goes to 0 as $t_n\to \infty$
 it follows that
 \begin{equation}
 \begin{aligned}\label{eq:thiseq}\liminf_{n\to \infty} \int_X  \frac{1}{t_n}&\Phi(\lfloor t_n\rfloor Y_n, (x,[\sigma_n(x)]))  \ d \mu_n(x)  \\&= \liminf_{n\to \infty} \frac{1}{t_n} \int_X  \log \|\calA(\exp(t_nY_n), x) \| \ d \mu_n(x)\\& \ge \epsilon >0.
 \end{aligned}\end{equation}




With the above objects and estimates we complete the proof of  Lemma \ref{lemma:firstexponents}.
\begin{proof}[Proof of Lemma \ref{lemma:firstexponents} \ref{lazylemmac}]
Consider first the expression $\int \Phi (Y_n, \cdot )\  d \td \eta_n.$  We have
\begin{align*}
\int \Phi  & (Y_n, \cdot ) \ d \td \eta_n
\\& =  \frac{1}{t_n}\int_0^{t_n} \int_X \Phi  \big(Y_n,\exp (tY_n) \cdot (x,  [\sigma_n(x)] )\big ) \ d\mu_n(x) \ d t
\\&= \frac{1}{t_n}  \int_0^{\lfloor t_n\rfloor}\int_X  \Phi    \big(Y_n,\exp (tY_n) \cdot (x,  [\sigma_n(x)] )\big )  \ d\mu_n(x) \ d t \\
&\quad
+ \frac{1}{t_n} \int_ {\lfloor t_n\rfloor} ^{t_n}\int_X \Phi    \big(Y_n,\exp (tY_n) \cdot (x,  [\sigma_n(x)] )\big ) \ d\mu_n(x)   \ d t
%\\&= \frac{1}{t_n}        \Bigg[    \Phi   \big(t_n Y_n,  (x_n,  [v_n] )  +    \int_0^{1}   \Phi    \big(-tY_n, \exp (tY_n) \cdot (x_n,  [v_n] )\big ) \ d t \\
%  &\quad \quad  -   \int_0^{1}   \Phi    \big(-tY_n, \exp ((t+\lfloor t_n\rfloor) Y_n) \cdot (x_n,  [v_n] )\big ) \ d t \Bigg]
\end{align*}
Note that the contribution of the second integral is bounded by
$$\left| \frac{1}{t_n} \int_ {\lfloor t_n\rfloor} ^{t_n}\int_X \Phi    \big(Y_n,\exp (tY_n) \cdot (x,  [\sigma_n(x)] )\big ) \ d\mu_n(x)   \ d t \right|
\le \frac{1}{t_n}(k\eta\inv \log t_n +  C_3)$$
which goes to zero as $t_n\to \infty$.


%\\& =  \frac{1}{t_n}\int_0^{1} \Phi  \big(t_nY_n,\exp (tY_n) \cdot (x,  [\sigma_n(x)] )\big ) \ d t
 Repeatedly applying the cocycle property \eqref{eq:cocycle} of $\Phi(Y_n, \cdot)$ we have  for $t_n\ge 1$ that
\begin{align*}
  \frac{1}{t_n}  \int_X  \int_0^{\lfloor t_n\rfloor} \Phi  & \big(Y_n,\exp (tY_n) \cdot (x,  [\sigma_n(x)] )\big ) \ d t \ d \mu_n(x)
  \\&= \frac{1}{t_n} \int_X  \int_0^{1}   \Phi    \big({\lfloor t_n\rfloor} Y_n, \exp (tY_n) \cdot (x,  [\sigma_n(x)] )\big ) \ d t\ d \mu_n(x)
  \\&= \frac{1}{t_n}  \int_X \int_0^{1}  \Big( \Phi    \big({\lfloor t_n\rfloor} Y_n,  (x,  [\sigma_n(x)] )\big )
  -  \Phi    \big( t Y_n,  (x,  [\sigma_n(x)] )\big )
  \\ & \quad\quad\quad +     \Phi    \big( t Y_n, \exp (\lfloor t_n\rfloor Y_n)\cdot  (x,  [\sigma_n(x)] )\big )
  \Big)
   \ d t\ d \mu_n(x)
   \\& =  \frac{1}{t_n}   \int_X \Phi    \big({\lfloor t_n\rfloor} Y_n,  (x,  [\sigma_n(x)] )  \ d \mu_n(x) +
 \frac{1}{t_n} \int_X \int_0^{1}  \Big(    -  \Phi    \big( t Y_n,  (x,  [\sigma_n(x)] )\big )   \\ & \quad\quad\quad +      \Phi    \big( t Y_n, \exp (\lfloor t_n\rfloor Y_n)\cdot  (x,  [\sigma_n(x)] )\big )
  \Big) \ dt\ d \mu_n(x)
%
%
%  \\&= \frac{1}{t_n}        \Bigg[   \sum _{k = 0}^{\lfloor t_n\rfloor}  \Phi    \big(Y_n,u_n^k   \cdot (x,  [\sigma_n(x)] )  +    \int_0^{1}   \Phi    \big(-tY_n, \exp (tY_n) \cdot (x,  [\sigma_n(x)] )\big ) \ d t \\
%  &\quad \quad  -   \int_0^{1}   \Phi    \big(-tY_n, \exp ((t+\lfloor t_n\rfloor) Y_n) \cdot (x,  [\sigma_n(x)] )\big ) \ d t \Bigg]
\end{align*}
From \eqref{eq:autocracy}, the contribution of the second and third integrals is bounded by
\begin{align*}
 &\left|\frac{1}{t_n}\int_X  \int_0^{1}  \Big(    -  \Phi    \big( t Y_n,  (x,  [\sigma_n(x)] )\big )
      +     \Phi    \big( t Y_n, \exp (\lfloor t_n\rfloor Y_n)\cdot  (x,  [\sigma_n(x)] )\big )
  \Big)  \ d t  \ d \mu_n(x)
 \right|
 \\&\quad \quad \quad \quad \le \frac{1}{t_n}  \int_0^{1}  2 ( k  \eta\inv \log t_n +  C_3)\ dt
 \\&\quad \quad \quad \quad = \frac{1}{t_n}    2 ( k  \eta\inv \log t_n +  C_3)
\end{align*}
which tend to zero as $t_n\to \infty$.  We then conclude from \eqref{eq:thiseq} that
\begin{equation}
\label{eq:keyless}
\liminf_{n\to \infty} \int \Phi   (Y_n, \cdot ) \ d \td \eta_n = \liminf_{n\to \infty}    \frac{1}{t_n} \int_X    \Phi    \big({\lfloor t_n\rfloor} Y_n,  (x,  [\sigma_n(x)] ) \ d \mu_n(x)
%\liminf  \frac{1}{t_n}\Phi(\lfloor t_n\rfloor Y_n, (x,[\sigma_n(x)])
\ge \epsilon >0.
\end{equation}

To complete the proof of \ref{lazylemmac}, for  $M>0$     take $\psi_M\colon X\to [0,1]$ continuous with $$ \text{$ \psi_M(x) = 1$ if $d(x,x_0) \le M$ and $\psi_M(x) = 0$ if $d(x,x_0)\ge M+ 1$}.$$
Let $\Psi_M\colon \P\calE \to [0,1]$ be $$\Psi_M(x,[v]) = \psi_M(x).$$
and define $\Phi_M \colon \lieg\times \P\calE\to \R$ to be
$$\Phi_M\big(Y,(x, [v])\big) :=\Psi_M(x, [v]) \Phi\big(Y,(x, [v])\big).$$

%Let $\Psi\colon  \P\calE\to \R$ be $$\Psi(x,[v]):= \sup_{\|Y\|\le 1} \Phi\big(Y,(x, [v])\big).$$
%As $\calA$ is tempered, we have for some $C>1,$ and $ k\ge 1$  that  $$|\Psi(x,[v])| \le \log(C) + {k d(x,x_0)}$$ for all $(x,[v])\in \P\calE$.
%In particular, if $|\Psi(x,[v])|\ge M$ then
%$d(x,x_0)\ge (M-\log C)/k.$
As the family $$  \mathcal N= \{\eta_n\} \cup\{\eta_\infty\}$$ has uniformly exponentially small mass in the cusps we have
$$\int e^{\eta d(x,x_0)} d \hat \eta <C$$
 and hence $\hat \eta\{x: d(x,x_0)\ge \ell \} \le Ce^{-\eta\ell}$
for all $\hat \eta \in   \mathcal N$.
It follows for all $\td  \eta\in  \{\td \eta_n\} \cup\{\td \eta_\infty\}$  that---letting $\hat  \eta\in \mathcal N$ denote the image of $ \td  \eta$ in $X$---we have for any $Y\in \lieg$  with $\|Y\|\le 1$ that
%\note{inequalities added and estimate clarified/fixed}
\begin{align*}
	\int_{ \P\calE} &|\Phi(Y, \cdot ) - \Phi_M(Y, \cdot )| \ d  \td  \eta
		\\&= \int_{\{(x,[v]\in \P\calE: d(x,x_0)\ge  M\}} |\Phi (Y, \cdot ) - \Phi_M (Y, \cdot ) | \ d  {\td \eta}
				\\&\le \int_{\{(x,[v]\in \P\calE: d(x,x_0)\ge  M\}}  |\Phi (Y, \cdot ) | \ d  {\td \eta}
		\\&\le \int_{\{x\in X: d(x,x_0)\ge  M\}}  \log(C) + {k d(x,x_0)} \ d \hat  \eta
%	\\& \le (\log C + k M)Ce^{-\eta M}   + \int_{\{x\in X : d(x,x_0)\ge  M\}}  \log(C) + {k d(x,x_0)} \ d \hat  \eta (x)
	\\& \le {\blue  (\log C + k M)Ce^{-\eta M}   + k \int_{ M}^\infty  \hat \eta \{ x:  { d(x,x_0)}\ge \ell\} \  d \ell}
	  \\ &\le (\log C + k M) C e^{-\eta M}  + k\frac{C e^{\eta  (- M)}}{\eta }.
		\end{align*}
%
%{\red \begin{align*}
%	\int_{ \P\calE} &|\Phi(Y, \cdot ) - \Phi_M(Y, \cdot )| \ d  \td  \eta
%		\\&= \int_{\{(x,[v]\in \P\calE: d(x,x_0)\ge  M\}} |\Phi (Y, \cdot ) - \Phi_M (Y, \cdot ) | \ d  {\td \eta}
%				\\&= \int_{\{(x,[v]\in \P\calE: d(x,x_0)\ge  M\}}  |\Phi (Y, \cdot ) | \ d  {\td \eta}
%		\\&= \int_{\{x\in X: d(x,x_0)\ge  M\}}  \log(C) + {k d(x,x_0)} \ d \hat  \eta
%	\\& \le (\log C + k M)Ce^{-\eta M}   + \int_{\{x\in X : d(x,x_0)\ge  M\}}  \log(C) + {k d(x,x_0)} \ d \hat  \eta (x)
%	  \\ &\le (\log C + k M) C e^{-\eta M}  + k\frac{C e^{\eta  (- M)}}{\eta }.
%		\end{align*}
%		}
In particular, given any $\delta>0$, by taking $M>0$ sufficiently large we may ensure that $$\int_{ \P\calE} |\Phi(Y, \cdot ) - \Phi_M(Y, \cdot )| \ d  \td  \eta\le \delta$$ for any   $$\td  \eta\in  \{\td \eta_n\} \cup\{\td \eta_\infty\}.$$

Since   the restriction of $\Phi_M$ to $\{ Y\in \lieg : \|Y\|\le 1\} \times \P\calE$ is compactly supported, it is uniformly continuous whence
$$\int  \Phi_M(Y_n, \cdot ) \ d \td \eta_n - \Phi_M(Y_\infty, \cdot )\ d \td \eta_\infty \to 0$$
as $n\to \infty.$
In particular given $\delta>0$ we may take $M$ and $n$ sufficiently large so that
\begin{align*}
\Big|\int_{ \P\calE}  &\Phi(Y_n, \cdot ) \ d \td \eta_n  - \int _{\P \calE} \Phi(Y_\infty, \cdot )  \ d \td  \eta_\infty\Big|
\\&\le  \int_{ \P\calE}  \left|  \Phi(Y_n, \cdot ) - \Phi_M(Y_n, \cdot ) \right |d \td \eta_n\\
&\quad \quad + \int_{ \P\calE}  \left|   \Phi_M(Y_n, \cdot ) - \Phi_M(Y_\infty, \cdot )\right |d \td \eta_n\\
&\quad \quad +  \int_{ \P\calE}  \left|  \Phi(Y_\infty, \cdot ) - \Phi_M(Y_\infty, \cdot ) \right |d \td \eta_\infty\\
&\le 3\delta.
\end{align*}

Let $g_\infty= \exp (Y_\infty)$.  Note for each $n$ that  $$\int _X\log \|\calA(g_\infty^n, x)\| \ d \eta_\infty(x) \ge  \int _{\P \calE} \log(\left \|\calA(g_\infty^n, x) v \right \| \|v\|\inv) \ d\td  \eta_\infty (x,[v]).$$ It then follows   for any $\delta>0$ %\note{clarified inequalities}
\begin{align*}
 \lambda_{\top, g_\infty, \eta,\calA }%&= \inf _{n\to \infty} \frac 1 n  \int_X \log \|\calA(g^n, x)\| \ d \eta_\infty (x)\\
  & = \lim _{n\to \infty} \frac 1 n  \int_X \log \|\calA(g_\infty^n, x)\| \ d \eta_\infty (x)\\
 &\ge  \liminf _{n\to \infty}  \frac 1 n \int _{\P \calE} \log\left( \left \|\calA(g_\infty^n, x) v \right \| \|v\|\inv \right) \ d\td  \eta_\infty (x,[v])\\
 &=  \liminf _{n\to \infty} \frac 1 n   \int _{\P \calE} \Phi (n Y_\infty , (x,[v]))   \ d\td  \eta_\infty (x,[v])\\
 &=    \int _{\P \calE} \Phi ( Y_\infty , (x,[v]))   \ d\td  \eta_\infty (x,[v])\\
 &\ge   \liminf _{n\to \infty} \int_{ \P\calE}   \Phi(Y_n, \cdot ) \ d \td \eta_n - 3\delta.
\end{align*}
where   the third equality follows from the invariance of $\td \eta_\infty$ and the cocycle property of $\Phi$.  %$L^1$ ergodic theorem.
Since %for any $\delta>0$ we may find $n$ so that
$$\liminf_{n\to \infty} \int_{ \P\calE}   \Phi(Y_n, \cdot ) \ d \td \eta_n \ge \epsilon $$
we conclude that \[\lambda_{\top, g_\infty,\eta,\calA}\ge\epsilon-3\delta\]% \qedhere\]
for any $\delta>0$ whence the result follows.
\end{proof}






%
%\begin{lemma}\note{rewrite for empirical measures of measures rather tthan points}
%\label{lemma:firstexponents}
%Suppose the action of $G$ on $(X,d)$ has uniform displacement and let $\calA\colon G\times \calE\to \calE$ be  tempered continuous cocycle.
%
%
%Let $Y_n\in \lieg$, $x_n \in X$, and  $t_n\ge 0$ be  sequences with  $\|Y_n\|=1$ for all $n$ and $t_n\to \infty$.  Assume that
%\begin{enumerate}\note{when we apply this, we use averages of emperical measures}
%\item the family of empirical measures  $\{\nu_n\} $ defined above has uniformly exponentially small mass in the cusps; and
%\item $\|\calA ( \exp (t_nY_n),x_n) \| \geq e^{\epsilon t_n}$.
%%\item $Y_n \to Y_\infty$ in $\lieg$.
%\end{enumerate}
%Then
%\begin{enumlemma}
%\item \label{lazylemmaa} the family $\{\nu_n\} $ is pre-compact;
%\item \label{lazylemmab}for any subsequential limit $Y_\infty = \lim_{j\to \infty}  Y_{n_j},$ any subsequential limit $\nu_\infty $ of $\{ \nu_{n_j}\}$ is invariant under the 1-parameter subgroup $\{ \exp (tY_\infty): t\in \R\}$;
%\item \label{lazylemmac}$\lambda_{\top,\nu,\calA, \exp (Y_\infty)}\ge\epsilon>0$.
%%converge to an $a^1$ invariant measure $\mu$ on $N^{\alpha}$ whose projection to $\Sl(2,\R)/\Sl(2,\Z)$ is Haar and for which we have $\lambda_{top,\mu, a^1}>0$. \note{Why would the projection be Haar? I don't think this is correct/}
%\end{enumlemma}
%\end{lemma}
%
%\begin{proof}[Proof of Lemma \ref{lemma:firstexponents} \ref{lazylemmaa} and  \ref{lazylemmab}]
%As in the proof of Lemma \ref{claim:jjjiiilllkkk} we have uniform bounds $$\nu_n (\{x: d(x,x_0) \ge \ell\})\le C e^{-\eta \ell}$$ for all $n$
%which, combined with the properness of $d$, establishes uniform tightness of the measure $\nu_n$ and  \ref{lazylemmaa}.
%
%For  \ref{lazylemmab},    let  $\phi \colon X \to \R$ be a compactly supported  continuous function.  Then for any $s>0$
%\begin{align*}
%\int_{X} \phi \circ \exp (sY_\infty) -  \phi \ d \nu_n
%&= \int_{X} \phi \circ \exp (sY_\infty)  -  \phi \circ \exp(s Y_n) \ d\nu_n\\
%&+ \int_{X}  \phi \circ \exp(sY_n) - \phi \ d\nu_n
%\end{align*}
%
%%For fixed $t$, t
%
%The first integral converges to zero as the functions $\phi \circ \exp (wY_\infty)  -  \phi \circ \exp(w Y_n)$ converges uniformly to zero in $n$ for fixed $w$.  The second integral clearly converges to zero  as
%\begin{align*}
%\int_{X}  \phi \circ \exp(s Y_n) - \phi \ d\nu_n&=
%\frac{1}{t_n} \int_0^{t_n} \phi \left( \exp\left((s+t) Y_n\right)  x_n \right) - \phi \left(\exp (tY_n) x_n\right) \ d t\\
%&=
%\frac{1}{t_n}\left[ \int_0^{s} \phi \left( \exp\left(t Y_n\right)  x_n \right) + \int_{t_n}^{t_n+s} \phi \left( \exp\left(t Y_n\right)  x_n \right) \ d t\right]
%\end{align*}
%which converges to 0 as $t_n\to \infty$.
%\end{proof}
%
%
%To prove Lemma \ref{lemma:firstexponents}\ref{lazylemmac} we first introduce a number of auxiliary objects.
% Let $\P \calE\to X$ denote the projectivization of the  tangent bundle    $\calE\to X$.
% We represent a point in $\P\calE$ as $(x,[v])$ where $[v]$ is an equivalence class of non-zero vectors in $\calE$.
% For each $n$, let $[v_n]\in \P\calE_{x_n} $ be such that
%$$\|\calA (x_n, \exp (t_nY_n)) (v_n)\| \geq e^{\epsilon t_n}\|v_n\|.$$
%The $G$ action on $\calE$ by vector-bundle automorphisms induces  a natural $G$-action on $\P \calE$ which restricts to projective transformations between each fiber.    For each $n$, let $\td \nu_n$ be the probability measure  on $\P\calE$ given as follows: given a bounded continuous $\phi\colon \P \calE\to \R$ define
%$$\int _{\P \calE} \phi  \ d \td \nu_n  :=
%\frac{1}{t_n} \int_0^{t_n} \phi \big(\exp (tY_n) \cdot (x_n,  [v_n] )\big ) \ d t.$$
%We have $\td \nu_n$ projects to $ \nu_n$ under the natural projection $\P\calE\to X$; moreover  any weak-$*$ subsequential limit $\td \nu_\infty$ of $\{ \td \nu_{n_j}\}$   projects to $ \nu_\infty$.
%
%
%Define  $\Phi\colon \lieg \times \P\calE\to \R$ by  $$\Phi\big(Y,(x, [v])\big) := \log \left(\left\| \calA\big(\exp
%(Y), x\big)v\right\|\|v\|\inv\right).$$
%Note that $\Phi$ satisfies a cocycle property:
%\begin{equation}\label{eq:cocycle}\Phi\big((s+t)Y,(x, [v])\big)= \Phi\big(tY,(x, [v])\big) +  \Phi\big(sY,\exp(tY)\cdot(x, [v])\big) \end{equation}
%
%By hypothesis, there are $C>1$, $ k\ge 1$ and $\eta>0$ so that $$\int e^{\eta d(x,x_0)} \ d \nu_n \le C$$ for all $n$ and
%$$ \frac{1}{C}  e^{-k d(x,x_0)} \le \left \|\calA(\exp (Y),x)v\right\| \|v\|\inv  \le  C  e^{k d(x,x_0)}$$ for all $(x,[v])\in \P\calE$ and $Y\in \lieg$ with $\|Y\|\le 1.$
%
%Let $$M_n = \sup _{0\le t\le t_n}\left\{d\left( \big(\exp (tY_n) x_n\big),x_0\right)\right\}.$$
%As we assume the  $G$-action on $(X,d)$ has uniform displacement, take $$C_1 = \sup_{\|Y\|\le 1, x\in X} \{d(\exp (Y)\cdot x, x)\}.$$
%As $$ \frac{1}{t_n} \int_0^{t_n} e^{\eta d\left( (\exp (tY_n) x_n,x_0\right)} \ d t= \int e^{\eta d(x,x_0)} \ d \nu_n \le C$$
%it follows if $t_n\ge 1$ that
%$$e^{\eta (M_n -  C_1)}  \le  Ct_n$$ \note{estimate here is a little unclear to AWB}
%whence $$M_n \le \eta\inv( \log  C  + \log t_n )  + C_1 =: \eta\inv \log t_n +  C_2.$$
%Thus since $\|Y_n \| = 1,$\begin{align*}
%\sup _{0\le t\le t_n, 0\le s\le 1} \Phi(sY_n, \exp(t Y_n)\cdot (x_n, [v_n]))
%&\le \log C + k M_n \\
%&\le \log C + k ( \eta\inv \log t_n +  C_2) \\
%&=:   k  \eta\inv \log t_n +  C_3.
%\end{align*}
%In particular, we have
% \begin{align*}\frac{1}{t_n} &\log \|\calA(t_nY_n, x_n) \| \\
% & = \frac{1}{t_n} \Phi(t_nY_n, (x_n,[v_n]))\\
% & = \frac{1}{t_n}\Phi(\lfloor t_n\rfloor Y_n, (x_n,[v_n]))+
% \frac{1}{t_n}\Phi((t-\lfloor t_n\rfloor) Y_n, \exp (\lfloor t_n\rfloor Y_n)\cdot (x_n,[v_n])).
% \end{align*}
% Since
% $$\left|\frac{1}{t_n}\Phi((t-\lfloor t_n\rfloor) Y_n, \exp (\lfloor t_n\rfloor Y_n)\cdot (x_n,[v_n]))\right|\le \frac{1}{t_n}( k  \eta\inv \log t_n +  C_3)$$ goes to 0 as $t_n\to \infty$
% it follows that
% \begin{equation}\label{eq:thiseq}\liminf_{n\to \infty}  \frac{1}{t_n}\Phi(\lfloor t_n\rfloor Y_n, (x_n,[v_n])) = \liminf_{n\to \infty} \frac{1}{t_n} \log \|\calA(t_nY_n, x_n) \| \ge \epsilon >0.\end{equation}
%
%With the above objects and observations we complete the proof of  Lemma \ref{lemma:firstexponents}.
%\begin{proof}[Proof of Lemma \ref{lemma:firstexponents} \ref{lazylemmac}]
%Consider first the expression $\int \Phi (Y_n, \cdot )\  d \td \nu_n.$  We have
%\begin{align*}
%\int \Phi  & (Y_n, \cdot ) \ d \td \nu_n
%\\& =  \frac{1}{t_n}\int_0^{t_n} \Phi  \big(Y_n,\exp (tY_n) \cdot (x_n,  [v_n] )\big ) \ d t
%\\&= \frac{1}{t_n}  \int_0^{\lfloor t_n\rfloor} \Phi    \big(Y_n,\exp (tY_n) \cdot (x_n,  [v_n] )\big ) \ d t
%+ \frac{1}{t_n} \int_ {\lfloor t_n\rfloor} ^{t_n}\Phi    \big(Y_n,\exp (tY_n) \cdot (x_n,  [v_n] )\big ) \ d t
%%\\&= \frac{1}{t_n}        \Bigg[    \Phi   \big(t_n Y_n,  (x_n,  [v_n] )  +    \int_0^{1}   \Phi    \big(-tY_n, \exp (tY_n) \cdot (x_n,  [v_n] )\big ) \ d t \\
%%  &\quad \quad  -   \int_0^{1}   \Phi    \big(-tY_n, \exp ((t+\lfloor t_n\rfloor) Y_n) \cdot (x_n,  [v_n] )\big ) \ d t \Bigg]
%\end{align*}
%Note that the contribution of the second integral is bounded by
%$$\left| \frac{1}{t_n} \int_ {\lfloor t_n\rfloor} ^{t_n}\Phi    \big(Y_n,\exp (tY_n) \cdot (x_n,  [v_n] )\big ) \ d t \right|
%\le \frac{1}{t_n}(k\eta\inv \log t_n +  C_3)$$
%which goes to zero as $t_n\to \infty$.
%
%
%%\\& =  \frac{1}{t_n}\int_0^{1} \Phi  \big(t_nY_n,\exp (tY_n) \cdot (x_n,  [v_n] )\big ) \ d t
% From the cocycle property \eqref{eq:cocycle} of $\Phi(Y_n, \cdot)$ we have  for $t_n\ge 1$ that
%\begin{align*}
%  \frac{1}{t_n}  \int_0^{\lfloor t_n\rfloor} \Phi  & \big(Y_n,\exp (tY_n) \cdot (x_n,  [v_n] )\big ) \ d t
%  \\&= \frac{1}{t_n}  \int_0^{1}   \Phi    \big({\lfloor t_n\rfloor} Y_n, \exp (tY_n) \cdot (x_n,  [v_n] )\big ) \ d t
%  \\&= \frac{1}{t_n}  \int_0^{1}  \Big( \Phi    \big({\lfloor t_n\rfloor} Y_n,  (x_n,  [v_n] )\big )
%  -  \Phi    \big( t Y_n,  (x_n,  [v_n] )\big )
%  \\ & \quad\quad\quad +     \Phi    \big( t Y_n, \exp (\lfloor t_n\rfloor Y_n)\cdot  (x_n,  [v_n] )\big )
%  \Big)
%   \ d t
%   \\& =  \frac{1}{t_n}   \Phi    \big({\lfloor t_n\rfloor} Y_n,  (x_n,  [v_n] )  +
% \frac{1}{t_n}  \int_0^{1}  \Big(    -  \Phi    \big( t Y_n,  (x_n,  [v_n] )\big )  \ dt
%  \\ & \quad\quad\quad +     \Phi    \big( t Y_n, \exp (\lfloor t_n\rfloor Y_n)\cdot  (x_n,  [v_n] )\big )
%  \Big) \ dt
%%
%%
%%  \\&= \frac{1}{t_n}        \Bigg[   \sum _{k = 0}^{\lfloor t_n\rfloor}  \Phi    \big(Y_n,u_n^k   \cdot (x_n,  [v_n] )  +    \int_0^{1}   \Phi    \big(-tY_n, \exp (tY_n) \cdot (x_n,  [v_n] )\big ) \ d t \\
%%  &\quad \quad  -   \int_0^{1}   \Phi    \big(-tY_n, \exp ((t+\lfloor t_n\rfloor) Y_n) \cdot (x_n,  [v_n] )\big ) \ d t \Bigg]
%\end{align*}
%The contribution of the second and third integrals is bounded by
%\begin{align*}
% &\left|\frac{1}{t_n}  \int_0^{1}  \Big(    -  \Phi    \big( t Y_n,  (x_n,  [v_n] )\big )
%      +     \Phi    \big( t Y_n, \exp (\lfloor t_n\rfloor Y_n)\cdot  (x_n,  [v_n] )\big )
%  \Big) \right|
% \\&\quad \quad \quad \quad \le \frac{1}{t_n}  \int_0^{1}  2 ( k  \eta\inv \log t_n +  C_3)\ dt
% \\&\quad \quad \quad \quad = \frac{1}{t_n}    2 ( k  \eta\inv \log t_n +  C_3)
%\end{align*}
%which to zero as $t_n\to \infty$.  We then conclude from \eqref{eq:thiseq} that
%\begin{equation}
%\label{eq:keyless}
%\liminf_{n\to \infty} \int \Phi   (Y_n, \cdot ) \ d \td \nu_n = \liminf_{n\to \infty}    \frac{1}{t_n}   \Phi    \big({\lfloor t_n\rfloor} Y_n,  (x_n,  [v_n] )
%%\liminf  \frac{1}{t_n}\Phi(\lfloor t_n\rfloor Y_n, (x_n,[v_n])
%\ge \epsilon >0.
%\end{equation}
%
%To complete the proof of \ref{lazylemmac}, for  $M>0$     take $\psi_M\colon X\to [0,1]$ with $$ \text{$ \psi_M(x) = 1$ if $d(x,x_0) \le M$ and $\psi_M(x) = 0$ if $d(x,x_0)\ge M+ 1$}.$$
%Let $\Psi_M\colon \P\calE \to [0,1]$ be $$\Psi_M(x,[v]) = \psi_M(x).$$
%and define $\Phi_M \colon \lieg\times \P\calE\to \R$ to be
%$$\Phi_M\big(Y,(x, [v])\big) :=\Psi_M(x, [v]) \Phi\big(Y,(x, [v])\big).$$
%
%%Let $\Psi\colon  \P\calE\to \R$ be $$\Psi(x,[v]):= \sup_{\|Y\|\le 1} \Phi\big(Y,(x, [v])\big).$$
%%As $\calA$ is tempered, we have for some $C>1,$ and $ k\ge 1$  that  $$|\Psi(x,[v])| \le \log(C) + {k d(x,x_0)}$$ for all $(x,[v])\in \P\calE$.
%%In particular, if $|\Psi(x,[v])|\ge M$ then
%%$d(x,x_0)\ge (M-\log C)/k.$
%As the family $$  \mathcal N= \{\nu_n\} \cup\{\nu_\infty\}$$ has uniformly exponentially small mass in the cusps we have
%$$\int e^{\eta d(x,x_0)} d \hat \nu <C$$ and
%%In particular, $\hat \nu\{x: d(x,x_0)\ge \ell \} \le Ce^{-\eta\ell}$
%for all $\hat \nu \in   \mathcal N$.
%It follows for all $\td  \nu\in  \{\td \nu_n\} \cup\{\td \nu_\infty\}$  that---letting $\hat  \nu\in \mathcal N$ denote the image of $ \td  \nu$ in $X$---we have for any $Y\in \lieg$  with $\|Y\|\le 1$ that
% \begin{align*}
%	\int_{ \P\calE} &|\Phi(Y, \cdot ) - \Phi_M(Y, \cdot )| \ d  \td  \nu
%		\\&= \int_{\{(x,[v]\in \P\calE: d(x,x_0)\ge  M\}} |\Phi (Y, \cdot ) - \Phi_M (Y, \cdot ) | \ d  {\td \nu}
%				\\&= \int_{\{(x,[v]\in \P\calE: d(x,x_0)\ge  M\}}  |\Phi (Y, \cdot ) | \ d  {\td \nu}
%		\\&= \int_{\{x\in X: d(x,x_0)\ge  M\}}  \log(C) + {k d(x,x_0)} \ d \hat  \nu
%	\\&=(\log C + k M)Ce^{-\eta M}   + \int_{\{x\in X : d(x,x_0)\ge  M\}}  \log(C) + {k d(x,x_0)} \ d \hat  \nu (x)
%	  \\ &\le (\log C + k M) C e^{-\eta M}  + k\frac{C e^{\eta  (- M)}}{\eta }.
%		\end{align*}
%
%In particular, given any $\delta>0$ by taking $M>0$ sufficiently large, we may ensure that $$\int_{ \P\calE} |\Phi(Y, \cdot ) - \Phi_M(Y, \cdot )| \ d  \td  \nu\le \delta$$ for any   $$\td  \nu\in  \{\td \nu_n\} \cup\{\td \nu_\infty\}.$$
%
%Since   there restriction of $\Phi_M$ to $\{ Y\in \lieg : \|Y\|\le 1\} \times X$ is compactly supported, it is uniformly continuous whence
%$$\int  \Phi_M(Y_n, \cdot ) \ d \td \nu_n - \Phi_M(Y_\infty, \cdot )\ d \td \nu_\infty \to 0$$
%as $n\to \infty.$
%In particular given $\delta>0$ we may take $M$ and $n$ sufficiently large so that
%\begin{align*}
%\Big|\int_{ \P\calE}  &\Phi(Y_n, \cdot ) \ d \td \nu_n  - \int _{\P \calE} \Phi(Y_\infty, \cdot )  \ d \td  \nu_\infty\Big|
%\\&\le  \int_{ \P\calE}  \left|  \Phi(Y_n, \cdot ) - \Phi_M(Y_n, \cdot ) \right |d \td \nu_n\\
%&\quad \quad + \int_{ \P\calE}  \left|   \Phi_M(Y_n, \cdot ) - \Phi_M(Y_\infty, \cdot )\right |d \td \nu_n\\
%&\quad \quad +  \int_{ \P\calE}  \left|  \Phi(Y_\infty, \cdot ) - \Phi_M(Y_\infty, \cdot ) \right |d \td \nu_\infty\\
%&\le 3\delta.
%\end{align*}
%
%Let $g_\infty= \exp (Y_\infty)$.  Note for each $n$ that  $$\int _X\|\calA(g_\infty^n, x)\| \ d \nu_\infty(x) \ge  \int _{\P \calE} \left \|\calA(g_\infty^n, x) v \right \| \|v\|\inv \ d\td  \nu_\infty (x,[v]).$$ It then follows   for any $\delta>0$
%\begin{align*}
% \lambda_{\top,\nu,\calA, g_\infty}%&= \inf _{n\to \infty} \frac 1 n  \int_X \log \|\calA(g^n, x)\| \ d \nu_\infty (x)\\
%  & = \lim _{n\to \infty} \frac 1 n  \int_X \log \|\calA(g_\infty^n, x)\| \ d \nu_\infty (x)\\
% &\ge  \liminf _{n\to \infty}  \frac 1 n \int _{\P \calE} \log\left( \left \|\calA(g_\infty^n, x) v \right \| \|v\|\inv \right) \ d\td  \nu_\infty (x,[v])\\
% &=  \liminf _{n\to \infty}    \int _{\P \calE} \Phi ( Y_\infty , (x,[v]))   \ d\td  \nu_\infty (x,[v])\\
% &\ge   \liminf _{n\to \infty} \int_{ \P\calE}   \Phi(Y_n, \cdot ) \ d \td \nu_n - 3\delta.
%\end{align*}
%where   the second equality follows from the $L^1$ ergodic theorem.
%Since %for any $\delta>0$ we may find $n$ so that
%$$\liminf_{n\to \infty} \int_{ \P\calE}   \Phi(Y_n, \cdot ) \ d \td \nu_n \ge \epsilon $$
%we conclude that \[\lambda_{\top,\nu,\calA, g_\infty}\ge\epsilon-3\delta\]% \qedhere\]
%for any $\delta>0$ whence the result follows.
%\end{proof}




%%%%%%%%%%%%%%%%% END AARON %%%%%%%%%%%%%%%%%%%%

\subsection{Oseledec's theorem for cocycles over actions by higher-rank abelian groups}
Let $A\subset G$ be a split Cartan subgroup.  Then $A\simeq \R^d$ where $d$ is the rank of $G$.
We have the following consequence of the higher-rank Oseledec's multiplicative ergodic theorem  (c.f.\ \cite[Theorem 2.4]{AWB-GLY-P1}).

Fix any norm $| \cdot |$ on $A\simeq \R^d$ and let $\eta\colon X\to \R$ be $$\eta(x) := \sup_{|a|\le 1} \log \| \calA(a, x)\|.$$
\begin{proposition}
\label{thm:higherrankMET}Let $\mu$ be an ergodic, $A$-invariant Borel probability measure on $X$ and
suppose $\eta\in L^{d,1}(\mu)$.  Then there are
	\begin{enumerate}
	\item an $\alpha$-invariant subset $\Lambda_0\subset X$ with $\mu(\Lambda_0)=1$;
%	\item a measurable function $r\colon \Lambda_0\to \N$,
  \item % a  family of measurable
   linear functionals $\lambda_i\colon A \to \R$ for $1\le i\le p$;  % for $1\le i\le r(x)$,
	\item   and splittings   $\calE(x)= \bigoplus _{i=1}^p E_{\lambda_i}(x)$ % $\R^k$
	into families of mutually transverse,  $\mu$-measurable  subbundles $E_{\lambda_i}(x)\subset \calE(x)$ defined  for $x\in \Lambda_0$

	\end{enumerate}
such that
\begin{enumlemma}	
	\item $\calA (s, x) E_{\lambda_i}(x)= E_{\lambda_i}(s\cdot x)$ and
	\item \label{lemma:partb} $\displaystyle \lim_{|s|\to \infty} \frac { \log \|  \calA (s,x) (v)\| - \lambda_i(s)}{|s|}=0$
\end{enumlemma}	
	for all $x\in \Lambda_0$ and all $ v\in  E_{\lambda_i}(p)\sm \{0\}$.  %and all $s\in A$.
 \end{proposition}
Note that \ref{lemma:partb} implies for $v\in E_{\lambda_i}(x)$ the weaker result that for $s\in A$,
$$\lim_{k\to\pm \infty} \tfrac {1} k \log \|  \calA (s^k,x) (v)\| =  \lambda_i(s).$$
Also note that for $s\in A$, and $\mu$ an $A$-invariant, $A$-ergodic measure that
\begin{equation}\label{eq:lameducksoup}\lambda_{\top, s, \mu,\calA} = \max _i \lambda_i(s).\end{equation}
If $\mu$ is not $A$-ergodic, we have the following.
\begin{claim} \label{claim:defenestratethepresident}
Let $\mu$ be an $A$-invariant measure with $\eta\in L^{d,1}(\mu)$ and $\lambda_{\top, s, \mu,\calA}>0$ for some $s\in A$. Then there is an $A$-ergodic component $\mu'$ of $\mu$ with
\begin{enumerate}
\item $\eta\in L^{d,1}(\mu')$;
\item there is non-zero   Lyapunov exponent $\lambda_j\neq 0$ for the $A$-action on $(X, \mu').$
\end{enumerate}
\end{claim}



We have the following which follows from the above definitions.
\begin{lemma}\label{lem:huntthelameduck}
Let $\mu$ be an $A$-invariant  probability  measure on $X$ with exponentially small mass in the cusps.
Suppose that $\calA$ is a tempered cocycle.
Then $\eta\in L^q(\mu)$ for all $q\ge 1$.   In particular, $\eta\in L^{d,1}(\mu)$.
\end{lemma}
%\begin{proof}
%
%\end{proof}




\subsection{Applications to the suspension action}
\label{subsection:repackagingpreliminaries}

We summarize the  previous  discussion in the setting in which we will apply the above results in the sequel.   Recall we work with in a    fiber bundle with compact fiber $$M \rightarrow M^{\alpha}=(G \times M)/\Gamma \xrightarrow{\pi} G/\Gamma$$
 over non-compact
base $G/\Gamma$.
From  the discussion in  \cite[Section  2.1]{AWBFRHZW-latticemeasure}, we may equip $G\times M$ with a $C^1$ metric that is \begin{enumerate}
\item  $\Gamma$-invariant;
\item the restriction to $G$-orbits coincides with the fixed right-invariant metric on $G$;
\item there is a Siegel fundamental set $D\subset G$ on which the restrictions to the fibers of the metrics are uniformly comparable.
\end{enumerate}
The metric then descends to a $C^1$ Riemannian metric on $M^\alpha$.  We fix this metric for the remainder.  It follows that the diameter of any fiber of $M^\alpha$ is uniformly bounded.
It then follows that if $\mu$ is a measure on $M^\alpha$ then the image $\nu= \pi_* \mu$ in $G/\Gamma$ has  exponentially small mass in the cusps if and only if $\mu$ does; {\blue moreover, a family $\{\mu_\zeta\}$ of probability measures on $M^\alpha$ has  uniformly exponentially small mass in the cusps if and only if the family of projected  measures  $\{\pi_*\mu_\zeta\}$ on $G/\Gamma$ does.}
 Note that by averaging the metric over the left-action of $K$, we may also assume that the metric is left-$K$-invariant.  This, in particular, implies the right-invariant metric on $G$ in $(2)$ above is left-$K$-invariant.

{\blue For the remainder, the cocycle of interest will be the fiberwise derivative cocycle on the fiberwise tangent bundle, $$\calA(g,x)\colon F\to F,\quad \calA(g,x)= \restrict{D_xg}{F}.$$
Given  $g\in G$ and a $g$-invariant probability measure on $M^\alpha$,
the average leading   Lyapunov exponent for the fiberwise derivative cocycle for translation by $g$  is written either as $\lambda^F_{\top,\mu,g}$
or as  $\lambda_{\top,\mu,g, \calA}$.}

%{\red say something about the cocycle is the fw derivative}

% but we work more generally with
%$$D \rightarrow E \xrightarrow {\pi} B$$
%\noindent a fiber bundle where $D$ is compact and $B$ and therefore $E$ are not.
%We let $\mathcal E$ be a vector bundle over $M^\alpha$ and  $\calA\colon G \times \calE \rightarrow \calE$ be a tempered cocycle over the $G$-action on $M^\alpha$.
%It is    straightforward to check  that if $\mu$ is a measure on $E$ and $\nu= \pi_* \mu$
%then $\mu$ has exponentially small mass in the cusp with exponent $\eta_0$ iff
%$\nu$ does.

%The following is a simple consequence of the work in the previous sections.
%
%\begin{lemma}\note{this lemma requires that $\{F_n \ast \mu\}$ have uniformly  exponentially small mass in the cusps}
%\label{lemma:fiberaveraging}
%For $s\in G$, let $\mu$ be an $s$-invariant measure on $M^\alpha$ with exponentially small mass in the cusps and let $\nu= \pi_* \mu$.
%For any amenable $H\subset C_G(s)$, if $\nu$ is $H$-invariant then for any regular \Folner sequence $F_n$ in $H$ and any limit measure
%$\mu'$ of $\{F_n \ast \mu\}$, we have $\lambda_{\top, \mu,s} \le \lambda_{\top, \mu',s}$.
%\end{lemma}
%
%\begin{proof}
%This is an immediate consequence of Lemma \ref{lemma:averaging}.
%\end{proof}

The next observation we need is a variant of a fairly standard observation about cocycle
over the suspension action.

\begin{lemma}
\label{lemma:tempered}
The fiberwise  derivative cocycle $\restrict{D_x g}{F}$ is tempered.
\end{lemma}


\begin{proof}
%\note{added sentence addressing Referee's 14}
{\blue Write $\pi\colon M^\alpha \to G/\Gamma$.  By the construction of the metric in the fibers of $M^\alpha$ there is a $C>0$ with the following properties: given  $x\in M^\alpha$ and $g\in G$, writing $\bar x = \pi (x) \in G/\Gamma$ we have $$\| \restrict{D_x g}{F}\| \le C^{\beta(g,\bar x)+1}$$
and
$$m(\restrict{D_x g}{F})\ge C^{-\beta(g,\bar x)-1}.$$}
The conclusion is then an immediate consequence of Lemma \ref{lemma:fromlmr}. %Note that this depends on our choice of fundamental domain.
\end{proof}

We now assemble the consequences of the results in this section in the form we will use them below in a pair of lemmas.
The first is just a special case of Corollary \ref{corollary:finite}.

\begin{lemma}
\label{lemma:finite}
Let $s\in A$ and let $\nu$ be an $s$-invariant measure on $G/\Gamma$ with exponentially small mass in the cusps.  Let $\mu$ be an $s$-invariant measure on $M^\alpha$ projecting to $\nu$.  Then the average leading   Lyapunov exponent for the fiberwise derivative cocycle, $\lambda^F_{\top,\mu,s},$ is finite.
\end{lemma}

The second lemma summarizes the above abstract results in the setting of $G$ acting on $M^\alpha$.  % what we will need for averaging arguments in our proofs.


%\note{clarified part (2)}
\begin{lemma}
\label{lemma:averagingsuspension}
Let $s\in A$ and let $\nu$ be an $s$-invariant measure on $G/\Gamma$ with exponentially small mass in the cusps.  % in the cusps with parameter $\eta_0$.
 Let $\mu$ be an $s$-invariant measure on $M^\alpha$ projecting to $\nu$.
\begin{enumerate}
\item \label{pla}For any amenable subgroup $H\subset C_G(s)$, if $\nu$ is $H$-invariant then
\begin{enumlemma}
\item  \label{playa} for any   \Folner sequence  of precompact sets $F_n$ in $H$,
%such that
the family $\{F_n \ast \mu\}$  has uniformly exponentially small mass in the cusps; and \item \label{playb} for any subsequential limit
$\mu'$ of $\{F_n \ast \mu\}$ we have
  $$\lambda_{\top, s, \mu}^F \le \lambda_{\top, s, \mu'}^F.$$ %\note{Should not be $\mathcal A$}
\end{enumlemma}
\item  %\note{this lemma requires that $\{F_n \ast \mu\}$ have uniformly  exponentially small mass in the cusps}
For any one-parameter unipotent subgroup $U$ centralized by $s$
%and any   \Folner sequence consisting of sets of the form $U_{T}$ in $U$
\begin{enumerate}
\item the family
 $\{U^T \ast \mu\}$ has uniformly exponentially small mass in the cusps;  and \item for any accumulation point   $\mu'$ of $\{U^T \ast \mu\}$  as $T\to \infty$ we have
 %, $\mu'$ has exponentially small mass in the cusps with parameter $\eta_0$ and
$$\lambda_{\top, s, \mu} ^F\le \lambda_{\top,s, \mu'}^F.$$
\end{enumerate}
\end{enumerate}
\end{lemma}


\begin{proof}
%\note{corrected refs from proof of (1)(b)}
Part \ref{playa} of the first conclusion is immediate since $H$-invariance of $\nu$ implies $\nu=\pi_*(F_n\ast \mu)$ for all $n$;  %hypothesis since $F_n\ast \mu$ has constant projection to $\Sl(m,\R)/\SL(m,\Z)$;
part \ref{playb} then follows from % Lemma \ref{lemma:translates},
Lemma \ref{lemma:averaging}.
The second conclusion follows from  Proposition \ref{prop:bananas} %  Lemma \ref{lemma:translates},
and Lemma \ref{lemma:averaging}.  %Both conclusions use Lemmas \ref{lemma:tempered} and Lemma \ref{lemma:finite}.
\end{proof}

We remark that we will also use Lemma \ref{lemma:firstexponents} in the proof of the main theorem, but we do not reformulate a special case of it here since the reformulation adds little clarity.

%{\blue I'm not sure what the benefit of stating Lemmas 2.14, 2.9, and 2.17(a) is.  I think it is clearer to just prove Lemma 2.15 and then summarize facts in Lemma 2.17 (of course with the missing hypothesis.}

\section{Subexponential growth of derivatives for unipotent elements}
\label{unipotents}

In this section we show that the restriction of the action $\alpha$ to certain unipotent elements in each copy  $\Lambda_{i,j} \cong \Sl(2, \Z)$ have  uniform  subexponential growth of   derivatives with respect to a right-invariant distance on $\Sl(2, \R)$.  Note that each $\Sl(2,\R)$ is geodesically embedded whence the   $\Sl(2,\R)$ distance is the same as  the $\Sl(m, \R)$ distance.  By  \cite{MR1244421, MR1828742},  the $\Sl(m,\R)$ distance is  quasi-isometric to the word-length in $\Sl(m,\Z)$.   Recall   that $d(\cdot,\cdot)$ denotes a right-invariant distance  on $\Sl(m, \R)$ and that $\Id$ is the identity in $\Sl(m,\R)$.

For $ 1\leq i < j \neq n$, let $\Lambda_{i,j} \cong \Sl(2, \Z)$ be the copy of $\Sl(2, \Z)$ in $\Sl(m,\Z)$ corresponding to the elements in $\Sl(m,\Z)$ which acts only on the lattice $\Z^2 < \Z^m$ generated by $\{e_i,e_j\}$.
Note that as all $\Lambda_{i,j}$ are conjugate under the Weyl group, it suffices to work with one of them.
%\noted{A: Why lower case? D: Matrices are upper case, vectors are lower case. Pretty sure that's standard.  We can repeat the other definition in terms of matrices instead if you prefer.}


Define the unipotent element $u := \begin{bmatrix} 1 & 1 \\ 0 & 1 \end{bmatrix}$ viewed as an element of $\Lambda_{i,j}$.  Note that any upper or  lower triangular unipotent element of $\Lambda_{i,j}$ is  conjugate to a power of $u$  under the Weyl group.

\begin{proposition}[Subexponential growth of derivatives for unipotent elements]
\label{unipotentisgood} For any $\Lambda_{i,j}$ and any $\e > 0$, there exists $N_{\e}>0$ such that for any $n \geq N_\e$: $$\|D(\alpha(u^n))\| \leq e^{\e d(u^n,\Id)}$$
\end{proposition}

To establish Proposition \ref{unipotentisgood}, we first show that generic elements in $\Sl(2, \Z)$ have uniform subexponential growth of derivatives.  This first part requires reusing most of the key arguments from \cite{BFH} in a slightly modified form.  We encourage the reader to read that paper first.



%
%{\blue
%\subsection*{Slow growth for ``most" elements in $\Sl(2,\Z)$}
%\label{subsection:slowgrowthgeneric}
%For $\epsilon, k >0$ and $x \in \Sl(2, \R)$, we make the following definitions:
%
%\begin{enumerate}
%
%\item Let $B_k(x)$ denote the ball of center $x$ and radius $k$ in $\Sl(2, \R)$. Let $|B_k|$ denote the volume of $B_k(x)$.
%\item Let $T_k := B_k(\Id) \cap \Sl(2, \Z)$.
%\item The set of $\e$-bad elements $M_{\e, k} := \{ g \in T_k \text{ such that }   \|D(\alpha(g))\| \geq e^{\e k }\}$
%\item The set of $\e$-good elements $G_{\e,k} := T_k \setminus M_{\e, k} $\\
%
%\end{enumerate}
%
%First we will show that the set $G_{\e,k}$ contains a positive proportion of $T_k$ if $k$ is large enough.  The next proposition
%is well known.
%
%\begin{proposition}\label{basic} There exists  positive constants $c, C$ such that for any $k \geq0$: $$ c|B_k| \leq |T_k| \leq C|B_k|.$$
%\end{proposition}
%
%\begin{proof}
%Observe that the volume of the ball $B_k$ is less than the area of the modular surface multiplied by the number of elements of $T_k$, this implies the lower bound  for $|T_k|$. For the upper bound, take a small ball $U$ in $\Sl(2,\R)/\Sl(2,\Z)$ and consider the lifts of $U$ to $\Sl(2,\R)$.
%{\red Counting the total volume of the lifts of $U$ in $\Sl(2,\R)$ inside $B_k$ we obtain the desired bound.} \note{If we are going to bother to prove this, lets not be half-assed about it.  Most of the balls hit the boundary i think}
%\end{proof}
%
%For an element $x \in \Sl(2,\R)$, let $\bar{x}$ be the projection in $\Sl(2,\R)/\Sl(2,\Z)$. Define
%\def\fib{\text{Fiber}}
%$$\|D_{\bar{x}} g\|_{\text{Fiber}} = \sup_{ \pi(y) = \bar{x}} \|D_y g\|$$
%
%\noindent and
%
%$$G'_{\e,k}(x) := \{ g \in B_k(\bar x) \text{ such that } \|D_{\bar{x}}g\|_{\text{Fiber}} \leq e^{\e d(g,\\Id)}\}$$
%
%%{\sout \begin{lemma}\label{uni1} There exists $x \in \Sl(2,\R)/\Sl(2,\Z)$ such that for any $\delta > 0$, if $k$ is large enough $|G'_{\e, k}(x)| > (1- \delta) |B_k|$
%%\end{lemma}}
%
%\begin{lemma}\label{uni1} For almost every  $x \in \Sl(2,\R)$ and  any $\delta > 0$ we have   $$|G'_{\e, k}(x)| > (1- \delta) |B_k|$$ for all    $k$ sufficiently  large.
%\end{lemma}
%\begin{proof}
%
%Fix the function $\psi(x) := e^{\eta d(x,x_0)}$ in $\Sl(2,\R)/\Sl(2,\Z)$, where $x_0$ is some arbitrary point in $\Sl(2,\R)/\Sl(2,\Z)$, and $\eta$ is small enough that $\psi$ is integrable with respect to Haar measure.  Let $$a^t :=  \begin{bmatrix} e^t&0\\0&e^{-t} \end{bmatrix}.$$ As the action of the one-parameter diagonal subgroup $\{a^t\}$  on $ \Sl(2,\R)/\Sl(2,\Z)$ is ergodic with respect to Haar measure, by Birkhoff's ergodic theorem for almost every  $x \in \Sl(2,\R)$ and almost every  $k_1 \in \SO(2)$ we have
%$$ \lim_{n \to \infty} \frac{1}{n} \int_{0}^n \psi(a^tk_1\bar{x}) \ dt = \int_{\Sl(2,\R)/\Sl(2,\Z)} \psi \ d\text{Haar} < \infty$$
%
%For fixed $\delta>0$  there exists  $n_{\delta}= n_\delta(x)$ and a set $K_{\delta} = K_\delta(x) \subset \So(2)$ such that \note{define this} {\red $|K_{\delta}| \geq (1-\delta/2)|\So(2)|$} with the property that  for any $k_1 \in K_ {\delta}$ and any $T \geq n_{\delta}$ we have
%
%\begin{equation}\label{tight}
%\frac{1}{T} \int_{0}^T \psi(a^tk_1\bar{x}) \ dt < 2\int_{\Sl(2,\R)/\Sl(2,\Z)} \psi \ d\text{Haar}
%\end{equation}
% The set of $x\in \Sl(2,\R)$ in the Lemma are the   Birkhoff generic points $x$ for which    \eqref{tight} holds.  We fix such a distinguished $x$ for the remainder.
%
% Define the set $$G''_{k}(x) := \{ g \in B_k \text{ such that } g= k_1a^tk_2 \text{ where } k_1 \in K, k_2 \in K_{\delta}(x) \text{ and } \delta' k <t < k\}$$
% where $\delta'>0$ is a fixed small number guaranteeing that $|G''_{k}(x)| \geq (1- \delta)|B_k|$.  Such a $\delta'$ exists because most of the mass of a ball in $\Sl(2,\R)$ is near the boundary sphere. {\red This looks like a bs explaination to me.  We need to justify this better.}
%
% To finish the proof of the lemma, we claim   that \begin{equation}\label{eq:okeqn}G''_{k}(x) \subset G'_{\e,k}(x)\end{equation} for $k$ sufficiently large.     For the sake of contradiction, suppose \eqref{eq:okeqn} fails.  Then there exist  $x_n \in \Sl(2,\R)$ with $x_n \in \SO(2)x$ such that $\|D_{x_n}(a^{t_n})\|_\fib \geq e^{\e t_n}$ for some sequence $t_n \to \infty$ and such that the corresponding {\red empirical measures} \note{wft} $\mu_n$ have exponential small mass in the cusps with parameter $\eta$ by equation \eqref{tight}.
%
%
%{\red By ............}   The measures $\mu_n$ converge to an $a^t-$invariant measure $\mu_0$ on $M^{\alpha}$ {\red w.t.f. is $M^\alpha$?} whose projection to $\Sl(m, \R)/\Sl(m,\Z)$ is Haar measure on $\Sl(2,\R)/\Sl(2, \Z)$ and which has positive fiberwise Lyapunov exponent {\red  by ........}. The fact that the projection is Haar derives from our generic choice of $x$. {\red bs, we  need to choose $x$ more carefully in order to guarantee this.  it only equidistributes with respect to sebastians function.}  The fact that the fiberwise Lyapunov exponent is positive is exactly the content of Lemma \ref{lemma:firstexponents}.  Since $a^t$ is ergodic on on $\Sl(2,\R)/\Sl(2, \Z)$, we can assume $\mu_0$ is ergodic by taking an ergodic component without changing any other properties.
%
%From here, we average as in \cite{BFH} to improve $\mu_0$ to a measure whose projection is Haar measure on all of $\Sl(m, \R)/\Sl(m,\Z)$.  The additional difficulties related to escape of mass are all handled by the the preliminaries in Section \ref{section:preliminaries}.
%
% As above, we note that there is a canonical copy of $H_2 = \Sl(m-2, \R)$ in $\Sl(m,\R)$ commuting with our chosen $H_1= \Sl(2,R)$. Fix $A$ to be a Cartan subgroup of $\Sl(m,\R)$ containing the one-parameter  group $\{a^t\}$, we let $A_1 = A \cap H_1 = \{a^t\}$ and $A_2 = A \cap H_2$ and $A' = A \cap H_1 \times H_2$. Note that $A' <A$ has codimension one.  Our chosen modular surface $\Sl(2,\R)/\Sl(2,\Z) \subset \Sl(m,\R)/\Sl(m,\Z)$ can be chosen so that $\Sl(2,\R)/\Sl(2,\Z) \subset \Sl(2,\R)/\Sl(2,\Z) \times \Sl(m-2 ,\R)/\Sl(m-2,\Z) \subset \Sl(m,\R)/\Sl(m,\Z)$.  We now define
% a measure $\mu_1$ on $M^{\alpha}$ that projects to Haar measure on $\Sl(2,\R)/\Sl(2,\Z) \times \Sl(m-2 ,\R)/\Sl(m-2,\Z)$, that is $A'$ invariant and $A'$ ergodic on $M^{\alpha}$ and that has positive first Lyaponuv exponent for $A_1$.  To do this we look at the restriction $M^{\alpha}_{m-2}$ of $M^{\alpha}$ to $\Sl(m-2,\R)/\Sl(m-2, \Z)$ and let $\mu'$ be any $A_2$ invariant, $A_2$ ergodic measure on  $M^{\alpha}_{m-2}$ which projects to Haar on $\Sl(m-2,\R)/\Sl(m-2, \Z)$.  Letting $\mu_1 = \mu_0 \times \mu_2$ supported on the subset of $M^{\alpha}$ defined by restricting the bundle to $\Sl(2,\R)/\Sl(2,\Z) \times \Sl(m-2 ,\R)/\Sl(m-2,\Z)$ produces the desired measure.
%
% We are now in a position to use Proposition \ref{proposition:averaging}.  On $A'$ we have a positive fiberwise Lyapunov exponent $\lambda$ and we would like to average over a unipotent direction frozen by some element $a_0 \in A'$ where $\lambda(a_0)>0$.  The only issue that can occur is if $\lambda$ is proportional to the root corresponding to the unipotent.  But in Proposition \ref{proposition:averaging}, at each step we have two choices of root and corresponding subgroup to work with and the linear functionals they define are not proportional on $A'$, so only one of the two can be proportional to $\lambda$.  We choose our first unipotent $U^{\beta'}$ accordingly to average over, choosing $a_0$ such that $a_0 \in \ker(\beta')$ and $a_0 \notin ker(\lambda)$.  This produces $\mu_2$ which has is invariant under $a_0$, has positive fiberwise Lyapunov exponent for $a_0$ and where $\Stab(\pi(\mu_2) > \Sl(2, \R) \times \Sl(m-2, \R)$ and also all unipotents corresponding to positive roots for $A$ by combining Lemma \ref{lemma:averagingsuspension} and Proposition \ref{proposition:averaging} and which has exponentially small mass in the cusps.  We may also assume $\mu_2$ is ergodic by passing to an ergodic component. \note{Do we want to state a lemma that says you can always pick an ergodic component with exponentially small cusps and positive exponent.  it is kinda obvious, but we prove easier things.}  To be able to use the second half of Proposition \ref{proposition:averaging}, we once again average $\mu_2$ over $A'$ to obtain $\mu_3$ which retains positive exponent for $a_0$ and exponentially small mass in the cusps by the first part of Lemma \ref{lemma:averagingsuspension}.   Once again we can pass to an $A'$ ergodic component of $\mu_2$ that retains the desired properties.
%
% The second step of averaging from Proposition \ref{proposition:averaging} works much as the first, using the choice of $\hat \beta$ described there and an appropriate (but possibly different) choice of $a_1$ with $\lambda(a_1)>0$ but $\hat \beta(a_1)=0$.  By Proposition \ref{proposition:averaging} and Lemma \ref{lemma:averagingsuspension}, we obtain a new measure $\mu_3$ where $\pi(\mu_3)$ is Haar measure on $\Sl(m, \R)/\Sl(m,Z)$.
% Lastly,we can now average $\mu_3$ over all of $A$ to obtain $\mu_4$, since now $\pi(\mu_4)$ is Haar and therefore $A$ invariant.  Again by Lemma \ref{lemma:averaging} $\mu_3$ and $\mu_4$ still have positive fiberwise exponents for $a_1$.  We note that $\pi(\mu_4)$ is $A$ ergodic and
%replace $\mu_4$ by an ergodic component with positive fiberwise exponent.
%
% Now arguing exactly as in \cite[Section 5.5]{BFH}, we can now use \cite[Proposition 5.1]{AWBFRHZW-latticemeasure} to show that $\mu_4$ is $G$ invariant on $M^{\alpha}$ which contradicts Zimmer's cocycle superrigidity theorem. To see that the argument applies, note that \cite[Proposition 5.1]{AWBFRHZW-latticemeasure} is proven for all lattices and not just cocompact ones and observe that at this phase of the argument we are only showing $\mu_r$ is invariant under various root subgroups corresponding to {\em non-resonant} roots  using \cite[Proposition 5.1]{AWBFRHZW-latticemeasure} and that at this phase of the argument we never change $\mu_4$.  The main point of the argument in \cite[Section 5.5]{BFH} is that when the dimension of $M$ is low enough, there are enough non-resonant roots of $\Sl(m,\R)$ to generate the group. Note that exactly as in \cite[Section 5.5]{BFH}, the argument works in dimension at most $m-2$ unless the action is volume preserving, in which case, it works in dimension at most $m-1$.
%
%\end{proof}
%
%
%As a consequence we have the following:
%
%\begin{corollary}\label{mainunipo} For any $\delta > 0$, if $k$ is large enough, the set $G_{\e,k}$ has at least $(1- \delta)|T_k|$ elements.
%\end{corollary}
%
%\begin{proof}
%{\red This proof is complete nonsense}
%We will use the same notation as in Proposition \ref{uni1}. Observe that the point $x$ in the
%proof of Proposition \ref{uni1} can be chosen as close to the coset of the identity as one wants. Let $c := d(x, \Id)$. Observe that if $g \in G''_{k, \epsilon/4}(x)$, then $gx \in G''_{k,\e/2}$ and therefore  $G''_{k,\e/4}(x) \subset G''_{k + c,\e/2}(\Id)$. This implies that for any $\delta>0$ and $k$ large enough we have  $|G''_{\e/2, k}(\Id)| > (1- \delta) |B_k|$.
%Then, one can argue as in \ref{basic}. Take a ball $U$ around the identity in $\Sl(2,\R)/\Sl(2,\Z)$ and look at all the lifts intersecting the ball $B_k$. If one of these lifted
%balls intersect $G''_{\e/2, k}(\Id)$, then the corresponding element  in $\Sl(2, (\Z)$ belongs to $G_{\e,k}$. As the proportion of such balls intersecting $G''_{\e/2, k}(\Id)$ must go
%to one if $k$ is large enough, the result follows.
%
%\end{proof}
%
%\begin{remark} Using large deviations, one can possibly make $\delta$ to be decreasing on $k$ so roughly $\delta_{k} = e^{-k^{1/1000}} $, but this is not necessary for the argument.  See \cite{MR2247652, MR2787598}
%\end{remark}
%
%}
%



\subsection{Slow growth for ``most" elements in $\Sl(2,\Z)$}
\label{subsection:slowgrowthgeneric}
For $\epsilon>0$, $k >0$, and $x \in \Sl(2, \R)$, we make the following definitions:

\begin{enumerate}
\item For $S\subset \Sl(2,\R)$  let $|S|$ denote the Haar-volume of $S$.
\item Let $K = \So(2) \subset \Sl(2,\R)$.  For $S\subset  K$   let $|S|$ denote the Haar-volume of $S$.
\item Let $B_k(x)$ denote the ball of radius $k$ centered at $x$ in $\Sl(2, \R)$.
\item Let $T_k := B_k(\Id) \cap \Sl(2, \Z)$.  Given $S\subset \Sl(2,\Z)$ write $|S|$ for the cardinality of $S$.
\item Define the  set of $\e$-bad elements to be $$M_{\e, k} := \{ \gamma \in T_k \text{ such that }   \|D(\alpha(\gamma))\| \geq e^{\e k }\}.$$
\item Define the set of $\e$-good elements to be $$G_{\e,k} := T_k \setminus M_{\e, k} .$$

\end{enumerate}

To establish Proposition \ref{unipotentisgood}, we first  show that the set $G_{\e,k}$ contains a positive proportion of $T_k$ when $k$ is large enough.
%\note{moved from corollary later to a proposition here since this is the main point of the subsection}
\begin{proposition}\label{mainunipo} For any $\delta > 0$,  the set $G_{\e,k}$ has at least $(1- \delta)|T_k|$ elements for every sufficiently large $k$. %\note{need something here about the change of norms.}
\end{proposition}



We have the following  well-known fact.  {\blue See for instance \cite[Section 2]{MR1230290}.}%\note{added EM ref.  The lower bound is actually subtle and uses exp mixing.} %. and removed nonsense proof.  Please check.}

\begin{lemma}\label{basic} There exist  positive constants $c, C$ such that for any $k \geq0$: $$ c|B_k| \leq |T_k| \leq C|B_k|.$$
\end{lemma}


%\begin{proof}\note{sentence 1 is obviously false}
%{\blue
%We recall a basic property of hyperbolic geometry: given $R>0$ there is a $\hat C>0$ such that for all $k$ sufficiently large,
%$$|B_{k+R}|\le \hat C |B_k|$$
%
%
%For the lower bound, take a compact $D$ with containing a neighborhood of $\id$ in a Dirichlet domain  $\mathcal D$ for the identity and let $\wtd D = D\cdot \Gamma$.  There is $L>1$ so that $|B_k| \le L |B_k \cap \wtd D$ for all $k$.  }
%
%
%{\red Given a fundamental domain $\calF$ for $\Gamma$ for $\Sl(2,\Z)$ in $\Sl(2,\R)$ we have $B_k \subset \calF cdot T_k$. Therefore the volume of the ball $B_k$ is less than the area of the modular surface multiplied by the number of elements of $T_k$. This implies the lower bound  for $|T_k|$.  GARBAGE} For the upper bound, take $U$ to be an open ball  of radius $0<r<1$ centered at the identity coset in $\Sl(2,\R)/\Sl(2,\Z)$.  Consider  lifts of $U$ to $\Sl(2,\R)$. There is some $\hat C>1$ such that  $$|B_{k+1}|\le \hat C |B_k|$$ for all $k$.
% Counting the total volume of the lifts of $U$ to $\Sl(2,\R)$ that intersect $B_k$ and comparing to the volume of $B_{k+1}$, we obtain the upper bound.
%\end{proof}



%\begin{proof}\note{sentence 1 is obviously false}
%{\red Given a fundamental domain $\calF$ for $\Gamma$ for $\Sl(2,\Z)$ in $\Sl(2,\R)$ we have $B_k \subset \calF cdot T_k$. Therefore the volume of the ball $B_k$ is less than the area of the modular surface multiplied by the number of elements of $T_k$. This implies the lower bound  for $|T_k|$.  STILL WRONG} For the upper bound, take $U$ to be an open ball  of radius $0<r<1$ centered at the identity coset in $\Sl(2,\R)/\Sl(2,\Z)$.  Consider  lifts of $U$ to $\Sl(2,\R)$. There is some $\hat C>1$ such that  $$|B_{k+1}|\le \hat C |B_k|$$ for all $k$.
% Counting the total volume of the lifts of $U$ to $\Sl(2,\R)$ that intersect $B_k$ and comparing to the volume of $B_{k+1}$, we obtain the upper bound.
%\end{proof}

For an element $x \in \Sl(2,\R)$, let $\bar{x}$ denote the projection in $\Sl(2,\R)/\Sl(2,\Z)$. Define
\def\fib{\text{Fiber}}
$$\|D_{\bar{x}} g\|_{\text{Fiber}} = \sup\{ \|\restrict{D_y g}{F}\| :y\in M^\alpha,  \pi(y) = \bar{x}\} .$$
Let $$G'_{\e,k}(x) := \{ g \in B_k( x) \text{ such that } \|D_{\bar{x}}g\|_{\text{Fiber}} \leq e^{\e d(g,\Id)}\}.$$



%{\sout \begin{lemma}\label{uni1} There exists $x \in \Sl(2,\R)/\Sl(2,\Z)$ such that for any $\delta > 0$, if $k$ is large enough $|G'_{\e, k}(x)| > (1- \delta) |B_k|$
%\end{lemma}}

\begin{lemma}\label{uni1} For almost every  $x \in \Sl(2,\R)$ and  any $\delta > 0$ we have   $$|G'_{\e, k}(x)| > (1- \delta) |B_k|$$ for all    $k$ sufficiently  large.
\end{lemma}\label{lem:aegood}
\def\calM{\mathcal M}
\begin{proof}
 Let $a^t\in \Sl(2,\R) $ be the matrix $$a^t :=  \begin{bmatrix} e^t&0\\0&e^{-t} \end{bmatrix}.$$ Recall that  the action of the one-parameter diagonal subgroup $\{a^t\}$  on $ \Sl(2,\R)/\Sl(2,\Z)$ is ergodic with respect to Haar measure.


%\note{fixed}
{\blue Let $\calM$ denote the set of Borel probability measures on $\Sl(2,\R)/\Sl(2,\Z)$ equipped with the standard topology (dual to bounded  continuous functions).  The topology on $\calM$ is metrizable (see \cite[Theorem 6.8]{MR1700749});  fix a metric on $\rho_\calM$ on $\calM$. } %\note{Aaron should add reference
% to Billingsley and clarify which weak$*$ this is.}



Consider the function $\psi\colon \Sl(2,\R)/\Sl(2,\Z)\to \R$ given by  $\psi(x) := e^{\eta d(x,x_0)}$ where $x_0=   \Sl(2,\Z)$ is the identity coset and $\eta>0$ is chosen sufficiently small so that $\psi$ is $L^1$ with respect to the Haar measure.
 By the pointwise ergodic theorem, for almost every  $x \in \Sl(2,\R)$ and almost every  $k_1 \in \SO(2)$ we have
\begin{equation}\label{eq:assin8thepresident} \lim_{T \to \infty} \frac{1}{T} \int_{0}^T \psi(a^tk_1\bar{x}) \ dt = \int_{\Sl(2,\R)/\Sl(2,\Z)} \psi \ d\text{Haar} < \infty.\end{equation}
Similarly, for almost every  $x \in \Sl(2,\R)$ and almost every  $k_1 \in \SO(2)$ we have  \begin{equation}\label{eq:grabtrumpbythepussy}\lim_{T \to \infty}  \frac{1}{T} \int_{0}^T \delta_{a^tk_1\bar{x}}  \ dt =  \text {Haar}.\end{equation}
Let $S\subset \Sl(2,\R)$ be the set of  $x\in \Sl(2,\R)$ such that \eqref{eq:assin8thepresident} and  \eqref{eq:grabtrumpbythepussy} hold for almost every $k_1 \in \So(2)$.  The set $S$ is $\Sl(2,\Z)$-invariant and co-null.  We show any $x\in S$ satisfies the conclusion of the lemma.

For fixed $x\in S$ and fixed $\delta>0$,  there exist   $T_{\delta}= T_\delta(x)$, a sequence $T_j = T_j (x)$ for $j\in \N$,  and a set $K_{\delta} = K_\delta(x) \subset \So(2)$ such that  $|K_{\delta}| \geq (1-\delta/2)|\So(2)|$ with the property that  for any $k_1 \in K_ {\delta}$ and any $T \geq T_{\delta}$ we have

\begin{equation}\label{tight}
\frac{1}{T} \int_{0}^T \psi(a^tk_1\bar{x}) \ dt < 2\int_{\Sl(2,\R)/\Sl(2,\Z)} \psi \ d\text{Haar}
\end{equation}
and for each $1\le j$
\begin{equation}\label{tight2}
\rho_\calM \left(\frac{1}{T} \int_{0}^{T}  \delta_{a^tk_1\bar{x}}  \ dt, \text{Haar}\right)< \frac{1}{j}.
\end{equation}
\noindent for all $T \geq T_j$.
 To finish the proof of the lemma, define the set $$G''_{k}(x) := \{   k_1a^tk_2 \text{ where } k_1 \in \So(2), k_2 \in K_{\delta}(x) \text{ and } (\delta/2) k <t < k\}.$$
 For $k$ large enough, we have that  $|G''_{k}(x)| \geq (1- \delta)|B_k|$.
   We claim   that \begin{equation}\label{eq:okeqn}G''_{k}(x) \subset G'_{\e,k}(x)\end{equation} for $k$ sufficiently large.     For the sake of contradiction, suppose \eqref{eq:okeqn} fails.  Using that the norm on $F$ is chosen to be $K$-invariant, there exists  $x_n \in \Sl(2,\R)$ with each $x_n $ in the  $K_\delta(x)$-orbit of $x$ such that $\|D_{x_n}(a^{t_n})\|_\fib \geq e^{\e t_n}$ for some sequence $t_n \to \infty$.  Moreover, the corresponding {empirical measures} $$\eta_n:=\frac{1}{t_n} \int_{0}^{t_n}  \delta_{a^t\bar{x}_n}  \ dt$$ have uniformly exponentially small mass in the cusps by equation \eqref{tight}.%\note{Figure out where to say we take a $K$ invariant metric on the bundle.}



By Lemma \ref{lemma:firstexponents} and  \eqref{tight2}, a subsequence of the measures $\eta_n$ converge to an $a^t$-invariant measure $\mu_0$ on $M^{\alpha}$  whose projection to $\Sl(m, \R)/\Sl(m,\Z)$ is Haar measure on the embedded modular surface $\Sl(2,\R)/\Sl(2, \Z)$ and has positive fiberwise Lyapunov exponent for the action of $a^1$. %The fact that the projection is Haar follows from \eqref{tight2}.  %The fact that the fiberwise Lyapunov exponent is positive is exactly the content of Lemma \ref{lemma:firstexponents}.
Since $a^t$ is ergodic on   $\Sl(2,\R)/\Sl(2, \Z)$, we can assume $\mu_0$ is ergodic by taking an ergodic component without changing any other properties.

We average as in \cite{BFH} to improve $\mu_0$ to a measure whose projection is the Haar measure on  $\Sl(m, \R)/\Sl(m,\Z)$.  Difficulties related to escape of mass are   handled by the   preliminaries in Section \ref{section:preliminaries}.

 As above, we note that there is a canonical copy of $H_2 = \Sl(m-2, \R)$ in $\Sl(m,\R)$ commuting with our chosen $H_1= \Sl(2,\R)$. {\blue Recall $A$ is the Cartan subgroup of $\Sl(m,\R)$ of positive diagonal matrices.  %\noted{clarified}
 The subgroup $A$ contains the one-parameter  group $\{a^t\}$ and a Cartan subgroup of $H_2$.} Let
 \begin{itemize}\item $A_1 = A \cap H_1 = \{a^t\}$,\item  $A_2 = A \cap H_2$, and \item $A' = A \cap H_1 \times H_2$. \end{itemize}Note that $A' <A$ has codimension one.  Our chosen modular surface $\Sl(2,\R)/\Sl(2,\Z) \subset \Sl(m,\R)/\Sl(m,\Z)$  is such that \begin{align*}\Sl(2,\R)/\Sl(2,\Z) &\subset \Sl(2,\R)/\Sl(2,\Z) \times \Sl(m-2 ,\R)/\Sl(m-2,\Z)\\& \subset \Sl(m,\R)/\Sl(m,\Z).\end{align*}

%\note{Non-trivial rearrangement follows here re: referee comment 23.}
 {\blue Define an $A'$-ergodic,  $A'$-invariant  measure $\mu_1$ on $M^{\alpha}$ that projects to Haar measure on $\Sl(2,\R)/\Sl(2,\Z) \times \Sl(m-2 ,\R)/\Sl(m-2,\Z)$ as follows: Let $M^{\alpha}_{2,m-2}$ denote the  restriction of the fiber-bundle  $M^{\alpha}$ to $\Sl(2, \R)/\Sl(2,\Z) \times \Sl(m-2,\R)/\Sl(m-2, \Z)$.  Pick point $y$ in $\Sl(m-2,\R)/\Sl(m-2, \Z)$ that equidistributes to the Haar measure on $\Sl(m-2,\R)/\Sl(m-2, \Z)$ under a \Folner sequence in $A_2$.  Consider $\mu_0$ as a measure on the restriction of $M^{\alpha}$ to $\Sl(2, \R)/\Sl(2,\Z) \times \{y\}$.  Now average $\mu_0$ over a \Folner   sequence  in $A_2$ and take a limit $\hat \mu_1$.  Note that  $\hat \mu_1$ has positive fiberwise Lyapunov exponent  $ \lambda_{\top, a^1,\mu_1}^F>0$. This can be seen by  mimicking the proof of   Lemma \ref{lemma:averaging}. Let $\mu_1$ be an $A'$ ergodic component of $\hat \mu_1$, then the measure $\mu_1$ has the desired properties and is  supported on the subset of $M^{\alpha}$ defined by restricting the bundle to $\Sl(2,\R)/\Sl(2,\Z) \times \Sl(m-2 ,\R)/\Sl(m-2,\Z)$.
}


We consider the $A'$-action on $(M^\alpha, \mu_1)$ and the fiberwise derivative cocycle $\calA(g,y) = \restrict{D_y g}{F}$.
By \eqref{eq:lameducksoup}, there is a   non-zero  Lyapunov exponent
$\lambda_{ \mu_1,A'}^F\colon A'\to \R$ for this action.   We apply the averaging procedure in   Proposition \ref{proposition:averaging} to this measure.
%    we have a positive fiberwise {\red average/ or top}  Lyapunov exponent $\lambda$ {\red lets not be dicks/use consistent notation here} and we would like to average over a unipotent direction frozen by some element $a_0 \in A'$ where $\lambda(a_0)>0$.  The only issue that can occur is if $\lambda$ is proportional to the root corresponding to the unipotent.  But in Proposition \ref{proposition:averaging}, at each step we have two choices of root and corresponding subgroup to work with and the linear functionals they define are not proportional on $A'$, so only one of the two can be proportional to $\lambda$.
    Take  $\beta'$ to be either $\alpha_2$ or $\delta$ so that $\beta'\colon A'\to \R$ is not proportional to $\lambda_{ \mu_1,A'}^F$.
Choose $a_0\in A'$ such that $a_0 \in \ker(\beta')$ and $\lambda_{ \mu_1,A'}^F(a_0)>0$.
Let $U = U^{\beta'}$ and let $\mu_2$ be any subsequential limit  of $U^T\ast \mu_1$ as $T \rightarrow \infty$. Then
 $\mu_2$  is  $a_0$-invariant, and has positive fiberwise Lyapunov exponent $ \lambda_{\top, a_0, \mu_2}^F >0$.  Moreover,
 $\pi_*\mu_2$ is $H$-invariant.
% and where $\Stab(\pi(\mu_2)) > \Sl(2, \R) \times \Sl(m-2, \R)$ and also all unipotents corresponding to positive roots for $A$ b
By Lemma \ref{lemma:averagingsuspension}  and Proposition \ref{proposition:averaging}, $\mu_2$ has exponentially small mass in the cusps.
We may also assume $\mu_2$ is ergodic by passing to an ergodic component and by Claim \ref{claim:defenestratethepresident} assume $\mu_2$ has a non-zero fiberwise  Lyapunov exponent $\lambda_{ \mu_2,A'}^F$ for the $A'$-action.







%  To be able to use the second half of Proposition \ref{proposition:averaging}, we once again
 We now average $\mu_2$ over $A'$ to obtain $\mu_3$.   Then $\mu_3$ has a non-zero fiberwise  Lyapunov exponent $\lambda_{\mu_3,A'}^F$ and has  exponentially small mass in the cusps by   Lemma \ref{lemma:averagingsuspension}(\ref{pla}).   Since $\pi_*\mu_2$ was $A'$-invariant, we have  $\pi_*\mu_2= \pi_*\mu_3$.  Once again, we may pass to an $A'$-ergodic component of $\mu_3$ that retains the desired properties.

Take $\hat \beta$ to be either $-\alpha_2$ or  $-\delta$ so that $\hat \beta$ is not proportional to $\lambda_{\mu_3,A'}^F$ on $A'$.
Select $a_1$ with $\lambda_{\mu_3,A'}^F(a_1)>0$ and  $\hat \beta(a_1)=0$.  By Proposition \ref{proposition:averaging} and Lemma \ref{lemma:averagingsuspension}, we obtain a new measure $\mu_4$ with  $\pi_*\mu_4$ the Haar measure on $\Sl(m, \R)/\Sl(m,Z)$.  We have $ \lambda_{\top, a_1,\mu_4}^F >0$.
Finally, average $\mu_4$ over all of $A$ to obtain $\mu_5$.  Since $\pi_*\mu_4$ is the Haar measure and thus   $A$-invariant, we have that $\pi_*\mu_4= \pi_*\mu_5$.
By Lemma \ref{lemma:averagingsuspension},      $\mu_5$ has a non-zero fiberwise Lyapunov exponent $\lambda_{\mu_5,A}^F$ for the action of $A$.  Replace  $\mu_5$ by an ergodic component with positive fiberwise Lyapunov exponent.

Exactly as in \cite[Section 5.5]{BFH}, we apply \cite[Proposition 5.1]{AWBFRHZW-latticemeasure} and conclude that  $\mu_5$ is a $G$-invariant measure on $M^{\alpha}$.  We then obtain a contradiction with  Zimmer's cocycle superrigidity theorem.
To conclude that $\mu_5$ is a $G$-invariant, note that \cite[Proposition 5.1]{AWBFRHZW-latticemeasure} holds for actions induced from actions of any lattice in $\Sl(m,\R)$ and shows that   $\mu_5$ is invariant under   root subgroups corresponding to \emph{non-resonant} roots.   Dimension counting exactly as in   \cite[Section 5.5]{BFH}  shows that the  non-resonant roots of $\Sl(m,\R)$  generate all of $G$ if  the  dimension of $M$ is at most $m-2$ or if the dimension of $M$ is $m-1$ and  the action is  preserves a  volume.
\end{proof}


%As a consequence we have the following:
%
%\begin{corollary}\label{mainunipo} For any $\delta > 0$, if $k$ is large enough, the set $G_{\e,k}$ has at least $(1- \delta)|T_k|$ elements. %\note{need something about the change of norms.}
%\end{corollary}
%We derive Proposition \ref{mainunipo} from Lemma \ref{uni1}.
%\begin{proof}[Proof of Proposition \ref{mainunipo}]
%Fix $0<c<1$ sufficiently small so that if $d(\Id,g)<c$ then $\|D_{\Gamma} g\|_{\text{Fiber}}\le e^{\epsilon/4}$.
%Fix a point $x\in \Sl(2,\R)$  as in   Lemma \ref{uni1} with $d(\Id, x)<c$.
%Observe that if $k\ge 1$ and $g \in G''_{k, \epsilon/4}(x)$, then $gx \in G'_{k+c,\e/2}(e)$.
%In particular, for any $\delta>0$  we have  for all $k$ sufficiently large that %$|G'_{\e/2, k+c}(\Id)| > (1- \delta) |B_k|$ for all $k$ sufficiently large.
%\begin{equation}\label{eq:sendninjastodecapitateTrumpinhissleep} |B_{k+c}\sm G'_{\e/2, k+c}(\Id)| < \delta \hat C |B_k|\end{equation} where $\hat C$ is a constant depending on $c$.
%
%Take  $U$ to be the ball of radius $c$ centered at  the identity coset in $\Sl(2,\R)/\Sl(2,\Z)$ and consider   lifts of $U$ to $\Sl(2,\R)$ intersecting the ball $B_k$. If a lift of $U$ intersects $G'_{\e/2, k+ c}(\Id)$, then the corresponding element of the deck group $\Sl(2, \Z)$ belongs to $G'_{3\e/4,k}(\id)$.
%
%From Lemma  \ref{basic} and \eqref{eq:sendninjastodecapitateTrumpinhissleep}, the ratio of lifts  of $U$ in $B_k$ that intersect $G''_{\e/2, k}(\Id)$ to all lifts of $U$ in $B_k$ goes
%to one as $k\to \infty$.  Finally, since the norms on the fiberwise of $M^\alpha$ above the identity coset and the original norm on $M$ are uniformly comparable,  the   result follows.
%\end{proof}


We derive Proposition \ref{mainunipo} from Lemma \ref{uni1}.
\begin{proof}[Proof of Proposition \ref{mainunipo}]
Fix $0<c<1$ sufficiently small so that if $d(\Id,g)<c$ then $\|D_{\Gamma} g\|_{\text{Fiber}}\le e^{\epsilon/4}$.
Fix a point $x\in \Sl(2,\R)$  as in   Lemma \ref{uni1} with $d(\Id, x)<c$.
Observe that if $k\ge 1$ and $g \in G'_{ \epsilon/4,k}(x)$, then $gx \in G'_{\e/2,k+c}(\Id)$.
In particular, for any $\delta>0$  we have  for all $k$ sufficiently large that %$|G'_{\e/2, k+c}(\Id)| > (1- \delta) |B_k|$ for all $k$ sufficiently large.
\begin{equation}\label{eq:sendninjastodecapitateTrumpinhissleep} |B_{k+c}\sm G'_{\e/2, k+c}(\Id)| < \delta \hat C |B_k|\end{equation} where $\hat C$ is a constant depending on $c$.

Take  $U$ to be the ball of radius $c$ centered at  the identity coset in $\Sl(2,\R)/\Sl(2,\Z)$ and consider   lifts of $U$ to $\Sl(2,\R)$ intersecting the ball $B_k$. If a lift of $U$ intersects $G'_{\e/2, k+ c}(\Id)$, then the corresponding element of the deck group $\Sl(2, \Z)$ belongs to $G'_{3\e/4,k}(\id)$.



{\blue Let $\tilde U$ be the set of lifts of $U$.  From Lemma  \ref{basic} and \eqref{eq:sendninjastodecapitateTrumpinhissleep}, it follows
that ratio of the measure of $\tilde{U} \cap B_k \cap  G'_{\e/2, k}(\Id)$ to the measure of $\tilde{U} \cap B_k$ goes to one as $k \to \infty$.}
%{\blue the fraction of all of lifts of $U$ contained in $B_k$ that intersect $G'_{\e/2, k}(\Id)$ goes to one as $k\to \infty$.}
Finally, since the norms on the fiber of $M^\alpha$ above the identity coset and the original norm on $M$ are uniformly comparable,  the   result follows.
\end{proof}


\begin{remark} Using large deviations, one can   make $\delta$ to be decreasing with $k$,  roughly as $\delta_{k} = e^{-k^{1/1000}} $.   See \cite{MR2247652, MR2787598}.  This is not necessary for our argument.
\end{remark}


\subsection{Subexponential growth of derivatives for unipotent elements in $\Sl(2,\Z)$}\label{sec:mutualmastication}

We work here with a specific copy of the group $\Sl(2,\R)\ltimes\R^2$ embedded in $\Sl(m,\R)$ and its intersection with the lattice $\Gamma$; the copy of $\Sl(2,\R)\ltimes\R^2$ corresponds to the elements of $\SL_m(\R)$ which differ from the identity matrix only in  the first two rows and  first three columns.
Any   unipotent element of   any $\Lambda_{i,j}\subset \Gamma$ considered in   the statement of Proposition \ref{unipotentisgood} is conjugate by an element of the  Weyl group to a power of the elementary matrix $E_{1,3}$.  Thus, after conjugation, any such element is contained in the  distinguished     copy of  $\Sl(2,\Z) \ltimes \Z^2$ generated by  $\Sl(2,\Z) = \Lambda_{1,2}$ and the normal subgroup $\Z^2$   generated by $E_{1,3}$ and $E_{2,3}$.

 For the reminder of this subsection, we   work with this fixed group. Identify $H_{1,2}$ with $\Sl(2,\R)$. Let $U_{1,2}:=  \{ u_{a,b}\}$ denote the abelian subgroup of $\Sl(m,\R)$ consisting of unipotent elements of the form % whose entries are zero off the diagonal \note{why are we saying the same thing twice?} except the  $(1,3)$ and $(2,3)$ entries.  % which are equal to $a,b \in \R$.  I.e. let
%That is, let
$$
u_{a,b}:= \left(\begin{array}{ccccc}1  &  0  & a  &  &   \\   0 & 1  &   b  & &   \\   0 & 0 & 1 &   &   \\  &   &   &  \ddots &  \\  &   &   &   &1  \end{array}\right)$$
%and let $U_{1,2}= \{ u_{1,b}\}.$
Clearly, $U_{1,2}$ is normalized by $H_{1,2}$ and  $H_{1,2} \ltimes U_{1,2}  \cong \Sl(2,\R) \ltimes \R^2$.    We have an embedding $$ \Sl(2,\R) \ltimes \R^2 / \Sl(2,\Z)\ltimes \Z^2 \to \Sl(m,\R)/\Sl(m,\Z)$$ where $\Z^2$ is identified with the subgroup generated by the unipotent elements $u_{1,0}$ and $u_{0,1}$. Note that $\Sl(2,\R) \ltimes \R^2 / \Sl(2,\Z)\ \ltimes \Z^2$ is a torus bundle over the unit-tangent bundle of the modular surface.

Equip $\Z^2$ with the $L_\infty$ norm with respect to the generating set $\{u_{1,0}, u_{0,1}\}$ and let $B_n(\Z^2)$ denote the closed ball of radius $n$ in $\Z^2$ centered at $0$ with respect to this norm.  Given $S\subset \Z^2$ let   $|S|$ denote the cardinality of the set  $S$.

Define the set of ``$\epsilon$-good unipotent  elements'' of $\Z^2\subset \Gamma$, denoted  by $GU_{\e,n}$, to be the following subset of $\Z^2$:
\begin{equation}\label{eq:racistPOTUS} GU_{\e,n} := \left\{ u_{a,b} \in B_n (\Z^2) \text{ such that } \|D(\alpha(u_{a,b}^{\pm1}))\| \leq e^{\e \log(n)}\right\}.\end{equation}

The main results of this subsection is the following.
\begin{proposition}\label{finalunipotent} For any $\e>0$, there exists $N_\e > 0$ such that if $n \geq N_{\e}$, then $GU_{\e,n} = B_n(\Z^2)$
\end{proposition}

Proposition \ref{unipotentisgood} follows from Proposition \ref{finalunipotent} using that any  subgroup $\langle u^n \rangle$ in Proposition \ref{unipotentisgood} is conjugate to a subgroup of the group $\Z^2$ and  the fact that  $d(u^n, \Id) = O(\log(n))$ from \eqref{unipotentgrowth}. %{\red This should be stated clearly somewhere.  We should define ``big-O'' if we leave it in.  Also we should probably %use it correctly if we leave it in}
The proof of Proposition \ref{finalunipotent} consists of conjugating elements of $U_{1,2}$ by elements of $G_{\e,n}$ in order to obtain a subset of $G_{\e,n}$ that contains  a positive density  of elements of $B_n(\Z^2)$.  Then, using the fact that $\Z^2$ is abelian, we  promote such a subset to all of $B_n(\Z^2)$  by taking sufficiently large sumsets  in Proposition \ref{babycombinatorics}.

%{\blue \begin{proposition}\label{conjuga}
%There exists $N_{\e}$ such that if $n \geq N_{\e}$ then for any  $A = \begin{bmatrix} a&b\\c&d \end{bmatrix}   \in G_{\e,n}$,
%the unipotents $u_{a,b}, u_{c,d}$ and its inverses  are in $GU_{3\e,n}$ for $n$ large enough.
%\end{proposition}}


%{\blue
%The following fact is elementary and is left to the reader:
%
%\begin{claim}\label{normdistance} Let $\|.\|$ be any matrix norm in $\Gl(2, \R)$, then there exists a constant $C > 0$ such that for any $A \in \Sl(2, \R)$ we have: $${\red 2\log \|A\| - C \leq d(A, \Id) \leq  2\log \|A\| + C}$$ \note{do we choose a conanonical metric or just an arbitrary one?}
%
%\end{claim}}




%{\blue
%\begin{lemma}\label{proportion}
%There is $\delta' > 0$ ({\red not depending on $\e$}) and $N_{\e}>0$ such that for $n \geq N_{\e}>0$ large enough $|GU_{\e,n}| \geq \delta' |B_n(\Z^2)|$\note{do we actually care about independence in $\e$?}
% \end{lemma}
%
%
%\begin{proof}
%
%Let $k := 2\log(n)$. Take $s$ to be a fixed number large enough so that for $k$ sufficiently large we have $|T_{k-s}| < 1/2|T_{k}|$. Let us define the set $$S_k := G_{\e,k}\cap (T_k \setminus T_{k-s})$$ From Proposition \ref{mainunipo}, we can assume that $$|S_{k}| \geq 1/4 |T_k|$$
%
%By Proposition \ref{normdistance}, there exists $C_1>0$ such that if $A = \begin{bmatrix} a&c\\b&d \end{bmatrix}$ belongs to $S_k$ then either $\|(a,b)\|_\infty \geq C_1e^{\frac{k-s}{2}}$ or $\|(c,d)\|_\infty \geq C_1e^{\frac{k-s}{2}}$. By simplicity we will assume that at least half of the elements of $S_k$ satisfy that $\|(a,b)\|_\infty \geq C_1e^{\frac{k-s}{2}}$.
%
%Now consider the map $P: S_{k} \to \Z^2$ given by sending  $A = \begin{bmatrix} a&c\\b&d \end{bmatrix}$ to $(a,b)$. By proposition \ref{normdistance} there exists $C_2 > 0$ such that the image $P(S_k)$ lies in the ball $B_{C_2e^{\frac{k}{2}}}(\Z^2) = B_{C_2n}(\Z^2)$ and by Proposition \ref{conjuga} if $(a,b) \in P(S_k)$, then $u_{a,b} \in GU_{(3\e,n)}$ , moreover the volume of $B_{n}(\Z^2)$ is of the order of $n^2$, but also $|S_k| \geq \frac{1}{4}|T_k| = O(e^k) = O(n^2)$. So to conclude the proof we need to show that the preimage $P^{-1} \{(a,b)\}$ consists of a bounded number of elements (depending on $s$).
%
%Observe that if $A,A' \in S_k$  satisfy $P(A) = P(A')$, then $A' = AU$, where $U = \begin{bmatrix} 1&m\\0&1 \end{bmatrix}$ for some $m \in \Z$.
%Therefore $$A' =   \begin{bmatrix} a&am+ c\\b&bm + d \end{bmatrix}.$$
%
%If $A'$ belongs to $T_k$ then $\|(am +c, bm+d)\|_\infty \leq C_2e^{\frac{k}{2}}$, and that $\|(c, d)\|_\infty \leq C_2e^{\frac{k}{2}}$. This imply that $|m|$ must be less or equal than $3\frac{C_2}{C_1}e^{\frac{s}{2}}$. Therefore the preimage $P^{-1}(a,b)$ has at most $6\frac{C_2}{C_1}e^{\frac{s}{2}} +1$ elements. Therefore ${\red |GU_{3\e,n}| = O(|S_k|) = O(n^2)}$ and we are done.\note{Its not clear to me that ``big-O''  gives a uniform $\delta'$ in $\epsilon$.  Also big-O is useless for lower bounds.}
%
%\end{proof}
%}


%Note that $G_{\e,n}$ was not defined to be symmetric.  However, taking sufficiently large

%%
%%\begin{claim}\label{conjuga}\note{This claim is wrong but not needed.  just state what we need in the next proof? I don't understand why this is wrong. DF.}\noted{The `n' in $G$ and $GU$ are log-related}
%%For any $\epsilon>0$, there exists $N_{\e}$ such that if $n \geq N_{\e}$ and if  $A = \begin{bmatrix} a&c\\b&d \end{bmatrix}   \in G_{\e,n}\cap G_{\e,n}\inv$ then  %and $A\inv \in G_{\e,n}$  then
%%the unipotent elements $u_{a,b}, u_{c,d}\in B_{e^n}(\Z^2)$ are contained in $GU_{3\e,e^n}$.
%%\end{claim}
%%
%%\begin{proof}
%%For $u_{a,b}$, we have that $$\alpha (u_{a,b})  =\alpha (A) \circ \alpha (u_{1,0})\circ  \alpha (A\inv)$$
%%whence $\|D \alpha (u_{a,b}) \| \le \|D\alpha (u_{1,0})\| e^{3\epsilon n}. $
%%%{\blue Follows by conjugating the unipotents $u_{1,0}$ and $u_{0,1}$ by $A$  and using the chain rule.}
%%\end{proof}
%%
%%
%%
%%\begin{lemma}\label{proportion}
%%There exists $\delta' > 0$ with the following properties: for any   $\e>0$ there is an $N_{\e}'>0$ such that for any $n \geq N_{\e}'$ we have  $$|GU_{\e,n}| \geq \delta' |B_n(\Z^2)|.$$
%% \end{lemma}
%%
%%
%%\begin{proof}
%%Recall that $T_k$ denotes the intersection of the ball of radius $k$ in $\Sl(2,\R)\simeq H_{1,2}$ with $\Sl(2,\Z)= \Lambda_{1,2}$ and $|T_k|$ denotes the cardinality of $T_k$.  As $|T_k|$ grows exponentially in $k$, we may take
%%%Let $k := 2\log(n)$.
%% $s$ fixed  so  that $|T_{k-s}| < \frac{1}{2}|T_{k}|$ for all $k$ sufficiently large. Given $\epsilon '>0$, define the subset $S_k\subset \Sl(2,\Z)$ to be   $$S_k := G_{\e',k}\cap G_{\e',k}\inv \cap  (T_k \setminus T_{k-s}).$$ From Proposition \ref{mainunipo}, we may  assume that $$|S_{k}| \geq \frac{1}{2} |T_k|$$
%%
%%From   \eqref{normdistance}, there exists $C_1>0$ such that if $A = \begin{bmatrix} a&c\\b&d \end{bmatrix}$ belongs to $S_k$ then either $$\text{$\|(a,b)\|_\infty \geq C_1e^{ \frac{1}{2}{(k-s)} }$ or $\|(c,d)\|_\infty \geq C_1e^{\frac{1}{2}{(k-s)} }$.}$$ With out loss of generality,   we     assume that at least half of the elements in $S_k$ satisfy   $\|(a,b)\|_\infty \geq C_1e^{ \frac{1}{2}{(k-s)} }$.
%%
%%Consider the map $P\colon S_{k} \to \Z^2$ that assigns     $A = \begin{bmatrix} a&c\\b&d \end{bmatrix}$ to $(a,b)$. By  \eqref{normdistance},  there is $C_2>1$ such that  the image $P(S_k)$ of $S_k$ lies in the norm-ball $B_{C_2e^{\frac{k}{2}}}(\Z^2)$ for all $k$.
%%
%% Let $k(n) = 2\log (n)$. Then $P(S_{k(n)}) \subset B_{C_2n}(\Z^2)$ and,   by Claim \ref{conjuga}, if  $n$ is sufficiently large  we have $u_{a,b} \in GU_{(3\e',n)}$ whenever $A = \begin{bmatrix} a&c\\b&d \end{bmatrix} \in S_{k(n)}$.
%%We have $|B_{  C_2 n}(\Z^2)|\le D_1 n^2$ for some $D_1\ge 1$.  Also, from \eqref{decapitationOrImpeachment?} and Lemma \ref{basic} we have   $|S_{k(n)}| \geq \frac{1}{2}|T_{k(n)}| \ge  \frac 1{D_2}e^{k(n)} = \frac 1{D_2}n^2$ for some $D_2\ge 1$.
%%
%%To to complete the proof, we show that the preimage $P^{-1} ((a,b))$ of any $(a,b)\in \Z^2$ has uniformly bounded cardinality  depending only on  $s$.
%% Observe that if $A,A'\in \Sl(2,\Z)$  satisfy $P(A) = P(A')$, then $A' = AU$, where $U = \begin{bmatrix} 1&m\\0&1 \end{bmatrix}$ for some $m \in \Z$ and we have
%% $$A=   \begin{bmatrix} a& c\\b&  d \end{bmatrix},\quad \text{and} \quad A' =   \begin{bmatrix} a&am+ c\\b&bm + d \end{bmatrix}.$$
%%If $A'$ belongs to $T_k$ then $\|(am +c, bm+d)\|_\infty \leq C_2e^{\frac{k}{2}}$ and if $A$ belongs to $T_k$ then $\|(c, d)\|_\infty \leq C_2e^{\frac{k}{2}}$.
%%We thus have   that $|am|\le 2 C_2 e^k$ and  $|bm|\le 2 C_2 e^k$.
%%As we assume that $$\|(a,b)\|_\infty \geq C_1e^{ {k-s} }$$ we have that  $|m|\le  2\frac{C_2}{C_1}e^{s}$.  Thus, the preimage $P^{-1}(a,b)$ has  at most $4\frac{C_2}{C_1}e^{s} +1$ elements.
%%
%%
%%With $\epsilon'=\frac 1 3 \epsilon$, having taken $n$ sufficiently large, we thus have
%%\begin{align*}\frac{|GU_{\e,n}|}{|B_n(\Z^2)|} \ge \frac{|GU_{\e,n}|}{|B_{C_2n}(\Z^2)| }\ge \frac{1}{2} \frac{ \frac{1}{D_2}n^2} {4\frac{C_2}{C_1}e^{s} +1}  \frac{1}{D_1n^2}=:\delta'
%%\end{align*}
%%which completes the proof.
%%%Therefore ${|GU_{3\e,n}| = O(|S_k|) = O(n^2)}$ and we are done.  %.\note{Its not clear to me that ``big-O''  gives a uniform $\delta'$ in $\epsilon$}
%%\end{proof}
%%

%
%\begin{claim}\label{conjuga}\note{This claim is wrong but not needed.  just state what we need in the next proof? I don't understand why this is wrong. DF.}\noted{The `n' in $G$ and $GU$ are log-related}
%For any $\epsilon>0$, there exists $N_{\e}$ such that if $n \geq N_{\e}$ and if  $A = \begin{bmatrix} a&c\\b&d \end{bmatrix}   \in G_{\e,n}\cap G_{\e,n}\inv$ then  %and $A\inv \in G_{\e,n}$  then
%the unipotent elements $u_{a,b}, u_{c,d}\in B_{e^n}(\Z^2)$ are contained in $GU_{3\e,e^n}$.
%\end{claim}
%
%\begin{proof}
%For $u_{a,b}$, we have that $$\alpha (u_{a,b})  =\alpha (A) \circ \alpha (u_{1,0})\circ  \alpha (A\inv)$$
%whence $\|D \alpha (u_{a,b}) \| \le \|D\alpha (u_{1,0})\| e^{3\epsilon n}. $
%%{\blue Follows by conjugating the unipotents $u_{1,0}$ and $u_{0,1}$ by $A$  and using the chain rule.}
%\end{proof}


%\note{claim removed and estimate incorporated into the following \\ also adjusted estimates}
\begin{lemma}\label{proportion}
There exists $\delta' > 0$ with the following properties: for any   $\e>0$ there is an $N_{\e}'>0$ such that for any $n \geq N_{\e}'$ we have  $$|GU_{\e,n}| \geq \delta' |B_n(\Z^2)|.$$
 \end{lemma}


\begin{proof}
Recall that $T_k$ denotes the intersection of the ball of radius $k$ in $\Sl(2,\R)\simeq H_{1,2}$ with $\Sl(2,\Z)= \Lambda_{1,2}$ and $|T_k|$ denotes the cardinality of $T_k$.  As $|T_k|$ grows exponentially in $k$, we may take
%Let $k := 2\log(n)$.
 $s$ fixed  so  that $|T_{k-s}| < \frac{1}{2}|T_{k}|$ for all $k$ sufficiently large. Given $\epsilon '>0$, define the subset $S_k\subset \Sl(2,\Z)$ to be   $$S_k := G_{\e',k}\cap G_{\e',k}\inv \cap  (T_k \setminus T_{k-s}).$$ From Proposition \ref{mainunipo}, we may  assume that $$|S_{k}| \geq \frac{1}{2} |T_k|.$$

From   \eqref{normdistance}, there exists $C_1>0$ such that if $A = \begin{bmatrix} a&c\\b&d \end{bmatrix}$ belongs to $S_k$ then either $$\text{$\|(a,b)\|_\infty \geq C_1e^{ \frac{1}{2}{(k-s)} }$ or $\|(c,d)\|_\infty \geq C_1e^{\frac{1}{2}{(k-s)} }$.}$$ Without loss of generality,   we     assume that at least half of the elements in $S_k$ satisfy   $\|(a,b)\|_\infty \geq C_1e^{ \frac{1}{2}{(k-s)} }$.

Consider the map $P\colon S_{k} \to \Z^2$ that assigns     $A = \begin{bmatrix} a&c\\b&d \end{bmatrix}$ to $(a,b)$. By  \eqref{normdistance},  there is $C_2>1$ such that  the image $P(S_k)$ of $S_k$ lies in the norm-ball $B_{C_2e^{\frac{k}{2}}}(\Z^2)$ for all $k$.

{\blue
 Let $k(n) = 2\log (n)-\log C_2 $. Then $P(S_{k(n)}) \subset B_{n}(\Z^2)$.  If  $n$ is sufficiently large and $A = \begin{bmatrix} a&c\\b&d \end{bmatrix} \in S_{k(n)}$ then  we have $u_{a,b} \in GU_{(5\e',n)}$; indeed $$\alpha (u_{a,b})  =\alpha (A) \circ \alpha (u_{1,0})\circ  \alpha (A\inv)$$
whence $$\|D \alpha (u_{a,b}) \| \le \|D\alpha (u_{1,0})\| e^{2 \epsilon' k(n)}. $$
We have $|B_{   n}(\Z^2)|\le D_1 n^2$ for some $D_1\ge 1$.  Also, from \eqref{decapitationOrImpeachment?} and Lemma \ref{basic} we have   $|S_{k(n)}| \geq \frac{1}{2}|T_{k(n)}| \ge \frac{1}{2} e^{k(n)} = %\frac 1{D_2}e^{k(n)} =
\frac 1{D_2}n^2$ for some $D_2\ge 1$.
}


To to complete the proof, we show that the preimage $P^{-1} ((a,b))$ {\blue in $S_k$} of any $(a,b)\in \Z^2$ {\blue satisfying $\|(a,b)\|_\infty \geq C_1e^{ \frac{1}{2}{(k-s)} }$} has uniformly bounded cardinality  depending only on  $s$.
 Observe that if $A,A'\in \Sl(2,\Z)$  satisfy $P(A) = P(A')$, then $A' = AU$, where $U = \begin{bmatrix} 1&m\\0&1 \end{bmatrix}$ for some $m \in \Z$ and we have %\note{added $\frac {1}{2}$ to exponents}
 $$A=   \begin{bmatrix} a& c\\b&  d \end{bmatrix},\quad \text{and} \quad A' =   \begin{bmatrix} a&am+ c\\b&bm + d \end{bmatrix}.$$
If $A'$ belongs to $T_k$ then $\|(am +c, bm+d)\|_\infty \leq C_2e^{\frac{k}{2}}$ and if $A$ belongs to $T_k$ then $\|(c, d)\|_\infty \leq C_2e^{\frac{k}{2}}$.
We thus have   that $|am|\le 2 C_2 e^{{\blue \frac{k}{2}}}$ and  $|bm|\le 2 C_2 e^{{\blue \frac{k}{2}}}$.
As we assume that $$\|(a,b)\|_\infty \geq C_1e^{ \frac{k-s}{2} }$$ we have that  $|m|\le  2\frac{C_2}{C_1}e^{{{\blue \frac{s}{2}}}}$.  Thus, the preimage $P^{-1}((a,b))$  has  at most $4\frac{C_2}{C_1}e^{{{\blue \frac{s}{2}}}} +1$ elements {\blue in $S_k$} .


With $\epsilon'=\frac 1 5 \epsilon$, having taken $n$ sufficiently large, we thus have
{\blue \begin{align*}\frac{|GU_{\e,n}|}{|B_n(\Z^2)|} \ge \frac {1} {D_1n^2} {\frac{\frac{1}{2}|S_{k(n)} |}{4\frac{C_2}{C_1}e^{s/2} +1  }}\ge
 \frac{1}{2} \frac{ \frac{1}{D_2}n^2} {4\frac{C_2}{C_1}e^{s/2} +1}  \frac{1}{D_1n^2}=:\delta'
\end{align*}}
which completes the proof.\end{proof}

%{\blue \begin{align*}\frac{|GU_{\e,n}|}{|B_n(\Z^2)|} \ge \frac {\tfrac{\frac{1}{2}|S_{k(n)} |}{4\frac{C_2}{C_1}e^{s} +1  }}{D_1n^2}\ge
% \frac{1}{2} \frac{ \frac{1}{D_2}n^2} {4\frac{C_2}{C_1}e^{s} +1}  \frac{1}{D_1n^2}=:\delta'
%\end{align*}}
%{\red \begin{align*}\frac{|GU_{\e,n}|}{|B_n(\Z^2)|} \ge \frac{|GU_{\e,n}|}{|B_{C_2n}(\Z^2)| }\ge \frac{1}{2} \frac{ \frac{1}{D_2}n^2} {4\frac{C_2}{C_1}e^{s} +1}  \frac{1}{D_1n^2}=:\delta'
%\end{align*}}
%Therefore ${|GU_{3\e,n}| = O(|S_k|) = O(n^2)}$ and we are done.  %.\note{Its not clear to me that ``big-O''  gives a uniform $\delta'$ in $\epsilon$}
%which completes the proof.
%\end{proof}








To complete the proof  of Proposition \ref{finalunipotent}, we show  that any element in $B_n(\Z^2)$ can be written as a product of a bounded number of elements in $GU_{\e,n}$ independent of $\e$.  This follows from the structure of sumsets of  abelian groups.  % of $B_n(\Z^2)$.

From the chain rule and submultiplicativity of norms, we have the following.

%\begin{claim}\label{trivialchain}
%For any positive integers $n,m$ and $\e_1, \e_2 > 0$ if  $u_{a,b} \in GU_{\e_1,n}$ and $u_{c,d} \in GU_{\e_2,m}$ , then the product $u_{a,b}u_{c,d} \in GU_{\e_1 + \e_2,n + m}$
%\end{claim}


\begin{claim}\label{trivialchain}
For any positive integers $n,m$ and $\e_1, \e_2 > 0$, if  $u_{a,b} \in GU_{\e_1,n}$ and $u_{c,d} \in GU_{\e_2,m}$   then the product $u_{a,b}u_{c,d} \in GU_{\max\{\e_1, \e_2\},n + m}$
\end{claim}
%\note{for $n,m\in \Z_+$ we have $n^{\e_1}m^{\e_2}\le (nm)^{\max\{\e_1, \e_2\}}$}


%\begin{proof} Follows easily from the chain rule.
%\end{proof}

For subsets $A, B \subset \Z^2 $ we denote by     $A + B$ the sumset of $A, B$.   % defined by $$A+B: = \{a + b \text{ such that } a \in A, b \in B\}$$

%{\blue \begin{proposition}\label{babycombinatorics}
%For any $\delta > 0$, there exists a positive integer $k_{\delta}$ and a finite set $F_{\delta} \subset \Z^2$ such that for any symmetric set $S_n \subset B_n(\Z^2)$ %($S_n = -S_n$)
%with  $|S_n| >  \delta|B_n|$, we have that $$B_n\subset F_\delta + \underbrace{S_n + S_n + ... + S_n}_{\text{$k_{\delta}$ times}}.$$
%\end{proposition}}
%{\blue
%\begin{proof}\note{rewrite this.}
%
%We will need some notation, {\red which already exists!} define for an integer $m$ the rectangle $R(m):= \big([-m,m]\times[-m,m] \big) \cap \Z^2$ and define the number  $N_{\delta} := \big( [\frac{2}{\delta}] + 1 \big)!$.
%For simplicity assume that $|S_n| > \delta |R(n)|$.
%
%\textbf{Claim:} Given a  {\red nonzero??} vector $v \in R(n\delta/100)$ of the form $v = (l,0)$  for some $l \in \Z$ (a horizontal vector) , then $N_{\delta}v \in  \sum_{N_{\delta}} S_n$.
%
%Before proving our claim, observe that this will finish the proof as by symmetry the same claim will be true for the vertical vectors of the form $(0,l)$, and therefore if we denote
%$F_{\delta} := R(N_{\delta})$ we have that $F_{\delta} +  S_n + \dots + S_n$ ($2N_{\delta}$ times) contains $R(n)$ as we wanted.
%
%To prove our claim, define an equivalence relation in $R(n)$ by saying that two elements $x,y \in R(n)$ are equivalent if $x-y$ is an integer multiple of $v$.
%Then $R(n) = \cup_{x \in R(n)} C_x$ where $C_x$ are defined as the different equivalent classes.
%Each equivalence class is of the form $$C_x = \{ ...,x - v, x, x + v, x + 2v,....\}$$ Observe also that each equivalence class contains at least {\red $100/\delta$ elements.} \note{This cant be true if $n$ is really small (e.g. forcing $v=0$}
%Therefore as  $|S_n| \geq \delta |R(n)|$, there exists at least one equivalence class such that $|C_x \cap S_n| \geq \delta |C_x|$,
%this implies that as the elements of $C_x$ are in arithmetic progression there is at least two elements $a, b$ in $C_x$ such that $a,b$ are $[\frac{2}{\delta}]$ apart in such progression, this implies that $a-b = ik$, where $ 0 \leq i \leq [\frac{2}{\delta}]$
%and so we have that $iv \in S_n + S_n$ and as as $i$ divides $N_{\delta}$, we have that $N_{\delta}v \in  \sum_{N_{\delta}} S_n$.
%
%\end{proof}
%}


\begin{claim}\label{babycombinatorics}
For any $0<\delta <1$, there exists a positive integer $k_{\delta}$ and a finite set $F_{\delta} \subset \Z^2$ such that for any $n$ and any symmetric set $S_n \subset B_n(\Z^2)$ %($S_n = -S_n$)
with  $|S_n| >  \delta|B_n|$, we have that $$B_n\subset F_\delta + \underbrace{S_n + S_n + ... + S_n}_{\text{$k_{\delta}$ times}}.$$
\end{claim}


\begin{proof}
Fix $M\in \Z_+$ with $\frac 1 M<\delta$.
Take    $N_{\delta} := \big( M + 1 \big)!$, $k_\delta = 4 N_\delta$, and $F_{\delta} := B_{N_{\delta}}(\Z^2).$
Consider a symmetric set $S_n \subset B_n (\Z^2)$ with $|S_n| > \delta |B_n(\Z^2)|$.

If $n\le N_\delta$ then $B_n (\Z^2)\subset F_\delta$ and we are done.  Thus, consider $n\ge N_\delta$.  To complete the proof the claim, we argue that the set
$$\sum_{k_\delta} S_n := \underbrace{S_n + S_n + ... + S_n}_{\text{$k_{\delta}$ times}}$$ contains the intersection of  the sublattice $N_\delta \Z^2$ with $B_n(\Z^2)$.  Adding $F_\delta$ to the sumset then implies the claim.
  Consider any  non-zero vector
$\td v\in N_\delta \Z^2\cap B_n(\Z^2)$ of the form $(\td \ell, 0)$ for some $\td \ell \in [-n,n]\cap N_\delta \Z$.
Then $\td v = N_\delta v$ where $v= (\ell,0)$ is such that $0< |\ell |\le \lfloor n N_\delta\inv \rfloor$.

Consider the equivalence relation in $B_n(\Z^2)$ defined by declaring that two elements $x,y \in R(n)$ are equivalent if $x-y$ is an integer multiple of $v$.
%Then $R(n) = \cup_{x \in R(n)} C_x$ where $C_x$ are defined as the different equivalent classes.
Each equivalence class is of the form $$C_x = \{ ...,x - v, x, x + v, x + 2v,....\}.$$
As  $|S_n| \geq \frac{1}{M} |B_n(\Z^2)|$, there exists  one equivalence class $C_x$  such that $|C_x \cap S_n| \geq \frac{1}{M} |C_x|$.
Since $0< |\ell |\le \lfloor n N_\delta\inv \rfloor$, each equivalence class contains at least $M+1$ elements and hence $C_x \cap S_n$ contains at least two elements $a,b$ with $b= a + i v $ for $|i|\le M$.
In particular, since  $a-b = iv$, we have $iv \in S_n + S_n$.  As $i$ divides $N_{\delta}$, we have that $\td v = N_{\delta}v \in  \sum_{2N_{\delta}} S_n$.

Similarly, for $n\ge N_\delta$ and any  $\td u\in N_\delta \Z^2\cap B_n(\Z^2)$ of the form $(0, \td \ell)$  we have
$\td u \in  \sum_{2N_{\delta}} S_n$.  Then $$\td u  + \td v\in \sum_{4N_{\delta}} S_n$$
completing the proof. \end{proof}

%for at least one equivalence class $C_x$.



%We claim that given any vector $v \in B_{n\delta/100}(\Z^2)$ of the form $v = (\ell,0)$  for some $\ell \in \Z$ , then $N_{\delta}v \in  \sum_{N_{\delta}} S_n$.
%$$\underbrace{S_n + S_n + ... + S_n}_{\text{$k_{\delta}$ times}}$$
%
%Before proving our claim, observe that this will finish the proof as by symmetry the same claim will be true for the vertical vectors of the form $(0,l)$, and therefore if we denote
%$F_{\delta} := R(N_{\delta})$ we have that $F_{\delta} +  S_n + \dots + S_n$ ($2N_{\delta}$ times) contains $R(n)$ as we wanted.

%To prove our claim, define an equivalence relation in $R(n)$ by saying that two elements $x,y \in R(n)$ are equivalent if $x-y$ is an integer multiple of $v$.
%Then $R(n) = \cup_{x \in R(n)} C_x$ where $C_x$ are defined as the different equivalent classes.
%Each equivalence class is of the form $$C_x = \{ ...,x - v, x, x + v, x + 2v,....\}$$ Observe also that each equivalence class contains at least $100/\delta$ elements.
%Therefore as  $|S_n| \geq \delta |R(n)|$, there exists at least one equivalence class such that $|C_x \cap S_n| \geq \delta |C_x|$,
%this implies that as the elements of $C_x$ are in arithmetic progression there is at least two elements $a, b$ in $C_x$ such that $a,b$ are $[\frac{2}{\delta}]$ apart in such progression, this implies that $a-b = ik$, where $ 0 \leq i \leq [\frac{2}{\delta}]$
%and so we have that $iv \in S_n + S_n$ and as as $i$ divides $N_{\delta}$, we have that $N_{\delta}v \in  \sum_{N_{\delta}} S_n$.

%\end{proof}




%\note{rewrite with better $\epsilon$ estimates}
%\begin{proof}[Proof of Proposition \ref{finalunipotent}] Given $\epsilon'>0$,  let  $\delta'$ and $ N_{\e'}'$ be given by Lemma \ref{proportion}.  Let $S_n := GU_{\e',n}$ be as in \eqref{eq:racistPOTUS}  and take $k_\delta'$ and $F_{\delta'}$ as in  Lemma \ref{babycombinatorics}.  Note that $GU_{\e',n}$  is symmetric by definition.
%Take $N\ge N'_{\e'}$  such  that $F_{\delta'} \in GU_{\e',n}$ whenever $n \geq N $. For $n\ge N$ and any $u_{a,b} \in B_n(\Z^2)$ we have that $u_{a,b} \in F_\delta + S_n + S_n+ ... + S_n $ ($k_{\delta'}$ times) by Proposition \ref{babycombinatorics}. Proposition \ref{trivialchain} then implies then that $u_{a,b} \in GU_{(k_{\delta'} + 1)\e', (k_{\delta'} +1)n}$ which implies that
% $\|D(u_{a,b})^{\pm1}\| \leq e^{(k_{\delta'} + 1)\e' \log ((k_{\delta'} +1)n) }$.  Taking $\epsilon '>0$ such that $ (k_{\delta'} + 2)\e'\le \epsilon$, take $N_\epsilon\ge N$ sufficiently large so that
%$$e^{\log ((k_{\delta'} +1)n) \log ((k_{\delta'} +1)n)} \le e^{(k_{\delta'} + 2)\e' n}$$
%for all $n\ge N_\epsilon$.  Then for $n\ge N_\epsilon$ and any $u_{a,b} \in B_n(\Z^2)$  we have
%$$\|D(u_{a,b})^{\pm1}\|\le e^{\epsilon n}.$$
%
% if $N_{\e}$ is large enough. As $k_{\delta'}$ is independent of $\e$ the result follows.
%\end{proof}
%
%
\begin{proof}[Proof of Proposition \ref{finalunipotent}] Given $\epsilon'>0$,  let  $\delta'$ and $ N_{\e'}'$ be given by Lemma \ref{proportion}.  Let $S_n := GU_{\e',n}$ be as in \eqref{eq:racistPOTUS}  and take $k_\delta'$ and $F_{\delta'}$ as in  Lemma \ref{babycombinatorics}.  Note that $GU_{\e',n}$  is symmetric by definition.
Take $N\ge N'_{\e'}$  such  that $F_{\delta'} \in GU_{\e',n}$ whenever $n \geq N $. For $n\ge N$ and any $u_{a,b} \in B_n(\Z^2)$ we have that $u_{a,b} \in F_\delta + S_n + S_n+ ... + S_n $ ($k_{\delta'}$ times) by Proposition \ref{babycombinatorics}. Proposition \ref{trivialchain} then implies   that $u_{a,b} \in GU_{ \e', (k_{\delta'} +1)n}$ so $\|D(u_{a,b})^{\pm1}\| \leq e^{ \e' \log( (k_{\delta'} +1)n ) }$.  With  $\epsilon '= \epsilon /2$, take $N_\epsilon\ge \max\{N,(k_{\delta'} +1)\}. $ Then for all $n\ge N_\epsilon$ we have $$ \e' \log( (k_{\delta'} +1)n) \le  \e \log(n)$$
whence
$$\|D(u_{a,b})^{\pm1}\|\le e^{\epsilon \log(n)}$$
and for $u_{a,b} \in B_n(\Z^2)$ with $n\ge N_\epsilon$.
% if $N_{\e}$ is large enough. As $k_{\delta'}$ is independent of $\e$ the result follows.
\end{proof}





\section{Proof of Theorem \ref{main2}}
\subsection{Reduction to the restriction of an action by $\Lambda_{i,j}$}
We recall the work of Lubotzky, Mozes, and Raghunathan, namely   \cite{MR1244421} and \cite{MR1828742}, which establishes quasi-isometry   between the word and Riemannian metrics on lattices in higher-rank semisimple Lie groups.
In the special case of $\Gamma= \Sl(m,\Z)$ for $m\ge 3$,
 in  \cite[Corollary 3]{MR1244421} it is shown that any element $\gamma$ of $\Sl(m,\Z)$ is written as a product of at most $m^2$ elements $\gamma_i$.  Moreover each $\gamma_i$ is contained some $\Lambda_{i,j}\simeq \Sl(2,\Z)$ and the word-length of each $\gamma_i$ is  proportional to the word-length of $\gamma$.

 Thus, to establish that an action $\alpha\colon \Gamma\to \Diff^1(M)$ has uniform subexponential growth of derivatives in  Theorem \ref{main2}, it  is sufficient to show that the restriction  $\restrict {\alpha}{\Lambda_{i,j}}\colon \Gamma\to \Diff^1(M)$ has uniform subexponential growth of derivatives for each $1\le i\neq j \le m$.  We emphasize that to measure subexponential growth of derivatives, the word-length on ${\Lambda_{i,j}}$ is measured as the word-length as embedded in $\Sl(m,\Z)$ (which is quasi-isometric to the Riemannian metric on $\Sl(m,\R)$) rather than the intrinsic word-length in $\Lambda_{i,j}\simeq \Sl(2,\Z)$ (which is not quasi-isometric to the Riemannian metric on $\Sl(2,\R)$).

As the Weyl group acts transitively on the set of all $\Lambda_{i,j}$, it is   sufficient to consider a fixed $\Lambda_{i,j}$.  Thus to deduce Theorem \ref{main2},  in the remainder of this section we establish the following, which is the  main proposition of the paper.
\begin{proposition}\label{prop:maybeweshouldstatethemainresultatsomepoint}
For any action $\alpha\colon \Gamma\to \Diff^1(M)$ as in  Theorem \ref{main2},  the restricted action $\restrict {\alpha}{\Lambda_{1,2}}\colon \Gamma\to \Diff^1(M)$ has  uniform subexponential growth of derivatives.
\end{proposition}





%[Reduction to $\Lambda_{i,j}$.  HERE]
\subsection{Orbits with large fiber growth yet low depth in the cusp}\label{maximal}

%In this section we show that if uniform subexponential growth of derivatives  in Theorem \ref{main2} fails for the restriction of the action $\alpha$ to a   canonical copy $\Lambda_{i,j} \cong \Sl(2,\Z)$ then there exist  orbits in $H_{i,j}/\Lambda_{i,j}\cong \Sl(2,\R)/\Sl(2, \Z)$ with controlled behavior in  the cusp and with exponential growth of the fiberwise derivative.  More precisely, we will obtain  orbits whose distance into the cusp is  sub-linear in the length of the orbit and which have   exponential growth of the fiberwise derivative.  The key idea to control the depth of orbits into the cusp is that   geodesic  paths in the modular surface that are deep in the cusp have unipotent monodromy and the growth  of the fiberwise derivative is subexponential along such paths from Section \ref{unipotents}.

To prove Proposition \ref{prop:maybeweshouldstatethemainresultatsomepoint}, as in Section \ref{sec:mutualmastication} we consider a canonical embedding $X=H_{1,2}/\Lambda_{1,2}$ of $\Sl(2,\R)/\Sl(2, \Z)$ in $\Sl(m,\R)/\Sl(m,\Z)$.
%Let $$X := H_{1,2}/\Lambda_{1,2}= \Sl(2,\R)/\Sl(2, \Z)$$ be this embedded unit tangent bundle of the modular surface.
Write  $$a^t := \text{diag}(e^{t/2}, e^{-t/2})\subset \Sl(2,\R)$$ for the geodesic flow on $X$.
Let $X_{\text{thick}}$ be a fixed compact   $\So(2)$-invariant ``thick part'' of $X$;  {\blue that is, relative to the Dirichlet  domain $\mathcal D$ in \eqref{eq:dirc}, points in $\So(2)\backslash X_{\text{thick}}$  corresponds to the points in  $\So(2) \backslash \mathcal D$  whose imaginary part is bounded above, say, by   $17$.}

%{\blue More precisely, view $\Sl(2,\R)$ acting  on the upper half-plane model of hyperbolic space $\mathbb H^2$ by M\"obius transformations.  There is a canonical fundamental domain $\mathcal D\subset \mathbb H^2= \So(2)\backslash \Sl(2,\R)$ for the $\Sl(2,\Z)$-action on $\mathbb H^2$ containing $i$ in its boundary and $\infty$ in its cusp.  Then $\So(2)\backslash X_{\text{thick}}$  corresponds to the points in  the  fundamental domain  $\mathcal D$ all of whose imaginary part is bounded above, say, by   $17$.}

A geodesic curve in the modular surface of length $t$ corresponds  to the image of  an orbit $\zeta= \{a^s (x)\}_{0\leq s \leq t}$ where $x \in X$ and $t \geq 0$.  Denote the length of such a curve by $l(\zeta)$. For an orbit   $\zeta= \{a^s (x)\}_{0\leq s \leq t}$ of $\{a^t\}$ in $X$ we define $$c(\zeta) := \log (\|D_x(a^t)\|_{\text{Fiber}}).$$


%\noindent For the rest of this section, for clarity of notation, we use simple norm bars for the fiberwise derivative and suppress the subscript.  {\red this is  bad notation though}

The following claim is straightforward from the compactness  $X_{\text{thick}}$ and the quasi-isometry between the word and Riemannian metrics on $\Gamma$.  %of and motivates the rest of the discussion below.

\begin{claim}\label{slowgrowththick} For an action $\alpha\colon \Sl(m,\Z) \to \diff^1(M)$, the following statements are equivalent:
\begin{enumerate}
\item  the restriction $\restrict \alpha {\Lambda_{1,2}}\colon \Lambda_{1,2} \to \diff^1(M)$ has uniform subexponential growth of  derivatives;
\item for any $\e>0$ there is a $t_{\e}>0$ such that for any orbit  $\zeta =  \{a^s (x)\}_{0\leq s \leq t}$ with  $x\in \Xt$, $a^t(x) \in \Xt$, and $l(\zeta) = t \geq {t_\e}$ we have  $$ c(\zeta) \leq \e l(\zeta).$$
\end{enumerate}
\end{claim}

Define the maximal fiberwise growth rate of orbits starting and  returning to $\Xt$ to be \begin{equation}\label{eq:fanorlamps}\chi_{\mathrm{max}} := \limsup_{t > 0} \left\{\sup \left\{   \frac{\log \|\restrict{D_x(a^t)}{{\text{Fiber}}}\|}{t} : x\in \Xt, a^t(x) \in \Xt\right\}\right\}.\end{equation} %where the maximum\note{put parameters into defn. restrict to the  fiber} is taken over all $x$ such that $x, a^t(x) \in X_{\text{thick}}$.
Using Claim \ref{slowgrowththick}, to establish Proposition \ref{prop:maybeweshouldstatethemainresultatsomepoint} it is sufficient to show  that $\chi_{\mathrm{max}} = 0$.

For an orbit  $\zeta = \{a^s (x)\}_{0\leq s \leq t}$,   define the following function which measures the depth of $\zeta$ into the cusp:
$$d(\zeta) = \max_{0 \leq s \leq t} \text{dist}(a^s(x), \Xt).$$
  %The function $d$ measures the maximal distance of a point in $\gamma$ to the thick part.
%The reason why we define such maximal exponent $\chi_{\text{max}}$ is the following lemma, which is the


The following lemma is the main result of this subsection.

\begin{lemma}\label{lemma:maximal} If  $\chi_{\text{max}} >0$ then  there exists a sequence of orbits $\zeta_n=\{a^s (x_n)\}_{0\leq s \leq t_n}$ with  $x_n\in \Xt$, $a^{t_n}(x_n) \in \Xt$, and   $t_n = l(\zeta_n) \to \infty$ % starting and ending at $X_{\text{thick}}$ such that:
such that
\begin{enumerate}
\item  $\displaystyle c(\zeta_n) \geq \frac{\chi_{\mathrm{max}}}{2} t_n$;
\item  $\displaystyle \lim_{n \to \infty} \frac{d(\zeta_n)}{t_n} = 0$.
\end{enumerate}
\end{lemma}

%  an immediate consequence of the fact the values of the cocycle coming from geodesics in the cusp of $X$ correspond to powers of unipotent elements and the slow growth of unipotent elements proven in
% Proposition \ref{unipotentisgood}.

%\note{w.t.f. is a "path"}
%\note{check that this is really true.  if start deep out, is word length really bounded by path length?}
%\begin{claim}\label{pathcusp} For any $\e>0$ there exists $t_{\e}$ with the following properties: for any $x\in X\sm \Xt$ and $t\ge t_\epsilon$ such that  $a^s(x)\in  X\sm \Xt$ for all $0\le s \le t$, for the orbit $\gamma =  \{a^s (x)\}_{0\leq s \leq t}$ we have
%%su%ch that for any $x \in  \partial(X_{\text{thick}})$ and any {\red path}  $\gamma $ starting at $x$, such that $\gamma$ lies outside the interior of $X_{\text{thick}}$ and $l(\gamma) > t_{\e}$ ,
%% then
%$$c(\gamma) \leq \e t = \epsilon l(\gamma).$$

We first have the following claim.
\begin{claim}\label{pathcusp}
 For any $\e>0$ there exists $t_{\e}$ with the following properties: for any $x\in \partial \Xt$ and $t\ge t_\epsilon$ such that  $a^s(x)\in  X\sm \Xt$ for all $0< s < t$ and $a^t(x) \in \partial \Xt$ then, for the orbit $\zeta =  \{a^s (x)\}_{0\leq s \leq t}$, we have
%su%ch that for any $x \in  \partial(X_{\text{thick}})$ and any {\red path}  $\gamma $ starting at $x$, such that $\gamma$ lies outside the interior of $X_{\text{thick}}$ and $l(\gamma) > t_{\e}$ ,
% then
$$c(\zeta) \leq \e t = \epsilon l(\zeta).$$
\end{claim}

Indeed, the claim follows from the fact that the value of the  return cocycle $\beta(a^s, x)$ is defined by geodesic in the cusp of $X$ is given by  a unipotent matrix of the form $\left(\begin{array}{cc}1 & n \\0 & 1\end{array}\right)\in \Lambda_{1,2}\subset \Sl(m,\Z)$ and  Proposition \ref{unipotentisgood}.


\begin{proof}[Proof of Lemma \ref{lemma:maximal}] Let $\zeta_n:= \{a^s (x_n)\}_{0\leq s \leq t_n}$ be a sequence of orbits with $x_n\in \Xt$, $a^{t_n}(x_n)\in \Xt$,  $t_n\to \infty$, and  such that  $$\chi_{\max} = \lim_{n \to \infty}\frac{ c(\zeta_n)}{ t_n}.$$
Replacing $\zeta_n$ with a subsequence, we may assume the following limit exists: $$\beta := \lim_{n \to \infty} \frac{d(\zeta_n)}{t_n}.$$

We aim to prove that $\beta = 0$. Arguing by contradiction,  suppose $0< \beta\le 1$. We decompose the orbit $$\zeta_n = %\omega_{k_n}
\alpha_{k_n}\omega_{k_{n-1}}\alpha_{k_{n-1}}\cdots \omega_{1}\alpha_1$$ %$\gamma_n$ into
as a concatenation of smaller orbit segments $\alpha_i, \omega_i$ with the following properties:
\begin{enumerate}
\item each orbit $\alpha_i$ is  such that $d(\alpha_i) \leq \frac{\beta}{2} t_n$;
\item the endpoints of each orbit $\alpha_i$ are contained in $\Xt$;
\item  each orbit $\omega_i$ is  contained entirely in $(X\sm \Xt) \cup \partial \Xt$ with endpoints contained in $\partial \Xt$;
\item each orbit $\omega_i$  satisfies $d(\omega_i) \geq \frac{\beta}{2} t_n$  whence $l(\omega_i)\ge \frac {\beta}{2} t_n$ for $t_n$ sufficiently large.
\end{enumerate}
% $\partial X_{\text{thick}}$ such that
% $$\gamma_n = \omega_{k_n}\alpha_{k_n}\omega_{k_{n-1}}\alpha_{k_{n-1}}....\omega_{1}\alpha_1$$
% and where each $\alpha_i$ is a segment such that $d(\alpha_i) \leq \frac{\beta}{2} l(\gamma_n)$ and each $\omega_i$ is a segment starting and ending at $\partial \Xt$, that lies outside $\Xt$ and satisfies  $d(\omega_i) \geq \frac{\beta}{2} l(\gamma_n)$.
%\note{consistent .9 vs 1/2}
Note for each  $n$, that $k_n \leq \lfloor \frac{2}{\beta}\rfloor+ 1$ and  thus $k_n$ is   bounded by some $k$ independent of $n$. Additionally,  since   $\Sl(m, \Z)$  is finitely generated and (equipped with the word metric) is quasi-isometrically embedded  in  $\Sl(m, \R)$,  there exists a constant $K$ such that for any orbit  segment $\zeta$ whose endpoints are contained  in $\Xt$, we have  $c(\zeta) \leq K l(\zeta)$.
By the definition of $\chi_{\max}$, for any $\e> 0$ there is a positive constant $M_{\e}$ such that for any orbit sub-segment $\alpha_i$ % one of the following holds:
\begin{enumerate}
\item $c(\alpha_i) \leq (\chi_{\max} + \e)l(\alpha_i)$    whenever  $l(\alpha_i) > M_\e$
\item $c(\alpha_i) \leq KM_\e$ whenever $l(\alpha_i) \leq M_{\e}$.
\end{enumerate}
From Claim  \ref{pathcusp}, for any   $\e>0$ we have, assuming that $n$ and hence $t_n$ are   sufficiently large,   that $$c(\omega_i) < \epsilon l(\omega_i)$$ for all orbit sub-segmants $\omega_i$.



Taking $n$  sufficiently large  we have
\begin{equation}\label{eqmax1}
(\chi_{\max} - \e)t_n < c(\zeta_n) \leq \sum_i c(\omega_i) + \sum_i c(\alpha_i).
\end{equation}
As we assume $\beta>0$, for all sufficiently large $n$ there exists at least one orbit sub-segment $\omega_i$ and thus for such $n$
\begin{equation}\label{eqmax2}
\sum_i c(\alpha_i) \leq kKM_{\e} + (\chi_{\max} + \e)\sum_i l(\alpha_i) \leq kKM_{\e} + (\chi_{\max} + \e)(1-  \beta/2   )t_n.
\end{equation}
From   \eqref{eqmax1} and \eqref{eqmax2} we obtain that
\begin{equation}
(\chi_{\max} - \e)t_n \leq   (\epsilon t_n) + \Big(kKM_{\e} + (\chi_{\max} + \e)(1- \beta /2)t_n\Big).
\end{equation}
Dividing by $t_n$ and taking $n \to \infty$ obtain
$$ \chi_{\max} - \e \leq \e  + (\chi_{\max} + \e)(1-  \beta/2).$$
As we assumed $\chi_{\max}>0$ and $\beta > 0$, we obtain a contradiction by taking $\e>0$ sufficiently small.  % enough, therefore $\beta = 0$.
\end{proof}
%{\red there is no contradiction if $\chi_{\mathrm{max}}=0$.  somewhere you assumed something.}


\subsection{Construction of a \Folner sequence and family averaged measures}
%\note{Note: measures are onstructed measures on $M^\alpha$, not  $UTM^\alpha$}
\label{section:folner}
Assuming that $\chi_{\mathrm{max}}$ in \eqref{eq:fanorlamps} is non-zero, we start from the orbit segments constructed in Lemma \ref{lemma:maximal} and perform an averaging procedure to obtain a family of measures $\{  \mu _n\}$ on  $M^\alpha$ whose properties lead to a contradiction.   In particular, %any
%In this section we define a sequence of measures $\td \mu _n$ in $UF$ constructed from the paths in Section \ref{maximal} such that
the projection of any weak-$*$ limit $  \mu _\infty$ of $  \mu _n$  to $M^\alpha$ will be $A$-invariant, well behaved at the cusps, and have non-zero Lyapunov exponents.  These measures on $M^\alpha$ are obtained by averaging certain Dirac measures against \Folner sequences in a certain amenable subgroup of $G$.

Consider the copy of  $\Sl(m-1, \R)\subset \Sl(m,\R)$ as the subgroup of matrices that differ from the identity away from the $m$th row and $m$th column.  Let $N'\simeq \R^{m-1}$ be the abelian  subgroup of unipotent elements that differ from the identity only in the $m$th column; that is given a vector $r = (r_1,r_2, \dots, r_{m-1})\in \R^{m-1}$ define $u^{r}$ to be the unipotent element
\begin{equation}\label{eq:formerlyknownas}
u^r= \left(\begin{array}{ccccc}1  &  0  & 0  & \dots  & r_1  \\  & 1  &   0 & \dots  & r_2  \\  &   &  \ddots &   &  \vdots  \\  &   &   &  1 &  r_{m-1} \\  &   &   &   &1  \end{array}\right)\end{equation}
and let $N'= \{u^r\}$.  $N'$ is normalized by $\Sl(m-1,\R)$.

Identifying $N'$ with $\R^{m-1}$ we have an embedding
   $\Sl(m-1, \R) \ltimes \R^{m-1}  \subset \Sl(m,\R)$.  % where $\R^{m-1}$ corresponds to the abelian group of unipotent elements that differ from the identity only in the $m$th column.
The subgroup $\Sl(m-1, \R) \ltimes \R^{m-1}$ has as a  lattice the subgroup $$\Sl(m-1, \Z) \ltimes \Z^{m-1}:= \Gamma \cap\big( \Sl(m-1, \R) \ltimes \R^{m-1}\big)$$ and there is a natural embedding given by the inclusion $$(\Sl(m-1,\R)\ltimes \R^{m-1}) / (\Sl(m-1, \Z^{-1}) \ltimes \Z^{m-1})) \subset \Sl(m,\R) / \Sl(m,\Z).$$

%We will construct a  measure $\mu$ in $UF$ which is $AN'$ invariant, where $N'$ is the abelian subgroup of unipotents corresponding to $\R^{m-1}$ above.
Recall $A$ is the group of diagonal matrices with positive entries.
Let $a^t, b^s \in A$ denote matrices
%\begin{center}
$$a^t = \text{diag}(e^{t/2}, e^{-t/2}, 1,1...,1)$$
$$b^s= \text{diag}(e^s, e^{s}, e^{s}..,e^{s}, e^{-s(m-1)}).$$
%\end{center}
%For a vector $r = (r_1,r_2, \dots, r_{m-1})$ define $u^{r}$ to be the unipotent in $N'$ given by:
%$$
%u^r= \left(\begin{array}{ccccc}1  &  0  & 0  & \dots  & r_1  \\  & 1  &   0 & \dots  & r_2  \\  &   &  \ddots &   &  \vdots  \\  &   &   &  1 &  r_{m-1} \\  &   &   &   &1  \end{array}\right).$$
Complete the set  $\{a,b\}$  to a spanning set  $\{a,b, c_1, c_2 \dots c_{m-3}\}$ of $A$ viewed as vector space where the $c_i$ are diagonal matrices whose $(m,m)$-entry is equal to $1$.



%We define a sequence of measures $\mu_n$ obtained by averaging the sequence of Dirac measures $\delta_{(x_n,v_n)}$ over a \Folner sequence $F_n$ of the subgroup $AN'$ which we define next:\\

Let $F_n\subset AN'$ be the subset of $G$ consisting of all the elements of the form
\begin{equation}\label{folner}
a^tb^{s} \prod_{c=1}^{m-3} c_i^{s_i} u^{r}
\end{equation}
where, for some $\delta>0$   to be determined later (in the proof of Proposition \ref{positiveexponent} below),
\begin{enumerate}
\item $0<t<t_n$;
\item$ \delta t_n/2<s < \delta t_n$;
\item $ 0 < s_i <  \sqrt{t_n}$;
\item $r \in B_{\R^{m-1}} (e^{200t_n})$.
\end{enumerate}

\begin{claim}
$\{F_n\}$ is \Folner sequence in $AN'$.
\end{claim}

Observe that $F_n$ is  linearly-long in the $a$-direction and exponentially-long in the $N'$-direction. From conditions (2) and (4), the $A$-component of $F_n$ is much longer in the $a^t$-direction than in the other directions.
%it is much longer in the $a$-direction compared to the other directions $a_i$s, because the idea is to get the limit measure $\mu$ to have $a$-positive Lyapunov exponent.
%\note{averaging unipotents tends to approach haar}
The condition (2) that  $\delta t_n/2<s$ is fundamental in our estimates in Section \ref{section:goodcusps} that ensure the  measures constructed below $\{\mu_n\}$ have uniformly  exponentially small mass in the cusps. These estimates are related to the fact that orbits of $N'$ correspond to the   unstable manifolds for the flow defined by $b_{s}$ in $\Sl(m,\R)/\Sl(m,\Z)$ and open subsets of unstable manifolds equidistribute to the Haar measure on $\Sl(m,\R)/\Sl(m,\Z)$  under the flow $b^{s}$.

Recall we have a sequence of fiber bundles $$F\to M^\alpha\to G/\Gamma$$ % and $\P F\to M^\alpha\to G/\Gamma$}$$
and may consider $F$  %and $\P F$
as a fiber bundle over $G/\Gamma$.  Given $x\in G/\Gamma$, let $F(x)\simeq TM$ denote the fiber of $F$ over $x$.  An element $v\in F(x)$ is a pair $v= (y,\xi)$ where, identifying the fiber of $M^\alpha$ through $x$ with $M$, we have  $y\in M$ and $\xi\in T_y M$.  Given $v= (y,\xi)\in F(x)$,  we write $\|v\| = \|\xi\|$ using our chosen norm on $F$.  % and if $\xi\neq 0$ write $[v]= (y, [\xi])$ for the corresponding element of $\P F$.
Given $v = (y,\xi)\in F(x)$, let $p(v) = y$ denote the footpoint of $v$ in the fiber of $M^\alpha$ through $x$.  %$M$.

If  uniform subexponential growth of derivatives fails for the restriction of the  $\alpha$ to $\Lambda_{1,2}$, then  there exist sequences $x_n\in \Xt$,   $v_n\in F(x_n)$  with $\|v_n\| = 1$, and $t_n \in \R$
as in Lemma \ref{lemma:maximal} and Claim \ref{slowgrowththick} with $t_n\to \infty$,   such that \begin{equation}\label{eq:cleansingDeluge}\|D_{x_n}a_n^{t_n}(v_n)\| \geq e^{\lambda t_n}\end{equation} for some $\lambda >0$.


%\begin{definition}
Note that $AN'$ is a solvable group.  We may equip $AN'$ with any left-invariant Haar measure.  Note that the ambient Riemannian metric induces a right-invariant Haar measure on $AN'$ but as $AN'$ is not unimodular these measures do not coincide.

For each $n$, take  $ \mu _n$ to be the measure on $M^\alpha$    obtained by averaging the Dirac measure $\delta_{(x_n,p(v_n))}$  over the set $F_n$: %\note{measures live on  $M^\alpha$}
$$  \mu _n:= \frac{1}{|F_n|_\ell} \int _{F_n}  g \cdot \delta(x_n, p(v_n))  \ d g$$
where $|F_n|_\ell$ is the volume of $F_n$ and $dg $ indicates integration with respect to left-invariant Haar measure on  $AN'$.

We expand the above integral in our coordinates introduced above.
%Recall that $A$ normalizes $N'$.  For $s\in A$ let $\psi(s) = \left|\det  \restrict {\Ad (s )}{\Lie(N')}\right |$.  Then  $\psi\colon A\to \R_+$   is a multiplicative homomorphism with the property  that for any Borel  subset $B\subset N'$ and $s\in A$
%$$ |s   B s\inv  |_G = \psi(s) |B|_G.$$
%Given $T>0$ let $$A_T:= \left\{ a^tb^{s} \prod_{c=1}^{m-3} c_i^{s_i} : 0<t<T,
% \delta T/2<s < \delta T,
%0 < s_i <  \sqrt{T}\right\}$$
%and $$N'_T:= \{  u^{r} :  r \in B_{\R^{m-1}} (e^{200T}) \}.$$
%Then for   any bounded continuous function $f\colon \P F \to \R$ integrating against the Riemannian coordinates on $N'$ and $A$  we have
%\note{we never use the measures on $\P F$ because all that reduction is baked in to section 3}
%\begin{align*}
%\int_{\P F} f \ d\td \mu_n &:=
%\frac{1}{|A_{t_n}|_G}
%\int _{A_{t_n}} \frac{\psi (s ) }{|N'_{t_n}|_G}
%\int _{N'_{t_n}} f\big((s u)\cdot (x_n, [v_n])\big) \ d u \  d s.
%%\\
%%&:=
%%{\red Fuck you}
%%\\
%% \frac{1}{t_n}{\red \frac{2}{\delta t_n}\bigg(\frac{1}{\sqrt{t_n}} }\bigg)^{m-3} {\red \frac{1}{|B(e^{200t_n})|}}\int_{0}^{t_n} \int_{\delta t_n /2}^{\delta t_n} \int_{[0, \sqrt{t_n}]^{m-3}}\int_{B_{\R^{m-1}}(e^{200t_n})} f(a^tb^s{a_i}^{s_i}u^r(x_n, v_n)) \  drd{s_i}dsdt
%%% \frac{1}{t_n}{\red \frac{2}{\delta t_n}\bigg(\frac{1}{\sqrt{t_n}} }\bigg)^{m-3} {\red \frac{1}{|B(e^{200t_n})|}}\int_{0}^{t_n} \int_{\delta t_n /2}^{\delta t_n} \int_{[0, \sqrt{t_n}]^{m-3}}\int_{B_{\R^{m-1}}(e^{200t_n})} f(a^tb^s{a_i}^{s_i}u^r(x_n, v_n)) \  drd{s_i}dsdt
% \end{align*}
Then for   any bounded continuous function $f\colon M^\alpha  \to \R$, integrating against our Euclidean parameters $t,s, s_i,$ and $r$ we have
\begin{equation}\label{eq:lazyAaron}
\begin{aligned}
\int\limits_{M^\alpha } &f \ d  \mu_n
\quad \quad
\\
&=  \frac{\displaystyle 2 \int\limits_{0}^{t_n} \int\limits_{\delta t_n /2}^{\delta t_n} \int\limits_{[0, \sqrt{t_n}]^{m-3}}
\int\limits_{B_{\R^{m-1}}(e^{200t_n})} f\left (a^tb^s\prod_{c=1}^{m-3} c_i^{s_i}u^r \cdot (x_n, p(v_n) )\right ) \  dr   \ d{s_i}\ ds\ dt  }{{t_n}{\delta t_n}{\sqrt{t_n}}^{m-3}{|B_{\R^{m-1}}(e^{200t_n})|}}
 %
 \end{aligned}
 \end{equation}

% \begin{equation}\label{eq:lazyAaron}
%\begin{aligned}
%\int\limits_{M^\alpha } &f \ d  \mu_n
%\quad \quad
%\\
%&=  \frac{1}{t_n}{  \frac{2}{\delta t_n}\bigg(\frac{1}{\sqrt{t_n}} }\bigg)^{m-3}\frac{ 1}{|B_{\R^{m-1}}(e^{200t_n})|}
%   \int\limits_{0}^{t_n} \int\limits_{\delta t_n /2}^{\delta t_n} \int\limits_{[0, \sqrt{t_n}]^{m-3}}
%   \\&\quad \quad
%\int\limits_{B_{\R^{m-1}}(e^{200t_n})} f\left (a^tb^s\prod_{c=1}^{m-3} c_i^{s_i}u^r \cdot (x_n, p(v_n) )\right ) \  dr   %\ d{s_i}\ ds\ dt
% %
% \end{aligned}
% \end{equation}

 \noindent where $|B_{\R^{m-1}}(e^{200t_n})|$ denotes the volume of $$B_{\R^{m-1}}(e^{200t_n})= N'_{t_n} \subset  N'$$ with respect to the Euclidean parameters $r$.




%for any compactly supported continuous function $f\colon \P F \to \R$ we have
%\begin{align*}
%\int_{UF} f d\mu_n &:=  \frac{1}{t_n}{\red \frac{2}{\delta t_n}\bigg(\frac{1}{\sqrt{t_n}} }\bigg)^{m-3} {\red \frac{1}{|B(e^{200t_n})|}}\int_{0}^{t_n} \int_{\delta t_n /2}^{\delta t_n} \int_{[0, \sqrt{t_n}]^{m-3}}\int_{B_{\R^{m-1}}(e^{200t_n})} f(a^tb^s{a_i}^{s_i}u^r(x_n, v_n)) \  drd{s_i}dsdt \end{align*}
%
%

%\note{took out the annoying equation because the normalization terms were false}

%{\blue More precisely, for any compactly supported continuous function $f\colon \P F \to \R$ we have
%\begin{align*}
%\int_{UF} f d\mu_n &:=  \frac{1}{t_n}{\red \frac{2}{\delta t_n}\bigg(\frac{1}{\sqrt{t_n}} }\bigg)^{m-3} {\red \frac{1}{|B(e^{200t_n})|}}\int_{0}^{t_n} \int_{\delta t_n /2}^{\delta t_n} \int_{[0, \sqrt{t_n}]^{m-3}}\int_{B_{\R^{m-1}}(e^{200t_n})} f(a^tb^s{a_i}^{s_i}u^r(x_n, v_n)) \  drd{s_i}dsdt \end{align*}

%We also define the probability measures $\mu_n$ on $M^\alpha$  to be the images of the measures $\td \mu _n$ under the canonical projection $\P F \to M^\alpha.$


%\begin{proof}
%%{\blue Since Aaron fucked up the disintegration above, we better write out this proof  }
%\end{proof}

For each $n$, let $\nu_n$ denote the image of the measure $\mu_n$ under the canonical projection from  $M^\alpha$ to $G/\Gamma$.
The following proposition is  shown in the next subsection.
\begin{proposition}\label{maincusps}
There exists $\eta>0$ such that the sequence of measures $\{\nu_n\}$ has uniformly exponentially small mass in the cusp with exponent $\eta$.
\end{proposition}
By the uniform comparability of distances in fibers of $M^\alpha$, this implies the family of measures $\{ \mu_n\}$ has uniformly exponentially  small measure in the  cusp.


By Lemma \ref{lemma:firstexponents}\ref{lazylemmaa} the families of measures    $\{\mu_n\}$  and $\{\nu_n\}$ are precompact families.
 As $F_n$ is a \Folner sequence in a solvable group, we have that
any weak-$*$ subsequential limit  of  $\{\mu _n\}$  or $\{\nu_n\}$   is  $AN'$-invariant.
Moreover, from Theorem \ref{thm:ratner}\ref{ratner3}, it follows that any  weak-$*$ subsequential limit $\nu_\infty$ of $\{\nu_n\}$ is invariant under the group $-N'$ generated by the root groups $U^{m,j}$ for each $1\le j\le m-1$.  Since $N'$ and $-N'$ generate all of $G$, we have that $\nu_\infty$ is a $G$-invariant measure on $G/\Gamma$.




%{\blue \subsection{The measures $\mu_n$ have exponentially smalls  cusps}\label{section:goodcusps}
%
%For each $n$, let $\nu_n$ denote the image of the measure $\mu_n$ under the canonical projection from  $M^\alpha$ to $G/\Gamma$.
%%The following proposition is the main result of the section:
%\begin{proposition}\label{maincuspsold}
%There exists $\eta>0$ such that the sequence of measures $\{\nu_n\}$ have exponentially small mass in the cusp with exponent $\eta$.
%\end{proposition}
%By the uniform comparability of distances in fibers of $M^\alpha$, this implies the family of measures $\{ \mu_n\}$ has uniformly exponentially  small measure in the  cusp.
%
%
%
%The idea behind the proof of this proposition comes from the following. The measures $\nu_n$ can be thought as follows: the maximal paths given in Proposition \ref{lemma:maximal} give us probability measures $\nu'_n$ whose support is at $o(t_n)$ distance of the thick part of  $\Sl(m, \R)/\Sl(m, \Z)$. Averaging such measures $\nu'_n$ by the subset of $N'$ in \ref{folner} we obtain probability measures $\nu''_n$.  One can check that the resulting measures $\nu''_n$ are measures whose support is contained in a $\T^{n-1}$-bundle over the maximal paths and so the support of $\nu''_n$ reminda at $o(t_n)$ distance from the thick part of $\Sl(m, \R)/\Sl(m, \Z)$. The measures $\nu''_n$ are then averaged by the $c_i$s as in \ref{folner} to obtain measures $\nu'''_n$ and as the range for such $c_i$s is quite small $O(\sqrt(t_n))$ the measures $\nu'''_n$ have support at $O(t_n)$ distance from the thick part. Observe that also the measures $\nu'''_n$ are ``foliated'' by large orbits of $N'$ and that orbits of $N'$ correspond to orbits of the full-unstable foliation of the flow $b_s$ defined in \ref{section:folner} and so if we average $\nu'''_n$ by $b_s$ as in \ref{folner} we obtain our measures $\nu_n$, and will expect that the probability measures $\nu_n$ will converge to Haar measure. The choice $\delta/2t_n < s$ in \ref{folner} was made with that in mind. We will now proceed to give the actual proof of proposition \ref{maincusps}.
%
%%We will use the following fact repeatedly:\\
%
%Recall that for any lattice $\Lambda \in \Sl(m,\Z)/Sl_m(\R)$, we have that $$\delta(\Lambda) := \min_{v \in \Lambda \setminus \{0\}} \|v\|$$ and for an element $g \in G/\Gamma$ we denote  $$\delta(g) := \delta (g \Lambda).$$
%We are interested in getting a uniform bound in the integrals: $$\int_{G/\Gamma} \delta^{-\eta}(g) d \mu_n(g\Gamma)$$
%
%
%
%%\begin{proposition}\label{marg} There exists positive constants  $C, K$  such that for any $L \in \Sl(m,\Z)$:  $$ -\frac{1}{K}\log \alpha(L) - C \leq d(L) \leq -K \log \alpha(L) + C$$
%%\end{proposition}
%
%%\begin{proof}
%%Reduction theory, and the fact that the Siegel set is  quasi-isometrically embedded.
%%\end{proof}
%
%We need to bound the following integrals independent of $n$:
%
%%\note{Wrong normalization, and completely wrong integral.  also the $a_i$ were called $c_i$}
%$$ \frac{1}{t_n}{  \frac{2}{\delta t_n}\bigg(\frac{1}{\sqrt{t_n}} \bigg)^{m-3}}\frac{1}{|B(e^{200t_n})|}\int_{0}^{t_n} \int_{\delta t_n /2}^{\delta t_n} \int_{0}^{\sqrt{t_n}}\int_{B_{\R^{m-1}}(e^{200t_n})} \delta(a^tb^s{a_i}^{s_i}u^rx_n)^{-\eta}\  drd{s_i}dsdt$$
%
%In order to show this, it is enough to show that the integrals:  $${   \frac{1}{|B(e^{200t_n})|}}\int_{B(e^{200t_n})}  \delta(a^tb^s{a_i}^{s_i}u^rx_n)^{-\eta} \  dr$$ are bounded independent of $n$ and where  $t,s, s_{i}$ are any values, depending on $n$, satisfying the conditions below equation \eqref{folner}.
%
%We will need to set up some notation. We will denote:
%
%\[u^r =
%\begin{pmatrix}
% r_1 \\ r_2 \\ \vdots \\ r_{m-1}
% \end{pmatrix}
% \]
%
%
%\begin{proposition} For every $t, s_i$ as above, there exists $A_n \in \Sl(m-1, \R)$ and $\gamma_n \in \Sl(m-1,\Z)$ such that:
%
%\begin{enumerate}\label{An}
%
%\item \[ {a_i}^{s_i}a^tx_n =
%\begin{pmatrix}
%  A_n\gamma_n & 0_{m-1 \times 1} \\ 0_{1\times m-1} & 1
%\end{pmatrix}
%\]
%
%\item \[ \lim_{n\to \infty} \log \frac{\|A_n\|}{t_n} = 0 \]
%
%\end{enumerate}
%
%\end{proposition}
%
%\begin{remark} Even though $A_n$ is depending in $s,t, s_i$, we do not point out such dependence to reduce notation. Also, the limit (2) converges uniformly in $s,t,s_i$.
%\end{remark}
%
%\begin{proof}
%This is an immediate consequence of  Proposition \ref{lemma:maximal} and the fact that $0\leq s_i \leq \sqrt{t_n}$.
%\end{proof}
%
%
%We let  $K_n \in \Sl(m-1, \R)$ be such that:
%
%\[ x_n =
%\begin{pmatrix}
%  K_n & 0_{m-1 \times 1} \\ 0_{1\times m-1} & 1
%\end{pmatrix}
%\]
%
%Since maximal paths begin and end in the thick part, we can suppose that the norms of $\|K_n\|$ are bounded by some constant $M_1$ for all $n$.
%
%A straightforward computation shows that:
%
%\[ {a_i}^{s_i}a^tu^rx_n =
%\begin{pmatrix}
%  A_n\gamma_n & A_n\gamma_{n}K_{n}^{-1}(u^r) \\ 0_{1\times m-1} & 1
%\end{pmatrix}
%\]
%
%and therefore:
%
%\[ b^{s}{a_i}^{s_i}a^tu^rx_n =
%\begin{pmatrix}
%  e^{s}A_n\gamma_n & e^{s}A_n\gamma_nK_{n}^{-1}(u^r) \\ 0_{1\times m-1} & e^{-(m-1)s}
%\end{pmatrix}
%\]
%
%%In order to bound the integrals, we need to measure the distance of $b^{s}{a_i}^{s_i}a^tu^rx_n$ to the thick part of $G/\Gamma$ as $u^r$ takes values in $B(e^{200t_n})$. In view of Proposition \ref{marg} we only need to get estimates for $\alpha(b^{s}{a_i}^{s_i}a^tu^rx_n)$.
%
%Let $z = [z_1,z_2,z_3, .... , z_{m}] \in \Z^m$, we need to estimate:
%
%\begin{equation}\label{integer}
%\delta( b^{s}{a_i}^{s_i}a^tu^rx_n) = \inf_{z \in \Z^n \setminus \{0\}} \bigg{\|}
%\begin{pmatrix}
%  e^{s}A_n\gamma_n & e^{s}A_n\gamma_nK_n^{-1}(u^r) \\ 0_{1\times m-1} & e^{-(m-1)s}
%\end{pmatrix}
%\begin{pmatrix}
%  z_1 \\ z_2 \\ \vdots \\ z_m
%\end{pmatrix}
%\bigg{\|}
%\end{equation}
%
%To reduce notation, we define: $$\beta(r) := -\log \delta (a^tb^s{a_i}^{s_i}u^rx_n).$$
%
%We need to estimate: $$\frac{1}{|B(e^{200t_n})|} \int_{B(e^{200t_n})} e^{\eta \beta(r)} dr$$
%
%We will from now on assume that $K_n = \Id$, the general case follows after the change of variables $ u^{r} \to K_n(u^r)$ and the fact that $\|K_n\| \leq M_1$. \\
%
%Let $C = [-1/2,1/2]^{m-1}$ be the  cube in $\R^{m-1}$ centered at $0$. We say that a cube is integral if it is a $\Z^n$ translate of $C$. Let $C_n$ the number of integral cubes contained in $B(e^{200t_n})$, observe that if $u^r - u^{r'} \in \Z^{m-1}$ then $\beta(r) = \beta(r')$, therefore we have that:
%
%$$\int_{B(e^{200t_n})} e^{\eta \beta(r)} dr = C_n \int_{C} e^{\eta \beta(r)} dr + \int_{D_n} e^{\eta \beta(r)} dr$$ Where $D_n$ is the complement in $B(e^{200t_n})$ of the union of all the integral cubes completely contained in $B(e^{200t_n})$.
%
%Because we average over the unipotents in $N'$ first and because we chose to do so on a closed $N'$ orbit, the average over $N'$ is very close
%to the average over a the closed $N'$ orbit which is measurably just the set $C$.  More precisely
%
%
%\begin{claim}
%There exists $M_2 > 0$ independent of $n$ such that for $\eta>0$ small enough the following hold:
%\begin{enumerate}
%\item \[  \frac{1}{|B(e^{200 t_n})|}\int_{D_n} e^{\eta \beta(r)} \ dr \leq M_2 \]
%\item \[ \frac{1}{|B(e^{200 t_n})|} \int_{B(e^{200t_n})} e^{\eta \beta(r)} dr  \leq  \int_C e^{\eta \beta(r)} dr + M_2 \]
%\end{enumerate}
%\end{claim}
%
%\noindent Therefore we only need to show that the integrals $\int_C e^{\eta \beta(r)} dr$ are bounded independent of $n,s,t,s_i$. Therefore Proposition \ref{maincusps} will follow immediately from the following proposition.
%
%\begin{proposition}\label{Tc} Let $c>0$ and let $$T_c  = \{ u^r \in C \text{ such that }\beta(r) > c\}.$$ There exists constants $M_3,M_4 >0$ just depending on $m$ such that for any $n$:
%$$Vol(T_c) \leq M_3e^{-cM_4}$$
%\end{proposition}
%
%
%\begin{proof}
%
%From equation \eqref{integer} it follows that for any $r \in \R^{m-1}$,  if $\beta(r) > c$ there exists a  non-zero $z = [z_1,z_2,z_3, .... , z_{m}] \in \Z^m$ such that:
%
%\[ e^s\Bigg{\|} A_n\gamma_n
%\begin{pmatrix}
%  z_1 \\ z_2 \\ \vdots \\ z_{m-1}
%\end{pmatrix}
% +
% z_mA_n\gamma_n
% \begin{pmatrix}
%  r_1 \\ r_2 \\ \vdots \\ r_{m-1}
%\end{pmatrix}
%\Bigg{\|}  < e^{-c}
%\ \ \text{ and } \ \
%|z_m| < e^{-c} e^{(m-1)s}
%\]
%
%As $\gamma_n \in \Sl_n(\Z)$, this is equivalent to:
%
%\[ e^s\Bigg{\|} A_n \bigg(
%\begin{pmatrix}
%  z_1 \\ z_2 \\ \vdots \\ z_{m-1}
%\end{pmatrix}
% +
% z_m\gamma_n
% \begin{pmatrix}
%  r_1 \\ r_2 \\ \vdots \\ r_{m-1}
%\end{pmatrix}
%\bigg)
%\Bigg{\|} < e^{-c}
%\ \ \text{ and } \ \
%|z_m| < e^{-c} e^{(m-1)s}
%\]
%
%Also as $\gamma_n$ defines a volume-preserving automorphism of $\R^{m-1}/\Z^{m-1}$, we can assume that $\gamma_n = \Id$. For every integer $k$ such that $|k| < e^{-c} e^{(m-1)s}$ Let $T_{c,k}$ be the subset of $C$ such that there exists $(z_1,z_2, \dots, z_{m-1})$  satisfying:
%
%\[ e^s\Bigg{\|} A_n \Bigg(
%\begin{pmatrix}
%  z_1 \\ z_2 \\ \vdots \\ z_{m-1}
%\end{pmatrix}
% +
% k
% \begin{pmatrix}
%  r_1 \\ r_2 \\ \vdots \\ r_{m-1}
%\end{pmatrix}
%\Bigg)
%\Bigg{\|}  < e^{-c}
%\]
%
%Observe that \[T_c \subseteq {\bigcup}_{|k| < e^{-c} e^{(m-1)s}} T_{c,k}\].
%
%\noindent So we are reduced to proving the following
%
%\begin{proposition}\label{rennes} There exists $M_5 \geq 0$ such that If $n$ is large enough, then: $\text|T_{c,k}| < M_5 e^{-(m-1)(s+c)}$
%\end{proposition}
%
%\begin{proof}
%
%If $k = 0$, then as $\delta n/2 <s$ and the $z_i$'s are integral and one of them is not zero, we have:
%
%\[ e^s\Bigg{\|}A_n
%\begin{pmatrix}
%  z_1 \\ z_2 \\ \vdots \\ z_{m-1}
%\end{pmatrix}
%\Bigg{\|}  > e^{\delta n/2}\|A_n\|^{-1}
%\]\note{These is should be a conorms}
%
%Proposition \ref{An} implies that if $n$ is large enough $\|A_n\| \leq e^{\delta n/4}$ and so the term in the left hand side above is greater than one, therefore $T_{c,0} = \emptyset$ for $n$ sufficiently large.
%
%If $k \neq 0$, recall that the map $\text{Mult}_k: \R^{m-1}/\Z^{m-1} \to \R^{m-1}/\Z^{m-1}$ given by multiplication $\text{Mult}_k(x) = kx$ preserves Lebesgue measure and this implies that $|T_{c,k}| = |T_{c,1}|$, therefore we can assume $k = 1$.
%
%Observe now that if $z_i \neq 0$ for some $1 \leq i \leq m-1$, then: \\
%$$ \Bigg{\|}
%\begin{pmatrix}
%  z_1 \\ z_2 \\ \vdots \\ z_{m-1}
%\end{pmatrix} +
%\begin{pmatrix}
%  r_1 \\ r_2 \\ \vdots \\ r_{m-1}
%\end{pmatrix}
%\Bigg{\|} \geq 1/2
% $$
%
% And therefore:
%
%\[ e^s\Bigg{\|} A_n \bigg(
%\begin{pmatrix}
%  z_1 \\ z_2 \\ \vdots \\ z_{m-1}
%\end{pmatrix}
% +
% \begin{pmatrix}
%  r_1 \\ r_2 \\ \vdots \\ r_{m-1}
%\end{pmatrix}
%\bigg)
%\Bigg{\|}  \geq 1/2e^s\|A_n\|^{-1}
%\]
%
%And so again by Proposition \ref{An} and as $s > \frac{\delta}{2}n$ this quantity is bigger than $e^{-c}$ if $n$ is large, therefore $z_i = 0$ for every $1 \leq i \leq m-1$.
%
%We are then interested in the set of $r \in C$ such that:
%
%\[ \Bigg{\|} A_n
% \begin{pmatrix}
%  r_1 \\ r_2 \\ \vdots \\ r_{m-1}
%\end{pmatrix}
%\Bigg{\|}  \leq e^{-s}e^{-c} \]
%
%and so this is the same as the set $A_n^{-1} (B(e^{-s}e^{-c})) \cap C$, therefore if $n$ is large enough:
%$$\text{Vol}(T_{c,1}) =  \text{Vol}(B_{\R^{m-1}}(e^{-s}e^{-c})) \leq M_{5} e^{-(m-1)(s+c)} $$
%for some constant $M_5$ depending only on $m$.
%
%\end{proof}
%
%To finish the proof of Proposition \ref{Tc} and Proposition \ref{maincusps}, observe that Proposition \ref{rennes} implies: $$\text{Vol}(T_c) \leq \sum_k \text{Vol}(T_{c,k}) \leq 2e^{-c} e^{(m-1)s} M_5 e^{-(m-1)(s+c)} \leq M_3e^{-cM_4}$$ for some constants $M_3, M_4$ independent of $n$.
%
%\end{proof}
%
%}



\subsection{Proof of Proposition \ref{maincusps}}
\label{section:goodcusps}
\subsubsection{Heuristics of the proof} The heuristic of the proof is the following.  Observe that for a fixed choice of $t$ and $s_i$ as given by the choice of \Folner set $F_n$, the point $$a^t\prod_{i=1}^{m-3}{c_i}^{s_i}(x_n)$$ lies at sub-linear distance to the thick part of $G/\Gamma$ with respect to $t_n$. Observe that the $N'$-orbit of such point is an embedded $(m-1)$-dimensional torus in $G/\Gamma$. As the range of points in $N'$ in the  \Folner set $F_n$ is quite large,  averaging a Dirac measure of the point $a^t\prod_{i=1}^{m-3}{c_i}^{s_i}(x_n)$ in the $N'$-direction in $F_n$ yields a measure quite close to Haar measure on the $N'$-orbit.

Observe that $N'$-orbits correspond to unstable manifolds for the action of the flow $b^s$ on  $\Sl(m,\R)/\Sl(m,\Z)$. As the action of $b_s$ is known to be mixing, we expect that if  $s$ is sufficiently large, flowing by $b_s$ the $N'$-orbit of $a^t\prod_{i=1}^{m-3}{c_i}^{s_i}(x_n)$ will become equidistributed and in particular it will intersect non-trivially the thick part of $G/\Gamma$. This is the reason why the condition $s > \delta/2 t_n$ is assumed.

While intuition about mixing motivates the proof, we do not use it explicitly.  Instead we use that for large enough $s$, the action of $b_s$ expands the $N'$-orbits in a way that forces them to hit the thick part. We verify this fact by explicit matrix multiplication.

%We have $d(x,\id)$ is uniformly bounded over compact subsets of $\Sl(m,\R)/\Sl(m,\Z)$.  %$x\in \Xt\subset \Sl(2,\R)/\Sl(2,\Z)= H_{1,2}/\Lambda_{1,2}$.%Given our choice of $N'$ above, for any $x\in \Xt$, we have a closed $N'$-orbit in $\Sl(m,\R)/\Sl(m,\Z)$ through $x$.  The $N'$-orbits are the unstable manifolds for the action of $b^s$ on  $\Sl(m,\R)/\Sl(m,\Z)$

%Given any $t_n$, flowing by $b^s$ for any  $\delta/2t_n<s<\delta t_n$, the $N'$-orbit of $x$ become very dense in a compact  thick part $K$ of $\Sl(m,\R)/\Sl(m,\Z)$.
%If we now take  $  x_n\in G/\Gamma$ to be the  points  $x_n\in \Xt\subset H_{1,2}/\Lambda_{1,2}$   satisfying \eqref{eq:cleansingDeluge},  we have that the $a^t$-orbit of $x_n$ stays sub-linearly (in $t_n$) close to $\Xt$ for $0<t<t_n$.  Thus, we expect the $a^t$-image of the $b^s$-image of the $N'$-orbit of $x_n$ intersects  a fixed compact  set $K\subset G/\Gamma$ for any $0<t<t_n$ and  $\delta/2t_n<s<\delta t_n$.  Flowing by $\prod_{i=1}^{m-3}{c_i}^{s_i} $ where each $0<s_i<\sqrt{t_n}$, we still expect the image under
%$$a^tb^s\prod_{i=1}^{m-3}{c_i}^{s_i}$$  of the $N'$-orbit of $x_n$
%to intersect   $K$.

 %As $A$ normalizes $N'$, the image under $a^tb^s\prod_{i=1}^{m-3}{c_i}^{s_i} $ of an $N'$-orbit of $x_n$ is an $N'$-orbit of $a^tb^s\prod_{i=1}^{m-3}{c_i}^{s_i} x_n$.
 %Then, for each $s\in A_{t_n}$, the measure $$\nu_{n,s}:=\frac{1}{|N'_{t_n}|_\ell} \int_{N'_{t_n}} (s u)\cdot \delta_{x_n} \ d u$$ is actually an average over a different piece of the unipotent $N'$ of a delta mass $\delta_{y_n}$ where $y_n$ is in the intersection of the $N'$-orbit of $a^tb^s(\Pi{c_i}^{s_i} ) x_n$ and the compact set $K$.

 As $b^s$ normalizes $N'$, the image under $b^s$ of the $N'$-orbit of $a^t\prod_{i=1}^{m-3}{c_i}^{s_i}(x_n)$ is the $N'$-orbit of a point $y_n$ in the thick part of $G/\Gamma$.
 Having in mind the quantitative non-divergence of unipotent flows as in the proof Proposition \ref{prop:bananas},   the $N'$-orbits  have uniformly (over all $n, s_i, $ and $t$) exponentially small mass in the cusps whence so do the measures $\nu_n$. % being an average (over $s_i$ and $t$) of such orbits.

  %$$\nu_n = \frac{1}{|A_{t_n}|_\ell} %\frac 1{\int _{A_{t_n}}  \psi(s) \ d s}
  %\int _{A_{t_n}} % \psi(s)
   %\nu_{n,s} \ ds.$$

 The following  proof of  Proposition \ref{maincusps} uses   explicit matrix calculations and estimates to verify  these heuristics.


%The idea behind the proof of this proposition comes from the following. The measures $\nu_n$ can be thought as follows: the maximal paths given in Proposition \ref{lemma:maximal} give us probability measures $\nu'_n$ whose support is at $o(t_n)$ distance of the thick part of  $\Sl(m, \R)/\Sl(m, \Z)$. Averaging such measures $\nu'_n$ by the subset of $N'$ in \ref{folner} we obtain probability measures $\nu''_n$.  One can check that the resulting measures $\nu''_n$ are measures whose support is contained in a $\T^{n-1}$-bundle over the maximal paths and so the support of $\nu''_n$ reminda at $o(t_n)$ distance from the thick part of $\Sl(m, \R)/\Sl(m, \Z)$. The measures $\nu''_n$ are then averaged by the $c_i$s as in \ref{folner} to obtain measures $\nu'''_n$ and as the range for such $c_i$s is quite small $O(\sqrt(t_n))$ the measures $\nu'''_n$ have support at $O(t_n)$ distance from the thick part. Observe that also the measures $\nu'''_n$ are ``foliated'' by large orbits of $N'$ and that orbits of $N'$ correspond to orbits of the full-unstable foliation of the flow $b_s$ defined in \ref{section:folner} and so if we average $\nu'''_n$ by $b_s$ as in \ref{folner} we obtain our measures $\nu_n$, and will expect that the probability measures $\nu_n$ will converge to Haar measure. The choice $\delta/2t_n < s$ in \ref{folner} was made with that in mind. We will now proceed to give the actual proof of proposition \ref{maincusps}.

%We will use the following fact repeatedly:\\
\subsubsection{Proof of Proposition \ref{maincusps}}
Recall that we identify   each coset $$g\Sl(m,\Z) \in \Sl(m,\R)/\Sl(m,\Z)$$ with a unimodular lattice
$\Lambda_g:= g\cdot \Z^m$ in $\R^m$.  We define the systole of a unimodular lattice $\Lambda\subset \R^m$ to be  $$\delta(\Lambda) := \min_{v \in \Lambda \setminus \{0\}} \|v\|$$ and for an element $g\in \Sl(m,\R)$,   we denote by $\delta(g)$ the systole $$\delta(g) = \delta(g\cdot \Z^m).$$ From \eqref{eq:ploy}, to prove Proposition \ref{maincusps} it is sufficient to find $\eta>0$ so that the  integrals $$\int_{G/\Gamma} \delta(g)^{-\eta}  \ d \nu_n(g\Gamma)$$
are uniformly bounded in $n$.

%Given $x\in M^\alpha$, let $\hat x\in G/\Gamma$ be the image of $x$ under the canonical projection $M^\alpha\to G/\Gamma$.
As discussed in the above heuristic, from \eqref{eq:lazyAaron} to bound the integrals $\int_{G/\Gamma} \delta^{-\eta}(g)  \ d \nu_n(g\Gamma)$ it is sufficient to show each integral $${\frac{1}{|B(e^{200t_n})|}}\int_{B(e^{200t_n})}  \delta(a^tb^s (\Pi {c_i}^{s_i})u^r   x_n)^{-\eta} \  dr$$
is uniformly bounded in $n$ and in all parameters $t,s, s_i$ for $0<t<t_n,$
$ \delta t_n/2<s < \delta t_n,$ and $
0 < s_i <  \sqrt{t_n}.$
Recall here that $  x_n\in G/\Gamma$ are the points  $x_n\in \Xt\subset H_{1,2}/\Lambda_{1,2}$   satisfying \eqref{eq:cleansingDeluge} used in the construction of the measures $\mu_n$.
%Recall that $\hat x_n$ is contained in $\Xt\subset \Sl(2,\R)/\Sl(2,\Z)= H_{1,2}/\Lambda_{1,2}$.


%\noted{Need to specify FD here. really using FD for modular surface????}
{\blue We have $H_{1,2}$ is canonically embedded in $\Sl(m,\R)$.   Given $x_n\in H_{1,2}/\Lambda_{1,2}$, let $$\td x_n\in H_{1,2}\subset \Sl(m,\R)$$ denote the element   mapping to $  x_n$ under the map $H_{1,2}\to H_{1,2}/\Lambda_{1,2}$ which is contained in a fundamental domain contained in the Dirichlet domain $\mathcal D\subset \SL(2,\R)$  in \eqref{eq:dirc} in Section \ref{sec:lolo}.}  Let $\|\cdot \|$ denote the operator norm on $\Sl(m,\R)$ and $m(\cdot )$ the associated conorm.

%\begin{proposition}\label{marg} There exists positive constants  $C, K$  such that for any $L \in \Sl(m,\Z)$:  $$ -\frac{1}{K}\log \alpha(L) - C \leq d(L) \leq -K \log \alpha(L) + C$$
%\end{proposition}

%\begin{proof}
%Reduction theory, and the fact that the Siegel set is  quasi-isometrically embedded.
%\end{proof}

%We need to bound the following integrals independent of $n$:
%

%$$ \frac{1}{t_n}{\red \frac{2}{\delta t_n}\bigg(\frac{1}{\sqrt{t_n}} \bigg)^{m-3}}\frac{1}{|B(e^{200t_n})|}\int_{0}^{t_n} \int_{\delta t_n /2}^{\delta t_n} \int_{0}^{\sqrt{t_n}}\int_{B_{\R^{m-1}}(e^{200t_n})} \delta(a^tb^s{c_i}^{s_i}u^rx_n)^{-\eta}\  drd{s_i}dsdt$$
%
%In order to show this, it is enough to show that the integrals:  $${\red  \frac{1}{|B(e^{200t_n})|}}\int_{B(e^{200t_n})}  \delta(a^tb^s{c_i}^{s_i}u^rx_n)^{-\eta} \  dr$$ are bounded independent of $n$ and where  $t,s, s_{i}$ are any values, depending on $n$, satisfying the conditions below equation \eqref{folner}.

%We will need to set up some notation. We will denote:

%\[u^r =
%\begin{pmatrix}
% r_1 \\ r_2 \\ \vdots \\ r_{m-1}
% \end{pmatrix}
% \]


\begin{claim} For every $n$,  $t\le t_n$, and $ 0\le s_i\le \sqrt{t_n}$ as above, there exist $$\text{$A_{n}= A_{n,t,s_1, \dots , s_{m-3}} \in \Sl(m-1, \R)$ and $\gamma_{n} = \gamma_{n,t,s_1, \dots , s_{m-3}} \in \Sl(m-1,\Z)$}$$ such that:

\begin{enumerate}\label{An}

\item $ \displaystyle a^t \prod_{i=1}^{m-3}{c_i}^{s_i} \td x_n =
\begin{pmatrix}
  A_n\gamma_n & 0_{m-1 \times 1} \\ 0_{1\times m-1} & 1
\end{pmatrix}
$

\item $ \displaystyle\lim_{n\to \infty} \sup_{t\le t_n,   0\le s_i\le \sqrt{t_n}}  \frac{\log\|A_n\|}{t_n} = 0 $ and  $ \displaystyle\lim_{n\to \infty} \inf_{t\le t_n,   0\le s_i\le \sqrt{t_n}}  \frac{\log(m(A_n))}{t_n} = 0 $


\end{enumerate}

\end{claim}

%\begin{remark} Even though $A_n$ is depending in $s,t, s_i$, we do not point out such dependence to reduce notation. Also, the limit (2) converges uniformly in $s,t,s_i$.
%\end{remark}

\begin{proof}(1) is immediate from construction.  The uniform limit in (2) follows from Lemma \ref{lemma:maximal}(2), equation \eqref{eq:easy}, and the fact that the  $s_i$ are chosen so that $0\leq s_i \leq \sqrt{t_n}$ whence $$\frac{d(  x_n , a^t  (\Pi {c_i}^{s_i}) \cdot   x_n )}{t_n}\to 0$$ uniformly in $t, s_i$.
\end{proof}

In the remainder, we will suppress the dependence of  choices on $t, s, s_i$.
We take  $K_n \in \Sl(m-1, \R)$ be such that
\[ \td x_n =
\begin{pmatrix}
  K_n & 0_{m-1 \times 1} \\ 0_{1\times m-1} & 1
\end{pmatrix}.
\]
{\blue Note that $K_n$ differs from the identity only in the first two rows and columns.}
Since each $x_n$ is contained in  $  \Xt$,  we have that the matrix norm  and conorm  $\|K_n\|$ and $m(K_n)$ are bounded above and below, respectively,  by   constants $M_1$ and $\frac 1 {M_1}$ independent of $n$.

Recall $r$ denotes a vector in $\R^{m-1}$ and $u^r\in \Sl(m,\R)$ is the unipotent element given by \eqref{eq:formerlyknownas}.
%$$
%u^r= \left(\begin{array}{ccccc}1  &  0  & 0  & \dots  & r_1  \\  & 1  &   0 & \dots  & r_2  \\  &   &  \ddots &   &  \vdots  \\  &   &   &  1 &  r_{m-1} \\  &   &   &   &1  \end{array}\right)$$
Matrix computation yields
\[ a^t  (\Pi {c_i}^{s_i})u^r \td  x_n =
\begin{pmatrix}
  A_n\gamma_n & A_n\gamma_{n}K_{n}^{-1}r \\ 0_{1\times m-1} & 1
\end{pmatrix}
\]
whence
\[ b^{s}(\Pi {c_i}^{s_i})a^tu^r \td x_n =
\begin{pmatrix}
  e^{s}A_n\gamma_n & e^{s}A_n\gamma_nK_{n}^{-1}r \\ 0_{1\times m-1} & e^{-(m-1)s}
\end{pmatrix}.
\]

%In order to bound the integrals, we need to measure the distance of $b^{s}{a_i}^{s_i}a^tu^rx_n$ to the thick part of $G/\Gamma$ as $u^r$ takes values in $B(e^{200t_n})$. In view of Proposition \ref{marg} we only need to get estimates for $\alpha(b^{s}{a_i}^{s_i}a^tu^rx_n)$.

%Let $z = [z_1,z_2,z_3, .... , z_{m}] \in \Z^m$
We have \begin{equation}\label{integer}
\begin{split}
\delta( b^{s}(\Pi {c_i}^{s_i})a^tu^r \td x_n) &= \delta( b^{s}(\Pi {c_i}^{s_i})a^tu^r    x_n) \\&= \inf_{z \in \Z^m \setminus \{0\}} \bigg{\|}
\begin{pmatrix}
  e^{s}A_n\gamma_n & e^{s}A_n\gamma_nK_n^{-1}r \\ 0_{1\times m-1} & e^{-(m-1)s}
\end{pmatrix}
z
%\begin{pmatrix}
%  z_1 \\ z_2 \\ \vdots \\ z_m
%\end{pmatrix}
\bigg{\|}
\end{split}
\end{equation}
 To reduce notation, for fixed $t,s,$ and $s_i$   define $$\beta(r) :=- \log \delta (a^tb^s(\Pi {c_i}^{s_i})u^r \td x_n).$$ We aim to find an upper bound of $$\frac{1}{|B(e^{200t_n})|} \int_{B(e^{200t_n})} e^{\eta \beta(r)} dr$$ that is independent of $n$ and $ t,s,$ and $s_i$.

%{\blue We will from now on assume that $K_n = \Id$, the general case follows after the change of variables $ u^{r} \to K_n(u^r)$
%and the fact that $\|K_n\| \leq M_1$. } {\red This looks suspicious }

Observe that if $r - {r'} $ differ by an element of the unimodular lattice $ K_n \Z^{m-1}\subset \R^{m-1}$, then $\beta(r) = \beta(r')$.
{\blue
Indeed, if $r'= r+ K_n z'$ for some $z'= (z_1', \dots, z_{m-1}')\in \Z^{m-1}$ and if $z\in \Z^m\sm\{0\}$ is $z= (z_1, \dots , z_m)$ then
$$\begin{pmatrix}
  e^{s}A_n\gamma_n & e^{s}A_n\gamma_nK_n^{-1}r' \\ 0_{1\times m-1} & e^{-(m-1)s}
\end{pmatrix}z=\begin{pmatrix}
  e^{s}A_n\gamma_n & e^{s}A_n\gamma_nK_n^{-1}r \\ 0_{1\times m-1} & e^{-(m-1)s}
\end{pmatrix} \td z
$$
where $\td z = (z_1 + z_m z'_1, \dots, z_{m-1} + z_m z'_{m-1}, z_m)\in \Z^m\sm\{0\}.$
} %\noted{extra just. added}
Thus   we have that $\beta\colon \R^{m-1}\to (0,\infty)$ descends to a function on the torus $\R^{m-1}/(K_n \Z^{m-1})$.

Let $D_n = K_n \cdot ( [-1/2,1/2]^{m-1})$ be a fundamental domain for this torus in $\R^{m-1}$ centered at $0$.
Let $c_n$ denote the number of $(K_n\Z^{m-1})$-translates of $D_n$ that intersect $B(e^{200t_n})$.
Then, if $t_n$ is sufficiently large we have that
\begin{equation*}
 \frac{1}{|B(e^{200 t_n})|}\int_{B(e^{200t_n})} e^{\eta \beta(r)} \ dr
  \le
  \frac{1}{|B(e^{200 t_n})|}c_n \int_{D_n} e^{\eta \beta(r)} \ dr
  \le 2\int_{D_n} e^{\eta \beta(r)} \ dr
 \end{equation*}
 The first inequality follows from inclusion.  The second inequality follows from the fact that the perimeter of $B(q)$ grows like $q^{m-2}$, the volume of $B(q)$ grows like $q^{m-1}$, and the domains $D_n = K_n  \cdot ( [-1/2,1/2]^{m-1})$ have uniformly comparable geometry over $n$.





% We say that a cube is integral if it is a $\Z^n$ translate of $C$. Let $C_n$ the number of integral cubes contained in $B(e^{200t_n})$, observe that if $u^r - u^{r'} \in \Z^{m-1}$ then $\beta(r) = \beta(r')$, therefore we have that:
%
%$$\int_{B(e^{200t_n})} e^{\eta \beta(r)} dr = C_n \int_{C} e^{\eta \beta(r)} dr + \int_{D_n} e^{\eta \beta(r)} dr$$ Where $D_n$ is the complement in $B(e^{200t_n})$ of the union of all the integral cubes completely contained in $B(e^{200t_n})$.
%
%Because we average over the unipotents in $N'$ first and because we chose to do so on a closed $N'$ orbit, the average over $N'$ is very close
%to the average over a the closed $N'$ orbit which is measurably just the set $C$.  More precisely
%{\blue $N'$-orbit in $G/\Gamma$ closed}
%
%\begin{claim}\note{There is something pretty delicate here}
%There exists $M_2 > 0$ independent of $n$ such that for $\eta>0$ {\red small enough} the following hold:
%\begin{enumerate}
%\item \[  \frac{1}{|B(e^{200 t_n})|}\int_{D_n} e^{\eta \beta(r)} \ dr \leq M_2 \]
%\item \[ \frac{1}{|B(e^{200 t_n})|} \int_{B(e^{200t_n})} e^{\eta \beta(r)} dr  \leq  \int_C e^{\eta \beta(r)} dr + M_2 \]
%\end{enumerate}
%\end{claim}

%\noindent Therefore we only need to show that the integrals $\int_C e^{\eta \beta(r)} dr$ are bounded independent of $n,s,t,s_i$. Therefore Proposition \ref{maincusps} will follow immediately from the following proposition.
It remains to estimate $\int_{D_n} e^{\eta \beta(r)} \ dr$.  Given $c>0$ and fixed  $n, t, s_i, $ and $s$  we define $$T_c  = \{  r \in D_n:\beta(r) > c\}.$$
Proposition \ref{maincusps}   follows immediately  from the estimate in the following lemma.
\begin{lemma}\label{Tc} % Given $c>0$  let %$$T_c  = \{ u^r \in D_n:\beta(r) > c\}.$$
There exists constants $M_3,M_4 >0$, independent of $n, t, s_i, $ and $s$,    such that
$$ |T_c| \leq M_3e^{-cM_4}.$$
\end{lemma}
Indeed, if $\eta\inv > { M_4}$ then
\begin{align*}\int_{D_n} e^{\eta \beta(r)} \ dr
&=\int_0^\infty |\{ r\in D_n: e^{\eta \beta(r) }\ge \tau\}|  \ d \tau
\le 1+ \int_1^\infty |\{ r\in D_n: e^{\eta \beta(r) }\ge \tau\}|  \ d \tau\\
&=1+ \int_1^\infty |\{ r\in D_n:   \beta(r) \ge \log\big(\tau^{\frac 1 \eta}\big)\}| \ d \tau
%&= 1 + \int_1^\infty |\{ r\in D_n:   \beta(r) \ge \log\big(\tau^{\frac 1 \eta}\big)\} |\ d \tau\\
=1+\int_1^\infty | T_{\log\big(\tau^{\frac 1 \eta}\big)}| \ d \tau\\
&= 1 + \int_1^\infty M_3  \tau^{\frac { -M_4} \eta} \ d \tau <\infty
\end{align*}
 and Proposition \ref{maincusps} follows.


\begin{proof}[Proof of Lemma \ref{Tc}]
From  \eqref{integer}, given any   $r \in \R^{m-1}$, if $\beta(r) > c$ then there exists a  non-zero $z = (z_1,z_2,z_3, .... , z_{m}) \in \Z^m$ such that
\[ e^s\left\| A_n\gamma_n
%\begin{pmatrix}
%  z_1 \\ z_2 \\ \vdots \\ z_{m-1}
%\end{pmatrix}
(z_1, \dots, z_{m-1})
 +
 z_mA_n\gamma_n K_n\inv
r
\right\|  < e^{-c}
\ \ \text{ and } \ \
|z_m| < e^{-c} e^{(m-1)s}
\]
which (as $\gamma_n \in \Sl(m-1, \Z)$) holds if and only if there is a non-zero $z = (z_1,z_2,z_3, .... , z_{m}) \in \Z^m$
\begin{equation}\label{eq:pleasetakeafreshmanlogiccourse} e^s\left\| A_n \Big(
%\begin{pmatrix}
%  z_1 \\ z_2 \\ \vdots \\ z_{m-1}
%\end{pmatrix}
(z_1, \dots, z_{m-1})
 +
 z_mK_n\inv (K_n\gamma_n K_n\inv) r
% z_m\gamma_n K_n\inv r
% \begin{pmatrix}
%  r_1 \\ r_2 \\ \vdots \\ r_{m-1}
%\end{pmatrix}
\Big)
\right\| < e^{-c}
\ \ \text{ and } \ \
|z_m| < e^{-c} e^{(m-1)s}
\end{equation}
As $K_n\gamma_n K_n\inv$ induces a volume-preserving automorphism of $\R^{m-1}/(K_n\Z^{m-1})$, the set of $r\in D_n$ satisfying \eqref{eq:pleasetakeafreshmanlogiccourse}  {\blue for some $z\in \Z^m$} has the same measure as the set of $r\in D_n$ satisfying
%\begin{equation}\label{eq:ThisiswhatitsoundslikeWhenPostdocsCry}
$$e^s\left\| A_n \Big(
%\begin{pmatrix}
%  z_1 \\ z_2 \\ \vdots \\ z_{m-1}
%\end{pmatrix}
(z_1, \dots, z_{m-1})
 +
 z_mK_n\inv   r
% z_m\gamma_n K_n\inv r
% \begin{pmatrix}
%  r_1 \\ r_2 \\ \vdots \\ r_{m-1}
%\end{pmatrix}
\Big)
\right\| < e^{-c}
\ \ \text{ and } \ \
|z_m| < e^{-c} e^{(m-1)s}$$
{\blue for some $z\in \Z^m.$}
%\end{equation}

For every integer $k$ satisfying  $|k| < e^{-c} e^{(m-1)s}$, let $T_{c,k}$ be the subset of $r\in D_n$ such that there exists $(z_1,z_2, \dots, z_{m-1})\in \Z^{m-1}$  satisfying
$$e^s\left\| A_n \Big(
%\begin{pmatrix}
%  z_1 \\ z_2 \\ \vdots \\ z_{m-1}
%\end{pmatrix}
(z_1, \dots, z_{m-1})
 +
k K_n\inv   r
% z_m\gamma_n K_n\inv r
% \begin{pmatrix}
%  r_1 \\ r_2 \\ \vdots \\ r_{m-1}
%\end{pmatrix}
\Big)
\right\|< e^{-c}.$$
 Then $|T_c| \leq   \sum_{|k| < e^{-c} e^{(m-1)s}} |T_{c,k}|.$   Thus the estimate reduces to the following.


%\[ e^s\left\| A_n \Big(
%\begin{pmatrix}
%  z_1 \\ z_2 \\ \vdots \\ z_{m-1}
%\end{pmatrix}
% +
% k
% \begin{pmatrix}
%  r_1 \\ r_2 \\ \vdots \\ r_{m-1}
%\end{pmatrix}
%\Bigg)
%\Bigg{\|}  < e^{-c}
%\]

%Observe that \[T_c \subseteq {\bigcup}_{|k| < e^{-c} e^{(m-1)s}} T_{c,k}\].

%\noindent So we are reduced to proving the following

\begin{claim}\label{rennes} There exists $M_5 \geq 0$ such that $|T_{c,k} |< M_5 e^{-(m-1)(s+c)}$  for all $n$ sufficiently  large.
\end{claim}
%\note{ReWrote this as I didn't understand the reduction to $K_n = \id$.}
\begin{proof}Recall that $\delta t_n/2 <s$.
If $k = 0$ then,   for any non-zero $(z_1, \dots,  z_{m-1})\in \Z^{m-1}$, we have \[ e^s\left \| A_n
%\begin{pmatrix}
%  z_1 \\ z_2 \\ \vdots \\ z_{m-1}
%\end{pmatrix}
(z_1, \dots,  z_{m-1})
\right \| > e^{\delta t_n/2} m(A_n).
\]
From Claim \ref{An}(2),  if $n$ is large enough then  $m(A_n)\ge e^{-\delta t_n/4}$ and so the term in the left hand side above is greater than one, therefore $T_{c,0} = \emptyset$ for $n$ sufficiently large.

If $k \neq 0$,
observe that the map $M_k\colon \R^{m-1}/K_n \Z^{m-1} \to \R^{m-1}/K_n\Z^{m-1}$ given by $$r+ K_n \Z^{m-1}\mapsto kr+ K_n \Z^{m-1}$$ preserves the Lebesgue measure on $\R^{m-1}/K_n \Z^{m-1}.$  In particular, this implies that $T_{c,k}$ and $T_{c,1}$ have the same volume.

%\noted{Fixed counting argument}
{\blue
We thus take $k = 1$.
Note that $K_n\inv D_n=[-1/2,1/2]$.   There is a $L\ge 1$, depending only on $m-1$, such that   the set
	$$Q=\{z'\in \Z^{m-1} :  |z'+r|\le 1 \text{ for some $r\in K_n\inv D_n$ }\}$$
	has cardinality at most $L$.    From Claim \ref{An}(2),  if $n$ is large enough then  $m(A_n)\ge e^{-\delta t_n/4}$ whence for all $ (z_1, \dots, z_{m-1}) \in \Z^{m-1} \sm Q$ and all $r\in D_n$,
	$$e^s\left\| A_n \Big(
 (z_1, \dots, z_{m-1}) + K_n\inv   r\Big)
\right\|  \ge 1.$$
We thus need only consider $(z_1, \dots, z_{m-1})\in Q$.
}


%{\red We thus take $k = 1$.
%Then there is a $L\ge 1$ depending only on $M_1$ (which is bounded on $\Xt$) such that the number of  $r\in D_n$ such that   $K_n\inv r\in \Z^{m-1}$ is bounded above by $L$.}

Given a fixed $z= (z_1, \dots,  z_{m-1})\in Q\subset  \Z^{m-1}$, using that $K_n\in \Sl(m-1, \R)$
we have $$\left | \{r\in \R^{m-1}  : \|z+K_n\inv  r\|\le \ell \} \right|\le (2\ell )^{m-1}$$
whence $$\left |\{ r\in \R^{m-1}  :  \|(z_1, \dots, z_{m-1}) + K_n\inv   r \| \le e^{-c}  \} \right| \le 2^{m-1} e^{-c(m-1)}.$$
%Since for $n$ sufficiently large we have
%\begin{align*}e^s\left\| A_n \Big(
% \|(z_1, \dots, z_{m-1}) + K_n\inv   r\Big)
%\right\|
%&\ge e^s m(A_n) \|(z_1, \dots, z_{m-1}) + K_n\inv   r\|  \\
%&\ge e^{t_n\delta/2}  e^{-t_n\delta/4} \|(z_1, \dots, z_{m-1}) + K_n\inv   r\|  \\
%&\ge   \|(z_1, \dots, z_{m-1}) + K_n\inv   r\|
%\end{align*}
%we have $$|T_{c,k}|\le L 2^{m-1} e^{-c(m-1)}.$$



If $r\in T_{c,1}$ so that
$$e^s\left\| A_n \Big(
 (z_1, \dots, z_{m-1}) + K_n\inv   r\Big)
\right\|  \le e^{-c}$$
then
\begin{equation}\label{eq:thisiswhatissoundslikewhenpostdocscry}\left\| A_n \Big(
 (z_1, \dots, z_{m-1}) + K_n\inv   r\Big)
\right\|  \le e^{-c-s}.\end{equation}
Since $A_n\in \Sl(m-1,\R)$ the set of $r\in \R^{m-1}$ satisfying  \eqref{eq:thisiswhatissoundslikewhenpostdocscry}
has the same volume as the set of $r\in \R^{m-1}$ satisfying
\[\left\|
 (z_1, \dots, z_{m-1}) + K_n\inv   r \right\|  \le e^{-c-s}. \]
{\blue  It follows that $|T_{c,1}|\le   2^{m-1} L e^{-(s+c)(m-1)}$.}\end{proof}
%&\ge e^s m(A_n) \|(z_1, \dots, z_{m-1}) + K_n\inv   r\|  \\
%&\ge e^{t_n\delta/2}  e^{-t_n\delta/4} \|(z_1, \dots, z_{m-1}) + K_n\inv   r\|  \\
%&\ge   \|(z_1, \dots, z_{m-1}) + K_n\inv   r\|
%\end{align*}
%%We thus have $$|T_{c,k}|\le L 2^{m-1} e^{-(c+s)(m-1)}.$$
%%
%%
%%$$ \Bigg{\|}
%%(
%%  z_1, z_2, \dots,  z_{m-1}
%%) - K_n r
%%  \ge 1/L
%% $$
%%
%% And therefore:
%%
%%\[ e^s\Bigg{\|} A_n \bigg(
%%\begin{pmatrix}
%%  z_1 \\ z_2 \\ \vdots \\ z_{m-1}
%%\end{pmatrix}
%% +
%% \begin{pmatrix}
%%  r_1 \\ r_2 \\ \vdots \\ r_{m-1}
%%\end{pmatrix}
%%\bigg)
%%\Bigg{\|}  \geq 1/2e^s\|A_n\|^{-1}
%%\]
%%
%%And so again by Proposition \ref{An} and as $s > \frac{\delta}{2}n$ this quantity is bigger than $e^{-c}$ if $n$ is large, therefore $z_i = 0$ for every $1 \leq i \leq m-1$.
%%
%%We are then interested in the set of $r \in C$ such that:
%%
%%\[ \Bigg{\|} A_n
%% \begin{pmatrix}
%%  r_1 \\ r_2 \\ \vdots \\ r_{m-1}
%%\end{pmatrix}
%%\Bigg{\|}  \leq e^{-s}e^{-c} \]
%%
%%and so this is the same as the set $A_n^{-1} (B(e^{-s}e^{-c})) \cap C$, therefore if $n$ is large enough:
%%$$\text{Vol}(T_{c,1}) =  \text{Vol}(B_{\R^{m-1}}(e^{-s}e^{-c})) \leq M_{5} e^{-(m-1)(s+c)} $$
%%for some constant $M_5$ depending only on $m$.
%%
%\end{proof}

To finish the proof of Lemma \ref{Tc}, from Claim \ref{rennes} we have $$|T_c| \leq \sum_{|k| < e^{-c} e^{(m-1)s}} |T_{c,k}| \leq (2e^{-c} e^{(m-1)s}+1) M_5 e^{-(m-1)(s+c)} \leq M_3e^{-cM_4}$$ for some constants $M_3, M_4$ independent of $n$.
\end{proof}




\subsection{Positive Lyapunov exponents for limit measures}
\label{section:ANLyapunov}
To deduce Proposition \ref{prop:maybeweshouldstatethemainresultatsomepoint}, having assumed that $\chi_{\max}$ in \eqref{eq:fanorlamps} is non-zero, we show that  any {weak-${*}$ subsequential limit of the sequence of measures $\{\mu_n\}$ has a positive Lyapunov exponent from which we derive a contradiction.


Recall from Section \ref{section:folner} that we  fixed sequences $x_n, v_n, t_n$ such  that  $\|D_{x_n}a^{t_n}(v_n)\| \geq e^{ \lambda t_n}$ for some fixed $\lambda > 0$.
Let $\calA\colon G\times F\to F$ be the fiberwise derivative cocycle over the action of $G$ on $M^\alpha$.


Our main result is the following.
\begin{proposition}\label{positiveexponent}
For any weak-$*$ subsequential limit $\mu_\infty$ of $\{\mu_n\}$ we have
$$\lambda_{\top, a, \mu_\infty,\calA}\ge \lambda/2 >0.$$
\end{proposition}


%If $n$ sufficiently large, we have that
% $$\int \log  \|\calA(a ^{t_n},x) \|  \ d \mu_n(x)= \int \log  \|D_{x_n}a ^{t_n} \|_\Fib   \ d \mu_n(x) \geq \epsilon t_n.$$
%
%
%%$\int \|\calA ( a^{t_n},x) \|  \ d \mu_n(x) \geq e^{\epsilon t_n}$.
%
%
% It follows from   Lemma \ref{lemma:firstexponents}  that for a weak-$*$ limit $\mu_\infty$ of $\{\mu_n\}$,


%  has a nonzero exponent




%{\blue Our main result is the following.
%\begin{proposition}\label{positiveexponentold} If $n$ sufficiently large, we have that  $\int_ {UF} \Phi \ d\td mu_n > \lambda/4.$
%\end{proposition} {\red lets state the main conclusion since the reduction is in earlier proof}
%A standard argument shows that the $a^t$ action on $M^\alpha$ has a
%}

%First, we will show that there is subexponential growth of derivatives along the $N'$-orbits of the sequence of points $x_n$. This is a consequence of the subexponential growth of unipotents proved in Proportion \ref{unipotentisgood} and the fact that the $N'$-orbits of $x_n$ are tori $\T^{m-1}$  so the cocycle $\beta$ has values which are unipotent elements in $\Z^{n-1}$. More precisely, we have the following

We first show that averaging over $N'$ does not change the Lyapunov exponents of the cocycle.

\begin{claim}\label{siegelunipotent} Given any $\e> 0$ there is $t_{\e}>0$ such that for any $t \geq t_{\e}$ and any $r \in B_{\R^{m-1}}(e^{t})$ we have  $$\| {D_{x}u^r}\|_{\Fib} \leq e^{\e t}$$ %\note{  need to redo with the  conorm? maybe not}
for any $x\in \Xt$.
\end{claim}
%{\red there is a simpler proof of this}
\begin{proof}
 Recall that the $N'$-orbit of any $x\in X:= H_{1,2}/\Lambda_{1,2}\subset \Sl(m,\R)/\Sl(m,\Z)$ is a closed torus.    Then the $N'$-orbit of $\Xt$ is compact.
{\blue Recall our fixed fundamental domain $\calF\subset \wtd {\mathcal D}$ contained in the Dirichlet  domain  $\wtd {\mathcal D}$ of the identity for $\Sl(m,\R)/\Sl(m,\Z)$ as discussed in Section \ref{sec:lolo}.} Given $x\in  \Sl(m,\R)/\Sl(m,\Z)$, let $\td x$ be the lift of $x$ in $\calF$.  Let  $\wtd X _{\mathrm{thick}}\subset H_{1,2}\cap \calF$ denote the lift of $\Xt$ to  $\calF$ and let $\hat X _{\mathrm{thick}}$ be the lift of the orbit $N'\Xt$ to $\calF$.  {\blue As discussed in Section \ref{sec:lolo}, we have that $\wtd X _{\mathrm{thick}}$ is contained in the Dirichlet domain $\mathcal D$ of the identity for the  $\Lambda_{1,2}$-action on $H_{1,2}$.  Moreover, $\hat X _{\mathrm{thick}}$ is precompact in $\Sl(m,\R)$.}

Fix $r\in \R^{m-1}$ and $x\in \Xt$.   Write $$\td x = \left(\begin{array}{cc}K& 0 \\0 & 1\end{array}\right)$$ for some $K\in \Sl(m-1, \R)$; we have $\|K\|\le M_1$ and $m(K) \ge \frac 1 {M_1}$ for all $x\in \Xt$.     The deck group of the orbit $N'\td x$ is $$\td x   \{ u^z : z\in \Z^{m-1}\} \td x\inv = \{  u^{K\cdot z} : z\in \Z^{m-1}\} .$$  Thus, there is $z\in \Z^{m-1}$ and $r'\in \R^{m-1}$ such that
$$u^r \td x = \left(\begin{array}{cc}K& r \\0 & 1\end{array}\right)= \left(\begin{array}{cc}K& r' +K z\\0 & 1\end{array}\right)= \left(\begin{array}{cc}1& r' \\0 & 1\end{array}\right) \left(\begin{array}{cc}K& 0\\0 & 1\end{array}\right) \left(\begin{array}{cc}1& z \\0 & 1\end{array}\right) = u^{r'} \td x u^z$$
and $u^{r'} \td x\in \hat X _{\mathrm{thick}}$.
Then
$$\|D_{x}u^r\|\le \|D_{x}\td x\inv \| _{\Fib} \cdot  \| D_{\Id \Gamma} u^z\| _{\Fib}  \cdot \| D_{\Id\Gamma} u^{r'}\td x\| _{\Fib}.$$
Since $\td x$ and $ u^{r'}\td x$ are in precompact sets, the first and last terms of the right hand side are uniformly bounded in $r$ and $x\in \Xt$.

There exists some $C$ such that
$$\| D_{\Id \Gamma} u^z\| _{\Fib}\le C \|D\alpha(u^z)\| .$$
Since $r\in B_{\R^{m-1}}(e^t)$ we  have $z\in B_{\R^{m-1}} (M_1 e ^t)$ whence $d(u^z,\id)\le C_2 t + C_3$ for some constants $C_2$ and $C_3$.
  Proposition \ref{unipotentisgood} implies for any $\e'$ that   $$\|D\alpha(u^z)\|  \leq e^{\e'  (C_2 t + C_3)}$$
and taking $\e'>0$ sufficiently small, the claim follows.
%
% If $\td x$ is an element in   fundamental domain of $\Sl(m,\R)$ containing the identity, write $$\td x = \left(\begin{array}{cc}K& 0 \\0 & 1\end{array}\right);$$
%
%
%\end{proof}
%\begin{proof}
%We will fix a fundamental domain $D$ for our lattice $\Sl(m,\Z)$. We can assume that $x_n \in D$. As in Section \ref{section:goodcusps} We let  $K_n \in \Sl(m-1, \R) \cap D$ be such that:
%
%\[ x_n =
%\begin{pmatrix}
%  K_n & 0_{m-1 \times 1} \\ 0_{1\times m-1} & 1
%\end{pmatrix}
%\]
%
%We can assume $\|K_n\| \leq C$ for some constant $C>0$. We have:
%\[ u^rx_n =
%\begin{pmatrix}
%  K_n & r \\ 0_{1\times m-1} & 1
%\end{pmatrix}
%\]
%An easy computation shows that:
%\[
%\begin{pmatrix}
%  K_n & r \\ 0_{1\times m-1} & 1
%\end{pmatrix}
%= \begin{pmatrix}
%  K_n & r'_n \\ 0_{1\times m-1} & 1
%\end{pmatrix}
%\begin{pmatrix}
%  \Id_{m-1} & r''_n \\ 0_{1\times m-1} & 1
%\end{pmatrix}
%\]
%
%Where $r''_n \in B_{\R_{m-1}}(Ce^{t}) \cap \Sl(m, \Z)$ and $r'_n \in B_{\R^{m-1}}(C)$. Let us define
%\[
%y_n := \begin{pmatrix}
%  K_n & r'_n \\ 0_{1\times m-1} & 1
%\end{pmatrix}
%\text{ and } \gamma'_n :=
%\begin{pmatrix}
%  \Id_{m-1} & r''_n \\ 0_{1\times m-1} & 1
%\end{pmatrix}
%\]
%
%Observe that the distance $d(y_n, x_n)$ is bounded independent of $n$ and so we can write $y_n = x_n'\gamma''_n$ where $\gamma''_n \in SL_n(\Z)$ and $x'_n$ lies in the fundamental domain $D$. Also, the distances $d(x_n, x'_n)$ and $d(\gamma''_n, \Id)$ are bounded independent of $n$. Therefore $u_rx_n = y_n\gamma'_n =  x'_n \gamma_n$, where $\gamma_n := \gamma''_n\gamma'_n $.
%
%It follows easily from propostion \ref{unipotentisgood} that $\|D(\gamma'_n)\| \leq e^{\e/2t}$ if $t$ is large enough. As the distances $d(x_n, x'_n)$ and $d(\gamma''_n, \Id)$ are bounded, there exists a constant $C$ such that for every $n$ and $t$ large enough: $\|D_{x_n}(u^r)\| \leq Ce^{{\e/2}t} \leq e^{\e t}$  and we are done.
\end{proof}

%We  proceed with the proof of Proposition \ref{positiveexponent}.
%{\blue
%\begin{proof}[Proof of Proposition \ref{positiveexponent}]
%\note{rewritten to better match section 3} Take the function $\psi\colon \P F \to \R$ defined by $$\Phi(x,[v]) := \log \frac{\| D_xa(v)\|}{\|v\|}.$$
%By  of the measures $\td\mu_n$ we have that:
%$$\int_{\P F} \Phi  \ d  \mu_n =   \frac{1}{t_n}\frac{2}{\delta t_n}\frac{1}{|B(e^{200t_n})|}\int_{0}^{t_n} \int_{\delta t_n /2}^{\delta t_n} \int_{B(e^{200t_n})} \Phi(a^tb^su^rv_n) \  drdsdt$$
%
%This is a telescoping integral and can be rewritten as:
%
%$$ \frac{2}{\delta t_n}\frac{1}{|B(e^{200t_n})|} \int_{0}^1 \int_{\delta t_n/2}^{\delta t_n} \int_{B(e^{200t_n})}  \frac{1}{t_n} \log \frac{||D_{x_n}a^{t_n}a^tb^su^r(v_n)||}{||D_{x_n}a^tb^su^r(v_n)||} \  drdsdt$$
%
%We will show that if $n$ is large, for each fixed $r,s,t$ in the integral of the domain above, the integrand is greater than $\lambda/4$. First, we will show that the denominator ${||D_{x_n}a^tb^su^r(v_n)||}$ is at most $e^{\lambda t_n/10}$ if $n$ is large enough. We have: $$D_{x_n}a^tb^su^r(v_n) = D_{u^{r}x_n}a^tb^s D_{x_n}u^r (v_n)$$
%
%Observe that as $u^r(x_n)$ lies in a fixed compact set of $(\Sl({m-1},\R)\ltimes \R^{m-1}) / (\Sl({m-1},\Z) \ltimes \Z^{m-1})$ Lubotzky-Mozes-Raghunathan \cite{MR1244421} implies that the norm of $D_{u^{r}x_n}a^tb^s$ should be less than $e^{\lambda t_n/100}$ if the constant $\delta$ in the definition of the \Folner set is chosen sufficiently small.
%
%Therefore, It suffices to show that the norm of $D_{x_n}u^r (v_n)$ is less than $e^{\lambda t_n/100}$ for $n$ large enough but this follows immediately by Proposition \ref{siegelunipotent}.
%
%We will now analyze the numerator. We define $r' \in \R^{n-1}$ such that $a^{t_n}u^{r} = u^{r'}a^{t_n}$, observe that $r' \in B_{\R^{m-1}}(e^{201t_n})$ and that we can rewrite the numerator as:
%$$D_{x_n}a^{t_n}a^tb^su^r(v_n) = D_{x_n} a^tb^s u^{r'}a^{t_n}(v_n) = D_{u^{r'}a^n(x_n)}a^tb^s D_{a^n(x_n)}u^{r'} D_{x_n}a_n (v_n)$$
%
%By our choice of $x_n, v_n$ we have $\|D_{x_n}a^{t_n} (v_n)\| \geq e^{\lambda t_n/2}$. Observe also that as $a^n(x_n)$ and $u^{r'}a^n(x_n)$ lie in a fixed compact set, the norm of $D_{u^{r'}a^n(x_n)}a^tb^s$   is less than $e^{\lambda t_n /100}$ if $n$ is chosen sufficiently large. Proposition \ref{siegelunipotent} implies that  $\|D_{a^n(x_n)}u^{r'}\|$ is less than $e^{\lambda t_n /100}$ if $n$ is chosen sufficiently large. Therefore the norm of $D_{x_n}a^{t_n}a^tb^su^r(v_n)$ is greater than $e^{ \lambda t_n/2 - \lambda t_n/100 - \lambda t_n/100}$ and we are done.
%
%\end{proof}
%}

By Lemma \ref{lemma:fromlmr}, the fact that $\Sl(m,\Z)$ is finitely generated, and the uniform comparability of the fibers of $M^\alpha$,  we also have the following.%\note{added $\pm 1$ to avoid conorm notation}
\begin{claim}\label{claim90} There are uniform constants $C_5$ and $C_6$ with the following property:
Let $x\in G/\Gamma$.  Then for any $X\in \lieg$ with $\|X\|\le 1$ we have $$\left\|\big(D_{x} \exp (tX) \big)^{\blue \pm 1} \right\|_{\Fib}\le e^{C_5 t + C_5 d(x, \id) + C_6}.$$
\end{claim}



We now prove Proposition \ref{positiveexponent}.
\begin{proof}[Proof of Proposition \ref{positiveexponent}]
Recall we take $x_n\in \Xt$, $t_n\to \infty $, and $v_n \in F(x_n)$ with $\|v_n\| =1$ such that
$\|D_{x_n}a^{t_n} (v_n)\| \geq e^{\lambda t_n}$
for some fixed $\lambda>0$  in \eqref{eq:cleansingDeluge} in Section \ref{section:folner}.    We  also write $\calA\colon G\times F\to F$ for the fiberwise derivative cocycle.


The measures $\mu_n$ constructed in Section \ref{section:folner} are   defined by averaging last along the orbit $a^t, 0\le t\le t_n$.
Let $\xi_n$ be the measure on $M^\alpha$ given by
\begin{equation*}
\begin{aligned}
\int_{M^\alpha } &f \ d \xi_n ={  \frac{2}{\delta t_n}\bigg(\frac{1}{\sqrt{t_n}} }\bigg)^{m-3}\frac{ 1}{|B_{\R^{m-1}}(e^{200t_n})|}
\\&\quad \quad     \int_{\delta t_n /2}^{\delta t_n} \int_{[0, \sqrt{t_n}]^{m-3}}
%   \\&\quad \quad
\int_{B_{\R^{m-1}}(e^{200t_n})} f\left (a^tb^s\prod_{c=1}^{m-3} c_i^{s_i}u^r \cdot (x_n, p(v_n) )\right ) \  dr   \ d{s_i}\ ds.
 %
 \end{aligned}
 \end{equation*}
%{\blue where $x_n\in \Xt\subset G/\Gamma$, $v_n\in F(x_n)$,  and $t_n$ are chosen as \eqref{eq:cleansingDeluge} in Section \ref{section:folner}.}

In the context of  Lemma \ref{lemma:firstexponents}, the measures $\mu_n= \int_0^{t_n} (a^t_*\xi_n ) \ d t$ constructed in Section \ref{section:folner} correspond to the empirical measures $\eta_n= \eta(\log a, t_n, \xi_n)$ appearing in the proof of Lemma \ref{lemma:firstexponents}.  From Lemma \ref{lemma:firstexponents}, to establish
 Proposition \ref{positiveexponent} it is sufficient to show that
$$\int \log  \|\calA(a ^{t_n},\cdot) \|  \ d \xi_n %= \int \log  \|D_{x}a ^{t_n} \|_\Fib   \ d \xi_n(x)
\geq \frac{\lambda}{2} t_n.$$

%Below, given $x\in G/\Gamma$, $v\in F(x)$, %$y$ in the fiber of $M^\alpha$ over $x$,
%and $g\in G$ we write $D_{x} g (v) \in F(g\cdot x)$ %\colon T \pi\inv(x) M^\alpha\to T_{g\cdot y}M^\alpha$
% for the derivative of translation by $g$ applied to $v$.  % in the fiberwise tangent bundle over $x$.   %Given $y$ in the fiber of $M^\alpha$ over $x$, let  $p\colon  F(y)\to y$ denote the projection to the base point $y$.
We have
\begin{equation*}
\begin{aligned}
%\int_{M^\alpha } & \log  \|D_{x_n}a ^{t_n} \|_\Fib  \ d \xi_n \\
 \int_{M^\alpha }  & {\blue  \log  \|\calA(a ^{t_n},\cdot) \|  \ d \xi_n} \\
&\quad  ={  \frac{2}{\delta t_n}\bigg(\frac{1}{\sqrt{t_n}} }\bigg)^{m-3}\frac{ 1}{|B_{\R^{m-1}}(e^{200t_n})|}
\\ &\quad   \quad
%\\
%&
    \int_{\delta t_n /2}^{\delta t_n} \int_{[0, \sqrt{t_n}]^{m-3}}
%   \\&\quad \quad
\int_{B_{\R^{m-1}}(e^{200t_n})}
{\blue \log \left\| \calA\big(  a^{t_n}, b^s\Pi  c_i^{s_i}u^r \cdot (x_n, p(v_n) )\big) \right\|\  dr \ ds_i \ d s}\\
%&\quad   \quad
%{\red
%\phantom{    \int_{\delta t_n /2}^{\delta t_n} \int_{[0, \sqrt{t_n}]^{m-3}}
%\int_{B_{\R^{m-1}}(e^{200t_n})}}
%\log \cancel  {\| D_{\big(b^s\Pi  c_i^{s_i}u^r \cdot (x_n, p(v_n) )\big)} a^{t_n} \| }\  dr \ ds_i \ d s}\\
&\quad  \ge
{  \frac{2}{\delta t_n}\bigg(\frac{1}{\sqrt{t_n}} }\bigg)^{m-3}\frac{ 1}{|B_{\R^{m-1}}(e^{200t_n})|}
\\ &\quad \quad
    \int_{\delta t_n /2}^{\delta t_n} \int_{[0, \sqrt{t_n}]^{m-3}}
\int_{B_{\R^{m-1}}(e^{200t_n})}
%\log \frac{\left \| D_{\big(b^s\Pi  c_i^{s_i}u^r \cdot x_n \big)} a^{t_n} \big( D_{x_n } (b^s\Pi  c_i^{s_i}u^r \cdot  v_n)\big) \right  \|    }{}
\log \frac{\left \| D_{ x_n  } \big( a^{t_n} b^s\Pi  c_i^{s_i}u^r\big) ( v_n)\right  \|    }{\left \| D_{ x_n  } \big(   b^s\Pi  c_i^{s_i}u^r\big) ( v_n) \right  \|  }
\  dr \ ds_i \ d s.\\
 \end{aligned}
 \end{equation*}
Consider fixed $r,$ $s, $ and $ s_i$.  Take $r'\in \R^{m-1}$ such that $a^{t_n} u^r = u^{r'}a^{t_n} $.  Then
 \begin{equation*}  \begin{aligned}
 \log &\frac{\left \| D_{ x_n  } \big( a^{t_n} b^s\Pi  c_i^{s_i}u^r\big) ( v_n) \right  \|    }{\left \| D_{ x_n  } \big(   b^s\Pi  c_i^{s_i}u^r\big) ( v_n)\big) \right  \|  }
  =  \log \frac{\left \| D_{ a^{t_n}\cdot x_n  } \big( b^s\Pi  c_i^{s_i} u^{r'}   \big)       \circ  D_{ x_n  } a^{t_n}   \big( v_n\big) \right  \|    }{\left \| D_{ x_n  } \big(   b^s\Pi  c_i^{s_i}u^r\big) ( v_n) \right  \|  }\\
 %
 %
 &\quad \ge \log \|D_{ x_n  } a^{t_n}   \big( v_n\big)\| -   \log \| \big (D_{ x_n  }   b^s\Pi  c_i^{s_i}u^r\big) \| _\fib- \log \| \big( D_{ a^{t_n}\cdot x_n  } \big( b^s\Pi  c_i^{s_i} u^{r'} \big)\big)^{\blue -1} \|_\fib \phantom{\Big\|}\\
%
 &\quad \ge \log \|D_{ x_n  } a^{t_n}   \big( v_n\big)\| -   \log \| D_{ x_n  } \big(    u^r\big)\| _\fib- \log \| D_{ u^r x_n  } \big(   b^s\Pi  c_i^{s_i} \big)\|_\fib
 \\&\quad \quad \quad  - \log \| \big (D_{ u^{r'}a^{t_n}\cdot x_n  } \big( b^s\Pi  c_i^{s_i}   \big) \big)^{\blue -1}\|_\fib -
 \log \| D_{ a^{t_n}\cdot x_n  } \big(  u^{{\blue -r'}}  \big)\|_\fib
 . \phantom{\Big\|}\\
%
 \end{aligned}
 \end{equation*}%\note {fixed notation in above estimates}




% \begin{equation*}
%{\red \cancel{ \begin{aligned}
% \log &\frac{\left \| D_{ x_n  } \big( a^{t_n} b^s\Pi  c_i^{s_i}u^r\big) ( v_n)\big) \right  \|    }{\left \| D_{ x_n  } \big(   b^s\Pi  c_i^{s_i}u^r\big) ( v_n)\big) \right  \|  }
%  =  \log \frac{\left \| D_{ a^{t_n}\cdot x_n  } \big( b^s\Pi  c_i^{s_i} u^{r'}   \big) ( v_n)\big)      \circ  D_{ x_n  } a^{t_n}   \big( v_n\big) \right  \|    }{\left \| D_{ x_n  } \big(   b^s\Pi  c_i^{s_i}u^r\big) ( v_n)\big) \right  \|  }\\
% %
% %
% &\quad \ge \log \|D_{ x_n  } a^{t_n}   \big( v_n\big)\| -   \log \| D_{ x_n  } \big(   b^s\Pi  c_i^{s_i}u^r\big)\| - \log \| D_{ a^{t_n}\cdot x_n  } \big( b^s\Pi  c_i^{s_i} u^{r'}  \big)\| \phantom{\Big\|}\\
%%
% &\quad \ge \log \|D_{ x_n  } a^{t_n}   \big( v_n\big)\| -   \log \| D_{ x_n  } \big(    u^r\big)\| - \log \| D_{ u^r x_n  } \big(   b^s\Pi  c_i^{s_i} \big)\|
% \\&\quad \quad \quad  - \log \| D_{ u^{r'}a^{t_n}\cdot x_n  } \big( b^s\Pi  c_i^{s_i}   \big)\| -
% \log \| D_{ a^{t_n}\cdot x_n  } \big(  u^{r'}  \big)\|
% . \phantom{\Big\|}\\
%%
% \end{aligned}}}
% \end{equation*}


%$$ \frac{2}{\delta t_n}\frac{1}{|B(e^{200t_n})|}  \int_{\delta t_n/2}^{\delta t_n} \int_{B(e^{200t_n})}  \frac{1}{t_n} \log \frac{||D_{x_n}a^{t_n}a^tb^su^r(v_n)||}{||D_{x_n}a^tb^su^r(v_n)||} \  dr \ ds_i \ d s $$




%From Lemma \ref{lemma:firstexponents} to establish that a weak-$*$ limit $\mu_\infty$ of $\{\mu_n\}$ has a nonzero exponent


%Take the function $\psi\colon \P F \to \R$ defined by $$\Phi(x,[v]) := \log \frac{\| D_xa(v)\|}{\|v\|}.$$



%$$\int_{\P F} \Phi  \ d\td \mu_n =   \frac{1}{t_n}\frac{2}{\delta t_n}\frac{1}{|B(e^{200t_n})|}\int_{0}^{t_n} \int_{\delta t_n /2}^{\delta t_n} \int_{B(e^{200t_n})} \Phi(a^tb^su^rv_n) \  drdsdt$$

%This is a telescoping integral and can be rewritten as:

%$$ \frac{2}{\delta t_n}\frac{1}{|B(e^{200t_n})|} \int_{0}^1 \int_{\delta t_n/2}^{\delta t_n} \int_{B(e^{200t_n})}  \frac{1}{t_n} \log \frac{||D_{x_n}a^{t_n}a^tb^su^r(v_n)||}{||D_{x_n}a^tb^su^r(v_n)||} \  drdsdt$$

%We will show that if $n$ is large, for each fixed $r,s,t$ in the integral of the domain above, the integrand is greater than $\lambda/4$. First, we will show that the denominator ${||D_{x_n}a^tb^su^r(v_n)||}$ is at most $e^{\lambda t_n/10}$ if $n$ is large enough. We have: $$D_{x_n}a^tb^su^r(v_n) = D_{u^{r}x_n}a^tb^s D_{x_n}u^r (v_n)$$

Observe that both  $u^r\cdot  x_n$ and $ u^{r'}a^{t_n}\cdot x_n$ are contained in a fixed compact subset of  $G/\Gamma$ and hence, by Claim \ref{claim90}, having taken $\delta>0$ sufficiently small in the construction of the \Folner sequence, from the constraints on $s_i$ and $s$ we have  $\|D_{u^{r}x_n}\Pi  c_i^{s_i}b^s\|_\fib\le  e^{\lambda t_n/100}$  and $\|\big(D_{u^{r'}a^{t_n}\cdot x_n}\Pi  c_i^{s_i}b^s\big)^{\blue -1}\|_\fib\le  e^{\lambda t_n/100}$ for all $n$ sufficiently large.

Moreover, from Claim  \ref{siegelunipotent},  we have $\|D_{x_n}u^r \|_\fib\le e^{\lambda t_n/100}$  for all $n$ sufficiently large.

Finally, there exists $\kappa>0$ such that $\|r'\| \le e^{\kappa t_n} \|r\|$ whence $r'\in   B_{\R^{m-1}}(e^{(200+\kappa)t_n})$.  Again   from Claim  \ref{siegelunipotent},
 we have $\|D_{a^{t_n}\cdot x_n}u^{\blue -r'} \|_\fib\le e^{\lambda t_n/100}$   for $n$ sufficiently large.
 Combined with  \eqref{eq:cleansingDeluge} we then have
\[\frac{1}{t_n} \int_{M^\alpha }   {\log  \|\calA(a ^{t_n},\cdot) \|  \ d \xi_n}   \ge \lambda- \frac 4{100} \lambda. \]
 Proposition \ref{positiveexponent} then follows from  Lemma \ref{lemma:firstexponents}.
%\[\frac{1}{t_n} \int_{M^\alpha }  \log  \|D_{x_n}a ^{t_n} \|_\Fib  \ d \xi_n \ge \lambda- \frac 4{100} \lambda.\qedhere \]
\end{proof}







%%
%%We now prove Proposition \ref{positiveexponent}.
%%\begin{proof}[Proof of Proposition \ref{positiveexponent}]
%%Recall we take $x_n\in \Xt$, $t_n\to \infty $, and $v_n \in F(x_n)$ with $\|v_n\| =1$ such that
%%$\|D_{x_n}a^{t_n} (v_n)\| \geq e^{\lambda t_n}$
%%for some fixed $\lambda>0$.  We  also write $\calA\colon G\times F\to F$ for the fiberwise derivative cocycle.
%%
%%
%%The measures $\mu_n$ constructed in Section \ref{section:folner} are   defined by averaging last along the orbit $a^t, 0\le t\le t_n$.
%%Let $\xi_n$ be the measure on $M^\alpha$ given by
%%\begin{equation*}
%%\begin{aligned}
%%\int_{M^\alpha } &f \ d \xi_n ={  \frac{2}{\delta t_n}\bigg(\frac{1}{\sqrt{t_n}} }\bigg)^{m-3}\frac{ 1}{|B_{\R^{m-1}}(e^{200t_n})|}
%%\\&\quad \quad     \int_{\delta t_n /2}^{\delta t_n} \int_{[0, \sqrt{t_n}]^{m-3}}
%%%   \\&\quad \quad
%%\int_{B_{\R^{m-1}}(e^{200t_n})} f\left (a^tb^s\prod_{c=1}^{m-3} c_i^{s_i}u^r \cdot (x_n, p(v_n) )\right ) \  dr   \ d{s_i}\ ds.
%% %
%% \end{aligned}
%% \end{equation*}
%%In the context of  Lemma \ref{lemma:firstexponents}, the measures $\mu_n= \int_0^{t_n} (a^t_*\xi_n ) \ d t$ constructed in Section \ref{section:folner} correspond to the empirical measures $\eta_n= \eta(\log a, t_n, \xi_n)$ appearing in the proof of Lemma \ref{lemma:firstexponents}.  From Lemma \ref{lemma:firstexponents}, to establish
%% Proposition \ref{positiveexponent} it is sufficient to show that
%%$$\int \log  \|\calA(a ^{t_n},x) \|  \ d \xi_n(x)= \int \log  \|D_{x}a ^{t_n} \|_\Fib   \ d \xi_n(x) \geq \frac{\lambda}{2} t_n.$$
%%
%%Given $x\in G/\Gamma$, $y$ in the fiber of $M^\alpha$ over $x$, and $g\in G$ we write $D_{(x,y)} g\colon T_{(x,y)}M^\alpha\to T_{g\cdot (x,y)}M^\alpha$ for the derivative of translation by $g$.   Given $y$ in the fiber of $M^\alpha$ over $x$, let  $p\colon  F(y)\to y$ denote the projection to the base point $y$.
%%We have
%%\begin{equation*}
%%\begin{aligned}
%%\int_{M^\alpha } & \log  \|D_{x_n}a ^{t_n} \|_\Fib  \ d \xi_n \\
%%&\quad  ={  \frac{2}{\delta t_n}\bigg(\frac{1}{\sqrt{t_n}} }\bigg)^{m-3}\frac{ 1}{|B_{\R^{m-1}}(e^{200t_n})|}
%%\\ &\quad   \quad
%%%\\
%%%&
%%    \int_{\delta t_n /2}^{\delta t_n} \int_{[0, \sqrt{t_n}]^{m-3}}
%%%   \\&\quad \quad
%%\int_{B_{\R^{m-1}}(e^{200t_n})}
%%\log \| D_{\big(b^s\Pi  c_i^{s_i}u^r \cdot (x_n, p(v_n) )\big)} a^{t_n} \|\  dr \ ds_i \ d s\\
%%&\quad  \ge
%%{  \frac{2}{\delta t_n}\bigg(\frac{1}{\sqrt{t_n}} }\bigg)^{m-3}\frac{ 1}{|B_{\R^{m-1}}(e^{200t_n})|}
%%\\ &\quad \quad
%%    \int_{\delta t_n /2}^{\delta t_n} \int_{[0, \sqrt{t_n}]^{m-3}}
%%\int_{B_{\R^{m-1}}(e^{200t_n})}
%%%\log \frac{\left \| D_{\big(b^s\Pi  c_i^{s_i}u^r \cdot x_n \big)} a^{t_n} \big( D_{x_n } (b^s\Pi  c_i^{s_i}u^r \cdot  v_n)\big) \right  \|    }{}
%%\log \frac{\left \| D_{ x_n  } \big( a^{t_n} b^s\Pi  c_i^{s_i}u^r\big) ( v_n)\big) \right  \|    }{\left \| D_{ x_n  } \big(   b^s\Pi  c_i^{s_i}u^r\big) ( v_n)\big) \right  \|  }
%%\  dr \ ds_i \ d s.\\
%% \end{aligned}
%% \end{equation*}
%%For each $dr,$ $ds, $ and $ s_i$,  take $r'\in \R^{m-1}$ such that $a^{t_n} u^r = u^{r'}a^{t_n} $.  Then
%%  \begin{equation*}
%%\begin{aligned}
%% \log &\frac{\left \| D_{ x_n  } \big( a^{t_n} b^s\Pi  c_i^{s_i}u^r\big) ( v_n)\big) \right  \|    }{\left \| D_{ x_n  } \big(   b^s\Pi  c_i^{s_i}u^r\big) ( v_n)\big) \right  \|  }
%%  =  \log \frac{\left \| D_{ a^{t_n}\cdot x_n  } \big( b^s\Pi  c_i^{s_i} u^{r'}   \big) ( v_n)\big)      \circ  D_{ x_n  } a^{t_n}   \big( v_n\big) \right  \|    }{\left \| D_{ x_n  } \big(   b^s\Pi  c_i^{s_i}u^r\big) ( v_n)\big) \right  \|  }\\
%% %
%% %
%% &\quad \ge \log \|D_{ x_n  } a^{t_n}   \big( v_n\big)\| -   \log \| D_{ x_n  } \big(   b^s\Pi  c_i^{s_i}u^r\big)\| - \log \| D_{ a^{t_n}\cdot x_n  } \big( b^s\Pi  c_i^{s_i} u^{r'}  \big)\| \phantom{\Big\|}\\
%%%
%% &\quad \ge \log \|D_{ x_n  } a^{t_n}   \big( v_n\big)\| -   \log \| D_{ x_n  } \big(    u^r\big)\| - \log \| D_{ u^r x_n  } \big(   b^s\Pi  c_i^{s_i} \big)\|
%% \\&\quad \quad \quad  - \log \| D_{ u^{r'}a^{t_n}\cdot x_n  } \big( b^s\Pi  c_i^{s_i}   \big)\| -
%% \log \| D_{ a^{t_n}\cdot x_n  } \big(  u^{r'}  \big)\|
%% . \phantom{\Big\|}\\
%%%
%% \end{aligned}
%% \end{equation*}
%%
%%
%%%$$ \frac{2}{\delta t_n}\frac{1}{|B(e^{200t_n})|}  \int_{\delta t_n/2}^{\delta t_n} \int_{B(e^{200t_n})}  \frac{1}{t_n} \log \frac{||D_{x_n}a^{t_n}a^tb^su^r(v_n)||}{||D_{x_n}a^tb^su^r(v_n)||} \  dr \ ds_i \ d s $$
%%
%%
%%
%%
%%%From Lemma \ref{lemma:firstexponents} to establish that a weak-$*$ limit $\mu_\infty$ of $\{\mu_n\}$ has a nonzero exponent
%%
%%
%%%Take the function $\psi\colon \P F \to \R$ defined by $$\Phi(x,[v]) := \log \frac{\| D_xa(v)\|}{\|v\|}.$$
%%
%%
%%
%%%$$\int_{\P F} \Phi  \ d\td \mu_n =   \frac{1}{t_n}\frac{2}{\delta t_n}\frac{1}{|B(e^{200t_n})|}\int_{0}^{t_n} \int_{\delta t_n /2}^{\delta t_n} \int_{B(e^{200t_n})} \Phi(a^tb^su^rv_n) \  drdsdt$$
%%
%%%This is a telescoping integral and can be rewritten as:
%%
%%%$$ \frac{2}{\delta t_n}\frac{1}{|B(e^{200t_n})|} \int_{0}^1 \int_{\delta t_n/2}^{\delta t_n} \int_{B(e^{200t_n})}  \frac{1}{t_n} \log \frac{||D_{x_n}a^{t_n}a^tb^su^r(v_n)||}{||D_{x_n}a^tb^su^r(v_n)||} \  drdsdt$$
%%
%%%We will show that if $n$ is large, for each fixed $r,s,t$ in the integral of the domain above, the integrand is greater than $\lambda/4$. First, we will show that the denominator ${||D_{x_n}a^tb^su^r(v_n)||}$ is at most $e^{\lambda t_n/10}$ if $n$ is large enough. We have: $$D_{x_n}a^tb^su^r(v_n) = D_{u^{r}x_n}a^tb^s D_{x_n}u^r (v_n)$$
%%
%%Observe that both  $u^r\cdot  x_n$ and $ u^{r'}a^{t_n}\cdot x_n$ are contained in a fixed compact subset of  $G/\Gamma$ and hence, by Claim \ref{claim90}, having taken $\delta>0$ sufficiently small in the construction of the \Folner sequence, from the constraints on $s_i$ and $s$ we have  $\|D_{u^{r}x_n}\Pi  c_i^{s_i}b^s\|\le  e^{\lambda t_n/100}$  and $\|D_{u^{r'}a^{t_n}\cdot x_n}\Pi  c_i^{s_i}b^s\|\le  e^{\lambda t_n/100}$.
%%
%%Moreover, from Claim  \ref{siegelunipotent},  we have $\|D_{x_n}u^r \|\le e^{\lambda t_n/100}$  for all $n$ sufficiently large.
%%
%%Finally, there exists $\kappa>0$ such that $\|r'\| \le e^{\kappa t_n} \|r\|$ whence $r'\in   B_{\R^{m-1}}(e^{(200+\kappa)t_n})$.  Again   from Claim  \ref{siegelunipotent},
%% we have $\|D_{a^{t_n}\cdot x_n}u^{r'} \|\le e^{\lambda t_n/100}$   for $n$ sufficiently large.
%% Then
%%\[\frac{1}{t_n} \int_{M^\alpha }  \log  \|D_{x_n}a ^{t_n} \|_\Fib  \ d \xi_n \ge \lambda- \frac 4{100} \lambda.\qedhere \]
%%\end{proof}
%We will now analyze the numerator. We define $r' \in \R^{n-1}$ such that $a^{t_n}u^{r} = u^{r'}a^{t_n}$, observe that $r' \in B_{\R^{m-1}}(e^{201t_n})$ and that we can rewrite the numerator as:
%$$D_{x_n}a^{t_n}a^tb^su^r(v_n) = D_{x_n} a^tb^s u^{r'}a^{t_n}(v_n) = D_{u^{r'}a^n(x_n)}a^tb^s D_{a^n(x_n)}u^{r'} D_{x_n}a_n (v_n)$$
%
%By our choice of $x_n, v_n$ we have $\|D_{x_n}a^{t_n} (v_n)\| \geq e^{\lambda t_n/2}$. Observe also that as $a^n(x_n)$ and $u^{r'}a^n(x_n)$ lie in a fixed compact set, the norm of $D_{u^{r'}a^n(x_n)}a^tb^s$ is less than $e^{\lambda t_n /100}$ if $n$ is chosen sufficiently large. Proposition \ref{siegelunipotent} implies that  $\|D_{a^n(x_n)}u^{r'}\|$ is less than $e^{\lambda t_n /100}$ if $n$ is chosen sufficiently large. Therefore the norm of $D_{x_n}a^{t_n}a^tb^su^r(v_n)$ is greater than $e^{ \lambda t_n/2 - \lambda t_n/100 - \lambda t_n/100}$ and we are done.



\subsection{Proof of Proposition \ref{prop:maybeweshouldstatethemainresultatsomepoint}}
Having   assumed that $\chi_{\mathrm{max}}$ in \eqref{eq:fanorlamps} is non-zero, we arrive at a contradiction.
Take any weak-$*$ subsequential limit $\mu_\infty$ of the sequence of measure  $\{\mu_n\}$ on $M^\alpha$.  We have that $\mu_\infty$ is $A$-invariant and has a non-zero fiberwise Lyapunov exponent for the fiberwise derivative  over the action of $a^t$.
Moreover, we have that $\mu_\infty$ projects to $\nu_\infty$ on $G/\Gamma$ which, as discussed above, is the Haar measure on $G/\Gamma$.
We may replace $\mu_\infty$ with an $A$-ergodic component $\mu'$ with the same properties as above.  Then $\mu$ is $A$-ergodic, projects to Haar, and the fiberwise derivative cocycle over the $A$-action on $(M^\alpha, \mu)$ has a non-zero Lyapunov exponent functional $\lambda_i\colon A\to \R.  $


%and consider its projection $\mu= \pi_0(\tilde \mu)$ to $M^{\alpha}$. By construction $\mu$ is $AN'$ invariant, has subexponential cusps and has positive fiberwise Lyaponuv exponent for $a^t$. Given a subgroup $F<\Sl(m,\R)$, we let $F^t$ be the subgroup of transposes of elements of $F$.
%Using consequences of Ratner's work for $\Sl(2,\R)$ triples as described in \cite[Theorem 5.1(d)]{BFH}, we see that $\pi(\mu)$ is also invariant under the group $N'' = (N')^t$.  It is easy to see that $N'$ and $N''$ generate all of $\Sl(m, \R)$ as $N'$ contains all elementary matrices with non-zero entries in the $\{i,n\}$ place and $N''$ contains all elementary matrices with non-zero entry in the $\{n,j\}$ place, so the group generated by them contains all elementary matrices.
 As in the conclusion of Lemma \ref{uni1},   the arguments of \cite[Section 5.5]{BFH} using
\cite[Proposition 5.1]{AWBFRHZW-latticemeasure}  imply that the measure  $\mu$ is, in fact, $\Sl(m,\R)$-invariant. As before, we note that \cite[Proposition 5.1]{AWBFRHZW-latticemeasure} does not assume $\Gamma$ is cocompact, so the algebraic argument applying that proposition in \cite[Section 5.5]{BFH} goes through verbatim.   For a more self-contained proof that applies since we only consider the case of $\SL(m,\R)$ see \cite[Proposition 4]{BDZ}.
We then obtain a contradiction with Zimmer's cocycle superrigidity by constraints on the dimension of the fibers of $M^\alpha$.
Thus we must have $\chi_{\mathrm{max}}=0$ and Proposition \ref{prop:maybeweshouldstatethemainresultatsomepoint} follows.
%
% and shows that we have subexponential growth of derivatives for $\Sl(2,\Z)$.
% As the argument does not depend anywhere on our choice of indices, we have this in fact for $\Lambda_{i,j}$ for every choice of $i$ and $j$.
%
%We now use the work of Lubotzky, Mozes and Raghunathan, particularly \cite[Corollary 3]{MR1244421} which shows that any element $\gamma$ of $\Sl(m,\Z)$ can be written as a product of at most $d^2$ elements $\gamma_i$, each one contained in a $\Lambda_{i,j}$ and with each $\gamma_i$ of size proportional to the size of $\gamma$.  Combined
%with the subexponential growth of derivatives for elements in each $\Lambda_{i,j}$ this immediately yields subexponential growth for elements of $\Gamma$.





%\subsection{Final arguments}
%\label{section:final}
%
%We begin by taking the sequence $\mu_n$ of measures constructed  above taking a subsequential   $\mu$ on $M^{\alpha}$. By construction $\mu$ is $AN'$ invariant, has subexponential cusps and has positive fiberwise Lyaponuv exponent for $a^t$. Given a subgroup $F<\Sl(m,\R)$, we let $F^t$ be the subgroup of transposes of elements of $F$.
%Using consequences of Ratner's work for $\Sl(2,\R)$ triples as described in \cite[Theorem 5.1(d)]{BFH}, we see that $\pi(\mu)$ is also invariant under the group $N'' = (N')^t$.  It is easy to see that $N'$ and $N''$ generate all of $\Sl(m, \R)$ as $N'$ contains all elementary matrices with non-zero entries in the $\{i,n\}$ place and $N''$ contains all elementary matrices with non-zero entry in the $\{n,j\}$ place, so the group generated by them contains all elementary matrices.  As at the end of the proof of Lemma \ref{uni1}, we can now apply the arguments of \cite[Section 5.5]{BFH} using
%\cite[Proposition 5.1]{AWBFRHZW-latticemeasure} to show that $\mu$ is in fact $\Sl(m,\R)$ invariant. As before, we note that \cite[Proposition 5.1]{AWBFRHZW-latticemeasure} does not assume $\Gamma$ is cocompact, so the algebraic argument applying that proposition in \cite[Section 5.5]{BFH} goes through verbatim. This then contradicts Zimmer's cocycle superrigidity and shows that we have subexponential growth of derivatives for $\Sl(2,\Z)$.  As the argument does not depend anywhere on our choice of indices, we have this in fact for $\Lambda_{i,j}$ for every choice of $i$ and $j$.
%
%We now use the work of Lubotzky, Mozes and Raghunathan, particularly \cite[Corollary 3]{MR1244421} which shows that any element $\gamma$ of $\Sl(m,\Z)$ can be written as a product of at most $d^2$ elements $\gamma_i$, each one contained in a $\Lambda_{i,j}$ and with each $\gamma_i$ of size proportional to the size of $\gamma$.  Combined
%with the subexponential growth of derivatives for elements in each $\Lambda_{i,j}$ this immediately yields subexponential growth for elements of $\Gamma$.



  \bibliographystyle{AWBmath}

\bibliography{bibliography}

\end{document} 