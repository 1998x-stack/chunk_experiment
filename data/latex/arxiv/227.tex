%%%%%%%%%%%%%%%%%%%%%%%%%%%%%%%%%%%%%%%%%%%%%%%%%
% bounds 
%%%%%%%%%%%%%%%%%%%%%%%%%%%%%%%%%%%%%%%%%%%%%%%%%

\section{Proofs of Properties of Elementary Swaps}
\label{sec:bounds}

In this section we prove Lemmas~\ref{lemma:elem-swap} and \ref{lemma:elem-swap-seq}.

%\comment{
To prove 
%In 
Lemma~\ref{lemma:elem-swap}, the idea is to look at paths in the \textit{double quadrilateral graph} $G_D$ that we will define below. Informally speaking, 
$G_D$ captures where pairs of non-crossing edges can go via flips, similar to the way the quadrilateral graph captures where a single edge can go via flips.
%$G_D$ has $O(n^4)$ vertices that correspond to pairs of non-crossing edges on point set $P$, and edges that correspond to flip sequences of length $O(n^2)$. 
We will show that there is an elementary swap between two labels in a triangulation if and only if there exists a path of certain type in the double quadrilateral graph. 
%}

\begin{proof}[Proof of Lemma~\ref{lemma:elem-swap}]
Construct a graph $G_D$ called the \emph{double quadrilateral graph}. 
%\note{(is there a better name? ``2-edge quadrilateral graph''?)}.  
Vertices of the graph $G_D$ are
pairs of non-crossing edges on the point set $P$, and we define two vertices
$(e_{1},f_{1})$ and $(e_{2},f_{2})$ of $G_D$ to be adjacent
if either $e_{1}=e_{2}$ and $f_{1}$ and $f_{2}$
are adjacent in the quadrilateral graph, 
or if $f_{1}=f_{2}$ and $e_{1}$ and $e_{2}$
are adjacent in the quadrilateral graph.
(Recall that two edges $a$ and $b$ are adjacent in the quadrilateral graph if $a$ and $b$ cross and their four endpoints form an empty quadrilateral.)
%\note{Let's make a figure for $G_D$.}

In the graph $G_D$ we identify some vertices as ``swap vertices''.  These are the vertices $(g,h)$ such that $g$ and $h$ are diagonals of some empty convex pentagon in the point set. 
Note that the swap vertices can be identified in polynomial time.

We claim that  there is an elementary swap of $e$ and $f$ in labelled triangulation $\cal T = (T,\ell)$  if and only if there is a path in $G_D$ from vertex $(e,f)$ to a swap vertex.  
For the forward direction, suppose there is such an elementary swap.
It begins with a sequence $\sigma$ of flips from $\cal T$ to a labelled triangulation ${\cal T}'$ in which
labels $\ell(e)$ and $\ell(f)$ are attached to two diagonals $g$ and $h$ of some empty convex pentagon.
The subsequence of $\sigma$ consisting of those flips that apply to an edge whose current label is $\ell(e)$ or $\ell(f)$ corresponds to a path in $G_D$ from $(e,f)$ to the swap vertex $(g,h)$.

For the other direction, let $\pi$ be a path in $G_D$ from $(e,f)$ to a swap vertex.  
It suffices to show that the path $\pi$ provides a sequence of flips, $\sigma$, that takes $\cal T$ to some labelled triangulation ${\cal T}'$ in which
labels $\ell(e)$ and $\ell(f)$ are attached to two diagonals of an empty convex pentagon,
because the rest of the elementary swap is then determined.  
Consider the first edge of $\pi$ and suppose without loss of generality that it goes from $(e,f)$ to $(e,f')$ (the case when $e$ changes is similar).   Then $e$ and $f'$ are non-crossing.  Because $f$ and $f'$ are adjacent in the quadrilateral graph, they cross and form an empty convex quadrilateral $Q$.
Note that $e$ does not intersect the interior of $Q$, since $Q$ is empty and $e$ does not cross $f$ or $f'$.
We apply the result that any constrained triangulation can be  flipped to any other with $O(n^2)$ flips.  Fix edges $e$ and $f$ in $T$ and flip $\cal T$ to a labelled triangulation that contains the edges of $Q$.  In this triangulation, we can flip $f$ to $f'$, transferring $\ell(f)$ to $f'$.  We continue in this way to realize each edge of $\pi$ via $O(n^2)$ flips, arriving finally at a labelled triangulation in which labels $\ell(e)$ and $\ell(f)$ are attached to edges that are the diagonals of some empty convex pentagon in the point set.  Fixing the two diagonals, we can flip to a triangulation that contains the edges of the convex pentagon, and at this point we are done.      

Because the graph $G_D$ has $O(n^4)$ vertices, the diameter of any of its connected components is $O(n^4)$.
Thus, if there is an elementary swap that exchanges the labels of edges $e$ and $f$, then there is one corresponding to a path in $G_D$ of length $O(n^4)$.
We can explicitly construct $G_D$ and find such a path in polynomial time.
As argued above, every edge of $G_D$ can be realized by $O(n^2)$ flips.
This proves that, for any elementary swap, we can construct a sequence of $O(n^6)$ flips to realize it, and the construction takes polynomial time.  
%It is unlikely that this bound is tight.
\end{proof}  


%%%%%%%%%%%%%%%%%%%%%%%%%%%%

%\comment{
As  
mentioned in Section~\ref{sec:reductions}, there is a group-theoretic argument proving a weaker version of Lemma~\ref{lemma:elem-swap-seq}.  The argument depends on the following claim: If a permutation group is generated by transpositions and contains a permutation that maps element $e$ to $f$ then the group contains the transposition of $e$ and $f$. To prove this claim, notice that if the group contains transpositions $(ab)$ and $(bc)$, then it also contains transposition $(ac)=(ab)(bc)(ab)$; and apply induction.   


To apply this claim in our situation, observe that
by the Elementary Swap Theorem, all label permutations achievable by flips in a triangulation $\cal T$ are compositions of elementary swaps, hence, these label permutations indeed form a group $G$ generated by transpositions. Moreover, by the assumption of Lemma~\ref{lemma:elem-swap-seq}, $G$ contains a permutation taking the label of edge $e$ to edge $f$. Hence, by the above claim, the group $G$ also contains a 
permutation, which is a composition of elementary swaps, whose effect is to transpose labels of edges $e$ and $f$.

In order to prove the full result of Lemma~\ref{lemma:elem-swap-seq}, i.e., that the label transposition of $e$ and $f$ can be done with a single elementary swap, we combine the techniques used in the proof of the group theory claim above with the structure of elementary swaps. 
%}

\begin{proof}[Proof of Lemma~\ref{lemma:elem-swap-seq}]
%\note{I'm not sure if the following paragraph adds insight any more.}
%In essence, we use the fact that if a permutation group is generated by transpositions and contains a permutation that maps $a$ to $b$ then the group contains the transposition of $a$ and $b$.
%However, we cannot simply appeal to this result since we need the details of how the 
%transposition of $a$ and $b$ is generated.
An elementary swap in triangulation $\cal T$ acts on two edges of $\cal T$.  
We define a graph $G_S$ called the \emph{elementary swap graph} of $\cal T$. 
$G_S$ has a vertex for every edge of $\cal T$, and we define vertices $e$ and $f$ to be adjacent in $G_S$ if there is an elementary swap  of $e$ and $f$ in $\cal T$. 
%Note that we can explicitly construct $G_S$ using Lemma~\ref{lemma:elem-swap}. 

By hypothesis, there is a sequence of elementary swaps that takes the label of edge $e$ to edge $f$.
Observe that no sequence of elementary swaps will take the label of edge $e$ outside the connected component of $G_S$ that contains $e$.
Therefore $e$ and $f$ must lie in the same connected component of $G_S$.  
We will now show that 
each connected component of $G_S$ is a clique.  This implies that there is an 
elementary swap of $e$ and $f$, and completes our proof.

Consider a simple path $(e_0,e_1), (e_1, e_2), \ldots, (e_{k-1}, e_k)$ in $G_S$.  
Let $\sigma_i$, $i=1, \ldots, k$ be a flip sequence that realizes the elementary swap $(e_{i-1}, e_i)$, and let $\sigma = \sigma_1 \sigma_2 \ldots \sigma_{k-1}$.  Observe that $\sigma$ takes the label of $e_0$ to $e_{k-1}$, and does not change the label of $e_k$ (by the assumption that the path is simple).
By definition of an elementary swap, the flip sequence $\sigma_k$ has the form $\rho \pi \rho^{-1}$ where $\rho$ is a sequence of flips that moves the labels of $e_{k-1}$ and $e_k$ into an empty convex pentagon, and $\pi$ is the sequence of five flips that exchanges the labels of $e_{k-1}$ and $e_k$. 

Consider the flip sequence  $\sigma \sigma_k \sigma^{-1} = \sigma \rho \pi \rho^{-1} \sigma^{-1} = \sigma \rho \pi (\sigma \rho)^{-1}$. 
The first part of this flip sequence, $\sigma \rho$, moves the labels of $e_0$ and $e_k$ into an empty convex pentagon; the middle part, $\pi$, exchanges them; and the final part, $(\sigma \rho)^{-1}$ reverses the first part.  Therefore this flip sequence realizes an elementary swap of $e_0$   
and $e_k$.  
\end{proof}



