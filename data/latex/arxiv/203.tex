\section{Proof of Lemma~\ref{lemma:clustertree}}
\label{sec:appendix-implicit}

\begin{proof}[{\bf Proof of Lemma~\ref{lemma:clustertree}}]
	We first show that this is true for the clusters defined by the
	primary centers $S$ ($\rho_0(v))$.
        We use the notation
        $\p(u,v)+\p(v,w)$ to indicate joining the two shortest paths
        at $v$. Consider a vertex $v$ with
        Consider a vertex $v$ with
	$\rho_0(v) = s$, and consider all the vertices $P$ on the shortest
	path from $v$ to $s$.  The claim is that for each $u \in P, \rho(u) =
	s$ and $\mb{SP}(u,s)$ is a subpath of $P$. This implies a
	rooted tree. To see that $\rho(u) = s$ note that the shortest path
	from $u$ to a primary vertex $t$ has length $L(\mb{SP}(u,t))$. We can write the
	length of the shortest path from $v$ to $t$ as
	$L(\mb{SP}(v,t)) \leq L(\mb{SP}(v,u) + \mb{SP}(u,t))$ and the length of the
	shortest path from $v$ to $s$ as
	$L(\mb{SP}(v,s)) = L(\mb{SP}(v,u) + \mb{SP}(u,s))$.
	We know that since $\rho_0(v) = s$
	that $L(\mb{SP}(v,s)) < L(\mb{SP}(v,t))$ $\forall t \neq s$. Through substitution and subtraction,
	we see that $L(\mb{SP}(u,s)) < L(\mb{SP}(u,t))$ $\forall t \neq s$. This means that $\rho_0(u) = s$.
	We know that $\mb{SP}(u,s)$ cannot contain the edge $b$ that $v$ takes to reach $u$
	in $\mb{SP}(v,s)$ since $u \in \mb{SP}(v,s)$. Since the search from $u$ excluding
	$b$ will have the same priorities as the search from $v$ when it
	reaches $u$, $\mb{SP}(u,s)$ is a subpath of $P$.
	
	Now consider the clusters defined by $\rho(v)$.  The secondary centers
	associated with a primary center $s$ partition the tree for $s$ into
	subtrees, each rooted at one of those centers and going down until
	another center is hit.  Each vertex in the tree for $s$ will be
	assigned the correct partition by $\rho(v)$ since each will be
	assigned to the first secondary center on the way to the primary
	center.
\end{proof}




% returns the smallest (according to the total
% order on vertices) vertex in the component.  In the graph connectivity,
% this vertex is the representative of the cluster, and in
% biconnectivity queries we can afford to expend such cluster with cost
% $O(k^2)$, if a vertex or an edge in this cluster is involved in the
% query, since the size of the cluster is bounded by $k$. 
