





\section{M\"obius transformation}\label{app:A}

The Riemann sphere $\mathbb{S}^2$ can be blobally
parametrized by stereographic projection by means of $\C:=\mathbb{C} \cup
\{ \infty \}$. A M\"obius transformation is a map 
\begin{align*}
\chi: \C & \longrightarrow \C, \\
c & \longrightarrow \chi(z) := \frac{a z + b}{c z + d}, \quad \quad
a,b,c,d \in \mathbb{C}, \quad \quad ad-bc = 1.
\end{align*}
The set of all M\"obius transformations define a group, denoted by $\Moeb$.


This group is isomorphic to the set of positively oriented conformal maps of 
$\mathbb{S}^2$ endowed with the standard round metrid.


In this appendix we prove the following theorem. 
\begin{thm}
\label{InvThm}
Let $\C := \mathbb{C} \cup \{ \infty \}$ be the Riemann sphere and 
$\Moeb$ 
%\mnote{doesn't there exist a standard notation for the set of all Mobiustrafos? (or is that the standard notation?) Wikipedia says:"The Möbius group is usually denoted $Aut(\hat{C})$"} 
the set of M\"obieus transformations. Let 
$c: \mathbb{S}^1 \longrightarrow \C$ be an embedding.
If there exists $\chi \neq \Id_{\C}$ that leaves $c$ invariant as a 
set, then $c$ is a generalized circle (i.e. a circle or a straight line with the 
point at infinity attached), or there exists $n \in \mathbb{N}$ such that
$\chi^n = \Id_{\C}$ and $c$ is conjugate to a closed curve invariant
under rotations of angle $\frac{2\pi m}{n}$, $m \in \mathbb{Z}$ around the
origin of $\mathbb{C}$.
\end{thm}

By ``invariant as a set'' we mean that there is a diffeomorphism
$f : \mathbb{S}^1 \longrightarrow \mathbb{S}^1$ such that
$\chi \circ c = c \circ f$ (the image of $c$ and $\chi \circ c$
are obviously the same).
A closed embedded curve $c$ is {\it conjugate} to another
closed embedded curve $c_1$ if there exists $\chi_1 \in \Moeb$ such that
$\chi_1 \circ c = c_1$.

\vspace{3mm}

%\begin{proof}
{\it Proof:}
We first note that the problem is invariant under conjugation:
for any  $\xi \in \Moeb$ the conjugate curve $c_{\xi} := 
\xi \circ c$ is invariant (as a set) under
the conjugate transformation $\chi_\xi := \xi \cdot \chi \cdot \xi^{-1}$, 
as it is obvious from:
\begin{align*}
\chi_\xi \circ c_{\xi} = 
(\xi \circ \chi \circ \xi^{-1} ) \circ (\xi \circ c ) =
\xi \circ (\chi \circ c) = \xi \circ c \circ f = c_{\xi} \circ f.
\end{align*}
It is well-known that 
all M\"obius transformations (different from the identity) can be
classified by conjugation into four 
disjoint classes:
parabolic, elliptic, hyperbolic or loxodromic. Each class admits
a canonical representative, in the sense that any element in the class
is conjugate to this representative. The representatives can be chosen as
follows:
\begin{align}
& \mbox{Parabolic:} \quad \quad & & \chi_P(z)= \frac{z}{1+z} &&  \label{list} \\
& \mbox{Elliptic:} \quad \quad & & \chi_E(z)=   e^{i \theta} z, \quad \quad
&&0 \neq \theta \in \mathbb{R}_{\mbox{mod } 2 \pi} \nonumber \\
& \mbox{Hyperbolic:} \quad \quad & & \chi_H(z)=  e^{\lambda} z \quad \quad 
&& \lambda \in 
\mathbb{R} \setminus \{ 0 \} \nonumber \\
& \mbox{Loxodromic:} \quad \quad & & \chi_L(z)=   k z \quad \quad && k \in \mathbb{C}
\setminus \{ \mathbb{R} \} \quad \quad \mbox{and} \quad \quad |k| \neq 1
\nonumber 
\end{align}
Thus, we may assume without loss of generality
that the transformation $\chi$ leaving $c$
invariant is one of these canonical
transformations. Obviously $\chi^m, m \in \mathbb{Z}$ also leaves $c$ invariant.
The action of $\chi^m$ is immediate to write down in the elliptic,
hyperbolic and loxodromic canonical cases.
In the parabolic case, a simple inductive argument
shows that:
\begin{align*}
\chi_P^m (z ) = \frac{z}{m z+1}, \quad \quad m \in \mathbb{Z}.
\end{align*} 
Thus, it follows
that  the cyclic group $\{ \chi^n ; n \in \mathbb{Z} \}$ is finite 
(i.e. $\chi^m = \mbox{Id}$ for some $m \in \mathbb{Z}$)
if and only if $\chi$ is elliptic
and $\frac{\theta}{2\pi} \in \mathbb{Q}_{\mbox{mod } 1}$.

Let us consider first the loxodromic, hyperbolic and parabolic cases. We start
by showing that  the embedded loop $c$ must pass through the
origin $z=0$  of the complex plane. Let $0 \neq z_0 \in \mathbb{C}$ be any point on the curve, i.e. $z_0 \in \mbox{Im} (c)$
and define, for each $m \in \mathbb{Z}$, $z_m := 
\chi^m (z_0) \in \C$. From invariance of the
curve under $\chi$, all points in the sequence $\{ z_m \}$ lie on
the image of the curve. From compactness of $\mbox{Im} (c) \subset \C$
it follows that the set of accumulation points of $\{ z_m\}$ is non-empty
and a subset of $\mbox{Im}(c)$. 


When $\chi$ 
is hyperbolic or loxodromic, the canonical form is
$\chi_k(z) := k z$ with $|k| \neq 1$. The sequences are now 
$z_m := \chi_k^m (z_0) =  k^m z_0$. If $|k| > 1$, the 
sequence converges to $z=0$ as $m \rightarrow -\infty$. If 
$|k| <1$, the sequence converges to $z=0$ as $m \rightarrow \infty$. In either
case $z=0$ is an accumulation point, so the loop $c$ passes throught $z=0$.
When $\chi$ is parabolic, the sequence is
$z_m = \frac{z_0}{m z_0 + 1 }$ which converges to $z=0$ as $|m| \rightarrow 
\infty$, and we reach the same conclusion.

We can now show that a loxodromic M\"obius transformation does not leave
any closed embedded loop invariant. Let us take differentials in the
invariance equation $\chi \circ c = c \circ f$ and evaluate at
the invariant point $p := \{ z=0\}$:
\begin{align*}
d \chi |_p (\dot c) = \dot{f} |_{c^{-1} (p)} \dot{c},
\end{align*}
which simply states the fact that
the differential map of $\chi_p$ must 
preserve the direction of $\dot{c}|_p$ (it may change its scale, but not 
the direction). The differential of $\chi(z)= k z$ at $z=0$
is  $d \chi |_{z=0} = k$. Thus, this differential
acts on a vector $v$ by scaling with $|k|$ and rotating by $\mbox{arg} (k)$.
When $k$ is not real, all vectors $v \neq 0$ change direction and
we reach  a contradiction. Thus, no embedded loop is invariant
under a  loxodromic M\"obius transformation.

We next consider the hyperbolic case. The canonical representative
is now $\chi = \chi_{H}$. Let $\xi$ be a rotation of the form
$\xi(z) = e^{i\alpha} z$ $\alpha \in \mathbb{R}$. Upon conjugation with
$\xi$, the map $\chi_H$ remains unchanged.  The conjugate
curve 
$\xi \circ c$ passes though $z=0$, and the parameter $\alpha$ can be
adjusted so that its tangent vector there points along the
real axis $x$.
%the
%real coordinates $\{ x,y \}$ of $\mathbb{C}$  are
%defined by $z = x + i  y$. 
Since $c$ is an embedded curve, there is a 
neighbourhood $U$ of $z=0$ such that $U \cap c$ is connected 
and in fact a graph over
the real axis. 
After restricting $U$ if necessary we may assume that
$U$ is an open disk centered at $z=0$.
We consider the curve $c_U := c \cap U$ from now on.
This curve can be parametrized by $x$, i.e.
$c(x) = x + i y(x)$ where $y(x)$
is a smooth function of $x \in 
(-\epsilon,\epsilon)$. The parameter $\lambda$
in the definition of 
$\chi_H$ can be assumed to be negative
(if it were positive simply replace 
$\chi_H$ by $\chi_H^{-1}$). Then $\chi_H$ maps $U$ into itself, and leaves
the curve $c_U$ invariant. So, it must be the case
that, for all $x \in (- \epsilon, \epsilon)$:
\begin{align}
e^{\lambda} ( x + i y(x) ) = x' (x) + i y(x'(x))  \quad
\Longleftrightarrow  \quad y(e^{\lambda} x) = e^{\lambda} y(x).
\label{invariance}
\end{align}
where $x'(x)$ indicates the reparametrization of the curve induced
by the M\"obius transformation $\chi_H$. Define the function
$P(u) := e^{-\lambda u} y(e^{\lambda u})$. By construction, $P(u)$ is smooth
on $(-\infty,\lambda^{-1} \ln \epsilon)$. In terms of $P$, the function $y(x)$ restricted to $x>0$ takes the form $y(x)= x P(\lambda^{-1} \ln x)$. 
The  invariance property (\ref{invariance}) becomes, when
applied at the point $x = e^{\lambda u}$:
\begin{align*}
P(u+1)= e^{-\lambda u} e^{-\lambda} y(e^{\lambda u} e^{\lambda} )
= e^{-\lambda u} y (e^{\lambda u} )= P(u).
\end{align*}
So $P(u)$ is a periodic function of period one. We can now compute
the derivative of $y(x)$ (prime denotes derivative with respect to $u$):
\begin{align*}
\frac{dy(x)}{dx} = 
P(\lambda^{-1} \ln x) + \lambda^{-1} P' |_{\lambda^{-1} \ln x}.
\end{align*}
If $P(u)$ is not a constant function
the combination $P(u) + \lambda^{-1} P'(u)$ does not converge 
as $u \rightarrow - \infty$. To show this, take the sequence $u_n = u_0 - n$
with $u_0 \in [-1,0)$ defined by the condition that
$P(u_0)$ attains the supremum of $P(u)$
and another sequence  $u'_n = u_1 - n$ where
$u_1 \in [-1,0)$ is the value where $P(u)$ attains the infimum.
By periodicity, the sequences $P(u_n)$ and $P(u'_n)$ are both constant.
Moreover,  $P'$ vanishes on all points $u_n$ and $u'_n$.
Thus, the sequences $\{ P(u_n) + \lambda^{-1} P'(u_n) \}$
and $\{ P(u'_n) + \lambda^{-1} P'(u'_n) \}$
 converge to the same
limit if and only if $P(u_0) = P(u_1)$, i.e.  if the function $P(u)$
is constant, as claimed.
As a consequence, $\frac{dy}{dx}$ converges
as $x \rightarrow 0^+$ if and only if $P(u) = a$ for some constant $a$,
or equivalently iff $y(x) = ax$. Since,
in our setup, $\frac{dy}{dx}=0$ at $x=0$  we conclude that $y(x)=0$.
We have proved this fact in a neighbourhood $U$ of $0$,
but this extends to the whole loop $c$ by applying repeatedly the transformation
$\chi_H$. In summary, we have shown that the only  embedded
loops invariant under the canonical
representative $\chi_H$ of hyperbolic M\"obius transformations is
the line $(x, y=0)$, and arbitrary rotations thereof around
the origin. We now use the property that
M\"obius transformations map generalized circles 
into generalized circles, and conclude that 
an embedded loop which is not a generalized circle can never be invariant
under a hyperbolic M\"obius transformation.

We want to use a similar argument for the parabolic case. To that aim, it is 
preferable to use a different representative.
More precisely, recall that for $\chi = \chi_P$ given in 
(\ref{list}) the invariant embedded loop  $c$ necessarily passes
through $z=0$. Let us apply a conjugation
with the inversion map $\hat{\xi}(z) = -1/z$. The conjugate
$\widehat{\chi}_P = \hat{\xi} \circ \chi_P \circ \hat{\xi}^{-1}$ is given by
$\widehat{\chi}_P (z) = z -1$ and the conjugate loop
$\hat{c}:= \hat{\xi} \circ c$ passes through the point at infinity. 
%We work from now on
%with $\widehat{\chi}_P$ and $\hat{c}$, and drop the primes for simplicity.
Consider the vector field:
\begin{align*}
\zeta = z^2 \partial_z + \overline{z}^2 \partial_{\overline{z}}.
\end{align*}
This field is smooth in a neighbourhood of the point at infinity.
Indeed,
the vector field
$\partial_{x'} =
\partial_{z'} + \partial_{\overline{z}'}$ is clearly smooth in a neighbourhood
of zero. The inversion map $z' = - \frac{1}{z}$ transforms
this neighbourhood of zero into a neighbourhood of infinity and transforms
the vector field $\partial_{x'}$ into $\zeta$, from which smoothness follows.
In the coordinates $\{x,y\}$ defined by $z= x+ i y$ this vector
field takes the form:
\begin{align*}
\zeta = \left (x^2 - y^2 \right ) \partial_x + 2 xy \partial_y.
\end{align*}
The property of invariance of an embedded loop 
under a M\"obius transformation 
is
preserved by reparametrizations of the curve, so we are free to choose
the parametrization of $\hat{c}$. However,  we must make sure that the 
parameter is smooth everywhere, including a neighbourhood
of infinity. To that aim
we choose to parametrize $\hat{c}$ with
arc length $s$ with respect to the round sphere metric:
\begin{align}
ds^2 = \frac{1}{\left ( 1+ \frac{1}{4} (x^2 + y^2 ) \right )^2}
(dx^2 + dy^2),
\label{metric}
\end{align}
which extends smoothly to the point at infinity.
As before, let $0 \neq z_0= \hat{c}(s_0) = (x_0,y_0) \in \mathbb{C}$
be a point on the curve. 
From the condition that the  tangent vector
$T|_p$ of the curve is unit with respect to (\ref{metric}),
there exists $\alpha \in  [0,2\pi)$ such that:
\begin{align*}
T|_p = F|_p \left ( \cos \alpha \partial_x + \sin \alpha
\partial_y \right ),
\end{align*}
with $F|_p$ determined by:
\begin{align*}
F|_p = \left . 1 + \frac{1}{4} \left ( x^2 + y^2 \right ) \right |_{(x_0,y_0)}.
\end{align*}
We compute the scalar product with the vector $\zeta$ to find:
\begin{align*}
\langle T|_p, \zeta |_p \rangle
=
\left . \frac{\cos \alpha (x^2 - y^2 ) + 2 \sin \alpha xy }
{1 + \frac{1}{4} \left ( x^2 + y^2 \right )}
\right |_{(x_0,y_0)}.
\end{align*}
Consider now the sequence of points $\{ z_m = (x_0 -m, y_0)\} $. From
invariance under $\widehat{\chi}_P$, they also also lie on the curve
$\hat{c}$. In fact, the set $\mbox{Im}(\hat{c})$
defines a periodic submanifold, in the sense that 
a unit translation along the $x$ axis leaves it invariant.
As a consequence, all the the tangent 
vectors $T_{p_m}$
of the curve at each point $z_m$ must be parallel to each other
(in the natural euclidean sense of the term).
Hence $\alpha$ is the same for all $z_m$. Let us
compute the limit along the sequence of the scalar product
$\langle T|_{p_m}, \zeta |_{p_m} \rangle$:
\begin{align*}
\lim_{m \longrightarrow \infty}
\langle T|_{p_m}, \zeta |_{p_m} \rangle
=
\lim_{m \longrightarrow \infty}
\frac{
\cos \alpha ( (x_0-m)^2 - y_0^2 ) + 2 \sin \alpha (x_0 - m)
y_0 }{1 + \frac{1}{4} \left ( (x_0-m)^2 + y_0^2 \right )} =
4 \cos \alpha.
\end{align*}
Given that the curve is smooth evergywhere, including infinity,
and that the sequence $\{ z_m\}$ converges to the point at infinity,
it follows that all the tangent vectors $T|_{p_m}$ must converge,
namely to the unit tangent vector $T_{\infty} $ to the curve there. The scalar
products above must then converge to a single finite value, and this must happen
independently of the initial point $z_0$.
Since the limit depends on $\alpha$ we conclude that $\alpha$ must be
the same for all points along the curve. If $\alpha = \frac{\pi}{2}$
or $\alpha = \frac{3\pi}{2}$
then the curve would be an infinite collection of vertical lines
in the $\{ x,y\}$ plane, all of them passing though the point
at infinity and the curve $\hat{c}$ would not be embedded.
Thus the tangent vector $T_p$ must have a non-zero component
along the $x$ axis everywhere along the curve.
This implies that it can be described as a graph $y(x)$ on the $x$
axis. Since $y(x)$ must reach a local maximum and $\alpha$ vanishes there
we conclude that $\alpha=0$ at all points, and hence that 
$y=y_0 = \mbox{const}$.
So, the embedded loop $\hat{c}$ must be the straight line $y =y_0$. This claim is
for embedded curves invariant under  the parabolic transformation $z \rightarrow 
z -1$. Upon conjugation, and using again that M\"oebius transformations
map generalized circles into generalized circes, we conclude
that the only
embedded closed loops invariant under a parabolic transformation
are generalized circles.

It only remains to consider the elliptic case, i.e. $\chi = \chi_E$.
Since $\chi_E$ is a rotation of angle $\theta$
of the complex plane around its origin, the invariant embedded loop $c$
defines a figure invariant under a rotation of angle $\theta \neq 2 \pi k$,
$k \in \mathbb{Z}$.
Consider the set of all angles $\beta \in (0,2\pi)$
under which this figure is invariant
and let $\beta_0$ be its infimum. If $\beta_0 =0$, the curve
must be a circle. If $\beta_0$ is different from zero, then there must exist 
$n \in \mathbb{N}$ such that $\beta_0 = \frac{2\pi}{n}$ (if such $n$
did not exist, define $n \in \mathbb{N}$ by
$n \beta_0 <2\pi < (n+1) \beta_0 $,  the angle $(n+1) \beta_0
- 2\pi$ is positive, smaller that $\beta_0$ and belongs to the set
of rotation angles that leave the figure invariant, which is a contradiction.)
Thus $\beta_0 = \frac{2\pi}{n}$ and in fact all other symmetry angles
must be a multiple of this (by a similar argument as before).
The number $n$ is called the {\it order of symmetry} of the figure.
In summary, the closed embedded loop $c$ is invariant under $\chi_E$
if and only if it is a circle centered at zero, or a figure with a discrete
rotational symmetry of order $n$. The statement of the theorem then follows
once again from the fact that the collection of generalized circles is preserved
under M\"obius transformations. $\hfill \Box$

\vspace{5mm}

As discussed in the main text, the shadow curve for suitable chosen
observers at any point in the class of black hole spacetimes under
consideration here has the property of being reflection symetric.
In precise terms, let  the map $r: \C
\longrightarrow \C$ be defined by reflection 
with respect to the real axis  $y=0$, i.e. $r(z) = \zb$.
A closed embedded loop
$c : \mathbb{S}^1  \longrightarrow \C$
is {\bf reflection symmetric} if there exists
a smooth map $f_1 : \mathbb{S}^1 \longrightarrow \mathbb{S}^1$ such that
$r \circ c = c \circ f_1$. One checks immediately that $f_1$ is 
a diffeomorphism of $\mathbb{S}^1$ (in fact an orientation reversing
diffeomorphism).  
Our aim is to determine
which elements $\chi \in \Moeb$ have the property that
the conjugate curve $\chi \circ c$ is also 
reflection symmetric.  Thus, we want to impose the 
condition that there exists a diffemorphism $f_2 : \mathbb{S}^1
\longrightarrow \mathbb{S}^1$ such that
$r \circ \chi \circ c = \chi \circ c \circ f_2$, which in turn is equivalent to
$\chi^{-1} \circ r \circ \chi \circ r^{-1} \circ c \circ f_1 = c \circ f_2$,
i.e. to:
\begin{align*}
\chi^{-1} \circ r \circ \chi \circ r^{-1} \circ c  = c \circ f,
\end{align*}
where $f := f_2 \circ f_1^{-1}$ is an orientation preserving
diffeomorphism of $\mathbb{S}^1$.
The map 
$\widetilde{\chi}:=
\chi^{-1} \circ r \circ \chi \circ r^{-1} $ is by construction an element
of the M\"obius group, and leaves the loop defined by $c$ invariant (as 
a submanifold). From Theorem \ref{InvThm} it follows
that $\widetilde{\chi}$ is the indentity map, unless either 
$\mbox{Im}(c)$
is conjugate to a figure with discrete rotational symmetry of order $n$
and, in addition, $\widetilde{\chi}$ is conjugate
to $\chi_{m,n} := z \rightarrow e^{i \frac{2 \pi m}{n}} z$ for some integer $m$
between $-n$ and $n$,  or else  $c$ is a generalized circle.  

In this paper we are interested in M\"obius transformations sufficiently close to
the identity that map reflection symmetric curves into 
reflection symmetric curves. Since, for fixed $n$
$\{ \xi_{m,n}; - n < m < n \}$ is discrete, it is disjoint
to a sufficiently
small neighbourhood of the identity map  $\Id_{\mathbb{C}}$, and we can ignore
the case of discrete rotational symmetry of order $n$.  Also, 
we restrict ourselves to non-degenerate spacetimes points, where the shadow curve
is not a generalized circle (for simplicity we call such
curves ``non--circular''). So, we conclude that
$\widetilde{\chi}$ must be the identity map, i.e.:
\begin{align*}
\chi^{-1} \circ r \circ \chi \circ r^{-1} = \Id_{\C} \quad \quad
\Longleftrightarrow \quad \quad
r \circ \chi \circ r^{-1}  = \chi.
\end{align*}
Letting $\chi$ correspond to the $\mbox{SL}(2,\mathbb{C})$ matrix:
\begin{eqnarray*}
\left ( \begin{array}{cc}
\alpha & \beta \\
\gamma & \delta \\
\end{array}\right ),
\end{eqnarray*}
it is immediate to compute that $ r \circ \chi \circ r^{-1}$ corresponds to the
$\mbox{SL}(2,\mathbb{C})$ matrix:
\begin{eqnarray*}
\left ( \begin{array}{cc}
\overline{\alpha} & \overline{\beta} \\
\overline{\gamma} & \overline{\delta} \\
\end{array} \right ).
\end{eqnarray*}
Thus, is $\chi$ is sufficiently close to the identity map and the
reflection symmetric curve $c$  is non-circular, it must be the
case that $\chi \in SL(2,\mathbb{R})$, i.e. all 
$\alpha, \beta, \gamma, \delta$ are real
parameters.

Our second aim is to identify the infinitessimal transformations with
generate this subgroup of M\"obius transformations. Consider
a one parameter subgroup $\tau: \mathbb{R} \longrightarrow 
\mbox{SL}(2,\mathbb{C})$ of $\mbox{SL}(2,\mathbb{C})$
and denote by
$\chi_{\tau(s)}$, $s \in \mathbb{R}$ 
the corresponding curve in the M\"obius group. A straightforward
computation gives, for each $z \in \mathbb{C}$:
\begin{align*}
\frac{d \chi_{\tau(s)}(z)}{d s}  =
\beta_0 + \left ( \alpha_0 - \delta_0 \right ) - \gamma_0 z^2,
\end{align*}
where $\alpha_0 = \left . \frac{d \alpha(s)}{ds} \right |_{s=0}$,
$\beta_0 = \left . \frac{d \beta(s)}{ds} \right |_{s=0}$,
$\gamma_0 = \left . \frac{d \gamma(s)}{ds}\right |_{s=0}$,
$\delta_0 = \left . \frac{d \delta(s)}{ds}\right |_{s=0}$. The condition that
the curve $\tau(s)$ takes values in $\mbox{SL}(2,\mathbb{C})$ requires that $\delta_0
= - \alpha_0$. Thus, the infinitessimal generator of this one-parameter
subgroup  is:
\begin{align*}
\xi = 
\left ( \beta_0 + 2 \alpha_0 z - \gamma_0 z^2 \right )  \partial_z 
+ 
\left ( \overline{\beta_0} + 2 \overline{\alpha_0} \, \zb- 
\overline{\gamma_0} \, \zb^2 \right )  \partial_{\zb}.
\end{align*}
Thus if we restrict ourselves to the subgroup of transformation 
preserving the reflection symmetry of a non-circular curve $c$,
the generators are: 
\begin{align*}
\xi =  \beta_0 \left ( \partial_z + \partial_{\zb} \right )
+2 \alpha_0 \left ( z\partial_z + \zb \partial_{\zb} \right )
- \gamma_0 \left ( z^2\partial_z + \zb^2 \partial_{\zb} \right ),
\quad \quad 
\alpha_0,\beta_0,\gamma_0 \in \mathbb{R}.
\end{align*}
In terms of Cartesian coordinates $\{ x,y\}$  on the stereograhic plane,
i.e. $z = x + i y$, this vector field becomes:
\begin{align*}
\xi =  \beta_0 \partial_x 
+2 \alpha_0 \left ( x \partial_x + y \partial_y \right )
- \gamma_0 \left ( \left ( x^2 - y^2 \right ) \partial_x
+ 2 x y \partial_{y} \right ).
\end{align*}
So, the three generators of  M\"obius transformations preserving 
reflection symmetry turn out to be  the translations along the
$x$ axis $\xi_1 = \partial_x$, the dilations about the origin
$\xi_2 = x \partial_y + y \partial_x$ and a third conformal Killing vector
given by
$\xi_3 = (x^2 - y^2) \partial_x + 2 xy \partial_y$. These vector fields
generate a Lie algebra with structure constants:
\begin{align*}
[ \xi_1, \xi_2] = \xi_1, \quad \quad
[ \xi_1, \xi_3] = 2\xi_2, \quad \quad
[\xi_2,\xi_3] = \xi_3.
\end{align*}
Note that the subset of reflection symmetric
transformations that leave the origin $\{x=0,y=0\}$
invariant is generated by $\{\xi_2,\xi_3\}$, which is, naturally,
a two-dimensional subalgebra. Another observation is that 
the only element in $\{ \xi_1,\xi_2,\xi_3\}$ which is a Killing vector
of $\mathbb{C} \cup \{ \infty \}$ endowed with the spherical metric
$ds^2 = \left ( 1 + \frac{1}{4} (x^2 + y^2) \right )^{-2} (dx^2 + dx^2)$,
is $4 \xi_1 + \xi_3$ (and its constant multiples). This Killing field
corresponds to rotations of the sphere leaving invariant the 
antipodal points for which the corresponding equator maps onto the real axis by stereographic projection.



