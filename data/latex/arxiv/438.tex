%Zarsiki tirplet and quartet


\section{Zariski triple and 4-ple for cubic-conic-line arrangements}\label{sec:3}

In \cite{bannai-tokunaga}, we give examples of Zariski $N$-ples for conic, 
conic-quartic arrangements. By similar  arguments to those in \cite{bannai-tokunaga},
we give a Zarisiki triple and $4$-ple for cubic-conic-line arrangements. 
Throughout this section, we use the terminology, notation  and results in \cite{bannai-tokunaga}, freely.
The combinatorics considered in this section is as follows:

\bigskip

{\bf Combinatorics 3.}
Let $\mcE$, $\mcL_o$ and $\mcC_i$ ($i = 1, 2, 3$) be as below. Put  $\mcB = \mcE  + \mcL_o + \sum_{i=1}^3 \mcC_i$:

\begin{enumerate}

\item[(i)] $\mcE$: (a)  a smooth cubic or  (b) a nodal cubic. 

\item[(ii)] $\mcL_o$: a transversal line to $\mcE$ and we put $\mcE \cap \mcL_o = \{p_1, p_2, p_3\}$.

\item[(iii)] $\mcC_i$ ($i = 1, 2, 3$):  contact conics to $\mcE + \mcL_o$. Each of them
is tangent to $\mcE + \mcL_o$ at four points

\item[(iv)] $\mcC_i$ and $\mcC_j$ intersect transversally for $i < j$ and  $\mcC_1\cap \mcC_2 \cap \mcC_3 = \emptyset$.

\end{enumerate}



%
%\bigskip
%
%{\bf Combinatorics 4}
%
%We put $\mcE$, $\mcL_o$ and $\mcC_i$ ($i = 1, 2, 3$) as follows:
%
%\begin{enumerate}
%
%\item[(i)] $\mcE$:  a nodal cubic curve.
%
%\item[(ii)] $\mcL_o$: a transversal line to $\mcE$ and we put $\mcE \cap \mcL_o = \{p_1, p_2, p_3\}$.
%
%\item[(iii)] $\mcC_i$ ($i = 1, 2, 3$):  contact conics to $\mcE + \mcL_o$. Each of them
%is tangent to $\mcE + \mcL_o$ at four points
%
%\item[(iv)] $\mcC_i$ and $\mcC_j$ intersect transversally for $i < j$ and 
%$\mcC_1\cap \mcC_2 \cap \mcC_3 = \emptyset$.
%
%\end{enumerate}
%
%We put $\mcB = \mcE  + \mcL_o + \sum_{i=1}^3 \mcC_i$.

%\bigskip
%
%
%We make use of the argument in  \cite{bannai-tokunaga} and have
%
%\begin{thm}\label{thm:z-tri-quar}{
%\begin{enumerate}
%
%\item[(i)] There exists a Zariski quartet for Comibnatorics 3.
%
%\item[(ii)] There exsits a Zariski triple for Combinatorics 4
%
%\end{enumerate}
%
%}
%\end{thm}
%
%
%\begin{rem} {\rm By increasing  the number of contact conics, our result can be generaiized to
%Zariski $N$-plet in the same way as in \cite{bannai-tokunaga}.
%%, bannai-tokunaga17}.
%}
%\end{rem}


 In the construction of plane curves with Combinatorics 3-(a) and (b),  how to find a contact conic $\mcC$ to 
 $\mcE + \mcL_o$ is crucial and we make use of bisections of elliptic surfaces as we did in \cite{bannai-tokunaga}.
 Let us recall that  a bisection is defined as follows:
 
 
 \begin{defin}\label{def:bisection}{\rm
 Let $\varphi : S \to C$ be an elliptic surface over $C$. Let $F$ be a general fiber of $\varphi$.
 A bisection of $\varphi$ is a horizontal curve $D$ with $FD = 2$. Here, a horizontal curve
 with respect to $\varphi$ is a curve that does not contain any fiber components.
 }
 \end{defin}

  Put $\mcQ = \mcE + \mcL_o$ and let $\varphi_{\mcQ, z_o} : S_{\mcQ, z_o} 
 \to \PP^1$ be the rational elliptic surface as before and let $f_{\mcQ, z_o} : S_{\mcQ, z_o} \to
 \PP^2$ be the generically $2$-$1$ morphism. Likewise \cite{bannai-tokunaga}, we construct
 a contact conic $\mcC$ as the image $f_{\mcQ, z_o}(D)$ of an irreducible bisection $D$.
  

 
{\sl Combinatorics 3-(a).} Our proof is almost parallel to that of \cite[Proposition 5]{bannai-tokunaga}.
Let $P_i$ ($i = 1, 2, 3, 4$) be the rational points introduced
in Proof for Combinatorics 1-(a) of  the previous section. Define $Q_i \in E_{\mcQ, z_o}(\CC(t))$ 
($i = 1, 2, 3, 4$)
by 
\[
[Q_0\dot{+}P_{\tau}\,\,  Q_1\,\,  Q_2\,\,  Q_3] = [P_0\,\,  P_1\,\,  P_2 \,\,  P_3] 
 \left [ \begin{array}{cccc}
         2 & -1 & -1 & -1 \\
        -1 & 2 & 0 & 0 \\
        -1 & 0 & 2 & 0 \\
        -1 & 0 & 0 & 2
          \end{array}
          \right ].
\]
%As $\gamma_{\mcQ, z_o}(Q_i) = [0, 0, 0, 0], (i = 0, 1, 2, 3)$, $\{Q_0, Q_1, Q_2, Q_3\}$ is a set of 
%generators of $E_{\mcQ, z_o}(\CC(t))^0$.

Likewise \cite[p. 234]{bannai-tokunaga}, we now consider  six irreducible bisections  $D_0, \ldots, D_5$ as
follows:
\begin{enumerate}      
\item[(i)] $s(D_0) = -s_{Q_0}$,  $s(D_i) = - s_{Q_1} (1 \le i \le 3)$, $s(D_4) = - s_{Q_2}$, 
$s(D_5) = -s_{Q_0 + Q_1}$. Here $s(D_i)$ is the section determined uniquely by $D_i$ in \cite[Lemma~5.1]{shioda90}. 
\item[(ii)] $f_{\mcQ, z_o}(D_i)$ $(i = 0, 1,\ldots, 5)$  are contact conics to $\mcQ$ tangent at
$4$ distinct points.

\item[(iii)]  $\mcC_i$ and $C_j$ intersect transversally if $i \neq j$ and $\mcC_1\cap \mcC_2 \cap \mcC_3 = \emptyset$.
\end{enumerate}

Now if we put
\[
%\begin{array}{ccc}
\mcB^1  :=  \mcQ + \mcC_1 + \mcC_2 + \mcC_3,  \,\mcB^2  :=   \mcQ + \mcC_0 + \mcC_1 + \mcC_2,\, 
\mcB^3  :=  \mcQ + \mcC_0 + \mcC_1 + \mcC_4, \, \mcB^4  :=   \mcQ + \mcC_0 + \mcC_1 + \mcC_5.
%\end{array}
\]
%\begin{eqnarray*}
%\mcB^1 & = & \mcQ + \mcC_1 + \mcC_2 + \mcC_3 \\
%\mcB^2 & =  & \mcQ + \mcC_0 + \mcC_1 + \mcC_2 \\
%\mcB^3 & = & \mcQ + \mcC_0 + \mcC_1 + \mcC_4 \\
%\mcB^4 & = &  \mcQ + \mcC_0 + \mcC_1 + \mcC_5
%\end{eqnarray*}

Then by \cite[Theorem~4, Corollary~3, Corollary~4]{bannai-tokunaga}, we have

\begin{prop}\label{prop:key}{
\begin{enumerate}
\item[(i)] $\Cov_b(\PP^2, 2\mcQ + p(\mcC_i + \mcC_j), D_{2p}) \neq \emptyset$ if and only if $\{i, j\} \subset \{1, 2, 3\}$.
\item[(ii)]  $\Cov_b(\PP^2, 2\mcQ + p(\mcC_i + \mcC_j +\mcC_k), D_{2p}) \neq \emptyset$ if and only if $\{i, j, k\} \subset \{1, 2, 3\}$ or $\{0, 1, 5\}$.
\end{enumerate}
}
\end{prop}


From Proposition~\ref{prop:key},  we have

\begin{prop}\label{prop:z-4plet}{If $D_0, \ldots, D_5$ as above exist for 
$\mcQ$, $(\mcB^1, \mcB^2, \mcB^3, \mcB^4)$  is a Zarisiki 4-ple.
}
\end{prop}

%As we see in the next section, the six bisections as above exist for $\mcQ$ given
%by an explicit equation. This shows that there exists a Zariski quartet for Combinatorics 3.
% 



%\bigskip 


{\sl  Combnatorics 3-(b).} Our proof is almost parallel to that of \cite[Thoerem 1]{bannai-tokunaga}.
Let $P_i$ ($i = 1, 2, 3$) be the rational points intruduced
in Proof for Combinatorics 1-(a) of  the previous section. Define $Q_i \in E_{\mcQ, z_o}(\CC(t))$ 
($i = 1, 2, 3$) by $Q_i := [2]P_i$ $(i = 1, 2, 3)$, respectively. We now consider $5$ bisections as
follows:

\begin{enumerate}      
\item[(i)]  $s(D_i) = - s_{Q_1} (1 \le i \le 3)$, $s(D_4) = - s_{Q_2}$, 
$s(D_5) = -s_{Q_3}$. Here $s(D_i)$ is the section determined uniquely by $D_i$ in \cite[Lemma~5.1]{shioda90}. 
\item[(ii)] $f_{\mcQ, z_o}(D_i)$ $(i =1,\ldots, 5)$  are contact conics to $\mcQ$ tangent at
$4$ distinct points.

\item[(iii)]  $\mcC_i$ and $C_j$ intersect transversally if $i \neq j$ and $\mcC_1\cap \mcC_2 \cap \mcC_3 = \emptyset$.
\end{enumerate}

Now if we put
\[
\begin{array}{ccccc}
\mcB^1  :=  \mcQ + \mcC_1 + \mcC_2 + \mcC_3,  & &
\mcB^2  :=   \mcQ + \mcC_1 + \mcC_2  + \mcC_4,  & &
\mcB^3 : =  \mcQ + \mcC_1 + \mcC_4 + \mcC_5. 
\end{array}
\]
%\begin{eqnarray*}
%\mcB^1 & = & \mcQ + \mcC_1 + \mcC_2 + \mcC_3 \\
%\mcB^2 & =  & \mcQ + \mcC_1 + \mcC_2  + \mcC_4\\
%\mcB^3 & = & \mcQ + \mcC_1 + \mcC_4 + \mcC_5 
%\end{eqnarray*}

Then by \cite[Theorem~4]{bannai-tokunaga}, we have

\begin{prop}\label{prop:key2}{
 $\Cov_b(\PP^2, 2\mcQ + p(\mcC_i + \mcC_j), D_{2p}) \neq \emptyset$ if and only if $\{i, j\} \subset \{1, 2, 3\}$.
}
\end{prop}

This shows that there exist no homeomrophism $(\PP^2, \mcB^i) \to (\PP^2, \mcB^j)$ if 
$i \neq j$ by a similar argument to the proof of \cite[Proposition~3]{bannai-tokunaga}. Thus we have

\begin{prop}\label{prop:z-triple}{If $D_1, \ldots, D_5$ as above exist for 
$\mcQ$, $(\mcB^1, \mcB^2, \mcB^3)$  is a Zarisiki triple.
}
\end{prop}


%\begin{rem}{\rm In Figure 2,  if we blow down $f(\Theta_{4,0})$, $f(O)$, $f(\Theta_{0,1})$, 
% $f(\Theta_{1,1})$,  $f(\Theta_{2,1})$ and $f(\Theta_{3,1})$, in this order, the the image 
% $\mcQ_1$of
% the branch locus $\Delta_f$ of $f$ consists of two smooth conics intersecting $4$ points. This 
% case is the one we consider in \cite[Theorem1]{bannai-tokunaga}.
%}
%\end{rem}

As we see in the next section, the six (resp. five) bisections as above exist for $\mcQ$ given
by an explicit equation. Thus we have

\begin{thm}\label{thm:z-tri-quar}{
 There exists a Zariski $4$-ple (resp. triple) for Comibnatorics 3-(a) (resp. (b)). 

%\begin{enumerate}
%
%\item[(i)] There exists a Zariski quartet for Comibnatorics 3-(a).
%
%\item[(ii)] There exsits a Zariski triple for Combinatorics 3-(b).
%
%\end{enumerate}

}
\end{thm}


\begin{rem} {\rm (i) By increasing  the number of contact conics, our result can be generalized to
Zariski $N$-plet in the same way as in \cite{bannai-tokunaga}.

(ii) In Figure 2,  if we blow down $f(\Theta_{4,0})$, $f(O)$, $f(\Theta_{0,1})$, 
 $f(\Theta_{1,1})$,  $f(\Theta_{2,1})$ and $f(\Theta_{3,1})$, in this order, the the image 
 $\mcQ_1$of
 the branch locus $\Delta_f$ of $f$ consists of two smooth conics intersecting $4$ points. This 
 is the one we consider in \cite[Theorem1]{bannai-tokunaga}.

%, bannai-tokunaga17}.
}
\end{rem}


