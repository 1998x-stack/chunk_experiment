%section 3

\section{Proof of Theorem~\ref{thm:comb-1}}
 
We first recall the double cover diagram for $S_{\mcQ, z_o}$. In our case,
$\widehat{\Sigma}_2$ can be successively blown down to $\PP^2$. We denote it by $\overline{q} : \widehat{\Sigma}_2 \to \PP^2$.
We then have the following combined diagram:
\[
\begin{CD}
S' @<{\mu}<< S_{\mcQ, z_o} \\
@V{f'}VV                 @VV{f}V \\
\Sigma_2@<<{q}< \widehat{\Sigma}_2 @>{\overline{q}}>>  \PP^2. 
\end{CD}
\]
Note that $f_{\mcQ, z_o} = \overline{q}\circ f$.

 Theorem~\ref{thm:comb-1} will be proved based on \cite[Theorem 3.2]{tokunaga14} and the following lemma.  

%In the following two cases, we give an explicit description of the images of sections under $f_{\mcQ, z_o}$ which play important roles to prove Theorem~\ref{thm:comb-1}.



 \begin{lem}\label{lem:key-comb1}{ Let $P_1, P_2, P_3\in E_{\mcQ,z_o}(\CC(t))$ be rational points such that $f_{\mcQ, z_o}(s_{P_i})=\mcL_i$. 

 Then there exists a $D_{2p}$-cover $\pi_p : X_p \to \widehat{\Sigma}_2$ such that
 
 \begin{itemize}
 
   \item $D(X_p/\widehat{\Sigma}_2) = S_{\mcQ, z_o}$ and 
   
   \item $\Delta_{\beta_2(\pi_p)} = \sum_{i=1}^3 s_{P_i} + \sigma_f^*(\sum_{i=1}^3 s_{P_i})$ 
   % (resp. $ =  \sum_{i=2}^4 s_{Q_i} + \sigma_f^*(\sum_{i=2}^4 s_{Q_i})$),
 \end{itemize}
 if and only if there exists a $D_{2p}$-cover $\overline{\pi}_p : \overline {X}_p \to \PP^2$ such that  
 \begin{itemize}
 
  \item $\Delta_{\overline{\pi}_p} = \mcQ + \sum_{i=1}^3 \mcL_i$ %(resp. $= \mcQ +  \sum_{i=2}^4 \mcL_i$) 
  and 
  \item the branch locus of $\beta_1(\overline{\pi}_p) = \mcQ$.
  \end{itemize}       
 }
 \end{lem}
 
 \proof Assume that there exists a $D_{2p}$-cover of $\widehat{\Sigma}_2$ described as above. Consider 
 the Stein factorizaiton $\overline{\pi}_p : \overline{X}_p \to \PP^2$ of $\overline{q}\circ \pi_p$. As the branch
 locus of $\overline{\pi}_p$ is $\overline {q}(\Delta_{\pi_p})$, 
 $\Delta_{\overline{\pi}_p} = \mcQ + \sum_{i=1}^3 \mcL_i$ %(resp. $= \mcQ +  \sum_{i=2}^4 \mcL_i$) 
  and the covering $\beta_1(\overline{\pi}_p)$ is branched along $\mcQ$. Conversely, suppose that there exists a $D_{2p}$-cover 
 $\overline{\pi}_p : \overline{X}_p \to \PP^2$ satisfying the above condition.
 Take the $\CC(\overline{X}_p)$-normalization $\pi_p : X_p \to \widehat{\Sigma}_2$ of $\widehat{\Sigma}_2$.
 Since $\Delta_{\beta_1(\overline{\pi}_p)} = \mcQ$, $\overline{q}(\Delta_{\beta_1(\pi)_p}) = \mcQ$. Hence we infer 
 that $D(X_p/\widehat{\Sigma}_2)$ is a double cover of $\widehat{\Sigma}_2$ branched along $f(O)$ and the 
 proper transform of $\mcQ$, i.e., $D(X_p/\widehat{\Sigma}_2) = S_{\mcQ, z_o}$ and $\beta_1(\pi_p) = f$. 
 Let $\beta_1(\pi_p) : X_p \to S_{\mcQ, z_o}$ be the $p$-cyclic cover determined by $\pi_p$. By \cite[Remark 3.1, (i)]{tokunaga14},
 any irreducible component of singular fibers can not be contained in the branch locus of $\beta_2(\pi_p)$. Hence
 $\overline{q}\circ f(\Delta_{\beta_2(\pi_p)}) = \sum_{i=1}^3 \mcL_i$ %(resp. $= \sum_{i=2}^4\mcL_i)$.
 \qed
 
By the above lemma, we will be able to choose sections appropriately in constructing our configurations so that dihedral covers exist or do not exist. The difference in (non-)existence allows us to distinguish our configurations topologically. 
\bigskip

{\bf Proof for Combinatorics 1-(a)} 
% \begin{lem}\label{lem:key-comb1}{ Let $P_1, P_2, P_3\in \MW (S_{\mcQ, z_o})$ be sections such that $\bar{q}\circ f(s_{P_i})=\mcL_i$ are lines in $\PP^2$. 
%
% Then there exists a $D_{2p}$-cover $\pi_p : X_p \to \widehat{\Sigma}_2$ such that
% 
% \begin{itemize}
% 
%   \item $D(X_p/\widehat{\Sigma}_2) = S_{\mcQ, z_o}$ and 
%   
%   \item $\Delta_{\beta_2(\pi_p)} = \sum_{i=1}^3 s_{P_i} + \sigma_f^*(\sum_{i=1}^3 s_{P_i})$ 
%   % (resp. $ =  \sum_{i=2}^4 s_{Q_i} + \sigma_f^*(\sum_{i=2}^4 s_{Q_i})$),
% \end{itemize}
% if and only if there exists a $D_{2p}$-cover $\overline{\pi}_p : \overline {X}_p \to \PP^2$ such that  
% \begin{itemize}
% 
%  \item $\Delta_{\overline{\pi}_p} = \mcQ + \sum_{i=1}^3 \mcL_i$ %(resp. $= \mcQ +  \sum_{i=2}^4 \mcL_i$) 
%  and 
%  \item the branch locus of $\beta_1(\overline{\pi}_p) = \mcQ$.
%  \end{itemize}       
% }
% \end{lem}
% 
% \proof Assume that there exists a $D_{2p}$-cover of $\widehat{\Sigma}_2$ described as above. Consider 
% the Stein factorizaiton $\overline{\pi}_p : \overline{X}_p \to \PP^2$ of $\overline{q}\circ \pi_p$. As the branch
% locus of $\overline{\pi}_p$ is $\overline {q}(\Delta_{\pi_p})$, 
% $\Delta_{\overline{\pi}_p} = \mcQ + \sum_{i=1}^3 \mcL_i$ %(resp. $= \mcQ +  \sum_{i=2}^4 \mcL_i$) 
%  and the covering $\beta_1(\overline{\pi}_p)$ is branched along $\mcQ$. Conversely, suppose that there exists a $D_{2p}$-cover 
% $\overline{\pi}_p : \overline{X}_p \to \PP^2$ satisfying the above condition.
% Take the $\CC(\overline{X}_p)$-normalization $\pi_p : X_p \to \widehat{\Sigma}_2$ of $\widehat{\Sigma}_2$.
% Since $\Delta_{\beta_1(\overline{\pi}_p)} = \mcQ$, $\overline{q}(\Delta_{\beta_1(\pi)_p}) = \mcQ$. Hence we infer 
% that $D(X_p/\widehat{\Sigma}_2)$ is a double cover of $\widehat{\Sigma}_2$ branched along $f(O)$ and the 
% proper transform of $\mcQ$, i.e., $D(X_p/\widehat{\Sigma}_2) = S_{\mcQ, z_o}$ and $\beta_1(\pi_p) = f$. 
% Let $\beta_1(\pi_p) : X_p \to S_{\mcQ, z_o}$ be the $p$-cyclic cover determined by $\pi_p$. By \cite[Remark 3.1, (i)]{tokunaga14},
% any irreducible component of singular fibers can not be contained in the branch locus of $\beta_2(\pi_p)$. Hence
% $\overline{q}\circ f(\Delta_{\beta_2(\pi_p)}) = \sum_{i=1}^3 \mcL_i$ %(resp. $= \sum_{i=2}^4\mcL_i)$.
% \qed
% 
$\mcE$: a smooth cubic. Choose $z_o \in \mcE\setminus \{p_1, p_2, p_3\}$ and let $S_{\mcQ, z_o}$ be
the rational elliptic surface as in the Introduction. 
%$\varphi_{\mcQ, z_o} : S_{\mcQ, z_o} \to \PP^1$ has $4$ reducible singular fibers. All of them are of type either $\I_2$ or $\III$ type. Hence $T_{\varphi_{\mcQ, z_o}}
% = A_1^{\oplus 4}$ and $G_{\Sing(\varphi_{\mcQ, z_o})} = (\ZZ/2\ZZ)^{\oplus 4}$. Since
%$\mcL_o$ gives rise to a $2$-torsion section,  by \cite{oguiso-shioda},
By Section 1,  $E_{\mcQ, z_o}(\CC(t)) \cong
D_4^*\oplus \ZZ/2\ZZ$. We choose generators $P_0, P_1, P_2, P_3$ of the  $D_4^*$ part such that
\[
[\langle P_i, P_j \rangle ] 
= \left [ 
   \begin{array}{cccc}
      2 & 1 & 1& 1 \\
      1 & 1 & \frac 12 & \frac 12 \\
      1 & \frac 12 & 1 & \frac 12 \\
      1 & \frac 12 & \frac 12 & 1 
    \end{array}
   \right ].
\]   
We denote the $2$-torsion point by $P_{\tau}$. Let $s_{P_i}$ be the corresponding section for each $P_i$ $(i = 0, 1, 2, 3, \tau)$.
Since $\langle P_i, P_j\rangle$ is determined by the intersection numbers, $s_{P_i}s_{P_j}$, $s_{P_i}O$, and
$\mathrm{Contr}_v(s_{P_i}),  \mathrm {Contr}_v(s_{P_i}, s_{P_j})$, i.e., 
at which component of each singular fiber $s_{P_i}$ intersects, we have may assume that the following:    

\begin{enumerate}
 
  \item[(i)] By \cite[Theorem 10.8]{shioda90}, we may assume that $s_{P_i}O = 0$ $(i = 0, 1, 2, 3)$.
  \[
  \sum_{v \in  \Red(\varphi)}\mathrm{Contr}_v(s_{P_i}),  
  \sum_{v \in  \Red(\varphi)}\mathrm {Contr}_v(s_{P_i}, s_{P_j}) = 0, 1/2, 1, 3/2, \mbox{\rm or}\,\,  2.
  \]
  \item[(ii)] For $P_i$ $(i = 1, 2, 3)$, $\langle P_i, P_i \rangle = 1$ implies that 
  $\sum_v \mathrm{Contr}_v(s_{P_i}) = 1$. Also for $\{i, j\} \subset \{1, 2, 3\}, i\neq j$, $\langle P_i, \pm P_j \rangle = 1/2$
  implies $s_{P_i}s_{P_j} = 0$, $\sum_v \mathrm{Contr}_v(s_{P_i}, s_{P_j}) = 1/2$.
  
  \item[(iii)] For $P_0$,  $\sum_v \mathrm{Contr}_v(s_{P_0}) = 0$.
  
  \item[(iv)] For  $P_{\tau}$, as $P_{\tau}$ is a torsion, $\langle P_{\tau}, P_{\tau} \rangle = 0$ and we have 
  $\sum_v \mathrm{Contr}_v(s_{P_{\tau}}) = 2$.
  
\end{enumerate}

From the facts as above, we have that 
\[
\begin{array}{cclcc}
\gamma_{\mcQ, z_o} (P_{\tau}) =   (1, 1, 1,1) & &
\gamma_{\mcQ, z_o} (P_0)  =  (0, 0, 0,0) &&
\gamma_{\mcQ, z_o} (P_1)  =  (1, 1, 0,0)  \\
\gamma_{\mcQ, z_o} (P_2)  =  (1, 0, 1,0) & &
\gamma_{\mcQ, z_o} (P_3)  =  (1, 0, 0,1)  \,\, \mbox{\rm or} \,\,(0, 1, 1, 0) & & 
\end{array}
\]
%\begin{eqnarray*}
%\gamma_{\mcQ, z_o} (P_{\tau}) & = & (1, 1, 1,1) \\
%\gamma_{\mcQ, z_o} (P_0) & = & (0, 0, 0,0) \\
%\gamma_{\mcQ, z_o} (P_1) & = & (1, 1, 0,0) \\
%\gamma_{\mcQ, z_o} (P_2) & = & (1, 0, 1,0) \\
%\gamma_{\mcQ, z_o} (P_3) & = & (1, 0, 0,1)  \,\, \mbox{\rm or} \,\,(0, 1, 1, 0)
%\end{eqnarray*}
By replacing $P_3$ by $P_3 \dot{+} P_\tau$, if necessary, we may assume that
 $\gamma_{\mcQ, z_o} (P_3)  =  (1, 0, 0,1)$.   Put $P_4:= P_2 \dot {-} P_3\dot{+}P_{\tau}$. Then
$\gamma_{\mcQ, z_o}(P_4) = (1, 1, 0, 0)$. We now label the irreducible components of the singular fibers as in the following figure:
%(cf. \cite[No. 13, p. 87-88]{tokunaga12}. Note that we use different labels.)
\begin{center}
\input No13b.tex

Figure 1
\end{center}
% 
%\cite[p. 88]{tokunaga12}. In Figure 13 in \cite[p. 88]{tokunaga12}, $s_i$ $(i = 1, 2, 3)$ correspond to $s_{P_i}$ ($i = 1, 2, 3$), 
%respectively.
(Note that we label singular fibers of type $\III$ similarly, if they exist.) 
%Also we label irreducible components differently from those in \cite[No. 13, p. 87- 88]{tokunaga12}). 
%Let $f: S_{\mcQ, z_o} \to \widehat{\Sigma}_2$ be the one in the
%double cover diagram.
We now blow down  smooth rational curves $f(\Theta_{0, 0}), f(O), f(\Theta_{1, 1}),
f(\Theta_{2, 1}),  f(\Theta_{3, 1})$  in this order. The resulting surface is $\PP^2$ and  this is the morphism
$\overline q$ in this case.
Note that $f_{\mcQ, z_o} = \overline{q}\circ f$ and $\overline {q}\circ f(O\cup \Theta_{0,0}) = z_o$. 

%By combining the double cover diagram, we have the diagram below:
%\[
%\begin{CD}
%S' @<{\mu}<< S_{\mcQ, z_o} \\
%@V{f'}VV                 @VV{f}V \\
%\Sigma_2@<<{q}< \widehat{\Sigma}_2 @>{\overline{q}}>>  \PP^2. 
%\end{CD}
%\]
%As $f_{\mcQ, z_o} = \overline{q}\circ f$, we have

\begin{lem}\label{lem:image-1}{
\begin{enumerate}
 \item[(i)]  The image of the fixed locus of $[-1]_{\varphi}$ is a smooth cubic $\mcE$ and a transversal line $\mcL_o$ to $\mcE$. Moreover, $\overline {q}\circ f(s_{P_{\tau}}) = \mcL_o$.
 
 \item[(ii)]  $\{f_{\mcQ, z_o}(\Theta_{1,1}), f_{\mcQ, z_o}(\Theta_{2,1}), f_{\mcQ, z_o}(\Theta_{3,1})\} =
 \mcE\cap \mcL_o$. We denote $p_i = f_{\mcQ, z_o}(\Theta_{i,1})$.
 
\item[(iii)]  Put $\mcL_i := f_{\mcQ, z_o}(s_{P_i})$. Then $f_{\mcQ, z_o}(s_{P_i})$ $(i =  1, 2, 3, 4)$ passes through $p_i$ ($i = 1, 2, 3$), respectively.
 Also  $\mcL_4$ passes through $p_1$.

 
 \item[(iv)] Furthermore, 
 (a) $\mcL_i$ is tangent to $\mcE$ at a point other than $p_i$ ($p_1$ for $\mcL_4$)   or 
 $p_i$ (resp. $p_1$) is an inflection point of $\mcE$ and $\mcL_i$ (resp. $\mcL_4$) is
 an inflectional tangent line.
 
 \end{enumerate}
 }
 \end{lem}
 
 \proof 
 The statements (i) and (ii) are immediate from our construction of $S_{\mcQ, z_o}$.
% 
%  For the statements (i), we refer \cite[\S 6, Observation~6.3 and Lemma~6.4]{miranda-persson}. Note that $s_{P_{\tau}}$ is 
% contained in the fixed locus of $[-1]_{\varphi}$ as $s_{P_{\tau}}$ is a $2$-torsion section. The statement (ii) is immediate from our construction.  
% 
 We show that the statements (iii) and (iv) hold for $\mcL_1= f_{\mcQ, z_o}(s_{P_1})$ only, as our proof for the remaining sections are the same.
 Since $\gamma_{\mcQ, z_o}(P_1) = (1, 1, 0, 0)$, $s_{P_1}\Theta_{2,1} = 1$.
 This shows that  $f_{\mcQ, z_o}(\Theta_{2,1}) \in \mcL_1$. 
 We now go on to (iv). 
% Again we  only prove the case for $\mcL_1$, as the remaining cases can be 
% proved similarly. 
 Since $\gamma_{\mcQ, z_o}(P_1) = (1, 1, 0, 0)$, $s_{P_1}\Theta_{1,0} = 0$ and $z_o \not\in \mcL_1$.
 As general fiber
 $F$ fo $\varphi_{\mcQ, z_o}$ is mapped to a line $l$ through $z_o$ and $s_{P_1}F = 1$,
 $z_o \not\in f_{\mcQ, z_o}(s_{P_1})$ implies that $\mcL_1\cap l$ consists of only one point.  This shows that
 $\mcL_1$ is a line. If $\mcL_1$ is not the line described in (iv) , the closure
 of $f_{\mcQ, z_o}^{-1}(\mcL_1 \setminus ( \mcL_1\cap \mcE))$ is irreducible. on the other hand, it must contains
 both $s_{P_1}$ and $[-1]_{\varphi_{\mcQ, z_o}}(s_{P_1})$, which is a contradiction. \qed
  %we have $s_{P_1}\Theta_{0,0} = 0$ and
  
\begin{remark}\label{rem:obs-1} Conversely, from the proof of Lemma \ref{lem:image-1} we observe that:

 \begin{enumerate}
 \item[(i)] Any tangent line to $\mcE$ through $p_i$ gives rise a point $P \in E_{\mcQ, z_o}(\CC(t))$ with $\langle P, P \rangle = 1$.
 
 \item[(ii)] Any contact conic to $\mcQ$ through $z_o$ gives rise a point $P \in E_{\mcQ, z_o}(\CC(t))$ with $\langle P, P \rangle = 2$.
 
 \end{enumerate}
\end{remark}
  
Now let
\[
\mcB^1=\mcE +\mcL_0+\sum_{i=1}^3 \mcL_i, 
\mcB^2=\mcE +\mcL_0+\sum_{i=2}^4 \mcL_i.
\]
Then by  \cite[Theorem 3.2]{tokunaga14} and  Lemma~\ref{lem:key-comb1}, a $D_{2p}$-cover branched at $2(\mcE+\mcL_0)+p(\sum_{i=2}^4 \mcL_i)$ exists, but does not exist for $2(\mcE +\mcL_0)+p(\sum_{i=1}^3 \mcL_i)$. Hence, $(\mcB^1, \mcB^2)$  is a Zariski pair
if their combinatorics are the same.
 \bigskip 
 
 %以下の証明は,(1-b)と共通で使う事ができるように記号をそろえて下さい.
 
%Theorem~\ref{thm:comb-1} for Combinatorics 1 follows from Theorem~\ref{thm:criterion} and the following 
% lemma:
% 
% \begin{lem}\label{lem:key-comb1}{ 
% There exists a $D_{2p}$-cover $\pi_p : X_p \to \widehat{\Sigma}_2$ such that
% 
% \begin{itemize}
% 
%   \item $D(X_p/\widehat{\Sigma}_2) = S_{\mcQ, z_o}$ and 
%   
%   \item $\Delta_{\beta_2(\pi_p)} = \sum_{i=1}^3 s_{P_i} + \sigma_f^*(\sum_{i=1}^3 s_{P_i})$ (resp. 
%   $ =  \sum_{i=2}^4 s_{P_i} + \sigma_f^*(\sum_{i=2}^4 s_{P_i})$),
% \end{itemize}
% if and only if there exists a $D_{2p}$-cover $\overline{\pi}_p : \overline {X}_p \to \PP^2$ such that
% 
% \begin{itemize}
% 
%  \item $\Delta_{\overline{\pi}_p} = \mcQ + \sum_{i=1}^3 \mcL_i$ (resp. $= \mcQ +  \sum_{i=2}^4 \mcL_i$) and 
%  \item the branch locus of $\beta_1(\overline{\pi}_p) = \mcQ$.
%  \end{itemize}       
% }
% \end{lem}
% 
% \proof Assume that there exists a $D_{2p}$-cover of $\widehat{\Sigma}_2$ described as above. Consider 
% the Stein factorizaiton $\overline{\pi}_p : \overline{X}_p \to \PP^2$ of $\overline{q}\circ \pi_p$. As the branch
% locus of $\overline{\pi}_p$ is $\overline {q}(\Delta_{\pi_p})$, 
% $\Delta_{\overline{\pi}_p} = \mcQ + \sum_{i=1}^3 \mcL_i$ (resp. $= \mcQ +  \sum_{i=2}^4 \mcL_i$)  and the
% $\beta_1(\overline{\pi}_p)$ is branched along $\mcQ$. Conversely, suppose that there exists a $D_{2p}$-cover 
% $\overline{\pi}_p : \overline{X}_p \to \PP^2$ satisfying the above condition.
% Take the $\CC(\overline{X}_p)$-normalization $\pi_p : X_p \to \widehat{\Sigma}_2$ of $\widehat{\Sigma}_2$.
% Since $\Delta_{\beta_1(\overline{\pi}_p)} = \mcQ$, $\overline{q}(\Delta_{\beta_1(\pi)_p}) = \mcQ$. Hence we infer 
% that $D(X_p/\widehat{\Sigma}_2)$ is a double cover of $\widehat{\Sigma}_2$ branched along $f(O)$ and the 
% proper transform of $\mcQ$, i.e., $D(X_p/\widehat{\Sigma}_2) = S_{\mcQ, z_o}$ and $\beta_1(\pi_p) = f$. 
% Let $\beta_1(\pi_p) : X_p \to S_{\mcQ, z_o}$ be the $p$-cyclic cover determined by $\pi_p$. By \cite[Remark 3.1, (i)]{tokunaga14},
% any irreducible component of singular fibers can not be contained in the branch locus of $\beta_2(\pi_p)$. Hence
% $\overline{q}\circ f(\Delta_{\beta_2(\pi_p)}) = \sum_{i=1}^3 \mcL_i$ (resp. $= \sum_{i=2}^4\mcL_i$.
% \qed
 
 \bigskip
 
 
{\bf Proof for Combinatorics 1-(b)}  $\mcE$: a nodal cubic. Choose $z_o \in \mcE\setminus \{p_1, p_2, p_3\}$ and let $S_{\mcQ, z_o}$ be
the rational elliptic surface as in the Introduction. 
%$\varphi_{\mcQ, z_o} : S_{\mcQ, z_o} \to \PP^1$ has $5$ reducible singular fibers. All of them are of type either $\I_2$ or $\III$ type. Hence $T_{\varphi_{\mcQ, z_o}} = A_1^{\oplus 5}$ and $G_{\Sing(\varphi_{\mcQ, z_o})} = (\ZZ/2\ZZ)^{\oplus 5}$. Since$\mcL_o$ gives rise to a $2$-torsion section,  by \cite{oguiso-shioda}, 
By Section 1, $E_{\mcQ, z_o}(\CC(t)) \cong
(A_1^*)^{\oplus 3}\oplus \ZZ/2\ZZ$. We choose generators $P_1, P_2, P_3$ of the  $(A_1^*)^{\oplus 3}$-part
such that
\[
[\langle P_i, P_j\rangle]=\left[ \begin{array}{ccc} \frac{1}{2}  & 0 & 0 \\ 0 & \frac{1}{2}  & 0 \\ 0 & 0 &\frac{1}{2} \end{array}\right]
\]
We denote the 2-torsion point by $P_{\tau}$. Let $s_{P_i}$ be the corresponding section for each $P_i$ ($i=1,2,3,\tau$)

\begin{enumerate}
\item[(i)] By \cite[Theorem 10.8]{shioda90}, we may assume that $s_{P_i}O=0$, $(i=0,1, 2,\tau)$ and
\[
\sum_{v \in  \Red(\varphi)}\mathrm{Contr}_v(s_{P_i}),  
  \sum_{v \in  \Red(\varphi)}\mathrm {Contr}_v(s_{P_i}, s_{P_j}) = 0, 1/2, 1, 3/2, \mbox{\rm or}\,\,  2, 5/2.
\]
\item[(ii)] For $P_i$ ($i=1,2,3$), $\langle P_i, P_i\rangle= 1/2$ implies that $\sum_v\mathrm{Contr}_v(s_{P_i})=3/2$. Also for $\left\{i,j\right\}\subset\{1,2,3\}$, $i\not=j$, $\langle P_i, P_j\rangle=0$ implies $s_{P_i}s_{P_j}=0$, $\sum_v\mathrm{Contr}_v(s_{P_i}, s_{P_j})=1$
\item[(iii)] For $P_\tau$, as $P_\tau$ is a torsion section, $ \langle P_\tau, P_\tau\rangle=0$ and we have $\sum_v\mathrm{Contr}_v(s_{P_i})=2$.
\item[(iv)] Furthermore, $\langle P_i, P_\tau\rangle=0$ implies  $\sum_v\mathrm{Contr}_v(s_{P_i}, s_{P_\tau})=1$.
\end{enumerate}

From the above facts, we can assume that
\[
\begin{array}{ccl}
\gamma_{\mcQ, z_o} (P_{\tau})  =  (1, 1, 1, 1 , 0) & &
\gamma_{\mcQ, z_o} (P_1)  =  (1, 1, 0, 0, 1) \\
\gamma_{\mcQ, z_o} (P_2)  =  (1, 0, 1, 0, 1) & &
\gamma_{\mcQ, z_o} (P_3)  =  (1, 0, 0,1, 1)  \,\, \mbox{\rm or} \,\,(0, 1, 1, 0, 1)
\end{array}
\]
%\begin{eqnarray*}
%\gamma_{\mcQ, z_o} (P_{\tau}) & = & (1, 1, 1, 1 , 0) \\
%\gamma_{\mcQ, z_o} (P_1) & = & (1, 1, 0, 0, 1) \\
%\gamma_{\mcQ, z_o} (P_2) & = & (1, 0, 1, 0, 1) \\
%\gamma_{\mcQ, z_o} (P_3) & = & (1, 0, 0,1, 1)  \,\, \mbox{\rm or} \,\,(0, 1, 1, 0, 1)
%\end{eqnarray*}
By replacing $P_3$  by $P_3\dot+P_\tau$  if necessary, we may assume that $\gamma_{\mcQ, z_o} (P_3)  =  (1, 0, 0,1, 1).$ We now label the irreducible components of the singular fibers as in the following figure. (cf.
\cite[No. 24, p. 90]{tokunaga12}. Note that we use different labelings.). 

%\begin{center}
%\input No24b.tex
%
%Figure 2
%\end{center}


%We now label the irreducible components of the singular fibers as in Figure 21 in \cite[No. 24, p. 90]{tokunaga12}. The sections $s_i$ ($i=1,2,3$) in Figure 21 correspond to $s_{P_i}$ $(i=1,2,3)$ respectively. 



Now consider $P_{i,j}^\pm=P_i\dot\pm P_j\dot+P_\tau$ ($\{i,j\}\subset\{1,2,3\}$, $i\not=j$). Then, since $\gamma_{\mcQ, z_o}$ is a group homomorphism, $s_{P_{i,j}^\pm}\Theta_{1,1}=s_{P_{i,j}^\pm}\Theta_{k,1}=1$, $k\not=i,j$. Also, since $\sum_v{\rm Contr}(s_{P_{i,j}^\pm})=1$, and $\langle P_{i,j}^+, P_{i,j}^+\rangle=\langle P_{i,j}^-, P_{i,j}^-\rangle=1$, we have $s_{P_{i,j}^\pm}O=0$.

Let $f: S_{\mcQ, z_o} \to \widehat{\Sigma}_2$ be the one in the
double cover diagram.
We now blow down  smooth rational curves $f(\Theta_{0, 0})$, $f(O)$, $f(\Theta_{1, 1})$,
$f(\Theta_{2, 1})$, $f(\Theta_{3, 1})$, $f(\Theta_{4, 1})$  in this order. The resulting surface is $\PP^2$ and this morphism coincides with $\overline q$ as in the previous case.

\begin{center}
\input No24b.tex

Figure 2
\end{center}
%Let $f: S_{\mcQ, z_o} \to \widehat{\Sigma}_2$ be the one in the
%double cover diagram.
%We now blow down  smooth rational curves $f(\Theta_{0, 0})$, $f(O)$, $f(\Theta_{1, 1})$,
%$f(\Theta_{2, 1})$, $f(\Theta_{3, 1})$, $f(\Theta_{4, 1})$  in this order. The resulting surface is $\PP^2$ and this morphism coincides with $\overline q$ as in the previous case.
%Note that $f_{\mcQ, z_o} = \overline{q}\circ f$ and $\overline {q}\circ f(O\cup \Theta_{0,0}) = z_o$. By combining the double cover diagram, we have the diagram below:
%\[
%\begin{CD}
%S' @<{\mu}<< S_{\mcQ, z_o} \\
%@V{f'}VV                 @VV{f}V \\
%\Sigma_2@<<{q}< \widehat{\Sigma}_2 @>{\overline{q}}>>  \PP^2. 
%\end{CD}
%\]
%As $f_{\mcQ, z_o} = \overline{q}\circ f$, we have


\begin{lem}\label{lem:image-2}{
\begin{enumerate}
 \item[(i)]  The image of the fixed locus of $[-1]_{\varphi}$ is a nodal cubic $\mcE$ and a transversal line $\mcL_o$ to $\mcE$. Moreover, $\overline {q}\circ f(s_{P_{\tau}}) = \mcL_o$ and the node is the image of $\Theta_{4,1}$.
 
 \item[(ii)]  $\{f_{\mcQ, z_o}(\Theta_{1,1}), f_{\mcQ, z_o}(\Theta_{2,1}), f_{\mcQ, z_o}(\Theta_{3,1})\} =
 \mcE\cap \mcL_o$. We denote $p_i = f_{\mcQ, z_o}(\Theta_{i,1})$
 
\item[(iii)]  For $\{i,j,k\}=\{1,2,3\}$, $f_{\mcQ, z_o}(s_{P_{i,j}^\pm})$ ($i\not=j$) passes through $p_k$, $k\not=i,j$ ($i = 1, 2, 3$), respectively.
 
 
 \item[(iv)] Put $\mcL_k^\pm := f_{\mcQ, z_o}(s_{P_{i,j}^\pm})$.  Then   either
 (a) $\mcL_k^\pm$ is tangent to $\mcE$ at a point distinct to $p_k$  or 
 $p_k$  is an inflection point of $\mcE$ and $\mcL_k^\pm$ is
 an inflectional tangent line.
 
 \end{enumerate}
 }
 \end{lem}

\proof We prove the part about the node in statement (i). Since $s_{p_\tau}\Theta_{4,0}=1$, the remaining component(s) of the ramification locus must intersect $\Theta_{4,1}$ at two distinct points. Hence the image of $f(\Theta_{4,1})$ gives rise to a node on  the branch locus that is not on $\mcL_o$. The remaining statements can be proved in a similar way as that in Lemma \ref{lem:image-1}.  
\qed


Finally, as in the Introduction, we put 
\begin{eqnarray*}
P_4:=P_{1,3}^+=P_1\dot+P_3\dot+P_\tau \quad\quad
P_5:=P_{1,2}^+=P_1\dot+P_2\dot+P_\tau\\
P_6:=P_{2,3}^+=P_2\dot+P_3\dot+P_\tau \quad\quad
P_7:=P_{3,1}^-:=P_3\dot-P_1\dot+P_\tau
\end{eqnarray*}

%\begin{eqnarray*}
%P_4:=P_{1,3}^+=P_1\dot+P_3\dot+P_\tau\\
%P_5:=P_{1,2}^+=P_1\dot+P_3\dot+P_\tau\\
%P_6:=P_{2,3}^+=P_2\dot+P_3\dot+P_\tau\\
%P_7:=P_{3,1}^-:=P_3\dot-P_1\dot+P_\tau
%\end{eqnarray*}

Now let
\[
\mcB^1=\mcE +\mcL_0+\sum_{i=4}^6 \mcL_i, \quad
\mcB^2=\mcE +\mcL_0+\sum_{i=5}^7 \mcL_i.
\]
Then by  \cite[Theorem 3.2]{tokunaga14} and  Lemma~\ref{lem:key-comb1}, a $D_{2p}$-cover branched at $2(\mcE+\mcL_0)+p(\sum_{i=5}^7 \mcL_i)$ exists, but does not exist for $2(\mcE +\mcL_0)+p(\sum_{i=4}^6 \mcL_i)$. Hence, $(\mcB^1, \mcB^2)$  is a Zariski pair
if their combinatorics are the same.
 \bigskip 
%\begin{lem}\label{lem:image-2}
%
%
%\end{lem}
%
%
\bigskip


{\bf Proof for Combinatorics 2}
We first note that our proof is similar to that of \cite[Theorem 5 (ii)]{tokunaga14}. Let $\varphi_{\mcQ, z_o} : S_{\mcQ, z_o} \to \PP^1$
 be the rational elliptic surface corresponding to Combinatorics 1-(b).  Under the same nation as before, Consider $[2]P_i$ ($i = 1, 2,3$)
 and their corresponding sections $s_{[2]P_i}$ $(i = 1, 2, 3)$, respectively. Since $\langle [2]P_i, [2]P_i \rangle = 2$ and
 $\gamma_{\mcQ, z_o}(s_{[2]P_i}) = (0, 0, 0, 0, 0)$,  we infer that $f_{\mcQ, z_o}(s_{[2]P_i})$ $(i = 1, 2, 3)$ are contact conics $\mcC_i$ ($i = 1, 2, 3)$  to
 $\mcQ$ and $z_o$ is one of the tangent points between $\mcC_i$ and $\mcQ$ by the argument in the proof of \cite[Theorem 5 (ii), p. 633]
 {tokunaga14}.  Now, for example, let  $\mcL = f_{\mcQ, z_o}(s_{P_1})$ and consider two curves
 \[
 \mcB^1 := \mcQ + \mcL + \mcC_1, \quad \mcB^2 := \mcQ + \mcL + \mcC_2.
 \]
 
 If there  exists a homeomorphism $h: (\PP^2, \mcB^1) \to (\PP^2, \mcB^2)$, it satisfies $h(\mcQ) = \mcQ$. Hence
 by \cite[Proposition~4.4]{tokunaga14}, there exist no homeomorphisms $(\PP^2, \mcB^1) \to (\PP^2, \mcB^2)$.
 Moreover, if both $\mcB^1$ and $\mcB^2$ have the Combinatorics 2, then $(\mcB^1, \mcB^2)$ is a Zariski pair. In \S 5, we show
 that such an example exists. 
 








      


                                             