% !TEX root = bottom.tex

%%%%%%%%%%%%%%%%%%%%%%%%%%%%%%%%%%%%%%%%%%%%%%%%%%%%%%
\section{Potential description of heavy-quarkonium}
\label{sec:pot}

Herein, we use a lattice QCD vetted, non-relativistic potential based description of heavy-quarkonium in a thermal medium to compute its real-time evolution in a heavy-ion collision. The potential originates in a systematic treatment of heavy quarkonium in QCD, based on the effective field theories non-relativistic QCD (NRQCD) and potential NRQCD \cite{Brambilla:2004jw,Brambilla:2008cx}. These frameworks exploit the inherent separation of scales between the heavy quark rest mass $m_{\rm c,b}$, the surrounding medium temperature, as well as the characteristic scale of QCD $\Lambda_{\rm QCD}$ to dramatically simplify the description of heavy quarkonium. Instead of having to consider a full quantum field-theoretical boundary value problem for Dirac spinor fields, one may go over in a first step to an initial-value problem for two-component Pauli spinors (NRQCD). In a second step this non-relativistic theory may be matched to a further simplified description in terms of coupled color singlet $ \psi_S({\bf r},t)$ and color octet wavefunctions $ \psi_O({\bf r},t)$ (pNRQCD). In the latter, the interaction among the heavy-quarks, as well as their interaction with the medium is captured in both potential and non-potential contributions. Depending on the concrete scale hierarchy of the problem at hand, the potential contributions may dominate and the relevant real-time evolution of heavy quarkonium reduces to a Schr\"odinger equation. 

For realistic settings, such as those encountered in heavy-ion collisions at RHIC and LHC, the matching coefficients of the effective theory, i.e. the potential cannot be reliably determined using perturbation theory. Nevertheless vital insight had been gained by evaluating pNRQCD using the hard-loop approximation \cite{Laine:2006ns,Brambilla:2008cx}. In particular it was found that  the proper in-medium potential must assume complex values at high temperatures, as was first discussed in \cite{Laine:2006ns} and subsequently extended to a momentum-space anisotropic medium in \cite{Dumitru:2007hy,Burnier:2009yu,Dumitru:2009fy,Nopoush:2017zbu}. This fact, in particular, implies that purely real model potentials, such as the popular color singlet free energies or the internal energies do not constitute valid descriptions of the relevant in-medium quarkonium physics. 

The proper heavy-quark potential is related to a real-time QCD quantity, the rectangular Wilson loop, via the process of matching,~i.e.~one selects a correlation function in the effective theory pNRQCD and in the underlying microscopic theory QCD which carry the same physics content and identifies them at an appropriate scale. In our case the unequal time correlation function of a heavy quarkonium singlet state may be identified with the Wilson loop in the static limit
\begin{eqnarray}
\nonumber \Big\langle \psi_S({\bf r},t)\psi_S({\bf r},0) \Big\rangle_{\rm pNRQCD} \overset{m\to\infty}{\equiv}  W_\square({\bf r},t)=\Big\langle {\rm Tr} \Big( {\rm exp}\Big[-ig\int_\square dx^\mu A_\mu^aT^a\Big] \Big) \Big\rangle_{\rm QCD}.
\label{Eq:ForwProp}
\end{eqnarray}
Since the Wilson loop obeys a simple equation of motion \cite{Laine:2006ns}
\begin{align}
i\partial_tW_\square(r,t)=\Phi(r,t)W_\square(r,t),
\end{align}
with a time- and space-dependent complex function $\Phi(r,t)$,  the potential picture is applicable as long as $\Phi$ asymptotes towards a time independent value at late times. This value in general is complex and the corresponding potential may be formally defined as 
\begin{align}
V_{\rm QCD}(r)=\lim_{t\to\infty} \frac{i\partial_t W(t,r)}{W(t,r)}\label{Eq:VRealTimeDef}.
\end{align}

As such, this real-time definition is not yet amenable to an evaluation in non-perturbative lattice QCD, which is simulated in unphysical Euclidean time. Instead, one has to take a detour via the spectral decomposition of the Wilson loop to relate the Euclidean and Minkowski domain \cite{Rothkopf:2009pk,Rothkopf:2011db}
\begin{align}
 W_\square(\tau,r)=\int d\omega e^{-\omega \tau} \rho_\square(\omega,r)\,
\leftrightarrow\, \int d\omega e^{-i\omega t} \rho_\square(\omega,r)= W_\square(t,r). \label{Eq:SpecDec}
\end{align}
Inserting \eqref{Eq:SpecDec} into \eqref{Eq:VRealTimeDef} tells us that the real and imaginary part of the potential are related to the position and width of the lowest lying peak structure within the Wilson loop spectrum. It has been shown \cite{Burnier:2012az} that if a potential picture is applicable, the Wilson loop spectrum actually contains a well defined lowest lying peak of skewed Lorentzian form from which the values of the potential can be straightforwardly extracted via a $\chi^2$-fit. 

Note however that the extraction of spectral functions from Euclidean lattice data is an ill-posed inverse problem, which has only recently been successfully tackled in the context of the heavy-quark potential. With the help of a novel Bayesian approach \cite{Burnier:2013nla} the reconstruction robustness was significantly improved compared to previous attempts based on the Maximum Entropy Method \cite{Rothkopf:2011db}. In practice, instead of the Wilson loop on the lattice, one considers Wilson line correlators fixed to Coulomb gauge, which are free from a class of divergences hampering the numerical determination of the Wilson loop.  Using this prescription, the values of the potential have been extracted, to date, in quenched QCD based on the naive Wilson action \cite{Burnier:2016mxc}, as well as for full QCD with $N_f=2+1$ light medium quark flavors based on simulations by the HotQCD collaboration. In both cases, the applicability of the potential picture was confirmed, at all temperatures considered, as the Wilson spectral functions showed a well defined peak of Lorentzian shape. 

To utilize the discrete values of the potential obtained from lattice QCD two further steps need to be taken, for which we follow the same strategy as laid out in \cite{Burnier:2015tda,Burnier:2016kqm}. The first task is to parametrize the values of the potential with an analytic formula, which allows the evaluation of Re[V] and Im[V] at intermediate separation distances not explicitly resolved on the lattice. The second task is to correct the parameters of this analytic parametrization for finite volume and finite lattice spacing artifacts, as no fully continuum extrapolated lattice QCD determination of the potential has been achieved so far.

The analytic parametrization we deploy in the following is based on the concept of a generalized Gauss law for the vacuum heavy quark potential. Lattice QCD studies have shown that over the phenomenologically relevant range of distances for quarkonium the $T=0$ potential is very well reproduced by the Cornell ansatz, consisting of a Coulombic and linearly rising term. Allowing the linear term to contribute down to the smallest distances mimics the effects of a running in the coupling. With the Cornell potential applicable in vacuum, we can consider the divergence of  the auxiliary (color) electric field $\vE=q r^{a-1} \rh$ arising from either the the Coulombic $a=-1, q=\alpha_s, [\alpha_s]=1$ or the string-like part $a=1, q=\sigma, [\sigma]={\rm GeV}^2$
\begin{align}
\vec\nabla \left(\frac{\vE}{r^{a+1}}\right)=   4\pi \,q\, \delta(\vr) \label{Eq:GenGauss}.
\end{align}
Three parameters enter this expression, which characterize the non-perturbative vacuum physics of the quarkonium bound state: the strong coupling $\alpha_s$, the string tension $\sigma$, and a constant shift $c$. Note that we absorb the factor $C_F$ into our definition of the strong coupling $\alpha_s=\frac{g^2C_F}{4\pi}$.

In order to introduce the effects of a thermal medium, we adopt a prescription well known in classical electrodynamics,~i.e.~in the case of the Coulombic contribution, one Fourier transforms Gauss' law and subsequently modifies the right hand side by dividing it with an in-medium permittivity $\epsilon$. We use here the permittivity of the QCD medium computed in hard-thermal loop perturbation theory  
\begin{align}
\eps^{-1}(\vp,m_D)=\frac{p^2}{p^2+m_D^2}-i\pi T \frac{p\, m_D^2}{(p^2+m_D^2)^2}.\label{Eq:HTLperm}
\end{align}
The idea is that the non-perturbative physics of the bound state is encoded in the Cornell form of the $T=0$ potential, whose modification is driven by a weakly-coupled gas of quarks and gluons. Combining \eqref{Eq:HTLperm} and \eqref{Eq:GenGauss} thus leads to integro-differential equations for the in-medium modified Coulombic and string-part of the $T=0$ potential as discussed in detail in \cite{Burnier:2015nsa}. Since the permittivity is complex, the in-medium potential also contains an imaginary part. In contrast to purely perturbative computations, which capture only the Coulombic contribution to the potential, the in-medium potential here receives additional contributions to its real and imaginary part from the string-like part of the $T=0$ Cornell potential. The explicit expressions for the Coulombic part are
\begin{align}
V_c(r)= -\alpha_s\left[\mD+\frac{e^{-\mD r}}{r}
+iT\phi(\mD r)\right] \label{Eq:VHTL},
\end{align}
with 
\begin{align}
\phi(x)=2 \int_0^\infty dz \frac{z}{(z^2+1)^2}\left(1-\frac{\sin(xz)}{xz}\right),\label{phi}
\end{align}
which coincide with the results of Ref.~\cite{Laine:2006ns}.
The additional and novel string-like contribution on the other hand reads
\begin{align}
{\rm Re}V_s(r)&=-\frac{\Gamma[\frac{1}{4}]}{2^{\frac{3}{4}}\sqrt{\pi}}\frac{\sigma}{\mu} D_{-\frac{1}{2}}\big(\sqrt{2}\mu r\big)+ \frac{\Gamma[\frac{1}{4}]}{2\Gamma[\frac{3}{4}]} \frac{\sigma}{\mu} \, ,
\end{align}
for the real part, where the strength of the in-medium modification is characterized by the parameter $\mu^4=m_D^2 \sigma/\alpha_s$. For its imaginary part we have 
\begin{eqnarray}
\Im V_s(r)&=&-i\frac{\sigma m_D^2 T}{\mu^4} \psi(\mu r)=-i\alpha_s T \psi(\mu r)\label{Eq:ImVSGenGauss},
\end{eqnarray}
where $\psi$ corresponds to the following Wronskian
\begin{eqnarray} 
 \notag\psi(x)&=&D_{-1/2}(\sqrt{2}x)\int_0^x dy\, {\rm Re}D_{-1/2}(i\sqrt{2}y)y^2 \phi(ym_D/\mu)\\\notag&&\hspace{-4mm}+{\rm Re}D_{-1/2}(i\sqrt{2}x)\int_x^\infty dy\, D_{-1/2}(\sqrt{2}y)y^2 \phi(ym_D/\mu)\\ \nonumber&&-D_{-1/2}(0)\int_0^\infty dy\, D_{-1/2}(\sqrt{2}y)y^2 \phi(ym_D/\mu).
\end{eqnarray}
An important aspect of these expressions is that, once the vacuum parameters of the Cornell potential are fixed, only a single temperature-dependent parameter remains, the Debye mass $m_D$.

While the analytic parametrization was derived in a straightforward fashion, it relies on several assumptions and thus needs to be validated on real lattice QCD data. Using both quenched \cite{Burnier:2016mxc} and full QCD simulations with $N_f=2+1$ light flavors \cite{Burnier:2015tda} it was shown that the lattice values of the potential were indeed excellently reproduced by the generalized Gauss law parametrization (see Fig.\ref{Fig:PotentialLat}). After fixing $\alpha_s$, $\sigma$, and $c$ using low temperature ensembles, the real-part of the potential was fitted by tuning $m_D$. Once $m_D$ is fixed, the parametrization makes a prediction for $\Im[V]$, which for quenched QCD simulations showed quantitative agreement at high temperatures and, as expected, became less accurate around the phase transition. In full QCD, no robust determination of the imaginary part has been achieved so far, however, the tentative values extracted, again, showed very good agreement with the Gauss-law parametrization at high and intermediate temperatures \cite{Burnier:2015tda}. The values for the Debye mass related to the full QCD in-medium potential also showed clear deviations from the perturbative predictions in the phenomenologically relevant regime between $T_C<T<3T_C$.

\begin{figure*}
\includegraphics[scale=0.4]{QQbarReVFullQCDRaw.pdf}
\includegraphics[scale=0.4]{QQbarImVFullQCDRaw.pdf}
\caption{The real (left) and imaginary (right) part of the in-medium heavy quark potential in full QCD with $N_f=2+1$ light quark flavors based on ensembles by the HotQCD collaboration (colored points, shifted for better readability). The vacuum parameters here are tuned using the $T\approx0$ ensembles at $\beta=6.9$ and $\beta=7.48$. By adjusting the Debye mass parameter $m_D$, the lattice QCD values of $\Re[V]$ are reproduced very well via the Gauss-law parametrization (solid lines) over all separation distances and temperatures. The theoretical error bars (shown as shaded regions surrounding the central line) arise from the fit uncertainty of $m_D$. For $\Im[V]$, the agreement at high temperatures and small distances is very good, while at $T\approx T_C$ deviations from the extracted lattice values are visible. (The crossover temperature on these lattices due to the relatively large pion mass of $m_\pi\approx300$MeV lies at $T_C=172.5$ MeV) }\label{Fig:PotentialLat}
\end{figure*}

Even though it has been established that the Gauss-law parametrization is capable of reproducing the lattice QCD in-medium potential, its values found on the lattice may not be applied directly to phenomenological computations due to the presence of lattice artifacts. If a continuum extrapolation of the $T=0$ potential in the thermodynamic limit were available, we could directly determine the parameters $\alpha_s$, $\sigma$ and $c$ from first principles. Here instead we select these parameters in a phenomenological fashion, i.e. we tune them such that the vacuum bottomonium spectrum below the B-meson threshold is reproduced. The correct quark mass to be used in such a computation is the renormalon subtracted mass, which for bottomonium may be perturbatively computed and takes the value $m^{\rm RS'}_b=4.882\pm0.041$GeV. Since the vacuum potential in full QCD was robustly determined only up to distances \mbox{$r\approx1$ fm} we enforce the flat asymptotics due to string breaking by hand at $r_{\rm SB}=1.25$ fm. Within this setting the best set of parameters is
\begin{align}
c=-0.1767\pm 0.0210~{\rm GeV}, \quad \alpha_s=0.5043\pm 0.0298,\quad \sqrt{\sigma}=0.415\pm0.015~{\rm GeV}.\label{T0const}
\end{align}

In this study we will compute the Debye mass self-consistently from the dynamical evolution of the bulk and use its value to implement the in-medium modification of the Cornell potential with the above parameters. In Fig.~\ref{Fig:Potential} we show the real (left) and imaginary part (right) of the actual potential used for different values of the Debye mass. In order understand better which values of $m_D$ play a role in the evolution of heavy quarkonium we note that lattice QCD studies showed that in a thermal QCD medium close to the crossover transition the ratio of $m_D/T\approx1$ and grows to $m_D/T\approx2$ as temperature is increased to $T=2T_C$.
\begin{figure*}
\includegraphics[scale=0.4]{ReV.pdf}
\includegraphics[scale=0.4]{ImV.pdf}
\caption{The real (left) and imaginary (right) part of the in-medium heavy quark potential used in this study. Their values are given for different values of the Debye mass of the QCD medium. The vacuum parameters $\alpha_S$, $\sigma$, and $c$ at $m_D=0$ were tuned such that the PDG bottomonium spectrum is reproduced. To this end string breaking is enforced at $r_{\rm sb}=1.25$ fm. The in-medium modification of the Cornell-type vacuum potential in our approach is governed by a single temperature-dependent parameter, the Debye mass $m_D$. Thermal effects lead to a characteristic (Debye) screening of the real-part and induce a finite imaginary part, which saturates at large distances.}\label{Fig:Potential}
\end{figure*}


%%%%%%%%%%%%%%%%%%%%%%%%%%%%%%%%%%%%%%%%%%%%%%%%%%%%%%