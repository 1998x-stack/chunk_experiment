

%\documentclass[journal]{IEEEtran}

\documentclass[letterpaper, 10 pt, conference]{ieeeconf}  % Comment this line out if you need a4paper

%\documentclass[a4paper, 10pt, conference]{ieeeconf}      % Use this line for a4 paper

\IEEEoverridecommandlockouts                              % This command is only needed if 
                                                          % you want to use the \thanks command

\overrideIEEEmargins                                      % Needed to meet printer requirements.




% Some very useful LaTeX packages include:
% (uncomment the ones you want to load)


% *** GRAPHICS RELATED PACKAGES ***
\usepackage{graphics} % for pdf, bitmapped graphics files
%\usepackage{epsfig} % for postscript graphics files
%\usepackage{mathptmx} % assumes new font selection scheme installed
%\usepackage{times} % assumes new font selection scheme installed
%\usepackage{amsmath} % assumes amsmath package installed
%\usepackage{amssymb}  % assumes amsmath package installed





% *** MATH PACKAGES ***
%
\usepackage{amsmath}

\usepackage{pgfplots}
\pgfplotsset{compat=1.5}


\usepackage{multirow}


% *** PDF, URL AND HYPERLINK PACKAGES ***
%
\usepackage{url}
% url.sty was written by Donald Arseneau. It provides better support for
% handling and breaking URLs. url.sty is already installed on most LaTeX
% systems. The latest version and documentation can be obtained at:
% http://www.ctan.org/pkg/url
% Basically, \url{my_url_here}.

\usepackage{hyperref}

\usepackage[noend]{algpseudocode}
\usepackage[ruled,noend]{algorithm2e}


\usepackage{multirow}
\usepackage{array}
\newcolumntype{P}[1]{>{\centering\arraybackslash}p{#1}}
\newcolumntype{M}[1]{>{\centering\arraybackslash}m{#1}}
\newcolumntype{L}[1]{>{\raggedleft\arraybackslash}m{#1}}
\newcolumntype{R}[1]{>{\raggedright\arraybackslash}m{#1}}

\usepackage{amsmath}
\usepackage{color}
\newcommand{\red}[1]{\textcolor{red}{#1}}
\newcommand{\todo}[1]{\red{TODO: #1}}


% correct bad hyphenation here
%\hyphenation{op-tical net-works semi-conduc-tor}


%
% paper title
\title{\LARGE \bf
Planning High-Quality Grasps using \\Mean Curvature Object Skeletons
}
%\title{Part-based Grasp Planning using \\Mean Curvature Object Skeletons}
		


\author{
Nikolaus Vahrenkamp, Eduard Koch, Mirko W\"achter and Tamim Asfour%
\thanks{The authors are with the High Performance Humanoid Technologies (H\textsuperscript{2}T) lab, Institute for Anthropomatics and Robotics (IAR), Karlsruhe Institute of Technology (KIT), Germany}
\thanks{The research leading to these results has received funding from the European Union’s Horizon 2020 Research and Innovation programme under grant agreement No 643950 (SecondHands).}
}


\begin{document}



\maketitle
\thispagestyle{empty}
\pagestyle{empty}
		
% As a general rule, do not put math, special symbols or citations
% in the abstract or keywords.
\begin{abstract}
In this work, we present a grasp planner which integrates two sources of information to generate robust grasps for a robotic hand. First, the topological information of the object model is incorporated by building the mean curvature skeleton and segmenting the object accordingly in order to identify object regions which are suitable for applying a grasp.
Second, the local surface structure is investigated to construct feasible and robust grasping poses by aligning the hand according to the local object shape. 
We show how this information can be used to derive different grasping strategies, which also allows to distinguish between precision and power grasps. We applied the approach to a wide variety of object models of the KIT and the YCB real-world object model databases and evaluated the approach with several robotic hands. The results show that the skeleton-based grasp planner is capable to autonomously generate high-quality grasps in an efficient manner. 
In addition, we evaluate how robust the planned grasps are against hand positioning errors as they occur in real-world applications due to perception and actuation inaccuracies. 
The evaluation shows that the majority of the generated grasps are of high quality since they can be successfully applied even when the hand is not exactly positioned.


%Due to the consideration of the local object shape, the resulting grasping poses tend to be more natural compared to the results of planners which just consider the surface normal for aligning the hand. In addition, the evaluation shows that we are able to generate high-quality grasping hypotheses and that the quality in terms of robustness outperforms other approaches. 
%We applied the approach to a wide variety of object models of the KIT and the YCB real-world object model databases and evaluated the approach with several robotic hands.
    \end{abstract}
    
    % Note that keywords are not normally used for peerreview papers.
    %\begin{IEEEkeywords}
    %   grasp planning
    %\end{IEEEkeywords}
    
    
    
    
    
    
    % For peer review papers, you can put extra information on the cover
    % page as needed:
    % \ifCLASSOPTIONpeerreview
    % \begin{center} \bfseries EDICS Category: 3-BBND \end{center}
    % \fi
    %
    % For peerreview papers, this IEEEtran command inserts a page break and
    % creates the second title. It will be ignored for other modes.
    \IEEEpeerreviewmaketitle
    
    
    \input{section_introduction}
    \input{section_related-work}
    \input{section_skeleton}
    \input{section_approach}
    \input{section_evaluation}
    \input{section_conclusion}
    
   

    \vspace{-0.2cm}
		
    % references section
    
    % can use a bibliography generated by BibTeX as a .bbl file
    % BibTeX documentation can be easily obtained at:
    % http://mirror.ctan.org/biblio/bibtex/contrib/doc/
    % The IEEEtran BibTeX style support page is at:
    % http://www.michaelshell.org/tex/ieeetran/bibtex/
    \bibliographystyle{IEEEtran}
    % argument is your BibTeX string definitions and bibliography database(s)
    %\bibliography{IEEEabrv,../bib/paper}

     \bibliography{literatur}
    
    % biography section
    % 
    % If you have an EPS/PDF photo (graphicx package needed) extra braces are
    % needed around the contents of the optional argument to biography to prevent
    % the LaTeX parser from getting confused when it sees the complicated
    % \includegraphics command within an optional argument. (You could create
    % your own custom macro containing the \includegraphics command to make things
    % simpler here.)
    %\begin{IEEEbiography}[{\includegraphics[width=1in,height=1.25in,clip,keepaspectratio]{mshell}}]{Michael Shell}
    % or if you just want to reserve a space for a photo:
    
    
	% \input{author-bio}
    
    % that's all folks
\end{document}
