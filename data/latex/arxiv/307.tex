\begin{figure}
\centering
  \includegraphics[width=.5\columnwidth]{figure/LG}
  \vspace{-1em}
  \caption{An example of a \localgraph{}.  The vertices in the shaded area are in one cluster.  The \localgraph{} contains the vertices in the shaded area and the \outvers{} shown in plaques.  Solid lines indicate the edges that are in the clusters and thick grey lines represent cluster tree edges connecting other clusters (which are shown in yellow pentagons).  The three neighbor clusters sharing the same cluster label are connected using two edges (dash curves).  Edges $e_1$ and $e_2$ are the edges that only has one endpoints in the cluster.  The other endpoint is set to be the \outver{} connecting the cluster of the other original endpoint of this edge in the cluster spanning tree.  Consequently $e_1'$ and $e_2'$ are the replaced edges for $e_1$ and $e_2$.
  }\label{fig:localgraph}
\end{figure}

