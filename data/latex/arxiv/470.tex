\section{System Design}
\label{sec:methodology} 

In this section, we describe the design, architecture, and implementation of the
OutWithFriendz system.  We also present \replaced{a walk-through example to 
illustrate the user workflow of our app for group event scheduling }{the typical user workflow through a
series of use cases}.

\subsection{System Architecture}
In order to understand group user behavior at scale, we designed versions of the
OutWithFriendz mobile client for both the iPhone and Android platforms, allowing
us to collect data on the scale of hundreds of users and user-generated
invitations. This mobile system provides opportunity for us to track group
event organization behavior on a regular, ongoing basis. We support the two most
popular mobile platforms, iPhone and Android, allowing us to accommodate groups composed of
users of both platforms.
Our system consists of three major components: the mobile client, the data
collection server, and the Google Cloud Messaging (GCM)
server~\cite{developers2014google}. We also call the Google Maps API to retrieve
location search results.  Figure~\ref{fig:architecture} shows the overall
architecture of our OutWithFriendz system.

\begin{figure}
\centering
\includegraphics[width=0.9\linewidth]{architecture}
\caption{The architecture of the OutWithFriendz system.}
\label{fig:architecture}
\end{figure}

The implementation of the OutWithFriendz system requires careful engineering to
handle the communication and data synchronization between clients. Our
OutWithFriendz server is implemented as a Java Web application using the Spring
application framework~\cite{Spring}. All required functionality to the client is
exposed through the server's REST APIs. MongoDB is also used to store and manage
all data on the server~\cite{mongodb}. To push notifications between server and
clients, GCM services are used to handle all aspects of queueing of messages and
delivery to client applications running on both mobile platforms. In addition,
information about each location is obtained from Google Map
Services~\cite{googlemaps}, including the name, street address, and
latitude/longitude coordinates.

\begin{figure}
\centering
\begin{subfigure}{.33\textwidth}
  \centering
  \includegraphics[width=0.9\linewidth]{screenshot_host}
  \caption{Create invitation.}
  \label{fig:screenshot_host}
\end{subfigure}%
\begin{subfigure}{.33\textwidth}
  \centering
  \includegraphics[width=0.9\linewidth]{screenshot_map}
  \caption{Add location through Google Map.}
  \label{fig:screenshot_map}
\end{subfigure}
\begin{subfigure}{.33\textwidth}
  \centering
  \includegraphics[width=0.9\linewidth]{screenshot_vote}
  \caption{\added{Suggest and} vote for preferences.}
  \label{fig:screenshot_vote}
\end{subfigure}
\caption{Main workflow of the OutWithFriendz mobile application.}
\end{figure}

\subsection{UI Design Challenges}
%User interface is a very important part in our OutWithFriendz system. 
In order to enable a natural group decision-making workflow, we continuously
streamlined the UI and workflow of the OutWithFriends app based on feedback
collected from user studies. We started with an initial usage survey
before releasing the app to the market. During our survey, we hired seven
students on campus who have different academic backgrounds. They formed three
groups to use our app and provided useful feedbacks for improving UI design.
For example, these users suggested: (1) Adding a chat board to allow group
members to discuss their opinions; (2) Allowing users to edit the location title
and provide detailed information for each location; (3) Allowing users to link
suggested locations with the Google Places application; (4) Pushing notifications
if an invitation is created or modified; and (5) Replacing text buttons with
interactive icon buttons. Implementing this functionality helped us improve our
app to better support the real-life group \replaced{event scheduling }{decision making} process.
We also added and altered application functionality to improve usability, based
on many user suggestions received during application usage.
At the beginning of the study, we focused on dining events only. Later we came
to realize that the users would also like to use the OutWithFriends app for
generic group gatherings, such as going for a hike or watching a movie. To
support this functionality, we shifted from integrating with Foursquare API to the more
suitable Google Places API. We also changed the workflow for the voting process
to make it more flexible. Initially, invitation participants were required to
decide on the meeting time before starting the voting process for the location.
However, our users preferred to perform time voting and location voting
concurrently, which is more flexible. Next, we describe the main workflow of our app.

\subsection{\replaced{A Walk-through Example }{Use Cases}}
\label{sec:flow}

\begin{figure}
\centering
\begin{subfigure}{.33\textwidth}
  \centering
  \includegraphics[width=0.9\linewidth]{screenshot_chat}
  \caption{Chat screen.}
  \label{fig:screenshot_chat}
\end{subfigure}%
\begin{subfigure}{.33\textwidth}
  \centering
  \includegraphics[width=0.9\linewidth]{screenshot_finalize}
  \caption{Invitation finalization screen.}
  \label{fig:screenshot_finalize}
\end{subfigure}
\begin{subfigure}{.33\textwidth}
  \centering
  \includegraphics[width=0.9\linewidth]{screenshot_list}
  \caption{Invitation list screen.}
  \label{fig:screenshot_list}
\end{subfigure}
\caption{Main functions of the OutWithFriendz mobile application.}
\end{figure}

\added{To better understand the workflow of our mobile application, 
we provide a walk-through example of how the main functions are 
used for group event scheduling.}

\subsubsection{A host invites two friends to meet for dinner}
In this use case, we describe the actions a user would take to invite some friends to meet for 
dinner. Here we call this user the host. When creating 
a new event invitation, the host will go to the window shown in
Figure~\ref{fig:screenshot_host} and perform the following steps:
(1) create a title for the invitation, \added{such as Friday Dinner}; 
(2) specify one or more possible dates and times for the 
invitation; (3) suggest meeting locations using Google Map Services, as
shown in Figure~\ref{fig:screenshot_map}; and (4) add one or more friends that
want to be included as participants in the invitation. Finally, when the host is
satisfied with the invitation settings, she taps the ``send invitation'' icon, to send the invitation to all selected participants.
\added{She can also start voting for her own preferences right after the new invitation shows up on her screen.}

\subsubsection{A user receives an invitation to meet several friends for dinner}
First, the user receives a notification from the OutWithFriendz application
indicating that she has received a new invitation. The user can express her 
preferences by voting on one or more possible options for meeting dates and
locations, as shown in Figure~\ref{fig:screenshot_vote}. \added{One important feature of our system is that} 
the user may also add
new proposed dates/time or locations to the invitation. Once the user has added a new
option, it will be automatically made visible to all other participants. \added{Users are also allowed to change their suggestions 
and votes throughout the voting process.}
In the ``Chat'' tab shown in Figure~\ref{fig:screenshot_chat}, \replaced{the }{user}
is also able to send text messages to other group members \added{for discussion and better coordination 
of the scheduling process}.

\subsubsection{Host finalizes the invitation based on voting results}
The voting process continues until the host decides to finalize the meeting time 
and location. Only the host is permitted to finalize, which is shown in
Figure~\ref{fig:screenshot_finalize}. After the host has finalized the
invitation, each participant receives a notification regarding this action. 
\added{To support unforeseen changes, the host could still update the final decision 
after it is finalized.}
Each user's main screen will show a list of invitations that she has 
participated in, as shown in Figure~\ref{fig:screenshot_list}.





