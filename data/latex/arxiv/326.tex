\documentclass[conference]{IEEEtran}
\IEEEoverridecommandlockouts
% The preceding line is only needed to identify funding in the first footnote. If that is unneeded, please comment it out.

\usepackage{subcaption}
\usepackage{multirow}
\usepackage{array}
\usepackage{caption}

\usepackage{cite}
\usepackage{amsmath,amssymb,amsfonts}
\usepackage{algorithmic}
\usepackage{graphicx}
\usepackage{textcomp}
\def\BibTeX{{\rm B\kern-.05em{\sc i\kern-.025em b}\kern-.08em
    T\kern-.1667em\lower.7ex\hbox{E}\kern-.125emX}}

\usepackage{soul}

%%%%%%%%%%%%%%%%%%%%%%%%%%%%%%%%%%%%%%%%%%%%%%%%
\ifodd 1
\usepackage{soul, xcolor}
\newcommand{\rev}[1]{{{\color{blue} #1}}}
\newcommand{\cmt}[1]{{{\color{red} \hl{(#1)}}}}
%\newcommand{\cmt}[1]{{{\color{red} \hl{}}}}
\newcommand{\red}[1]{{{\color{red} #1}}} %revise of the text
\newcommand{\revs}[1]{{{\color{blue} #1}}}
\fi
%%%%%%%%%%%%%%%%%%%%%%%%%%%%%%%%%%%%%%%%%%%%%%%%%


\hyphenation{time-stamps}

\begin{document}

\title{Cloaking the Clock: Emulating Clock Skew in Controller Area Networks
%{\footnotesize \textsuperscript{*}Note: Sub-titles are not captured in Xplore and should not be used}
%\thanks{Identify applicable funding agency here. If none, delete this.}
}

\author{Sang Uk Sagong$^{\ast}$, Xuhang Ying$^{\ast}$, Andrew Clark$^{\dagger}$, Linda Bushnell$^{\ast}$, and Radha Poovendran$^{\ast}$\\
$^{\ast}$ Department of Electrical Engineering, University of Washington, Seattle, WA 98195. \\
$^{\dagger}$ Department of Electrical and Computer Engineering, Worcester Polytechnic Institute, Worcester, MA 01609. \\
Email: \{sagong, xhying, lb2, rp3 \}@uw.edu, aclark@wpi.edu
}

%\author{\IEEEauthorblockN{Sang Sagong, Xuhang Ying, Andrew Clark, Radha Poovendran, Linda Poovendran}
%\IEEEauthorblockA{\textit{Department of Electrical Engineering, University of Washington} \\
%\textit{name of organization (of Aff.)}\\
%City, Country \\
%email address}
%\and
%\IEEEauthorblockN{2\textsuperscript{nd} Given Name Surname}
%\IEEEauthorblockA{\textit{dept. name of organization (of Aff.)} \\
%\textit{name of organization (of Aff.)}\\
%City, Country \\
%email address}
%\and
%\IEEEauthorblockN{3\textsuperscript{rd} Given Name Surname}
%\IEEEauthorblockA{\textit{dept. name of organization (of Aff.)} \\
%\textit{name of organization (of Aff.)}\\
%City, Country \\
%email address}
%\and
%\IEEEauthorblockN{4\textsuperscript{th} Given Name Surname}
%\IEEEauthorblockA{\textit{dept. name of organization (of Aff.)} \\
%\textit{name of organization (of Aff.)}\\
%City, Country \\
%email address}
%\and
%\IEEEauthorblockN{5\textsuperscript{th} Given Name Surname}
%\IEEEauthorblockA{\textit{dept. name of organization (of Aff.)} \\
%\textit{name of organization (of Aff.)}\\
%City, Country \\
%email address}
%\and
%\IEEEauthorblockN{6\textsuperscript{th} Given Name Surname}
%\IEEEauthorblockA{\textit{dept. name of organization (of Aff.)} \\
%\textit{name of organization (of Aff.)}\\
%City, Country \\
%email address}
%
%}

\maketitle

%>>> ABSTRACT
\input{./sections/abstract}

%\begin{abstract}
%This document is a model and instructions for \LaTeX. This and the IEEEtran.cls file define the components of your paper [title, text, heads, etc.]. *CRITICAL: Do Not Use Symbols, Special Characters, Footnotes, or Math in Paper Title or Abstract.
%\end{abstract}

\begin{IEEEkeywords}
CPS Security, Controller Area Network, Intrusion Detection System, Masquerade Attack, Clock Skew
\end{IEEEkeywords}

%>>> SECTION: Introduction
\input{./sections/introduction}

%>>> SECTION: Overview of Clock-based IDS
\input{./sections/overview_of_CIDS}


%>>> SECTION: NTP-based CIDS
\input{./sections/NTP_based_CIDS}


%>>> SECTION: Cloaking Attack
\input{./sections/cloaking_attack}


%>>> SECTION: Evaluation
%\input{./sections/evaluation}
\section{Evaluation}
\label{sec:evaluation}
In this section, we  evaluate the performance of the proposed cloaking attack on two CAN bus testbeds, and demonstrate that the cloaking attack is able to bypass both the state-of-the-art and the NTP-based IDSs.
We first describe our testbeds, followed by an illustration of a single trial run of our proposed attack. We then give detailed results for the cloaking attack against both the clock skew and correlation detectors.

\input{./sections/evaluation_AB}

\input{./sections/evaluation_C}

\input{./sections/evaluation_D}



%>>> SECTION: Discussion
% \input{./sections/discussion}


%>>> SECTION: Conclusion
\input{./sections/conclusion}


%>>> SECTION: Acknowledgment
% \section*{Acknowledgment}


%>>> SECTION: Reference
\bibliographystyle{IEEEtran}
\bibliography{./sections/sang_bib}

%\section*{References}
%\begin{thebibliography}{00}
%\bibitem{b1} G. Eason, B. Noble, and I. N. Sneddon, ``On certain integrals of Lipschitz-Hankel type involving products of Bessel functions,'' Phil. Trans. Roy. Soc. London, vol. A247, pp. 529--551, April 1955.
%\bibitem{b2} J. Clerk Maxwell, A Treatise on Electricity and Magnetism, 3rd ed., vol. 2. Oxford: Clarendon, 1892, pp.68--73.
%\bibitem{b3} I. S. Jacobs and C. P. Bean, ``Fine particles, thin films and exchange anisotropy,'' in Magnetism, vol. III, G. T. Rado and H. Suhl, Eds. New York: Academic, 1963, pp. 271--350.
%\bibitem{b4} K. Elissa, ``Title of paper if known,'' unpublished.
%\bibitem{b5} R. Nicole, ``Title of paper with only first word capitalized,'' J. Name Stand. Abbrev., in press.
%\bibitem{b6} Y. Yorozu, M. Hirano, K. Oka, and Y. Tagawa, ``Electron spectroscopy studies on magneto-optical media and plastic substrate interface,'' IEEE Transl. J. Magn. Japan, vol. 2, pp. 740--741, August 1987 [Digests 9th Annual Conf. Magnetics Japan, p. 301, 1982].
%\bibitem{b7} M. Young, The Technical Writer's Handbook. Mill Valley, CA: University Science, 1989.
%\end{thebibliography}


%>>> SECTION: Appendix
\input{./sections/appendices}

\end{document}
