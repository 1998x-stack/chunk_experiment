\documentclass{article}
\usepackage{color}
\usepackage{ amssymb }
\usepackage{amsmath}
\usepackage{mathtools}
\DeclarePairedDelimiter{\ceil}{\lceil}{\rceil}
\DeclarePairedDelimiter{\floor}{\lfloor}{\rfloor}
\DeclarePairedDelimiter{\round}{\lceil}{\rfloor}
\DeclarePairedDelimiter{\Round}{\left\lceil}{\right\rfloor}
\DeclareMathOperator*{\argmax}{arg\,max}
\DeclareMathOperator*{\argmin}{arg\,min}
\title{Response to the reviewers}

\begin{document}
\maketitle
\section*{Review 1}

\begin{itemize}
\item  My concerns are mostly addressed (some are a matter of taste and I do not want to force my taste on the authors' writing). 

\item Overall, the work has some theoretical value (not major, but fair), and little practical value (the interesting cases are when S is small, where they see very small gain). Having said that, I see no harm in the publication of this work. 

\item The revised version is improved compared to the previous version. The presentation can still be improved (especially improving the motivation can help). Also, I like to see a  paragraph added which clearly discusses the shortcomings of this work (e.g., little practical benefit and the low rate of the codes) and another one which motivates the work mostly from a theoretical point of view (e.g, would the theory behind this work open new research venues?). 

\textcolor{blue}{We have added Remark 2 to the introduction in order to emphasize that the asymptotic results are of theoretical and not practical interest. Furthermore, we separately discussed examples with $s =3$ for the numerical analysis of both functional and exact repair models, to show that FCRS already improves the repair bandwidth at the more practical regime of small $s$ (albeit by less than the asymptotic values). 
}

\end{itemize} 
\pagebreak


\section*{Review 2}

The authors have addressed my comments in a satisfactory manner. I still have a few inconveniences and suggestions that are listed below.

\subsection*{Major comments:}

\begin{itemize}
\item      Section II, under ``repair-by-transfer". The ``broader notion" you mention is unclear. This requirement on the entropy seems to invalidate the very essence of ``exact repair". If I understand correctly, this merely implies that $X_\ell^{(r)}$ is an injective function of $Y_{\ell,[d]}^{(r,i)}$. If so, how is that a broader interpretation of exact repair? Also, where do you use this assumption in your analysis?

\textcolor{blue}{We are only relying on one additional assumption in our new converse bound: $H(Y_{\ell,[d]}^{(r,i)}|X_\ell^{(r)}) = 0$. This property tells us that the helper nodes do not transmit any ``useless" information to the newcomer. This is naturally implied by the ``repair-by-transfer" model, since there is no computation whatsoever. The helpers transmit blocks of data to the newcomer, and the newcomer stores all the received blocks. Nevertheless, in the broader notion of ``exact repair", we may not have $H(Y_{\ell,[d]}^{(r,i)}|X_\ell^{(r)}) = 0$. Hypothetically speaking, let us say that the newcomer must reconstruct $A+B+C$ where $A$, $B$, and $C$ are stored on different servers. In this case, we would have $(Y_1,Y_2,Y_3) = (A,B,C)$ and $X = A +B + C$. These variables do not satisfy $H(Y_{[3]}|X) = 0$.\\
 We do share the same intuition with you that at the MBR point $X_\ell^{(r)}$ will be an injective function of $Y_{\ell,[d]}^{(r,i)}$ and therefore, $H(Y_{\ell,[d]}^{(r,i)}|X_\ell^{(r)}) = 0$. However, we do not have a proof for this conjecture.}

\item       I strongly suggest to compose a clear list of open problems in a designated location in the paper. 

\textcolor{blue}{We added this to the ``Conclusion and Open Problems" section.}

\end{itemize}
\subsection*{Minor comments:}
\begin{itemize}
\item      Introduction - It is unclear what is the ``data recovery criterion", how does it relate to the MDS property, and what is required relation between the parameters in order to satisfy it.

\textcolor{blue}{Thanks. We added equation labels and specifically addressed the data recovery criterion in Section II. Please see Equation (2). We also made a reference in the introduction to the corresponding section. }

\item      The repeated use of the notation mod(n,s) is a bit odd. I assume that you mean n mod s.

\textcolor{blue}{Fixed.}


\item      Section II - The sentence ``Upon receiving ... a function of Y" is unclear.

\textcolor{blue}{Fixed.}

\item      Remark 2 - ``servers fail" should be ``servers that fail" or ``failed servers".

\textcolor{blue}{Fixed.}

\item      Section II - Under ``repair-by-transfer in r'th" should be ``in the r'th".

\textcolor{blue}{Fixed.}

\end{itemize}

\pagebreak


\section*{Review 3}

The authors have made reasonable attempts to address my previous concerns, however, the current state of the paper still appears to be a borderline case. Both the technical results and the connection to practical applications are on the weak side. In terms of the presentation, the paper is actually quite nicely written, and there are a few observations that may be useful. Though I do not feel comfortable recommending for an acceptance of the paper at this point, I will also not strongly recommend against it. Below my concerns are (re)-stated.

\begin{itemize}

\item My main concern is the motivation and problem formulation itself. The availability issue is of course an important concern in practical data storage systems, but this should not be viewed as of such importance that it needs to be achieved at significant expense of the storage cost. Of course, a small sacrifice in the storage cost may be justifiable. However, due to the Fixed Cluster formulation in this work, the code constructions for the proposed framework are essentially only of theoretical value from the outset, due to the immediate consequence of only allowing very low rate codes in this framework. This understanding is immediately clear even without the effort reported in the current work, and in this regard, the current work simply confirms it with a more quantitative analysis. To make things worse, the focus of the work is on the MBR point instead of the MSR point, which means an even more elevated storage cost. This is exactly the reason that MBR codes for the original regenerating codes
did not receive any essential attention in any meaningful application.

\textcolor{blue}{We are fully aware of the limitations of this work. As another reviewer suggested, we have added a paragraph to the introduction (Remark 2) to put the weaknesses and the strengths of the paper into perspective.}


\item One of my early suggestions was to include results on MSR code construction, but no such code construction was introduced in the revision. The authors did make some efforts to include more converse results on the proposed MBR code construction, however, there are significant restrictions on this optimality result. In the literature, there are three classes of
codes proposed: general exact-repair codes, help-by-transfer codes, and repair-by-transfer codes. The last class is the most restrictive, which is the setting the new results are given. Thus this additional result in the revision does not significantly improve the technical contribution of the work.

\textcolor{blue}{Please note that as Figure 4 and the analysis in Section III suggest, FCRS performs well close to the MBR point, whereas the model in [1] outperforms FCRS close to the MSR point. As we explained in the previous response letter and in the introduction of the paper, there is a plethora of works in the literature concerning the trade-off between storage and availability, many of which outperform [1]. The main objective of the current paper is to guarantee high availability while maintaining a low repair bandwidth. Knowing  these, designing exact-repair MSR codes for FCRS would be a futile effort, and would contradict the spirit of the paper.\\
Unfortunately, proving optimality of Cubic Codes under the help-by-transfer model appears to be as hard as the general exact repair model. As we responded to another reviewer, the main property that helps us prove the optimality of Cubic Codes under repair-by-transfer is $H(Y_{\ell,[d]}^{(r,i)}|X_\ell^{(r)}) = 0$ which may not hold under help-by-transfer.}

\item The improvement factor (maximum 0.79) of the proposed code, comparing to the original regenerating codes, is small, and thus the result is not very exciting in this aspect. Moreover, this baseline itself should in fact be updated, perhaps to [19,29].

\textcolor{blue}{Reference [29] considers a generalization of locally repairable codes where we can repair $\delta - 1$ simultaneous failures with the help of only $r$ servers. The construction proposed in [29] does not perform well in terms of the repair bandwidth vs. availability trade-off (not surprisingly, since availability of data is not studied in [29]). To see why, let us look at the proposed construction.\\
Firstly, we encode the data using Gabidulin codes. We take the codeword symbols of this Gabidulin code and partition them into $t$ banks. Each bank of symbols is then encoded with an MBR code. Subsequently, each of these local MBR codes has the property that erased symbols can be reconstructed by reading only a few other symbols (within the same partition) while generating small traffic. This is exactly the property that the authors of [29] have been aiming to achieve. Therefore, this construction performs very well from their perspective. Nevertheless, in order to achieve high availability, we should be able to reconstruct a codeword symbol by reading the other banks too. This is generally possible with Maximum Rank Distance codes such as Gabidulin codes, but results in a large repair bandwidth. MRD codes have been repeatedly used in the literature to minimize the storage for locally repairable codes \footnote{\textcolor{blue}{See for instance ``Optimal Locally Repairable Codes via Rank Metric Codes" by Silberstein et al.}} and do not perform well at the MBR point. We would like to reiterate that the focus of the current paper is on the MBR point and not the MSR.\\
Reference [19] introduces the notion of Fractional Repetition Codes which have been uses as a baseline for our construction. In other words, Cubic Codes are examples of FR Codes. Nevertheless, the specific FR constructions in [19] do not perform well in terms of availability vs. repair bandwidth trade-off, which is not surprising, given the focus of the paper. Furthermore, neither of these two references considers the functional repair problem.\\
Reference [1] substantially outperforms any other constructions that we have seen in the literature in terms of the availability vs. repair bandwidth trade-off.   }
\end{itemize}

\subsection*{Some other comments:}
\begin{itemize}
\item Many references should be updated to their journal versions, essentially for all conference publications or arxiv pre-prints dated on or before 2015.

\textcolor{blue}{We fixed this to the extent possible.}

\item The capitalization in the references is not done correctly: e.g., ``mbr" should be ``MBR" in several places.

\textcolor{blue}{Fixed.}

\end{itemize}
\end{document}