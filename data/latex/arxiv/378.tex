\section{No multiplicative orthomorphisms exist for $n > 2$}
\label{sec:nomult}
Throughout this section, $n \ge 2$ is a fixed integer,
and $\sigma : \{1, \dots, n-1\} \to \{1, \dots, n-1\}$
is a multiplicative orthomorphism.
Our aim is to show $n = 2$.

We first provide the following definition.
\begin{definition}
	Given $x \in \ZZ/n$,
	we define the \emph{rank} $R_n(x) = \gcd(x, n)$.
\end{definition}
We observe that $R_n(ab) \ge \max \left\{ R_n(a), R_n(b) \right\}$.
In particular, $R_n(x\sigma(x)) \ge \max\left\{ \sigma(x), x \right\}$.
However, the sequences $x$, $\sigma(x)$, $x\sigma(x)$
are supposed to be permutations of each other,
and in particular they have the same multisets of ranks.
Therefore this is only possible if
\[ R_n \left( x \sigma(x) \right) = R_n(x) = R_n(\sigma(x)) \]
for every $x$.

With this, we may begin by proving:
\begin{proposition}
	The number $n$ must be squarefree.
	\label{prop:mult_squarefree}
\end{proposition}
\begin{proof}
	Assume $q$ is a prime with $q^2 \mid n$.
	Then consider elements $x \in \ZZ/n$
	for which the exponent of $q$ in $x$ is either $0$ or $1$.
	For those elements, we necessarily have $q \nmid \sigma(x)$,
	otherwise $R_n(x\sigma(x)) \ge q R_n(x) > R_n(x)$ which is a contradiction.

	Thus at least $\frac{q^2-1}{q^2} n$ of the $\sigma(x)$'s
	need to be not divisible by $q$.
	But $\sigma$ should be a permutation of $\{1, \dots, n\}$
	which only has $\frac{q-1}{q} n$ elements not divisible by $q$,
	contradiction.
\end{proof}

Let $q$ now be any prime divisor of $n$,
and let $m = n/q$.
Since $n$ is squarefree we have $\gcd(m, n) = 1$.
Consider the set $S$ consisting of the $q-1$ elements of rank $m$,
namely \[ S = \{m, 2m, \dots, (q-1)m \}. \]
Then $\sigma(x)$ and $x\sigma(x)$ both induce permutations on $S$,
and therefore we have
\[  \left( \prod_{i=1}^{q-1} im  \right)^2
	\equiv \prod_{i=1}^{q-1} im \cdot \sigma(im)
	\equiv \prod_{i=0}^{q-1} im \pmod{n}. \]
and from this we deduce
\[ 1 \equiv \prod_{i=1}^{q-1} im
	\equiv (q-1)! \cdot m^{q-1} \equiv 1 \pmod q. \]
By Fermat's little theorem we know $m^{q-1} \equiv 1 \pmod q$.
On the other hand, $(q-1)! \equiv -1 \pmod q$ by Wilson's theorem.
Consequently, we conclude $-1 \equiv 1 \pmod q$,
and therefore $q = 2$.

Since $q$ was any prime dividing $n$, and $n$ is squarefree,
we conclude $n = 2$ is the only possible value.
