%\section{Estimating the partition functions}
\section{Proof of main theorem}

Before putting everything together, we show how to estimate the partition functions. 
We will apply the following to $g_1(x) = e^{-\be_{\ell}f(x)}$ and $g_2(x) = e^{-\be_{\ell+1} f(x)}$. 
\begin{lem}[Estimating the partition function to within a constant factor]
Suppose that $p_1(x) =\fc{g_1(x)}{Z_1}$ and $p_2(x)=\fc{g_2(x)}{Z_2}$ are probability distributions on $\Om$. %with $\fc{Z_2}{Z_1}\in [\rc C, C]$, where $Z_i = \int_\Om g_i(x)\dx$.
 %, with $q_1,q_2$ being unnormalized versions of the probability distribution. 
%Let $\wt p_1, p_1,p_2$ be distributions on $\Om$. Suppose that $q_2=kp_2$ for some constant $k$ ($q_2$ is an unnormalized version of the probability distribution). 
Suppose $\wt p_1$ is a distribution such that $d_{TV}(\wt p_1, p_1)<\fc{\ep}{2C^2}$, and $\fc{g_2(x)}{g_1(x)}\in [0, C]$ for all $x\in \Om$. Given $n$ samples from $\wt p_1$, define the random variable
\begin{align}
\ol r = \rc{n} \sumo in \fc{g_2(x_i)}{g_1(x_i)}.
\end{align}
Let
\begin{align}
r = \EE_{x\sim p_1}\fc{g_2(x)}{g_1(x)} = \fc{Z_2}{Z_1}
\end{align}
and suppose $r\ge \rc{C}$. 
Then  with probability $\ge 1-e^{-\fc{n\ep^2}{2C^4}}$, 
\begin{align}
\ab{\fc{\ol r}{r}-1}& \le \ep.
\end{align}
\label{l:partitionfunc}
\end{lem}
\begin{proof}
We have that 
\begin{align}
\label{eq:est-Z1}
\ab{\EE_{x\sim \wt p_1} \fc{g_2(x)}{g_1(x)} - \EE_{x\sim p_1}\fc{g_2(x)}{g_1(x)}}
&\le Cd_{TV}(\wt p_1, p_1)\le \fc{\ep}{2C}.
\end{align}
The Chernoff bound gives
\begin{align}
\Pj
\pa{
\ab{r - \EE_{x\sim \wt p_1} \fc{g_2(x)}{g_1(x)}} \ge \fc{\ep}{2C}
}
&\le 
e^{-\fc{n\pf{\ep}{2C}^2}{2\pf{C}2^2}} = 
e^{-\fc{n\ep^2}{2C^4}}.\label{eq:est-Z2}
\end{align}
Combining \eqref{eq:est-Z1} and \eqref{eq:est-Z2} using the triangle inequality,
\begin{align}
\Pj\pa{|\ol r- r|\ge \rc{\ep}C} \le e^{-\fc{n\ep^2}{2C^4}}.
\end{align}
Dividing by $r$ and using $r\ge \rc{C}$ gives the result.
\end{proof}


\begin{lem}\label{lem:delta}
%Assume the setting of Lemma~\ref{lem:close-to-sum}. 
Suppose that $f(x) = -\ln \ba{\sumo in  w_i e^{-\fc{\ve{x-\mu_i}^2}2}}$, $p(x)\propto e^{-f(x)}$, and for $\al\ge 0$ let $p_\al(x) \propto e^{-\al f(x)}$, $Z_\al =\int_{\R^d} e^{-\al f(x)}\dx$. 
Suppose that $\ve{\mu_i}\le D$ for all $i$. 
%Given $\be_j$, let $p_{j}(x)\propto e^{-\be_{j}f(x)}$. 

If $\al<\be$, then
\begin{align}
\ba{
\int_{A} \min\{p_{\al}(x),p_\be(x)\} \dx 
}/p_\be(A)
&\ge
\min_x\fc{p_\al(x)}{p_\be(x)} \ge \fc{Z_\be}{Z_\al}\\
\fc{Z_\be}{Z_\al}&
\in \ba{
\rc2 e^{-2(\be-\al)\pa{D+\rc{\sqrt\al}\pa{\sqrt d + \sqrt{\ln \pf{2}{w_{\min}}}}}^2}, 1}.
\end{align}
Choosing $\be-\al=O\prc{D^2+\fc{d}{\al}+\rc{\al}\ln\prc{w_{\min}}}$, this quantity is $\Om(1)$.
\end{lem}
\begin{proof}
Let $\wt p_\al(x) \propto \sumo in w_i e^{-\al\ve{x-\mu_i}^2/2}$. 

Let %$C=D+\fc{2\sqrt d}{\be}$. 
$C= D +\rc{\sqrt\al} \pa{\sqrt{d}+\sqrt{2\ln \pf{2}{w_{\min}}}}$. Then by Lemma~\ref{lem:close-to-sum}, 
\begin{align}
\Pj_{x\sim p} (\ve{x}\ge C) 
&\le \rc{w_{\min}} \Pj_{x\sim \wt p_{\al}}(\ve{x}\ge C)\\
&\le \rc{w_{\min}}
\sumo in w_i \Pj_{x\sim \wt g_{\al}} (\ve{x}\ge C)\\
&\le  \rc{w_{\min}}\sumo in w_i \Pj_{x\sim N\pa{0,\rc{\sqrt \al} I_d}}(\ve{x}^2 \ge (C-D)^2)\\
&= \rc{w_{\min}} \Pj_{x\sim N(0, I_d)}\ba{\ve{x}^2 \ge \pa{\sqrt{d} + \sqrt{2\ln \pf{2}{\ep w_{\min}}}}^2}\\
&\le \rc{w_{\min}} \Pj_{x\sim N(0, I_d)}\ba{\ve{x}^2 \ge 
d+2\pa{\sqrt{d \ln\pf{2}{\ep w_{\min}}} + \ln \pf{2}{\ep w_{\min}}}}
\\
&\le  \rc{w_{\min}}\fc{w_{\min}}2 =\rc2
\end{align}
using the $\chi_d^2$ tail bound $\Pj_{y\sim \chi_d^2}(y\ge 2(\sqrt{dx}+x))\le e^{-x}$ from \cite{laurent2000adaptive}.

Thus, using $f(x)\ge 0$, 
\begin{align}
\ba{
\int_{A} \min\{p_{\al}(x),p_\be(x)\} \dx 
}/p_\be(A)
&\ge \int_A \min\{\fc{p_{\al}(x)}{p_\be(x)},1\} p_\be(x)\dx\Big/p_\be(A)\\
&\ge \int_A \min\{\fc{Z_\be}{Z_\al} e^{(-\be + \al)f(x)}, 1\} p_\be(x)\dx\Big /p_\be(A)\\
&\ge \fc{Z_\be}{Z_\al}
\\
& = \fc{\int e^{-\be f(x)}\dx}{\int e^{-\al f(x)}\dx}\\
&=\int_{\R^d} e^{(-\be + \al)f(x)}p_\al(x)\dx\\
&\ge \int_{\ve{x}\le D +\rc{\sqrt\al} \pa{\sqrt{d}+\sqrt{2\ln \pf{2}{w_{\min}}}}} e^{(-\be + \al)f(x)}p_\al(x)\dx\\
&\ge \rc2 e^{-(\be-\al)\max_{\ve{x}\le D +\fc2{\sqrt\al} \pa{\sqrt{d}+\sqrt{\ln \pf{2}{w_{\min}}}}}(f(x))}\\
&\ge \rc 2 e^{-2(\be-\al)\pa{D+\rc{\sqrt \al}(\sqrt d + \sqrt{\ln \pf{2}{w_{\min}}})}^2}
%\rc2 e^{-(\be-\al) } 
\end{align}
\end{proof}

\begin{lem}\label{lem:chi-sq-mixture}
If $p=\sumo in w_i p_i$ where $p_i$ are probability distributions and $w_i>0$ sum to 1, then 
\begin{align}
\chi^2(p||q) &\le \sumo in w_i \chi^2(p_i||q).
\end{align}
\end{lem}
\begin{proof}
We calculate 
\begin{align}
\chi^2(p||q)&=\sumo in \fc{q(x)^2}{\sumo in w_ip_i(x)} \dx -1\\
&\le \int  \pa{\sumo in w_i}\pa{\sum in w_i\fc{q(x)^2}{p_i(x)}}\dx -1\\
& = \sumo in w_i\pa{\int \fc{q(x)^2}{p_i(x)}\dx-1} = \sumo in w_i \chi^2(p_i||q).
\end{align}
\end{proof}

\begin{lem}\label{lem:a1-correct}
Suppose that Algorithm~\ref{a:stlmc} is run on temperatures $0<\be_1<\cdots< \be_\ell\le 1$, $\ell\le L$ with partition function estimates $\wh{Z_1},\ldots, \wh{Z_\ell}$ satisfying
\begin{align}\label{eq:Z-ratio-correct}
\ab{\fc{\wh{Z_i}}{Z_i} - \fc{\wh{Z_1}}{Z_1}}\le \pa{1+\rc L}^{i-1}
\end{align} 
for all $1\le i\le \ell$
and with parameters satisfying
\begin{align}
\label{eq:beta1}
\be_1 &= O\pf{\si^2}{D^2}\\
\label{eq:beta-diff}
\be_i-\be_{i-1} &= 
%O\prc{\pf{D}{\si}^2 + \fc{d}{\be_{i-1}} + \rc{\be_{i-1}} \ln \prc{w_{\min}}}\\
O\pf{\si^2}{D^2\pa{d+\ln \prc{w_{\min}}}}\\
T&=\Om\pa{D^2\ln \prc{w_{\min}}}\\
t&=\Omega \left(\frac{D^8 \left(d^4 + \ln\left(\frac{1}{w_{\min}}\right)^4 \right)}{\sigma^8 w^4_{\min}} \ln\left(\frac{1}{\epsilon} \frac{D^2 d \ln\left(1/w_{\min}\right)}{\sigma^8 w_{\min}}\right) \max\left(\frac{D^2}{\sigma^2}, \frac{m^{16}}{\ln(1/w_{\min})^4}\right)\right)\\
%\Om\pf{\ln \prc{\ep}m^8}{w_{\min}^2D^2}\fixme{\de^2} w_{\min}^4D^8\\
\eta &= %O\pa{\si^2\min\bc{ \fc{\si}{tD \sqrt{\fc{D}{\si}+d} \sqrt{\ln (1/w_{\min})}}, \rc d}}.
O\pf{\ep \si^2}{dtT}.
\end{align}
Let $q^0$ be the distribution $\pa{N\pa{0,\fc{\si^2}{\be_1}}, 1}$ on $\R^d\times [\ell]$. 
The distribution $q^t$ after running for $t$ steps satisfies $
\ve{p-q^t}_1\le \ep
$.

Setting $\ep = O\prc{L}$ above and taking $m=\Om\pa{\ln \prc{\de}}$ samples, with probability $1-\de$ the estimate 
\begin{align}\wh Z_{\ell+1}&=
\wh Z_\ell \pa{\rc{m}\sumo jm e^{(-\be_{\ell+1} + \be_\ell)f_i(x_j)}}
\end{align} also satisfies~\eqref{eq:Z-ratio-correct}.
\end{lem}

\begin{proof}
First consider the case $\si=1$.

Consider simulated tempering $M_{\st}|_{B_R}$: the type 1 transitions are running continuous Langevin for time $T$, and  if a type 1 transition would leave $B_R$, then instead stay at the same location. 
Let $p^t|_{B_R}$ be the distribution of $M_{st}|_{B_R}$ after $t$ steps starting from $p^0$. 
%Define $p^t|_{B_R}$ the same except that the type 1 transitions are running continuous Langevin for time $T$. We have that
By the triangle inequality,
\begin{align}
\ve{p-q^t}_1 &\le \ve{p-p|_{B_R}}_1
+ \ve{p|_{B_R} - p^t|_{B_R}}_1 
+ \ve{p^t|_{B_R} - p^t}_1
+ \ve{p^t-q^t}_1
\end{align}
Note that $\ve{p-p|_{B_R}}_1$ and $\ve{p^t|_{B_R}-p^t}_1$ approach 0 as $R\to \iy$, so we concentrate on the other two terms.

Let $M_i$ be the chain at inverse temperature $\be_i$.  Consider first $i\ge 2$. 
By Lemma~\ref{lem:rest-large}, for any $\ep>0$ we can choose $R$ such that for all $x\in B_R$, $P_T(x,B^c) \le e^{-\be T/2}$ and
\begin{align}
\la_n(I-P_T|_{B_R}) \ge (1-\ep)(\la_n(I-P_T) - e^{-\be T/2})
\end{align}
For $T=\Om\pa{\ln \prc{w_{\min}}D^2}$, we have $e^{-\be T/2} = o\pf{Tw_{\min}}{D^2}$.
% such that $\Gap(\ol{M_i}^{\cal P_i})\ge \Gap(\ol{M_i})$. 
Thus we can ensure
\begin{align}
\la_n(I-P_T|_{B_R}) \ge \fc 34\pa{\la_n(I-P_T) - o\pf{Tw_{\min}}{D^2}}
\end{align}
By Theorem~\ref{thm:bakry-emery}, a Poincar\'e inequality holds for $\be_i f_j$ with constant $\fc{8}{\be_i}\le O(D^2)$. Letting $\sL'$ be the generator for Langevin on $\wt g_{\be_i}(x)=\sumo jn w_i e^{-\be_i f_j(x)}$, 
\begin{align}
\la_{n+1}(\sL)&\ge w_{\min}\la_{n+1}(\sL')&\text{by Lemma~\ref{lem:close-to-sum} and Lemma~\ref{lem:poincare-liy}}\\
& \ge \Om\pf{w_{\min}}{D^2}&\text{by Lemma~\ref{lem:m+1-eig}}
\end{align}
By Lemma~\ref{lem:small-poincare} on $p_\be$, a Poincar\'e inequality holds with constant $O\pf{D^2}{w_{\min}}$. on sets of size $\le \fc{w_{\min}^2}2$.

By Lemma~\ref{lem:limit-chain},
for each $i$, we can choose a partition $\cal Q_i$ of $B_R$ such that for every compact $K$ consisting of a union of sets in $\cal Q_i$, 
\begin{align}
\Gap({M_i|_K}) &\ge (1-\ep)\Gap(\ol{M_i|_K}^{\cal Q_i}).
\end{align}•

All the conditions of Lemma~\ref{lem:any-partition} are satisfied with $C=O\pf{D^2}{w_{\min}}$. We obtain a partition $\cal P_i$ that is a 
$
\pa{\Om\pa{w_{\min}^2(\ln \rc{w_{\min}})^2{m^8}}, O\pf{Tw_{\min}}{D^2}}
$-clustering under the projected chain with each set in the partition having measure at least $\fc{w_{\min}^2}4$ under $p_i$. 
By having chose the partition fine enough, and by Cheeger's inequality~\ref{thm:cheeger}, for each set $A$ in the $\cal P_i$, 
\begin{align}
\Gap(M_{i}|_A)\ge 
(1-\ep)
\Gap(\ol{M_{i}|_{A}}^{\cal P_i})\ge \Om\pf{(\ln \rc{w_{\min}})^4}{m^{16}}.
\end{align}
For the highest temperature, by Lemma~\ref{lem:hitempmix}, we have
\begin{align}
\Gap(M_1) &=\Om\pa{ \be_1 e^{-2\be_1 D^2}} = \Om\prc{D^2}. 
\end{align}

By Lemma~\ref{lem:delta}, since the condition on $\be_i-\be_{i-1}$ is satisfied, $\de((\cal P_i)_{i=1}^\ell)=\Om(1)$. By assumption on $\wh{Z_i}$, $r=\Om(1)$ ($r$ is defined in Assumption~\ref{asm}). 
By Theorem~\ref{t:temperingnochain}, the spectral gap of the simulated tempering chain is
\begin{align}
G:=\Gap(M_{\st}) = \Om\pa{\fc{r^4\de^2p_{\min}^2}{\ell^4}\min\bc{\pf{(\ln \rc{w_{\min}})^4}{m^{16}}, \rc{D^2}}} = \fc{w_{\min}^4}{\ell^4}\min\bc{\pf{{\ln \rc{w_{\min}}}^4}{m^{16}}, \rc{D^2}}.
\end{align}
Choosing $t=\Om\pf{\ln \pf{\ell}{\ep w_{\min}}}G$, we get by Cauchy-Schwarz and~\eqref{eq:gap-mix} that 
\begin{align}\label{eq:chi-sq-final}
\ve{p-q^t}_1&\le \chi^2(p||q^t)\le  (1-G)^t\chi_2(p||q^0) \le e^{-Gt}\chi^2(p||q^0)
=O\pf{\ep w_{\min}}{\ell}\chi^2(p||q^0)
%=\fc{\ep}{3}.
\end{align}
(Note that $G<2-\la_{\max}$ because the chain is somewhat lazy; it stays with probability $\rc{2\ell}$.)
To calculate $\chi^2(q||p_0)$, first note the $\chi^2$ distance between $N(0,\si^2I_d)$ and $N(\mu, \si^2I_d)$ is $\le e^{\ve{\mu}^2/\si^2}$:
\begin{align}
\chi^2(N(0,\si^2I_d), N(\mu,\si^2 I_d))& = \rc{(2\pi\si^2)^{\fc d2}}\int_{\R^d} e^{2(-\fc{\ve{x-\mu}^2}{2\si^2}) + \fc{\ve{x}^2}{2\si^2}}\dx-1\\
&\le \rc{(2\pi\si^2)^{\fc d2}}\int_{\R^d}  e^{(-\fc{\ve{x}^2}2 + 2\an{x,\mu} - 2\ve{\mu}^2)/\si^2} e^{\ve{\mu}^2/\si^2}\dx= e^{\ve{\mu}^2/\si^2}. 
\end{align}•
Then by Lemma~\ref{lem:chi-sq-mixture} and Lemma~\ref{lem:close-to-sum},
\begin{align}
\chi^2(p||q^0) &\le O\pf{\ell}{w_{\min}} \chi^2\pa{\wt p_\be||N\pa{0, \rc{\be_1}I_d}} \\
&= O\pf{\ell}{w_{\min}}\sumo im \chi^2\pa{N\pa{\mu_i,\rc{\be_1}I_d)||N(0, \rc{\be_1}I_d}}\\
&= O\pf{e^{D^2\be_1}\ell}{w_{\min}} =O\pf{\ell}{w_{\min}}. 
\end{align}
Together with~\eqref{eq:chi-sq-final} this gives $\ve{p-q^t}_1\le \fc\ep3$.

For the term $\ve{p^t-q^t}_1$, use Pinsker's inequality and Lemma~\ref{l:maindiscretize} to get
\begin{align}
\ve{p^t-q^t}_1
&\le 
 \sqrt{2\KL(p^t||q^t)}\\
 &=O\pa{ 
\eta^2[(D^2+d) Tt^2 + D^2] +\eta dtT}\le \fc{\ep}3
\end{align}
for %$\eta = O\pa{\min\bc{ \rc{tD \sqrt{D+d} \sqrt{\ln (1/w_{\min})}}, \rc d}}$. 
$\eta = O\pa{\ep \min\bc{\rc{\sqrt T t(D+\sqrt d)}, \rc{dtT}}} = O(\frac{\ep}{dtT})$. 

This gives $\ve{p-q^t}_1 \le \ep$. 

For the second part, setting $\ep=O\prc{\ell L}$ gives that $\ve{p_l - q^t_l}=O\prc{L}$.
By Lemma~\ref{l:partitionfunc}, noting Lemma~\ref{lem:delta} gives $C=O(1)$, 
 after collecting $n=\Om\pa{L^2\ln \prc{\de}}$ samples, with probability $\ge 1-\de$, $\ab{\fc{\wh{Z_{\ell+1}}/\wh{Z_\ell}}{Z_{\ell+1}/Z_\ell}-1}\le \rc L$. 
Set $\wh{Z_{\ell+1}} = \ol r\wh{Z_\ell}$. Then 
$\fc{\wh{Z_{\ell+1}}}{\wh{Z_\ell}} \in [1-\rc L , 1+\rc L] \fc{Z_{\ell+1}}{Z_\ell}$ and 
$\fc{\wh{Z_{\ell+1}}}{\wh{Z_1}} \in 
\ba{\pa{1-\rc L}^\ell, \pa{1+\rc L}^\ell}\fc{Z_{\ell+1}}{Z_1}$.

Now consider general $\si$. We can transform the problem to the problem where $\si=1$ by the change of variables $x\mapsfrom x\si$. This changes $D$ to $\fc{D}{\si}$. Note that running the discretized chain on this transformed problem with step size $\eta$ corresponds to running the discretized chain on the original problem with step size $\eta \si^2$. This is because a step $Y_{t+1} =Y_t - \eta \nb  g(Y_t) \,dt + \sqrt{2 \eta }\xi_k$ with $g(x) = f\pf{x}{\si}$ corresponds to a step 
 $X_{t+1} = X_t - \eta \si\nb g\pf{X_t}{\si} + \sqrt{2 \eta }\si\xi_k
 = X_t - \eta\si^2 \nb f(X_t) +\sqrt{2\eta \si^2}\xi_k$.
\end{proof}
%
%Take $T=1$.
%By Lemma ?, the spectral gap for the simulated tempering chain is 
%\begin{align}
%\min\Gap(M|_A)&=\Om\pa{
%\min\bc{
%\min_{j\ge 2}\fc{w_{\min}^2T^2}{C_{\be_j}^2 m^8}, \rc{m^2} , \rc{C_\be}w_{\min}T, \be_{1}^{-1} e^{2\be_{1} D^2T}
%}
%}\\
%\Gap(M) &\ge O\pf{r^4 \de^2 p_{\min}^2}{L^4} \min\Gap(M|_A)
%= O\pa{\fixme{\de^2} w_{\min}^4D^8}\min\Gap(M|_A)
%\end{align}
%Note $\be = \rc{D^2T}$ gives $ \be_{1}^{-1} e^{2\be_{1} D^2T} = O(D^2)$, and $C_{\be_j}^2 \ge \rc{D^2}$. 
%
%By Lemma ?, 
%\begin{align}
%KL_t := KL(p^t||q^t)&=O\pa{\fc{\eta^2}{\si^6}[(D+d) Tt^2 + 2D^2] + \fc{\eta}{\si^4}d + T\eta}.
%\end{align}
%
%By the triangle inequality, Cauchy-Schwarz, and Pinsker's inequality, 
%\begin{align}
%\ve{p-q^t}_1&\le \ve{p-p^t}_1 + \ve{p^t-q^t}_1\\
%&\le \chi^2(p||p^t) + \sqrt{2KL_t}\\
%&\le (1-G)^t + \sqrt{2KL_t}\\
%&\le e^{-Gt} + \sqrt{2KL_t}.
%\end{align}
%For $t\ge \rc{G}\ln \prc{\ep}$, this is $O(\ep)$. We now choose $\eta$ such that 
%\begin{align}
%\fc{\eta^2}{\si^6}[(D+d) Tt^2 + 2D^2] + \fc{\eta}{\si^4}d + T\eta &= O(\ep^2). 
%\end{align}
%\end{proof}

Now we prove the main theorem, Theorem~\ref{thm:main}.
%\begin{thm}[Main theorem]\label{thm:main}
%Algorithm~\ref{a:mainalgo} gives a sample from a distribution $p'$ with $\ve{p-p'}_1\le \ep$ in time $\poly\pa{w_{\min}, D, d, \ln \prc{\ep}}$.
%\end{thm}
\begin{proof}[Proof of Theorem~\ref{thm:main}]
Choose $\de=\fc{\ep}{2L}$ where $L$ is the number of temperatures. 
Use Lemma~\ref{lem:a1-correct} inductively, with probability $1-\fc{\ep}2$ each estimate satisfies $\fc{\wh{Z_l}}{\wh{Z_1}}\in [\rc e,e]$. Estimating the final distribution within $\fc{\ep}2$ accuracy gives the desired sample.
\end{proof}

\begin{rem}
One reason that the large powers appear in Lemma~\ref{lem:a1-correct} is that we are going between conductance and spectral gap multiple times, and each time we lose a square by Cheeger's inequality. We care about a spectral gap within sets of the partition, but Theorem~\ref{thm:gt14} controls the conductance rather than the spectral gap. It may be possible to tighten the bound by proving a variant of the theorem that controls the spectral gap directly.
\end{rem}

