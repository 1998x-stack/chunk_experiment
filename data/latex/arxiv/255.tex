\section{Conclusion and Future Work} 
\label{sec:conclusion} 

In this paper, we have presented the results of a large-scale study of OutWithFriendz,
a newly designed mobile application for group event \replaced{scheduling }{organization}. We summarize
our key findings as follows: (1) User mobility has a significant impact on group event attendance.
(2) Users would like to vote for locations near their frequented places. On weekdays, they would 
like to meet after work while on weekends, they have wider time options. (3) A group host 
has a higher impact on the group decision making process than other members. (4) Early
voters are more likely to vote for more options while late voters tend to coincide with existing
results to find a mutually agreeable option. We believe the results presented in this paper
are a good start towards \replaced{better }{accurate} understanding of group \added{event scheduling} behaviors in real life. Our analysis
of this first-of-a-kind real world user study of mobile group dynamics has provided very 
important guidance for \replaced{group event scheduling and recommendation }{mobile group recommendation and mobile group formation.}

\added{We plan to pursue several directions for future work. First,
although we have collected more than 300 completed group events, we hope to grow
our user base through more effective advertising, so that we may achieve viral
adoption and gather data at even larger scales. Second, to obtain a user's
friend list, OutWithFriendz currently only allows users to log in through their
Facebook accounts. Users may want to invite people who are not already a
Facebook friend or do not use Facebook at all. This limits our app's ability to
support larger groups. We plan to design an ``Add Friend'' function which enables
users to log in and connect with other users directly within the application.
Third, currently, polls in OutWithFriends are designed to be open
polls, allowing late-coming voters to see existing voting results, which may
influence their own votes. We plan to add a closed poll option. For closed polls,
existing voting results will be hidden from new voters. This functionality will
allow us to examine how a closed poll mechanism influences the group
event scheduling process. Lastly, we have seen from our work on OutWithFriendz
that a number of factors influence group decision making. We believe that
group context can be seen as inhabiting a latent trait space, similar to how
users inhabit a latent user trait space in the matrix factorization framework
for individual recommendation. Furthermore, our work has revealed that both host and
individual members within a group play an important role in the group
event scheduling process. In our future work, we intend to pursue the development of
a group recommendation system that incorporates these ideas into a probabilistic
model for group preferences and make group event recommendations in real-world 
settings. %. Building a distributed system implementation of mobile group recommendation 
This will help us gain a better understanding of group event dynamics and provide useful
suggestions for group event organizers.}
