\section{Introduction}

The ability to robustly plan feasible grasps for arbitrary objects is an important capability of autonomous robots in the context of mobile manipulation. It allows to generate grasping configurations in an autonomous and unsupervised manner based on the object's 3D mesh information.
Part-based grasp planning is an approach to deal with the complexity of arbitrary shaped objects by segmenting them into simpler parts which are more suitable for shape analysis for grasp generation. This procedure complies also with the behavior of humans which prefer to structure objects into smaller segments (see e.g. the recognition-by-components paradigm of Biederman~\cite{biederman1987}).
Representing an object by its parts fits very well to robotic grasp planning, since a robotic gripper or hand can usually only grasp an object by accessing a local surface area for establishing contacts.
\begin{figure}[t]%
\centering
%\includegraphics[width=1\columnwidth]{fig/airplane-all2.png}%
\includegraphics[width=0.9\columnwidth]{fig/teaser.png}%
\caption{The object mesh model (1) is used to build the mean curvature object skeleton (2). The skeleton is segmented (3) and the corresponding surface segments (4) are depicted in red (segment end points), yellow (connecting segments), and blue (segment branches).
The results of the skeleton-based grasp planner are visualized (5) by indicating the approach directions of the planned grasps via green lines. In addition, a selection of grasps is fully visualized.}
\label{fig:teaser}%
\end{figure}
The part-based grasp planner presented in this work is based on the assumption that an object can be segmented according to its skeleton structure. The object skeleton simplifies the topological representation and provides information about the connectivity of the segments.
Object segmentation is realized by analyzing the skeleton according to branches and crossings in the topology. 
The object segments are analyzed separately to determine if several grasping strategies (e.g. power or precision grasps) can be applied according to the local surface information (see \autoref{fig:teaser}).

Since robustness is essential for robotic grasp planning, it is desirable to plan grasps which can be executed robustly in realistic setups. Hence, we aim at planning grasps that are robust to disturbances and inaccuracies as they appear during execution due to noise in perception and actuation. In this work, we show that the analysis of the object skeleton in combination with the usage of local surface properties leads to grasping hypotheses which are robust to inaccuracies in robot hand positioning, which indicates a reliable execution on real robot platforms. An implementation of the presented grasp planning approach based on the Simox library \cite{Vahrenkamp12b} is provided as C++ open source project\footnote{https://gitlab.com/Simox/simox-cgal}. 


%In addition, robustness is is of high importance in robotic grasp planning. Hence, the quality of a planned grasp should consider potential inaccuracies during execution. With traditional quality indexes such as the grasp wrench space approach \cite{XY} 

%We show in this work that the mean curvature flow approach (\cite{XY}) 

